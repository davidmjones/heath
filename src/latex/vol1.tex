\documentclass{heath}

\usepackage{wrapfig}

\usepackage{verbatim}

\begin{document}

\frontmatter

\title{The Thirteen Books of Euclid's Elements}

\begin{titlepage}

\begin{center}

\begingroup

\openup\bigskipamount

{\LARGE \textbf{THE THIRTEEN BOOKS OF}}

\bigskip

{\Huge \textbf{EUCLID'S ELEMENTS}}

\medskip

TRANSLATED FROM THE TEXT OF HEIBERG

\bigskip

{\Large WITH INTRODUCTION AND COMMENTARY}

\endgroup

\bigskip
\bigskip
\bigskip

\begingroup

BY

\medskip

Sir {\large THOMAS L. HEATH},

{\small K.C.B., K.C.V.O., F.R.S.,}

\medskip

{\SMALL SC.D. CAMB., HON. D.SC. OXFORD}

{\SMALL HONORARY FELLOW (SOMETIME FELLOW) OF TRINITY COLLEGE CAMBRIDGE}

\endgroup

\vfill

\emph{SECOND EDITION}

\medskip

\emph{REVISED WITH ADDITIONS}

\vfill

VOLUME I

\bigskip

INTRODUCTION AND BOOKS I, II

\vfill
\vfill

\end{center}

\clearpage

\thispagestyle{empty}

\null

\vfill

Published in Canada by General Publishing
Company, Ltd., 30 Lesmill Road, Don Mills, Toronto,
Ontario.

Published in the United Kingdom by Constable
and Company, Ltd., 10 Orange Street, London
WC 2.

\vfill

\emph{Library of Congress Catalog Card Number: 56-4336}

\end{titlepage}

\chapter*{Preface}

``There never has been, and till we see it we never
shall believe that there can be, a system of geometry
worthy of the name, which has any material departures (we do
not speak of \emph{corrections} or \emph{extensions} or \emph{developments}) from
the plan laid down by Euclid.'' De Morgan wrote thus in
October 1848 (\emph{Short supplementary remarks on the first six
Books of Euclid's Elements} in the \emph{Companion to the Almanac}
for 1849); and I do not think that, if he had been living
to-day, he would have seen reason to revise the opinion so
deliberately pronounced sixty years ago. It is true that in the
interval much valuable work has been done on the continent
in the investigation of the first principles, including the
formulation and classification of axioms or postulates which
are necessary to make good the deficiencies of Euclid's own
explicit postulates and axioms and to justify the further
assumptions which he tacitly makes in certain propositions,
content apparently to let their truth be inferred from observation of
the figures as drawn; but, once the first principles are
disposed of, the body of doctrine contained in the recent text-books of
elementary geometry does not, and from the nature 
of the case cannot, show any substantial differences from that
set forth in the \emph{Elements}. In England it would seem that far
less of scientific value has been done; the efforts of a multitude
of writers have rather been directed towards producing
alternatives for Euclid which shall be more suitable, that is to say,
easier, for schoolboys. It is of course not surprising that, in
these days of short cuts, there should have arisen a movement
to get rid of Euclid and to substitute a ``royal road to
geometry''; the marvel is that a book which was not written
for schoolboys but for grown men (as all internal evidence
shows, and in particular the essentially theoretical character
of the work and its aloofness from anything of the nature of
``practical'' geometry) should have held its own as a
school-book for so long. And now that Euclid's proofs and
arrangement are no longer required from candidates at examinations
there has been a rush of competitors anxious to be first in the
field with a new text-book on the more ``practical'' lines which
now find so much favour. The natural desire of each teacher
who writes such a text-book is to give prominence to some
special nostrum which he has found successful with pupils.
One result is, too often, a loss of a due sense of proportion;
and, in any case, it is inevitable that there should be great
diversity of treatment. It was with reference to such a danger
that Lardner wrote in 1846: ``Euclid once superseded, every
teacher would esteem his own work the best, and every school
would have its own class book. All that rigour and exactitude
which have so long excited the admiration of men of science
would be at an end. These very words would lose all definite
meaning. Every school would have a different standard;
matter of assumption in one being matter of demonstration in
another; until, at length, \textsc{Geometry}, in the ancient sense of
the word, would be altogether frittered away or be only
considered as a particular application of Arithmetic and
Algebra.'' It is, perhaps, too early yet to prophesy what will
be the ultimate outcome of the new order of things; but it
would at least seem possible that history will repeat itself and
that, when chaos has come again in geometrical teaching,
there will be a return to Euclid more or less complete for the
purpose of standardising it once more.

But the case for a new edition of Euclid is independent of
any controversies as to how geometry shall be taught to
schoolboys. Euclid's work will live long after all the text-books
of the present day are superseded and forgotten. It is one
of the noblest monuments of antiquity; no mathematician
worthy of the name can afford not to know Euclid, the real
Euclid as distinct from any revised or rewritten versions
which will serve for schoolboys or engineers. And, to know
Euclid, it is necessary to know his language, and, so far as it
can be traced, the history of the ``elements'' which he
collected in his immortal work.

This brings me to the \emph{raison d'\^etre} of the present edition.
A new translation from the Greek was necessary for two
reasons. First, though some time has elapsed since the
appearance of Heiberg's definitive text and prolegomena,
published between 1883 and 1888, there has not been, so far
as I know, any attempt to make a faithful translation from it
into English even of the Books which are commonly read.
And, secondly, the other Books, \book{vii} to \book{x} and
\book{xiii}, were not
included by Simson and the editors who followed him, or
apparently in any English translation since Williamson's
(1781–8), so that they are now practically inaccessible to
English readers in any form.

In the matter of notes, the edition of the first six Books
in Greek and Latin with notes by Camerer and Hauber
(Berlin, 1824–5) is a perfect mine of information. It would
have been practically impossible to make the notes more
exhaustive at the time when they were written. But the
researches of the last thirty or forty years into the history of
mathematics (I need only mention such names as those of
Bretschneider, Hankel, Moritz Cantor, Hultsch, Paul Tannery,
Zeuthen, Loria, and Heiberg) have put the whole subject
upon a different plane. I have endeavoured in this edition
to take account of all the main results of these researches up
to the present date. Thus, so far as the geometrical Books
are concerned, my notes are intended to form a sort of
dictionary of the history of elementary geometry, arranged
according to subjects; while the notes on the arithmetical
Books \book{vii}—\book{ix} and on Book~\book{x} follow the same plan.

I desire to express here my thanks to my brother,
Dr R. S. Heath, Vice-Principal of Birmingham University,
for suggestions on the proof sheets and, in particular, for the
reference to the parallelism between Euclid's definition of
proportion and Dedekind's theory of irrationals, to Mr R. D.
Hicks for advice on a number of difficult points of translation,
to Professor A. A. Bevan for help in the transliteration of
Arabic names, and to the Curators and Librarian of the
Bodleian Library for permission to reproduce, as frontispiece,
a page from the famous Bodleian MS. of the \emph{Elements}.
Lastly, my best acknowledgments are due to the Syndics of
the Cambridge University Press for their ready acceptance
of the work, and for the zealous and efficient cooperation of
their staff which has much lightened the labour of seeing the
book through the Press.

\byline{T. L H.}{\emph{November}, 1908.}

\chapter*{Preface to the Second Edition}

I like to think that the exhaustion of the first edition of
this work furnishes a new proof (if such were needed)
that Euclid is far from being defunct or even dormant, and
that, so long as mathematics is studied, mathematicians will
find it necessary and worth while to come back again and
again, for one purpose or another, to the twenty-two-centuries-old
book which, notwithstanding its imperfections, remains the
greatest elementary textbook in mathematics that the world is
privileged to possess.

The present edition has been carefully revised throughout,
and a number of passages (sometimes whole pages) have been
rewritten, with a view to bringing it up to date. Some not
inconsiderable additions have also been made, especially in the
Excursuses to Volume~I, which will, I hope, find interested
readers.

Since the date of the first edition little has happened in the
domain of geometrical teaching which needs to be chronicled.
Two distinct movements however call for notice.

The first is a movement having for its object the mitigation
of the difficulties (affecting in different ways students, teachers
and examiners) which are found to arise from the multiplicity
of the different textbooks and varying systems now in use for
the teaching of elementary geometry. These difficulties have
evoked a widespread desire among teachers for the
establishment of an agreed sequence to be generally adopted in teaching
the subject. One proposal to this end has already been made:
but the chance of the acceptance of an agreed sequence has in
the meantime been prejudiced by a second movement which
has arisen in other quarters.

I refer to the movement in favour of reviving, in a modified
form, the proposal made by Wallis in 1663 to replace Euclid's
Parallel-Postulate by a Postulate of Similarity (as to which see
pp.~210–11 of Volume~I of this work). The form of Postulate
now suggested is an assumption that ``Given one triangle,
there can be constructed, on any arbitrary base, another triangle
equiangular with (or similar to) the given triangle.'' It may
perhaps be held that this assumption has the advantage of not
referring, in the statement of it, to the fact that a straight line
is of unlimited length; but, on the other hand, as is well known,
Saccheri showed (1733) that it involves more than is necessary
to enable Euclid's Postulate to be proved. In any case it
would seem certain that a scheme based upon the proposed
Postulate, if made scientifically sound, must be more difficult
than the procedure now generally followed. This being so,
and having regard to the facts (1)~that the difference between
the suggested Postulate and that of Euclid is in effect so slight
and (2)~that the historic interest of Euclid's Postulate is so
great, I am of opinion that the proposal is very much to be
deprecated.

\byline{T. L. H.}{\emph{December} 1925.}

\tableofcontents

\mainmatter

\part*{Introduction}

\chapter{Euclid and the Traditions about Him}

As in the case of the other great mathematicians of Greece, so in
Euclid's case, we have only the most meagre particulars of the life
and personality of the man.

Most of what we have is contained in the passage of Proclus' summary
relating to him, which is as follows\footnote{Proclus, ed. Friedlein,
  p.~68, 6--20.}:

``Not much younger than these (sc.\ Hermotimus of Colophon and
Philippus of Medma) is Euclid, who put together the Elements,
collecting many of Eudoxus' theorems, perfecting many of Theaetetus',
and also bringing to irrefragable demonstration the things which were
only somewhat loosely proved by his predecessors. This man
lived\footnote{The word \greek{γέγονε} must apparently mean
  ``flourished,'' as Heiberg understands it
  (\emph{Litterargeschichtliche Studien über Euclid}, 1882, p.~26),
  not ``was born,'' as Hankel took it; otherwise part of Proclus'
  argument would lose its cogency.} in the time of the first
Ptolemy. For Archimedes, who came immediately after the first
(Ptolemy)\footnote{So Heiberg understands \greek{ἐπιβαλὼν τῷ πρώτῳ}
  (sc. \greek{Πτολεμαίῳ}). Friedlein's text has \greek{καί} between
  \greek{ἐπιβαλὼν} and \greek{τῷ πρώτῳ} and it is right to remark
  that another reading is \greek{καί ἐν τῷ πρώτῳ} (without
  \greek{ἐπιβαλών}) which has been translated ``in his first
  \emph{book},'' by which is understood \emph{On the Sphere and
    Cylinder}~\textsc{i}., where (1)~in Prop.~2 are the words ``let
  $BC$ be made equal to $D$ \emph{by the second \emph{(proposition)}
    of the first} of Euclid's (books),'' and (2)~in Prop.~6 the words
  ``For these things are handed down in the Elements'' (without the
  name of Euclid).  Heiberg thinks the former passage is referred to,
  and that Proclus must therefore have had before him the words ``by
  the second of the first of Euclid'': a fair proof that they are
  genuine, though in themselves they would be somewhat suspicious.},
makes mention of Euclid: and, further, they say that Ptolemy once
asked him if there was in geometry any shorter way than that of the
elements, and he answered that there was no royal road to
geometry\footnote{The same story is told in Stobaeus,
  \emph{Ecl}.\ (\textsc{ii}. p.~228, 30, ed.\ Wachsmuth) about
  Alexander and Menaechmus. Alexander is represented as having asked
  Menaechmus to teach him geometry concisely, but he replied: ``O
  king, through the country there are royal roads and roads for common
  citizens, but in geometry there is one road for all.''}.  He is then
younger than the pupils of Plato but older than Eratosthenes and
Archimedes; for the latter were contemporary with one another, as
Eratosthenes somewhere says.''

This passage shows that even Proclus had no direct knowledge of
Euclid's birthplace or of the date of his birth or death. He proceeds
by inference. Since Archimedes lived just after the first Ptolemy, and
Archimedes mentions Euclid, while there is an anecdote about
\emph{some} Ptolemy and Euclid, \emph{therefore} Euclid lived in the
time of the first Ptolemy.

We may infer then from Proclus that Euclid was intermediate between
the first pupils of Plato and Archimedes. Now Plato died in 347/6,
Archimedes lived 287–212, Eratosthenes \emph{c}.~284-204~\bc. Thus
Euclid must have flourished \emph{c}.~300~\bc, which date agrees well with
the fact that Ptolemy reigned from 306 to 283~\bc.

It is most probable that Euclid received his mathematical training in
Athens from the pupils of Plato; for most of the geometers who could
have taught him were of that school, and it was in Athens that the
older writers of elements, and the other mathematicians on whose works
Euclid's \emph{Elements} depend, had lived and taught. He may himself
have been a Platonist, but this does not follow from the statements of
Proclus on the subject. Proclus says namely that he was of the school
of Plato and in close touch with that philosophy\footnote{Proclus,
  p. 68, 20, \greek{καὶ τῇ προαιπέσει δέ Πλατωνικός ἐστι καὶ τῇ
    φιλοσοφίᾳ ταύτῃ οἰκεῖος}.}. But this was only an attempt of a New
Platonist to connect Euclid with his philosophy, as is clear from the
next words in the same sentence, ``for which reason also he set before
himself, as the end of the whole Elements, the construction of the
so-called Platonic figures.'' It is evident that it was only an idea
of Proclus' own to infer that Euclid was a Platonist because his
\emph{Elements} end with the investigation of the five regular solids,
since a later passage shows him hard put to it to reconcile the view
that the construction of the five regular solids was the end and aim
of the \emph{Elements} with the obvious fact that they were intended
to supply a foundation for the study of geometry in general, ``to make
perfect the understanding of the learner in regard to the whole of
geometry\footnote{\emph{ibid}. p.~71, 8.}.'' To get out of the
difficulty he says\footnote{\emph{ibid}. p.~70, 19 sqq.} that, if one
should ask him what was the aim (\greek{σκοπός}) of the treatise, he
would reply by making a distinction between Euclid's intentions (1)~as
regards the subjects with which his investigations are concerned,
(2)~as regards the learner, and would say as regards (1)~that ``the
whole of the geometer's argument is concerned with the cosmic
figures.'' This latter statement is obviously incorrect It is true
that Euclid's \emph{Elements} end with the construction of the five
regular solids; but the planimetrical portion has no direct relation
to them, and the arithmetical no relation at all; the propositions
about them are merely the conclusion of the stereometrical division of
the work.

One thing is however certain, namely that Euclid taught, and founded a
school, at Alexandria. This is clear from the remark of Pappus about
Apollonius\footnote{Pappus, \r7, p.~678, 10–11,
  \greek{συσχολάσας τοῖς ὑπὸ Εὐκλείδου μαθηταῖς ἐν Άλεξανδρείς\?
    πλεῖστον χρόνον, ὅθεν ἔσχε καὶ τὴν τοιαύτην ἕξιν οὐκ ἀμαθῆ}.}:
``he spent a very long time with the pupils of Euclid at Alexandria,
and it was thus that he acquired such a scientific habit of thought,''

It is in the same passage that Pappus makes a remark which might, to
an unwary reader, seem to throw some light on the personality of
Euclid.  He is speaking about Apollonius' preface to the first book of
his \emph{Conics}, where he says that Euclid had not completely worked
out the synthesis of the ``three- and four-line locus,'' which in fact
was not possible without some theorems first discovered by himself.
Pappus says on this\footnote{Pappus, \r7.\ pp.~676, 25–678,
  6. Hultsch, it is true, brackets the whole passage pp.~676, 25–678,
  15, but apparently on the ground of the diction only,}: ``Now
Euclid—regarding Aristaeus as deserving credit for the discoveries he
had already made in conics, and without anticipating him or wishing to
construct anew the same system (such was his scrupulous fairness and
his exemplary kindliness towards all who could advance mathematical
science to however small an extent), being moreover in no wise
contentious and, though exact, yet no braggart like the other
[Apollonius]—wrote so much about the locus as was possible by means of
the conics of Aristaeus, without claiming completeness for his
demonstrations.'' It is however evident, when the passage is examined
in its context, that Pappus is not following any tradition in giving
this account of Euclid: he was offended by the terms of Apollonius'
reference to Euclid, which seemed to him unjust, and he drew a fancy
picture of Euclid in order to show Apollonius in a relatively
unfavourable light.

Another story is told of Euclid which one would like to believe true.
According to Stobaeus\footnote{Stobaeus, \emph{l.c.}}, ``some one who
had begun to read geometry with Euclid, when he had learnt the first
theorem, asked Euclid, 'But what shall I get by learning these
things?' Euclid called his slave and said 'Give him threepence, since
he must make gain out of what he learns.'\,''

In the middle ages most translators and editors spoke of Euclid as
Euclid \emph{of Megara}. This description arose out of a confusion
between our Euclid and the philosopher Euclid of Megara who lived
about 400~\bc. The first trace of this confusion appears in Valerius
Maximus (in the time of Tiberius) who says\footnote{\r8.~12,
  exṭ~1.} that Plato, on being appealed to for a solution of the
problem of doubling the cubical altar, sent the inquirers to ``Euclid
the geometer.'' There is no doubt about the reading, although an early
commentator on Valerius Maximus wanted to correct ``Eucliden'' into
``\emph{Eudoxum},'' and this correction is clearly right. But, if
Valerius Maximus took Euclid the geometer for a contemporary of Plato,
it could only be through confusing him with Euclid of Megara. The
first specific reference to Euclid as Euclid of Megara belongs to the
14th century, occurring in the \greek{} of Theodorus Metochita
(d.~1332) who speaks of ``Euclid of Megara, the Socratic philosopher,
contemporary of Plato,'' as the author of treatises on plane and solid
geometry, data, optics etc.: and a Paris \textsc{ms}.\ of the 14th
century has ``Euclidis philosophi Socratici liberelementorum,'' The
misunderstanding was general in the period from Campanus' translation
(Venice 1482) to those of Tartaglia (Venice 1565) and Candalla (Paris
1566). But one Constantinus Lascaris (d.~about 1493) had already made
the proper distinction by saying of our Euclid that ``he was different
from him of Megara of whom Laertius wrote, and who wrote
dialogues''\footnote{Letter to Fernandus Acuna, printed in Maurolycus,
  \emph{Historia Siciliae}, fol.~21~r.\ (see Heiberg,
  \emph{Euklid-Studien}, pp.~22–3, 25).}; and to Commandinus belongs
the credit of being the first translator\footnote{Preface to
  translation (Pisauri, 1572).} to put the matter beyond doubt: ``Let
us then free a number of people from the error by which they have been
induced to believe that our Euclid is the same as the philosopher of
Megara'' etc.

Another idea, that Euclid was born at Gela in Sicily, is due to the
same confusion, being based on Diogenes Laertius'
description\footnote{Diog.\ L.\ \r2.~106, p.~58 ed.\ Cobet.} of
the philosopher Euclid as being ``of Megara, or, according to some, of
Gela, as Alexander says in the \greek{Διαδοχαί}.''

In view of the poverty of Greek tradition on the subject even as early
as the time of Proclus (410–485~\ad), we must necessarily take
\emph{cum grano} the apparently circumstantial accounts of Euclid
given by Arabian authors; and indeed the origin of their stories can
be explained as the result (1)~of the Arabian tendency to romance, and
(2)~of misunderstandings.

We read\footnote{Casiri, \emph{Bibliotheca Arabico-Hispana
    Escurialensis}, \r1.~p.~339, Casiri's source is al-Qifṭī
  (d.~1248), the author of the \emph{Ta'rīkh al-Ḥukamā}, a collection
  of biographies of philosophers, mathematicians, astronomers etc.}
that ``Euclid, son of Naucrates, grandson of Zenarchus\footnote{The
  \emph{Fihrist} says ``son of Naucrates, the son of Berenice~(?)''
  (see Suter's translation in \emph{Abhandlungren zur
    Gesch.\ d.\ Math}. Heft, 1892. p.~16).}, called the author of
geometry, a philosopher of somewhat ancient date, a Greek by
nationality domiciled at Damascus, born at Tyre, most learned in the
science of geometry, published a most excellent and most useful work
entitled the foundation or elements of geometry, a subject in which no
more general treatise existed before among the Greeks: nay, there was
no one even of later date who did not walk in his footsteps and
frankly profess his doctrine. Hence also Greek, Roman and Arabian
geometers not a few, who undertook the task of illustrating this work,
published commentaries, scholia, and notes upon it, and made an
abridgment of the work itself. For this reason the Greek philosophers
used to post up on the doors of their schools the well-known notice:
`Let no one come to our school, who has not first learned the elements
of Euclid.'\,” The details at the beginning of this extract cannot be
derived from Greek sources, for even Proclus did not know anything
about Euclid's father, while it was not the Greek habit to record the
names of grandfathers, as the Arabians commonly did. Damascus and Tyre
were no doubt brought in to gratify a desire which the Arabians always
showed to connect famous Greeks in some way or other with the
East. Thus Naṣīraddīn, the translator of the \emph{Elements}, who was
of Ṭūs in Khurāsān, actually makes Euclid out to have been
``Thusinus'' also\footnote{The same predilection made the Arabs
  describe Pythagoras as a pupil of the wise Salomo, Hipparchus as the
  exponent of Chaldaean philosophy or as the Chaldaean, Archimedes as
  an Egyptian etc. (Ḥājī Khalfa, \emph{Lexicon Bibliographicum}, and
  Casiri).}. The readiness of the Arabians to run away with an idea is
illustrated by the last words of the extract. Everyone knows the story
of Plato's inscription over the porch of the Academy: ``let no one
unversed in geometry enter my doors''; the Arab turned geometry into
\emph{Euclid's} geometry, and told the story of Greek philosophers in
general and ''\emph{their} Academies.''

Equally remarkable are the Arabian accounts of the relation of Euclid
and Apollonius\footnote{The authorities for these statements quoted by
  Casiri and Ḥājī Khalfa are al-Kindī's tract \emph{de instituto libri
    Euclidis} (al-Kindī died about 873) and a commentary by Qāḍīzāde
  ar-Rūmī (d.\ about 1440) on a book called \emph{Ashkāl at-ta' sīs}
  (fundamental propositions) by Ashraf Shanuaddīn as-Samarqandī
  (\emph{c}.~1176) consisting of elucidations of 35 propositions
  selected from the first books of Euclid.  Naṣīraddīn likewise says
  that Euclid cut out two of 15 books of elements then existing and
  published the rest under his own name.  According to Qāḍīzāde the
  king heard that there was a celebrated geometer named Euclid at
  \emph{Tyre}: Naṣīraddīn says that he sent for Euclid of Ṭūs.}.
According to them the \emph{Elements} were originally written, not by
Euclid, but by a man whose name was Apollonius, a carpenter, who wrote
the work in 15 books or sections\footnote{So says the \emph{Fihrist},
  Suter (\emph{op.\ cit.}  p.~49) thinks that the author of the
  \emph{Fihrist} did not suppose Apollonius \emph{of Perga} to be the
  writer of the \emph{Elements}, as later Arabian authorities did, but
  that he distinguished another Apollonius whom he calls ``a
  carpenter.''  Suter's argument is based on the fact that the
  \emph{Fihrist}'s article on Apollonius (of Perga) says nothing of
  the \emph{Elements}; and that it gives the three great
  mathematicians, Euclid, Archimedes and Apollonius, in the correct
  chronological order.}.  In the course of time some of the work was
lost and the rest became disarranged, so that one of the kings at
Alexandria who desired to study geometry and to master this treatise
in particular first questioned about it certain learned men who
visited him and then sent for Euclid who was at that time famous as a
geometer, and asked him to revise and complete the work and reduce it
to order. Euclid then rewrote it in 13 books which were thereafter
known by his name. (According to another version Euclid composed the
13 books out of commentaries which he had published on two books of
Apollonius on conics and out of introductory matter added to the
doctrine of the five regular solids.) To the thirteen books were added
two more books, the work of others (though some attribute these also
to Euclid) which contain several things not mentioned by Apollonius.
According to another version Hypsicles, a pupil of Euclid at
Alexandria, offered to the king and published Books \book{xiv} and
\book{xv}, it being also stated that Hypsicles had ``discovered'' the
books, by which it appears to be suggested that Hypsicles had edited
them from materials left by Euclid.

We observe here the correct statement that Books \book{xiv} and
\book{xv} were not written by Euclid, but along with it the incorrect
information that Hypsicles, the author of Book \book{xiv}, wrote Book
\book{xv} also.

The whole of the fable about Apollonius having preceded Euclid and
having written the \emph{Elements} appears to have been evolved out of
the preface to Book \book{xiv} by Hypsicles, and in this way; the Book
must in early times have been attributed to Euclid, and the inference
based upon this assumption was left uncorrected afterwards when it was
recognised that Hypsicles was the author. The preface is worth
quoting:

``Basilides of Tyre, O Protarchus, when he came to Alexandria and met
my father, spent the greater part of his sojourn with him on account
of their common interest in mathematics. And once, when examining the
treatise written by Apollonius about the comparison between the
dodecahedron and the icosahedron inscribed in the same sphere,
(showing) what ratio they have to one another, they thought that
Apollonius had not expounded this matter properly, and accordingly
they emended the exposition, as I was able to learn from my
father. And I myself, later, fell in with another book published by
Apollonius, containing a demonstration relating to the subject, and I
was greatly interested in the investigation of the problem. The book
published by Apollonius is accessible to all—for it has a large
circulation, having apparently been carefully written out later—but I
decided to send you the comments which seem to me to be necessary, for
you will through your proficiency in mathematics in general and in
geometry in particular form an expert judgment on what I am about to
say, and you will lend a kindly ear to my disquisition for the sake of
your friendship to my father and your goodwill to me.''

The idea that Apollonius preceded Euclid must evidently have been
derived from the passage just quoted. It explains other things
besides. Basilides must have been confused with \greek{Βασιλεύς}, and
we have a probable explanation of the ``Alexandrian king,'' and of the
``learned men who visited'' Alexandria. It is possible also that in
the ``Tyrian'' of Hypsicles' preface we have the origin of the notion
that Euclid was born in Tyre. These inferences argue, no doubt, very
defective knowledge of Greek: but we could expect no better from those
who took the \emph{Organon} of Aristotle to be ``instrumentum musicum
pneumaticum,'' and who explained the name of Euclid, which they
variously pronounced as \emph{Uclides} or \emph{Icludes}, to be
compounded of \emph{Ucli} a key, and \emph{Dis} a measure, or, as some
say, geometry, so that \emph{Uclides} is equivalent to the \emph{key
  of geometry}!

Lastly the alternative version, given in brackets above, which says
that Euclid made the \emph{Elements} out of commentaries which he
wrote on two books of Apollonius on conics and prolegomena added to
the doctrine of the five solids, seems to have arisen, through a like
confusion, out of a later passage\footnote{Heiberg's Euclid,
  vol.~v.~p.~6.} in Hypsicles' Book \book{xiv}: ``And this is
expounded by Aristaeus in the book entitled 'Comparison of the five
figures,' and by Apollonius in the second edition of his comparison of
the dodecahedron with the icosahedron.'' The ``doctrine of the five
solids'' in the Arabic must be the ``Comparison of the five figures''
in the passage of Hypsicles, for nowhere else have we any information
about a work bearing this title, nor can the Arabians have had. The
reference to the \emph{two books} of Apollonius on \emph{conics} will
then be the result of mixing up the fact that Apollonius wrote a book
on conics with the \emph{second edition} of the other work mentioned
by Hypsicles.  We do not find elsewhere in Arabian authors any mention
of a commentary by Euclid on Apollonius and Aristaeus: so that the
story in the passage quoted is really no more than a variation of the
fable that the \emph{Elements} were the work of Apollonius.

\chapter{Euclid's Other Works}

In giving a list of the Euclidean treatises other than the \emph{Elements},
I shall be brief: for fuller accounts of them, or speculations with
regard to them, reference should be made to the standard histories of
mathematics\footnote{See, for example, Loria, \emph{Le scienze esatte nell' antica
  Grecia}, 1914, pp.~245–268; T. L. Heath, \emph{History of Greek
  Mathematics}, 1921, \r1.~pp.~421–446. Cf. Heiberg,
  \emph{Litterargeschichtliche Studien über Euklid}, pp.~36–153;
  \emph{Euclidis opera omnia}, ed.\ Heiberg and Menge,
  Vols.\ \r6.–\r8.}.

I will take first the works which are mentioned by Greek authors,

\section{The \emph{Pseudaria}}

I mention this first because Proclus refers to it in the general
remarks in praise of the \emph{Elements} which he gives immediately after
the mention of Euclid in his summary. He says\footnote{Proclus, p.~70, 1–18.}; ``But, inasmuch
as many things, while appearing to rest on truth and to follow from
scientific principles, really tend to lead one astray from the principles
and deceive the more superficial minds, he has handed down methods
for the discriminative understanding of these things as well, by the
use of which methods we shall be able to give beginners in this study
practice in the discovery of paralogisms, and to avoid being misled.
This treatise, by which he puts this machinery in our hands, he
entitled (the book) of Pseudaria, enumerating in order their various
kinds, exercising our intelligence in each case by theorems of all
sorts, setting the true side by side with the false, and combining
the refutation of error with practical illustration. This book then is
by way of cathartic and exercise, while the Elements contain the
irrefragable and complete guide to the actual scientific investigation
of the subjects of geometry.''

The book is considered to be irreparably lost. We may conclude
however from the connexion of it with the \emph{Elements} and the reference
to its usefulness for beginners that it did not go outside the domain
of elementary geometry\footnote{Heiberg points out that Alexander Aphrodisiensis appears to allude
  to the work in his commentary on Aristotle's \emph{Sophistici
  Elenchi} (fol.~25~\emph{b}): ``Not only those (\greek{ἔλεγχοι})
  which do not start from the principles of the science under which
  the problem is classed\dots but also those which do start from the
  proper principles of the science but in some respect admit a
  paralogism, e.g.\ the \emph{Pseudographemata} of Euclid.'' Tannery
  (\emph{Bull. des sciences math. et astr.} 2\tsup{e} Série,
  \textsc{vi}., 1882, 1\tsup{ère} Partie, p.~147) conjectures that it
  may be from this treatise that the same commentator got his
  Information about the quadratures of the circle by Antiphon and
  Bryson, to say nothing of the lunules of Hippocrates. I think
  however that there is is objection to this theory so far as regards
  Bryson; for Alexander distinctly says that Bryson's quadrature did
  \emph{not} start from the proper principles of geometry, but from
  some principles more general.}.

\section{The \emph{Data}}

The \emph{Data} (\greek{δεδομένα}) are included by Pappus in the
\emph{Treasury of Analysis} (\greek{}), and he describes their
contents\footnote{Pappus, \r7.\ p.~638.}.  They are still concerned with elementary
  geometry, though forming part
of the introduction to higher analysis. Their form is that of
propositions proving that, if certain things in a figure are given (in
magnitude, in species, etc.), something else is given. The
subject-matter is much the same as that of the planimetrical books of the
\emph{Elements}, to which the \emph{Data} are often supplementary. We shall see
this later when we come to compare the propositions in the \emph{Elements}
which give us the means of solving the general quadratic equation
with the corresponding propositions of the \emph{Data} which give the
solution. The \emph{Data} may in fact be regarded as elementary
  exercises
in analysis.

It is not necessary to go more closely into the contents, as we
have the full Greek text and the commentary by Marinus newly
edited by Menge and therefore easily accessible\footnote{Vol. \r6. in
  the Teubner edition of \emph{Euclidis opera omnia} by Heiberg and
  Menge. A translation of the \emph{Data} is also included in Simson's
  Euclid (though naturally his text left much to be desired).}.

\section{The book \emph{On divisions (of figures)}}

This work (\greek{περὶ διαιρέσεων βιβλίον}) is mentioned by
  Proclus\footnote{Proclus, p.~69, 4.}.
In one place he is speaking of the conception or definition (\greek{λόγος})
of figure, and of the divisibility of a figure into others differing from
it in kind; and he adds: ``For the circle is divisible into parts unlike
in definition or notion (\greek{ἀνόμοια τῷ λόγῳ}), and so is each of
  the
rectilineal figures; this is in fact the business of the writer of the
Elements in his Divisions, where he divides given figures, in one case
into like figures, and in another into unlike\footnote{\ibid\ 144,
  22–26.}.'' ``Like'' and ``unlike''
here mean, not ``similar'' and ``dissimilar'' in the technical sense, but
``like'' or ``unlike \emph{in definition} or \emph{notion}''
  (\greek{λόγῳ}): thus to divide a
triangle into triangles would be to divide it into ``like'' figures, to
divide a triangle into a triangle and a quadrilateral would be to
divide it into ``unlike'' figures.

The treatise is lost in Greek but has been discovered in the
Arabic.  First John Dee discovered a treatise \emph{De divisionibus}
  by one
Muhammad Bagdadinus\footnote{Stetnschneider places him in the 10th
  c. H. Suter (\emph{Bibliotheca Mathematica}, \r4\tsub{3}, 1903,
  pp.~24, 27) identifies him with Abū (Bekr) Muḥ. b. `Abdalbāqī
  al-Baġdādī, Qādī (Judge) of Māristān (\emph{circa} 1070–1141), to
  whom he also attributes the \emph{Liber judei \emph{(?judicis)}
  super decimum Euclidis} translated by Gherard of Cremona.} and
  handed over a copy of it (in Latin) in
1563 to Commandinus. who published it, in Dee's name and his own,
in 1570\footnote{\emph{De superficierum divisionibus liber Machometo
  Bagdadino adscriptus, nunc primum Ioannis Dee Londinensis et
  Federici Commandini Urbinatis opera in lucem editus}, Pisauri, 1570,
  afterwards included in Gregory's Euclid (Oxford, 1703).}.  Dee did
  not himself translate the tract from the Arabic; he
found it in Latin in a \textsc{ms.}\ which was then in his own
  possession but
was about 20 years afterwards stolen or destroyed in an attack by a
mob on his house at Mortlake\footnote{R.~C. Archibald, \emph{Euclid's
  Book on the Division of Figures with a restoration based on
  Woepcke's text and on the Practica geometriae of Leonardo Pisano},
  Cambridge, 1915, pp.~4–9.}. Dee, in his preface addressed to
Commandinus, says nothing of his having \emph{translated} the book, but
only remarks that the very illegible \textsc{ms.}\ had caused him much trouble
and (in a later passage) speaks of ``the actual, very ancient, copy
  from
which I \emph{wrote out}…'' (in ipso unde descripsi vetustissimo
  exemplari).
The Latin translation of this tract from the Arabic was probably made
by Gherard of Cremona (1114–1187), among the list of whose numerous
translations a ``liber divisionum'' occurs.  The Arabic original
  cannot
have been a direct translation from Euclid, and probably was not even
a direct adaptation of it; it contains mistakes and unmathematical
expressions, and moreover does not contain the propositions about
the division of a circle alluded to by Proclus. Hence it can scarcely
have contained more than a fragment of Euclid's work.

But Woepcke found in a \textsc{ms.}\ at Paris a treatise in Arabic on
  the
division of figures, which he translated and published in
  1851\footnote{\emph{Journal Asiatique}, 1851, p.~233~sqq.}. It is
expressly attributed to Euclid in the \textsc{ms.}\ and corresponds to
  the
description of it by Proclus. Generally speaking, the divisions are
divisions into figures of the same kind as the original figures,
  e.g.\ of
triangles into triangles; but there are also divisions into ``unlike ``
figures, e.g.\ that of a triangle by a straight line parallel to the base.
The missing propositions about the division of a circle are also here:
``to divide into two equal parts a given figure bounded by an arc
of a circle and two straight lines including a given angle'' and ``to
draw in a given circle two parallel straight lines cutting off a certain
part of the circle.'' Unfortunately the proofs are given of only four
propositions (including the two last mentioned) out of~36, because
the Arabic translator found them too easy and omitted them. To
illustrate the character of the problems dealt with I need only take
one more example: ``To cut off a certain fraction from a
  (parallel-)trapezium by a straight line which passes through a given
  point lying
inside or outside the trapezium but so that a straight line can be
drawn through it cutting both the parallel sides of the trapezium.''
The genuineness of the treatise edited by Woepcke is attested by the
facts that the four proofs which remain are elegant and depend on
propositions in the \emph{Elements}, and that there is a lemma with a
  true
Greek ring: ``to apply to a straight line a rectangle equal to the
rectangle contained by \emph{$AB$, $AC$ and deficient by a square}.''
  Moreover
the treatise is no fragment, but finishes with the words ``end of the
treatise,'' and is a well-ordered and compact whole. Hence we may
safely conclude that Woepcke's is not only Euclid's own work but
the whole of it. A restoration of the work, with proofs, was attempted
by Ofterdinger\footnote{L. F. Ofterdinger, \emph{Beiträge zur
  Wiederherstellung der Schrift des Euklides über die Theilung der
  Figuren}, Ulm, 1853.}, who however does not give Woepcke's
  props. 30, 31, 34, 35, 36.  We have now a satisfactory restoration,
  with ample notes and an introduction, by R.~C. Archibald, who used
  for the purpose Woepcke's text and a section of Leonardo of Pisa's
  \emph{Practica geometriae} (1220)\footnote{There is a remarkable
  similarity between the propositions of Woepcke's text and those of
  Leonardo, suggesting that Leonardo may have had before him a
  translation (perhaps by Gherard of Cremona) of the Arabic tract.}.

\section{The \emph{Porisms}}

It is not possible to give in this place any account of the
controversies about the contents and significance of the three lost books
of Porisms, or of the important attempts by Robert Simson and
Chasles to restore the work. These may be said to form a whole
literature, references to which will be found most abundantly given
by Heiberg and Loria, the former of whom has treated the subject
from the philological point of view, most exhaustively, while the
latter, founding himself generally on Heiberg, has added useful
details, from the mathematical side, relating to the attempted
restorations, etc.\footnote{Heiberg, \emph{Euklid-Studien}, pp.~56–79,
  and Loria, \emph{op.\ cit.}, pp.~253–265.} It must suffice here to
  give an extract from the only
original source of information about the nature and contents of the
\emph{Porisms}, namely Pappus\footnote{Pappus, ed. Hultsch,
  \r7.\ pp.~648–660. I put in square brackets the words bracketed by
  Hultsch.}. In his general preface about the books
composing the \emph{Treasury of Analysis} (\greek{τόπος ἀναλυόμενος})
  he says:

``After the Tangencies (of Apollonius) come, in three books, the
Porisms of Euclid, [in the view of many] a collection most ingeniously
devised for the analysis of the more weighty problems, [and] although
nature presents an unlimited number of such porisms\footnote{I adopt
  Heiberg's reading of a comma here instead of a full stop.}, [they
  have
added nothing to what was written originally by Euclid, except that
some before my time have shown their want of taste by adding to a
few (of the propositions) second proofs, each (proposition) admitting
of a definite number of demonstrations, as we have shown, and
Euclid having given one for each, namely that which is the most
lucid. These porisms embody a theory subtle, natural, necessary,
and of considerable generality, which is fascinating to those who can
see and produce results].

``Now all the varieties of porisms belong, neither to theorems nor
problems, but to a species occupying a sort of intermediate position
[so that their enunciations can be formed like those of either theorems
or problems], the result being that, of the great number of geometers,
some regarded them as of the class of theorems, and others of pro-
blems, looking only to the form of the proposition. But that the
ancients knew better the difference between these three things is
clear from the definitions. For they said that a theorem is that
which is proposed with a view to the demonstration of the very
thing proposed, a problem that which is thrown out with a view to
the construction of the very thing proposed, and a porism that which
is proposed with a view to the producing of the very thing proposed.
[But this definition of the porism was changed by the more recent
writers who could not produce everything, but used these elements
and proved only the fact that that which is sought really exists, but
did not produce it\footnote{\begin{wrapfigure}{r}{0pt}
\includegraphics{introI_1}
\end{wrapfigure}
Heiberg points out that Props.~5–9 of Archimedes' treatise \emph{On
  Spirals} are porisms in this sense. To take Prop.~5 as an example,
  $DBF$ is a tangent to a circle with centre $K$. It is then possible,
  says Archimedes, to draw a straight line $KHF$, meeting the
  circumference in $H$ and the tangent in~$F$, such that
  \[ FH : HK < (\arc BH) : c \]
  where $c$ is the circumference of \emph{any} circle. To prove this
  he assumes the following construction. $E$ being any straight line
  greater than $c$, he says: let $KG$ be parallel to $DF$, ``and let
  the line $GH$ equal to $E$ be placed \emph{verging} to the point
  $B$.''  Archimedes must of course nave known how to effect this
  construction, which requires conics. But that it is \emph{possible}
  requires very little argument, for if we draw any straight line
  $BHG$ meeting the circle in $H$ and $KG$ in $G$, it is obvious that
  as $G$ moves away from~$C$, $HG$ becomes greater and greater and may
  be made as great as we please. The ``later writers ``would no doubt
  have contented themselves with this consideration without actually
  \emph{constructing} $HG$.} and were accordingly confuted by the
  definition
and the whole doctrine. They based their definition on an incidental
characteristic, thus: A porism is that which falls short of a
locus-theorem in respect of its hypothesis\footnote{As Heiberg says,
  this translation is made certain by a preceding passage of Pappus
  (p.~648, 1–3) where he compares two enunciations, the latter of
  which ``falls short of the former in \emph{hypothesis} but goes
  beyond it in \emph{requirement}.''  E.g.\ the first enunciation
  requiring us, given three circles, to draw a circle touching all
  three, the second may require us, given only two circles (one less
  datum), to draw a circle touching them and \emph{of a given size}
  (an extra requirement).}. Of this kind of porisms loci
are a species, and they abound in the Treasury of Analysis; but
this species has been collected, named and handed down separately
from the porisms, because it is more widely diffused than the other
species]. But it has further become characteristic of porisms that,
owing to their complication, the enunciations are put in a contracted
form, much being by usage left to be understood; so that many
geometers understand them only in a partial way and are ignorant of
the more essential features of their contents,

``[Now to comprehend a number of propositions in one enunciation
is by no means easy in these porisms, because Euclid himself has not
in fact given many of each species, but chosen, for examples, one or a
few out of a great multitude\footnote{I translate Heiberg's reading
  with a full stop here followed by \greek{πρὸς ἀρχῇ δὲ ὄμως}
  [\greek{πρὸς ἀρχὴν (δεδομένον)} Hultsch] \greek{τοῦ πρώτου
  βιβλίου}\dots.}. But at the beginning of the first book
he has given some propositions, to the number of ten, of one species,
namely that more fruitful species consisting of loci.]  Consequently,
finding that these admitted of being comprehended in one enunciation,
we have set it out thus:
\begin{quote}
If, in a system of four straight lines\footnote{The four straight
  lines are described in the text as (the sides) \greek{ὑπτίου ἤ
  παρυπτίου} i.e. sides of two' sorts of quadrilaterals which Simson
  tries to explain (see p.~120 of the \emph{Index Graecitatis} of
  Hultsch's edition of Pappus).} which cut each other
two and two, three points on one straight line be given while the
rest except one lie on different straight lines given in position,
the remaining point also will lie on a straight line given in
position\footnote{In other words (Chasles, p.~23; Loria, p.~256), if a
  triangle be so deformed that each of its sides turns about one of
  three points in a straight line, and two of its vertices lie on two
  straight lines given in position, the third vertex will also lie on
  a straight line.}.
\end{quote}

``This has only been enunciated of four straight lines, of which not
more than two pass through the same point, but it is not known (to
most people) that it is true of any assigned number of straight lines
if enunciated thus:
\begin{quote}
If any number of straight lines cut one another, not more
than two (passing) through the same point, and all the points
(of intersection situated) on one of them be given, and if each of
those which are on another (of them) lie on a straight line given
in position—
\end{quote}
or still more generally thus:
\begin{quote}
if any number of straight lines cut one another, not more than
two (passing) through the same point, and all the points (of
intersection situated) on one of them be given, while of the other
points of intersection in multitude equal to a triangular number
a number corresponding to the side of this triangular number lie
respectively on straight lines given in position, provided that of
these latter points no three are at the angular points of a triangle
(\emph{sc.}\ having for sides three of the given straight lines)—each
  of the
remaining points will lie on a straight line given in
  position\footnote{Loria (p.~256, \emph{n}.~3) gives the meaning of
  this as follows, pointing out that Simson was the discoverer of it:
  ``If a complete $n$-lateral be deformed so that its sides
  respectively turn about $n$ points on a straight line, and $n - 1$
  of its $n(n - 1)/2$ vertices move on as many straight lines, the
  other $(n-1_(n-2)/2$ of its vertices likewise move on as many
  straight lines; but it is necessary that it should be impossible to
  form with the $(n - 1)$ vertices any triangle having for sides the
  sides of the polygon.''}.
\end{quote}
``It is probable that the writer of the Elements was not unaware
of this but that he only set out the principle; and he seems, in the
case of all the porisms, to have laid down the principles and the
seed only [of many important things], the kinds of which should be
distinguished according to the differences, not of their hypotheses, but
of the results and the things sought [All the hypotheses are different
from one another because they are entirely special, but each of the
results and things sought, being one and the same, follow from many
different hypotheses.]

``We must then in the first book distinguish the following kinds of
things sought:
\begin{quote}

``At the beginning of the book\footnote{Reading, with Heiberg,
  \greek{τοῦ Βιβλίου} [\greek{τοῦ ζ’} Hultsch].} is this proposition:

I. `\emph{If from two given points straight tines be drawn meeting
on a straight line given in position, and one cut off from a straight
line given in position (a segment measured) to a given point on it,
the other will also cut off from another (straight line a segment)
liaving to the first a given ratio.}'

``Following on this (we have to prove)

II. that such and such a point lies on a straight line given
in position;

III. that the ratio of such and such a pair of straight lines
is given;''

etc. etc, (up to \r39.).
\end{quote}

``The three books of the porisms contain 38 lemmas; of the
theorems themselves there are 171.''

Pappus further gives lemmas to the \emph{Porisms} (pp.~866–918, ed.\
Hultsch).

With Pappus' account of Porisms must be compared the passages
of Proclus on the same subject.  Proclus distinguishes two senses in
which the word \greek{pórisma} is used. The first is that of
  \emph{corollary} where
something appears as an incidental result of a proposition, obtained
without trouble or special seeking, a sort of bonus which the
investigation has presented us with\footnote{Proclus, pp.~212, 14;
  301, 22.}.  The other sense is that of Euclid's
\emph{Porisms}\footnote{\ibid~pp.212, 12. ``The term porism is used of
  certain problems, like the \emph{Porisms} written by Euclid.''}, In
  this sense\footnote{\ibid~pp.~301, 25 sqq.} ``\emph{porism} is the
  name given to things which
are sought, but need some finding and are neither pure bringing into
existence nor simple theoretic argument. For (to prove) that the
angles at the base of isosceles triangles are equal is a matter of
theoretic argument, and it is with reference to things existing that
such knowledge is (obtained).  But to bisect an angle, to construct a
triangle, to cut off, or to place—all these things demand the making
of something; and to find the centre of a given circle, or to find the
greatest common measure of two given commensurable magnitudes,
or the like, is in some sort between theorems and problems.  For in
these cases there is no bringing into existence of the things sought,
but finding of them, nor is the procedure purely theoretic.  For it is
necessary to bring that which is sought into view and exhibit it to
the eye. Such are the porisms which Euclid wrote, and arranged in
three books of Porisms.''

Proclus' definition thus agrees well enough with the first, ``older,''
definition of Pappus. A porism occupies a place between a theorem
and a problem: it deals with something already \emph{existing}, as a theorem
does, but has to \emph{find} it (e.g.\ the centre of a circle), and,
  as a certain
operation is therefore necessary, it partakes to that extent of the
nature of a problem, which requires us to construct or produce
something not previously existing. Thus, besides \prop{3}{1} of the
  \emph{Elements}
and \prop{10}{3, 4} mentioned by Proclus, the following propositions
  are real porisms: \prop{3}{25}, \prop{6}{11–13}, \prop{7}{33, 34,
  36, 39}, \prop{8}{2, 4}, \prop{10}{10}, \prop{13}{18}. Similarly in
  Archimedes \emph{On the Sphere and Cylinder} \prop{1}{2–6}
might be called porisms.

The enunciation given by Pappus as comprehending ten of Euclid's
propositions may not reproduce the \emph{form} of Euclid's enunciations;
but, comparing the result to be proved, that certain points lie on
straight lines given in position, with the class indicated by
  \r2.\ above,
where the question is of such and such a point lying on a straight line
given in position, and with other classes, e.g.\ (\r5.) that such and such a
line is given in position, (\r6.) that such and such a line verges to
  a given
point, (\r23.) that there exists a given point such that straight
  lines
drawn from it to such and such (circles) will contain a triangle given
in species, we may conclude that a usual form of a porism was ``to
prove that it is possible to find a point with such and such a property''
or ``a straight line on which lie all the points satisfying given
conditions'' etc.

Simson defined a porism thus: ``Porisma est propositio in qua
proponitur demonstrare rem aliquant, vel plures datas esse, cui, vel
quibus, ut et cuilibet ex rebus innumeris, non quidem datis, sed quae
ad ea quae data sunt eandem habent relationem, convenire ostendendum
est affectionem quandam communem in propositione
descriptam\footnote{This was thus expressed by Chasles: ``Le porisme
  est une proposition dans laquelle on demande de démontrer qu'une
  chose ou plusieurs choses sont \emph{données}, qui, ainsi que l'une
  quelconque d'une infinité d'autres choses non données, mais dont
  chacune est avec des choses données dans une même relation, ont une
  certaine propriété commune, décrite dans la proposition.''}.''

From the above it is easy to understand Pappus' statement that
\emph{loci} constitute a large class of porisms. A \emph{locus} is
well defined by Simson thus: ``Locus est propositio in qua propositum
est datam esse demonstrare, vel invenire lineam aut superficiem cuius
quodlibet punctum, vel superficiem in qua quaelibet linea data lege
descripta, communem quandam habet proprietatem in propositione
descriptam.''  Heiberg cites an excellent instance of a \emph{locus}
  which is a \emph{porism}, namely the following proposition quoted by
  Eutocius\footnote{Commentary on Apollonius' \emph{Conics}
    (vol.~\r2.~p.~180, ed.\ Heiberg).} from the \emph{Plane Loci} of
  Apollonius:

``Given two points in a plane, and a ratio between unequal straight
  lines, it is possible to draw, in the plane, a circle such that the
  straight lines drawn from the given points to meet on the
  circumference of the circle have (to one another) a ratio the same
  as the given ratio,''

A difficult point, however, arises on the passage of Pappus, which
says that a porism is ``that which, in respect of its hypothesis,
falls short of a locus-theorem ``(\greek{τοπικοῦ θεωρήματος}). Heiberg
explains it by comparing the porism from Apollonius' \emph{Plane Loci}
just given with Pappus' enunciation of the same thing, to the effect
that, if from two given points two straight lines be drawn meeting in
a point, and these straight lines have to one another a given ratio,
the point will lie on either a straight line or a circumference of a
circle given in position.  Heiberg observes that in this latter
enunciation something is taken into the hypothesis which was not in
the hypothesis of the enunciation of the porism, viz.\ ``that the
ratio of the straight lines is the same.''  I confess this does not
seem to me satisfactory: for there is no real difference between the
enunciations, and the supposed difference in hypothesis is very like
playing with words. Chasles says: ``\emph{Ce qui constitue le porisme
  est ce qui manque à l'hypothèse d'un théorème local} (en d'autres
termes, le porisme est inférieur, par l'hypothèse, au théorème local;
e'est-á-dire que quand quelques parties d'une proposition locale n'ont
pas dans l'énoncé la détermination qui ieur est propre, cette
proposition cesse d'être regardée comme un théorème et devient un
porisme).'' But the subject still seems to require further
elucidation.

While there is so much that is obscure, it seems certain (1)~that the
\emph{Porisms} were distinctly part of higher geometry and not of
elementary geometry, (2)~that they contained propositions belonging to
the modern theory of transversals and to projective geometry. It
should be remembered too that it was in the course of his researches
on this subject that Chasles was led to the idea of \emph{anharmonic
  ratios}.

Lastly, allusion should be made to the theory of
Zeuthen\footnote{\emph{Die Lehre von den Kegelschnitten im Altertum},
  chapter~\r8.}  on the subject of the porisms. He observes that the
only porism of which Pappus gives the complete enunciation, ``If from
two given points straight lines be drawn meeting on a straight line
given in position, and one cut off from a straight line given in
position (a segment measured) towards a given point on it, the other
will also cut off from another (straight line a segment) bearing to
the first a given ratio,'' is also true if there be substituted for
the first given straight line a conic regarded as the ``locus with
respect to four lines,'' and that this extended porism can be used for
completing Apollonius' exposition of that locus. Zeuthen concludes
that the \emph{Porisms} were in part by-products of the theory of
conics and in part auxiliary means for the study of conics, and that
Euclid called them by the same name as that applied to corollaries
because they were corollaries with respect to conics. But there
appears to be no evidence to confirm this conjecture.

\section{The \emph{Surface-loci} (\greek{τόποι πρὸς ἐπιφανείᾳ})}

The two books on this subject are mentioned by Pappus as part of the
\emph{Treasury of Analysis}\footnote{Pappus, \r7.\ p.~636.}.  As the
other works in the list which were on plane subjects dealt only with
straight lines, circles, and conic sections, it is \emph{a priori}
likely that among the loci in this treatise (loci which are surfaces)
were included such loci as were cones, cylinders and spheres. Beyond
this all is conjecture based on two lemmas given by Pappus in
connexion with the treatise.

(1)~The first of these lemmas\footnote{\ibid~\r7.~p.~1004.} and the
figure attached to it are not satisfactory as they stand, but a
possible restoration is indicated by Tannery\footnote{\emph{Bulletin
    des sciences math.\ et astron.}, 2\tsup{e} Série, \r6.~149.}.  If
the latter is right, it suggests that one of the loci contained all
the points on the elliptical parallel sections of a cylinder and was
therefore an oblique circular cylinder.  Other assumptions with regard
to the conditions to which the lines in the figure may be subject
would suggest that other loci dealt with were cones regarded as
containing all points on particular elliptical parallel sections of
the cones\footnote{Further particulars will be found in \emph{The
    Works ef Archimedes}, pp.~lxii–lxiv, and in Zeuthen, \emph{Die
    Lehre von Kegelschnitten}, p.~425~sqq.}.

(2)~In the second lemma Pappus states and gives a complete proof of
the focus-and-directrix property of a conic, viz.\ that \emph{the
  locus of a point whose distance from a given point is in a given
  ratio to its distance from a fixed line is a conic section, which is
  an ellipse, a parabola or a hyperbola according as the given ratio
  is less than, equal to, or greater than unity}\footnote{Pappus,
  \r7.~pp.~1006–1014, and Hultsch's Appendix, pp.~1270–3.}.  Two
conjectures are possible as to the application of this theorem in
Euclid's \emph{Surface-loci}, (\emph{a})~It may have been used to
prove that the locus of a point whose distance from a given straight
line is in a given ratio to its distance from a given plane is a
certain cone.  (\emph{b})~It may have been used to prove that the
locus of a point whose distance from a given point is in a given ratio
to its distance from a given plane is the surface formed by the
revolution of a conic about its major or conjugate axis\footnote{For
  further details see \emph{The Works of Archimedes}, pp.~lxiv, lxv,
  and Zeuthen, \emph{l.~c.}}.  Thus Chasles may have been correct in
his conjecture that the \emph{Surface-loci} dealt with surfaces of
revolution of the second degree and sections of the
same\footnote{\emph{Aperçu historique}, pp.~273–44.}.

\section{The \emph{Conics}}

Pappus says of this lost work: ``The four books of Euclid's Conics
were completed by Apollonius, who added four more and gave us eight
books of Conics\footnote{Pappus, \r7.~p.~672.}.'' It is probable that
Euclid's work was lost even by Pappus' time, for he goes on to speak
of ``Aristaeus, who wrote the \emph{still extant} five books of Solid
Loci connected with the conics.''  Speaking of the relation of
Euclid's work to that of Aristaeus on conics regarded as loci, Pappus
says in a later passage (bracketed however by Hultsch) that Euclid,
regarding Aristaeus as deserving credit for the discoveries he had
already made in conics, did not (try to) anticipate him or construct
anew the same system. We may no doubt conclude that the book by
Aristaeus on solid loci preceded Euclid's on conics and was, at least
in point of originality, more important.  Though both treatises dealt
with the same subject-matter, the object and the point of view were
different; had they been the same, Euclid could scarcely have
refrained, as Pappus says he did, from attempting to improve upon the
earlier treatise.  No doubt Euclid wrote on the general theory of
conics as Apollonius did, but confined himself to those properties
which were necessary for the analysis of the \emph{Solid Loci} of
Aristaeus. The \emph{Conics} of Euclid were evidently superseded by
the treatise of Apollonius.

As regards the contents of Euclid's \emph{Conics}, the most important
source of our information is Archimedes, who frequently refers to
propositions in conics as well known and not needing proof, adding in
three cases that they are proved in the ``elements of conics'' or in
``the conics,'' which expressions must clearly refer to the works of
Aristaeus and Euclid\footnote{For details of these propositions see my
  \emph{Apollonius of Perga}, pp.~xxxv, xxxvi.}.

Euclid still used the old names for the conics (sections of a
right-angled, acute-angled, or obtuse-angled cone), but he was aware
that an ellipse could be obtained by cutting a cone in any manner by a
plane not parallel to the base (assuming the section to lie wholly
between the apex of the cone and its base) and also by cutting a
cylinder. This is expressly stated in a passage from the
\emph{Phaenomena} of Euclid about to be
mentioned\footnote{\emph{Phaenomena}, ed.\ Menge, p.~6: ``if a cone or
  a cylinder be cut by a plane not parallel to the base, the section
  is a section of an acute-angled cone, which is like a shield
  (\greek{θυρεός}).''}.

\section{The \emph{Phaenomena}}

This is an astronomical work and is still extant. A much interpolated
version appears in Gregory's Euclid.  An earlier and better recension
is however contained in the \textsc{ms.}\ Vindobonensis
philos.\ Gr.~103, though the end of the treatise, from the middle of
prop.~16 to the last (18), is missing. The book, now edited by
Menge\footnote{\emph{Euclidis opera omnia}, vol.~\r8., 1916,
  pp.~2–156.} , consists of propositions in \emph{spheric} geometry.
Euclid based it on Autolycus' work \greek{περὶ κινουμένης σφαίρας},
but also, evidently, on an earlier text-book of \emph{Sphaerica} of
exclusively mathematical content.  It has been conjectured that the
latter textbook may have been due to Eudoxus\footnote{Heiberg,
  \emph{Euklid-Studien}, p.~46; Hultsch, \emph{Autolycus}, p.~xii;
  A. A. Björnbo, \emph{Studien über Menelaos' Sphärk}
  (\emph{Abhandlungen zur Geschichte der mathematischen
    Wissenschaften}, \r14.~1902), p.~56~sqq.}.

\section{The \emph{Optics}}

This book needs no description, as it has been edited by Heiberg
recently\footnote{\emph{Euclidis opera omnia}, vol.~\r7. (1895).},
both in its genuine form and in the recension by Theon.  The
\emph{Catoptrica} published by Heiberg in the same volume is not
genuine, and Heiberg suspects that in its present form it may be
Theon's. It is not even certain that Euclid wrote \emph{Catoptrica} at
all, as Proclus may easily have had Theon's work before him and
inadvertently assigned it to Euclid\footnote{Heiberg, Euclid's
  \emph{Optics}, \emph{etc}. p.~l.}.

\section{}

Besides the above-mentioned works, Euclid is said to have written the
\emph{Elements of Music}\footnote{Proclus, p.~69, 3.} (\greek{αἱ κατὰ
  μουσικὴν στοιχειώσεις}). Two treatises are attributed to Euclid in
our \textsc{mss.}\ of the \emph{Musici}, the \greek{}, \emph{Sectio
  canonis} (the theory of the intervals), and the \greek{}
(introduction to harmony)\footnote{Both treatises edited by Jan in
  \emph{Musici Scriptores Graeci}, 1895, pp.~113–166, 167–207;, and by
  Menge in \emph{Euclidis opera omnia}, vol.~\r8., 1916, pp.~157–183,
  185–223.}. The first, resting on the Pythagorean theory of music, is
mathematical, and the style and diction as well as the form of the
propositions mostly agree with what we find in the
\emph{Elements}. Jan thought it genuine, especially as almost the
whole of the treatise (except the preface) is quoted \emph{in
  extenso}, and Euclid is twice mentioned by name, in the commentary
on Ptolemy's \emph{Harmonica} published by Wallis and attributed by
him to Porphyry.  Tannery was of the opposite
opinion\footnote{\emph{Comptes rendus de l'Acad.\ des inscriptions et
    belles-lettres}, Paris, 1904, pp.~439—445.  Cf.\ \emph{Bibliotheca
    Mathematica}, \r6\tsub{3}, 1905–6, p.~225, note~1.}. The latest
editor, Menge, suggests that it may be a redaction by a less competent
hand from the genuine Euclidean \emph{Elements of Music}. The second
treatise is not Euclid's, but was written by Cleonides, a pupil of
Aristoxenus\footnote{Heiberg, \emph{Euklid-Studien}, pp.~52–55; Jan,
  \emph{Musici Scriptores Graeci}, pp.~169–174.}.

Lastly, it is worth while to give the Arabians' list of Euclid's
works.  I take this from Suter's translation of the list of
philosophers and mathematicians in the \emph{Fihrist}, the oldest
authority of the kind that we possess\footnote{H. Suter, \emph{Das
    Mathematiker-Verzeichnis im Fihrist} in \emph{Abhandlungen zur
    Geschichte der Mathematik}, \r6., 1892, pp.~1–87 (see especially
  p.~17).  Cf.\ Casiri, 1.~339, 340, and Gartz, \emph{De interpretibus
    et explanatoribus}, 1823, pp.~4,~5.}. ``To the writings of Euclid
belong further [in addition to the \emph{Elements}]: the book of
Phaenomena; the book of Given Magnitudes [\emph{Data}]; the book of
Tones, known under the name of Music, not genuine; the book of
Division, emended by Thābit; the book of Utilisations or Applications
[\emph{Porisms}], not genuine; the book of the Canon; the book of the
Heavy and Light; the book of Synthesis, not genuine; and the book of
Analysis, not genuine.''

It is to be observed that the Arabs already regarded the book of Tones
(by which must be meant the \greek{εἰσαγωγὴ ἁρμονική}) as spurious.
The book of Division is evidently the book on \emph{Divisions (of
  figures)}.  The next book is described by Casiri as ``liber de
utilitate suppositus.''  Suter gives reason for believing the
\emph{Porisms} to be meant\footnote{Suter, \emph{op.~cit.}\ pp.~49,
  50, Wenrich translated the word as ``utilia.'' Suter says that the
  nearest meaning of the Arabic word as of ``porism'' is \emph{use},
  \emph{gain} (Nutzen, Gewinn), while a further meaning is
  explanation, observation, addition: a gain arising out of what has
  preceded (cf.\ Proclus' definition of the porism in the sense of a
  corollary).}, but does not apparently offer any explanation of why
the work is supposed to be spurious. The book of the Canon is clearly
the \greek{κατατομὴ κανόνος}.  The book on ``the Heavy and Light'' is
apparently the tract \emph{De levi et ponderoso}, included in the
Basel Latin translation of 1537, and in Gregory's edition. The
fragment, however, cannot safely be attributed to Euclid, for (1)~we
have nowhere any mention of his having written on mechanics, (2)~it
contains the notion of specific gravity in a form so clear that it
could hardly be attributed to anyone earlier than
Archimedes\footnote{Heiberg, \emph{Euklid-Studien}, pp.~9,~10.}. Suter
thinks\footnote{Suter, \emph{op.~cit.}\ p.~50.} that the works on
Analysis and Synthesis (said to be spurious in the extract) may be
further developments of the \emph{Data} or \emph{Porisms}, or may be
the interpolated proofs of \emph{Eucl.}\ \prop{13}{1–5}, divided into
\emph{analysis} and \emph{synthesis}, as to which see the notes on
those propositions.

\chapter{Greek Commentators on the \emph{Elements} other
than Proclus}

That there was no lack of commentaries on the \emph{Elements} before
the time of Proctus is evident from the terms in which Proclus refers
to them; and he leaves as in equally little doubt as to the value
which, in his opinion, the generality of them possessed. Thus he says
in one place (at the end of his second prologue)\footnote{Proclus,
  p.~84,~8.}:

``Before making a beginning with the investigation of details, I warn
those who may read me not to expect from me the things which have been
dinned into our ears \emph{ad nauseam} (\greek{διατεθρύληται}) by
those who have preceded me, viz.\ lemmas, cases, and so forth. For I
am surfeited with these things and shall give little attention to
them.  But I shall direct my remarks principally to the points which
require deeper study and contribute to the sum of philosophy, therein
emulating the Pythagoreans who even had this common phrase for what I
mean `a figure and a platform, but not a figure and
sixpence\footnote{i.e.\ we reach a certain height, use the platform so
  attained as a base on which to build another stage, then use that as
  a base and so on.}.'\,”

In another place\footnote{Proclus, p.~200, 10.} he says: ``Let us now
turn to the elucidation of the things proved by the writer of the
Elements, selecting the more subtle of the comments made on them by
the ancient writers, while cutting down their interminable
diffuseness, giving the things which are more systematic and follow
scientific methods, attaching more importance to the working-out of
the real subject-matter than to the variety of cases and lemmas to
which we see recent writers devoting themselves for the most part.''

At the end of his commentary on Eucl.~\r1.\ Prockis
remarks\footnote{\ibid~p.~432, 15.} that the commentaries then in
vogue were full of all sorts of confusion, and contained no account of
\emph{causes}, no dialectical discrimination, and no philosophic
thought.

These passages and two others in which Proclus refers to ``the
commentators\footnote{\ibid~p.~389, 11; p.~328, 16.}'' suggest that
these commentators were numerous.  He does not however give many
names; and no doubt the only important commentaries were those of
Heron, Porphyry, and Pappus.

\section{Heron}

Proclus alludes to Heron twice as Heron
\emph{mechanicus}\footnote{Proclus, p.~305, 24; p.~346, 13.}, in
another place\footnote{\ibid~p.~41, 10.} he associates him with
Ctesibius, and in the three other passages\footnote{\ibid~p.~196, 16;
  p.~323, 7: p.~429, 13.} where Heron is mentioned there is no reason
to doubt that the same person is meant, namely Heron of
Alexandria. The date of Heron is still a vexed question. In the early
stages of the controversy much was made of the supposed relation of
Heron to Ctesibius. The best \textsc{ms.}\ of Heron's
\emph{Belopoeica} has the heading \greek{Ἥρωνος Κτησιβίου Βελοποιϊκά},
and an anonymous Byzantine writer of the tenth century, evidently
basing himself on this title, speaks of Ctesibius as Heron's
\greek{καθηγητής}, ``master'' or ``teacher.'' We know of two men of
the name of Ctesibius. One was a barber who lived in the time of
Ptolemy Euergetes~II, i.e.\ Ptolemy~VII, called Physcon (died
117~\bc)), and who is said to have made an improved
water-organ\footnote{Athenaeus, \emph{Deipno-Soph.} iv., c.~75, p.~174
  \emph{b}–\emph{c}.}.  The other was a mechanician mentioned by
Athenaeus as having made an elegant drinking-horn in the time of
Ptolemy Philadelphus (285–247~\bc)\footnote{\ibid~xi., c.~97, p.~497
  \emph{b}–\emph{c}.}.  Martin\footnote{Martin, \emph{Recherches sur
    la vie et les ouvrages d' Héron d' Alexandrie}, Paris, l854,
  p.~27.} took the Ctesibius in question to be the former and
accordingly placed Heron at the beginning of the first century~\bc,
say 126–50~\bc. But Philo of Byzantium\footnote{Philo,
  \emph{Mechan.\ Synt.}, p.~50, 38, ed.~Schöne.}, who repeatedly
mentions Ctesibius by name, says that the first mechanicians had the
advantage of being under kings who loved fame and supported the
arts. Hence our Ctesibius is more likely to have been the earlier
Ctesibius who was contemporary with Ptolemy~II Philadelphus.

But, whatever be the date of Ctesibius, we cannot safely conclude that
Heron was his immediate pupil.  The title ``Heron's (edition of)
Ctesibius's Belopoeica'' does not, in fact, justify any inferenee as
to the interval of time between the two works.

We now have better evidence for a \emph{terminus post quem}. The
\emph{Metrica} of Heron, besides quoting Archimedes and Apollonius,
twice refers to ``the books about straight lines (chords) in a
circle'' (\greek{}). Now we know of no work giving a Table of Chords
earlier than that of Hipparchus.  We get, therefore, at once,
150~\bc. or thereabouts as the \emph{terminus post quem}.  But, again,
Heron's \emph{Mechanica} quotes a definition of ``centre of gravity''
as given by ``Posidonius, a Stoic'': and, even if this Posidonius
lived before Archimedes, as the context seems to imply, it is certain
that another work of Heron's, the \emph{Definitions}, owes something
to Posidonius of Apamea or Rhodes, Cicero's teacher (135–51~\bc).
This brings Heron's date down to the end of the first century~\bc, at
least.

We have next to consider the relation, if any, between Heron and
Vitruvius. In his \emph{De Architectura} brought out apparently in
14~\bc, Vitruvius quotes twelve authorities on \emph{machinationes}
including Archytas (second), Archimedes (third), Ctesibius (fourth)
and Philo of Byzantium (sixth), but does not mention Heron.  Nor is it
possible to establish inter-dependence between Vitruvius and Heron;
the differences between them seem on the whole more numerous and
important than the resemblances (e.g.\ Vitruvius uses~3 as the value
of~$\pi$, while Heron always uses the Archimedean value~3). The
inference is that Heron can hardly have written earlier than the first
century~\ad

The most recent theory of Heron's date makes him later than Claudius
Ptolemy the astronomer (100–178~\ad). The arguments are mainly
these. (1)~Ptolemy claims as a discovery of his own a method of
measuring the distance between two places (as an arc of a great circle
on the earth's surface) in the case where the places are neither on
the same meridian nor on the same parallel circle.  Heron, in his
\emph{Dioptra}, speaks of this method as of a thing generally known to
experts. (2)~The dioptra described in Heron's work is a fine and
accurate instrument, much better than anything Ptolemy had at his
disposal. (3)~Ptolemy, in his work \greek{Περὶ ῥοπῶν}, asserted that
water with water round it has no weight and that the diver, however
deep he dives, does not feel the weight of the water above him. Heron,
strangely enough, accepts as true what Ptolemy says of the diver, but
is dissatisfied with the explanation given by ``some,'' namely that it
is because water is uniformly heavy—this seems to be equivalent to
Ptolemy's dictum that water in water has no weight—and he essays a
different explanation based on Archimedes. (4)~It is suggested that
the Dionysius to whom Heron dedicated his \emph{Definitions} is a
certain Dionysius who was \emph{praefectus urbi} in 301~\ad.

On the other hand Heron was earlier than Pappus, who was writing under
Diocletian (284–305~\ad), for Pappus alludes to and draws upon the
works of Heron. The net result, then, of the most recent research is
to place Heron in the third century~\ad\ and perhaps little earlier
than Pappus. Heiberg\footnote{\emph{Heronis Alexandrini opera},
  vol. \r5. (Teubner, 1914), p.~ix.} accepts this conclusion, which
may therefore, perhaps, be said to hold the field for the
present\footnote{Fuller details of the various arguments will be found
  in my \emph{History of Greek Mathematics}, 1921, vol.~\r2.,
  pp.~298–306.}.

That Heron wrote a systematic commentary on the \emph{Elements} might
be inferred from Proclus, but it is rendered quite certain by
references to the commentary in Arabian writers, and particularly in
an-Nairīzī's commentary on the first ten Books of the \emph{Elements}.
The \emph{Fihrist} says, under Euclid, that ``Heron wrote a commentary
on this book [the \emph{Elements}], endeavouring to solve its
difficulties\footnote{\emph{Das Matkematiker-Verzeichniss im Fihrist}
  (tr.\ Suter), p.~16.}''; and under Heron, ``He wrote: the book of
explanation of the obscurities in Euclid\footnote{\ibid~p.~22.}…''
An-Nairīzī's commentary quotes Heron by name very frequently, and
often in such a way as to leave no doubt that the author had Heron's
work actually before him. Thus the extracts are given in the first
person, introduced by ``Heron says'' (``Dixit Yrinus'' or ``Heron'');
and in other places we are told that Heron ``says nothing,'' or ``is
not found to have said anything,'' on such and such a proposition. The
commentary of an-Nairīzī is in part edited by Besthorn and Heiberg
from a Leiden \textsc{ms.}\ of the translation of the \emph{Elements}
by al-Ḥajjāj with the commentary attached\footnote{\emph{Codex
    Leidensis} 399, 1. \emph{Euclidis Elementa ex interpretatione
    al-Hadschdschadschii cum commentariis al-Narizii}. Five parts
  carrying the work to the end of Book~\r6.\ were issued in
  1893. 1897, 1900, 1905 and 1910 respectively.}.  But this
\textsc{ms.}\ only contains six Books, and several pages in the first
Book, which contain the comments of Simplicius on the first twenty-two
definitions of the first Book, are missing. Fortunately the commentary
of an-Nairīzī has been discovered in a more complete form, in a Latin
translation by Gherardus Cremonensis of the twelfth century, which
contains the missing comments by Simplicius and an-Nairīzī's comments
on the first ten Books. This valuable work has recently been edited by
Curtze\footnote{\emph{Anaritii in decem libros priores elementorum
    Euclidis commentarii ex interpretatione Gherardi
    Cremonensis…edidit} Maximilianus Curtze (Teubner, Leipzig,
  1899).}.

Thus from the three sources, Proclus, and the two versions of
an-Nairīzī, which supplement one another, we are able to form a very
good idea of the character of Heron's commentary. In some cases
observations given by Proclus without the name of their author are
seen from an-Nairīzī to be Heron's; in a few cases notes attributed by
Proclus to Heron are found in an-Nairīzī without Heron's name; and,
curiously enough, one alternative proof (of \prop{1}{25}) given as
Heron's by Proclus is introduced by the Arab with the remark that he
has not been able to discover who is the author.

Speaking generally, the comments of Heron do not seem to have
contained much that can be called important. We find

(1)~A few general notes, e.g. that Heron would not admit more than
three axioms.

(2)~Distinctions of a number of particular cases of Euclid's
propositions according as the figure is drawn in one way or in
another.

Of this class are the different cases of \prop{1}{35, 36}, \prop{3}{7,
  8} (where the chords to be compared are drawn on different sides of
the diameter instead of on the same side), \prop{3}{12} (which is not
Euclid's, but Heron's own, adding the case of external contact to that
of internal contact in \prop{3}{11}), \prop{6}{19} (where the triangle
in which an additional line is drawn is taken to be the \emph{smaller}
of the two), \prop{7}{19} (where he gives the particular case of three
numbers in continued proportion, instead of four proportionals).

(3)~Alternative proofs. Of these there should be mentioned
(\emph{a})~the proofs of \prop{2}{1–10} ``without a figure,'' being
simply the algebraic forms of proof, easy but uninstructive, which are
so popular nowadays, the proof of \prop{3}{25} (placed after
\prop{3}{30} and starting from the arc instead of the chord),
\prop{3}{10} (proved by \prop{3}{9}), \prop{3}{13} (a proof preceded
by a lemma to the effect that a straight line cannot meet a circle in
more than two points). Another class of alternative proof is
(\emph{b})~that which is intended to meet a particular objection
(\greek{ἔνστασις}) which had been or might be raised to Euclid's
construction. Thus in certain cases he avoids \emph{producing} a
particular straight line, where Euclid produces it, in order to meet
the objection of any one who should deny our right to assume that
there is \emph{any space available}\footnote{Cf.\ Proclus, 275, 7
  \greek{εἰ δὲ λέγοι τις τόπον μὴ εἰδέναι}…, 289, 18 \greek{λέγει οὖν
    τις ὅτι οὐκ ἔστι τόπος}….}.  Of this class are Heron's proofs of
\prop{1}{11}, \prop{1}{20}, and his note on \prop{1}{16}. Similarly on
\prop{1}{48} he supposes the right-angled triangle which is
constructed to be constructed on the \emph{same} side of the common
side as the given triangle is. A third class (\emph{c})~is that which
avoids \emph{reductio ad absurdum}. Thus, instead of indirect proofs,
Heron gives direct proofs of \prop{1}{19} (for which he requires, and
gives, a preliminary lemma), and of \prop{1}{25}.

(4)~Heron supplies certain \emph{converses} of Euclid's propositions,
e.g.\ converses of
\prop{2}{12, 13},
\prop{8}{27}.

(5)~A few additions to, and extensions of, Euclid's propositions are
also found. Some are unimportant, e.g.\ the construction of isosceles
and scalene triangles in a note on~\prop{1}{1}, the construction of
\emph{two} tangents in \prop{3}{17}, the remark that \prop{7}{3} about
finding the greatest common measure of three numbers can be applied to
as many numbers as we please (as Euclid tacitly assumes in
\prop{7}{31}). The most important extension is that of \prop{3}{20} to
the case where the angle at the circumference is greater than a right
angle, and the direct deduction from this extension of the result of
\prop{3}{22}. Interesting also are the notes on \prop{1}{37} (on
\prop{1}{24} in Proclus), where Heron proves that two triangles with
two sides of one equal to two sides of the other and with the included
angles supplementary are equal, and compares the areas where the sum
of the two included angles (one being supposed greater than the other)
is less or greater than two right angles, and on \prop{1}{47}, where
there is a proof (depending on preliminary lemmas) of the fact that,
in the figure of the proposition, the straight lines $AL$, $BK$, $CF$
meet in a point. After \prop{4}{16} there is a proof that, in a
regular polygon with an even number of sides, the bisector of one
angle also bisects its opposite, and an enunciation of the
corresponding proposition for a regular polygon with an odd number of
sides.

Van Pesch\footnote{\emph{De Procli fontibus} Lugduni-Batavorum, 1900,}
gives reason for attributing to Heron certain other notes found in
Proclus, viz.\ that they are designed to meet the same sort of points
as Heron had in view in other notes undoubtedly written by him. These
are (\emph{a})~alternative proofs of \prop{1}{5}, \prop{1}{17},
and~\prop{1}{32}, which avoid the producing of certain straight lines,
(\emph{b})~an alternative proof of \prop{1}{9} avoiding the
construction of the equilateral triangle on the side of $BC$ opposite
to~$A$; (\emph{c})~partial converses of \prop{1}{35–38}, starting from
  the equality of the areas and the fact of the parallelograms or
  triangles being in the same parallels, and proving that the bases
  are the same or equal, may also be Heron's.  Van Pesch further
  supposes that it was in Heron's commentary that the proof by
  Menelaus of \prop{1}{25} and the proof by Philo of \prop{1}{8} were
  given.

The last reference to Heron made by an-Nairīzī occurs in the note on
\prop{8}{27}, so that the commentary of the former must at least have
reached that point.

\section{Porphyry}

The Porphyry here mentioned is of course the Neo-Platonist who lived
about 232–304~\ad\ Whether he really wrote a systematic commentary on
the \emph{Elements} is uncertain. The passages in Proclus which seem
to make this probable are two in which he mentions him (1)~as having
demonstrated the necessity of the words ``not on the same side'' in
the enunciation of \prop{1}{14}\footnote{Proclus, pp.~297, 1–298,
  10.}, and (2)~as having pointed out the necessity of understanding
correctly the enunciation of \prop{1}{26}, since, if the particular
injunctions as to the sides of the triangles to be taken as equal are
not regarded, the student may easily fall into
error\footnote{\ibid~p.~351, 13, 14 and the pages preceding.}.  These
passages, showing that Porphyry carefully analysed Euclid's
\emph{enunciations} in these cases, certainly suggest that his remarks
were part of a systematic commentary.  Further, the list of
mathematicians in the \emph{Fihrist} gives Porphyry as having written
``a book on the Elements.'' It is true that Wenrich takes this book to
have been a work by Porphyry mentioned by Suidas and Proclus
(\emph{Theolog.\ Platan.}), \greek{περὶ ἀρχῶν}
libri~\r2.\footnote{\emph{Fihrist} (tr.\ Suter), p.~9, 10 and p.~45
  (note~5).}

There is nothing of importance in the notes attributed to Porphyry
by Proclus.

(1)~Three alternative proofs of \prop{1}{20}, which avoid
\emph{producing} a side of the triangle, are assigned to Heron and
Porphyry without saying which belonged to which.  If the first of the
three was Heron's, I agree with van Pesch that it is more probable
that the two others were both Porphyry's than that the second was
Heron's and only the third Porphyry's. For they are similar in
character, and the third uses a result obtained in the
second\footnote{Van Pesch, \emph{De Procli fontibus}, pp.~129, 130.
  Heiberg assigned them as above in his \emph{Euklid-Studien}
  (p.~160), but seems to have changed his view later.  (See
  Besthorn-Heiberg, \emph{Codex Leidensis}, p.~93, note~2.)}.

(2)~Porphyry gave an alternative proof of \prop{1}{18} to meet a
childish objection which is supposed to require the part of $AC$ equal
to $AB$ to be cut off from $CA$ and not from~$AC$.

Proclus gives a precisely similar alternative proof of \prop{1}{6} to
meet a similar supposed objection; and it may well be that, though
Proclus mentions no name, this proof was also Porphyry's, as van Pesch
suggests\footnote{Van Pesch, \emph{op.~cit.}\ pp.~130–1.}.

Two other references to Porphyry found in Proclus cannot have anything
to do with commentaries on the \emph{Elements}. In the first a work
called the \greek{Συμμικτά} is quoted, while in the second a
philosophical question is raised.

\section{Pappus}

The references to Pappus in Proclus are not numerous; but we have
other evidence that he wrote a commentary on the \emph{Elements}.
Thus a scholiast on the definitions of the \emph{Data} uses the phrase
``as Pappus says at the beginning of his (commentary) on the 10th
(book) of Euclid\footnote{Euclid's \emph{Data},
  ed.\ Menge. p.~262.}.'' Again in the \emph{Fihrist} we are told that
Pappus wrote a commentary to the tenth book of Euclid in two
parts\footnote{\emph{Fihrist} (tr.\ Sutter), p.~22.}.  Fragments of
this still survive in a \textsc{ms.}\ described by
Woepcke\footnote{\emph{Mémoires présentés à ĺacadémie des sciences},
  1856, \r14. pp.~658–719.}, Paris. No.~952. 2 (supplément arabe de la
Bibliothèque impériale), which contains a translation by Abū `Uthmān
(beginning of 10th century) of a Greek commentary on Book~\r10. It is
in two books, and there can now be no doubt that the author of the
Greek commentary was Pappus\footnote{Woepcke read the name of the
  author, in the title of the first book, as \emph{B.los} (the dot
  representing a missing vowel). He quotes also from other
  \textsc{mss.}\ (e.g.\ of the \emph{Ta'rīkh al-Ḥukamā} and of the
  \emph{Fihrist}) where he reads the name of the commentator as
  \emph{B.lis}, \emph{B.n.s} or \emph{B.l.s}.  Woepcke takes this
  author to be Valens, and thinks it possible that he may be the same
  as the astrologer Vettius Valens.  This Heiberg
  (\emph{Euklid-Studien}, pp.~169, 170) proves to be impossible,
  because, while one of the \textsc{mss.}\ quoted by Woepcke says that
  ``\emph{B.n.s}, le \emph{Roûmi} (late-Greek) was later than Claudius
  Ptolemy and the \emph{Fihrist} says ``\emph{B.l.s}, le \emph{Roûmi}
  wrote a commentary on Ptolemy's \emph{Planisphaerium}, Vettius
  Valens seems to have Lived under Hadrian, and must therefore have
  been an elder contemporary of Ptolemy. But Suter shows
  (\emph{Fihrist}, p.~22 and p.~54, note~92) that \emph{Banos} is only
  distinguished from \emph{Babos} by the position of a certain dot,
  and \emph{Balos} may also easily have arisen from an original
  \emph{Babos} (there is no P in Arabic), so that Pappus must be the
  person meant.  This is further confirmed by the fact that the
  \emph{Fihrist} gives this author and Valens as the subjects of two
  separate paragraphs, attributing to the latter astrological works
  only.}.  Again Eutocius, in his note on Archimedes, \emph{On the
  Sphere and Cylinder} \r1.~13, says that Pappus explained in his
commentary on the \emph{Elements} how to inscribe in a circle a
polygon similar to a polygon inscribed in another circle; and this
would presumably come in his commentary on Book~\r12., just as the
problem is solved in the second scholium on Eucl.~\prop{12}{1}. Thus
Pappus' commentary on the \emph{Elements} must have been pretty
complete, an additional confirmation of this supposition being
forthcoming in the reference of Marinus (a pupil and follower of
Proclus) in his preface to the \emph{Data} to ``the commentaries of
Pappus on the book\footnote{Heiberg, \emph{Euklid-Studien}, p.~173;
  Euclid's \emph{Data}, ed.\ Menge, pp.~256, lii.}.''

The actual references to Pappus in Proclus are as follows:

(1)~On the Postulate~(4) that all right angles are equal, Pappus is
quoted as saying that the converse, viz.\ that all angles equal to a
right angle are right, is not true\footnote{Proclus, pp.~189, 190.},
since the angle included between the arcs of two semicircles which are
equal, and have their diameters at right angles and terminating at one
point, is equal to a right angle, but is not a right angle.

(2) On the axioms Pappus is quoted as saying that, in addition to
Euclid's axioms, others are on record as well
(\greek{συναναγράφεσθαι}) about unequals added to equals and equals
added to unequals\footnote{\ibid~p.~197, 6–10.}; these, says Proclus,
follow from the Euclidean axioms, while others given by Pappus are
involved by the definitions, namely those which assert that ``all
parts of the plane and of the straight line coincide with one
another,'' that ``a point divides a straight line, a line a surface,
and a surface a solid,'' and that ``the infinite is (obtained) in
magnitudes both by addition and diminution\footnote{\ibid~p.~198,
  3–15.}.''

(3)~Pappus gave a pretty proof of \prop{1}{5}. This proof has, I
think, been wrongly understood; on this point see my note on the
proposition.

(4)~On \prop{1}{47} Proclus says\footnote{Proclus, p. 429, 9–15.}:
``As the proof of the writer of the Elements is manifest, I think that
it is not necessary to add anything further, but that what has been
said is sufficient, since indeed those who have added more, like Heron
and Pappus, were obliged to make use of what is proved in the sixth
book, without attaining any important result.'' We shall see what
Heron's addition consisted of; what Pappus may have added we do not
know, unless it was something on the lines of his extension of
\prop{1}{47} found in the \emph{Synagoge} (\r4.~p.~176, ed.\ Hultsch).

We may fairly conclude, with van Pesch\footnote{Van Pesch, \emph{De
    Procli fontibus}, p.~134 sqq.}, that Pappus is drawn upon in
various other passages of Proclus where he quotes no authority, but
where the subject-matter reminds us of other notes expressly assigned
to Pappus or of what we otherwise know to have been favourite
questions with him. Thus:

1.~We are reminded of the curvilineal angle which is equal to but not
a right angle by the note on \prop{1}{32} to the effect that the
converse (that a figure with its interior angles together equal to two
right angles is a triangle) is not true unless we confine ourselves to
rectilineal figures. This statement is supported by reference to a
figure formed by four semicircles whose diameters form a square, and
one of which is turned inwards while the others are turned outwards.
The figure forms two angles ``equal to'' right angles in the sense
described by Pappus on Post.\ref{post:4}, while the other curvilineal
angles are not considered to be angles at all, and are left out in
summing the internal angles. Similarly the allusions in the notes on
\prop{1}{4, 23} to curvilineal angles of which certain moon-shaped
angles (\greek{μηνοειδεῖς}) are shown to be ``equal to'' rectilineal
angles savour of Pappus.

2.~On \prop{1}{9} Proclus says\footnote{Proclus, p.~272, 10.} that
``Others, starting from the Archimedean spirals, divided any given
rectilineal angle in any given ratio.''  We cannot but compare this
with Pappus \r4.~p.~286, where the spiral is so used; hence this note,
including remarks immediately preceding about the conchoid and the
quadratrix, which were used for the same purpose, may very well be due
to Pappus.

3.~The subject of isoperimetric figures was a favourite one with
Pappus, who wrote a recension of Zenodorus' treatise on the
subject\footnote{Pappus, v.\ pp.~304–350; for Zenodorus' own treatise
  see Hultsch's Appendix, pp.~1189–1211.}.  Now on \prop{1}{35}
Proclus speaks\footnote{Proclus, pp.~396–8.} about the paradox of
parallelograms having equal area (between the same parallels) though
the two sides between the parallels may be of any length, adding that
of parallelograms with equal perimeter the rectangle is greatest if
the base be given, and the square greatest if the base be not given
etc. He returns to the subject on \prop{1}{37} about
triangles\footnote{\ibid~pp.~403–4.}.
Compare\footnote{\ibid~pp.~236–7.} also his note on \prop{1}{4}. These
notes may have been taken from Pappus.

4.~Again, on \prop{1}{21}, Proclus remarks on the paradox that
straight lines may be drawn from the base to a point within a triangle
which are (1)~together greater than the two sides, and (2)~include a
less angle, provided that the straight lines may be drawn from points
in the base other than its extremities. The subject of straight lines
satisfying condition (1)~was treated at length, with reference to a
variety of cases, by Pappus\footnote{Pappus, \r3. pp.~104–130.}, after
a collection of ``paradoxes'' by Erycinus, of whom nothing more is
known. Proclus gives Pappus' first case, and adds a rather useless
proof of the possibility of drawing straight lines satisfying
condition (2)~\emph{alone}, adding that ``the proposition stated has
been proved by me without using the parallels of the
commentators\footnote{Proclus, p.~328, 15.}.''  By ``the
commentators'' Pappus is doubtless meant.

5.~Lastly, the ``four-sided triangle,'' called by Zenodorus the
``hollow-angled,''\footnote{Proclus, p.~165, 24; cf.\ pp.~328, 329.}
is mentioned in the notes on \r1.~Def.~24–29 and \prop{1}{21}.  As
Pappus wrote on Zenodorus' work in which the term
occurred\footnote{See Pappus, ed.\ Hultsch, pp.~1154, 1206.}, Pappus
may be responsible for these notes.

\section{Simplicius}.

According to the \emph{Fihrist}\footnote{\emph{Fihrist} (tr.\ Suter),
  p.~21.}, Simplicius the Greek wrote ``a commentary to the beginning
of Euclid's book, which forms an introduction to geometry.'' And in
fact this commentary on the definitions, postulates and axioms
(including the postulate known as the Parallel-Axiom) is preserved in
the Arabic commentary of an-Nairīzī\footnote{An-Nairīzī,
  ed.\ Besthorn-Heiberg, pp.~9–41, 119–133, ed.\ Curtis, pp.~1–37,
  65–73. The \emph{Codex Leidensis}, from which Besthorn and Heiberg's
  edition is taken, has unfortunately lost some leaves, so that there
  is a gap from Def.~1 to Def.~13 (parallels). The loss is, however,
  made good by Curtze's edition of the translation by Gherard of
  Cremona.}. On two subjects this commentary of Simplicius quotes a
certain ``Aganis,'' the first subject being the definition of an
angle, and the second the definition of parallels and the
parallel-postulate. Simplicius gives word for word, in a long passage
placed by an-Nairīzī after \prop{1}{29}, an attempt by ``Aganis'' to
prove the parallel-postulate. It starts from a definition of parallels
which agrees with Geminus' view of them as given by
Proclus\footnote{Proclus, p.~177, 21.}, and is closely connected with
the definition given by Posidonius\footnote{\ibid~p.~176, 7.}. Hence
it has been assumed that ``Aganis'' is none other than Geminus, and
the historical importance of the commentary of Simplicius has been
judged accordingly. But it has been recently shown by Tannery that the
identification of ``Aganis'' with Geminus is practically
impossible\footnote{\emph{Bibliotheca Mathematica}, \r2\tsub{3}, 1900,
  pp.~9–11.}.  In the translation of Besthorn-Heiberg Aganis is called
by Simplicius in one place ``philosophus Aganis,'' in another
``magister noster Aganis,'' in Gherard's version he is ``socius
Aganis'' and ``socius noster Aganis.'' These expressions seem to leave
no doubt that Aganis was a contemporary and friend, if not master, of
Simplicius; and it is impossible to suppose that Simplicius
(fl.\ about 500~\ad) could have used them of a man who lived four and
a half centuries before his time. A phrase in Simplicius'
word-for-word quotation from Aganis leads to the same conclusion. He
speaks of people who objected ``even in ancient times'' (iam
antiquitus) to the use by geometers of this postulate. This would not
have been an appropriate phrase had Geminus been the writer. I do not
think that this difficulty can he got over by Suter's
suggestion\footnote{\emph{Zeitschrift für Math.\ u.\ Physik}, \r44.,
  hist.-litt.\ Abth.\ p.~61.} that the passages in question may have
been taken out of \emph{Heron's} commentary, and that an-Nairīzī may
have forgotten to name the author; it seems clear that Simplicius is
the person who described ``Aganis.''  Hence we are driven to suppose
that Aganis was not Geminus, but some unknown contemporary of
Simplicius\footnote{The above argument seems to me quite
  insuperable. The other arguments of Tannery do not, however, carry
  conviction to my mind. I do not follow the reasoning based on
  Aganis' definition of an angle. It appears to me a pure assumption
  that Geminus would have seen that Posidonius' definition of
  parallels was not admissible. Nor does it seem to me to count for
  much that Proclus, while telling us that Geminus held that the
  postulate ought to be proved and warned the unwary against hastily
  concluding that two straight lines approaching one another must
  necessarily meet (cf.\ a curve and its asymptote), gives no hint
  that Geminus did try to prove the postulate. It may well be that
  Proclus omitted Geminus' ``proof'' (if he wrote one) because he
  preferred Ptolemy's attempt which he gives
  (pp.~365–7).}. Considerable interest will however continue to attach
to the comments of Simplicius so fortunately preserved.

Proclus tells us that one Aegaeas (? Aenaeas) of Hierapolis wrote an
epitome of the \emph{Elements}\footnote{Proclus, p.~361, 21.}; but we
know nothing more of him or of it.

\chapter{Proclus and His Sources\protect\footnote{My task in this chapter is made easy by the appearance, in the nick
  of time, of the dissertation \emph{De Procli fontibus} by J. G. van
  Pesch (Lugduni-Batavorum, Apud L. van Nifterik,
  \textsc{mdcccc}). The chapters dealing directly with the subject
  show a thorough acquaintance on the part of the author with all the
  literature hearing on it; he covers the whole field and he exercises
  a sound and sober judgment in forming his conclusions.  The same
  cannot always be said of his only predecessor in the same inquiry,
  Tannery (in \emph{La Géométrie grecque}, 1887), who often robs his
  speculations of much of their value through his proneness to run
  away with an idea; he does so in this case, basing most of his
  conclusions on an arbitrary and unwarranted assumption as to the
  significance of the words \greek{οἰ περί τινα} (e.g.\ \greek{Ἤρωνα},
  \greek{Ποσειδώνιον} etc) as used in Proclus.}}

It is well known that the commentary of Proclus on Eucl.\ Book \r1.
is one of the two main sources of information as to the history of
Greek geometry which we possess, the other being the \emph{Collection}
of Pappus. They are the more precious because the original works of
the forerunners of Euclid, Archimedes and Apollonius are lost, having
probably been discarded and forgotten almost immediately after the
appearance of the masterpieces of that great trio.

Proclus himself lived 410–485~\ad, so that there had already passed a
sufficient amount of time for the tradition relating to the
pre-Euclidean geometers to become obscure and defective. In this
connexion a passage is quoted from Simplicius\footnote{Simplicius on
  Aristotle's \emph{Physics}, ed. Diels, pp.~54–69.} who, in his
account of the quadrature of certain lunes by Hippocrates of Chios,
while mentioning two authorities for his statements, Alexander
Aphrodisiensis (about 220~\ad) and Eudemus, says in one
place\footnote{\ibid~p.~68, 32.}, ``As regards Hippocrates of Chios we
must pay more attention to Eudemus, \emph{since he was nearer the
  times}, being a pupil of Aristotle.''

The importance therefore of a critical examination of Proclus'
commentary with a view to determining from what original sources he
drew need not be further emphasised.

Proclus received his early training in Alexandria, where Olympiodorus
was his instructor in the works of Aristotle, and mathematics was
taught him by one Heron\footnote{Cf.\ Martin, \emph{Recherches sur la
    vie et les ouvrages d'Héron d'Alexandrie}, pp.~240–2.} (of course
a different Heron from the ``\emph{mechanicus} Hero'' of whom we have
already spoken). He afterwards went to Athens where he was imbued by
Plutarch, and by Syrianus, with the Neo-Platonic philosophy, to which
he then devoted heart and soul, becoming one of its most prominent
exponents. He speaks everywhere with the highest respect of his
masters, and was in turn regarded with extravagant veneration by his
contemporaries, as we learn from Marinus his pupil and biographer. On
the death of Syrianus he was put at the head of the Neo-Platonic
school. He was a man of untiring industry, as is shown by the number
of books which he wrote, including a large number of commentaries,
mostly on the dialogues of Plato. He was an acute dialectician, and
pre-eminent among his contemporaries in the range of his
learning\footnote{Zeller calls him ``Der Gelehrte, dem kein Feld
  damaligen Wissens verschlossen ist.''}; he was a competent
mathematician; he was even a poet. At the same time he was a believer
in all sorts of myths and mysteries and a devout worshipper of
divinities both Greek and Oriental.

Though he was a competent mathematician, he was evidently much more a
philosopher than a mathematician\footnote{Van Pesch observes that in
  his commentaries on the \emph{Timaeus} (pp.~671–2) he speaks as no
  real mathematician could have spoken. In the passage referred to the
  question is whether the sun occupies a middle place among the
  planets. Proclus rejects the view of Hipparchus and Ptolemy because
  ``\greek{}'' (sc.\ the Chaldean, says Zeller) thinks otherwise,
  ``whom it is not lawful to disbelieve.''  Martin says rather neatly,
  ``Pour Proclus, les Éléments d'Euclide ont ĺheureuse chance de
  n'être contredits ni par les Oracles chaldaïques, ni par les
  spéculations des pythagoriciens anciens et nouveaux……''}.  This is
shown even in his commentary on Eucl.\ \r1., where, not only in the
Prologues (especially the first), but also in the notes themselves, he
seizes any opportunity for a philosophical digression. He says himself
that he attaches most importance to ``the things which require deeper
study and contribute to the sum of philosophy\footnote{Proclus, p.~84,
  13.}''; alternative proofs, cases, and the like (though he gives
many) have no attraction for him; and, in particular, he attaches no
value to the addition of Heron to \prop{1}{47}\footnote{\ibid~p.~429,
  12.}, which is of considerable mathematical interest.  Though he
esteemed mathematics highly, it was only as a handmaid to philosophy.
He quotes Plato's opinion to the effect that ``mathematics, as making
use of hypotheses, falls short of the non-hypothetical and perfect
science\footnote{\ibid~p.~31, 20. }''…“Let us then not say that Plato
excludes mathematics from the sciences, but that he declares it to be
secondary to the one supreme science\footnote{\ibid~p.~32, 2.}.''  And
again, while ``mathematical science must be considered desirable in
itself, though not with reference to the needs of daily life,'' ``if
it is necessary to refer the benefit arising from it to something
else, we must connect that benefit with intellectual knowledge
(\greek{νοερὰν γνῶσιν}), to which it leads the way and is a
propaedeutic, clearing the eye of the soul and taking away the
impediments which the senses place in the way of the knowledge of
universals (\greek{τῶν ὅλων})\footnote{\ibid~p.~27, 27 to 28, 7;
  cf.\ also p.~21, 25, pp.~46, 47.}.''

We know that in the Neo-Platonic school the younger pupils learnt
mathematics; and it is clear that Proclus taught this subject, and
that this was the origin of the commentary. Many passages show him as
a master speaking to scholars. Thus ``we have illustrated and made
plain all these things in the case of the first problem, but it is
necessary that \emph{my hearers} should make the same inquiry as
regards the others as well\footnote{Proclus, p.~210, 18.},'' and ``I
do not indicate these things as a merely incidental matter but as
preparing us beforehand for the doctrine of the
Timaeus\footnote{\ibid~p. 384, 2.}.''  Further, the pupils whom he was
addressing were \emph{beginners} in mathematics; for in one place he
says that he omits ``for the present'' to speak of the discoveries of
those who employed the curves of Nicomedes and Hippias for trisecting
an angle, and of those who used the Archimedean spiral for dividing an
angle in any given ratio, because these things would be too difficult
for beginners (\greek{δυσθεωρήτους τοῖς
  εἰσαγομένοις})\footnote{\ibid~p.~272, 12.}. Again, if his pupils had
not been beginners, it would not have been necessary for Proclus to
explain what is meant by saying that sides subtend certain
angles\footnote{\ibid~p.~238, 12.}, the difference between
\emph{adjacent} and \emph{vertical} angles\footnote{\ibid~p.~298, 14.}
etc., or to exhort them, as he often does, to work out other
particular cases for themselves, for practice (\greek{γυμνασίας
  ἕνεκα})\footnote{Cf.\ p.~224, 15 (on \r1.2).}.

The commentary seems then to have been founded on Proclus' lectures to
beginners in mathematics. But there are signs that it was revised and
re-edited for a larger public; thus he gives notice in one
place\footnote{\ibid~p. 84, 9.} ``to those who shall come upon'' his
work (\greek{}). There are also passages which could not have heen
understood by the beginners to whom he lectured, e.g.\ passages about
the cylindrical helix\footnote{\ibid~p.~105.}, conchoids and
cissoids\footnote{\ibid~p.~113.}. These passages may have been added
in the revised edition, or, as van Pesch conjectures, the explanations
given in the lectures may have been much fuller and more
comprehensible to beginners, and they may haw; been shortened on
revision.

In his comments on the propositions of Euclid, Proclus generally
proceeds in this way: first he gives explanations regarding Euclid's
proofs, secondly he gives a few different cases, mainly for the sake
of practice, and thirdly he addresses himself to refuting objections
raised by cavillers to particular propositions. The latter class of
note he deems necessary because of ``sophistical cavils'' and the
attitude of the people who rejoiced in finding paralogisms and in
causing annoyance to scientific men\footnote{\ibid~p.~375, 8.}. His
commentary does not seem to have been written for the purpose of
correcting or improving Euclid. For there are very few passages of
mathematical content in which Proclus can be supposed to be
propounding anything of his own; nearly all are taken from the works
of others, mostly earlier commentators, so that, for the purpose of
improving on or correcting Euclid, there was no need for his
commentary at all. Indeed only in one place does he definitely bring
forward anything of his own to get over a difficulty which he finds in
Euclid\footnote{\ibid~pp.~368–373.}; this is where he tries to prove
the parallel-postulate, after first giving Ptolemy's attempt and then
pointing out objections to it. On the other hand, there are a number
of passages in which he extols Euclid; thrice\footnote{Proclus,
  p.~280, 9; p.~282, 10; pp.~335, 336.} also he supports Euclid
against Apollonius where the latter had given proofs which he
considered better than Euclid's (\prop{1}{10, 11, and 23}).

Allusion must be made to the debated question whether Proclus
continued his commentaries beyond Book~\r1, His intention to do so is
clear from the following passages. Just after the words above quoted
about the trisection etc.\ of an angle by means of certain curves he
says, ``For we may perhaps more appropriately examine these things on
the third book, where the writer of the Elements bisects a given
circumference\footnote{\ibid~p.~272, 14.}.'' Again, after saying that
of all parallelograms which have the same perimeter the square is the
greatest ``and the rhomboid least of all,'' he adds: ``But this we
will prove in another place; for it is more appropriate to the
(discussion of the) hypotheses of the second
book\footnote{\ibid~p.~398, 18.}.''  Lastly, when alluding (on
\prop{1}{45}) to the squaring of the circle, and to Archimedes'
proposition that any circle is equal to the right-angled triangle in
which the perpendicular is equal to the radius of the circle and the
base to its perimeter, he adds, ``But of this
elsewhere\footnote{\ibid~p.~423, 6.}''; this may imply, an intention
to treat of the subject on Eucl.~\r12., though Heiberg doubts
it\footnote{Heiberg, \emph{Euklid-Studien}, p.~165, note.}.  But it is
clear that, at the time when the commentary on Book~\r1.\ was written,
Proclus had not yet begun to write on the other Books and was
uncertain whether he would be able to do so: for at the end he
says\footnote{Proclus, p.~432, 9,}, ``For my part, if I should be able
to discuss the other books\footnote{The words in the Greek are:
  \greek{εἰ μὲν δυνηθείημεν καὶ τοῖς λοιποῖς τὸν αὐτὸν τρόπον
    ἐξελθεῖν}.  For \greek{ἐξελθεῖν} Heiberg would read
  \greek{ἐπεξελθεῖν}.} in the same manner, I should give thanks to the
gods; but, if other cares should draw me away, I beg those who are
attracted by this subject to complete the exposition of the other
books as well, following the same method, and addressing themselves
throughout to the deeper and better defined questions involved''
(\greek{τὸ πραγματειῶδες πανταχοῦ καὶ εὐδιαίρετον μεταδιώκοντας}).

There is in fact no satisfactory evidence that Proclus did actually
write any more commentaries than that on Book~\r1.\footnote{True, a
  Vatican \textsc{ms.}\ has a collection of scholia on
  Books~\r1.\ (extracts from the extant commentary of Proclus), \r2.,
  \r5., \r6., \r10.\ headed \greek{Εἰς τὰ Εὐκλείδου στοιχεῖα
    προλαμβανόμενα ἐκ τῶν Πρόκλου στροπάδην καὶ κατ’ ἐπιτομήν}.
  Heiberg holds that this title itself suggests that the authorship of
  Proclus was limited to the scholia on Book~\r1.; for
  \greek{προλαμβανόμενα ἐκ τῶν Πρόκλου} suits extracts from Proclus'
  \emph{prologues}, but hardly scholia to later Books.  Again, a
  certain scholium (Heiberg in \emph{Hermes}. \r38., 1903, p.~341,
  No.~17) purports to quote words from the end of ``a scholium of
  Proclus'' on \prop{10}{9}. The words quoted are from the scholium
  \r10.\ No.~62, one of the Scholia Vaticana. But none of the other,
  older, sources connect Proclus' name with \r10.\ No.~62; it is
  probable therefore that a Byzantine, who had in his \textsc{ms.}\ of
  Euclid the collection of Schol.\ Vat.\ and knew that those on
  Book~\r1.\ came from Proclus, himself attached Proclus' name to the
  others.} The contrary view receives support from two facts pointed
out by Heiberg, viz.\ (1)~that the scholiast's copy of Proclus was not
so much better than our \textsc{mss.}\ as to suggest that the
scholiast had further commentaries of Proclus which have vanished for
us\footnote{While one class of scholia (Schol.\ Vat.)\ have some
  better readings than our \textsc{mss.}\ of Proclus have, and partly
  fill up the gaps at \prop{1}{36, 37} and \prop{1}{41–43}, the other
  class (Schol.\ Vind.)\ derive from an inferior Proclus
  \textsc{ms.}\ which also had the same lacunae.1}; (2)~that there is
no trace in the scholia of the notes which Proclus promised in the
passages quoted above.

Coming now to the question of the sources of Proclus, we may say that
everything goes to show that his commentary is a compilation, though a
compilation ``in the better sense'' of the term\footnote{Knoche,
  \emph{Untersuchungenüber des Proklus Diadochus Commentar zu Euklid's
    Elementen} (1862), p.~11.}. He does not even give us to understand
that we shall find in it much of his own; ``let us,'' he says, ``now
turn to the exposition of the theorems proved by Euclid, selecting the
more subtle of the comments made on them by the ancient writers, and
cutting down their interminable diffuseness…\footnote{Proclus, p.~200,
  10—13.}'': not a word about anything of his own. At the same time,
he seems to imply that he will not necessarily on each occasion quote
the source of each extract from an earlier commentary; and, in fact,
while he quotes the name of his authority in many places, especially
where the subject is important, in many others, where it is equally
certain that he is not giving anything of his own, he mentions no
authority. Thus he quotes Heron by name six times; but we now know,
from the commentary of an-Nairīzī, that a number of other passages,
where he mentions no name, are taken from Heron, and among them
the. not unimportant addition of an alternative proof to
\prop{1}{19}. Hence we can by no means conclude that, where no authority
is mentioned, Proclus is giving notes of his own. The presumption is
generally the other way; and it is often possible to arrive at a
conclusion, either that a particular note is not Proclus' own, or that
it is definitely attributable to someone else, by applying the
ordinary principles of criticism. Thus, where the note shows an
unmistakable affinity to another which Proclus definitely attributes
to some commentator by name, especially when both contain some
peculiar and distinctive idea, we cannot have much doubt in assigning
both to the same commentator\footnote{Instances of the application of
  this criterion will be found in the discussion of Proclus'
  indebtedness to the commentaries of Heron, Porphyry and
  Pappus.}. Again, van Pesch finds a criterion in the form of a note,
where the explanation is so condensed as to be only just intelligible;
the note is that in which a converse of \prop{1}{32} is
proved\footnote{Van Pesch attributes this converse and proof to
  Pappus, arguing from the fact that the proof is followed by a
  passage which, on comparison with Pappus' note on the postulate that
  all right angles are equal, he feels justified in assigning to
  Pappus, I doubt if the evidence is sufficient.} the proposition
namely that a rectilineal figure which has all its interior angles
together equal to two right angles is a triangle.

It is not safe to attribute a passage to Proclus himself because he
uses the first person in such expressions as ``I say'' or ``I will
prove''—for he was in the habit of putting into his own words the
substance of notes borrowed from others—nor because, in speaking of an
objection raised to a particular proposition, he uses such expressions
as ``perhaps someone may object'' (\greek{ἴσως δ’ ἄν τινες
  ἐνσταῖεν}…): for sometimes other words in the same passage, indicate
that the objection had actually been taken by someone\footnote{Van
  Pesch illustrates this by an objection refuted in the note on
  \prop{1}{9}, p.~273, 11~sqq.  After using the above expression to
  introduce the objection, Proclus uses further on (p.~273, 25) the
  term ``they say'' (\greek{φασίν}).}.  Speaking generally, we shall
not be justified in concluding that Proclus is stating something new
of his own unless he indicates this himself in express terms.

As regards the form of Proclus' references to others by name, van
Pesch notes that he very seldom mentions the particular \emph{work}
from which he is borrowing. If we leave out of account the references
to Plato's dialogues, there are only the following references to
books: the \emph{Bacchae} of Philolaus\footnote{Proclus, p.~22, 15.},
the \emph{Symmitka} of Porphyry\footnote{\ibid~p.~56, 25.}, Archimedes
\emph{On the Sphere and Cylinder}\footnote{\ibid~p.~71, 18.},
Apollonius \emph{On the cochlias}\footnote{\ibid~p.~105, 5.}, a book
by Eudemus on \emph{The Angle}\footnote{\ibid~p.~125, 8.}, a whole
book of Posidonius directed against Zeno of the Epicurean
sect\footnote{\ibid~p.~200, 2.}, Carpus'
\emph{Astronomy}\footnote{\ibid~p.~241, 19.}, Eudemus' \emph{History
  of Geometry}\footnote{\ibid~p. 352, 15.}, and a tract by Ptolemy on
the parallel-postulate\footnote{\ibid~p.~362, 15.}.

Again, Proclus does not always indicate that he is quoting something
at second-hand. He often does so, e.g.\ he quotes Heron as the
authority for a statement about Philippus, Eudemus as attributing a
certain theorem to Oenopides etc.; but he says on \prop{1}{12} that
``Oenopides first investigated this problem, thinking it useful for
astronomy'' when he cannot have had Oenopides' work before him.

It has been said above that Proclus was in the habit of stating in his
own words the substance of the things which he borrowed. We are
prepared for this when we find him stating that he will select the
best things from ancient commentaries and ``cut short their
interminable diffuseness,'' that he will ``briefly describe''
(\greek{}) the other proofs of \prop{1}{20} given by Heron and
Porphyry and also the proofs of \prop{1}{25} by Menelaus and
Heron. But the best evidence is of course to be found in the passages
where he quotes works still extant, e.g.\ those of Plato, Aristotle
and Plotinus. Examination of these passages shows great divergences
from the original; even where he purports to quote textually, using
the expressions ``Plato says,'' or ``Plotinus says,'' he by no means
quotes word for word\footnote{See the passages referred to by van
  Pesch (p.~70). The most glaring case is a passage (p.~21, 10) where
he quotes Plotinus, using the expression ``Plotinus says……''
Comparison with Plotinus, \emph{Ennead}.\ 1.~3, 3, shows that
\emph{very few} words are those of Plotinus himself; the rest
represent Plotinus' views in Proclus' own language.}. In fact, he
  seems to have had a positive distaste for quoting textually from
  other works. He cannot conquer this even when quoting from Euclid;
  he says in his note on \prop{1}{22}, ``we will follow the words of
  the geometer'' but fails, nevertheless, to reproduce the text of
  Euclid unchanged\footnote{Proclus, p.~330, 19 sqq.}.

We now come to the sources themselves from which Proclus drew in
writing his commentary. Three have already been disposed of,
viz.\ Heron, Porphyry and Pappus, who had all written commentaries on
the \emph{Elements}\footnote{See pp.~20 to 27 above.}. We go on to

\textbf{Eudemus}, the pupil of Aristotle, who, among other works,
wrote a history of arithmetic, a history of astronomy, and a history
of geometry.  The importance of the last mentioned work is attested by
the frequent use made of it by ancient writers. That there was no
other history of geometry written after the time of Eudemus seems to
be proved by the remark of Proclus in the course of his famous
summary: ``Those who compiled histories bring the development of this
science up to this point. \emph{Not much younger than these is
  Euclid}\footnote{Proclus, p.~68, 4–7.}….'' The loss of Eudemus'
history is one of the gravest which fate has inflicted upon us, for it
cannot be doubted that Eudemus had before him a number of the actual
works of earlier geometers, which, as before observed, seem to have
vanished completely when they were superseded by the treatises of
Euclid, Archimedes and Apoilonius. As it is, we have to be thankful
for the fragments from Eudemus which such writers as Proclus have
preserved to us.

I agree with van Pesch\footnote{\emph{De Procli fontibus}. pp.~73–75.}
that there is no sufficient reason for doubting that the work of
Eudemus was accessible to Proclus at first hand. For the later writers
Simplicius and Eutocius refer to it in terms such as leave no room for
doubt that they had it before them.  I have already quoted a passage
from Simplicius' account of the lunes of Hippocrates to the effect
that Eudemus must be considered the best authority since he lived
nearer the times\footnote{See above, p.~29.}. In the same place
Simplicius says\footnote{Simplicius, \emph{loc.\ cit.}, ed.\ Diels,
  p.~60, 17.}, ``I will set out what Eudemus says word for word
(\greek{κατὰ λέξιν λεγόμενα}), adding only a little explanation in the
shape of reference to Euclid's Elements \emph{owing to the
  memorandum-like style of Eudemus} (\greek{διὰ τὸν ὑπομνηματικὸν
  τρόπον τοῦ Εὐδήμον}) who sets out his explanations in the
abbreviated form usual with ancient writers.  Now in the second book
of the history of geometry he writes as follows\footnote{5\?}.'' It is
not possible to suppose that Simplicius would have written in this way
about the style of Eudemus if he had merely been copying certain
passages second-hand out of some other author and had not the original
work itself to refer to. In like manner, Eutocius speaks of the
paralogisms handed down in connexion with the attempts of Hippocrates
and Antiphon to square the circle\footnote{Archimedes, ed.\ Heiberg,
  vol.~\r3.\ p.~228.}, ``with which I imagine that those are
accurately acquainted who have examined (\greek{ἐπεσκεμμένους}) the
geometrical history of Eudemus and know the Ceria Aristotelica.''  How
could the contemporaries of Eutocius have \emph{examined} the work of
Eudemus unless it was still extant in his time?

The passages in which Proclus quotes Eudemus by name as his authority
are as follows:

(1)~On \prop{1}{26} he says that Eudemus in his history of geometry
referred this theorem to Thales, inasmuch as it was necessary to
Thales' method of ascertaining the distance of ships from the
shore\footnote{Proclus, p.~352, 14–18.}.

(2)~Eudemus attributed to Thales the discovery of
Eucl.\ \prop{1}{15}\footnote{\ibid~p.~299, 3.}, and

(3)~to Oenopides the problem of \prop{1}{23}\footnote{\ibid~p.~333, 5.}.

(4)~Eudemus referred the discovery of the theorem in \prop{1}{32} to
the Pythagoreans, and gave their proof of it, which Proclus
reproduces\footnote{\ibid~p.~379, 1–16.}.

(5)~On \prop{1}{44} Proclus tells us\footnote{\ibid~p.~419, 15–18.} that Eudemus says
that ``these things are ancient, being discoveries of the Pythagorean
muse, the application (\greek{παραβοή}) of areas, their exceeding
(\emph{ὑπερβολή}) and their falling short (\greek{ἔλλειψις}).'' The
next words about the appropriation of these terms (parabola, hyperbola
and ellipse) by later writers (i.e.\ Apollonius) to denote the conic
sections are of course not due to Eudemus.

Coming now to notes where Eudemus is not named by Proclus, we may
fairly conjecture, with van Pesch, that Eudemus was really the
authority for the statements (1)~that Thales first proved that a
circle is bisected by its diameter\footnote{\ibid~p.~157, 10, 11.}
(though the proof by \emph{reductio ad absurdum} which follows in
Proclus cannot be attributed to Thales\footnote{Cantor
  (\emph{Gesch.\ d.\ Math.} \r1\tsub{3}, p.~221) points out the
  connexion between the \emph{reductio ad absurdum} and the analytical
  method said to have been discovered by Plato, Proclus gives the
  proof by \emph{reductio ad absurdum} to meet an imaginary critic who
  desires a mathematical proof; possibly Thales may have been
  satisfied with the argument in the same sentence which mentions
  Thales, ``the cause of the bisection being the unswerving course of
  the straight line through the centre.''}), (2)~that ``Plato made
over to Leodamas the analytical method, by means of which \emph{it is
  recorded} (\greek{ἱστόρηται}) that the latter too made many
discoveries in geometry\footnote{Proclus, p.~211, 19–23.},'' (3)~that
the theorem of \prop{1}{5} was due to Thales, and that for equal
angles he used the more archaic expression ``similar''
angles\footnote{\ibid~p.~250, 20.}, (4)~that Oenopides first
investigated the problem of \prop{1}{12}, and that he called the
perpendicular the \emph{gnomonic} line (\greek{κατὰ
  γνώμονα})\footnote{\ibid~p.~283, 7–10.}, (5)~that the theorem that
only three sorts of polygons can fill up the space round a point,
viz.\ the equilateral triangle, the square and the regular hexagon,
was Pythagorean\footnote{\ibid~pp.~304, 11–305, 3.}. Eudemus may also
be the authority for Proclus' description of the two methods, referred
to Plato and Pythagoras respectively, of forming right-angled
triangles in whole numbers\footnote{\ibid~pp.~428, 7–429, 9. }.

We cannot attribute to Eudemus the beginning of the note on
\prop{1}{47} where Proclus says that ``if we listen to those who like
to recount ancient history, we may find some of them referring this
theorem to Pythagoras and saying that he sacrificed an ox in honour of
his discovery\footnote{\ibid~p.~426, 6–9.}.'' As such a sacrifice was
contrary to the Pythagorean tenets, and Eudemus could not have been
unaware of this, the story cannot rest on his authority. Moreover
Proclus speaks as though he were not certain of the correctness of the
tradition; indeed, so far as the story of the sacrifice is concerned,
the same thing is told of Thales in connexion with his discovery that
the angle in a semi-circle is a right angle\footnote{Diogenes
  Laertius, \r1. 24, p.~6, ed.\ Cobet.}, and Plutarch is not certain
whether the ox was sacrificed on the discovery of \prop{1}{47} or of
the problem about application of areas\footnote{Plutarch, \emph{non
    posse suaviter vivi secundum Epicurum}, \r2; \emph{Symp.} \r8,
  2.}. Plutarch's doubt suggests that he knew of no evidence for the
story beyond the vague allusion in the distich of Apollodorus
``Logisticus'' (the ``calculator'') cited by Diogenes Laertius also\footnote{Diog.\ Laert.\ \r8. 12, p.~207, ed.\ Cobet:
\begin{verse}
\greek{Ἡνίκα Πυθαγόρης τὸ περικλεές εὔρετο βράμμα,}\\
\greek{κεῖν’ ἐφ’ ὅτῳ κλεινὴν ἤγαγε Βουθυσίην.}
\end{verse}
See on this subject Tannery, \emph{La Géométrie grecque}, p.~105.};
and Proclus may have had in mind this couplet with the passages of
Plutarch.

We come now to the question of the famous historical summary given by
Proclus\footnote{Proclus, pp.~64–70.}. No one appears to maintain that
Eudemus is the author of even the early part of this summary in the
form in which Proclus gives it. It is, as is well known, divided into
two distinct parts, between which comes the remark, ``Those who
compiled histories\footnote{The plural is well explained by Tannery,
  \emph{La Géométrie grecque}, pp.~73, 74. No doubt the author of the
  summary tried to supplement Eudemus by means of any other histories
  which threw light on the subject. Thus e.g.\ the allusion (p.~64,
  21) to the Nile recalls Herodotus. Cf.\ the expression in Proclus,
  p. 64, 19, \greek{παρὰ τῶν πολλῶν ἱστόρηται}.} bring the development
of this science up to this point.  Not much younger than these is
Euclid, who put together the Elements, collecting many of the theorems
of Eudoxus, perfecting many others by Theaetetus, and bringing to
irrefragable demonstration the things which had only been somewhat
loosely proved by his predecessors.'' Since Euclid was later than
Eudemus, it is impossible that Eudemus can have written this. Yet the
style of the summary after this point does not show any such change
from that of the former portion as to suggest different
authorship. The author of the earlier portion recurs frequently to the
question of the origin of the elements of geometry in a way in which
no one would be likely to do who was not later than Euclid; and it
must be the same hand which in the second portion connects Euclid's
Elements with the work of Eudoxus and Theaetetus\footnote{Tannery,
  \emph{La Géométrie grecque}, p.~75.}.

If then the summary is the work of one author, and that author not
Eudemus, who is it likely to have been? Tannery answers that it is
Geminus\footnote{\ibid~pp.~66–75.}; but I think, with van Pesch, that
he has failed to show why it should be Geminus rather than
another. And certainly the extracts which we have from Geminus' work
suggest that the sort of topics which it dealt with was quite
different; they seem rather to have been general questions of the
content of mathematics, and even Tannery admits that historical
details could only have come incidentally into the
work\footnote{\ibid~p.~19.}.

Could the author have been Proclus himself? Circumstances which seem
to suggest this possibility are (1)~that, as already stated, the
question of the origin of the Elements is kept prominent, (2)~that
there is no mention of Democritus, whom Eudemus would not be likely to
have ignored, while a follower of Plato would be likely enough to do
him the injustice, following the example of Plato who was an opponent
of Democritus, never once mentions him, and is said to have wished to
burn all his writings\footnote{Diog.\ Laertius, \r9. 40, p.~237,
  ed.\ Cobet.}, and (3)~the allusion at the beginning to the
``inspired Aristotle'' (\greek{ὁ δαιμονιος
  Ἀριστοτέλης})\footnote{Proclus, p. 64, 8.}, though this may easily
have been inserted by Proclus in a quotation made by him from someone
else. On the other hand there are considerations which suggest that
Proclus himself was not the writer.  (1)~The style of the whole
passage is not such as to point to him as the author. (2)~If he wrote
it, it is hardly conceivable that he would have passed over in silence
the discovery of the analytical method, the invention of Plato to
which he attached so much importance\footnote{Proclus, p. 211, 19
  sqq.; the passage is quoted above, p.~36\?.}.

There is nothing improbable in the conjecture that Proclus quoted
the summary from a compendium of Eudemus' history made by some
later writer: but as yet the question has not been definitely settled.
All that is certain is that the early part of the summary must have
been made up from scattered notices found in the great work of
Eudemus.

Proclus refers to another work of Eudemus besides the history, viz.\ a
book on \emph{The Angle} (\greek{})\footnote{\ibid, p.~125,
  8.}. Tannery assumes that this must have been part of the history,
and uses this assumption to confirm his idea that the history was
arranged according to \emph{subjects}, not according to chronological
order\footnote{Tannery, \emph{La Géométrie gecque}, p.~26.}. The
phraseology of Proclus however unmistakably suggests a separate work;
and that the history was \emph{chronologically} arranged seems to be
clearly indicated by the remark of Simplicius that Eudemus ``also
counted Hippocrates among the more ancient writers''
(\greek{})\footnote{Simplicius, ed.\ Diels, p.~69, 23.}.

The passage of Simplicius about the lunes of Hippocrates throws
considerable light on the style of Eudemus' history. Eudemus wrote in
a memorandum-like or summary manner (\greek{})\footnote{\ibid~p.~60,
  29,} when reproducing what he found in the ancient writers;
sometimes it is clear that he left out altogether proofs or
constructions of things by no means easy\footnote{Cf.\ Simplicius,
  p.~63, 19 sqq.; p.~64. 25 sqq.; also Usener's note ``de supplendis
  Hippocratis quas omisit Eudemus constructionibus'' added to Diels'
  preface, pp. xxiii—xxvi.}.

\subsection*{Geminus}

The discussions about the date and birthplace of Geminus form a whole
literature, as to which I must refer the reader to Manitius and
Tittel\footnote{Manitius, \emph{Gemini elementa astronomiae} (Teubner,
  1898), pp. 237- — 151; Tittel, art.\ ``Geminos'' in Pauly-Wissowa's
  \emph{Real-Encyclopädie der classischen Altertumswissenschaft},
  vol.~\r7. 1910.}, Though the name looks like a Latin name (Gemǐnus),
Manitius concluded that, since it appears as \greek{Γεμῖνος} in all
Greek \textsc{mss.}\ and as \greek{Γεμεῖνος} in some inscriptions, it
is Greek and possibly formed from \greek{γεμ} as \greek{Ἐργῖνος} is
from \greek{ἐργ} and \greek{Ἀλεξῖνος} from \greek{ἀλεξ} (cf.\ also
\greek{Ἰκτῖνος}, \greek{Κρατῖνος}). Tittel is equally positive that it
is Gemǐnus and suggests that \greek{Γεμῖνος} is due to a false analogy
with \greek{Ἀλεξῖνος} etc.\ and \greek{Γεμεῖνος} wrongly formed on the
model of \greek{Ἀντωνεῖνος}, \greek{Ἀγριππεῖνα}.  Geminus, a Stoic
philosopher, born probably in the island of Rhodes, was the author of
a comprehensive work on the classification of mathematics, and also
wrote, about 73–67~\bc, a not !ess comprehensive commentary on the
meteorological textbook of his teacher Posidonius of Rhodes.

It is the former work in which we are specially interested here.
Though Proclus made great use of it, he does not mention its title,
unless we may suppose that, in the passage (p.~177, 24) where, after
quoting from Geminus a classification of lines which never meet, he
says, ``these remarks I have selected from the \greek{φιλοκαλία} of
Geminus,'' \greek{φιλοκαλία} is a title or an alternative
title. Pappus however quotes a work of Geminus ``on the classification
of the mathematics'' (\greek{ἐν τῷ περὶ τῆς τῶν μαθημάτων
  τάξεως})\footnote{Pappus, ed,\ Hultsch, p.~1026, 9.}, while Eutocius
quotes from ``the sixth book of the doctrine of the mathematics''
(\greek{ἐν τῷ ἕκτῳ τῆς τῶν μαθημάτων θεωρίας})\footnote{Apollonius,
  ed.\ Heiberg, vol.~\r2. p.~170.}.  Tannery\footnote{Tannery,
  \emph{La Géométrie gecque}, pp.~18, 19.} pointed out that the former
title corresponds well enough to the long extract\footnote{Proclus,
  pp.~38, 1–42, 8.} which Proclus gives in his first prologue, and
also to the fragments contained in the \emph{Anonymi variae
  collectiones} published by Hultsch at the end of his edition of
Heron\footnote{Heron, ed.\ Hultsch, pp.~246, 16–249, 12.}; but it does
not suit most of the other passages borrowed by Proclus, The correct
title was therefore probably that given by Eutocius, \emph{The
  Doctrine}, or \emph{Theory}, \emph{of the Mathematics}; and Pappus
probably refers to one particular portion of the work, say the first
Book. If the sixth Book treated of conics, as we may conclude from
Eutocius, there must have been more Books to follow, because Proclus
has preserved us details about higher curves, which must have come
later. If again Geminus finished his work and wrote with the same
fulness about the other branches of mathematics as he did about
geometry, there must have been a considerable number of Books
altogether. At all events it seems to have been designed to give a
complete view of the whole science of mathematics, and in fact to be a
sort of encyclopaedia of the subject.

I shall now indicate first the certain, and secondly the probable,
obligations of Proclus to Geminus, in which task I have only to follow
van Pesch, who has embodied the results of Tittel's similar inquiry
also\footnote{Van Pesch, \emph{De Procli fontibus}, pp.~97–113.  The
  dissertation of Tittel is entitled \emph{De Gemini Stoici studiis
    mathematicis} (1895).}.  I shall only omit the passages as regards
which a case for attributing them to Geminus does not seem to me to
have been made out.

First come the following passages which must be attributed to
Geminus, because Proclus mentions his name:

(1)~(In the first prologue of Proclus\footnote{Proclus, pp.~38, 1–42,
  8, except the allusion in p.~41, 8–10, to Ctesibius and Heron and
  their pneumatic devices (\greek{θαυματοποιϊκή}), as regards which
  Proclus' authority may be Pappus (\r8.~p.~1024, 14–27) who uses very
  similar expressions.  Heron, even if not later than Geminus, could
  hardly have been included in a historical work by him.  Perhaps
  Geminus may have referred to Ctesibius only, and Proclus may have
  inserted ``and Heron'' himself.}) on the division of mathematical
sciences into arithmetic, geometry, mechanics, astronomy, optics,
geodesy, canonic (science of musical harmony), and logistic
(apparently arithmetical problems);

(2)~(in the note on the definition of a straight line) on the
classification of lines (including curves) as simple (straight or
circular) and mixed, composite and incomposite, uniform
(\greek{ὁμοιομερεῖς}) and non-uniform (\greek{ἀνομοιομερεῖς}), lines
``about solids'' and lines produced by cutting solids, including conic
and spiric sections\footnote{Proclus, pp.~103, 21–107, 10; pp.~111,
  1–113, 3.};

(3)~(in the note on the definition of a plane surface) on similar
distinctions extended to surfaces and solids\footnote{\ibid~pp.~117,
  14–120, 12, where perhaps in the passage pp.~117, 22–118, 23, where
  perhaps in the passage pp.~117, 22–118, 23 we may have Geminus' own
  words.};

(4)~(in the note on the definition of parallels) on lines which
\emph{do not meet} (\greek{ἀσύμπτωτοι}) but which are not on that
account parallel, e.g.\ a curve and its asymptote, showing that the
property of \emph{not meeting} does not make lines parallel—a
favourite observation of Geminus—and, incidentally, on \emph{bounded}
lines or those which \emph{enclose a figure} and those which do
not\footnote{\ibid~pp.~176, 18–177, 25; perhaps also p.~175.  The note
  ends with the words ``These things too we have selecled from
  Geminus' \greek{φιλοκαλία} for the elucidation of the matters in
  question.''  Tannery (p.~27) takes these words coming at the end of
  the commentary on the definitions as referring to the whole of the
  portion of the commentary dealing with the definitions.  Van Pesch
  properly regards them as only applying to the note on
  \emph{parallels}. This seems to me clear from the use of the word
  \emph{too} (\greek{τοσαῦτα καί}).};

(5)~(in the same note) the definition of parallels given by
Posidonius\footnote{Proclus, p.~176, 5–17.};

(6)~on the distinction between postulates and axioms, the futility of
trying to prove axioms, as Apollonius tried to prove Axiom~1, and the
equal incorrectness of assuming what really requires proof, ``as
Euclid did in the fourth postulate [equality of right angles] and in
the fifth postulate [the
  parallel-postulate]\footnote{\ibid~pp.~178–182, 4; pp.~183, 14–184,
  10; cf.\ p.~188, 3–11.}'';

(7)~on Postulates 1, 2, 3, which Geminus makes depend on the idea of a
straight line being described by the motion of a
point\footnote{\ibid~p.~185, 6–25.};

(8)~(in the note on Postulate~5) on the inadmissibility in geometry of
an argument which is merely plausible, and the danger in this
particular case owing to the existence of lines which do converge ad
infinitum and yet never meet\footnote{\ibid~p.~192, 5–29.};

(9)~(in the note on \prop{1}{1}) on the subject-matter of geometry,
theorems, problems and \greek{διορισμοί} (conditions of possibility)
for problems\footnote{\ibid~pp.~200, 21–202, 25.};

(10)~(in the note on \prop{1}{5}) on a generalisation of \prop{1}{5}
by Geminus through the substitution for the rectilineal base of ``one
uniform line (curve),'' by means of which he proved that the only
``uniform lines'' (alike in all their parts) are a straight line, a
circle, and a cylindrical helix\footnote{Proclus, p.~251, 2–11.};

(11)~(in the note on \prop{1}{10}) on the question whether a line is
made up of indivisible parts (\greek{ἀμερῆ}), as affecting the problem
of bisecting a given straight line\footnote{\ibid~pp.~277, 25–279,
  11.};

(12) (in the note on \prop{1}{35}) on \emph{topical}, or
\emph{locus}-theorems\footnote{\ibid~pp.~394, 11–395, 2 and p.~395,
  13–21}, where the illustration of the equal parallelograms described
between a hyperbola and its asymptotes may also be due to
Geminus\footnote{\ibid~p.~395, 8–12.}.

Other passages which may fairly be attributed to Gem in us, though his
name is not mentioned, are the following:

(1)~in the prologue, where there is the same allusion as in the
passage~(8) above to a remark of Aristotle that it is equally absurd
to expect scientific proofs from a rhetorician and to accept mere
plausibilities from a geometer\footnote{\ibid~pp.~33. 21–34, 1.};

(2)~a passage in the prologue about the subject-matter, methods, and
bases of geometry, the latter including axioms and
postulates\footnote{\ibid~pp.~57, 9–58, 3.};

(3)~another on the definition and nature of
\emph{elements}\footnote{\ibid~pp.~72, 3–75, 4. };

(4)~a remark on the Stoic use of the term axiom for every simple
statement (\greek{ἀπόφανσις ἁπλῆ})\footnote{\ibid~p.~77, 3–6.};

(5)~another discussion on theorems and problems\footnote{\ibid~pp.~77,
  7–78, 13, and 79, 3–81, 4.}, in the middle of which however there
are some sentences by Proclus himself\footnote{\ibid~pp.~78, 13–79,
  2.}.

(6)~another passage, in connexion with Def.~3, on lines including or
not including a figure (with which cf.\ part of the passage~(4)
above)\footnote{\ibid~pp.~102, 22–103, 18.};

(7)~a classification of different sorts of angles according as they
are contained by simple or mixed lines (or
curves)\footnote{\ibid~pp.~126, 7–127, 16.};

(8)~a similar classification of figures\footnote{\ibid~pp.~159,
  12–160, 9.}, and of plane figures\footnote{\ibid~pp.~162, 27–164,
  6.};

(9)~Posidonius' definition of a \emph{figure}\footnote{\ibid~p.~143, 5–11. };

(10)~a classification of triangles into seven
kinds\footnote{\ibid~p.~168, 4–12.};

(11)~a note distinguishing lines (or curves) producible indefinitely
or not so producible, whether forming a figure or not forming a figure
(like the ``single-turn spiral'')\footnote{\ibid~p.~187, 19–27.};

(12)~passages distinguishing different sorts of
problems\footnote{\ibid~pp.~220, 7–222, 14; also p.~330, 6–9.},
different sorts of theorems\footnote{\ibid~pp.~244, 14–246, 12.}, and
two sorts of converses (complete and partial)\footnote{\ibid~pp.~252,
  5–254, 20.};

(13)~the definition of the term ``porism ``as used in the title of
Euclid's \emph{Porisms}, as distinct from the other meaning of
``corollary''\footnote{\ibid~pp.~301, 21–302, 13.};

(14)~a note on the Epicurean objection to \prop{1}{20} as being
obvious even to an ass\footnote{\ibid~pp.~322, 4–323. 3.};

(15)~a passage on the properties of parallels, with allusions to
Apollonius' \emph{Conics}, and the curves invented by Nicomedes,
Hippias and Perseus\footnote{Proclus, pp.~355, 20–356, 16. };

(16)~a passage on the parallel-postulate regarded as the converse of
\prop{1}{17}\footnote{\ibid~p.~364, 9–11; pp.~364, 20–365, 4.}.

Of the authors to whom Proclus was indebted in a less degree the most
important is \textbf{Apollonius of Perga}. Two passages allude to his
\emph{Conics}\footnote{\ibid~p.~71, 19; p.~356, 8, 6.}, one to a work
on irrationals\footnote{\ibid~p.~74, 23, 24.} and two to a treatise
\emph{On the cochlias} (apparently the cylindrical helix) by
Apollonius\footnote{\ibid~pp.~105, 5, 6, 14, 15. }. But more important
for our purpose are six references to Apollonius in connexion with
elementary geometry,

(1)~He appears as the author of an attempt to explain the idea of a
line (possessing length but no breadth) by reference to daily
experience, e.g.\ when we tell someone to measure, merely, the length
of a road or of a wall\footnote{\ibid~p.~100, 5–19.}; and doubtless
the similar passage showing how we may in like manner get a notion of
a surface (without depth) is his also\footnote{\ibid~p.~114, 20–25}.

(2)~He gave a new general definition of an
angle\footnote{\ibid~p.~123, 15–19 (cf.\ p.~124, 17. p.~125, 17).}.

(3)~He tried to prove certain axioms\footnote{\ibid~p.~183, 13, 14.},
and Proclus gives his attempt to prove Axiom~1, word for
word\footnote{\ibid~pp.~194, 25–195, 5.}.

Proclus further quotes:

(4)~Apollonius' solution of the problem in Eucl.\ \prop{1}{10},
avoiding Euclid's use of \prop{1}{9}\footnote{\ibid~pp.279, 16–280,
  4.}.

(5)~his solution of the problem in \prop{1}{11}, differing only
slightly from Euclid's\footnote{\ibid~p.~282, 8–19.}, and

(6)~his solution of the problem in
\prop{1}{23}\footnote{\ibid~pp.~335, 16–336, 5.}.

Heiberg\footnote{\emph{Philologus}, vol. \r43. p.~489.} conjectures
that Apollonius departed from Euclid's method in these propositions
because he objected to solving problems of a more general, by means of
problems of a more particular, character. Proclus however considers
all three solutions inferior to Euclid's; and his remarks on
Apollonius' handling of these elementary matters generally suggest
that he was nettled by criticisms of Euclid in the work containing the
things which he quotes from Apollonius, just as we conclude that
Pappus was offended by the remarks of Apollonius about Euclid's
incomplete treatment of the ``three- and four-line locus\footnote{See
  above, pp.~2, 3.}.'' If this was the case, Proclus can hardly have
got his information about these things at second-hand; and there seems
to be no reason to doubt that he had the actual work of Apollonius
before him. This work may have been the treatise mentioned by Marinus
in the words ``Apollonius in his general treatise''
(\greek{})\footnote{\emph{Marinus in Euclidis Data}, ed.\ Menge,
  p.~234, 16.}.  If the notice in the
\emph{Fihrist}\footnote{\emph{Fihrist}, tr.\ Suter, p.~19.}  stating,
on the authority of Thābit b.~Qurra, that
Apollonius wrote a tract on the parallel-postulate be correct, it may
have been included in the same work. We may conclude generally
that, in it, Apollonius tried to remodel the beginnings of geometry,
reducing the number of axioms, appealing, in his definitions of lines,
surfaces etc., more to experience than to abstract reason, and
substituting for certain proofs others of a more general character.

The probabilities are that, in quoting from the tract of Ptolemy in
which he tried to prove the paralel-postulate, Proclus had the actual
work before him. For, after an allusion to it as ``a certain
book\footnote{Proclus, p.~191, 23.}'' he gives two long
extracts\footnote{\ibid~pp.~362, 14–363, 19; pp.~365, 7–367, 27.}, and
at the beginning of the second indicates the title of the tract, ``in
the (book) about the meeting of straight lines produced from (angles)
less than two right angles,'' as he has very rarely done in other
cases.

Certain things from \textbf{Posidonius} are evidently quoted at
second-hand, the authority being Geminus (e.g.\ the definitions of
\emph{figure} and \emph{parallels}); but besides these we have
quotations from a separate work which he wrote to controvert Zeno of
Sidon, an Epicurean who had sought to destroy the whole of
geometry\footnote{\ibid~p.~200, 1–3.}. We are told that Zeno had
argued that, even if we admit the fundamental principles
(\greek{ἀρχαί}) of geometry, the deductions from them cannot be proved
without the admission of something else as well, which has not been
included in the said principles\footnote{\ibid~pp.~199, 11—200,
  1.}. On \prop{1}{1} Proclus gives at some length the arguments of
Zeno and the reply of Posidonius as regards this
proposition\footnote{\ibid~pp.~214, 18–215, 13; pp.~216, 10–218, 11.}.
In this case Zeno's ``something else'' which he considers to be
assumed is the fact that two straight lines cannot have a common
segment, and then, as regards the ``proof'' of it by means of the
bisection of a circle by its diameter, he objects that it has been
assumed that two \emph{circumferences} (arcs) of circles cannot have a
common part. Lastly, he makes up, for the purpose of attacking it,
another supposed ``proof'' of the fact that two straight lines cannot
have a common part. Proclus appears, more than once, to be quoting the
actual words of Zeno and Posidonius; in particular, two expressions
used by Posidonius about ``the acrid Epicurean'' (\greek{τὸν δριμὺν
  Ἐπικούρειον})\footnote{\ibid~p.~216, 21.} and his
``misrepresentations'' (\greek{Ποσειδώνιός φησι τὸν Ζήνωνα
  συκοφαντεῖν})\footnote{\ibid~p.218, 1.}. It is not necessary to
suppose that Proclus had the original work of Zeno before him, because
Zeno's arguments may easily have been got from Posidonius' reply; but
he would appear to have quoted direct from the latter at all events.

The work of Carpus \emph{mechanicus} (a treatise on astronomy) quoted
from by Proclus\footnote{\ibid~pp.~241, 19–243, 11.} must have been
accessible to him at first-hand, because a portion of the extract from
it about the relation of theorems and problems\footnote{\ibid~pp.~242,
  22—243, 11.} is reproduced word for word. Moreover, if he were not
using the book itself, Proclus would hardly be in a position to
question whether the introduction of the subject of theorems and
problems was opportune in the place where it was found (\greek{εἰ μὲν
  κατὰ καιρὸν ἤ μή, παρείσθω πρὸς τὸ παρόν})\footnote{Proclus, p.~241,
  21, 22.}.

It is of course evident that Proclus had before him the original works
of Plato, Aristotle, Archimedes and Plotinus, as well as the
\greek{Συμμικτά} of Porphyry and the works of his master Syrianus
(\greek{ὁ ἡμέτερος καθηγεμόν})\footnote{\ibid~p.~123, 19,}, from whom
he quotes in his note on the definition of an angle. Tannery also
points out that he must have had before him a group of works
representing the Pythagorean tradition on its mystic, as distinct from
its mathematical, side, from Philolaus downwards, and comprising the
more or less apocryphal \greek{ἱερός λόγος} of Pythagoras, the Oracles
(\greek{λόγια}), and Orphic verses\footnote{Tannery, \emph{La
    Géométrie grecque}, pp.~25, 26.}.

Besides quotations from writers whom we can identify with more or less
certainty, there are many other passages which are doubtless quoted
from other commentators whose names we do not know. A list of such
passages is given by van Pesch\footnote{Van Pesch, \emph{De Procli
    fontibus}, p.~139.}, and there is no need to cite them here.

Van Pesch also gives at the end of his work\footnote{\ibid~p.~155.} a
convenient list of the books which, as the result of his
investigation, he deems to have been accessible to and directly used
by Proclus. The list is worth giving here, on the same ground of
convenience. It is as follows:

Eudemus: \emph{history of geometry}.

Geminus: \emph{the theory of the mathematical sciences}.

Heron: \emph{commentary on the Elements of Euclid}.

Porphyry: \emph{commentary on the Elements of Euclid}.

Pappus: \emph{commentary on the Elements of Euclid}.

Apollonius of Perga: a work relating to elementary geometry.

Ptolemy: \emph{on the parallel-postulate}.

Posidonius: a book controverting Zeno of Sidon.

Carpus: \emph{astronomy}.

Syrianus: a discussion on the \emph{angle}.

Pythagorean philosophical tradition.

Plato's works.

Aristotle's works.

Archimedes' works,

Plotinus: \emph{Enneades}.

Lastly we come to the question what passages, if any, in the
commentary of Proclus represent his own contributions to the subject.
As we have seen, the \emph{onus probandi} must be held to rest upon
him who shall maintain that a particular note is original on the part
of Proclus, Hence it is not enough that it should be impossible to
point to another writer as the probable source of a note; we must have
a positive reason for attributing it to Proclus, The criterion must
therefore be found either (1)~in the general terms in which Proclus
points out the deficiencies in previous commentaries and indicates the
respects in which his own will differ from them, or (2)~in specific
expressions used by him in introducing particular notes which may
indicate that he is giving his own views. Besides indicating that he
paid more attention than his predecessors to questions requiring
deeper study (\greek{τὸ πραγματειῶδες}) and ``pursued clear
distinctions'' (\greek{τὸ εὐδιαίρετον
  μεταδιώκοντας})\footnote{Proclus, p.~84, 13, p.~432, 14, 15.}—by
which he appears to imply that his predecessors had confused the
different departments of their commentaries, viz.\ lemmas, cases, and
objections (\greek{ἐνστάσεις})\footnote{cf.\ \ibid~p.~289, 11–15;
  p.~432, 15–17.}—Proclus complains that the earlier commentators had
failed to indicate the ultimate grounds or \emph{causes} of
propositions\footnote{\ibid~p.~432, 17.}. Although it is from Geminus
that he borrowed a passage maintaining that it is one of the proper
functions of geometry to inquire into causes (\greek{τὴν αἰτίαν καὶ τὸ
  διὰ τί})\footnote{\ibid~p.~202, 9–25.}, yet it is not likely that
Geminus dealt with Euclid's propositions one by one; and consequently,
when we find Proclus, on \prop{1}{8, 16, 17, 18, 32, and
  47}\footnote{See Proclus, p.~270, 5–14 (\prop{1}{8}); pp.~309,
  3–310, 8 (\prop{1}{16}); pp.~310, 19–311, 23 (\prop{1}{17}), p.~316,
  14–318, 2 (\prop{1}{18}); p.~384, 13–21 (\prop{1}{32}); pp.~426,
  22–427, 8 (\prop{1}{47}).}, endeavouring to explain \emph{causes},
we have good reason to suppose that the explanations are his own.

Again, his remarks on certain things which he quotes from Pappus
can scarcely be due to anyone else, since Pappus is the latest of the
commentators whose works he appears to have used. Under this
head, come

(1)~his objections to certain new axioms introduced by
Pappus\footnote{\ibid~p.~198, 5–15.},

(2)~his conjecture as to how Pappus came to think of his alternative
proof of \prop{1}{5}\footnote{\ibid~p.~250, 12–19.},

(3)~an addition to Pappus' remarks about the curvilineal angle which
is equal to a right angle without being one\footnote{\ibid~.p~190,
  9–23.}.

The defence of Geminus against Carpus, who combated his view of
theorems and problems, is also probably due to
Proclus\footnote{\ibid~p.~243, 12–29..}, as well as an observation on
\prop{1}{38} to the effect that \prop{1}{35–38} are really
comprehended in \prop{6}{1} as particular
cases\footnote{\ibid~pp.~405, 6–406, 9.}.

Lastly, we can have no hesitation in attributing to Proclus himself
(1)~the criticism of Ptolemy's attempt to prove the
parallel-postulate\footnote{\ibid~p.~368, 1–23.}, and (2)~the other
attempted proof given in the same note\footnote{\ibid~pp.~371, 11–373,
  2.} (on \prop{1}{29}) and assuming as an axiom that ``if from one
point two straight lines forming an angle be produced \emph{ad
  infinitum} the distance between them when so produced \emph{ad
  infinitum} exceeds any finite magnitude (i.e.\ length),'' an
assumption which purports to be the equivalent of a statement in
Aristotle\footnote{Aristotle, \emph{de caelo}, \prop{1}{5}
  (271~b~28–30).}.  It is introduced by words in which the writer
appears to claim originality for his proof: ``To him who desires to
see this proved (\greek{κατασκευαζόμενον}) \emph{let it be said by us}
(\greek{λεγέσθω παρ’ ἤμῶν})'' etc.\footnote{Proclus, p..~371, 10.}
Moreover, Philoponus, in a note on Aristotle's
\emph{Anal.\ post.}\ \r1.~10, says that ``the geometer (Euclid)
assumes this as an axiom, but it wants a great deal of proof, insomuch
that both Ptolemy and Proclus wrote a whole book upon
it\footnote{Berlin Aristotle, vol.~\r4. p.~214~a~9–12.}.''

\chapter{The Text\protect\footnote{The material for the whole of this chapter is taken from Heiberg's
  edition of the \emph{Elements}, introduction to vol.~\r5., and from
  the same scholar's \emph{Litterargeschichtliche Studien über
    Euklid}, p.~174 sqqf\. and \emph{Paralipomena zu Euklid} in
  \emph{Hermes}, \r38., 2903}.}

It is well known that the title of Simson's edition of Euclid (first
brought out in Latin and English in 1756) claims that, in it, ``the
errors by which Theon, or others, have long ago vitiated these books
are corrected, and some of Euclid's demonstrations are restored''; and
readers of Simson's notes are familiar with the phrases used, where
anything in the text does not seem to him satisfactory, to the effect
that the demonstration has been spoiled, or things have been
interpolated or omitted, by Theon ``or some other unskilful editor.''
Now most of the \textsc{mss.}\ of the Greek text prove by their titles
that they proceed from the recension of the \emph{Elements} by Theon;
they purport to be either ``from the edition of Theon'' (\greek{ἐκ τῆς
  Θέωνος}) or ``from the lectures of Theon'' (\greek{ἀπὸ συνουσιῶν τοῦ
  Θέωνος}). This was Theon of Alexandria (4th c.~\ad) who also wrote a
commentary on Ptolemy, in which there occurs a passage of the greatest
importance in this connexion\footnote{\r1. p.~201 ed.\ Halma = p.~50
  ed.~Basel.}: ``But that sectors in equal circles are to one another
as the angles on which they stand \emph{has been proved by me in my
  edition of the Elements at the end of the sixth book}.'' Thus Theon
himself says that he edited the \emph{Elements} and also that the
second part of \prop{6}{33}, found in nearly all the \textsc{mss.}, is
his addition.

This passage is the key to the whole question of Theon's changes in
the text of Euclid; for, when Peyrard found in the Vatican the
\textsc{ms.}~190 which contained neither the words from the titles of
the other \textsc{mss.}\ quoted above nor the interpolated second part
of \prop{6}{33}, he was justified in concluding, as he did, that in
the Vatican \textsc{ms.}\ we have an edition more ancient than
Theon's. It is also clear that the copyist of~P, or rather of its
archetype, had before him the two recensions and systematically gave
the preference to the earlier one; for at \prop{13}{6} in~P the first
hand has added a note in the margin: ``This theorem is not given in
most copies of the \emph{new edition}, but is found in those of the
old.''  Thus we are more fortunate than Simson, since our judgment of
Theon's recension can be formed on the basis, not of mere conjecture,
but of the documentary evidence afforded by a comparison of the
Vatican \textsc{ms.}\ just mentioned with what we may conveniently
call, after Heiberg, the Theonine \textsc{mss.}

The \textsc{mss.}\ used for Heiberg's edition of the \emph{Elements}
are the following:

(1)~P = Vatican \textsc{ms.}\ numbered 190, 4to, in two volumes
(doubtless one originally); 10th~c.

This is the \textsc{ms.}\ which Peyrard was able to use; it was sent
from Rome to Paris for his use and bears the stamp of the Paris
Imperial Library on the last page. It is well and carefully written.
There are corrections some of which are by the original hand, but
generally in paler ink, others, still pretty old, by several different
hands, or by one hand with different ink in different places (P m.~2),
and others again by the latest hand (P m.~rec.). It contains, first,
the \emph{Elements} \r1.–\r13.\ with scholia, then Marinus' commentary
on the Data (without the name of the author), followed by the
\emph{Data} itself and scholia, then the \emph{Elements} \r14.,
\r15.\ (so called), and lastly three books and a part of a fourth of a
commentary by Theon \greek{εἰς τοὺς προχείρους κανόνας Πτολεμαίου}.

The other \textsc{mss.}\ are ``Theonine.''

(2)~F = \textsc{ms.}\ \r38,~3, in the Laurentian Library at Florence,
4to; 10th~c.

This \textsc{ms.}\ is written in a beautiful and scholarly hand and
contains the \emph{Elements} \r1.–\r15., the \emph{Optics} and the
\emph{Phaenomena}, but is not well preserved.  Not only is the
original writing renewed in many places, where it had become faint, by
a later hand of the 16th~c., but the same hand has filled certain
smaller lacunae by gumming on to torn pages new pieces of parchment,
and has replaced bodily certain portions of the \textsc{ms.}, which
had doubtless become illegible, by fresh leaves. The larger gaps so
made good extend from Eucl.\ \prop{7}{12} to \prop{9}{15}, and from
\prop{12}{3} to the end; so that, besides the conclusion of the
\emph{Elements}, the \emph{Optics} and \emph{Phaenomena} are also in
the later hand, and we cannot even tell what in addition to the
\emph{Elements} \r1.–\r13.\ the original
\textsc{ms.}\ contained. Heiberg denotes the later hand by \greek{φ}
and observes that, while in restoring words which had become faint and
filling up minor lacunae the writer used no other \textsc{ms.}, yet in
the two larger restorations he used the Laurentian
\textsc{ms.}\ \r28,~6, belonging to the 13th–14th~c. The latter
\textsc{ms.}\ (which Heiberg denotes by~f) was copied from the
Viennese \textsc{ms.}~(V) to be described below.

(3)~B = Bodleian \textsc{ms.}, D'Orville \r10. 1 inf.~2, 30, 4to;
\ad~888.  This \textsc{ms.}\ contains the \emph{Elements}
\r1.–\r15.\ with many scholia. Leaves 15–118 contain \prop{1}{14}
(from about the middle of the proposition) to the end of Book~\r6.,
and leaves 123–387 (wrongly numbered~397) Books \r7.–\r15, in one and
the same elegant hand (9th~c.). The leaves preceding leaf~15 seem to
have been lost at some time, leaves 6 to~14 (containing Elem.\ \r1. to
the place in \prop{1}{14} above referred to) being carelessly written
by a later hand on thick and common parchment (13th~c.). On leaves 2
to~4 and~122 are certain notes in the hand of Arethas, who also wrote
a two-line epigram on leaf~5, the greater part of the scholia in
uncial letters, a few notes and corrections, and two sentences on the
last leaf, the first of which states that the \textsc{ms.}\ was
written by one Stephen \emph{clericus} in the year of the world 6397
(= 888~\ad), while the second records Arethas' own acquisition of it.
Arethas lived from, say, 865 to 939~\ad.  He was Archbishop of
Caesarea and wrote a commentary on the Apocalypse. The portions of his
library which survive are of the greatest interest to palaeography on
account of his exact notes of dates, names of copyists, prices of
parchment etc.  It is to him also that we owe the famous Plato
\textsc{ms.}\ from Patmos (Cod.\ Clarkianus) which was written for him
in November~895\footnote{See Pauly-Wissowa, \emph{Real-Encyclopädie
    der class.\ Altertumswissenschaft}, vol.~\r2, 1896,}.

(4) V - Viennese \textsc{ms.}\ Philos.\ Gr. No.~103; probably 12th~c.

This \textsc{ms.}\ contains 292 leaves, Eucl.\ \emph{Elements}
\r1.–\r15.\ occupying leaves 1 to~254, after which come the
\emph{Optics} (to leaf~71), the \emph{Phaenomena} (mutilated at the
end) from leaf~272 to leaf~282, and lastly scholia, on leaves 283
to~292, also imperfect at the end. The different material used for
different parts and the varieties of handwriting make it necessary for
Heiberg to discuss this \textsc{ms.}\ at some length\footnote{Heiberg,
  vol.~\r5. pp.~xxix—xxxiii.}. The handwriting on leaves 1 to~183
(Book~\r1.\ to the middle of \prop{10}{105}) and on leaves 203 to~234
(from \prop{11}{31}, towards the end of the proposition, to
\prop{13}{7}, a few lines down) is the same; between leaves 184
and~202 there are two varieties of handwriting, that of leaves 184
to~189 and that of leaves 200 (verso) to~202 being the same. Leaf~235
begins in the same handwriting, changes first gradually into that of
leaves 184 to~189 and then (verso) into a third more rapid cursive
writing which is the same as that of the greater part of the scholia,
and also as that of leaves 243 and~282, although, as these leaves are
of different material, the look of the writing and of the ink seems
altered.  There are corrections both by the first and a second hand,
and scholia by many hands.  On the whole, in spite of the apparent
diversity of handwriting in the \textsc{ms.}, it is probable that the
whole of it was written at about the same time, and it may (allowing
for changes of material, ink etc) even have been written by the same
man.  It is at least certain that, when the Laurentian
\textsc{ms.}\ \r28,~6 was copied from it, the whole \textsc{ms.}\ was
in the condition in which it is now, except as regards the later
scholia and leaves 283 to~292 which are not in the Laurentian
\textsc{ms.}, that \textsc{ms.}\ coming to an end where the
\emph{Phaenomena} breaks off abruptly in~V\@.  Hence Heiberg
attributes the whole \textsc{ms.}\ to the 12th~c.

But it was apparently in two volumes originally, the first consisting
of leaves 1 to~183; and it is certain that it was not all copied at
the same time or from one and the same original. For leaves 184 to~202
were evidently copied from two \textsc{mss.}\ different both from one
another and from that from which the rest was copied. Leaves 184 to
the middle of leaf~189 (recto) must have been copied from a
\textsc{ms.}\ similar to~P, as is proved by similarity of readings,
though not from~P itself. The rest, up to leaf~202, were copied from
the Bologna \textsc{ms.}~(b) to be mentioned below. It seems clear
that the content of leaves 184 to~202 was supplied from other
\textsc{mss.}\ because there was a lacuna in the original from which
the rest of~V was copied.

Heiberg sums up his conclusions thus. The copyist of~V first copied
leaves 1 to~183 from an original in which two \emph{quaterntones} were
missing (covering from the middle of Eucl.\ \prop{10}{105} to near the
end of \prop{11}{31}). Noticing the lacuna he put aside one
\emph{quaternio} of the parchment used up to that point. Then he
copied onwards from the end of the lacuna in the original to the end
of the \emph{Phaenomena}.  After this he looked about him for another
\textsc{ms.}\ from which to fill up the lacuna; finding one, he copied
from it as far as the middle of leaf 189 (recto). Then, noticing that
the \textsc{ms.}\ from which he was copying was of a different class,
he had recourse to yet another \textsc{ms.}\ from which he copied up
to leaf~202, At the same time, finding that the lacuna was longer than
he had reckoned for, he had to use twelve more leaves of a different
parchment in addition to the quatemio which he had put aside. The
whole \textsc{ms.}\ at first formed two volumes (the first containing
leaves 1 to~183 and the second leaves 184 to~282); then, after the
last leaf had perished, the two volumes were made into one to which
two more \emph{quaterniones} were also added. A few leaves of the
latter of these two have since perished,

(5)~b = \textsc{ms.}\ numbered 18–19 in the Communal Library at
Bologna, in two volumes, 4to; 11th~c.

This \textsc{ms.}\ has scholia in the margin written both by the first
hand and by two or three later hands; some are written by the latest
hand, Theodorus Cabasilas (a descendant apparently of Nicolaus
Cabasilas, 14th~c.) who owned the \textsc{ms.}\ at one time. It
contains (\emph{a}) in 14 \emph{quaterniones} the definitions and the
enunciations (without proofs) of the \emph{Elements} \r1.–\r13, and of
the \emph{Data}, (\emph{b})~in the remainder of the volumes the
\emph{Proem to Geometry} (published among the \emph{Variae
  Collectiones} in Hultsch's edition of Heron, pp.~252, 24 to~274, 14)
followed by the \emph{Elements} \r1.–\r13.\ (part of \prop{13}{18} to
the end being missing), and then by part of the \emph{Data} (from the
last three words of the enunciation of Prop.~38 to the end of the
penultimate clause in Prop.~87, ed.\ Menge). From \prop{11}{36}
inclusive to the end of \r12.\ this \textsc{ms.}\ appears to represent
an entirely different recension. Heiberg is compelled to give this
portion of~b separately in an appendix. He conjectures that it is due
to a Byzantine mathematician who thought Euclid's proofs too long and
tiresome and consequently contented himself with indicating the course
followed\footnote{\emph{Zeitschrift für Math.\ u.\ Physik}, \r29.,
  hist.-litt.\ Abtheilung, p.~13.}.  At the same time this Byzantine
must have had an excellent \textsc{ms.}\ before him, probably of the
ante-Theonine variety of which the Vatican \textsc{ms.}~190 (P) is the
sole representative.

(6)~p = Paris \textsc{ms.}\ 2466, 4to; 12th~c.

This manuscript is written in two hands, the finer hand occupying
leaves 1 to~53 (recto), and a more careless hand leaves 53 (verso)
to~64, which are of the same parchment as the earlier leaves, and
leaves 65 to~239, which are of a thinner and rougher parchment showing
traces of writing of the 8th–9th~c.\ (a Greek version of the Old
Testament). The \textsc{ms.}\ contains the \emph{Elements}
\r1.–\r13.\ and some scholia after Books \r11., \r12.\ and \r13.

(7)~q = Paris \textsc{ms.}\ 2344, folio; 12th c.

It is written by one hand but includes scholia by many hands.  On
leaves 1 to~16 (recto) are scholia with the same title as that found
by Wachsmuth in a Vatican \textsc{ms.}\ and relied upon by him to
prove that Proclus continued his commentaries beyond
Book~\r1.\footnote{\greek{[εἰς τ]ὰ τοῦ Εὐκλείδου στοιχεῖα
    προλαμβανόμενα ἐκ το›ν Πρόκλου σποράδην καὶ κατ’ έπιτομήν}.
  cf.\ p.~32, note 8, above.} Leaves 17 to~357 contain the
\emph{Elements} \r1.–\r13.\ (except that there is a lacuna from the
middle of \prop{8}{25} to the \greek{ἔκθεσισ} of \prop{9}{14}); before
Books \r7.\ and \r10.\ there are some leaves filled with scholia only,
and leaves 358 to~366 contain nothing but scholia.

(8)~Heiberg also used a palimpsest in the British Museum
(Add.\ 17211). Five pages are of the 7th–8th~c.\ and are contained
(leaves 49–53) in the second volume of the Syrian
\textsc{ms.}\ Brit.\ Mus.\ 687 of the 9th~c.; half of leaf~50 has
perished. The leaves contain various fragments from
Book~\r10.\ enumerated by Heiberg, Vol.~\r3., p.~v, and nearly the
whole of \prop{13}{14}.

Since his edition of the \emph{Elements} was published, Heiberg has
collected further material bearing on the history of the
text\footnote{Heiberg, \emph{Paralipomena zu Euklid} in \emph{Hermes},
  \r38., 1903, pp.~46–74, 161–201, 321—356.}.  Besides giving the
results of further or new examination of \textsc{mss.}, he has
collected the fresh evidence contained in an-Nairīzī's commentary, and
particularly in the quotations from Heron's commentary given in it
(often word for word), which enable us in several cases to trace
differences between our text and the text as Heron had it, and to
identify some interpolations which actually found their way into the
text from Heron's commentary itself; and lastly he has dealt with some
valuable fragments of ancient papyri which have recently come to
light, and which are especially important in that the evidence drawn
from them necessitates some modification in the views expressed in the
preface to Vol.~\r5.\ as to the nature of the changes made in Theon's
recension, and in the principles laid down for differentiating between
Theon's recension and the original text, on the basis of a comparison
between P and the Theonine \textsc{mss.}\ alone.

The fragments of ancient papyri referred to are the following.

1.~\emph{Papyrus Herculanensis} No.~1061\footnote{Described by Heiberg
  in \emph{Oversigt over det kngl.\ danske Videnskabernes Selskabs
    Forhandlinger}, 1900, p.~161.}.

This fragment quotes Def.~15 of Book~\r1.\ in Greek, and omits the
words \greek{ἢ καλεῖται περιφέρεια}, ``which is called the
circumference,'' found in all our \textsc{mss.}, and the further
addition \greek{πρὸς τὴν τοῦ κύκλου περιφέρειαν} also found in
practically all the \textsc{mss.}  Thus Heiberg's assumption that both
expressions are interpolations is now confirmed by this oldest of all
sources.

2.~\emph{The Oxyrhynchus Papyri}~\r1.\ p.~58, No.~\r29.\ of the 3rd or
4th~c.

This fragment contains the enunciation of Eucl.\ \prop{2}{5} (with
figure, apparently without letters, immediately following, and not, as
usual in our \textsc{mss.}, at the end of the proof) and before it the
part of a word \greek{περιεχομε} belonging to \prop{2}{4} (with room
for -\greek{νῳ ὀρθογωνίῳ ὅπερ ἔδει δεῖξαι} and a stroke to mark the
end), showing that the fragment \emph{had not} the Porism which
appears in all the Theonine \textsc{mss.}\ and (in a later hand) in~P,
and thereby confirming Heiberg's assumption that the Porism was due to
Theon.

3.~A fragment in \emph{Fayum towns and their papyri}, p.~96,
No.~\r9.\ of 2nd or 3rd~c.

This contains \prop{1}{39} and \prop{1}{41} following one another and
almost complete, showing that \prop{1}{40} was wanting, whereas it is
found in all the \textsc{mss.}\ and is recognised by Proclus.
Moreover the text of the beginning of \prop{1}{39} is better than
ours, since it has no double \greek{διορισμός} but omits the first
(``I say that they are also in the same parallels ``) and has
``\emph{and}'' instead of ``\emph{for} let $AD$ be joined ``in the
next sentence.  It is clear that \prop{1}{40} was interpolated by
someone who thought there ought to be a proposition following
\prop{1}{39} and related to it as \prop{1}{38} is related to
\prop{1}{37} and \prop{1}{36} to \prop{1}{35}, although Euclid nowhere
uses \prop{1}{40}, and therefore was not likely to include it.  The
same interpolator failed to realise that the words ``let $AD$ be
joined'' were part of the \greek{ἔκθεσις} or \emph{setting-out}, and
took them for the \greek{κατασκευή} or ``construction'' which
generally follows the \greek{διορισμός} or ``particular statement ``
of the conclusion to be proved, and consequently thought it necessary
to insert a \greek{διορισμός} \emph{before} the words.

The conclusions drawn by Heiberg from a consideration of particular
readings in this papyrus along with those of our \textsc{mss.}\ will
be referred to below.

We now come to the principles which Heiberg followed, when preparing
his edition, in differentiating the original text from the Theonine
recension by means of a comparison of the readings of~P and of the
Theonine \textsc{mss.}  The rules which he gives are subject to a
certain number of exceptions (mostly in cases where one
\textsc{ms.}\ or the other shows readings due to copyists' errors),
but in general they may be relied upon to give conclusive results.

The possible alternatives which the comparison of~P with the Theonine
\textsc{mss.}\ may give in particular passages are as follows:

I.~There may be \emph{agreement} in three different degrees.

(1)~P and \emph{all} the Theonine \textsc{mss.}\ may agree.

In this case the reading common to all, even if it is corrupt or
interpolated, is more ancient than Theon, i.e.\ than the 4th~c.

(2) P may agree with \emph{some} (only) of the Theonine \textsc{mss.}

In this case Heiberg considered that the latter give the true reading
of Theon's recension, and the other Theonine \textsc{mss.}\ have
departed from it

(3) P and \emph{one} only of the Theonine \textsc{mss.}\ may agree.

In this case too Heiberg assumed that the \emph{one} Theonine
\textsc{ms.}\ which agrees with~P gives the true Theonine reading, and
that this rule even supplies a sort of measure of the quality and
faithfulness of the Theonine \textsc{mss.}  Now none of them agrees
alone with~P in preserving the true reading so often as~F. Hence F
must be held to have preserved Theon's recension more faithfully than
the other Theonine \textsc{mss.}; and it would follow that in those
portions where F fails us P must carry rather more weight even though
it may differ from the Theortine \textsc{mss.}\ BVpq. (Heiberg gives
many examples in proof of this, as of his main rules generally, for
which reference must be made to his \emph{Prolegomena} in Vol.~\r5.)
The specially close relation of F and P is also illustrated by
passages in which they have the same errors; the explanation of these
common errors (where not due to accident) is found by Heiberg in the
supposition that they existed, but were not noticed by Theon, in the
original copy in which he made his changes.

Although however F is by far the best of the Theonine \textsc{mss.},
there are a considerable number of passages where one of the others
(B, V, p or~q) \emph{alone} with~P gives the genuine reading of
Theon's recension.

As the result of the discovery of the papyrus fragment containing
\prop{1}{39, 41}, the principles above enunciated under (2) and~(3)
are found by Heiberg to require some qualification. For there is in
some cases a remarkable agreement between the papyrus and the Theonine
\textsc{mss.}\ (some or all) as against~P\@. This shows that Theon
took more trouble to follow older \textsc{mss.}, and made fewer
arbitrary changes of his own, than has hitherto been supposed. Next,
when the papyrus agrees with some of the Theonine
\textsc{mss.}\ against~P, it must now be held that these
\textsc{mss.}\ (and not, as formerly supposed, those which agree
with~P) give the true reading of Theon. If it were otherwise, the
agreement between the papyrus and the Theonine \textsc{mss.}\ would be
accidental: but it happens too often for this. It is clear also that
there must have been contamination between the two recensions;
otherwise, whence could the Theonine \textsc{mss.}\ which agree with~P
and not with the papyrus have got their readings? The influence of the
P class on the Theonine~F is especially marked.

II.~There may be \emph{disagreement} between P and all the Theonine
\textsc{mss.}

The following possibilities arise,

(1)~The Theonine \textsc{mss.}\ differ also among themselves.

In this case Heiberg considered that P nearly always has the true
reading, and the Theonine \textsc{mss.}\ have suffered interpolation
in different ways after Theon's time.

(2)~The Theonine \textsc{mss.}\ all combine against~P.

In this case the explanation was assumed by Heiberg to be one or other
of the following.
\begin{enumerate}
% TBD: lowercase greek labels

\item The common reading is due to an error which cannot be
imputed to Theon (though it may have escaped him when putting
together the archetype of his edition); such error may either have
arisen accidentally in all alike, or (more frequently) may be
referred to a common archetype of all the \textsc{mss.}

\item There may be an accidental error in~P; e.g.\ something has
  dropped out of~P in a good many places, generally through
  \greek{ὁμοιοτέλευτον}.

\item There may be words interpolated in~P.

\item Lastly, \emph{we may have in the Theonine \textsc{mss.}\ a
  change made by Theon himself}.
\end{enumerate}
(The discovery of the ancient papyrus showing readings agreeing with
some, or with all, of the Theonine \textsc{mss.}\ against~P now makes
tt necessary to be very cautious in applying these criteria.)

It is of course the last class~(\greek{δ}) of changes which we have to
investigate in order to get a proper idea of Theon's recension.

Heiberg first observes, as regards these, that we shall find that
Theon, in editing the \emph{Elements}, altered hardly anything without
some reason, often inadequate according to our ideas, but still some
reason which seemed to him sufficient.  Hence, in cases of very slight
differences where both the Theonine \textsc{mss.}\ and~P have readings
good and probable in themselves, Heiberg is not prepared to put the
differences down to Theon. In those passages where we cannot see the
least reason why Theon, if he had the reading of~P before him, should
have altered it, Heiberg would not at once assume the superiority of~P
unless there was such a consistency in the differences as would
indicate that they were due not to accident but to design. In the
absence of such indications, he thinks that the ordinary principles of
criticism should be followed and that proper weight should be attached
to the antiquity of the sources. And it cannot be denied that the
sources of the Theonine version are the more ancient. For not only is
the British Museum palimpsest~(L), which is intimately connected with
the rest of our \textsc{mss.}, at least two centuries older than~P,
but the other Theonine \textsc{mss.}\ are so nearly allied that they
must be held to have had a common archetype intermediate between them
and the actual edition of Theon; and, since they themselves are as old
as, or older than~P, their archetype must have been much
older. Heiberg gives (pp.~xlvi, xlvii) a list of passages where, for
this reason, he has followed the Theonine \textsc{mss.}\ in preference
to~P.

It has been mentioned above that the copyist of~P or rather of its
archetype wished to give an ancient recension. Therefore (apart from
clerical errors and interpolations) the first hand in~P may be relied
upon as giving a genuine reading even where a correction by the first
hand has been made at the same time. But in many places the first hand
has made corrections afterwards; on these occasions he must have used
new sources, e.g.\ when inserting the scholia to the first Book which
P alone has, and in a number of passages he has made additions from
Theonine \textsc{mss.}

We cannot make out any ``family tree'' for the different Theonine
\textsc{mss.}  Although they all proceeded from a common archetype
later than the edition of Theon itself, they cannot have been copied
one from the other; for, if they had been, how could it have come
about that in one place or other each of them agrees \emph{alone}
with~P in preserving the genuine reading?  Moreover the great variety
in their agreements and disagreements indicates that they have all
diverged to about the same extent from their archetype. As we have
seen that P contains corrections from the Theonine family, so they
show corrections from~P or other \textsc{mss.}\ of the same
family. Thus V has part of the lacuna in the \textsc{ms.}\ from which
it was copied filled up from a \textsc{ms.}\ similar to~P, and has
corrections apparently derived from the same; the copyist, however, in
correcting~V, also used another \textsc{ms.}\ to which he alludes in
the additions to \prop{9}{19} and 30 (and also on \prop{10}{23} Por.):
``in the book of the Ephesian (this) is not found,'' Who this Ephesian
of the 12th~c.\ was, we do not know.

We now come to the alterations made by Theon in his edition of the
\emph{Elements}.  I shall indicate \emph{classes} into which these
alterations may be divided but without details (except in cases where
they affect the mathematical content as distinct from form or language
pure and simple)\footnote{Exhaustive details under all the different
  heads are given by Heiberg (Vol.~\r5. pp.~lii–lxxv).}.

I.~\emph{Alterations made by Theon where he found, or thought he
  found, mistakes in the original.}

1. Real blots in the original which Theon saw and tried to remove.

(\emph{a}) Euclid has a porism (corollary) to \prop{6}{19}, the
enunciation of which speaks of similar and similarly described
\emph{figures} though the proposition itself refers only to triangles,
and therefore the porism should have come after \prop{6}{20}. Theon
substitutes \emph{triangle} for \emph{figure} and proves the more
general porism after~\prop{6}{20}.

(\emph{b}) In \prop{9}{19} there is a statement which is obviously
incorrect.  Theon saw this and altered the proof by reducing four
alternatives to two, with the result that it fails to correspond to
the enunciation even with Theon's substitution of ``if'' for ``when''
in the enunciation.

(\emph{c}) Theon omits a porism to \prop{9}{11}, although it is necessary for
the proof of the succeeding proposition, apparently because, owing to
an error in the text (\greek{κατὰ τὸν} corrected by Heiberg into
\greek{ἐπὶ τὸ}), he could not get out of it the right sense.

(\emph{d}) I should also put into this category a case which Heiberg
classifies among those in which Theon merely fancied that he found
mistakes, viz.\ the porism to \prop{5}{7} stating that, if four
magnitudes are proportional, they are proportional inversely. Theon
puts this after \prop{5}{4} with a proof, which however has no
necessary connexion with \prop{5}{4} but is obvious from the
definition of proportion,

(\emph{e}) I should also put under this head \prop{11}{1}, where
Euclid's argument to prove that two straight lines cannot have a
common segment is altered.

2.~Passages which seemed to Theon to contain blots, and which he
therefore set himself to correct, though more careful consideration
would have shown that Euclid's words are right or at least may be
excused and offer no difficulty to an intelligent reader. Under this
head come:

(\emph{a}) an alteration in \prop{3}{24}.

(\emph{b}) a perfectly unnecessary alteration, in \prop{6}{14}, of
``equiangular parallelograms'' into ``parallelograms having one angle
equal to one angle,'' where Theon followed the false analogy of
\prop{6}{15}.

(\emph{c}) an omission of words in \prop{5}{26}, owing to his having
been misled by a wrong figure.

(\emph{d}) an alteration of the order of \r11.\ Deff.~27,~28.

(\emph{e}) the substitution of ``parallelepipedal solid'' for ``cube''
in \prop{11}{38}, because Theon observed, correctly enough, that it
was true of the parallelepipedal solid in general as well as of the
cube, but failed to give weight to the fact that Euclid must have
given the particular case of the cube for the simple reason that that
was all he wanted for use in \prop{13}{17}.

(\emph{f}) the substitution of the letter \greek{Φ} for \greek{Ω} ($V$
for~$Z$ in my figure) because he saw that the perpendicular from
\greek{Κ} to \greek{ΒΦ} would fall on~\greek{Φ} itself, so that
\greek{Φ}, \greek{Ω} coincide. But, if the substitution is made, it
should be proved that \greek{Φ}, \greek{Ω} coincide. Euclid can hardly
have failed to notice the fact, but it may be that he deliberately
ignored it as unnecessary for his purpose, because he did not want to
lengthen his proposition by giving the proof.

II.~\emph{Emendations intended to improve the form or diction of Euclid.}

Some of these emendations of Theon affect passages of appreciable
length. Heiberg notes about ten such passages; the longest is in
Eucl.\ \prop{12}{4} where a whole page of Heiberg's text is affected
and Theon's version is put in the Appendix. The kind of alteration may
be illustrated by that in \prop{9}{15} where Euclid uses successively
the propositions \prop{7}{24, 25}, quoting the enunciation of the
former but not of the latter; Theon does exactly the reverse. In a few
of the cases here quoted by Heiberg, Theon shortened the original
somewhat.

But, as a rule, the emendations affect only a few words in each
sentence. Sometimes they are considerable enough to alter the
conformation of the sentence, sometimes they are trifling alterations
``more magistellorum ineptorum'' and unworthy of Theon.  Generally
speaking, they were prompted by a desire to change anything which was
out of the common in expression or in form, in order to reduce the
language to one and the same standard or norm. Thus Theon changed the
order of words, substituted one word for another where the latter was
used in a sense unusual with Euclid (e.g.\ \greek{ἐπειδήπερ},
``since,'' for \greek{ὅτι} in the sense of ``because''), or one
expression for another in like circumstances (e.g.\ where, finding
``that which was enjoined would be done'' in a \emph{theorem},
\prop{7}{31}, and deeming the phrase more appropriate to a
\emph{problem}, he substituted for it ``that which is sought would be
manifest''; probably also and for similar reasons he made certain
variations between the two expressions usual at the end of
propositions \greek{ὅπερ ἔδει δεῖξαι}, and \greek{ὅπερ ἔδει ποιῆσαι}
\emph{quod erat demonstrandum} and \emph{quod erat
  faciendum}). Sometimes his alterations show carelessness in the use
of technical terms, as when he uses \greek{ἅπτεσθαι} (to \emph{meet})
for \greek{ἐφάπτεσθαι} (to \emph{touch}) although the ancients
carefully distinguished the two words. The desire of keeping to a
standard phraseology also led Theon to omit or add words in a number
of cases, and also, sometimes, to change the lettering of figures.

But Theon seems, in editing the \emph{Elements}, to have bestowed the
most attention upon

III.~\emph{Additions designed to supplement or explain Euclid.}

First, he did not hesitate to interpolate whole propositions where he
thought there was room or use for them. We have already mentioned the
addition to \prop{6}{33} of the second part relating to
\emph{sectors}, for which Theon himself takes credit in his commentary
on Ptolemy.  Again, he interpolated the proposition commonly known as
\prop{7}{22} (\emph{ex aequo in proportione perturbata} for numbers,
corresponding to \prop{5}{23}), and perhaps also \prop{7}{20}, a
particular case of \prop{7}{19} as \prop{6}{17} is of \prop{6}{16}.
He added a second case to \prop{6}{27}, a porism to \prop{2}{4}, a
second porism to \prop{3}{16}, and a lemma after \prop{10}{12};
perhaps also the porism to \prop{5}{19} and the first porism to
\prop{6}{20}. He also inserted alternative proofs here and there,
e.g.\ in \prop{2}{4} (where the alternative differs little from the
original) and in \prop{7}{31}; perhaps also in \prop{10}{1, 6, and 9}.

Secondly, he sometimes repeats an argument where Euclid had said ``For
the same reason,'' adds specific references to points, straight lines
etc.\ in the figures in order to exclude the possibility of mistake
arising from Euclid's reference to them in general terms, or inserts
words to make the meaning of Euclid more plain,
e.g.\ \emph{componendo} and \emph{alternately}, where Euclid had left
them out. Sometimes he thought to increase by his additions the
mathematical precision of Euclid's language in enunciations or
elsewhere, sometimes to make smoother and clearer things which Euclid
had expressed with unusual brevity and harshness or carelessness, in
reliance on the intelligence of his readers.

Thirdly, he supplied intermediate steps where Euclid's argument seemed
too rapid and not easy enough to follow. The form of these additions
varies; they are sometimes placed as a definite intermediate step with
``therefore'' or ``so that,'' sometimes they are additions to the
statement of premisses, sometimes phrases introduced by ``since,''
``for'' and the like, after the inference.

Lastly, there is a very large class of additions of a word, or one or
two words, for the sake of clearness or consistency. Heiberg gives a
number of examples of the addition of such nouns as ``triangle,''
``square,'' ``rectangle,'' ``magnitude,'' ``number,'' ``point,''
``side,'' ``circle,'' ``straight line,'' ``area'' and the like, of
adjectives such as ``remaining,'' ``right,'' ``whole,''
``proportional,'' and of other parts of speech, even down to words
like ``is'' (\greek{ἐστί}) which is added 600 times, \greek{δή},
\greek{ἄρα}, \greek{μέν}, \greek{γάρ}, \greek{καί} and the like.

IV.~\emph{Omissions by Theon}

Heiberg remarks that, Theon's object having been, as above shown, to
amplify and explain Euclid, we should not naturally have expected to
find him doing much in the contrary process of compression, and it is
only owing to the recurrence of a certain sort of omissions so
frequently (especially in the first Books) as to exclude the
hypothesis of their being all due to chance that we are bound to
credit him, with alterations making for greater brevity. We have seen,
it is true, that he made omissions as well as additions for the
purpose of reducing the language to a certain standard form. But there
are also a good number of cases where in the enunciation of
propositions, and in the \emph{exposition} (the re-statement of them
with reference to the figure), he has left out words because,
apparently, he regarded Euclid's language as being \emph{too} careful
and precise.  Again, he is apparently responsible for the frequent
omission of the words \greek{ὅπερ ἔδει δεῖξαι} (or \greek{ποιῆσαι}),
\textsc{q.e.d.}\ (or \textsc{f.}), at the end of propositions. This is
often the case at the end of porisms, where, in omitting the words,
Theon seems to have deliberately departed from Euclid's practice. The
\textsc{ms.}~P seems to show clearly that, where Euclid put a porism
at the end of a proposition, he omitted the \textsc{q.e.d.}\ at the
end of the proposition but inserted it at the end of the porism, as if
he regarded the latter as being actually a part of the proposition
itself. As in the Theonine \textsc{mss.}\ the \textsc{q.e.d.}\ is
generally omitted, the omission would seem to have been due to Theon.
Sometimes in these cases the \textsc{q.e.d.}\ is interpolated at the
end of the proposition.

Heiberg summed up the discussion of Theon's edition by the remark that
Theon evidently took no pains to discover and restore from
\textsc{mss.}\ the actual words which Euclid had written, but aimed
much more at removing difficulties that might be felt by learners in
studying the book. His edition is therefore not to be compared with
the editions of the Alexandrine grammarians, but rather with the work
done by Eutocius in editing Apollonius and with an interpolated
recension of some of the works of Archimedes by a certain Byzantine,
Theon occupying a position midway between these two editors, being
superior to the latter in mathematical knowledge but behind Eutocius
in industry (these views now require to be somewhat modified, as above
stated). But however little Theon's object may be approved by those of
us who would rather know the \emph{ipsissima verba} of Euclid, there
is no doubt that his work was approved by his pupils at Alexandria for
whom it was written; and his edition was almost exclusively used by
later Greeks, with the result that the more ancient text is only
preserved to us in one \textsc{ms.}

As the result of the above investigation, we may feel satisfied that,
where P and the Theonine \textsc{mss.}\ agree, they give us (except in
a few accidental instances) Euclid as he was read by the Greeks of the
4th~c. But even at that time the text had been passed from hand to
hand through more than six centuries, so that it is certain that it
had already suffered changes, due partly to the fault of copyists and
partly to the interpolations of mathematicians.  Some errors of
copyists escaped Theon and were corrected in some \textsc{mss.}\ by
later hands. Others appear in all our \textsc{mss.}\ and, as they
cannot have arisen accidentally in all, we must put them down to a
common source more ancient than Theon. A somewhat serious instance is
to be found in \prop{3}{8}; and the use of \greek{ἁπτέσθω} for
\greek{ἐφαπτέσθω} in the sense of ``touch'' may also be mentioned, the
proper distinction between the words having been ignored as it was by
Theon also.  But there are a number of imperfections in the
ante-Theonine text which it would be unsafe to put down to the errors
of copyists, those namely where the good \textsc{mss.}\ agree and it
is not possible to see any motive that a copyist could have had for
altering a correct reading.  In these cases it is possible that the
imperfections are due to a certain degree of carelessness on the part
of Euclid himself; for it is not possible ``Euclidem ab omni naevo
vindicare,'' to use the words of Saccheri\footnote{\emph{Euclides ab
    omni naevo vindicatus}, Mediolani, 1733.}, and consequently Simson
is not right in attributing to Theon and other editors all the things
in Euclid to which mathematical objection can be taken. Thus, when
Euclid speaks of ``the ratio compounded of the sides'' for ``the ratio
compounded of the \emph{ratios of the} sides,'' there is no reason for
doubting that Euclid himself is responsible for the more slip-shod
expression.  Again, in the Books \r11.–\r13.\ relating to solid
geometry there are blots neither few nor altogether unimportant which
can only be attributed to Euclid himself\footnote{Cf.\ especially the
  assumption, without proof or definition, of the criterion for
  \emph{equal} solid angles, and the incomplete proof of
  \prop{12}{17}.}; and there is the less reason for hesitation in so
attributing them because solid geometry was then being treated in a
thoroughly systematic manner for the first time.  Sometimes the
conclusion (\greek{συμπέρασμα}) of a proposition does not correspond
exactly to the enunciation, often it is cut short with the words
\greek{καὶ τὰ ἑξῆς} ``and the rest'' (especially from
Book~\r10.\ onwards), and very often in Books \r8., \r9.\ it is
omitted.  Where all the \textsc{mss.}\ agree, there is no ground for
hesitating to attribute the abbreviation or omission to Euclid;
though, of course, where one or more \textsc{mss.} have the longer
form, it must be retained because this is one of the cases where a
copyist has a temptation to abbreviate.

Where the true reading is preserved in one of the Theonine
\textsc{mss.}\ alone, Heiberg attributes the wrong reading to a
mistake which arose before Theon's time, and the right reading of the
single \textsc{ms.}\ to a successful correction.

We now come to the most important question of the \emph{Interpolations
  introduced before Theon's time}.

I. Alternative proofs or additional cases.

It is not in itself probable that Euclid would have given two proofs
of the same proposition; and the doubt as to the genuineness of the
alternatives is increased when we consider the character of some of
them and the way in which they are introduced.  First of all, we have
those of \prop{6}{20} and \prop{12}{17} introduced by ``we shall prove
this otherwise \emph{more readily} (\greek{προχειρότερον})'' or that
of \prop{10}{90} ``it is possible to prove \emph{more shortly}
(\greek{συντομώτερον}).'' Now it is impossible to suppose that Euclid
would have given one proof as that definitely accepted by him and then
added another with the express comment that the latter has certain
advantages over the former. Had he considered the two proofs and come
to this conclusion, he would have inserted the latter in the received
text instead of the former. These alternative proofs must therefore
have been interpolated. The same argument applies to alternatives
introduced with the words ``or even this'' (\greek{ἢ καὶ οὕτως}), ``or
even otherwise'' (\greek{ἣ καὶ ἄλλως}). Under this head come the
alternatives for the last portions of \prop{3}{7, 8}; and Heiberg also
compares the alternatives for parts of \prop{3}{31} (that the angle in
a semicircle is a right angle) and \prop{13}{18}, and the alternative
proof of the lemma after \prop{10}{32}. The alternatives to
\prop{10}{105 and 106}, again, are condemned by the place in which
they occur, namely after an alternative proof to \prop{10}{115}.  The
above alternatives being all admitted to be spurious, suspicion must
necessarily attach to the few others which are in themselves
unobjectionable, Heiberg instances the alternative proofs to
\prop{3}{9}, \prop{3}{10}, \prop{6}{30}, \prop{6}{31} and
\prop{11}{22}, observing that it is quite comprehensible that any of
these might have occurred to a teacher or editor and seemed to him,
rightly or wrongly, to be better than the corresponding proofs in
Euclid.  Curiously enough, Simson adopted the alternatives to
\prop{3}{9, 10} in preference to the genuine proofs. Since Heiberg's
preface was written, his suspicion has been amply confirmed as regards
\prop{3}{10} by the commentary of an-Nairīzī (ed.\ Curtze) which shows
not only that this alternative is Heron's, but also that the
substantive proposition \prop{3}{12} in Euclid is also Heron's, having
been given by him to supplement \prop{3}{11} which must originally
have been enunciated of circles ``touching one another'' simply,
i.e.\ so as to include the case of external as well as internal
contact, though the proof covered the case of internal contact
only. ``Euclid, in the 11th proposition,'' says Heron, ``supposed two
circles touching one another internally and wrote the proposition on
this case, proving what it was required to prove in it.  \emph{But I
  will show how it is to be proved if the contact be
  external}\footnote{An-Nairīzī, ed.\ Curtze, p.~121.}.'' This
additional proposition of Heron's is by way of adding another case,
which brings us to that class of interpolation. It was the practice of
Euclid and the ancients to give only one case (generally the most
difficult one) and to leave the others to be investigated by the
reader for himself.  One interpolation of a second case (\prop{6}{27})
is due, as we have seen, to Theon.  The two extra cases of
\prop{11}{23} were manifestly interpolated before Theon's time, for
the preliminary distinction of three cases, ``(the centre) will either
be within the triangle $LMN$, or on one of the sides, or
outside. First let it be within,'' is a spurious addition (B and V
only). Similarly an unnecessary case is interpolated in \prop{3}{11}.

II. Lemmas,

Heiberg has unhesitatingly placed in his Appendix to \r3.\ certain
lemmas interpolated either by Theon (on \prop{10}{13}) or later
writers (on \prop{10}{27, 29, 31, 32, 33, 34}, where V only has the
lemmas).  But we are here concerned with the lemmas found in all the
\textsc{mss.}, which however are, for different reasons, necessarily
suspected. We will deal with the Book~\r10.\ lemmas last.

(1) There is an \emph{a priori} ground of objection to those lemmas
which come \emph{after} the propositions to which they relate and
prove properties used in those propositions; for, if genuine, they
would be a sign of faulty arrangement such as would not be likely in a
systematic work so carefully ordered as the \emph{Elements}. The lemma
to \prop{6}{22} is one of this class, and there is the further
objection to it that in \prop{6}{28} Euclid makes an assumption which
would equally require a lemma though none is found. The lemma after
\prop{12}{4} is open to the further objections that certain altitudes
are used but are not drawn in the figure (which is not in the manner
of Euclid), and that a peculiar expression ``parallelepipedal solids
\emph{described on} (\greek{ἀναγραφόμενα ἀπό}) \emph{prisms}'' betrays
a hand other than Euclid's. There is an objection on the score of
language to the lemma after \prop{13}{2}, The lemmas on \prop{11}{23},
\prop{13}{13}, \prop{13}{18}, besides coming after the propositions to
which they relate, are not very necessary in themselves and, as
regards the lemma to \prop{13}{13}, it is to be noticed that the
writer of a gloss in the proposition could not have had it, and the
words ``as will be proved afterwards'' in the text are rightly
suspected owing to differences between the \textsc{ms.}\ readings. The
lemma to \prop{12}{2} also, to which Simson raised objection, comes
\emph{after} the proposition; but, if it is rejected, the words ``as
was proved before'' used in \prop{12}{5 and 18}, and referring to this
lemma, must be struck out.

(2)~Reasons of substance are fatal to the lemma before \prop{10}{60},
which is really assumed in \prop{10}{44} and therefore should have
appeared there if anywhere, and to the lemma on \prop{10}{20}, which
tries to prove what is already stated in \r10.~Def.~4.

We now come to the remaining lemmas in Book~\r10., eleven in number,
which come \emph{before} the propositions to which they relate and
remove difficulties in the way of their demonstration. That before
\prop{10}{42} introduces a set of propositions with the words ``that
the said irrational straight lines are uniquely divided…we will prove
after premising the following lemma,'' and it is not possible to
suppose that these words are due to an interpolator; nor are there any
objections to the lemmas before \prop{10}{14, 17, 22, 33, 54}, except
perhaps that they are rather easy. The lemma before \prop{10}{10} and
\prop{10}{10} itself should probably be removed from the
\emph{Elements}; for \prop{10}{10} really uses the following
proposition \prop{10}{11}, which is moreover numbered~10 by the first
hand in~P, and the words in \prop{10}{10} referring to the lemma ``for
we learnt (how to do this)'' betray the interpolator. Heiberg gives
reason also for rejecting the lemmas before \prop{10}{19 and 24} with
the words ``in any of the aforesaid ways'' (omitted in the Theonine
\textsc{mss.})\ in the enunciations of \prop{10}{19, 24} and in the
\emph{exposition} of \prop{10}{20}. Lastly, the lemmas before
\prop{10}{29} may be genuine, though there is an addition to the
second of them which is spurious.

Heiberg includes under this heading of interpolated lemmas two which
purport to be substantive propositions, \prop{11}{38} and
\prop{13}{6}. These must be rejected as spurious for reasons which
will be found in detail in my notes on \prop{11}{37} and \prop{13}{6}
respectively.  The latter proposition is only quoted once (in
\prop{13}{17}); probably the words quoting it (with \greek{γραμμή}
instead of \greek{εὐθεῖα}) are themselves interpolated, and Euclid
thought the fact stated a sufficiently obvious inference from
\prop{13}{1}.

III. Porisms (or corollaries).

Most of the porisms in the text are both genuine and necessary; but
some are shown by differences in the \textsc{mss.}\ not to be so,
e.g.\ those to \prop{1}{15} (though Proclus has it), \prop{3}{31} and
\prop{6}{20} (Por.~2).  Sometimes parts of porisms are interpolated.
Such are the last few lines in the porisms to \prop{4}{5},
\prop{6}{8}; the latter addition is proved later by means of
\prop{6}{4, 8}, so that the writer of these proofs could not have had
the addition to \prop{6}{8}~Por.\ before him.  Lastly, interpolators
have added a sort of proof to some porisms, as though they were not
quite obvious enough; but to add a demonstration is inconsistent with
the idea of a porism, which, according to Proclus, is a by-product of
a proposition appearing without our seeking it

IV.~Scholia.

Several interpolated scholia betray themselves by their wording,
e.g.\ those given by Heiberg in the Appendix to Book~\r10.\ and
containing the words \greek{καλεῖ}, \greek{ἐκάλεσε} (``he calls'' or
``called''); these scholia were apparently written as marginal notes
before Theon's time, and, being adopted as such by Theon, found their
way into the text in~P and some of the Theonine \textsc{mss.}  The
same thing no doubt accounts for the interpolated analyses and
syntheses to \prop{13}{1–5}, as to which see my note on \prop{13}{1}.

V.~Interpolations in Book~\r10.

First comes the proposition ``\emph{Let it be proposed to us} to show
that in square figures the diameter is incommensurable in length with
the side,'' which, with a scholium after it, ends the tenth Book. The
form of the enunciation is suspicious enough and the proposition, the
proof of which is indicated by Aristotle and perhaps was Pythagorean,
is perfectly unnecessary when \prop{10}{9} has preceded. The scholium
ends with remarks about commensurable and incommensurable solids,
which are of course out of place before the Books on solids. The
scholiast on Book~\r10.\ alludes to this particular scholium as being
due to ``Theon and some others.'' But it is doubtless much more
ancient, and may, as Heiberg conjectures have been the beginning of
Apollonius' more advanced treatise on incommensurables. Not only is
everything in Book~\r10.\ after \prop{10}{115} interpolated, but
Heiberg doubts the genuineness even of \prop{10}{112–115}, on the
ground that \prop{10}{111} rounds off the theory of incommensurables as
we want it in the Books on solid geometry, while \prop{10}{112–115}
are not really connected with what precedes, nor wanted for the later
Books, but seem to form the starting-point of a new and more elaborate
theory of irrationals,

VI.~Other minor interpolations are found of the same character as
those above attributed to Theon. First there are two places
(\prop{11}{35} and \prop{11}{26}) where, after ``similarly we shall
prove'' and ``for the same reason,'' an actual proof is nevertheless
given.  Clearly the proofs are interpolated; and there are other
similar interpolations. There are also interpolations of intermediate
steps in proofs, unnecessary explanations and so on, as to which I
need not enter into details.

Lastly, following Heiberg's order, I come to

VII.~Interpolated definitions, axioms etc.

Apart from \r6.~Def.~5 (which may have been interpolated by Theon
although it is found written in the margin of~P by the first hand),
the definition of a segment of a circle in Book~\r1.\ is interpolated,
as is clear from the fact that it occurs in a more appropriate place
in Book~\r3.\ and Proclus omits it.  \prop{6}{Def.~2} (reciprocal
figures) is rightly condemned by Simson—perhaps it was taken from
Heron—and Heiberg would reject \prop{7}{Def.~10}, as to which see my
note on that definition. Lastly the double definition of a solid angle
(\prop{11}{Def.~11}) constitutes a difficulty. The use of the word
\greek{ἐπιφάνεια} suggests that the first definition may have been
older than Euclid, and he may have quoted it from older
\emph{elements}, especially as his own definition which follows only
includes solid angles contained by \emph{planes}, whereas the other
includes other sorts (cf.\ the words \greek{γραμμῶν},
\greek{γραμμαῖς}) which are also distinguished by Heron (Def.~22). If
the first definition had come last, it could have been rejected
without hesitation: but it is not so easy to reject the first part up
to and including ``otherwise'' (\greek{ἄλλως}).  No difficulty need be
felt about the definitions of ``oblong,'' ``rhombus,'' and
``rhomboid,'' which are not actually used in the \emph{Elements}; they
were no doubt taken from earlier \emph{elements} and given for the
sake of completeness.

As regards the axioms or, as they are called in the text, \emph{common
  notions} (\greek{κοιναὶ ἔννοιαι}), it is to be observed that Proclus
says\footnote{Proclus, pp.~194, 10 sqq.} that Apollonius tried to
prove ``the axioms,'' and he gives Apollonius' attempt to prove
Axiom~1.  This shows at all events that Apollonius had \emph{some} of
the axioms now appearing in the text. But how could Apollonius have
taken a controversial line against Euclid on the subject of axioms if
these axioms had not been Euclid's to his knowledge? And, if they had
been interpolated between Euclid's time and his own, how could
Apollonius, living so comparatively short a time after Euclid, have
been ignorant of the fact? Therefore \emph{some} of the axioms are
Euclid's (whether he called them \emph{common notions}, or
\emph{axioms}, as is perhaps more likely since Proclus calls them
axioms): and we need not hesitate to accept as genuine the first three
discussed by Proclus, viz.\ (1)~things equal to the same equal to one
another, (2)~if equals be added to equals, wholes equal, (3)~if equals
be subtracted from equals, remainders equal. The other two mentioned
by Proclus (whole greater than part, and congruent figures equal) are
more doubtful, since they are omitted by Heron, Martianus Capella, and
others. The axiom that ``two lines cannot enclose a space'' is however
clearly an interpolation due to the fact that \prop{1}{4} appeared to
require it. The others about equals added to unequals, doubles of the
same thing, and halves of the same thing are also interpolated; they
are connected with other interpolations, and Proclus clearly used some
source which did not contain them.

Euclid evidently limited his formal axioms to those, which seemed to
him most essential and of the widest application; for he not
unfrequently assumes other things as axiomatic, e.g.\ in \prop{7}{28}
that, if a number measures two numbers, it measures their difference.

The differences of reading appearing in Proclus suggest the question
of the comparative purity of the sources used by Proclus, Heron and
others, and of our text. The omission of the definition of a segment
in Book~\r1.\ and of the old gloss ``which is called the
circumference'' in \prop{1}{Def.~15} (also omitted by Heron, Taurus,
Sextus Empiricus and others) indicates that Proclus had better sources
than we have; and Heiberg gives other cases where Proclus omits words
which are in all our \textsc{mss.}\ and where Proclus' reading should
perhaps be preferred. But, except in these instances (where Proclus
may have drawn from some ancient source such as one of the older
commentaries), Proclus' \textsc{ms.}\ does not seem to have been among
the best.  Often it agrees with our worst \textsc{mss.}, sometimes it
agrees with~F where F alone has a certain reading in the text, so that
(e.g.\ in \prop{1}{15} Por.)\ the common reading of Proclus and F must
be rejected, thrice only does it agree with P alone, sometimes it
agrees with P and some Theonine \textsc{mss.}, and once it agrees with
the Theonine \textsc{mss.}\ against P and other sources.

Of the other external sources, those which are older than Theon
generally agree with our best \textsc{mss.}, e.g.\ Heron, allowing for
the difference in the plan of his definitions and the somewhat free
adaptation to his purpose of the Euclidean definitions in Books \r10.,
\r11.

Heiberg concludes that the \emph{Elements} were most spoiled by
interpolations about the 3rd~c., for Sextus Empiricus had a correct
text, while Iamblichus had an interpolated one; but doubtless the
purer text continued for a long time in circulation, as we conclude
from the fact that our \textsc{mss.}\ are free from interpolations
already found in Iamblichus' \textsc{ms.}

\begin{comment}

\chapter{The Scholia}

Heiberg has collected scholia, to the number of about 1500, in
Vol. V. of his edition of Euclid, and has also discussed and classified
them in a separate short treatise, in which he added a few others 1 .

These scholia cannot be regarded as doing much to facilitate the
reading of the Elements. As a rule, they contain only such observa-
tions as any intelligent reader could make for himself. Among the
few exceptions are XI. Nos. 33, 35 (where \prop{11}{22}, 23 are extended to
solid angles formed by any number of plane angles), xil. No. 85
(where an assumption tacitly made by Euclid in \prop{12}{17} is proved),
IX. Nos. 28, 29 (where the scholiast has pointed out the error in the
text of \prop{9}{19}).

Nor are they very rich in historical information; they cannot be
compared in this respect with Proclus' commentary on Book I. or
with those of Eutocius on Archimedes and Apollonius. But even
under this head they contain some things of interest, e.g.\ II. No. 1 1
explaining that the gnomon was invented by geometers for the sake of
brevity, and that its name was suggested by an incidental characteristic,
namely that ``from it the whole is known (yvaplTai), either of the
whole area or of the remainder, when it (the yva>ftmv) is either placed
round or taken away''; 11. No, 13, also on the gnomon; IV. No. 2
stating that Book IV. was the discovery of the Pythagoreans;
V. No. 1 attributing the content of Book v. to Eudoxus; x. No. 1 with
its allusion to the discovery of incommensurability by the Pytha-
goreans and to Apollonius' work on irrationals; x. No. 62 definitely
attributing \prop{10}{9} to Theaetetus; XIII. No. I about the ``Platonic'' figures,
which attributes the cube, the pyramid, and the dodecahedron to the
Pythagoreans, and the octahedron and icosahedron to Theaetetus.

Sometimes the scholia are useful in connexion with the settlement
of the text, (1) directly, e.g.\ III. No. 16 on the interpolation of the
word ``within'' (eWot) in the enunciation of 111, 6, and x. No. 1
alluding to the discussion by ``Theon and some others'' of irrational
``surfaces'' and ``solids,'' as well as ``lines,'' from which we may
(2) indirectly in that they sometimes throw light on the connexion
of certain \textsc{mss.}\ 

1 Heiberg, \emph{Om Scholierne til Euklids Elementer}, Kjøbenhavn,
  1888. The tract is written in Danish, but, fortunately for those who
  do not read Danish easily, the author has appended (pp.~70–78) a
  résumé in French.

conclude that the scholium at the end of Book x, is not genuine;

Lastly, they have their historical importance as enabling us to
judge of the state of mathematical science at the times when they
were written.

Before passing to the classification of the scholia, Heiberg remarks
that we must separate from them a number of additions in the nature
of scholia which are found in the text of our \textsc{mss.}\  but which can, in
one way or another, be proved to be spurious. As they are found
both in P and in the Theonine \textsc{mss.}, they must have been in the \textsc{mss.}\ 
anterior to Theon (4th c). But they are, in great part, only found in
the margin of P and the Theonine \textsc{mss.}\ ; in V they are half in the
text and half in the margin. This can hardly be explained except
on the supposition that these additions were originally (in the \textsc{mss.}\ 
before Theory) in the margin, and that Theon kept them there in his
edition, but that they afterwards found their way gradually into the
text of P as well as of the Theonine \textsc{mss.}, or were omitted altogether,
while particular \textsc{mss.}\  have in certain places preserved the old arrange-
ment Of such spurious additions Heiberg enumerates the following:
the axiom about equals subtracted from unequals, the last lines of the
porism to vi. 8, second porisms to v. 19 and to vi. 20, the porism
to hi. 31, vi. Def. 5, various additions in Book X., the analyses and
syntheses of XI 1 1. 1 — Si an  the proposition \prop{13}{6}.

The two first classes of scholia distinguished by Heiberg are
denoted by the convenient abbreviations ``Schol. Vat.'' and ``Schol.
Vind.''

I. Schol, Vat.

It is first necessary to set out the letters by which Heiberg
denotes certain collections of scholia.

P = Scholia in P written by the first hand.

B = Scholia in B by a hand of the same date as the \textsc{ms.}\  itself,
generally that of Aretha s.

F = Scholia in F by the first hand.

Vat = Scholia of the Vatican \textsc{ms.}\  204 of the 10th c, which has
these scholia on leaves 198 — 205 (the end is missing) as an independent
collection. It does not contain the text of the Elements.

V c = Scholia found on leaves 283 — 292 of V and written in the
same hand as that part of the \textsc{ms.}\  itself which begins at leaf 233.

Vat 192 = a Vatican \textsc{ms.}\  of the 14th c. which contains, after
(l) the Elements I. — XIII. (without scholia), (2) the Data with scholia,

(3) Marinus on the Data, the Schol. Vat as an independent collection
and in their entirety, beginning with 1. No. 88 and ending with xm.
No. 44.

The Schol. Vat., the most ancient and important collection of
scholia, comprise those which are found in PBF Vat. and, from VII, 12
to \prop{9}{15}, in PB Vat. only, since in that portion of the Elements
F was restored by a later hand without scholia; they also include 1,

No. 88 which only happens to be erased in F, and IX. Nos. 28, 29
which may be left out because F, here has a different text In F
and Vat. the collection ends with Book x.; but it must also include
Schol. FB of Books xi. — xill., since these are found along with Schol.
Vat. to Books I. — X. in several \textsc{mss.}\  (of which Vat. 192 is one) as a
separate collection. The Schol. Vat. to Books X. — XIII. are also
found in the collection V c (where, curiously enough, xill. Nos, 43, 44
are at the beginning). The Schol. Vat. accordingly include Schol.
PBV C Vat, 192, and doubtless also those which are found in two of
these sources. The total number of scholia classified by Heiberg as
Schol. Vat. is 138.

As regards the contents of Schol. Vat. Heiberg has the following
observations. The thirteen scholia to Book I. are extracts made
from Proclus by a writer thoroughly conversant with the subject,
and cleverly recast (with some additions). Their author does not
seem to have had the two lacunae which our text of Proclus has
(at the end of the note on 1. 36 and the beginning of the next note,
and at the beginning of the note on \prop{1}{43}), for the scholia I. Nos. 125
and 137 seem to fill the gaps appropriately, at least in part. In
some passages he had better readings than our \textsc{mss.}\  have. The rest
of Schol. Vat. (on Books II. — xill.) are essentially of the same
character as those on Book 1., containing prolegomena, remarks on
the object of the propositions, critical remarks on the text, converses,
lemmas; they are, in general, exact and true to tradition. The
reason of the resemblance between them and Proclus appears to be
due to the fact that they have their origin in the commentary of
Pappus, of which we know that Proclus also made use. In support
of the view that Pappus is the source, Heiberg places some of the
Schol, Vat. to Book X. side by side with passages from the com-
mentary of Pappus in the Arabic translation discovered by Woepcke 1;
he also refers to the striking confirmation afforded by the fact that
XII, No. 2 contains the solution of the problem of inscribing in a
given circle a polygon similar to a polygon inscribed in another circle,
which problem Eutocius says' that Pappus gave in his commentary
on the Elements.

But, on the other hand, Schol. Vat. contain some things which
cannot have come from Pappus, e.g.\ the allusion in X. No. 1 to Theon
and irrational surfaces and solids, Theon being later than Pappus;
in. No. 10 about porisms is more like Proclus' treatment of the
subject than Pappus', though one expression recalls that of Pappus
about forming (cTj/MtTifeo-ftit) the enunciations of porisms like those
of either theorems or problems.

The Schol, Vat. give us important indications as regards the
text of the EUmtntt as Pappus had it. In particular, they show that
he could not have had in his text certain of the lemmas in Book X.
For example, three of these are identical with what we find in Schol,

1 \emph{Om Scholierne til Euklids Elementer}, pp.~11, 12:
  cf.\ \emph{Euklid-Studien}, pp.~170, 171; Woepcke, \emph{Mémoires
  présent.\ à l'Acad.\ des Sciences}, 1856, \r14. p.~658~sqq.
  
* Archimedes, ed.\ Heiberg, \r3. p.~28, 19–22.

Vat (the lemma to \prop{10}{17} = Schol. X. No. 106, and the lemmas to
\prop{10}{54}, 60 come in Schol. X. No. 328); and it is not possible to suppose.
that these lemmas, if they were already in the text, would also be
given as scholia. Of these three lemmas, that before \prop{10}{60} has
already been condemned for other reasons; the other two, un-
objectionable in themselves, must be rejected on the ground now
stated. There were four others against which Heiberg found nothing
to urge when writing his prolegomena to Vol. v., viz.\ the lemmas
before \prop{10}{42}, \prop{10}{14}, \prop{10}{2}Z and \prop{10}{33}. Of these, the lemma to \prop{10}{22}
is not reconcilable with Schol, x. No. 161, which takes up the
assumption in the text of Eucl. \prop{10}{22} as if no lemma had gone before.
The lemma to \prop{10}{42}, which, on account of the words introducing it
(see p.~60 above), Heiberg at first hesitated to regard as an inter-
polation, is identical with Schol. X. No. 27a It is true that in
Schol. x. No. 269 we find the words ``this lemma has been proved
before (£p tow ipTrpoaStv), but it shall also be proved now for
convenience' sake (rov eroinov evexa,)'' and it is possible to suppose
that ``before ``may mean in Euclid's text before x. 42; but a proof
in that place would surely have been as ``convenient ``as could be
desired, and it is therefore more probable that the proof had been
given by Pappus in some earlier place. (It may be added that the
lemma to \prop{10}{14}, which is identical with the lemma to \prop{11}{23}, con-
demned on other grounds, is for that reason open to suspicion.)

Heiberg's conclusion is that all the lemmas are spurious, and that
most or alt of them have found their way into the text from Pappus'
commentary, though at a time anterior to Theon's edition, since
they are found in all our \textsc{mss.}\  This enables us to fix a date for these
interpolations, namely the first half of the 4th c.

Of course Pappus had not in his text the interpolations which,
from the fact of their appearing only in some of our \textsc{mss.}, are seen to
be later than those above-mentioned. Such are the lemmas which
are found in the text of V only after \prop{10}{29} and \prop{10}{31} respectively and
are given in Heiberg's Appendix to Book X. (numbered 10 and 11).
On the other hand it appears from Woepcke's tract 1 that Pappus
already had x, 115 in his text: though it does not follow from this
that the proposition is genuine but only that interpolations began
very early.

Theon interpolated a proposition (or lemma) between \prop{10}{12} and
\prop{10}{13} (No, S in Heiberg's Appendix). Schol. Vat. has the same
thing (X. No. 125). The writer of the scholia therefore did not find
this lemma in the text. Schol. Vat IX. Nos. 28, 29 show that neither
did he find in his text the alterations which Theon made in Eucl. IX.
19; the scholia in fact only agree with the text of P, not with Theon's.
This suggests that Schol. Vat. were written for use with a \textsc{ms.}\  of the
ante-Theonine recension such as F is. This probability is further
confirmed by a certain independence which P shows in several places
when compared with the Theon ine \textsc{mss.}\  Not only has P better
readings in some passages, but more substantial divergences; and,

1 Woepcke, \emph{op.\ cit.}\ p.~702.

in particular, the absence in P of three notes of a historical character
which are added, wholly or partly from Proclus, in the Theonine \textsc{mss.}\ 
attests an independent and more primitive point of view in P.

In view of the distinctive character of P, it is possible that some
of the scholia found in it in the first hand, but not in the other
sources of Schol. Vat., also belong to that collection; and several
circumstances confirm this. Schol. XIII. No. 45, found in P only,
which relates to a passage in Eucl. \prop{13}{13}, shows that certain words
in the text, though older than Theon, are interpolated; and, as the
scholium is itself older than Theon, is headed ``third lemma,'' and
follows a ``second lemma'' relating to a passage in the text im-
mediately preceding, which ``second lemma'' belongs to Schol. Vat.
and is taken from Pappus, the ``third'' in all probability came from
Pappus also. The same is true of Schol. XII. No. 72 and xm. No. 69,
which are respectively identical with the propositions vulgo \prop{11}{38}
(Heiberg, A pp.~to Book xi., No, 3) and XII \prop{1}{6}; for both of these
interpolations are older than Theon. Moreover most of the scholia
which P in the first hand alone has are of the same character as
Schol, Vat Thus VII. No. 7 and XIII. No. I introducing Books VII.
and xm. respectively are of the same historical character as several
of Schol Vat; that vil. No. 7 appears in the text of P at the
beginning of Book VII. constitutes no difficulty. There are a number
of converses, remarks on the relation of propositions to one another,
explanations such as XII. No. 89 in which it is remarked that <f>, fl
in Euclid's figure to xil. 17 {Z, V in my figure) are really the same
point but that this makes no difference in the proof. Two other
Schol. P on \prop{12}{17} are connected by their headings with XII. No. 72
mentioned above, xi. No. 10 (P) is only another form of xi.
No. 1 1 (B); and B often, alone with P, has preserved Schol. Vat
On the whole Heiberg considers some 40 scholia found in P alone to
belong to Schol. Vat.

The history of Schol. Vat. appears to have been, in its main
outlines, the following. They were put together after 500 A.D., since
they contain extracts from Proclus, to which we ought not to assign
a date too near to that of Proclus' work itself; and they must at least
be earlier than the latter half of the 9th c, in which B was written.
As there must evidently have been several intermediate links between
the archetype and B, we must assign them rather to the first half of
the period between the two dates, and it is not improbable that they
were a new product of the great development of mathematical studies
at the end of the 6th c. (Isidorus of Miletus). The author extracted
what he found of interest in the commentary of Proclus on Book I.
and in that of Pappus on the rest of the work, and put these extracts
in the margin of a \textsc{ms.}\  of the class of p.~As there are no scholia to
\prop{1}{1} — 22, the first leaves of the archetype or of one of the earliest
copies must have been lost at an early date, and it was from that
mutilated copy that partly P and partly a \textsc{ms.}\  of the Theonine class
were taken, the scholia being put in the margin in both. Then the
collection spread through the Theonine \textsc{mss.}, gradually losing some

scholia which could not be read or understood, or which were
accidentally or deliberately omitted. Next it was extracted from
one of these \textsc{mss.}\  and made into a separate work which has been
preserved, in part, in its entirety (Vat. 192 etc.) and, in part, divided
into sections, so that ihe scholia to Books X. — xni. were detached
(V c ). It had the same fate in the mss, which kept the original
arrangement (in the margin), and in consequence there are some \textsc{mss.}\ 
where the scholia to the stereometric Books are missing, those Books
having come to be less read in the period of decadence. It is from
one of these \textsc{mss.}\  that the collection was extracted as a separate work
such as we find it in Vat. ( roth c).

II. The second great division of the scholia is Schol. Vind.

This title is taken from the Viennese \textsc{ms.}\  (V), and the letters used
by Heiberg to indicate the sources here in question are as follows.

V* = scholia in V written by the same hand that copied the \textsc{ms.}\ 
itself from fol. 235 onward.

q = scholia of the Paris \textsc{ms.}\  2344 (q) written by the first hand.

1 = scholia of the Florence ms. Laurent xxvin, 2 written in the
13th — 14th c, mostly in the first hand, but partly in two later
hands.

V b = scholia in V written by the same hand as the first part
(leaves 1 — 1S3) of the \textsc{ms.}\  itself; V'' wrote his scholia after V''.

q 1 = scholia of the Paris \textsc{ms.}\  (q) found here and there in another
hand of early date.

Schol. Vind. include scholia found in V m q. 1 is nearly related to
q; and in fact the three Mss. which, so far as Euclid's text is con-
cerned, show no direct interdependence, are. as regards their scholia,
derived from one original. Heiberg proves this by reference to the
readings of the three in two passages (found in Schol. I, No. 109 and
X. No, 39 respectively). The common source must have contained,
besides the scholia found in the three \textsc{mss.}\  V a ql, those also which
are contained in two of them, for it is more unlikely that two of the
three should contain common interpolations than that a particular
scholium should drop out of one of them. Besides V'' and q, the
scholia V b and q 1 must equally be referred to Schol. Vind., since the
greater part of their scholia are found in 1. There is a lacuna in q
from Eucl. \prop{8}{25} to \prop{9}{14}, so that for this portion of the Elements
Schol. Vind. are represented by VI only, Heiberg gives about 450
numbers in all as belonging to this collection.

Schol. Vind. did not all come from one source; this is shown by
differences of substance, e.g.\ between X. Nos. 36 and 39, and by
differences of time of writing: e.g.\ vi. No. 52 refers at the beginning
to No. 55 with the words ``as the scholium has it'' and is therefore
later than that scholium; X. No. 247 is also later than x. No. 246.

The scholia to Book I. are here also extracts from Proclus, but
more copious and more verbatim than in Schol. Vat. The author
has not always understood Proclus; and he had a text as bad as
that of our \textsc{mss.}, with the same lacunae. The scholia to the other

Books are partly drawn (i) from Schol. Vat., the \textsc{mss.}\  representing
Schol. Vind, and Schol. Vat. in these cases showing nearly all possible
combinations; but there is no certain trace in Schol. Vind. of the
scholia peculiar to p.~The author used a copy of Schol. Vat. in the
form in which they were attached to the Theonine text; thus Schol.
Vind. correspond to BF Vat., where these diverge from P, and
especially closely to B. Besides Schol. Vat., the editors of Schol.
Vind. used {2) other old collections 0/ scholia of which we find traces
in B and F; Schol. Vind. have also some scholia common with b.
The scholia which Schoi. Vind. have in common with BF come from
two different sources, and were apparently afterwards introduced
into the other \textsc{mss.}\ ; one result of this is that several scholia are
reproduced twice.

But, besides the scholia derived from these sources, Schol. Vind.
contain a large number of others of late date, characterised by in-
correct language or by triviality of content (there are many examples
in numbers, citations of propositions used, absurd diroplai, and the
like). Unlike Schol. Vat, these scholia often quote words from Euclid
as a heading (in one case a heading is inserted in Schol. Vind. where
a scholium without the heading is quoted from Schol. Vat, see V.
No. 14). The explanations given often presuppose very little know-
ledge on the part of the reader and frequently contain obscurities
and gross errors.

Schol. Vind. were collected for use with a \textsc{ms.}\  of the Theonine
class; this follows from the fact that they contain a note on the
proposition vulgo \prop{7}{22} interpolated by Theon (given in Heiberg's
App.~to Vol. II. p.~430), Since the scholium to vn. 39 given in V and
p in the text after the title of Book VIII. quotes the proposition as
\prop{7}{39}, it follows that this scholium must have been written before
the interpolation of the two propositions vulgo \prop{7}{20}, 22; Schol.
Vind. contain (vn. No. 80) the first sentence of it, but without the
heading referring to \prop{7}{39}. Schol. VII. No. 97 quotes \prop{7}{33} as
\prop{7}{34}, so that the proposition vulgo vn. 22 may have stood in the
scholiast's text but not the later interpolation vulgo vn. 20 (later
because only found in B in the margin by the first hand). Of course
the scholiast had also the interpolations earlier than Theon.

For the date of the collection we have a lower limit in the date
(12th c.) of \textsc{mss.}\  in which the scholia appear. That it was not much
earlier than the 12th c. is indicated (1) by the poverty of its contents,
(2) by the quality of the ms. of Proclus which was used in the
compilation of it (the Munich \textsc{ms.}\  used by Friedlein with which the
scholiast's excerpts are essentially in agreement belongs to the I ith —
12th c), (3) by the fact that Schol. Vind. appear only in \textsc{mss.}\  of the
12th c. and no trace of them is found in our \textsc{mss.}\  belonging to
the 9th — 10th c. in which Schol. Vat. are found. The collection may
therefore probably be assigned to the 1 ith c. Perhaps it may be in
part due to Psellus who lived towards the end of that century: for in
a Florence \textsc{ms.}\  (Magliabecch. XI, 53 of the 15 th c.) containing a
mathematical compendium intended for use in the reading of Aristotle

the scholia i. Nos. 40 and 49 appear with the name of Psellus
attached.

Schoi. Vind. are not found without the admixture of foreign
elements in any of our three sources. In 1 there are only very few
such in the first hand. In q there are several new scholia in the first
hand, for the most part due to the copyist himself. The collection of
scholia on Book x. in q (Heiberg's q=) is also in the first hand; it is
not original, and it may perhaps be due to Psellus (Maglb. has some
definitions of Book x. with a heading ``scholia of... Michael Psellus
on the definitions of Euclid's 10th Element'' and Schol. X. No. 9),
whose name must have been attached to it in the common source of
Maglb. and q; to a great extent it consists of extracts from Schol.
Vind. taken from the same source as VI. The scholia q 1 (in an
ancient hand in q), confined to Book II., partly belong to Schol, Vind.
and partly correspond to b 1 (Bologna \textsc{ms.}\ ), q* and q b are in one hand
(Theodorus Antiochita), the nearest to the first hand of q; they are
doubtless due to an early possessor of the \textsc{ms.}\  of whom we know
nothing more.

V* has, besides Schol. Vind., a number of scholia which also appear
in other \textsc{mss.}, one in BFb, some others in P, and some in v (Codex
Vat. IO38, 13th c.); these scholia were taken from a source in which
many abbreviations were used, as they were often misunderstood by V 1 .
Other scholia in V'' which are not found in the older sources — some
appearing in V* alone— are also not original, as is proved by mistakes
or corruptions which they contain; some others may be due to the
copyist himself.

V b seldom has scholia common with the other older sources; for
the most part they either appear in V b alone or only in the later
sources as v or F* (later scholia in F), some being original, others not.
In Book X. V b has three series of numerical examples, ( 1 ) with Greek
numerals, (2) alternatives added later, also mostly with Greek numerals,
(3) with Arabic numerals. The last class were probably the work of
the copyist himself. These examples (cf.\ p.~74 below) show the facility
with which the Byzantines made calculations at the date of the \textsc{ms.}\ 
(12th c). They prove also that the use of the Arabic numerals (in the
East- Arabian form) was thoroughly established in the 1 2th c.; they
were actually known to the Byzantines a century earlier, since they
appear, in the first hand, in an Escurial \textsc{ms.}\  of the 1 1 th c.

Of collections in other hands in V distinguished by Heiberg (see
preface to Vol. v.), V 1 has very few scholia which are found in other
sources, the greater part being original; V ! , V s are the work of the
copyist himself; V* are so in part only, and contain several scholia
from Schol. Vat. and other sources. V* and V J are later than 13th
— 14th c, since they are not found in f (cod. Laurent XXVHl, 6) which
was copied from V and contains, besides V'' V b , the greater part of
V 1 and vi. No. 20 of V (in the text).

In P there are, besides P* (a quite late hand, probably one of the
old Scriptores Grace i at the Vatican), two late hands (P 1 ), one of
which has some new and independent scholia, while the other has

added the greater part of Schol. Vind., partly in the margin and
partly on pieces of leaves stitched on.

Our sources for Schol. Vat. also contain other elements. In P
there were introduced a certain number of extracts from Proclus, to
supplement Schol. Vat. to Book I.; they are all written with a
different ink from that used for the oldest part of the \textsc{ms.}, and the
text is inferior. There are additions in the other sources of Schol.
Vat. (F and B) which point to a common source for FB and which
are nearly all found in other mss., and, in particular, in Schol. Vind,,
which also used the same source; that they are not assignable to
Schol. Vat. results only from their not being found in Vat. Of other
additions in F, some are peculiar to F and some common to it and b;
but they are not original. F s (scholia in a later hand in F) contains
three original scholia; the rest come from V. B contains, besides
scholia common to it and F, b or other sources, several scholia which
seem to have been put together by Arethas, who wrote at least a part
of them with his own hand.

Heiberg has satis6ed himself, by a closer study of b, that the
scholia which he denotes by b, ji and b 1 are by one hand; they are
mostly to be found in other sources as well, though some are original.
By the same hand (Theodoras Cabasilas, 15th c.) are also the scholia
denoted by b'', B', b* and B ! . These scholia come in great part from
Schol. Vind., and in making these extracts Theodoras probably used
one of our sources, 1, mistakes in which often correspond to those of
Theodoras. To one scholium is attached the name of Demetrius (who
must be Demetrius Cydonius, a friend of Nicolaus Cabasilas, 14th c);
but it could not have been written by him, since it appears in B antl
Schol. Vind. Nor are all the scholia which bear the name of
Theodoras due to Theodoras himself, though some are so.

As B' (a late hand in B) contains several of the original scholia of
b*, B* must have used b itself as his source, and, as all the scholia in
B* are in b, the latter is also the source of the scholia in B 8 which are
found in other \textsc{mss.}\  B and b were therefore, in the 15th c, in the
hands of the same person; this explains, too, the fact that b in a late
hand has some scholia which can only come from B. We arrive then
at the conclusion that Theodoras Cabasilas, in the 15th c, owned both
the \textsc{mss.}\  B and b, and that he transferred to B scholia which he had
before written in b, either independently or after other sources, and
inversely transferred some scholia from B to b. Further, B' are
earlier than Theodoras Cabasilas, who certainly himself wrote B* as
well as b' and b 8 .

An author's name is also attached to the scholia VI. No. 6 and
X. No. 223, which are attributed to Maximus Planudes (end of 13th c)
along with scholia on \prop{1}{31}, x. 14 and \prop{10}{18} found in 1 in a quite late
hand and published on pp.~46, 47 of Heiberg's dissertation. These
seem to have been taken from lectures of Planudes on the Elements
by a pupil who used 1 as his copy.

There are also in 1 two other Byzantine scholia, written by a late
hand, and bearing the names Ioannes and Pediasimus respectively;

these must in like manner have been written by a pupil after lectures
of Ioannes Pediasimus (first half of 14th c), and this pupil must also
have used 1.

Before these scholia were edited by Heiberg, very few of them had
been published in the original Greek. The Basel editio princeps has a
few (v. No. 1, VI. Nos. 3, 4 and some in Book X.) which are taken,
some from the Paris \textsc{ms.}\  (Paris. Gr, 2343) used by Grynaeus, others
probably from the Venice \textsc{ms.}\  (Marc. 301) also used by him; one
published by Heiberg, not in his edition of Euclid but in his paper
on the scholia, may also be from Venet. 301, but appears also in
Paris. Gr. 2342. The scholia in the Basel edition passed into the
Oxford edition in the text, and were also given by August in the
Appendix to his Vol. II.

Several specimens of the two series of scholia (Vat. and Vind.)
were published by C. Wachsmuth {Rhein, Mus. xvm, p.~132 sqq.)
and by Knoche {Untersuchungen iiber die neu aufgefundenen Scholien
des Proklus, Herford, 1 865).

The scholia published in Latin were much more numerous. G.
Valla {De expetendis et fugiendis rebus, 1 501) reproduced apparently
some 200 of the scholia included in Heiberg's edition. Several of
these he obtained from two Modena \textsc{mss.}\  which at one time were
in his possession (Mutin. Ill B, 4 and II E, 9, both of the 15th c.);
but he must have used another source as well, containing extracts
from other series of scholia, notably Schol. Vind. with which he has
some 87 scholia in common. He has also several that are new.

Commandinus included in his translation under the title ``Scholia
antiqua ``the greater part of the Schol, Vat. which he certainly
obtained from a \textsc{ms.}\  of the class of Vat. 192; on the whole he
adhered closely to the Greek text. Besides these scholia Com-
mandinus has the scholia and lemmas which he found in the Basel
editio princeps, and also three other scholia not belonging to Schol.
Vat., as well as one new scholium (to Xii. 13) not included in
Heiberg's edition, which are distinguished by different type and were
doubtless taken from the Greek \textsc{ms.}\  used by him along with the
Basel edition.

In Conrad Dasypodius' Lexicon matkematicum published in 1573
there is (on fol. 42—44) ``Graecum scholion in definitiones Euclidis
libri quinti elementorum append ids loco propter pagellas vacantes
annexum.'' This contains four scholia, and part of two others,
published in Heiberg's edition, with some variations of readings, and
with some new matter added (for which see pp.~64 — 6 of Heiberg's
pamphlet). The source of these scholia is revealed to us by another
work of Dasypodius, haaci Monachi Scholia in Euclidis elementorum
geometriae sex prions tibros per C. Dasypodium in latinum sermonem
trans lata et in lucem edita (1579). This work contains, besides
excerpts from Proclus on Book I. (in part closely related to Schol.
Vind.), some 30 scholia included in Heiberg's edition, several new
scholia, and the above-mentioned scholia to the definitions of Book v.
published in Greek in 1573. After the scholia follow ``Isaaci Monachi

prolegomena in Euclidis Elementorum geometriae libros'' (two
definitions of geometry) and ``Varia miscellanea ad geometriae cogni-
tionem necessaria ab Isaaco Monacho collecta ``(mostly the same as
pp.~252, 24 — 272, 27 in the Varieu Collectiones included in Hultsch's
Heron); lastly, a note of Dasypodius to the reader says that these
scholia were taken ``ex clarissimi viri Joannis Sambuciantiquocodice
manu propria Isaaci Monachi scrip to.'' Isaak Monachus is doubtless
Isaak Argyrus, 14th c.; and Dasypodius used a \textsc{ms.}\  in which, besides
the passage in Hultsch's Variae CoIUctiotus, there were a number of
scholia marked in the margin with the name of Isaak (cf.\ those in b
under the name of Theodorus Cabasilas). Whether the new scholia
are original cannot be decided until they are published in Greek; but
it is not improbable that they are at all events independent arrange-
ments of older scholia. All but five of the others, and all but one of
the Greek scholia to Book V., are taken from Schol. Vat.; three of the
excepted ones are from Schol, Vind., and the other three seem to
come from F (where some words of them are illegible, but can be
supplied by means of Mut. Ill B, 4, which has chese three scholia and
generally shows a certain likeness to Isaak's scholia).

Dasypodius also published in 1564 the arithmetical commentary
of Barlaam the monk (14th c.) on Eucl. Book 11., which finds a place
in Appendix IV. to the Scholia in Heiberg's edition.

Hultsch has some remarks on the origin of the scholia 1 . He
observes that the scholia to Book I. contain a considerable portion
of Geminus' commentary on the definitions and are specially valuable
because they contain extracts from Geminus only, whereas Proclus,
though drawing mainly upon him, quotes from others as well. On the
postulates and axioms the scholia give more than is found in Proclus.
Hultsch conjectures that the scholium on Book V., No. 3, attributing
the discovery of the theorems to Eudoxus but their arrangement to
Euclid, represents the tradition going back to Geminus, and that the
scholium XIII., No. 1, has the same origin.

A word should be added about the numerical illustrations of
Euclid's propositions in the scholia to Book x. They contain a large
number of calculations with sexagesimal fractions'; the fractions go
as far as fourth-sixtieths (i/6o*). Numbers expressed in these fractions
are handled with skill and include some results of surprising accuracy*

1 Art.\ ``Eukleides'' in Pauly-Wissowa's \emph{Real-Encyclopädie}.

2 Hultsch has written upon these in \emph{Bibliotheca Mathematica},
  \r5\tsub{3}, 1904, pp.~225–233.

* Thus $\sqrt{(27)}$ is given (allowing for a slight correction by
  means of the context) as $5$ $11́$ $46''$ $10'''$, which gives for
  $\sqrt{3}$ the value $1$ $43́$ $55''$ $23'''$, being the same value
  as that given by Hipparchus in his Table of Chords, and correct to
  the seventh decimal place. Similarly $\sqrt{8}$ is given as $2$
  $49'$ $42''$ $20'''$ $10''''$, which is equivalent to $\sqrt{2} =
  1.41421335$.  Hultsch gives instances of the various operations,
  addition, subtraction, etc., carried out in these fractions, and
  shows how the extraction of the square root was
  effected. Cf.\ T.~L. Heath, \emph{History of Greek Mathematics},
  \r1., pp.~59–63.

\chapter{Euclid in Arabia}

We are told by Hajl Khalfa' that the Caliph al-MansQr (754-775)
sent a mission to the Byzantine Emperor as the result of which he
obtained from him a copy of Euclid among other Greek books, and
again that the Caliph al-Ma'mun (813-833) obtained manuscripts of
Euclid, among others, from the Byzantines. The version of the
Elements by al-Hajjaj b. Yusuf b. Matar is, if not the very first, at
least one of the first books translated from the Greek into Arabic'.
According to the Fikrist* it was translated by al-Hajjaj twice; the
first translation was known as ``Haruni'' (``for Harun''), the second
bore the name ``Ma'muni'' (``for al-Ma'mun'') and was the more trust-
worthy. Six Books of the second of these versions survive in a Leiden
\textsc{ms.}\  (Codex Leidensis 399, 1) now in part published by Besthorn
and Heiberg*. In the preface to this ms. it is stated that, in the reign
of Harun ar-Rashid (786-809), al-Hajjaj was commanded by Yahya
b. Khalid b. Barmak to translate the book into Arabic. Then, when
al-Ma'mun became Caliph, as he was devoted to learning, al-Hajjaj
saw that he would secure the favour of al-Ma'mun ``if he illustrated
and expounded this book and reduced it to smaller dimensions. He
accordingly left out the superfluities, filled up the gaps, corrected or
removed the errors, until he had gone through the book and reduced
it, when corrected and explained, to smaller dimensions, as in this
copy, but without altering the substance, for the use of men endowed
with ability and devoted to learning, the earlier edition being left in
the hands of readers.''

The Fikrist goes on to say that the work was next translated by
Ishaq b. Hunain, and that this translation was improved by Thabit b,
Qurra. This Abu Ya'qub Ishaq b. Hunain b. Ishaq al-Tbadi (d. 910)
was the son of the most famous of Arabic translators, Hunain b. Ishaq
al-'lbadi (809-873), a Christian and physician to the Caliph al-
Mutawakkil (847-861). There seems to be no doubt that Ishaq, who

1 \emph{Lexicon bibliogr.\ et encyclop.}\ ed.\ Flügel, \v3. pp.~91,
  92.

1 Klamroth, \emph{Zeitschrift der Deutschen Morgenländischen
  Gesellschaft}, \r35. p.~303.

3 \emph{Fihrist} (tr.\ Suter), p.~16.

4 \emph{Codex Leidensis} 399, 1. \emph{Euclidis Elementa ex
  interpretatione al-Hadschdschadschii cum commentariis al-Narizii},
  Hauniae, part \r1. i.\ 1893, part \r1. ii.\ 1897, part \r2. i\ 1900,
  part \r2. ii.\ 1905, part \r3. i.\ 1910.

must have known Greek as well as his father, made his translation
direct from the Greek. The revision must apparently have been the
subject of an arrangement between Ishaq and Thabit, as the latter
died in 901 or nine years before Ishaq, Thabit undoubtedly consulted
Greek \textsc{mss.}\  for the purposes of his revision. This is expressly stated
in a marginal note to a Hebrew version of the Elements, made from
Ishaq's, attributed to one of two scholars belonging to the same family,
viz.\ either to Moses b. Tibbon (about 1 244- 1 274) or to Jakob b. Machir
(who died soon after 1306) 1 . Moreover Thabit observes, on the pro-
position which he gives as ix. 31, that he had not found this proposition
and the one before it in the Greek but only in the Arabic; from which
statement Klamroth draws two conclusions, (1) that the Arabs had
already begun to interest themselves in the authenticity of the text
and (2) that Thabit did not alter the numbers of the propositions in
Ishaq's translation'. The Fihrist also says that Yuhanna al-Qass (i.e.
``the Priest ``) had seen in the Greek copy in his possession the pro-
position in Book I. which Thabit took credit for, and that this was
confirmed by Nazlf, the physician, to whom Yuhanna had shown it
This proposition may have been wanting in Ishaq, and Thabit may
have added it, but without claiming it as his own discovery*. As
a fact, t. 45 is missing in the translation by al-Hajjaj.

The original version of Ishaq without the improvements by Thabit
has probably not survived any more than the first of the two versions
by al-Hajjaj; the divergences between the MSS, are apparently due to
the voluntary or involuntary changes of copyists, the former class
varying according to the degree of mathematical knowledge possessed
by the copyists and the extent to which they were influenced by
considerations of practical utility for teaching purposes*. Two \textsc{mss.}\ 
of the Ishaq-Thabit version exist in the Bodleian Library (No. 279
belonging to the year 1238, and No. 280 written in 1260-1) 11; Books
I. — XIII. are in the Ishaq-Thabit version, the non- Euclidean Books
XIV., XV. in the translation of Qusta b. Luqa at-Ba'labakki (d. about
912). The first of these \textsc{mss.}\  (No. 279) is thafctX)) used by Klamroth
for the purpose of his paper on the Arabian Euclid. The other \textsc{ms.}\ 
used by Klamroth is (K) Kjobenhavn LXXXI, undated but probably
of the 13th c, containing Books v. — xv., Boiks V. — X. being in the
Ishaq-Thabit version, Books XI. — XIII. purporting to be in al-Hajjaj's
translation, and Books XIV, xv. in the version of Qusta b. Luqa. In
not a few propositions K and O show not the slightest difference, and,
even where the proofs show considerable differences, they are generally
such that, by a careful comparison, it is possible to reconstruct the
common archetype, so that it is fairly clear that we have in these cases,
not two recensions of one translation, but arbitrarily altered and

1 Steinschneider, \emph{Zeitschrift für Math.\ u.\ Physik}, \r31.,
  hist.-litt.\ Abtheilung, pp.~85, 86, 99.

2 Klamroth, p.~279.

3 Steinschneider, p.~88.

4 Klamroth, p.~306.

5 These \textsc{mss.} are described by Nicoll and Pusey,
  \emph{Catatogus cod.\ mss.\ orient,\ bibl.\ Bodleianae},
  pt.~\r2. 1835 (pp.~257—262).

shortened copies of one and the same recension 1 . The Bodleian \textsc{ms.}\ 
No. 280 contains a preface, translated by Nicoll, which cannot be by
Thibit himself because it mentions Avkenna (980-1037) and other
later authors. The \textsc{ms.}\  was written at Maraga in the year 1260-1 and
has in the margin readings and emendations from the edition of
Nasiraddln at-Tusi (shortly to be mentioned) who was living at Maraga
at the time, is it possible that at-Tusl himself is the author of the
preface*? Be this as it may, the preface is interesting because it
throws light on the liberties which the Arabians allowed themselves
to take with the text. After the observation that the book (in spite
of the labours of many editors) is not free from errors, obscurities,
redundancies, omissions etc., and is without certain definitions neces-
sary for the proofs, it goes on to say that the man has not yet been
found who could make it perfect, and next proceeds to explain
(1) that Avicenna ``cut out postulates and many Definitions'' and
attempted to clear up difficult and obscure passages, (2) that Abu'l
Wafa al-Buzjanl (939-99?) ``introduced unnecessary additions and
left out many things of great importance and entirely necessary,''
inasmuch as he was too long in various places in Book VI. and too
short in Book X. where he left out entirely the proofs of the apotomae,
while he made an unsuccessful attempt to emend \prop{12}{14}, (3) that Abu
Ja'far al-Khazin (d, between 961 and 971) arranged the postulates
excellently but ``disturbed the number and order of the propositions,
reduced several propositions to one ``etc.\ Next the preface describes
the editor's own claims' and then ends with the sentences, ``But we
have kept to the order of the books and propositions in the work itself
(i.e.\ Euclid's) except in the twelfth and thirteenth books. For we have
dealt in Book xni, with the (solid) bodies and in Book XII. with the
surfaces by themselves.''

After Thabit the Fihrist mentions Abu 'Uthman ad-Dimashql as
having translated some Books of the Elements including Book X. (It
is Abu 'Uthman's translation of Pappus' commentary on Book X.
which Woepcke discovered at Paris.) The Fikrist adds also that
``Nazif the physician told me that he had seen the tenth Book of
Euclid in Greek, that it had 40 propositions more than the version
in common circulation which had 109 propositions, and that he had
determined to translate it into Arabic.''

But the third form of the Arabian Euclid actually accessible to us
is the edition of AbQ Ja'far Muh. b. Muh. b. al-Hasan Nasiraddln
at-Tusi (whom we shall call at-Tusi for short), born at Tus ( m
Khurasan) in 1201 (d. 1274). This edition appeared in two forms, a
larger and a smaller. The larger is said to survive in Florence only
(Pal. 272 and 313, the latter \textsc{ms.}\  containing only six Books); this was
published at Rome in 1 594, and, remarkably enough, some copies of

1 Klamroth, pp.~306–8.

2 Steinschneider, p.~98. Heiberg has quoted the whole of this preface
  in the \emph{\ZMP}, \r29., hist.-litt.\ Abth.\ p.~16.

3 This seems to include a rearrangement of the contents of Books
  \r14., \r15. added to the \emph{Elements}.

this edition are to be found with 12 and some with 13 Books, some
with a Latin title and some without'. But the book was printed in
Arabic, so that Kastner remarks that he will say as much about it as
can be said about a book which one cannot read*. The shorter form,
which however, in most \textsc{mss.}, is in 15 Books, survives at Berlin, Munich,
Oxford, British Museum (974, 1334*, 1335), Paris (2465, 2466), India
Office, and Constantinople; it was printed at Constantinople in
1 80 1, and the first six Books at Calcutta in 1824*.

At-Tusi's work is however not a trattslatiott of Euclid's text, but a
re-written Euclid based on the older Arabic translations. In this
respect it seems to be like the Latin version of the Elements by
Campanus (Campano), which was first published by Erhard Ratdolt
at Venice in 1482 (the first printed edition of Euclid*). Campanus
(13th c.) was a mathematician, and it is likely enough that he allowed
himself the same liberty as at-TGst in reproducing Euclid. What-
ever may be the relation between Campanus' version and that of
Athelhard of Bath (about 1 1 20), and whether, as Curtze thinks*, they
both used one and the same Latin version of 10th — 1 ith c, or whether
Campanus used Athel hard's version in the same way as at- T fist used
those of his predecessors 7 , it is certain that both versions came from
an Arabian source, as is evident from the occurrence of Arabic words
in them*. Campanus' version is not of much service for the purpose
of forming a judgment on the relative authenticity of the Greek and
Arabian tradition; but it sometimes preserves traces of the purer
source, as when it omits Theon's addition to vi. 33'. A curious
circumstance is that, while Campanus' version agrees with at-Tusi's
in the number of the propositions in all the genuine Euclidean Books
except V. and IX., it agrees with At hd hard's in having 34 propositions
in Book V. (as against 25 in other versions), which confirms the view
that the two are not independent, and also leads, as Klamroth says,
to this dilemma: either the additions to Book V. are Athelhard's
own, or he used an Arabian Euclid which is not known to us''.
Heiberg also notes that Campanus' Books XIV., XV. show a certain
agreement with the preface to the Thabit-Ishiq version, in which the
author claims to have (1) given a method of inscribing spheres in the
five regular solids, (2) carried further the solution of the problem how
to inscribe any one of the solids in any other and (3) noted the cases
where this could not be done 1 .

1 Suter, \emph{Die Mathematiker und Astronomen der Araber},
  p.~151. The Latin title is \emph{Euclidis elementorum geometricorum
  libri tredecim. Ex traditione doctissimi Nasiridini Tusini nunc
  primum arabice impressi}. Romae in typographia Medicea
  \textsc{mdxciv}. Cum licentia superiorum.

2 Kästner, \emph{Geschichte der Mathematik}, \r1. p.~367.

3 Suter has a note that this \textsc{ms.} is very old, having been
  copied from the original in the author's lifetime.

4 Suter, p.~151.

5 Described by Kästner, \emph{Geschichte der Mathematik},
  \r1. pp.~289–299, and by Weissenbom, \emph{Die Ubersetzungen des
  Euklid durch Campano und Zamberti}, Halle a.\ S., 1882, pp.~1–7. See
  also \emph{infra}, Chapter~\r8, p.~97.

6 Sonderabdruck des \emph{Jahresberichtes über die Fortschritte der
  klassischen Alterhumswissenschaft vom Okt}. 1879–1882, Berlin, 1884.

7 Klamroth, p.~271.

8 Curtze, \emph{op.~cit.} p.~20; Heiberg, \emph{Euklid-Studien},
  p.~178.

9 Heiberg's Euclid, vol.~\r5. p.~ci.

10 Klamroth, pp.~273—4.

With a view to arriving at what may be called a common measure
of the Arabian tradition, it is necessary to compare, in the first place,
the numbers of propositions in the various Books. Haji Khalfa says
that al-Hajjaj's translation contained 468 propositions, and Thabit's
478; this is stated on the authority of at-TQsI, whose own edition
contained 468*. The fact that Thabit's version had 478 propositions
is confirmed by an index in the Bodleian \textsc{ms.}\  279 (called O by
Klamroth). A register at the beginning of the Codex Leiden sis 399, 1
which gives Ishaq's numbers (although the translation is that of
al-FIajjaj) apparently makes the total 479 propositions (the number in
Book XIV. being apparently 11, instead of the 10 of O 1 ). I subjoin a
table of relative numbers taken from Klamroth, to which I have added
the corresponding numbers in August's and Heiberg's editions of the
Greek text

\begin{table}[H]
The Arabian Euclid

The Greek Euclid

\end{table}

The numbers in the case of Heiberg include all propositions which
he has printed in the text; they include therefore xiii. 6 and in. 12
now to be regarded as spurious, and X. .1 12 — 115 which he brackets
as doubtful. He does not number the propositions in Books XIV., XV.,
but I conclude that the numbers in P reach at least 9 in xiv., and 9
in XV.

1 Heiberg, \emph{\ZMP}, \r29,, hist.-litt.\ Abtheilung, p.~21.

2 Klamroth, p.~274; Steinschneider, \emph{\ZMP}, \r31.,
  hist.-litt.\ Abth.\ p.~98.

3 Besthorn-Heiberg read ``11?'' as the number, Klamroth had read it
  as 21 (p.~273).

The Fihrist confirms the number 109 for Book X., from which
K lam roth concludes that Ishaq's version was considered as by far the
most authoritative.

In the text of O, Book IV. consists of 1 7 propositions and Book
XIV. of 12, differing in this respect from its own table of contents; IV.
15, 16 in O are really two proofs of the same proposition.

In al-Hajjaj's version Book I. consists of 47 propositions only, \prop{1}{45}
being omitted. It has also one proposition fewer in Book III., the
Heron ic proposition m. 12 being no doubt omitted.

In speaking of particular propositions, I shall use Heiberg's
numbering, except where otherwise stated.

The difference of 10 propositions between Tha bit- Ishaq and
at-TusT is accounted for thus:

(1) The three propositions vi. 12 and X, 28, 39 which both Ishaq
and the Greek text have are omitted in at-Tusi,

(2) Ishaq divides each of the propositions xm. 1 — 3 into two,
making six instead of three in at-Tusi and in the Greek.

{3) Ishaq has four propositions (numbered by him vm. 24, 2$,

\prop{9}{30}, 3 1) which are neither in the Greek Euclid nor in at-Tusi.
Apart from the above differences al-iiajjaj (so far as we know),

Ishaq and at-Tusi agree , but their Euclid shows many differences
from our Greek text. These differences we will classify as follows 1 .

1. Prepositions.

The Arabian Euclid omits \prop{7}{20}, 22 of Gregory's and August's
editions (Heiberg, App.~to VoL 11. pp.~428-32); vm. 16, 17; \prop{10}{7}, 8,
13, 16, 24, 112, 113, 114, besides a lemma vulgo \prop{10}{13}, the proposition

\prop{10}{1} 1 7 of Gregory's edition, and the scholium at the end of the Book
(see for these Heiberg's Appendix to Vol. III. pp.~382, 408 — 416);

\prop{11}{38} in Gregory and August (Heiberg, App.~to Vol. I v. p.~354);

\prop{12}{6}, 13, 14; (also all but the first third of Book xv.).

The Arabian Euclid makes \prop{3}{11}, 12 into one proposition, and
divides some propositions (\prop{10}{31}, 32; xi, 31, 34; xin. 1 — 3) into two
each.

The order is also changed in the Arabic to the following extent.
v. 12, 13 are interchanged and the order in Books Vi,, vn, IX. —

XIII. is:

VL 1—8, 13, II, 12, 9, IO, 14—17, 19, 20, 18, 21, 22, 24, 26, 23,
25. 27—30, 32, 31, 33.

VII. I — 20, 22, 21, 23 — 28, 31, 32, 29, 30, 33 — 39.

IX i — 13, 20, 14—19, 21 — 25, 27, 26, 28 — 36, with two new pro-
positions coming before prop.~30.

x. 1—6, 9—12. IS. 14. i7— 2 3. 26—28, 25, 29—30, 31, 32, 33—
in, us-

XL 1—30,31.32,34.33.35—39

xii. 1 — 5, 7, 9, 8, 10, 12, 11, 15, 16 — 18.
xm. 1—3, 5. 4, 6, 7, 12, 9, io, 8, 11, 13, is, 14. 16—18.

1 See Klamroth, pp.~275–6, 280, 282—4, 314–15, 326; Heibeig,
  vol.~\r5. pp.~xcvi, xcvii.

2. Definitions.

The Arabic omits the following definitions: iv. Deff. 3 — 7, VII.
Def. 9 (or io), xi. Deff, 5 — 7, 15, 17. 23, 25—28; but it has the
spurious definitions VI. Deff. 2, 5, and those of proportion and ordered
proportion in Book V. (Deff. 8, 19 August), and wrongly interchanges
v. Deff. 1 1, 12 and also vi. Deff. 3, 4.

The order of the definitions is also different in Book VII. where,
after Def. II, the order is 12, 14, 13, 15, 16, 19, 20, 17, 18, 21, 22, 23,
and in Book xi. where the order is 1, 2, 3,4, 8, io, 9, 13, 14, 16, 12, 21,
22, 18, 19, 20, 11, 24.

3. Lemmas and porisms.

All are omitted in the Arabic except the porisms to vi. 8, vin, 2,
\prop{10}{3}; but there are slight additions here and there, not found in the
Greek, e.g.\ in vm. 14, 15 (in K).

4. Alternative proofs.

These are all omitted in the Arabic, except that in X, 105, 106 they
are substituted for the genuine proofs; but one or two alternative
proofs are peculiar to the Arabic (vi. 32 and vm. 4, 6).

The analyses and syntheses to XIII. I — 5 are also omitted in the
Arabic.

K lam roth is inclined, on a consideration of all these differences, to
give preference to the Arabian tradition over the Greek (1) ``on
historical grounds,'' subject to the proviso that no Greek \textsc{ms.}\  as
ancient as the 8th c, is found to contradict his conclusions, which are
based generally (2) on the improbability that the Arabs would have
omitted so much if they had found it in their Greek \textsc{mss.}, it being clear
from the Fihrist that the Arabs had already shown an anxiety for a
pure text, and that the old translators were subjected in this matter to
the check of public criticism. Against the ``historical grounds,'' Heiberg
is able to bring a considerable amount of evidence 1 . First of all there
is the British Museum palimpsest (L) of the 7th or the beginning of
the 8th c. This has fragments of propositions in Book X. which are
omitted in the Arabic; the numbering of one proposition, which agrees
with the numbering in other Greek ms., is not comprehensible on
the assumption that eight preceding propositions were omitted in it,
as they are in the Arabic; and lastly, the readings in L are tolerably
like those of our \textsc{mss.}, and surprisingly tike those of B. It is also to
be noted that, although P dates from the 10th c. only, it contains,
according to all appearance, an ante-Theonine recension.

Moreover there is positive evidence against certain omissions by
the Arabians. At-Tusi omits \prop{6}{12}, but it is scarcely possible that,
if Eutocius had not had it, he would have quoted Vi. 23 by that
number*. This quotation of \prop{6}{23} by Eutocius also tells against
Ishaq who has the proposition as vi. 25. Again, Simplicius quotes VI.
10 by that number, whereas it is \prop{6}{13} in Ishaq; and Pappus quotes,
by number, \prop{13}{2} (Ishaq 3, 4), X.\prop{3}{4} (Ishaq 8), \prop{13}{16} (Ishaq 19).

1 Heiberg in \emph{\ZMP}, \r29., hist.-litt.\ Abth.\ p.~3~sqq.

2 Apollonius, ed.\ Heiberg, vol.~\r2. p.~218, 3—5.

On the other hand the contraction of III. II, 12 into one proposition
in the Arabic tells in favour of the Arabic.

Further, the omission of certain porisms in the Arabic cannot be
supported; for Pappus quotes the porism to \prop{13}{17} 1 , Proclus those
to \prop{2}{4}, in. 1, vii. 2\ and Simplicius that to \prop{4}{15}.

Lastly, some propositions omitted in the Arabic are required in
later propositions. Thus \prop{10}{13} is used in \prop{10}{18}, 22, 23, 26 etc.; \prop{10}{17}
is wanted in \prop{10}{18}, 26, 36; xn. 6, 13 are required for \prop{12}{1} 1 and XII.
15 respectively.

It must also be remembered that some of the things which were
properly omitted by the Arabians are omitted or marked as doubtful
in Greek \textsc{mss.}\  also, especially in P, and others are rightly suspected for
other reasons (e.g.\ a number of alternative proofs, lemmas, and porisms,
as well as the analyses and syntheses of XII \prop{1}{1}—5). On the other
hand, the Arabic has certain interpolations peculiar to our inferior
\textsc{mss.}\  (cf.\ the definition VI. Def. 2 and those of proportion and ordered
proportion),

Heiberg comes to the general conclusion that, not only is the
Arabic tradition not to be preferred offhand to that of the Greek mss.,
but it must be regarded as inferior in authority. It is a question
how far the differences shown in the Arabic are due to the use of
Greek \textsc{mss.}\  differing from those which have been most used as the
basis of our text, and how far to the arbitrary changes made by
the Arabians themselves. Changes of order and arbitrary omissions
could not surprise us, in view of the preface above quoted from the
Oxford \textsc{ms.}\  of Thabit-Ishaq, with its allusion to the many important
and necessary things left out by Abu '1 Wafa and to the author's,
own rearrangement of Books XII., xm. But there is evidence of
differences due to the use by the Arabs of other Greek Mss. Heiberg'
is able to show considerable resemblances between the Arabic text
and the Bologna \textsc{ms.}\  b in that part of the \textsc{ms.}\  where it diverges so
remarkably from our other \textsc{mss.}\  (see the short description of it above,
p.~49); in illustration he gives a comparison of the proofs of \prop{12}{7} in b
and in the Arabic respectively, and points to the omission in both of
the proposition given in Gregory's edition as XI 38, and to a remark-
able agreement between them as regards the order of the propositions
of Book XII. As above stated, the remarkable divergence of b only
affects Books xi. (at end) and XII.; and Book xm. in b shows none
of the transpositions and other peculiarities of the Arabic. There
are many differences between b and the Arabic, especially in the
definitions of Book XX, as well as in Book xm. It is therefore a
question whether the Arabians made arbitrary changes, or the Arabic
form is the more ancient, and b has been altered through contact
with other \textsc{mss.}\  Heiberg points out that the Arabians must be alone
responsible for their definition of a prism, which only covers a prism
with a triangular base. This could not have been Euclid's own, for
the word prism already has the wider meaning in Archimedes, and

1 Pappus, \r5.~p.~436, 5.

2 Proclus, pp.~303–4.

3 \emph{\ZMP}, \r29., hist.-litt.\ Abth.\ p.~6~sqq.

Euclid himself speaks of prisms with parallelograms and polygons
as bases (xi. 39; \prop{12}{10}). Moreover, a Greek would not have been
likely to leave out the definitions of the ``Platonic ``regular solids.

Heiberg considers that the Arabian translator had before him
a \textsc{ms.}\  which was related to b, but diverged still further from the rest
of our \textsc{mss.}\  He does not think that there is evidence of the existence of
a redaction of Books I. — X. similar to that of Books XI., XII. in b; for
K I am roth observes that it is the Books on solid geometry (XI. — XIIL)
which are more remarkable than the others for omissions and shorter
proofs, and it is a noteworthy coincidence that it is just in these
Books that we have a divergent text in b.

An advantage in the Arabic version is the omission of VII. Def. 10,
although, as Iambliehus had it, it may have been deliberately omitted
by the Arabic translator. Another advantage is the omission of the
analyses and syntheses of XI II. I — 5; but again these may have been
omitted purposely, as were evidently a number of porisms which
are really necessary.

One or two remarks may be added about the Arabic versions
as compared with one another, Al-Hajjaj's object seems to have
been less to give a faithful reflection of the original than to write
a useful and convenient mathematical text-book. One characteristic
of it is the careful references to earlier propositions when their results
are used. Such specific quotations of earlier propositions are rare in
Euclid; but in al-Hajjaj we find not only such phrases as ``by prop,
so and so,'' ``which was proved ``or ``which we showed how to do in
prop, so and so,'' but also still longer phrases. Sometimes he repeats
a construction, as in \prop{1}{44} where, instead of constructing ``the parallelo-
gram BEFG equal to the triangle C in the angle EBG which is equal
to the angle })'' and placing it in a certain position, he produces AB
to G, making BG equal to half DE (the base of the triangle CDE in
his figure), and on GB so constructs the parallelogram BHKG by
\prop{1}{42} that it is equal to the triangle CDE, and its angle GBH is equal
to the given angle.

Secondly, al-Hajjaj, in the arithmetical books, in the theory of
proportion, in the applications of the Pythagorean 1. 47, and generally
where possible, illustrates the proofs by numerical examples. It is
true, observes Klamroth, that these examples are not apparently
separated from the commentary of an-Nairtzi, and might not there-
fore have been due to al-Hajjaj himself; but the marginal notes to
the Hebrew translation in Municn \textsc{ms.}\  36 show that these additions
were in the copy of al-Hajjaj used by the translator, for they expressly
give these proofs in numbers as variants taken from al-Hajjaj 1 .

These characteristics, together with al-Hajjaj 's freer formulation
of the propositions and expansion of the proofs, constitute an in-
telligible reason why Ishaq should have undertaken a fresh translation
from the Greek. Klamroth calls Ishaq's version a model of a good
translation of a mathematical text; the introductory and transitional

1 Klamroth, p.~310; Steinschneider, pp.~85–6.

phrases are stereotyped and few in number, the technical terms are
simply and consistently rendered, and the less formal expressions
connect themselves as closely with the Greek as is consistent with
intelligibility and the character of the Arabic language. Only in
isolated cases does the formulation of definitions and enunciations
differ to any considerable extent from the original. In general, his
object seems to have been to get rid of difficulties and unevennesses
in the Greek text by neat devices, while at the same time giving a
faithful reproduction of it 1 .

There are curious points of contact between the versions of
al-Hajjaj and Thabit-Ishaq. For example, the definitions and
enunciations of propositions are often word for word the same.
Presumably this is owing to the fact that Ishaq found these de-
finitions and enunciations already established in the schools in his
time, where they would no doubt be leamt by heart, and refrained
from translating them afresh, merely adopting the older version with
some changes'. Secondly, there is remarkable agreement between
the Arabic versions as regards the figures, which show considerable
variations from the figures of the Greek text, especially as regards
the letters; this is also probably to be explained in the same way,
all the later translators having most likely borrowed al-Hajjaj's
adaptation of the Greek figures'. Lastly, it is remarkable that the
version of Books XI. — XIII. in the KjfSbenhavn \textsc{ms.}\  (K), purporting
to be by al-Hajjaj, is almost exactly the same as the Thabit-Ishaq
version of the same Books in O. Klamroth conjectures that Ishaq
may not have translated the Books on solid geometry at all, and that
Thabit took them from al-Hajjaj, only making some changes in order
to fit them to the translation of Ishaq'.

From the facts (l) that at-TusI's edition had the same number
of propositions (468) as al-Hajjaj's version, while Thabit- 1 shaq's had
478, and (2) that at-Tusl has the same careful references to earlier
propositions, Klamroth concludes that at-Tusi deliberately preferred
al-Hajjaj's version to that of Ishaq', Heiberg, however,, points out
(1) that at-Tusi left out \prop{6}{12} which, if we may judge by Klamroth's
silence, al-Hajjaj had, and (2) al-Hajjaj's version had one proposition
less in Books 1. and in. than at-Tusl has. Besides, in a passage quoted
by Hajl Khalfa' from at-Tusi, the latter says that ``he separated the
things which, in the approved editions, were taken from the archetype
from the things which had been added thereto,'' indicating that he
had compiled his edition from both the earlier translations'.

There were a large number of Arabian commentaries on, or
reproductions of, the Elements or portions thereof, which will be

1 Klamroth, p.~290, illustrates Isḥāq's method by his way of
  distinguishing \emph{ἐφαρμόζειν} (to be congruent with) and
  \emph{e)farmózesqai} (to be applied to), the confusion of which by
  translators was animadverted on by Savile.  Isḥāq avoided the
  confusion by using two entirely different words.

2 Klamroth, pp.~310–1.

3 \ibid~p.~287.

4 \ibid~pp.~304—5.

5 \ibid~p.~274.

6 Hājī Khalfa, \r1. p.~383.

7 Heiberg, \emph{\ZMP}, \r29., hist.-litt.\ Abth.\ pp.~2,~3.

found fully noticed by Steinschn eider 1 . I shall mention here the
commentators etc.\ referred to in the Fikrist, with a few others.

1. Abu '1 'Abbas al-Fadl b. Hatim an-Nairlzi (born at Nairiz,
died about 922) has already been mentioned'. His commentary
survives, as regards Books 1. — VI., in the Codex Leidensis 399, 1, now
edited, as to four Books, by Besthorn and Heiberg, and as regards
Books I. — x. in the Latin translation made by Gherard of Cremona
in -the 12th c. and now published by Curtze from a Cracow \textsc{ms.}\ * Its
importance lies mainly in the quotations from Heron and Simplicius.

2. Ahmad b. 'Umar al-KarablsI (date uncertain, probably 9th —
10th c), ``who was among the most distinguished geometers and
arithmeticians'.''

3. A 1-' Abbas b. Sa'ld al-Jauhan (fl. 830) was one of the astro-
nomical observers under al-Ma'mun, but devoted himself mostly to
geometry. He wrote a commentary to the whole of the Elements,
from the beginning to the end; also the ``Book of the propositions
which he added to the first book of Euclid*.''

4. Muh. >. 'Isa Abu 'Abdallah al-Mahan! (d. between 874 and
884) wrotej according to the Fikrist, (1) a commentary on Eucl.
Book v., (2) ``On proportion,'' (3) ``On the 26 propositions of the
first Book of Euclid which are proved without reductio ad absurdum*.''
The work ``On proportion ``survives and is probably identical with, or
part of, the commentary on Book \prop{5}{7} He also wrote, what is not
mentioned by the Fihrist, a commentary on Eucl. Book X., a fragment
of which survives in a Paris \textsc{ms.}\ ''

5. Abu Ja'far al- Khazin (i.e.\ ``the treasurer ``or ``librarian ``), one
of the first mathematicians and astronomers of his time, was born in
Khurasan and died between the years 961 and 971. The Fikrist
speaks of him as having written a commentary on the whole of the
Elements*, but only the commentary on the beginning of Book X.
survives (in Leiden, Berlin and Paris); therefore either the notes on
the rest of the Books have perished, or the Fihrist is in error 10 . The
latter would seem more probable, for, at the end of his commentary,
al-Khazin remarks that the rest had already been commented on by
Sulaiman b. 'Usma (Leiden \textsc{ms.}\ ) 11 or ``Oqba (Surer), to be mentioned
below. At-Khazin's method is criticised unfavourably in the preface
to the Oxford \textsc{ms.}\  quoted by Nicoll (see p.~77 above).

6. Abu '1 WaiS al-Buzjanl (940-997), one of the greatest
Arabian mathematicians, wrote a commentary on the Elements, but

1 Steinschneider, \emph{\ZMP}, \r31., hist.-litt.\ Abth.\ pp.~86~sqq.

2 Steinschneider, p.~86, \emph{Fihrist} (tr.\ Suter), pp.~16, 67;
  Suter, \emph{Die Mathematiker und Astronomen der Araber} (1900),
  p.~45.

3 \emph{Supplementum ad Euclidis opera omnia}, ed.\ Heiberg and Menge,
  Leipzig, 1899.
  
4 \emph{Fihrist}, pp.~16, 38; Steinschneider, p.~87; Suter, p.~65.

5 \emph{Fihrist}, pp.~16, 25; Steinschneider, p.~88. Suter, p.~12.

6 \emph{Fihrist}, pp.~16, 25, 58.

7 Suter, p.~26, note, quotes the Paris \textsc{ms.} 2467, 16°
  containing the work ``on proportion'' as the authority for this
  conjecture.

8 \text{ms.} 2457, 39° (cf.\ Woepcke in \emph{Mém.\ prés.\ à
  l'acad.\ des sciences}, \r14., 1856, p.~669).

9 \emph{Fihrist}, p.~17.

10 Suter, p.~58, note~b.

11 Steinschneider, p.~89.

did not complete it 1 . His method is also unfavourably regarded in
the same preface to the Oxford \textsc{ms.}\  28a According to Haji Khalfa, he
also wrote a book on geometrical constructions, in thirteen chapters.
Apparently a book answering to this description was compiled by a
gifted pupil from lectures by Abu '1 Wafa, and a Paris \textsc{ms.}\  (Anc. fonds
169) contains a Persian translation of this work, not that of Abu '1 Wafa
himself. An analysis of the work was given by Woepcke*, and some
particulars will be found in Cantor'. Abu '1 Wafa also wrote a
commentary on Diophantus, as well as a separate ``book of proofs
to the propositions which Diophantus used in his book and to what
he (Abu '1 Wafa) employed in his commentary*.''

7. Ibn Rahawaihi al-Arjanl also commented on Eucl. Book X.*

8. 'All b, Ahmad Abu '1-Qasim al- AntakI (d. 987) wrote a
commentary on the whole book 1; part of it seems to survive (from
the 5th Book onwards) at Oxford (Catal. \textsc{mss.}\  orient. \prop{2}{28} 1) 7 .

9. Sind b. 'AH Abu 't-Taiyib was a Jew who went over to
Islam in the time of al-Ma'mun, and was received among his astro-
nomical observers, whose head ht became* (about 830); he died after
864. He wrote a commentary on the whole of the Elements; ``Abu
'All saw nine books of it, and a part of the tenth 8 .'' His book ``On
the Apotomae and the Medials,'' mentioned by the Fihrist, may be
the same as, or part of, his commentary on Book x.

10. Abu Yusuf Ya'qQb b, Muh. ar-Razi ``wrote a commentary
on Book X., and that an excellent one, at the instance of Ibn al-
•Amid''.''

11. The Fihrist next mentions al-Kindi (Abu Yusuf Ya'qub b.
Ishaq b. as-Sabbah al-Kindi, d. about 873), as the author (1) of a
work * on the objects of Euclid's book,'' in which occurs the statement
that the Elements were originally written by Apollonius, the carpenter
(see above, p.~5 and note), (2) of a book ``on the improvement of
Euclid's work,'' and (3) of another ``on the improvement of the 14th
and 15th Books of Euclid.'' ``He was the most distinguished man
of his time, and stood alone in the knowledge of the old sciences
collectively; he was called ' the philosopher of the Arabians '; his
writings treat of the most different branches of knowledge, as logic,
philosophy, geometry, calculation, arithmetic, music, astronomy and
others''.'' Among the other geometrical works of al-Kindi mentioned
by the Fihrist 1 * are treatises on the closer investigation of the results
of Archimedes concerning the measure of the diameter of a circle in
terms of its circumference, on the construction of the figure of the two
mean proportionals, on the approximate determination of the chords

1 \emph{Fihrist}, p.~17.

2 Woepcke, \emph{Journal Asiatique}, Sér.\ v.\ T.\ v.\ pp.~218–256 and
  309–359.

3 \emph{Gesch.\ d.\ Math.} vol.~\r1\tsub{3}, pp.~743–6.

4 \emph{Fihrist}, p.~39; Suter, p.~71.

5 \emph{Fihrist}, p.~17; Suter, p.~17.

6 \emph{Fihrist}, p.~17.

7. Suter, p.~64.

8 \emph{Fihrist}, p.~17, 29; Suter, pp.~13, 14.

9 \emph{Fihrist}, p.~17.

10 \emph{Fihrist}, p.~17; Suter, p.~66.

11 \emph{Fihrist}, p.~17, 10–15.

12 The mere catalogue of al-Kindī's works on the various branches of
  science takes up four octavo pages (11–15) of Suter's translation of
  the \emph{Fihrist}.

I

of the circle, on the approximate determination of the chord (side) of
the nonagon, on the division of triangles and quadrilaterals and con-
structions for that purpose, on the manner of construction of a circle
which is equal to the surface of a given cylinder, on the division of
the circle, in three chapters etc.

12. The physician Nazif b. Yumn (or Yaman) al-Qass (``the
priest ``) is mentioned by the Fikrist as having seen a Greek copy
of Eucl. Book X. which had 40 more propositions than that which
was in general circulation (containing 109), and having determined
to translate it into Arabic 1 . Fragments of such a translation exist
at Paris, Nos. 18 and 34. of the \textsc{ms.}\  24s 7 (952, 2 Suppl. Arab, in
Woepcke's tract); No. 18 contains ``additions to some propositions
of the 10th Book, existing in the Greek language'.'' Nazif must have
died about 990*,

13. YGhanna b. Yusuf b. al-Harith b. al-Bitriq al-Qass (d. about
980) lectures' on the Elements and other geometrical books, made
translations from the Greek, and wrote a tract on the ``proof'' of the
case of two straight lines both meeting a third and making with it,
on one side, two angles together less than two right angles*. Nothing
of his appears to survive, except that a tract ``on rational and irrational
magnitudes,'' No. 48 in the Paris \textsc{ms.}\  just mentioned, is attributed
to him.

14. Abu Muh. al- Hasan b. 'Ubaidallah b. Sulatman b. Wahb
(d. 901) was a geometer of distinction, who wrote works under the
two distinct titles ``A commentary on the difficult parts of the work
of Euclid ``and ``The Book on Proportion'.'' Suter thinks that an-
other reading is possible in the case of the second title, and that it
may refer to the Euclidean work ``on the divisions (of figures)*.''

15. Qusta b. Luqi al-Ba'labakkl (d. about 912), a physician,
philosopher, astronomer, mathematician and translator, wrote ``on the
difficult passages of Euclid's book'' and ``on the solution of arith-
metical problems from the third book of Euclid 7 ``; also an ``intro-
duction to geometry,'' in the form of question and answer''.

16. Thabit b. Qurra (826-90 1), besides translating some parts
of Archimedes and Books V. — VII. of the Conks of Apollonius, and
revising Ishaq's translation of Euclid's Elements, also revised the trans-
lation of the Data by the same Ishaq and the book On divisions of

figures translated by an anonymous writer. We are told also
that he wrote the following works: (1) On the Premisses (Axioms,
Postulates etc.) of Euclid, (2) On the Propositions of Euclid, (3) On
the propositions and questions which arise when two straight lines
are cut by a third (or on the ``proof'' of Euclid's famous postulate).
The last tract is extant in the \textsc{ms.}\  discovered by Woepcke (Paris
2 457> 3*°)- He is also credited with ``an excellent work'' in the
shape of an ``Introduction to the Book of Euclid,'' a treatise on

1 \emph{Fihrist}, pp.~16, 17.

2 Woepcke, \emph{Mém.\ prés.\ à l'acad.\ des science}, \r14. pp.~666, 668.

3 Suter, p.~68.

4 \emph{Fihrist}, p.~38; Suter. p.~60.

5 \emph{Fihrist}, p.~26, and Suter's note, p.~60.

6 Suter, p.~211, note~23,

7 \emph{Fihrist}, p.~43,

8 \emph{Fihrist}, p.~43; Suter, p.~41.

Geometry dedicated to Ismail b, Bulbul, a Compendium of Geometry,
and a large number of other works for the titles of which reference
may be made to Suter, who also gives particulars as to which are
extant 1 .

17. Abu Sa'ld Sinan b. T habit b, Qurra, the son of the translator
of Euclid, followed in his father's footsteps as geometer, astronomer

and physician. He wrote an ``improvement of the book of on

the Elements of Geometry, in which he made various additions to the
original.'' It is natural to conjecture that Euclid is the name missing
in this description (by Ibn abl Usaibi'a); Casiri has the name Aqaton 1 .
The latest editor of the Ta'rikk al-Hukamd, however, makes the name
to be Iflaton (= Plato), and he refers to the statement by the Fikrist
and Ibn al-Qiftl attributing to Plato a work on the Elements of
Geometry translated by Qusta. It is just possible, therefore, that at
the time of Qusta the Arabs were acquainted with a book on the
Elements of Geometry translated from the Greek, which they attri-
buted to Plato*. Sinan died in 94.3.

18. Abu Sahl Wijan (or Waijan) b. Rustam al-Kuhi (ft 988),
born at Kuh in Tabaristan, a distinguished geometer and astronomer,
wrote, according to the Fikrist, a ``Book of the Elements'' after that
of Euclid*; the 1st and 2nd Books survive at Cairo, and a part of
the 3rd Book at Berlin (5922)'. He wrote also a number of other
geometrical works: Additions to the 2nd Book of Archimedes on
the Sphere and Cylinder (extant at Paris, at Leiden, and in the India
Office), On the finding of the side of a heptagon in a circle (India
Office and Cairo), On two mean proportionals (India Office), which
last may be only a part of the Additions to Archimedes' On the Sphere
and Cylinder, etc.

19. Abu Nasr Muh. b. Muh. b. Tarkhan b. Uzlag al-Farabl
(870-950) wrote a commentary on the difficulties of the introductory
matter to Books I. and V. s This appears £0 survive in the Hebrew
translation which is, with probability, attributed to Moses b. Tibbon'.

20. Abu 'All al-Hasan b. al- Hasan b, al-Haitham (about 965-
1039), known by the name Ibn al-Haitham or Abu 'AHal-Basri, was a
man of great powers and knowledge, and no one of his time approached
him in the field of mathematical science. He wrote several works on
Euclid the titles of which, as translated by Woepcke from Usaibi'a,
are as follows 5:

1. Commentary and abridgment of the Etements.

2. Collection of the Elements of Geometry and Arithmetic,
drawn from the treatises of Euclid and Apollonius.

3. Collection of the Elements of the Calculus deduced from
the principles laid down by Euclid in his Elements.

1 Suter, pp.~34–8.

2 \emph{Fihrist} (ed.\ Suter), p.~59, note 132; Suter, p.~52, note~b.

3 See Suter in \emph{Bibliotheca Mathematica} \r4\tsub{3}, 1903–4,
  pp.~296–7, review of Julius Lippert's \emph{Ibn al-Qiftī.  Ta'rīch
  al-ḥukama=}, Leipzig, 1903.

4 \emph{Fihrist}, p.~40.

5 Suter, p.~75,

6 Suter, p.~55. 

7 Steinschneider, p.~92.

9 Steinschneider, pp.~92–3.



4. Treatise on ``measure ``after the manner of Euclid's
Elements,

5. Memoir on the solution of the difficulties in Book I.

6. Memoir for the solution of a doubt about Euclid, relative
to Book v,

7. Memoir on the solution of a doubt about the stereometric
portion.

8. Memoir on the solution of a doubt about Book XII.

9. Memoir on the division of the two magnitudes mentioned
in X. r (the theorem of exhaustion).

10. Commentary on the definitions in the work of Euclid
(where Steinschneider thinks that some more general expression
should be substituted for ``definitions'').

The last-named work (which Suter calls a commentary on the
Postulates of Euclid) survives in an Oxford ms. (Catal. \textsc{mss.}\  orient.
\prop{1}{908}) and in Algiers (1446, i'').

A Leiden \textsc{ms.}\  (966) contains his Commentary ``on the difficult
places ``up to Book v. We do not know whether in this commentary,
which the author intended to form, with the commentary on the
Musadarat, a sort of complete commentary, he had collected the
separate memoirs on certain doubts and difficult passages mentioned
in the above list

A commentary on Book V. and following Books found in a
Bodleian \textsc{ms.}\  (Catal. II. p.~262) with the title ``Commentary on Euclid
and solution of his difficulties ``is attributed to b. Haitham; this might
be a continuation of the Leiden US.

The memoir on \prop{10}{1} appears to survive at St Petersburg, \textsc{ms.}\  de
l'lnstitut des langues orient. 192, 5° (Rosen, Catal. p.~125).

21. Ibn Sina, known as Avicenna (980-1037), wrote a Com-
pendium of Euclid, preserved in a Leiden \textsc{ms.}\  No. 1445, and forming
the geometrical portion of an encyclopaedic work embracing Logic,
Mathematics, Physics and Metaphysics 1 .

22. Ahmad b. al-Husain al-Ahwazt al-Katib wrote a com-:
mentary on Book X., a fragment of which (some 10 pages) is to be
found at Leiden (970), Berlin (5923) and Paris (2467, 1 8°)*.

25. Naslraddln at-TusT (1 201-1274) who, as we have seen,
brought out a Euclid in two forms, wrote:

1. A treatise on the postulates of Euclid (Paris, 2467, 5 ).

2. A treatise on the 5th postulate, perhaps only a part of
the foregoing (Berlin, 5942, Paris, 2467, 6°).

3. Principles of Geometry taken from Euclid, perhaps
identical with No. I above (Florence, Pal. 298).

4. 105 problems out of the Elements (Cairo). He also edited
the Data (Berlin, Florence, Oxford etc.)*,

24. Muh. b. Ashraf Shamsaddin as-Samarqandi (fl. 1276) wrote
``Fundamental Propositions, being elucidations of 35 selected proposi-

1 Steinschneider, p.~92; Suter, p.~89.

2 Suter, p.~57.

3 Suter, pp.~150–1.

tions of the first Books of Euclid,'' which are extant at Gotha (1496
and 1497), Oxford (Catal. \prop{1}{967}, 2% and Brit Mus. 1 .

25. Musa b. Muh. b. Mahmud, known as Qadizade ar-Rumi (i.e.
the son of the judge from Asia Minor), who died between 1436 and
1446, wrote a commentary on the ``Fundamental Propositions'' just
mentioned, of which many \textsc{mss.}\  are extant 1 . It contained biographical
statements about Euclid alluded to above (p.~5. note),

26. Abu Da'Gd Sulaiman b. 'Uqba, a contemporary of al-Khazin
(see above, No. 5), wrote a commentary on the second half of Book X.,
which is, at least partly, extant at Leiden (974) under the title ``On
the binomials and apotomae found in the loth Book of Euclid*.''

27. The Codex Leidensis 399, 1 containing al-yajjaj's transla-
tion of Books I. — VL is said to contain glosses to it by Sa id b Mas'ud
b. al-Qass, apparently identical with Abu Nasr Gars al-Na'ma, son of
the physician Mas'ud b. al-Qass al-Bagdadl, who lived in the time of
the last Caliph al-Musta'sim {d. 1258)'.

28. Abu Muhammad b. AbdalbaqT al-Bagdadl al-Faradl (d.
1141, at the age of over 70 years) is stated in the Ta'rikk al-Hukama
to have written an excellent commentary on Book X. of the Elements,
in which he gave numerical examples of the propositions'. This is
published in Curtze's edition of an-Nairīzī where it occupies pages
252 — 386''.

29. Yahya b. Muh b. 'Abdan b. 'Abdalwahid, known by the
name of Ibn al-Lubudi (1 210-1268), wrote a Compendium of Euclid,
and a short presentation of the postulates*.

30. Abu 'Abdallah Muh. b. Mu'adh al-Jayyanl wrote a com-
mentary on Eucl. Book V. which survives at Algiers (1446, 3°)'.

31. Abu Nasr Mansur b. 'All b. 'Iraq wrote, at the instance of
Muh. b. Ahmad Abu 'r-Raihan al-Blruni (973-1048), a tract ``on
a doubtful (difficult) passage in Eucl. Book XIII.'' (Berlin, 5925)*,

1 Suter, p.~157.

2 \ibid~~p.~175.

3 \ibid~p.~56.

4 \ibid~pp.~153–4, 227.

5 Gartz, p.~14; Steinschneider, pp.~94—5.

6 Suter in \emph{Bibliotheca Mathematica}, \r4\tsub{3}, 1903, pp.~25,
  295; Suter has also an article an its contents, \emph{Bibliotheca
  Mathematica}, \r7\tsub{3}, 1906–7, pp.~234–251.

* Steinschneider, p.~94; Suter, p.~146.

* Suter, \emph{Nachträge und Berichtigungen}, in \emph{Abhandlungen
  zur Gesch.\ der math.\ Wissenschaften}, \r14., 1902, p.~170,

* Suter, p.~81, and \emph{Nachträge}, p.~172.

\chapter{Principal Translations and Editions of the Elements}

Cicero is the first Latin author to mention Euclid 1; but it is not
likely that in Cicero's time Euclid had been translated into Latin or
was studied to any considerable extent by the Romans; for, as Cicero
says in another place ', while geometry was held in high honour
among the Greeks, so that nothing was more brilliant than their
mathematicians, the Romans limited its scope by having regard only
to its utility for measurements and calculations. How very little
theoretical geometry satisfied the Roman agrimensores is evidenced
by the work of Balbus de mensuris', where some of the definitions of
Eucl. Book i, are given. Again, the extracts from the Elements found
in the fragment attributed to Censorinus (fl. 238 A.D )• are confined to
the definitions, postulates, and common notions. But by degrees the
Elements passed even among the Romans into the curriculum of a
liberal education; for Martianus Capella speaks of the effect of the
enunciation of the proposition ``how to construct an equilateral
triangle on a given straight line ``among a company of philosophers,
who, recognising the first proposition of the Elements, straightway
break out into encomiums on Euclid''. But the Elements were then
(c. 470 A.D.) doubtless read in Greek; for what Martianus Capella
gives* was drawn from a Greek source, as is shown by the occurrence
of Greek words and by the wrong translation of I. def. 1 (``punctum
vero est cuius pars nihil est''). Martianus may, it is true, have
quoted, not from Euclid himself, but from Heron or some other ancient
source.

But it is clear from a certain palimpsest at Verona that some
scholar had already attempted to translate the Elements into Latin.
This palimpsest 7 has part of the ``Moral reflections on the Book of
Job ``by Pope Gregory the Great written in a hand of the 9th c. above
certain fragments which in the opinion of the best judges date from
the 4th c. Among these are fragments of Vergil and of Livy, as well
as a geometrical fragment which purports to be taken from the I4tn
and 1 5th Books of Euclid, As a matter of fact it is from Books XIL
and XIII. and is of the nature of a free rendering, or rather a new

1 \emph{De oratore} \r3. 132.

2 \emph{Tusc.}\ \r1.~5.

3 \emph{Gromatici veteres}, \r1. 97 sq.\ (ed.\ F. Blume, K. Lachmann
  and A. Rudorff.  Berlin, 1848, 1852).

4 Censorinus, ed.\ Hultsch, pp.~60–3.

5 Martianus Capella, \r6. 724.

6 \ibid~\r6, 708~sq.

7 Cf.\ Cantor, \r1\tsub{3}, p.~565.

arrangement, of Euclid with the propositions in different order 1 . The
mh. was evidently the translator's own copy, because some words are
struck out and replaced by synonyms. We do not know whether the
translator completed the translation of the whole, or in what relation
his version stood to our other sources.

Magnus Aurelius Cassiodorus (b. about 475 A.D.) in the geometrical
part of his encyclopaedia De artibus ac disciplines liberalium literarum
says that geometry was represented among the Greeks by Euclid,
Apollonius, Archimedes, and others, ``of whom Euclid was given us
translated into the Latin language by the same great man Boethius .'';
also in his collection of letters* is a letter from Theodoric to Boethius
containing the words, ``for in your translations ... Nicomachus the
arithmetician, and Euclid the geometer, are heard in the Ausonian
tongue.'' The so-called Geometry of Boethius which has come down
to us by no means constitutes a translation of Euclid. The \textsc{mss.}\ 
variously give five, four, three or two Books, but they represent only
two distinct compilations, one normally in five Books and the other
in two. Even the latter, which was edited by Friedlein, is not
genuine*, but appears to have been put together in the 11 th c, from
various sources. It begins with the definitions of Eucl. I., and in these
are traces of perfectly correct readings which are not found even in
the \textsc{mss.}\  of the 10th c, but which can be traced in Proclus and other
ancient sources; then come the Postulates (five only), the Axioms
(three only), and after these some definitions of Eucl. 1 1., ill., IV.
Next come the enunciations of Eucl. I., of ten propositions of Book II.,
and of some from Books HI., IV., but always without proofs; there
follows an extraordinary passage which indicates that the author will
now give something of his own in elucidation of Euclid, though what
follows is a literal translation of the proofs of Eucl. \prop{1}{1} — 3. This
latter passage, although it affords a strong argument against the
genuineness of this part of the work, shows that the Pseudoboethius
had a Latin translation of Euclid from which he extracted the three
propositions.

Curtze has reproduced, in the preface to his edition of the trans-
lation by Gherard of Cremona of an-Nairizt's Arabic commentary on
Euclid, some interesting fragments of a translation of Euclid taken
from a Munich \textsc{ms.}\  of the 10th c. They are on two leaves used
for the cover of the \textsc{ms.}\  (Bibliothecae Regiae Universitatis Monacensis
2° 757) and consist of portions of Eucl. \prop{1}{37}, 38 and \prop{2}{8}, translated
literally word for word from the Greek text. The translator seems to
have been an Italian (cf.\ the words ``capitolonono'' used for the ninth
prop, of Book II.) who knew very little Greek and had moreover little
mathematical knowledge. For example, he translates the capital letters
denoting points in figures as if they were numerals: thus to. A Br,

1 The fragment was deciphered by W. Studemund, who communicated his
  results to Cantor.

2 Cassiodorus, \emph{Variae}, \r1. 45, p.~40, 12 ed. Mommsen.

* See especially Weissenborn in \emph{Abhandlungen zur
  Gesch.\ d.\ Math.}\ \r2. p.~185 sq.; Heiberg in \emph{Philologus},
  \r43. p.~507 sq.; Cantor, \r1\tsub{3}, p.~580 sq.

AEZ is translated ``que primo secundo et tertio quarto quinto et
sept i mo,'' T becomes ``tricentissimo ``and so on. The Greek \textsc{ms.}\  which
he used was evidently written in uncials, for AEZ8 becomes in one
place ``quod autem septimo nono,'' showing that he mistook AE for
the particle Si, and xal 6 2TU is rendered ``sicut tricentissimo et
quadringentissimo,'' showing that the letters must have been written
KAIOCTU.

The date of the Englishman Athelhard (Ethelhard) is approxi-
mately fixed by some remarks in his work Perdifficites Quaestiones
Naturaks which, on the ground of the personal allusions they contain,
must be assigned to the first thirty years of the 12th c. 1 He wrote a
number of philosophical works. Little is known about his life. He
is said to have studied at Tours and Lao 11, and to have lectured at the
latter school. He travelled to Spain, Greece, Asia Minor and Egypt,
and acquired a knowledge of Arabic, which enabled him to translate
from the Arabic into Latin, among other works, the Elements of
Euclid. The date of this translation must be put at about 1120.
\textsc{mss.}\  purporting to contain Atheihard's version are extant in the
British Museum (Harleian No. 5404 and others), Oxford (Trin. Coll.
47 and Ball. Coll. 257 of 12th c), Niirnberg (Johannes Regiomontanus'
copy) and Erfurt.

Among the very numerous works of Gherard of Cremona (I 1 14 —
1 1 87) are mentioned translations of ``1 5 Books of Euclid ``and of the
Data*. Till recently this translation of the Elements was supposed to
be lost; but Axel Anthon Bjornbo has succeeded (1904) in discovering
a translation from the Arabic which is different from the two others
known to us (those by Athelhard and Campanus respectively), and
which he, on grounds apparently convincing, holds to be Gherard's.
Already in 190 1 Bjornbo had found Books X.— XV. of this translation
in a \textsc{ms.}\  at Rome (Codex Reginensis lat. 1268 of 14th c.)*; but three
years later he had traced three \textsc{mss.}\  containing the whole of the same
translation at Paris (Cod. Paris. 7216, 15th a), Baulogne-sur-Mer
(Cod. Bononiens. 196, 14th a), and Bruges (Cod, Brugens. 521, 14th c),
and another at Oxford (Cod. Digby [74, end of 12th c.) containing a
fragment, \prop{11}{2} to XIV. The occurrence of Greek words in this
translation such as rombus, romboides (where Athelhard keeps the
Arabic terms), ambligonius, ortAogonius, gnomo, fyramis etc., show
that the translation is independent of Atheihard's. Gherard appears
to have had before him an old translation of Euclid from the Greek
which Athelhard also often followed, especially in his terminology,
using it however in a very different manner. Again, there are some
Arabic terms, e.g.\ meguar for axis of rotation, which Athelhard did not
use, but which is found in almost all the translations that are with
certainty attributed to Gherard of Cremona; there occurs also the

1 Cantor, \emph{Gesch.\ d.\ Math.}\ \r1\tsub{3}, p.~906.

2 Boncompagni, \emph{Della vita e delle opere di Gherardo Cremonese},
  Rome, 1851, p.~5.

3 Described in an appendix to \emph{Studien über Menelaos' Sphärik}
  (\emph{Abhandlungen zur Geschichte der mathematischen
  Wissenschaften}, \r14., 1902).

4 See \emph{Bibliotheca Mathematica}, \r7\tsub{3}, 1905–6, pp.~242–8.

expression ``superficies equidistantium laterum et rcctorum angulorum,''
found also in Gherard's translation of an-Nairīzī, where Athelhard says
``parallelogrammum rectangulum.'' The translation is much clearer
than Athelhard's: it is neither abbreviated nor ``edited'' as A the! hard's
appears to have been; it is a word-for-word translation of an Arabic
\textsc{ms.}\  containing a revised and critical edition of Thabit's version. It
contains several notes quoted from Thabit himself ( Tftebit dixit), e.g.
about alternative proofs etc.\ which Thabit found ``in another Greek
\textsc{ms.},'' and is therefore a further testimony to Thabit's critical treatment
of the text after Greek \textsc{mss.}\  The new editor also added critical
remarks of his own, e.g.\ on other proofs which he found in other
Arabic versions, but not in the Greek: whence it is clear that he
compared the Thabit version before him with other versions as care-
fully as Thabit collated the Greek mss. Lastly, the new editor speaks
of ``Thebit qui transtulit hunc librum in arabicam linguam'' and of
``translatio Thebit,'' which may tend to confirm the statement of al-Qiftl
who credited Thabit with an independent translation, and not (as the
Fihrist does) with a mere improvement of the version of Ishaq b,
Hunain.

Gherard's translation of the Arabic commentary of an-Nairlzi on
the first ten Books of the Elements was discovered by Maximilian
Curtze in a \textsc{ms.}\  at Cracow and published as a supplementary volume
to Heiberg and Menge's Euclid 1: it will often be referred to in this
work.

Next in chronological order comes Johannes Campanus (Campano)
of Novara. He is mentioned by Roger Bacon (12 14-1 294) as a
prominent mathematician of his time 1 , and this indication of his date
is confirmed by the fact that he was chaplain to Pope Urban IV, who
was Pope from 1261 to 1281''. His most important achievement was
his 'edition of the Elements including the two Books XIV. and XV.
which are not Euclid's. The sources of Athelhard's and Campanus'
translations, and the relation between them, have been the subject of
much discussion, which does not seem to have led as yet to any
definite conclusion. Cantor (Hi, p.~91) gives references* and some
particulars. It appears that there is a Ms. at Munich (Cod. lat. Mon.
13021} written by Sigboto in the 12th c. at Priifning near Regensburg,
and denoted by Curtze by the letter R, which contains the enunciations
of part of Euclid. The Munich \textsc{mss.}\  of Athelhard and Campanus'
translations have many enunciations textually identical with those in
R, so that the source of all three must, for these enunciations, have

1 \emph{Anaritii in decem libros priores Elementorum Euclidis
  Commentarii ex interpretatione Gherardi Cremonensis in codice
  Cracoviensi 569 servata} edidit Maximilianus Curtze, Leipzig
  (Teubner), 1899.

2 Cantor, \r2\tsub{1}, p.~88.

3 Tiraboschi, \emph{Storia della letteratura italiana}, IV, \r4. 145–160.

4 H. Weissenborn in \emph{\ZMP}, \r25., Supplement, pp.~143–166, and
  in his monograph. \emph{Die Übersetzungen des Euklid durch Campano
  und Zamberti} (1882); Max.\ Curtze in \emph{Philologische Rundschau}
  (1881), \r1. pp.~943–950, and in \emph{Jahresbericht über die
  Fortschritte der classischen Alterthumswissenschaft}, \r40. (1884,
  \r3.)\ pp.~19–22; Heiberg in \emph{\ZMP}, \r35., hist.-litt.\ Abth.,
  pp.~48–58 and pp.~81–6.

been the same; in others Athelhard and Campanus diverge com-
pletely from R, which in these places follows the Greek text and is
therefore genuine and authoritative. In the 32nd definition occurs the
word ``elinuam,'' the Arabic term for ``rhombus,'' and throughout the
translation are a number of Arabic figures. But R was not translated
from the Arabic, as is shown by (among other things) its close
resemblance to the translation from Euclid given on pp.~377 sqq. of
the Gromatici Ve teres and to the so-called geometry of Boethius. The
explanation of the Arabic figures and the word ``elinuam ``in Def. 32
appears to be that R was a late copy of an earlier original with
corruptions introduced in many places; thus in Def. 32 a part of the
text was completely lost and was supplied by some intelligent copyist
who inserted the word ``elinuam,'' which was known to him, and also
the Arabic figures. Thus Athelhard certainly was not the first to
translate Euclid into Latin; there must have been in existence before
the nth c, a Latin translation which was the common source of R,
the passage in the Gromatici, and ``Boethius.'' As in the two latter
there occur the proofs as well as the enunciations of 1. 1 — 3, it is
possible that this translation originally contained the proofs also.
Athelhard must have had before him this translation of the
enunciations, as well as the Arabic source from which he obtained his
proofs. That some sort of translation, or at least fragments of one,
were available before Athelhard's time even in England is indicated
by some old English verses 1:

``The clerk Euclide on this wyse hit fonde
Thys craft of gemetry yn Egypte londe
Yn Egypte he tawghte hyt fill wyde,
In dyvers londe on every syde.
Mony erys after warde y understonde
Yer that the Craft com ynto thys londe.
Thys Craft Coin into England, as y yow say,
Yn tyme of good kyng Adelstone's day,''

which would put the introduction of Euclid into England as far back
as 924-940 A.D.

We now come to the relation between Athelhard and Campanus.
That their translations were not independent, as Weissenborn would
have us believe, is clear from the fact that in all \textsc{mss.}\  and editions,
apart from orthographical differences and such small differences as
are bound to arise when \textsc{mss.}\  are copied by persons with some
knowledge of the subject-matter, the definitions, postulates, axioms,
and the 364 enunciations are word for word identical in Athelhard
and Campanus; and this is the case not only where both have the
same text as R but where they diverge from it. Hence it would seem
that Campanus used Athelhard's translation and only developed the
proofs by means of another redaction of the Arabian Euclid. It is
true that the difference between the proofs of the propositions in the
two translations is considerable; Athelhard's are short and com-

1 Quoted by Halliwell in \emph{Rara Mathematica} (p.~56 note) from
  \textsc{ms.} Bib.\ Reg.\ Mus.\ Brit.\ 17 A.~\r1. f.~2\tsup{b}–3.

pressed, Campari us' clearer and more complete, following the Greek
text more closely, though still at some distance. Further, the
arrangement in the two is different; in Athelhard the proofs regularly
precede the enunciations, Campanus follows the usual order. It is a
question how far the differences in the proofs, and certain additions in
each, are due to the two translators themselves or go back to Arabic
originals. The latter supposition seems to Curtze and Cantor the
more probable one. Curtze's general view of the relation of Campanus
to Athelhard is to the effect that A thel hard's translation was gradually
altered, from the form in which it appears in the two Erfurt \textsc{mss.}\ 
described by Weissenborn, by successive copyists and commentators
who had Arabic originals before them, until it took the form which
Campanus gave it and in which it was published. In support of this
view Curtze refers to Regiomontanus' copy of the Athelhard-Campanus
translation. In Regiomontanus' own preface the title is given, and
this attributes the translation to Athelhard; but, while this copy
agrees almost exactly with Athelhard in Book I., yet, in places where
Campanus is more lengthy, it has similar additions, and in the later
Books, especially from Book HI. onwards, agrees absolutely with
Campanus; Regiomontanus, too, himself implies that, though the
translation was Athelhard's, Campanus had revised it; for he has
notes in the margin such as the following, ``Campani est hec,'' ``dubito
an demonstret hie Campanus ``etc.

We come now to the printed editions of the whole or of portions
of the Elements, This is not the place for a complete bibliography,
such as Riccardi has attempted in his valuable memoir issued in five
parts between 1 88/ and 1893, which makes a large book in itself 1 .
I shall confine myself to saying something of the most noteworthy
translations and editions. It will be convenient to give first the Latin
translations which preceded the publication of the editio prineeps of
the Greek text in 1533, next the most important editions of the Greek
text itself, and after them the most important translations arranged
according to date of first appearance and languages, first the Latin
translations after 1533, then the Italian, German, French and English
translations in order.

It may be added here that the first allusion, in the West, to the
Greek text as still extant is found in Boccaccio's commentary on the
Divina Commedia of Dante*. Next Johannes Regiomontanus, who
intended to publish the Elements after the version of Campanus, but
with the latter's mistakes corrected, saw in Italy (doubtless when
staying with his friend Bessarion) some Greek \textsc{mss.}\  and noticed how
far they differed from the Latin version (see a letter of his written in
the year 1471 to Christian Roder of Hamburg)*.

1 \emph{Saggio di una Bibliografia Euclidea}, memoria del
  Prof.\ Pietro Riccardi (Bologna, 1887, 1888, 1890, 1893).

2 \r1. p.~404.

3 Published in C.~T. de Murr's \emph{Memorabilia Bibliothecarum
  Norimbergensium}, Part~\r1. p.~190~sqq.

I. Latin translations prior to 1533.

1482. In this year appeared the first printed edition of Euclid,
which was also the first printed mathematical book of any import-
ance. This was printed at Venice by Erhard Ratdolt and contained
Campanus' translation 1 , Ratdolt belonged to a family of artists at
Augsburg, where he was born about 1443. Having learnt the trade
of printing at home, he went in 1475 to Venice, and founded there a
famous printing house which he managed for 1 1 years, after which he
returned to Augsburg and continued to print important books until
1516. He is said to have died in 1528. Kastner 1 gives a short
description of this first edition of Euclid and quotes the dedication to
Prince Mocenigo of Venice which occupies the page opposite to the
first page of text. The book has a margin of 2i inches, and in this
margin are placed the figures of the propositions. Ratdolt says in
his dedication that at that time, although books by ancient and
modern authors were printed every day in Venice, little or nothing
mathematical had appeared: a fact which he puts down to the diffi-
culty involved by the figures, which no one had up to that time
succeeded in printing. He adds that after much labour he had
discovered a method by which figures could be produced as easily as
letters*. Experts are in doubt as to the nature of Ratdolt's discovery.
Was it a method of making figures up out of separate parts of figures,
straight or curved lines, put together as letters are put together to
make words? In a life of Joh. Gottlob Immanuel Breitkopf, a con-
temporary of Kastner's own, this member of the great house of
Breitkopf is credited with this particular discovery. Experts in that
same house expressed the opinion that Ratdolt's figures were wood-
cuts, while the letters denoting points in the figures were like the
other letters in the text; yet it was with carved wooden blocks that
printing began. If Ratdolt was the first to print geometrical figures,
it was not long before an emulator arose; for in the very same year
Mattheus Cordonis of Windischgratz employed woodcut mathematical
figures in printing Oresme's De latitudintbus*. How eagerly the
opportunity of spreading geometrical knowledge was seized upon is
proved by the number of editions which followed in the next few
years. Even the year 1482 saw two forms of the book, though they
only differ in the first sheet Another edition came out in i486
(Vlmae, apud lo. Regeruni) and another in 149 1 {Vincentiae per

1 Curtze (An-Nairīzī, p.~xiii) reproduces the heading of the first
  page of the text as follows (there is no title-page)\?[old enlish]:
  Preclarissimū opus elemento\? Euclidis megarēsis \=vna cū cōmentis
  Campani pspicacissimi in artē geometriā incipit felicit', after
  which the definitions begin at once.  Other copies have the shorter
  heading: Preclarissimus liber elementorum Euclidis perspicacissimi:
  in artem Geometrie incipit quam foelicissime.  At the end stands the
  following: \?[XXX] Opus elementorū euclidis megarensis in geometriā
  artē Jn id quoqz\?[eth] Campani pspicacissimi Cōmentationes finiūt.
  Erhardus ratdolt Augustensis impressor solertissimus . venetijs
  impressit . Anno salutis . M.cccc.lxxxij . Octauis . Caleñ . Juñ
  . Lector . Vale.

2 Kästner, \emph{Geschichte der Mathematik}, \r1. p.~289 sqq. See also
  Weissenborn, \emph{Die Übersetzungen des Euklid durch CAmpano und
  Zamberti}, pp.~1–7.

3 ``Mea industria non sine maximo labore effect vt qua facilitate
  litterarum elementa imprimuntur ea etiam geometrice figure
  conficerentur. ``

4 Curtze in \emph{\ZMP}, \r20., hist.-litt.\ Abth.\ p.~58.

Leonardum de Basika et Gttlielmum de Papia), but without the dedi-
cation to Mocenigo who had died in the meantime (1485). If Cam-
pan us added anything of his own, his additions are at all events not
distinguished by any difference of type or otherwise; the enunciations
are in large type, and the rest is printed continuously in smaller type.
There are no superscriptions to particular passages such as Euclides
ex Campano, Campanus, Campani additio, or Campani annotatio, which
are found for the first time in the Paris edition of 15 16 giving
both Campanus' version and that of Zamberti (presently to be men-
tioned).

1 50 1. G. Valla included in his encyclopaedic work De expetendis
et fugiendis rebus published in this year at Venice (in aedibus Aldi
Romani) a number of propositions with proofs and scholia translated
from a Greek \textsc{ms.}\  which was once in his possession (cod, Mutin. Ill
B, 4 of the 15th c.).

1505. In this year Bartolomeo Zamberti (Zambertus) brought out
at Venice the first translation, from the Greek text, of the whole of the
Elements. From the title 1 , as well as from his prefaces to the Catoptrica
and Data, with their allusions to previous translators ``who take some
things out of authors, omit some, and change some,'' or ``to that most
barbarous translator ``who filled a volume purporting to be Euclid's
``with extraordinary scarecrows, nightmares and phantasies,'' one object
of Zamberti's translation is clear. His animus against Campanus
appears also in a number of notes, e.g.\ when he condemns the terms
``helmuain'' and ``helmuariphe ``used by Campanus as barbarous,
un- Latin etc., and when he is roused to wrath by Campanus' unfortu-
nate mistranslation of V. Def. 5. He does not seem to have had the
penetration to see that Campanus was translating from an Arabic,
and not from a Greek, text Zamberti tells us that he spent
seven years over his translation of the thirteen Books of the
Elements. As he seems to have been born in 1473, and the Elements
were printed as early as 1 500, though the complete work (including the
Pkaenomena, Optica, Catoptrica, Data etc.) has the date 1505 at the
end, he must have translated Euclid before the age of 30. Heiberg
has not been able to identify the Ms, of the Elements which Zamberti
used; but it is clear that it belonged to the worse class of mss., since
it contains most of the interpolations of the Theonine variety. Zam-
berti, as his title shows, attributed the proofs to Theon.

1 509. As a counterblast to Zamberti, Luca Faciuolo brought out
an edition of Euclid, apparently at the expense of Ratdolt, at Venice
{per Paganinunt de Pagattinis), in which he set himself to vindicate
Campanus. The title-page of this now very rare edition' begins thus:
``The works of Euclid of Megara, a most acute philosopher and without

1 The title begins thus: ``Euclidis megaresis philosophi platonicj
  mathematicarum disciplinarum Janitoris: Habent in hoc volumine
  quicunque ad mathematicam substantiam aspirant: elementorum libros
  xiij cum expositione Theonis insignis mathematici.  quibus multa
  quae deerant ex lectione gracea sumpta addita sunt nec non plurima
  peruersa et praepostere: voluta in Campani interpretatione: ordinata
  digesta et castigata sunt etc.''  For a description of the book see
  Weissenborn, p.~12 sqq.

2 See Weissenborn, p.~30 sqq.

question the chief of all mathematicians, translated by Campanus their
most faithful interpreter'' It proceeds to say that the translation had
been, through the fault of copyists, so spoiled and deformed that it
could scarcely be recognised as Euclid, Luca Faciuolo accordingly
has polished and emended it with the most critical judgment, has
corrected 129 figures wrongly drawn and added others, besides supply-
ing short explanations of difficult passages. It is added that Scipio
Vegius of Milan, distinguished for his knowledge ``of both languages''
(i.e.\ of course Latin and Greek), as well as in medicine and the more
sublime studies, had helped to make the edition more perfect. Though
Zamberti is not once mentioned, this latter remark must have refer-
ence to Zamberti's statement that his translation was from the Greek
text; and no doubt Zamberti is aimed at in the wish of Paciuolo's
``that others too would seek to acquire knowledge instead of merely
showing off, or that they would not try to make a market of the
things of which they are ignorant, as it were (selling) smoke'.''
Weissenborn observes that, while there are many trivialities in Paci-
uolo's notes, they contain some useful and practical hints and explana-
tions of terms, besides some new proofs which of course are not
difficult if one takes the liberty, as Paciuolo does, of diverging from
Euclid's order and assuming for the proof of a proposition results not
arrived at till later. Two not inapt terms are used in this edition to
describe the figures of H\prop{1}{7}, 8, the former of which is called the
goose's foot (pes anseris), the second the peacock's tail (cauda pavonis)
Paciuolo as the castigator of Campanus' translation, as he calls himself,
failed to correct the mistranslation of v, Def. 5 s , Before the fifth
Book he inserted a discourse which he gave at Venice on the
15th August, 1508, in S. Bartholomew's Church, before a select
audience of 500, as an introduction to his elucidation of that Book.
1516. The first of the editions giving Campanus' and Zamberti's
translations in conjunction was brought out at Paris (in officina Henrici
Stephani e regiene scholae Decretorum). The idea that only the enun-
ciations were Euclid's, and that Campanus was the author of the proofs
in his translation, while Theon was the author of the proofs in the Greek
text, reappears in the title of this edition; and the enunciations of the
added Books XIV., XV. are also attributed to Euclid, Hypsicles being
credited with the proofs 1 . The date is not on the title-page nor at the

1  ``Atque utinam et alii cognoscere vellent non ostentare aut ea quae
  nesciunt veluti fumum venditare non conarentur.''

* Campanus' translation in Ratdolt's edition is. as follows:
  ``Quantitates quae dicuntur continuam habere proportionalitatem,
  sunt, quarum equè multiplicia aut equa sunt aut equè sibi sine
  interruptione addunt aut minuunut'' (!), to which Campanus adds the
  note: ``Continuè proportionalia sunt quorum omnia multiplicia
  equalia sunt continuè proportionalie.  Sed noluit ipsam
  diffinitionem proponere sub hac forma, quia tunc diffiniret idem per
  idem, aperte (?a parte) tamen rei est istud cum sua diffinitione
  convertibile.''

3 ``Euclidis Megarensis Geometricorum Elementorum Libri \r15\@.
  Campani Galli transalpini in eosdem commentariorum libri~\r15\@.
  Theonis Alexandrini Bartholomaeo Zamberto Veneto interprete, in
  tredecim priores, commentationum libri~\r13\@.  Hypsiclis Alexandrini
  in duos posteriores, eodem Bartholomaeo Zamberto Veneto interprete,
  commentariorum libri~\r2.''  On the last page (261) is a similar
  statement of content, but wish the difference that the expression
  ``ex Campani…deinde Theonis…et Hypsiclis…\emph{traditionibus}.''
  For description see Weissenborn, p.~56~sqq.

end, but the letter of dedication to Francois Brieonnet by Jacques
Leftvre is dated the day after the Epiphany, 15 16. The figures are
in the margin. The arrangement of the propositions is as follows:
first the enunciation with the heading Eudides ex Catnpano, then the
proof with the note Campanus, and after that, as Campani additio, any
passage found in the edition of Campanus' translation but not in the
Greek text; then follows the text of the enunciation translated from
the Greek with the heading Euclides ex Zamberto, and lastly the proof
headed Theo ex Zamberto. There are separate figures for the two proofs,
This edition was reissued with few changes in 1537 and 1546 at Basel
(apud Jehannem Heruagium), but with the addition of the Pkaenomena,
Optica, Catoptrica etc.\ For the edition of 1537 the Paris edition of
15 16 was collated with ``a Greek copy'' (as the preface says) by
Christian Herlin, professor of mathematical studies at Strasshurg,
who however seems to have done no more than correct one or two
passages by the. help of the Basel editio princeps (1533), and add the
Greek word in cases where Zamberti's translation of it seemed unsuit-
able or inaccurate
We now come to

II. Editions of the Greek text,

1 533 is the date of the editio princeps, the title-page of which reads
as follows:

ETKAEIAOT STOlXEIflN BIBA>- IE>

EK TON 6EON02 ZYNOYSICN.

Et? tov avrov to wpTOV, ifyyriftdTv UpoicXov /9*/9\. 5.

Adiecta praefatiuncula in qua de disciplinis

Mathematicis nonnihil.

BASILEAE APVD 10 AN. HERVAGIVM ANNO

M.D.XXXIII. MENSE SEPTEMBRI.

The editor was Simon Grynaeus the elder (d. 1541), who, after
working at Vienna and Ofen, Heidelberg and Tubingen, taught last
of all at Basel, where theology was his main subject. His ``prae-
fatiuncula ``is addressed to an Englishman, Cuthbert Tonstall { 1474-
559)i who, having studied first at Oxford, then at Cambridge, where
he became Doctor of Laws, and afterwards at Padua, where in addi-
tion he leamt mathematics — mostly from the works of Regiomontanus
and Paciuolo— wrote a book on arithmetic 1 as ``a farewell to the
sciences,'' and then, entering politics, became Bishop of London and
member of the Privy Council, and afterwards (1530) Bishop of Durham.
Grynaeus tells us that he used two \textsc{mss.}\  of the text of the Elements,
entrusted to friends of his, one at Venice by ``Lazarus Bayfius''
(Lazare de BaJf, then the ambassador of the King of France at Venice),
the other at Paris by ``loann. Rvellius ``(J ean Ruel, a French doctor
and a Greek scholar), while the commentaries of Proclus were put at

1 \emph{De arte supputandi libri quatuor.}

the disposal of Grynaeus himself by ``loann. Claymundus'' at Oxford,
Heiberg has been able to identify the two mss. used for the text;
they are (i) cod. Venetus Marcianus 301 and (2) cod. Paris, gr. 2343
of the 16th c, containing Books 1. — xv., with some scholia which are
embodied in the text. When Grynaeus notes in the margin the
readings from ``the other copy,'' this ``other copy ``is as a rule the
Paris \textsc{ms.}, though sometimes the reading of the Paris \textsc{ms.}\  is taken
into the text and the ``other copy ``of the margin is the Venice \textsc{ms.}\ 
Besides these two mss. Grynaeus consulted Zamberti, as is shown by
a number of marginal notes referring to ``Zampertus ``or to ``latin um
exemplar'' in certain propositions of Books IX. — XL When it is con-
sidered that the two \textsc{mss.}\  used by Grynaeus are among the worst, it
is obvious how entirely unauthoritative is the text of the editio princeps.
Yet it remained the source and foundation of later editions of the
Greek text for a long period, the editions which followed being
designed, not for the purpose of giving, from other \textsc{mss.}, a text more
nearly representing what Euclid himself wrote, but of supplying a
handy compendium to students at a moderate price.

1536. Orontius Finaeus (Oronce Fine) published at Paris {apurf
Simonem Colinaeum) ``demonstrations on the first six books of Euclid's
elements of geometry,'' ``in which the Greek text of Euclid himself is
inserted in its proper places, with the Latin translation of Barth.
Zamberti of Venice,'' which seems to imply that only the enunciations
were given in Greek. The preface, from which Kastner quotes', says
that the University of Paris at that time required, from all who
aspired to the laurels of philosophy, a most solemn oath that they
had attended lectures on the said first six Books. Other editions of
Fine's work followed in 1544 and 1551.

1545. The enunciations of the fifteen Books were published in
Greek, with an Italian translation by Angelo Caiani, at Rome (apud
Antonium Bladum Asulanum). The translator claims to have cor-
rected the books and ``purged them of six hundred things which did
not seem to savour of the almost divine genius and the perspicuity of
Euclid'``

1549. Joachim Camerarius published the enunciations of the first
six Books in Greek and Latin (Leipzig). The book, with preface,
purports to be brought out by Rhaeticus (1514-1576), a pupil of
Copernicus. Another edition with proofs of the propositions of the
first three Books was published by Moritz Steinmetz in 1 577 (Leipzig);
a note by the printer attributes the preface to Camerarius himself.

155a loan. Scheubel published at Basel (also per loan. Her-
vagium) the first six Books in Greek and Latin ``together with true
and appropriate proofs of the propositions, without the use of letters ``
(i.e.\ letters denoting points in the figures), the various straight lines
and angles being described in words 1 .

1557 (also 1558). Stephanus Gracilis published another edition
(repeated 1 573, 1 578, 1 598) of the enunciations (alone) of Books I, — XV.

1 Kästner, \r1. p.~260.

2 Heiberg, vol.~\r5. p.~cvii.

3 Kästner, \r1. p.~359.

in Greek and Latin at Paris {apud Gulielmurn Cavellai). He remarks
in the preface that for want of time he had changed scarcely anything
in Books I. — VI., but; n the remaining Books he had emended what
seemed obscure or inelegant in the Latin translation, while he had
adopted in its entirety the translation of Book x. by Pierre Mondore
(PetrusMontaureus), published separately at Paris in 1551. Gracilis
also added a few ``scholia.''

1564. In this year Conrad Dasypodius (Rauchfuss), the inventor
and maker of the clock in Strassburg cathedral, similar to the present
one, which did duty from 1571 to 1789, edited (Strassburg, Chr.
Mylius) (1) Book 1. of the Elements in Greek and Latin with scholia,
(2) Book 11. in Greek and Latin with Barlaam's arithmetical version
of Book II., and (3) the enunciations of the remaining Books III. — XIII.
Book 1. was reissued with ``vocabula quaedam geometrica ``of Heron,
the enunciations of all the Books of the Elements, and the other works
of Euclid, all in Greek and Latin. In the preface to (1) he says that it
had been for twenty-six years the rule of his school that all who were
promoted from the classes to public lectures should learn the first
Book, and that he brought it out, because there were then no longer
any copies to be had, and in order to prevent a good and fruitful
regulation of his school from falling through. In the preface to the
edition of 157 1 he says that the first Book was generally taught in all
gymnasia and that it was prescribed in his school for the first class.
In the preface to (3) he tells us that he published the enunciations of
Books in.— xill. in order not to leave his work unfinished, but that, as
it would be irksome to carry about the whole work of Euclid in
extenso, he thought it would be more convenient to students of
geometry to learn the Elements if they were compressed into a smaller
book.

1620. Henry Briggs {of Briggs' logarithms) published the first
six Books in Greek with a Latin translation after Commandinus,
``corrected in many places'' (London, G. Jones).

1703 is the date of the Oxford edition by David Gregory which,
until the issue of Heiberg and Menge's edition, was still the only
edition of the complete works of Euclid 1 . In the Latin translation
attached to the Greek text Gregory says that he followed Comman-
dinus in the main, but corrected numberless passages in it by means
of the books in the Bodleian Library which belonged to Edward
Bernard (1638- 1 696), formerly Savilian Professor of Astronomy, who
had conceived the plan of publishing the complete works of the ancient
mathematicians in fourteen volumes, of which the first was to contain
Euclid's Elements I. — xv. As regards the Greek text, Gregory tells us
that he consulted, as far as was necessary, not a few MSS, of the better
sort, bequeathed by the great Savile to the University, as well as the
corrections made by Savile in his own hand in the margin of the Basel
edition. He had the help of John Hudson, Bodley's Librarian, who

1 \greek{ΕΥΚΛΕΙΔΟΥ ΤΑ ΣΩΖΟΜΕΝΑ}.  Euclidis quae supersunt omnia.  Ex
  recensione Davidis Gregorii M.D. Astronomia Professoris Saviliani et
  R.S.S.  Oxoniae, e Theatro Sheldoniano, An.\ Dom.\ \r1703.

punctuated the Basel text before it went to the printer, compared the
Latin version with the Greek throughout, especially in the Elements
and Data, and, where they differed or inhere he suspected the Greek text,
consulted the Greek msS. and put their readings in the margin if
they agreed with the Latin and, if they did not agree, affixed an
asterisk in order that Gregory might judge which reading was geo-
metrically preferable. Hence it is clear that no Greek \textsc{ms.}, but the
Basel edition, was the foundation of Gregory's text, and that Greek
\textsc{mss.}\  were only referred to in the special passages to which Hudson
called attention.

1 is 14-1 818. A most important step towards a good Greek text
was taken by F. Peyrard, who published at Paris, between these years,
in three volumes, the Elements and Data in Greek, Latin and French 1 .
At the time (1808) when Napoleon was having valuable \textsc{mss.}\  selected
from Italian libraries and sent to Paris, Peyrard managed to get two
ancient Vatican \textsc{mss.}\  (190 and 1038) sent to Paris for his use (Vat.
304 was a ' so at Paris at the time, but all three were restored to their
owners in 1 8 14). Peyrard noticed the excellence of Cod. Vat. 190,
adopted many of its readings, and gave in an appendix a conspectus
of these readings and those of Gregory's edition; he also noted here
and there readings from Vat. 1038 and various Paris \textsc{mss.}\  He there-
fore pointed the way towards a better text, but committed the error
of correcting the Basel text instead of rejecting it altogether and
starting afresh.

1824-1825. A most valuable edition of Books I. — VI. is that of
J. G. Camerer (and C. F, Hauber) in two volumes published at
Berlin*. The Greek text is based on Peyrard, although the Basel
and Oxford editions were also used. There is a Latin translation
and a collection of notes far more complete than any other I have
seen and well nigh inexhaustible. There is no editor or commentator
of any mark who is not quoted from; to show the variety of important
authorities drawn upon by Camerer, I need only mention the following
names: Proclus, Pappus, Tartaglia, Command inus, Clavius, Peletier,
Barrow, Borelli, Watlis, Tacquet, Austin, Simson, Playfair. No words
of praise would be too warm for this veritable encyclopaedia of
information.

1825. J. G. C. Neide edited, from Peyrard, the text of Books
I. — VI., XI. and XII. (Halts Saxoniae).

1826-9. The last edition of the Greek text before Heiberg's is
that of E. F. August, who followed the Vatican \textsc{ms.}\  more closely
than Peyrard did, and consulted at all events the Viennese \textsc{ms.}\ 
Gr, 103 (Heiberg's V). August's edition (Berlin, 1826-9) contains
Books I. — XIII.

1 \emph{Euclidis quae supersunt.}  \emph{Les \?OEuvres d'Euclide, en
  Greç en Latin et en Français d'après un manuscrit très-ancien, qui
  était resté inconnu jusqu'à nos jours.}  Par F. Peyrarḍ Ouvrage
  approuvé par l'Institut de France (Paris, chez M. Patris).

1 \emph{Euclidis elementorum libri sex priores graece et latine
  commentario e scriptis veterum ac recentiorum mathematicorum et
  Pfleidereri maxime illustrati} (Berolini, sumptibus G. Reimeri).
  Tom.~\r1. 1824; tom.~\r2. 1825.

III. Latin versions or commentaries after 1533.

1545. Petrus Ramus (Pierre de la Ramee, 151 5-1572) is credited
with a translation of Euclid which appeared in 1545 and again in
1549 at Paris 1 . Ramus, who was more rhetorician and logician than
geometer, also published in his Scholae mathematical ( iSS9i Frankfurt;
1569, Base!) what amounts to a series of lectures on Euclid's Elements,
in which he criticises Euclid's arrangement of his propositions, the
definitions, postulates and axioms, all from the point of view of logic.

1557. Demonstrations to the geometrical Elements of Euclid, six
Books, by Peletarius (Jacques Peletier). The second edition (1610)
contained the same with the addition of the ``Greek text of Euclid'';
but only the enunciations of the propositions, as well as the defini-
tions etc., are given in Greek (with a Latin translation), the rest is
in Latin only. He has some acute observations, for instance about
the ``angle'' of contact

•559- Johannes Buteo, or Borrel (1492-1573), published in an
appendix to his book De quadratura circuit some notes ``on the errors
of Campanus, Zambertus, Orontius, Peletarius, Pena, interpreters of
Euclid.'' Buteo in these notes proved, by reasoned argument based
on original authorities, that Euclid himself and not Theon was the
author of the proofs of the propositions.

1 566. Franciscus Flussates Candalla (Francois de Foix, Comte de
Candale, 1 502-1 594) ``restored'' the fifteen Books, following, as he
says, the terminology of Zamberti's translation from the Greek, but
drawing, for his proofs, on both Campanus and Theon (i.e.\ Zamberti)
except where mistakes in them made emendation necessary. Other
editions followed in 1578, 1602, 1695 (in Dutch).

1572. The most important Latin translation is that of Com-
mand irtus (1509-1575) of Urbino, since it was the foundation of most
translations which followed it up to the time of Peyrard, including
that of Simson and therefore of those editions, numerous in England,
which give Euclid ``chiefly after the text of Simson.'' Simson 's first
(Latin) edition (1756) has ``ex versione Latina Federici Commandini''
on the title-page. Commandinus not only followed the original Greek
more closely than his predecessors but added to his translation some
ancient scholia as well as good notes of his own. The title of his
work is

Euclidis elementorum libri xv, una cum sckeliis antiquis.
A Federico Contmandino Urbinate nuper in latinum conversi,
commentariisque quibusdam illustrati (Pisauri, apud Camillum
Francischinum).

He remarks in his preface that Orontius Finaeus had only edited
six Books without reference to any Greek ms., that Peletarius had
followed Campanus' version from the Arabic rather than the Greek
text, and that Candalla had diverged too far from Euclid, having
rejected as inelegant the proofs given in the Greek text and
substituted faulty proofs of his own. Commandinus appears to have

1 Described by Boncompagni, \emph{Bullettino}, \r2. p.~389.

used, in addition to the Basel editio princeps, some Greek ms., so far
not identified; he also extracted his ``scholia antiqua ``from a US.
of the class of Vat. 192 containing the scholia distinguished by
Heiberg as ``Schol. Vat.'' New editions of Commandinus' translation
followed in 1575 (in Italian), 1619, 1749 (in English, by Keill and
Stone), 1756 (Books 1. — VI., XI., XII. in Latin and English, by Simson),
1763 (Keill). Besides these there were many editions of parts of the
whole work, e.g.\ the first six Books.

[574. The first edition of the Latin version by Clavius'
(Christoph Klau [?], born at Bamberg 1537, died 1612) appeared
in 1574, and new editions of it in 1589, 1591, 1603, 1607, 1612. It is
not a translation, as Clavius himself states in the preface, but it
contains a vast amount of notes collected from previous commentators
and editors, as well as some good criticisms and elucidations of his
own. Among other things, Clavius finally disposed of the error by
which Euclid had been identified with Euclid of Megara. He speaks
of the differences between Campanus who followed the Arabic
tradition and the ``commentaries of Theon,'' by which he appears to
mean the Euclidean proofs as handed down by Theon; he complains
of predecessors who have either only given the first six Books, or
have rejected the ancient proofs and substituted worse proofs of their
own, but makes an exception as regards Commandinus, ``a geometer
not of the common sort, who has lately restored Euclid, in a Latin
translation, to his original brilliancy.'' Clavius, as already stated, did
not give a translation of the Elements but rewrote the proofs, com-
pressing them or adding to them, where he thought that he could
make them clearer. Altogether his book is a most useful work.

1 62 1. Henry Savile's lectures {Praelectiones tresdecttn in prin-
cipium EUtnentorum Etulidis Oxoniae habitae MDC.XX., Oxonii 1621),
though they do not extend beyond \prop{1}{8}, are valuable because they
grapple with the difficulties connected with the preliminary matter,
the definitions etc., and the tacit assumptions contained in the first
propositions,

1654 Andre 1 Tacquet's Elementa geometriae planae et solidae
containing apparently the eight geometrical Books arranged for
general use in schools. It came out in a large number of editions up
to the end of the eighteenth century.

1655, Barrow's Euclidis Eiementorum Libri XV breviter demon-
strati is a book of the same kind. In the preface (to the edition of
[659) he says that he would not have written it but for the fact that
Tacquet gave only eight Books of Euclid. He compressed the work
into a very small compass (less than 400 small pages, in the edition
of 1659, for the whole of the fifteen Books and the Data) by abbre-
viating the proofs and using a large quantity of symbols (which, he
says, are generally Oughtred's). There were several editions up to
1732 (those of 1660 and 1732 and one or two others are in English).

1 \emph{Euclidis elementorum librĭ\r15.  Accessit \r16. de solidorum
  regularium comparatione.  Omnes perspicuis demonstrationibus,
  accuratisque scholiis illustrati.  Auctore Christophoro Clavio}
  (Romae, apud Vincentium Accoltum), 2~vols.

1658. Giovanni Alfonso Borelli (1608- 1 679) published Euclides
restitutus, on apparently similar lines, which went through three more
editions (one in Italian, 1663).

166a Claude Francois Milliet Dechales' eight geometrical Books
of Euclid's Elements made easy. Dechales' versions of the Elements
had great vogue, appearing in French, Italian and English as well
as Latin. Riccardi enumerates over twenty editions.

1733. Saccheri's Euclides ab omni naevo vindicatus sive const us
geometrkus quo stabiliuntur prima ipsa geomelriae principia is
important for his elaborate attempt to grove the parallel-postulate,
forming an important stage in the history of the development of non-
Euclidean geometry.

1756. Simson's first edition, in Latin and in English. The Latin
title is

Euclidis elementorum libri priores sex, item undecimus et duo-
decimus, ex versione latina Federici Commandtni; sublatis iis
quibus olim libri hi a Tkeone, aliisve, vitiali sunt, et quibusdam
Euclidis demonstrationibus restitutis. A Roberto Simson M.D.
Glasguae, in aedibus Academicis excudebant Robertus et Andreas
Foulis, Academiae typographi.

1802. Euclidis elementorum libri priores XII ex Commandini et
Gregorii versianibus latinis. In usum juventutis Academicae..,hy
Samuel Horsley, Bishop of Rochester. (Oxford, Clarendon Press.)

IV. Italian versions or commentaries.

1543. Tartaglia's version, a second edition of which was pub-
lished in 1565', and a third in 1585. It does not appear that he used
any Greek text, for in the edition of 1565 he mentions as available
only ``the first translation by Campano,'' ``the second made by
Bartolomeo Zamberto Veneto who is still alive,'' ``the editions of
Paris or Germany in which they have included both the aforesaid
translations,'' and ``our own translation into the vulgar (tongue).''

IS7S- Commandinus' translation turned into Italian and revised
by him.

1613. The first six Books ``reduced to practice'' by Pietro
Antonio Cataldi, re-issued in 1620, and followed by Books VH, — ix.
(1621) and Book X. (1625).

1663. Borelli's Latin translation turned into Italian by Domenico
Magni.

1680. Euclide restitute by Vitale Giordano.

169a Vincenzo Vivian i's Eletnenti piani e solid t di Euclide
(Book v. in 1674).

1 The title-page of the edition of 1565; is as follows:  \emph{Euclide
  Megarense philosopho, solo introduttore delle scientie mathematice,
  diligentemente rassettato, et alla intregrità ridotto, per il degno
  professore di tal scientie Nicolo Tartalea Brisciano.  secundo le
  due tradottioni.  con una ampla espositione dello istesso tradottore
  di nuouo aggiunta.  talmente chiara, che ogni mediocre ingegno,
  senza la notitia, ouer suffragio di alcun altra scientia con
  facilità serà capace a poterlo intendere.} In Venetia,
  Appresso Curtio Troiano, 1565.

173 1. Elementi geomttrici ptant e solid 'i di Eu elide by Guido
Grandi. No translation, but an abbreviated version, of which new
editions followed one another up to 1806.

1749. Italian translation of'' Dechales with Ozanam's corrections
and additions, re-issued 1785, 1797.

1752. Leonardo Ximenes (the first six Books). Fifth edition,
1 819.

181 8. Vincenzo Flauti's Corso di geometries eUtnentare e sublime
(4 vols.) contains (Vol. I.) the first six Books, with additions and a
dissertation on Postulate 5, and (Vol. II.) Books xi„ XH. Flauti
also published the first six Books in 1827 and the Elements of geometry
of Euclid m 1843 and 1854

V. German.

1558. The arithmetical Books vir. — ix. by Scheubel'' (cf.\ the
edition of the first six Books, with enunciations in Greek and Latin,
mentioned above, under date 1 5 50).

1562. The version of the first six Books by Wilhelm Holtzmann
(Xylander)*. This work has its interest as the first edition in German,
but otherwise it is not of importance. Xylander tells us that it was
written for practical people such as artists, goldsmiths, builders etc.,
and that, as the simple amateur is of course content to know facts,
without knowing how to prove them, he has often left out the proofs
altogether. He has indeed taken the greatest possible liberties with
Euclid, and has not grappled with any of the theoretical difficulties,
such as that of the theory of parallels.

1651. Heinrich Hoffmann's Teutscher Euclides (2nd edition 1653),
not a translation.

1694. Ant Ernst Burkh. v. Pirckenstein's Teutsch Redender
Euclides (eight geometrical Books), ``for generals, engineers etc.''
``proved in a new and quite easy manner.'' Other editions 1699,
1744.

1697. Samuel Reyher's In teutscher Sfiraclie vorgestellter Euclides
(six Books), ``made easy, with symbols algebraical or derived from the
newest art of solution.''

1714. Euclidu xv Bilcker teutsch, ``treated in a special and
brief manner, yet completely,'' by Chr. Schessler (another edition in
1729).

1773. The first six Books translated from the Greek for the
use of schools by J. F, Lorenz. The first attempt to reproduce
Euclid in German word for word.

1 78 1. Books XI., xji. by Lorenz (supplementary to the pre-
ceding). Also Euklid's Etemente fUnfzekn Biicker translated from

1 \emph{Das sibend acht und neunt buch des hochberümbten Mathematici
  Euclidis Megarensis…durch Magistrum Johann Scheybl, der löblichen
  universitet zu Tübingen, des Euclidis und Arithmetic Ordinarien,
  auss dem latein ins teutsch gebracht….}

2 \emph{Die sechs erste Bücher Euclidis vom anfang oder grund der
  Geometrj…Auss Griechischer sprach in die Teütsch gebracht aigentlich
  erklärt…Demassen vormals in Teütscher sprach nie gesehen
  worden…Durch Wilhelm Holtzman genant Xylander von Augspurg.}
  Getruckht zu Basel.

the Greek by Lorenz (second edition 1798; editions of 1809, 1818,
1824 by Mollweide, of 1840 by Dippe). The edition of 1824, and
I presume those before it, are shortened by the use of symbols and
the compression of the enunciation and ``setting-out'' into one.

1807. Books 1.— vi., xi. ( xil. ``newly translated from the Greek,''
by J. K. F. Hauff.

1828. The same Books by Joh. Jos. Ign. Hoffmann ``as guide
to instruction in elementary geometry,'' followed in 1832 by observa-
tions on the text by the same editor.

1833. Die Geometric des Euklid und das Wesen derselben by
E. S. Unger; also 1838, 1851.

1901. Max Simon, Euclid und die seeks planimetrischen Bucket.

VI, French.

1564-1566. Nine Books translated by Pierre Forcadel, a pupil
and friend of p.~de la Ramee.

1604. The first nine Books translated and annotated by Jean
Errard de Bar-Ie-Duc; second edition, 1605.

161 5. Denis Hen r ion's translation of the 15 Books (seven
editions up to 1676),

1639. The first six Books ``demonstrated by symbols, by a
method very brief and intelligible,'' by Pierre Hengone, mentioned
by Barrow as the only editor who, before him, had used symbols for
the exposition of Euclid.

1672. Eight Books ``rend us plus faciles'' by Claude Francois
Mi Diet Dec hales, who also brought out Les ilemens d'Euclide ex-
pliquis d'une maniere nouvelie et trh facile, which appeared in many
editions, 1672, 1677, 1683 etc.\ (from 1709 onwards revised by Ozanam),
and was translated into Italian (1749 etc.) and English (by William
Halifax, 1685).

1 804. In this year, and therefore before his edition of the Greek
text, F. Peyrard published the Elements literally translated into
French. A second edition appeared in 1 809 with the addition of the
fifth Book, As this second edition contains Books 1. — vi. XL, xn.
and x. 1, it would appear that the first edition contained Books 1. — iv.,
VI., XI., XII. Peyrard used for this translation the Oxford Greek text
and Simson.

VII. Dutch.

1606. Jan Pieterszoon Dou (six Books). There were many later
editions. Kastner, in mentioning one of 1702, says that Dou explains
in his preface that he used Xylander's translation, but, having after-
wards obtained the French translation of the six Books by Errard
de Bar-le-Duc (see above),' the proofs in which sometimes pleased
him more than those of the German edition, he made his Dutch
version by the help of both.

1 61 7. Frans van Schooten, ``The Propositions of the Books of
Euclid's Elements''; the fifteen Books in this version ``enlarged ``by
Jakob van Leest in 1662.

1695. C. J. Vooght, fifteen Books complete, with Candalla's ``16th.''

1702. Kendfik Coets, six Books (also in Latin, 1692); several
editions up to 1752. Apparently not a translation: but an edition for
school use.

1763. Pybo Steenstra, Books I. — vr„ XI., XII., likewise an abbre-
viated version, several times reissued until 1825.

VIII. English.
1570 saw the first and the most important translation, that of Sir
Henry Billingsley. The title-page is as follows:

THE ELEMENTS

OF GEOMETRIE

of the most auncient Philosopher

E VCLWE

of Megara

Faithfully (now first) translated into the English toung,

by H. Billingsley, Citizen of London. Whereunto are annexed

certaine Scholies, Annotations, and Inuentions, of the best

Mathematiciens, both of time past, and in this our age.

With a very fruitfull Preface by M. I. Dee, specifying the
chiefe Mathetnaticall Sciiees, what they are, and whereunto
commodious: where, also, are disclosed certaine new Secrets
Mathetnaticall and Afechanicall, vntill these our dales, greatly
missed.

Imprinted at London by John Daye.

The Preface by the translator, after a sentence observing that with-
out the diligent study of Euclides Efementes it is impossible to attain
unto the perfect knowledge of Geometry, proceeds thus. ``Wherefore
considering the want and lacke of such good authors hitherto in our
Englishe tounge, lamenting also the negligence, and lacke of zeale to
their countrey in those of our nation, to whom God hath geuen both
knowledge and also abilitie to translate into our tounge, and to
publishe abroad such good authors and bookes (the chiefe instrumentes
of all learninges): seing moreouer that many good wittes both of
gentlemen and of others of all degrees, much desirous and studious of
these artes, and seeking for them as much as they can, sparing no
paines, and yet frustrate of their intent, by no meanes attaining to
that which they seeke: I haue for their sakes, with some charge and
great trauaile, faithfully translated into our vulgar e touge, and set
abroad in Print, this booke of Euclide. Whereunto I haue added
easie and plaine declarations and examples by figures, of the defini-
tions. In which booke also ye shall in due place finde manifolde
additions, Scholies, Annotations, and Inuentions; which I haue
gathered out of many of the most famous and chiefe Mathematicies,
both of old time, and in our age: as by diligent reading it in course,
ye shall well perceaue...,''

It is truly a monumental work, consisting of 464 leaves, and there-
fore 928 pages, of folio size, excluding the lengthy preface by Dee.
The notes certainly include all the most important that had ever been
written, from those of the Greek commentators, Proclus and the others
whom he quotes, down to those of Dee himself on the la3t books.
Besides the fifteen Books, Billingsley included the ``sixteenth'' added
by Candalla, The print and appearance of the book are worthy of its
contents; and, in order that it may be understood how no pains were
spared to represent everything in the clearest and most perfect form,
I need only mention that the figures of the propositions in Book XI.
are nearly all duplicated, one being the figure of Euclid, the other an
arrangement of pieces of paper (triangular, rectangular etc.) pasted at
the edges on to the page of the book so that the pieces can be turned
up and made to show the real form of the solid figures represented.

Billingsley was admitted Lady Margaret Scholar of St John's
College, Cambridge, in 1551, and he is also said to have studied at
Oxford, but he did not take a degree at either University. He was
afterwards apprenticed to a London haberdasher and rapidly became
a wealthy merchant. Sheriff of London in 1584, he was elected Lord
Mayor on 31st December, 1596, on the death, during his year of office,
of Sir Thomas Skinner. From 1589 he was one of the Queen's four
``customers,'' or farmers of customs, of the port of London. In 1 591
he founded three scholarships at St John's College for poor students,
and gave to the College for their maintenance two messuages and
tenements in Tower Street and in Mark Lane, Allhallows, Barking.
He died in 1606.

1651. Elements of Geometry. The first VI Bootks: In a compen-
dious form contracted and demonstrated by Captain Thomas Rudd, with
the mathematicall preface of John Dee (London).

1660. The first English edition of Barrow's Euclid (published in
Latin in 1655)1 appeared in London. It contained ``the whole fifteen
books compendiously demonstrated''; several editions followed, in
1705, 1722, 1732, 1751.

1 66 1. Euclid 's Elements of Geometry, with a supplement of divers
Propositions and Corollaries. To which is added a Treatise of regular
Soli lis by Campane and Ftussat; likewise Euclid's Data and Marinus
his Preface. Also a Treatise of the Divisions of Superficies, ascribed to
Machomet Bagdedine, but published by Commandine at the request of
J, Dee of London, Published by care and industry of John Leeke and
Geo. Serle, students in the Math. (London). According to Potts this
was a second edition of Billingsley's translation.

1685. William Halifax's version of Dechales' ``Elements of Euclid
explained in a new but most easy method ``(London and Oxford).

1705. The English Euclide; being tlie first six Elements of
Geometry, translated out of the Greek, with annotations and useful/
supplements by Edmund Scarburgh (Ox ford ). A noteworthy and
useful edition.

1708. Books I. — VL, XL, xii., translated from Command in us' Latin
version by Dr John Keill, Savilian Professor of Astronomy at Oxford.

Keill complains in his preface of the omissions by such editors as
Tacquet and Dechales of many necessary propositions (e.g.\ \prop{6}{27} — 29),
and of their substitution of proofs of their own for Euclid's. He praises
Barrow's version on the whole, though objecting to the ``algebraical ``
form of proof adopted in Book n., and to the excessive use of notes
and symbols, which (he considers) make the proofs too short and
thereby obscure: his edition was therefore intended to hit a proper
mean between Barrow's excessive brevity and Clavius' prolixity.

Keill's translation was revised by Samuel Cunn and several times
reissued. 1749 saw the eighth edition, 1772 the eleventh, and 1782
the twelfth.

1714. W. Whiston's English version (abridged) of The Elements
of Euclid with select theorems out of Archimedes by the learned Andr.
Tacquet.

1756. Simson's first English edition appeared in the same year as
his Latin version under the title:

The Elements of Euclid, vis. the first six Books together with
the eleventh and twelfth. In this Edition the Errors by which
Theon or others have long ago vitiated these Books are corrected and
some of Euclid's Demonstrations are restored. By Robert Simson
(Glasgow).

As above stated, the Latin edition, by its title, purports to be ``ex
version© latina Federici Commandini,'' but to the Latin edition, as well
as to the English editions, are appended

Notes Critical and Geometrical; containing an Account of those
things in which this Edition differs from the Greek text; and the
Reasons of the Alterations which have been made. As also Obser-
vations on some of the Propositions.

Simson says in the Preface to some editions (e.g.\ the tenth, of
1799) that ``the translation is much amended by the friendly assistance
of a learned gentleman.''

Simson's version and his notes are so well known as not to need
any further description. The book went through some thirty suc-
cessive editions. The first five appear to have been dated 1756, 1762,
1767, 1772 and 1775 respectively; the tenth 1799, the thirteenth 1806,
the twenty-third 1830, the twenty-fourth 1834, the twenty-sixth 1844.
The Data ``in like manner corrected '' was added for the first time in
the edition of 1762 (the first octavo edition).

1781, 1788. In these years respectively appeared the two volumes
containing the complete translation of the whole thirteen Books by
James Williamson, the last English translation which reproduced
Euclid word for word. The title is

The Elements of Euclid, with Dissertations intended to assist
and encourage a critical examination of these Elements, as the most
effectual means of establishing a j'uster taste upon mathematical
subjects than that which at present prevails. By James Williamson.
In the first volume (Oxford, 1781) he is described as ``M.A.
Fellow of Hertford College,'' and in the second (London, printed by
T. Spilsbury, 1788) as ``B.D.'' simply. Books v., VI. with the Con-
clusion in the first volume are paged separately from the rest

1 78 1 . 4 n examination of the first six Books of Euclid's Elements,
by William Austin (London).

'79S- John Playfair's first edition, containing ``the first six Books
of Euclid with two Books on the Geometry of Solids.'' The book
reached a fifth edition in 1819, an eighth in 1831, a ninth in 1836, and
a tenth in 1846.

1826. Riccardi notes under this date Euclid's Elements of Geo-
metry containing the whole twelve Books translated into English, from the
edition of Peyrard, by George Phillips. The editor, who was President
of Queens' College, Cambridge, 1857-1892, was born in 1804 and
matriculated at Queens' in 1 826, so that he must have published the
book as an undergraduate.

1828. A very valuable edition of the first six Books is that of
Dionysius Lardner, with commentary and geometrical exercises, to
which he added, in place of Books XI., XII., a Treatise on Solid
Geometry mostly based on Legend re, Lardner compresses the pro-
positions by combining the enunciation and the setting-out, and he
gives a vast number of riders and additional propositions in smaller
print The book had reached a ninth edition by 1846, and an eleventh
by 1855. Among other things, Lardner gives an Appendix ``on the
theory of parallel lines,'' in which he gives a short history of the
attempts to get over the difficulty of the parallel' postulate, down to
that of Legendre.

1833. T. Perronet Thompson's Geometry without axioms, or the
first Book of Euclid's Elements with alterations and notes; and an
intercalary book in which the straight line and plane are derived from
properties of the sphere, with an appendix containing notices of methods
proposed for getting over the difficulty in the twelfth axiom of Euclid.

Thompson (1783-1869) was 7th wrangler 1802, midshipman 1803,
Fellow of Queens* College, Cambridge, 1804, and afterwards general
and politician. The book went through several editions, but, having
been well translated into French by Van Tenac, is said to have
received more recognition in France than at home.

1 845. Robert Potts' first edition (and one of the best) entitled:

Euclid's Elements of Geometry chiefly from the text of
Dr Simson with explanatory notes... to which is prefixed an
introduction containing a brief outline of the History of Geometry.
Designed for the use of the higher forms in Public Schools and
students in the Universities (Cambridge University Press, and
London, John W. Parker), to which was added (1847) An
Appendix to the larger edition of Euclid's Elements of Geometry,
containing additional notes on the Elements, a short tract on trans-
versals, and hints for the solution of the problems etc.
1862. Todhunter's edition.

The later English editions 1 will not attempt to enumerate; their
name is legion and their object mostly that of adapting Euclid for school
use, with all possible gradations of departure from his text and order.

IX, Spanish.
1576. The first six Books translated into Spanish by Rodrigo
Camorano.

1637. The first six Books translated, with notes, by L. Carduchi.

1689. Books 1.— vi, XI, XII, translated and explained by Jacob
Knesa.

X. Russian.

1739. Ivan Astaroff* (translation from Latin).

1 789. Pr. Suvoroff and Yos. Nikitin {translation from Greek).

1880. Vachtchenko-Zakhartchenko.

{181 7, A translation into Polish by Jo. Czecha.)

XI. Swedish.

1744. Mitten Stromer, the first six Books; second edition 1748.
The third edition (1753) contained Books XI.— XU. as well; new
editions continued to appear till 1884,

1836. H. Falk, the first six Books.

1844, 1845, 1859. p.~R. Brakenhjelm, Books I. — VI., XL, xn.

1850. F. A. A. Lundgren.

1850. H. A. Witt and M. E. Areskong, Books I.— VI., XI., XII.

XII. Danish.

1745. Ernest Gottlieb Ziegenbalg.
1803. H. C. Linderup, Books I. — VI.

XIII. Modern Greek.
1820. Benjamin of Lesbos,

I should add a reference to certain editions which have appeared
in recent years.

A Danish translation (Euklid's Eletnenter oversat af Thyra Eibe)
was completed in 1912; Books I. — II. were published (with an Intro-
duction by Zeuthen) in 1897, Books in. — IV. in 1900, Books v. — VI.
in 1904, Books VII. — XIII. in 19 1 2,

The Italians, whose great services to elementary geometry are
more than once emphasised in this work, have lately shown a note-
worthy disposition to make the ipsissima verba of Euclid once more
the object of study. Giovanni Vacca has edited the text of Book I.
(// prima libro degli Elementi. Testo greco, versione italiana, intro-
duzione e note, Firenze 1 916,) Federigo Enriques has begun the
publication of a complete Italian translation (Gli Elementi d' Enclide
e la critica antica e moderna); Books I. — IV. appeared in 1925 (Alberto
Stock, Roma).

An edition of Book I. by the present writer was published in 1918
{Euclid in Greek \ Book [., with Introduction and Notes, Camb. Univ.
Press).

\chapter{}

\section{On the Nature of \emph{Elements}}

It would not be easy to find a more lucid explanation of the terms
element and elementary, and of the distinction between them, than
is found in Proclus \ who is doubtless, here as so often, quoting
from Geminus. There are, says Proclus, in the whole of geometry
certain leading theorems, bearing to those which follow the relation of
a principle, all- pervading, and furnishing proofs of many properties.
Such theorems are called by the name of elements; and their function
may be compared to that of the letters of the alphabet in relation to
language, letters being indeed called by the same name in Greek
(orotyeta).

The term elementary, on the other hand, has a wider application:
it is applicable to things ``which extend to greater multiplicity, and,
though possessing simplicity and elegance, have no longer the same
dignity as the elements, because their investigation is not of general
use in the whole of the science, e.g.\ the proposition that in triangles
the perpendiculars from the angles to the transverse sides meet in a
point.''

``Again, the term element is used in two senses, as Menaechmus
says. For that which is the means of obtaining is an element of that
which is obtained, as the first proposition in Euclid is of the second,
and the fourth of the fifth. In this sense many things may even be
said to be elements of each other, for they are obtained from one
another. Thus from the fact that the exterior angles of rectilineal
figures are (together) equal to four right angles we deduce the number
of right angles equal to the internal angles (taken together)*, and
vice versa. Such an element is like a lemma. But the term element is
otherwise used of that into which, being more simple, the composite is
divided; and in this sense we can no longer say that everything is an
element of everything, but only that things which are more of the
nature of principles are elements of those which stand to them in the
relation of results, as postulates are elements of theorems. It is

1 Proclus, \emph{Comm.\ on Eucl.}\ \r1., ed.\ Friedlein, pp.~72~sqq.

2 \greek{τὸ πλῆθος τῶν ἐντὸς ὀρθαῖς ἴσων.} If the text is right, we
  must apparently take it as ``the number of the angles equal to right
  angles that there are inside,'' i.e.\ that are made up by the
  internal angles.

according to this signification of the term element that the elements
found in Euclid were compiled, being partly those of plane geometry,
and partly those of stereometry. In like manner many writers have
drawn up elementary treatises in arithmetic and astronomy.

``Now it is difficult, in each science, both to select and arrange in
due order the elements from which all the rest proceeds, and into
which all the rest is resolved. And of those who have made the
attempt some were able to put together more and some less; some
used shorter proofs, some extended their investigation to an indefinite
length; some avoided the method of reductio ad absurdum, some
avoided proportion-, some contrived preliminary steps directed against
those who reject the principles; and, in a word, many different
methods have been invented by various writers of elements.

``It is essential that such a treatise should be rid of everything
superfluous (for this is an obstacle to the acquisition of knowledge);
it should select everything that embraces the subject and brings it to
a point (for this is of supreme service to science); it must have great
regard at once to clearness and conciseness (for their opposites trouble
our understanding); it must aim at the embracing of theorems in
general terms (for the piecemeal division of instruction into the more
partial makes knowledge difficult to grasp). In all these ways
Euclid's system of elements will be found to be superior to the rest;
for its utility avails towards the investigation of the primordial
figures 1 , its clearness and organic perfection are secured by the
progression from the more simple to the more complex and by the
foundation of the investigation upon common notions, while generality
of demonstration is secured by the progression through the theorems
which are primary and of the nature of principles to the things sought.
As for the things which seem to be wanting, they are partly to be
discovered by the same methods, like the construction of the scalene
and isosceles (triangle), partly alien to the character of a selection of
elements as introducing hopeless and boundless complexity, like the
subject of unordered irrationals which Apoilonius worked out at
length'', and partly developed from things handed down (in the
elements) as causes, like the many species of angles and of lines.
These things then have been omitted in Euclid, though they have
received full discussion in other works; but the knowledge of them is
derived from the simple (elements).''

Proclus, speaking apparently on his own behalf, in another place
distinguishes two objects aimed at in Euclid's Elements. The first
has reference to the matter of the investigation, and here, like a good
Platonist, he takes the whole subject of geometry to be concerned
with the ``cosmic figures,'' the five regular solids, which in Book XHi.

1 \greek{τῶν ἀρχικῶν σχημάτων}, by which Proclus probably means the
  regular polyhedra (Tannery, p.~143\emph{n}.).

2 We have no more than the most obscure indications of the character
  of this work in an Arabic \textsc{ms.}\ analysed by Woepcke,
  \emph{Essai d'une restitution de travaux perdus d'Apollonius sur les
  quantités irrationelles d'après des indications tirées d'un
  manuscrit arabe} in \emph{Mémoires présentés à l'académie des
  sciences}, \r14 658–720, Paris, 1856.  Cf.\ Cantor,
  \emph{Gesch.\ d.\ Math.}\ \r1\tsub{3}, pp.~348–9: details are also
  given in my notes to book~\book{10}.

are constructed, inscribed in a sphere and compared with one another.
The second object is relative to the learner; and, from this standpoint,
the elements may be described as ``a means of perfecting the learner's
understanding with reference to the whole of geometry. For, starting
from these (elements), we shall be able to acquire knowledge of the
other parts of this science as well, while without them it is impossible
for us to get a grasp of so complex a subject, and knowledge of the
rest is unattainable. As it is, the theorems which are most of the
nature of principles, most simple, and most akin to the first hypotheses
are here collected, in their appropriate order; and the proofs of all
other propositions use these theorems as thoroughly well known, and
start from them. Thus Archimedes in the books on the sphere and
cylinder, Apollonius, and all other geometers, clearly use the theorems
proved in this very treatise as constituting admitted principles 1 .''

Aristotle too speaks of elements of geometry in the same sense.
Thus: ``in geometry it is well to be thoroughly versed in the
elements* ``; ``in general the first of the elements are, given the
definitions, e.g.\ of a straight line and of a circle, most easy to prove,
although of course there are not many data that can be used to
establish each of them because there are not many middle terms'``;
``among geometrical propositions we call those 'elements' the proofs of
which are contained in the proofs of all or most of such propositions'``;
``(as in the case of bodies), so in like manner we speak of the elements
of geometrical propositions and, generally, of demonstrations; for the
demonstrations which come first and are contained in a variety of
other demonstrations are called elements of those demonstrations...
the term element is applied by analogy to that which, being one and
small, is useful for many purposes •.''

\section{\emph{Elements} Anterior to Euclid's}

The early part of the famous summary of P rod us was no doubt
drawn, at least indirectly, from the history of geometry by Eudemus;
this is generally inferred from the remark, made just after the mention
of PhiHppus of Medma, a disciple of Plato, that ``those who have
written histories bring the development of this science up to this
point.'' We have therefore the best authority for the list of writers of
elements given in the summary. Hippocrates of Chios (fl, in second
half of 5th c.) is the first; then Leon, who also discovered diorismi,
put together a more careful collection, the propositions proved in it
being more numerous as well as more serviceable*. Leon was a little
older than Eudoxus (about 408-355 n.C.) and a little younger than
Plato (428/7-347/6 B.C.), but Hid not belong to the latter's school. The

1 Proclus, pp.~70, 19–71, 21.

2 \emph{Topics} \r8.~14, 163~b~23.

3 \emph{Topics} \r8.~3, 158~b~35. 

4 \emph{Metaph.}\ 998~a~25.

5 \emph{Metaph.}\ 1014~a~35–b~5.

6 Proclus, p.~66, 20 \greek{ὥστε τὸν Λέοντα καὶ τὰ στοιχεῖα συνθεῖναι
  τῷ τε πλήθει καὶ τῇ χρείᾳ τῶν δεικνυμένων ἐπιμελέστερον}.

geometrical text-book of the Academy was written by Theudius of
Magnesia, who, with Amyclas of Heraclea, Menaechmus the pupil of
Eudoxus, Menaechmus' brother Dinostratus and Athenaeus of Cyzicus
consorted together in the Academy and carried on their investigations
in common. Theudius ``put together the elements admirably, making
many partial (or limited) propositions more general 1 .'' Eudemus
mentions no text- book after that of Theudius, only adding that Her-
motimus of Colophon ``discovered many of the elements 1 .'' Theudius
then must be taken to be the immediate precursor of Euclid, and no
doubt Euclid made full use of Theudius as well as of the discoveries of
Hermotimus and all other available material. Naturally it is not in
Euclid's Elements that we can find much light upon the state of the
subject when he took it up; but we have another source of informa-
tion in Aristotle. Fortunately for the historian of mathematics,
Aristotle was fond of mathematical illustrations; he refers to a con-
siderable number of geometrical propositions, definitions etc., in a
way which shows that his pupils must have had at hand some text-
book where they could find the things he mentions; and this text -book
must have been that of Theudius. Heiberg has made a most valuable
collection of mathematical extracts from Aristotle*, from which much
is to be gathered as to the changes which Euclid made in the methods
of his predecessors; and these passages, as well as others not included
in Heiberg's selection, will often be referred to in the sequel.

\section{First Principles: Definitions, Postulates,
and Axioms}

On no part of the subject does Aristotle give more valuable
information than on that of the first principles as, doubtless, generally
accepted at the time when he wrote. One long passage in the
Posterior Analytics is particularly full and lucid, and is worth quoting
m txtenso. After laying it down that every demonstrative science
starts from necessary principles', he proceeds':

``By first principles in each genus I mean those the truth of which
it is not possible to prove. What is denoted by the first (terms) and
those derived from them is assumed; but, as regards their existence,
this must be assumed for the principles but proved for the rest. Thus
what a unit is, what the straight (line) is, or what a triangle is (must
be assumed); and the existence of the unit and of magnitude must
also be assumed, but the rest must be proved. Now of tie premisses
used in demonstrative sciences some are peculiar to each science and
others common (to all), the latter being common by analogy, for of
course they are actually useful in so far as they are applied to the sub-
ject-matter included under the particular science. Instances of first

1 Proclus, p.~67, 14 \greek{καὶ γὰρ τὰ στοιχεῖα καλῶς συνέταξεν καὶ
  πολλὰ τῶν μερικῶν [\greek{ὁρικῶν} (?) Friedlein] καθολικω/τερα
  ἐποίησεν}.

2 Proclus, p.~67, 22 \greek{τῶν στοιχείων πολλὰ ἀνεῦρε}.

3 \emph{Mathematisches zu Aristoteles} in \emph{Abhandlungen zur
  Gesch.\ d.\ math.\ Wissenschaften}, \r18. Heft (1904), pp.~1—49.

4 \emph{Anal.\ post.}\ \r1.~6, 74~b~5.

5 \ibid~\r1.~10, 76~a~31–77~a~4.

principles peculiar to a science are the assumptions that a line is of
such and such a character, and similarly for the straight (line); whereas
it is a common principle, for instance, that, if equals be subtracted
from equals, the remainders are equal. But it is enough that each of
the common principles is true so far as regards the particular genus
(subject-matter); for (in geometry) the effect will be the same even if
the common principle be assumed to be true, not of everything, but
only of magnitudes, and, in arithmetic, of numbers.

``Now the things peculiar to the science, the existence of which
must be assumed, are the things with reference to which the science
investigates the essential attributes, e.g.\ arithmetic with reference to
units, and geometry with reference to points and l: nes. With these
things it is assumed that they exist and that they are of such and
such a nature. But, with regard to their essential properties, what is
assumed is only the meaning of each term employed: thus arithmetic
assumes the answer to the question what is (meant by) ' odd ' or
'even,' 'a square' or 'a cube,' and geometry to the question
what is (meant by) ' the irrational ' or ' deflection ' or (the so-called)
' verging ' (to a point); but that there are such things is proved by
means of the common principles and of what has already been
demonstrated. Similarly with astronomy. For every demonstrative
science has to do with three things, (i) the things which are assumed
to exist, namely the genus (subject-matter) in each case, the essential
properties of which the science investigates, (2) the common axioms
so-called, which are the primary source of demonstration, and (3) the
properties with regard to which all that is assumed is the meaning of
the respective terms used. There is, however, no reason why some
sciences should not omit to speak of one or other of these things.
Thus there need not be any supposition as to the existence of the
genus, if it is manifest that it exists (for it is not equally clear that
number exists and that cold and hot exist); and, with regard to the
properties, there need be no assumption as to the meaning of terms if
it is clear r just as in the common (axioms) there is no assumption as
to what is the meaning of subtracting equals from equals, because it is
well known. But none the less is it true that there are three things
naturally distinct, the subject-matter of the proof, the things proved,
and the (axioms) from which (the proof starts).

``Now that which isperse necessarily true, and must necessarily be
thought so, is not a hypothesis nor yet a postulate. For demon-
stration has not to do with reasoning from outside but with the
reason dwelling in the soul, j ust as is the case with the syllogism.
It is always possible to raise objection to reasoning from outside,
but to contradict the reason within us is not always possible. Now
anything that the teacher assumes, though it is matter of proof,
without proving it himself, is a hypothesis if the thing assumed is
believed by the learner, and it is moreover a hypothesis, not abso-
lutely, but relatively to the particular pupil; but, if the same thing
is assumed when the learner either has no opinion on the subject
or is of a contrary opinion, it is a postulate. This is the difference

between 3 hypothesis and a postulate; for a postulate is that which
is rather contrary than otherwise to the opinion of the learner, or
whatever is assumed and used without being proved, although matter
for demonstration. Now definitions are not hypotheses, for they do
not assert the existence or non-existence of anything, while hypotheses
are among propositions. Definitions only require to be understood:
a definition is therefore not a hypothesis, unless indeed it be asserted
that any audible speech is a hypothesis. A hypothesis is that from
the truth of which, if assumed, a conclusion can be established. Nor
are the geometer's hypotheses false, as some have said: I mean those
who say that ' you should not make use of what is false, and yet the
geometer falsely calls the line which he has drawn a foot long when
it is not, or straight when it is not straight' The geometer bases no
conclusion on the particular line which he has drawn being that which
he has described, but (he refers to) what is illustrated by the figures.
Further, the postulate and every hypothesis are either universal or
particular statements; definitions are neither'' (because the subject
is of equal extent with what is predicated of it).

Every demonstrative science, says Aristotle, must start from in-
demonstrable principles: otherwise, the steps of demonstration would
be endless. Of these indemonstrable principles some are (a) common
to all sciences, others are (0) particular, or peculiar to the particular
science; (a) the common principles are the axioms, most commonly
illustrated by the axiom that, if equals be subtracted from equals, the
remainders are equal. Coming now to (6) the principles peculiar to
the particular science which must be assumed, we have first the genus
or subject-matter, the existence of which must be assumed, viz.\ magni-
tude in the case of geometry, the unit in the case of arithmetic. Under
this we must assume definitions of manifestations or attributes of the
genus, e.g.\ straight lines, triangles, deflection etc.\ The definition in
itself says nothing as to the existence of the thing defined: it only
requires to be understood. But in geometry, in addition to the genus
and the definitions, we have to assume the existence of a few primary
things which are defined, viz.\ points and lines only: the existence
of everything else, e.g.\ the various figures made up of these, as
triangles, squares, tangents, and their properties, e.g, incommensur-
ability etc., has to be proved (as it is proved by construction arid
demonstration). In arithmetic we assume the existence of the unit:
but, as regards the rest, only the definitions, e.g, those of odd, even,
square, cube, are assumed, and existence has to be proved. We have then
clearly distinguished, among the indemonstrable principles, axioms
and definitions, A postulate is also distinguished from a hypothesis,
the latter being made with the assent of the learner, the former
without such assent or even in opposition to his opinion (though,
strangely enough, immediately after saying this, Aristotle gives a
wider meaning to ``postulate'' which would cover ``hypothesis'' as well,
namely whatever is assumed, though it is matter for proof, and used
without being proved). Heiberg remarks that there is no trace in
Aristotle of Euclid's Postulates, and that ``postulate'' in Aristotle has
a different meaning. He seems to base this on the alternative
description of postulate, indistinguishable from a hypothesis; but,
if we take the other description in which it is distinguished from a
hypothesis as being an assumption of something which is a proper
subject of demonstration without the assent or against the opinion of
the learner, it seems to fit Euclid's Postulates fairly well, not only the
fi rst th ree ( post u I at i ng t h re e con struc t i on s), bu t e m i n en tl y al so the other
two, that all right angles are equal, and that two straight lines meeting
a third and making the internal angles on the same side of it less than
two right angles will meet on that side. Aristotle's description also
seems to me to suit the ``postulates'' with which Archimedes begins
his book On the equilibrium of planes, namely that equal weights balance
at equal distances, and that equal weights at unequal distances do not
balance but that the weight at the longer distance will prevail.

Aristotle's distinction also between hypothesis and definition, and
between hypothesis and axiom, is clear from the following passage:
``Among immediate syllogistic principles, I call that a thesis which-
it is neither possible to prove nor essential for any one to hold who
is to learn anything; but that which it is necessary for any one to
hold who is to learn anything whatever is an axiom: for there are
some principles of this kind, and that is the most usual name by
which we speak of them. But, of theses, one kind is that which
assumes one or other side of a predication, as, for instance, that
something exists or does not exist, and this is a hypothesis; the other,
which makes no Such assumption, is a definition. For a definition is
a thesis: thus the arithmetician posits (rlOcTat) that a unit is that
which is indivisible in respect of quantity; but this is not a hypo-
thesis, since what is meant by a unit and the fact that a unit exists
are different things 1 .''

Aristotle uses as an alternative term for axioms ``common (things),''
to, Kowd, or ``common opinions'' (koiwiI Sofoi)> as in the following
passages. ``That, when equals are taken from equals, the remainders
are equal is (a) common (principle) in the case of alt quantities, but
mathematics takes a separate department (atroXaovo'a) and directs its
investigation to some portion of its proper subject-matter, as e.g.\ lines
or angles, numbers, or any of the other quantities*.' 1 ``The common
(principles), e.g.\ that one of two contradictories must be true, that

equals taken from equals etc., and the like'' ````With regard to the

principles of demonstration, it is questionable whether they belong to
one science or to several. By principles of demonstration I mean the
common opinions from which all demonstration proceeds, e.g.\ that one
of two contradictories must be true, and that it is impossible for the
same thing to be and not be 4 .'' Similarly ``every demonstrative
(science) investigates, with regard to some subject-matter, the essential
attributes, starting from the common opinions'.'' We have then here,
as Heiberg says, a sufficient explanation of Euclid's term for axioms,

1 \emph{Anal.\ post.}\ \r1. 2, 72~a~14–24.

2 \emph{Metaph.}\ 1061~b~19–24.

3 \emph{Anal.\ post.}\ \r`.~11, 77~a~30.

4 \emph{Metaph.}\ 996~b~26–30.

5 \emph{Metaph.}\ 997~a~20–22.

viz.\ common notions (koivoI evvoiat), and there is no reason to suppose
it to be a substitution for the original term due to the Stoics: cf.
Proclus' remark that, according to Aristotle and the geometers, axiom
and common notion are the same thing 1 .

Aristotle discusses the indemonstrable character of the axioms
in the Metaphysics. Since ``all the demonstrative sciences use the
axioms','' the question arises, to what science does their discussion
belong*? The answer is that, like that of Being (oiJct-mi), it is the
province of the (first) philosopher 1 . It is impossible that there should
be demonstration of everything, as there would be an infinite series of
demonstrations: if the axioms were the subject of a demonstrative
science, there would have to be here too, as in other demonstrative
sciences, a subject-genus, its attributes and corresponding axioms 1 \ thus
there would be axioms behind axioms, and so on continually. The
axiom is the most firmly established of all principles*. It is ignorance
alone that could lead anyonetotryto prove the axioms'; the supposed
proof would be a.petilio principii'. If it is admitted that not every-
thing can be proved, no one can point to any principle more truly
indemonstrable'. If any one thought he could prove them, he could
at once be refuted; if he did not attempt to say anything, it would
be ridiculous to argue with him i he would be no better than a
vegetable 10 . The first condition of the possibility of any argument
whatever is that words should signify something both to the speaker
and to the hearer: without this there can be no reasoning with any one.
And, if any one admits that words can mean anything to both hearer
and speaker, he admits that something can be true without demon-
stration. And so on''.

It was necessary to give some sketch of Aristotle's view of the
first principles, if only in connexion with Proclus' account, which is
as follows. As in the case of other sciences, so ``the compiler of
elements in geometry must give separately the principles of the
science, and after that the conclusions from those principles, not
giving any account of the principles but only of their consequences.
No science proves its own principles, or even discourses about them:
they are treated as self-evident. . .Thus the first essential was to dis-
tinguish the principles from their consequences. Euclid carries out
this plan practically in every book and, as a preliminary to the whole
enquiry, sets out the common principles of this science. Then he
divides the common principles themselves into hypotheses, postulates,
and axioms. For all these are different from one another: an axiom,
a postulate and a hypothesis are not the same thing, as the inspired
Aristotle somewhere says. But, whenever that which is assumed and
ranked as a principle is both known to the learner and convincing in
itself, such a thing is an axiom, e.g.\ the statement that things which
are equal to the same thing are also equal to one another. When, on

1 Proclus, p.~194, 8.

2 \emph{Metaph.}\ 997~a~10.

3 \ibid~996~b~26, 

4 \ibid~1005~a~21–b~11.

5 \ibid~997~a~5–8.

6 \ibid~1005~b~11–17.

7 \ibid~1006~a~5.

8 \ibid~1006~a~17.

9 \ibid~1006~a~10

10 \ibid~1006~a~11–15.

11 \ibid~1006~a~18~sqq.

the other hand, the pupil has not the notion of what is told him
which carries conviction in itself, but nevertheless lays it down and
assents to its being assumed, such an assumption is a hypothesis.
Thus we do not preconceive by virtue of a common notion, and
without being taught, that the circle is such and such a figure, but,
when we are told so, we assent without demonstration. When again
what is asserted is both unknown and assumed even without the
assent of the learner, then, he says, we call this a postulate, e.g.\ that
all right angles are equal. This view of a postulate is clearly implied
by those who have made a special and systematic attempt to show,
with regard to one of the postulates, that it cannot be assented to by
any one straight off. According then to the teaching of Aristotle, an
axiom, a postulate and a hypothesis are thus distinguished 1 .''

We observe, first, that Proclus in this passage confuses hypotheses
and definitions, although Aristotle had made the distinction quite
plain. The confusion may be due to his having in his mind a passage
of Plato from which he evidently got the phrase about ``not giving
an account of'' the principles. The passage is a: ``I think you know
that those who treat of geometries and calculations (arithmetic) and
such things take for granted (inroQefievoi) odd and even, figures,
angles of three kinds, and other things akin to these in each subject,
implying that they know these things, and, though using them as
hypotheses, do not even condescend to give any account of them
either to themselves or to others, but begin from these things and
then go through everything else in order, arriving ultimately, by
recognised methods, at the conclusion which they started in search
of.'' But the hypothesis is here the assumption, e.g, ' that there may
he sttch a thing as length without breadth, henceforward called a line','
and so on, without any attempt to show that there is such a thing;
it is mentioned in connexion with the distinction between Plato's
'superior' and 'inferior' intellectual method, the former of which
uses successive hypotheses as stepping-stones by which it mounts
upwards to the idea of Good.

We pass now to Proclus' account of the difference between postu-
lates and axioms. He begins with the view of Geminus, according
to which ``they differ from one another in the same way as theorems
are also distinguished from problems. For, as in theorems we propose
to see and determine what follows on the premisses, while in problems
we are told to find and do something, in like manner in the axioms
such things are assumed as are manifest of themselves and easily
apprehended by our untaught notions, while in the postulates we
assume such things as are easy to find and effect (our understanding
suffering no strain in their assumption), and we require no complication
of machinery*.''...'' Both must have the characteristic of being simple

1 Proclus, pp.~75, 10–77, 2.

2 \emph{Republic}, \r6.~510~c. Cf.\ Aristotle, \emph{Nic.\ Eth.}\ 1151~a~17.

3 H.~Jackson, \emph{Journal of Philology}, vol.~\r10. p.~144.

4 Proclus, pp.~178, 12—179, 8.  In illustration Proclus contrasts the
  drawing of a straight Line or a circle with the drawing of a
  ``single-turn spiral ``or of an equilateral triangle, the spiral
  requiring more complex machinery and even the equilateral triangle
  needing a certain method. ``For the geometrical intelligence will
  say that by conceiving a straight line fixed at one end but, as
  regards the other end, moving round the fixed end, and a point
  moving along the straight line from the fixed end, I have described
  the single-turn spiral; for the end of the straight line describing
  a circle, and the point moving on the straight line simultaneously,
  when they arrive and meet at the same point, complete such a spiral.
  And again, if I draw equal circles, join their common point to the
  centres of the circles and draw a straight line from one of the
  centres to the other, I shall have the equilateral triangle.  These
  things then are far from being completed by means of a single act or
  of a moment's thought'' (p.~180. 8–21).

and readily grasped, I mean both the postulate and the axiom; but
the postulate bids us contrive and find some subject-matter (uXj,) to
exhibit a property simple and easily grasped, while the axiom bids us
assert some essential attribute which is self-evident to the learner,
just as is the fact that fire is hot, or any of the most obvious things 1 ,''

Again, says Proclus, ``some claim that alt these things are alike
postulates, in the same way as some maintain that all things that are
sought are problems. For Archimedes begins his first book on /«-
equilibrium' 1 with the remark ' I postulate that equal weights at equal
distances are in equilibrium,' though one would rather call this an
axiom. Others call them all axioms in the same way as some regard
as theorems everything that requires demonstration'.''

``Others again will say that postulates are peculiar to geometrical
subject-matter, while axioms are common to all investigation which
is concerned with quantity and magnitude. Thus it is the geometer
who knows that all right angles are equal and how to produce in
a straight line any limited straight line, whereas it is a Common notion
that things which are equal to the same thing are also equal to one
another, and it is employed by the arithmetician and any scientific
person who adapts the general statement to his own subject 1 .''

The third view of the distinction between a postulate and an axiom
is that of Aristotle above described'.

The difficulties in the way of reconciling Euclid's classification
of postulates and axioms with any one of the three alternative views
are next dwelt upon. If we accept the first view according to which
an axiom has reference to something known, and a postulate to
something done, then the 4th postulate (that all right angles are
equal) \% not a postulate; neither is the 5th which states that, if a
straight line falling on two straight lines makes the interior angles
on the same side less than two right angles, the straight lines, if
produced indefinitely, will meet on that side on which are the angles
less than two right angles. On the second view, the assumption that
two straight lines cannot enclose a space, ``which even now,'' says
Proclus, ``some add as an axiom,'' and which is peculiar to the
subject-matter of geometry, like the fact that all right angles are
equal, is not an axiom. According to the third (Aristotelian) view,
``everything which is confirmed (irt<not>Tat) by a sort of demonstration

1 Proclus, p.~181, 4–11.

2 It is necessary to coin a word to render \greek{ἀνισορροπιῶν}, which
  is moreover in the plural.  The title of the treatise as we have it
  is \emph{Equilibria of planes or centres of gravity of planes} in
  Book~\r1\ and \emph{Equilibria of planes} in Book~\r2.

3 Proclus, p.~181, 16–23.

4 \ibid~p.~182, 6–14.

5 Pp.~118, 119.

will be a postulate, and what is incapable of proof will be an axiom 1 .''
This last statement of Proclus is loose, as regards the axiom, because
it omits Aristotle's requirement that the axiom should be a self-
evident truth, and one that must be admitted by any one who is to
learn anything at all, and, as regards the postulate, because Aristotle
calls a postulate something assumed without proof though it is
``matter of demonstration'' (aTroSetierhv Sv), but says nothing of a
quasi -demon strati on of the postulates. On the whole I think it is
from Aristotle that we get the best idea of what Euclid understood
by a postulate and an axiom or common notion. Thus Aristotle's
account of an axiom as a principle common to all sciences, which is
self-evident, though incapable of proof, agrees sufficiently with the
contents of Euclid's common notions as reduced to five in the most
recent text (not omitting the fourth, that ``things which coincide are
equal to one another''). As regards the postulates, it must be borne
in mind that Aristotle says elsewhere' that, ``other things being equal,
that proof is the better which proceeds from the fewer postulates or
hypotheses or propositions.'' I f then we say that a geometer must
lay down as principles, first certain axioms or common notions, and
then an irreducible minimum of postulates in the Aristotelian sense
concerned only with the subject-matter of geometry, we are not far
from describing what Euclid in fact does. As regards the postulates
we may imagine him saying:  Besides the common notions there are
a few other things which I must assume without proof, but which
differ from the common notions in that they are not self-evident.
The learner may or may not be disposed to agree to them; but he
must accept them at the outset on the superior authority of his
teacher, and must be left to convince himself of their truth in the
course of the investigation which follows. In the first place certain
simple constructions, the drawing and producing of a straight line,
and the drawing of a circle, must be assumed to be possible, and with
the constructions the existence of such things as straight lines and
circles; and besides this we must lay down some postulate to form
the basis of the theory of parallels.'' It is true that the admission of
the 4th postulate that all right angles are equal still presents a
difficulty to which we shall have to recur.

There is of course no foundation for the idea, which has found
its way into many text-books, that ``the object of the postulates is to
declare that the only instruments the use of which is permitted in
geometry are the rule and compass'.''

\section{Theorems and Problems}

``Again the deductions from the first principles,'' says Proclus,
``are divided into problems and theorems, the former embracing the

1 Proclus, pp.~i82, 21–183, 13.

2 \emph{Anal.\ post.}\ \r1.~25, 86~a~33–35.

3 cf.\ Lardner's Euclid: also Todhunter.

generation, division, subtraction or addition of figures, and generally
the changes which are brought about in them, the latter exhibiting
the essential attributes of each 1 ,''

``Now, of the ancients, some, like Speusippus and Amphinomus,
thought proper to call them all theorems, regarding the name of
theorems as more appropriate than that of problems to theoretic
sciences, especially as these deal with eternal objects. For there is
no becoming in things eternal, so that neither could the problem
have any place with them, since it promises the generation and
making of what has not before existed, e.g.\ the construction of an
equilateral triangle, or the describing of a square on a given straight
line, or the placing of a straight line at a given point. Hence they
say it is better to assert that all (propositions) are of the same kind,
and that we regard the generation that takes place in them as
referring not to actual making but to knowledge, when we treat things
existing eternally as if they were subject to becoming: in other words,
we may say that everything is treated by way of theorem and not
by way of problem* (irdvra 6eapr)fjiaTiKw oX* ov TrpQJ3\iifiaTtic<io<i
\etfi0 ui' hit (tat).

``Others on the contrary, like the mathematicians of the school
of Menaechmus, thought it right to call them all problems, describing
their purpose as twofold, namely in some cases to furnish (yropi-
ffatrat) the thing sought, in others to take a determinate object
and see either what it is, or of what nature, or what is its property,
or in what relations it stands to something else.

``In reality both assertions are correct. Speusippus is right
because the problems of geometry are not like those of mechanics,
the latter being matters of sense and exhibiting becoming and change
of every sort. The school of Menaechmus are right also because the
discoveries even of theorems do not arise without an issuing- forth
into matter, by which I mean intelligible matter. Thus forms going
out into matter and giving it shape may fairly be said to be like
processes of becoming. For we say that the motion of our thought
and the thro wing-out of the forms in it is what produces the figures
in the imagination and the conditions subsisting in them. It is in
the imagination that constructions, divisions, placings, applications,
additions and subtractions (take place), but everything in the mind is
fixed and immune from becoming and from every sort of change 1 .''

``Now those who distinguish the theorem from the problem say
that every problem implies the possibility, not only of that which is
predicated of its subject-matter, but also of its opposite, whereas
every theorem implies the possibility of the thing predicated but not
of its opposite as well. By the subject-matter I mean the genus
which is the subject of inquiry, for example, a triangle or a square
or a circle, and by the property predicated the essential attribute,
as equality, section, position, and the like. When then any or-e

1 Proclus, p.~77, 7–12.

2 \ibid~pp.~77, 15–78, 8.

3 \ibid~pp.~78, 8–79, 2.

enunciates thus, To inscribe an equilateral triangle in a circle, he states
a problem; for it is also possible to Inscribe in it a triangle which
is not equilateral. Again, if we take the enunciation On a given
limited straight line to construct an equilateral triangle, this is a
problem; for it is possible also to construct one which is not equi-
lateral. But, when any one enunciates that In isosceles triangles tlu
angles at the base are equal, we must say that he enunciates a theorem;
for it is not also possible that the angles at the base of isosceles
triangles should be unequal. It follows that, if any one were to use
the form of a problem and say In a semicircle to describe a right angle,
he would be set down as no geometer. For every angle in a semi-
circle is right 1 .''

'' Zenodotus, who belonged to the succession of Oenoptdes, but
was a disciple of Andron, distinguished the theorem from the problem
by the fact that the theorem inquires what is the property predicated
of the subject-matter in it, but the problem what is the cause of what
effect (rivoi Sirros ri icrriv). Hence too Posidonius denned the one
(the problem) as a proposition in which it is inquired whether a thing
exists or not (et ffo-rte t) /*$), the other (the theorem*) as a proposition
in which it is inquired what (a thing) is or of what nature (ri itrrtv fj
•irolov Tt); and he said that the theoretic proposition must be put in a
declaratory form, e.g., Any triangle has two sides (together) greater than
the remaining side and In any isosceles triangle the angles at the base
are equal, but that we should state the problematic proposition as If
inquiring whether it is possible to construct an equilateral triangle
upon such and such a straight line. For there is a difference between
inquiring absolutely and indeterminately (a-n-Xaq re KaX dopl<rrto<i)
whether there exists a straight line from such and such a point at
right angles to such and such a straight line and investigating which
is the straight line at right angles*.''

``That there is a certain difference between the problem and the
theorem is clear from what has been said; and that the Elements of
Euclid contain partly problems and partly theorems will be made
manifest by the individual propositions, where Euclid himself adds at
the end of what is proved in them, in some cases, 'that which it was
required to do,' and in others, ' that which it was required to prove, 1
the latter expression being regarded as characteristic of theorems, in
spite of the fact that, as we have said, demonstration is found in
problems also. In problems, however, even the demonstration is for
the purpose of (confirming) the construction: for we bring in the
demonstration in order to show that what was enjoined has been
done; whereas in theorems the demonstration is worthy of study for
its own sake as being capable of putting before us the nature of the
thing sought. And you will find that Euclid sometimes interweaves
theorems with problems and employs them in turn, as in the first

1 Proclus, pp.~79, 11–80, 5.

2 In the text we have \greek{τὸ δὲ πρόβλημα} answering to \greek{τὸ
  μὲν} without substantive: \greek{πρόβλημα} was obviously inserted in
  error,

3 Proclus, pp.~80, 15–81, 4.

book, while at other times he makes one or other preponderate.
For the fourth book consists wholly of problems, and the fifth of
theorems 1 .''

Again, in his note on Eucl. \prop{1}{4}, Proclus says that Carpus, the
writer on mechanics, raised the question of theorems and problems in
his treatise on astronomy. Carpus, we are told, ``says that the class
of problems is in order prior to theorems. For the subjects, the
properties of which are sought, are discovered by means of problems.
Moreover in a problem the enunciation is simple and requires no
skilled intelligence; it orders you plainly to do such and such a
thing, to construct an equilateral triangle, or, given two straight lines, to
cut off from the greater (a straight line) equal to the lesser, and what is
there obscure or elaborate in these things ? But the enunciation of a
theorem is a matter of labour and requires much exactness and
scientific judgment in order that it may not turn out to exceed or
fail short of the truth; an example is found even in this proposition
(\prop{1}{4}), the first of the theorems. Again, in the case of problems, one
general way has been discovered, that of analysis, by following which
we can always hope to succeed; it is this method by which the more
obscure problems are investigated. But, in the case of theorems, the
method of setting about them is hard to get hold of since ' up to our
time,' says Carpus, ' no one has been able to hand down a general
method for their discovery. Hence, by reason of their easiness, the
class of problems would naturally be more simple/ After these
distinctions, he proceeds: ' Hence it is that in the Elements too
problems precede theorems, and the Elements begin from them; the
first theorem is fourth in order, not because the fifth 11 is proved from
the problems, but because, even if it needs tor its demonstration none
of the propositions which precede it, it was necessary that they should
be first because they are problems, while it is a theorem. In fact, in
this theorem he uses the common notions exclusively, and in some
sort takes the same triangle placed in different positions; the
coincidence and the equality proved thereby depend entirely upon
sensible and distinct apprehension. Nevertheless, though the demon-
stration of the first theorem is of this character, the problems properly
preceded it, because in general problems are allotted the order of
precedence*.'``

Proclus himself explains the position of Prop.~4 after Props, i- — 3
as due to the fact that a theorem about the essential properties of
triangles ought not to be introduced before we know that such a
thing as a triangle can be constructed, nor a theorem about the
equality of sides or straight lines until we have shown, by constructing
them, that there can be two straight lines which are equal to one
another 4 . It is plausible enough to argue in this way that Props. 2
and 3 at all events should precede Prop, 4. And Prop, 1 is used in

1 Proclus, p.~81, 5–22.

2 \greek{τὸ πέμπτον} This should apparently be the fourth because in
  the next words it is implied that none of the first three
  propositions ate required in proving it.

3 Proclus, pp.~241, 19–243, 11.

4 \ibid~pp.~233, 21–234, 6.

Prop.~2, and must therefore precede it But Prop.~1 showing how to
construct an equilateral triangle on a given base is not important, in
relation to Prop.~4, as dealing with the ``production of triangles ``in
general: for it is of no use to say, as Proclus does, that the construc-
tion of the equilateral triangle is ``common to the three species (of
triangles) 1 ,'' as we are not in a position to know this at such an early
stage. The existence of triangles in general was doubtless assumed as
following from the existence of straight lines and points in one plane
and from the possibility of drawing a straight line from one point to
another.

Proclus does not however seem to reject definitely the view of
Carpus, for he goes on*: ``And perhaps problems are in order before
theorems, and especially for those who need to ascend from the arts
which are concerned with things of sense to theoretical investigation.
But in dignity theorems are prior to problems.... It is then foolish to
blame Geminus for saying that the theorem is more perfect than the
problem. For Carpus himself gave the priority to problems in respect
of order, and Geminus to theorems in point of more perfect dignity,''
so that there was no real inconsistency between the two.

Problems were classified according to the number of their possible
solutions. Amphinomus said that those which had a unique solution
(itowatSt) were called ``ordered ``(the word has dropped out in
Proclus, but it must be rerayftiva, in contrast to the third kind,
aTattra); those which had a definite number of solutions ``inter-
mediate'' (fiea-a); and those with an infinite variety of solutions ``un-
ordered ``(ajaMTa)'. Proclus gives as an example of the last the
problem To divide a given straight line into three parts in continued
proportion*. This is the same thing as solving the equations %+y+s=a,
xs =_y i . Proclus' remarks upon the problem show that it was solved,
like all quadratic equations, by the method of ``application of areas.''
The straight line a was first divided into any two parts, (jr-M)and_y,
subject to the sole limitation that (x + s) must not be less than 2y,
which limitation is the Stopto-fws, or condition of possibility. Then
an area was applied to (x + z), or («— y), ``falling short by a square
figure'' {eWetirov tXhei rerpaywy) and equal to the square on y. This
determines x and z separately in terms of a and y. For, if b be the
side of the square by which the area (i.e.\ rectangle) ``falls short,'' we
have {{a —y) — z\z ™j**, whence 2z «= (a —y) ± n/[(a —yf — 4y 3 }. And
y may be chosen arbitrarily, provided that it is not greater than 0/3.
Hence there are an infinite number of solutions. If y = a(i, then, as
Proclus remarks, the three parts are equal.

Other distinctions between different kinds of problems are added
by Proclus. The word ``problem,'' he says, is used in several senses.
In its widest sense it may mean anything ``propounded ``(irpojewo-
pi-i>ov), whether for the purpose of instruction {paBrio-em) or construc-
tion (n-oojo'fci)?). (In this sense, therefore, it would include a theorem.)

1 Proclus, p.~234, 21

2 \ibid~p.~243, 12–25.

3 \ibid~p.~220, 7—12.

4 \ibid~pp.~220, 16–221, 6.

But its special sense in mathematics is that of something ``propounded
with a view to a theoretic construction 1 .''

Again you may apply the term (in this restricted sense) even to
something which is impossible, although it is more appropriately used
of what is possible and neither asks too much nor contains too little in
the shape of data. According as a problem has one or other of these
defects respectively, it is called (1) a problem in excess (irXeovdgov) or
(2) a deficient problem (iK\nre<; irp60\T}p.a), The problem in excess
(1) is of two kinds, (a) a problem in which the properties of the
figure to be found are either inconsistent (atrii p/Sara) or non-existent
{avvirapicTa), in which case the problem is called impossible, or (b) a
problem in which the enunciation is merely redundant: an example
of this would be a problem requiring us to construct an equilateral
triangle with its vertical angle equal to two-thirds of a right angle;
such a problem is possible and is called ``more than a problem '' (p.elv
7j Trp6ft\T)fta). The deficient problem (2) is similarly called ``less than
a problem ``(tKaatrov rj Trpdft\i)f*a), its characteristic being that
something has to be added to the enunciation in order to convert it
from indeterminateness (aopttr-ria) to order (raf «) and scientific deter-
minateness (Spas evta-TtjtioviKos): such would be a problem bidding
you ``to construct an isosceles triangle,'' for the varieties of isosceles
triangles are unlimited. Such ``problems'' are not problems in the
proper sense {/cupim<! \cy6ftei/a vpofiXjjpaTa), but only equivocally''.

\section{The Formal Divisions of a Proposition}

``Every problem,'' says Proclus', ``and every theorem which is
complete with all its parts perfect purports to contain in itself all of
the following elements: enunciation (TrpoVao-is), setting-out (e*0Wt<:),
definition or specification (Stop tvr fiat), construction or machinery
(xaratrKt utj), proof (airoSeifiv), conclusion (uvp/!repai7jt,a). Now of
these the enunciation states what is given and what is that which is
sought, the perfect enunciation consisting of both these parts. The
setting-out marks off what is given, by itself, and adapts it before-
hand for use in the investigation. The definition or specification
states separately and makes clear what the particular thing is which
is sought. The construction or machinery adds what is wanting to the
datum for the purpose of finding what is sought .The proof draws
the required inference by reasoning scientifically from acknowledged
facts. The conclusion reverts again to the enunciation, confirming
what has been demonstrated. These are all the parts of problems
and theorems, but the most essential and those which are found in all
are enunciation, proof, conclusion. For it is equally necessary to know
beforehand what is sought, to prove this by means of the intermediate
steps, and to state the proved fact as a conclusion; it is impossible
to dispense with any of these three things. The remaining parts
are often brought in, but are often left out as serving no purpose.

1 Proclus, p.~221, 7–11.

2 \ibid~pp.~221, 13—222, 14.

3 \ibid~pp.~203, 1–204, 13; 204, 23–205, 8.

Thus there is neither setting-out nor definition in the problem of
constructing an isosceles triangle having each of the angles at the
base double of the remaining angle, and in most theorems there
is no construction because the setting'Oiit suffices without any addition
for proving the required property from the data. When then do
we say that the setting-out is wanting ? The answer is, when there
is nothing given in the enunciation; for, though the enunciation is
in general divided into what is given and what is sought, this
is not always the case, but sometimes it states only what is sought,
i.e.\ what must be known or found, as in the case of the problem
just mentioned. That problem does not, in fact, state beforehand
with what datum we are to construct the isosceles triangle having
each of the equal angles double of the remaining angle, but (simply)
that we are to find such a triangle When, then, the enuncia-
tion contains both (what is given and what is sought), in that case
we find both definition and setting-out, but, whenever the datum
is wanting, they too are wanting. For not only is the setting-out
concerned with the datum, but so is the definition also, as, in the
absence of the datum, the definition will be identical with the
enunciation. In fact, what could you say in defining the object of
the aforesaid problem except that it is required to find an isosceles
triangle of the kind referred to ? But that is what the enunciation
stated. If then the enunciation does not include, on the one hand,
what is given and, on the other, what is sought, there is no setting-out
in virtue of there being no datum, and the definition is left out in
order to avoid a mere repetition of the enunciation.''

The constituent parts of an Euclidean proposition will be readily
identified by means of the above description. As regards the defi-
nition or specification (S*opt<r/ios) it is to be observed that we have
here only one of its uses. Here it means a closer definition or descrip-
tion of the object aimed at, by means of the concrete lines or figures
set out in the eKeai<; instead of the general terms used in the enun-
ciation; and its purpose is to rivet the attention better, as Proclus
indicates in a later passage (rpoirov rtvk irpotrextia? itrrlv aiVioe o

The other technical use of the word to signify the limitations to
which the possible solutions of a problem are subject is also described
by Proclus, who speaks of hiopio-fioi determining ``whether what is
sought is impossible or possible, and how far it is practicable and in
how many ways 1 ``; and the Biopur/ioit in this sense appears in Euclid
as well as in Archimedes and Apollonius. Thus we have in Eucl. I,
22 the enunciation ``From three straight lines which are equal to
three given straight lines to construct a triangle,'' followed imme-
diately by the limiting condition ($tap«rfi6<;). ``Thus two of the
straight lines taken together in any manner must be greater than the
remaining one.'' Similarly in \prop{6}{28} the enunciation ``To a given
straight line to apply a parallelogram equal to a given rectilineal

1 Proclus, p.~2o8, 21.

2 \ibid~p.~202, 3.

figure and falling short by a parallelogrammic figure similar to a
given one ``is at once followed by the necessary condition of possi-
bility: ``Thus the given rectilineal figure must not be greater than
that described on half the line and similar to the defect.''

Tannery supposed that, in giving the other description of the
Biopta/tos as quoted above, Proclus, or rather his guide, was using the
term incorrectly. The Btopta-fios in the better known sense of the
determination of limits or conditions of possibility was, we are told,
invented by Leon. Pappus uses the word in this sense only. The
other use of the term might, Tannery thought, be due to a confusion
occasioned by the use of the same words (Bu }) in introducing the
parts of a proposition corresponding to the two meanings of the word
Btopur/io 1 . On the other hand it is to be observed that Eutocius
distinguishes clearly between the two uses and implies that the differ-
ence was well known 1 . The SiopuTft.6 in the sense of condition of
possibility follows immediately on the enunciation, is even part of it;
the iopitry.6s in the other sense of course comes immediately after the
setting-out.

Proclus has a useful observation respecting the conclusion of a
proposition 3 . ``The conclusion they are accustomed to make double
in a certain way: 1 mean, by proving it in the given case and then
drawing a general inference, passing, that is, from the partial con-
clusion to the general. For, inasmuch as they do not make use of
the individuality of the subjects taken, but only draw an angle or a
straight line with a view to placing the datum before our eyes, they
consider that this same fact which is established in the case of the
particular figure constitutes a conclusion true of every other figure of
the same kind. They pass accordingly to the general in order that
we may not conceive the conclusion to be partial. And they are
justified in so passing, since they use for the demonstration the par-
ticular things set out, not qud particulars, but qua typical of the rest
For it is not in virtue of such and such a size attaching to the angle
which is set out that I effect the bisection of it, but in virtue of its
being rectilineal and nothing more. Such and such size is peculiar to
the angle set out, but its quality of being rectilineal is common to all
rectilineal angles. Suppose, for example, that the given angle is a
right angle. If then I had employed in the proof the fact of its being
right, I should not have been able to pass to every species of recti-
lineal angle; but, if I make no use of its being right, and only consider
it as rectilineal, the argument will equally apply to rectilineal angles
in general.''

1 \emph{La Géométrie grecque}, p.~149 note. Where \greek{δεῖ δὴ}
  introduces the closer description of the pioblem we may translate,
  ``it is then requited'' or ``thus it is required'' (to construct
  etc.); when it introduces the condition of possibility we may
  translate ``thus it is necessary etc.''  Heiberg originally wrote
  \greek{δεῖ δὲ} in the latter sense in \prop{1}{21} on the authority
  of Proclus and Eutocius, and against that of the \textsc{mss.}
  Later, on the occasion of \prop{11}{13}, he observed that he should
  have followed the \textsc{mss.}\ and written \greek{δεῖ δὴ} which he
  found to be, after all, the right reading in Eutocius (Apollonius,
  ed.\ Heiberg, 11. p.~178). \greek{δεῖ δὴ} is also the expression
  used by Diophantus for introducing conditions of possibility.

2 See the passage of Eutocius referred to in last note.

3 Proclus, p.~207, 4–25.

\section{Other Technical Terms}

I. Things said to be given.

Proclus attaches to his description of the formal divisions of a
proposition an explanation of the different senses in which the word
give ft or datum (SeSoevoi') is used in geometry. ``Everything that is
given is given in one or other of the following ways, in position, in
ratio, in tnagnitude, or in species. The point is given in position only,
but a line and the rest may be given in all the senses 1 .''

The illustrations which Proclus gives of the four senses in which a
thing may be given are not altogether happy, and, as regards things
which are given in position, in magnitude, and in species, it is best, I
think, to follow the definitions given by Euclid himself in his book of
Data. Euclid does not mention the fourth class, things given in ratio,
nor apparently do any of the great geometers.

(i) Given in position really needs no definition; and, when Euclid
says {Data, Def 4) that ``Points, lines and angles are said to be given
in position which always occupy the same place,'' we are not really
the wiser.

(3) Given in tnagnitude is defined thus {Data, Def. 1): ``Areas,
lines and angles are called given in magnitude to which we can find
equals.'' Proclus' illustration is in this case the following: when, he
says, two unequal straight lines are given from the greater of which
we have to cut off a straight line equal to the lesser, the straight lines
are obviously given in magnitude, ``for greater and less, and finite
and infinite are predications peculiar to magnitude.'' But he does not
explain that part of the implication of the term is that a thing is given
in magnitude only, and that, for example, its position is not given and
is a matter of indifference

(3) Given in species. Euclid's definition {Data, Def. 3) is:
``Rectilineal figures are said to be given in species in which the angles
are severally given and the ratios of the sides to one another are
given.'' And this is the recognised use of the term (cf.\ Pappus,
passim) Proclus uses the term in a much wider sense for which I am
not aware of any authority. Thus, he says, when we speak of (bisect-
ing) a given rectilineal angle, the angle is given in species by the word
rectilineal, which prevents our attempting, by the same method, to
bisect a curvilineal angle I On Eucl. 1. 9, to which he here refers, he
says that an angle is given in species when e.g.\ we say that it is right
or acute or obtuse or rectilineal or ``mixed,'' but that the actual angle
in the proposition is given in species only. As a matter of fact, we
should say that the actual angle in the figure of the proposition is
given in magnitude and not in species, part of the implication of given
in species being that the actual magnitude of the thing given in species
is indifferent; an angle cannot be given in species in this sense at all.
The confusion in Proclus' mind is shown when, after saying that a
right angle is given in species, he describes a third of a right angle as
given in magnitude.

1 Proclus, p.~205, 13–15.

No better example of what is meant by given in species, in its
proper sense, as limited to rectilineal figures, can be quoted than the
given parallelogram in Eucl. \prop{6}{28}, to which the required parallelo-
gram has to be made similar; the former parallelogram is in fact
given in species, though its actual size, or scale, is indifferent

(4) Given in ratio presumably means something which is given
by means of its ratio to some other given thing. This we gather from
Proclus' remark (in his note on 1. 9) that an angle may be given in
ratio ``as when we say that it is double and treble of such and such an
angle or, generally, greater and less.'' The term, however, appears to
have no authority and to serve no purpose. Proclus may have
derived it from such expressions as ``in a given ratio'' which are
common enough.

2. Lemma.

``The term lemma'' says Proclus 1 , ``is often used of any proposition
which is assumed for the construction of something else: thus it is a
common remark that a proof has been made out of such and such
lemmas. But the special meaning of lemma in geometry is a
proposition requiring confirmation. For when, in either construction
or demonstration, we assume anything which has not been proved but
requires argument, then, because we regard what has been assumed as
doubtful in itself and therefore worthy of investigation, we call it a
lemma*, differing as it does from the postulate and the axiom in being
matter of demonstration, whereas they are immediately taken for
granted, without demonstration, for the purpose of confirming other
things. Now in the discovery of lemmas the best aid is a mental
aptitude for it. For we may see many who are quick at solutions and
yet do not work by method; thus Cratistus in our time was able to
obtain the required result from first principles, and those the fewest
possible, but it was his natural gift which helped him to the discovery.

1 Proclus, pp.~211, 1–212, 4.

* It would appear, says Tannery (p.~151~\emph{n}), that Geminus
  understood a lemma as being simply \greek{λαμβανόμενον}, something
  assumed (cf.\ the passage of Proclus, p.~73, 4, relating to
  Menaechmus' view of \emph{elements}): hence we cannot consider
  ourselves authorised in attributing to Geminus the more technical
  definition of the term here given by Proclus, according to which it
  is only used of propositions not proved beforehand. This view of a
  lemma must be considered as relatively modern. It seems to have had
  its origin in an imperfection of method. In the course of a
  demonstration it was necessary to assume a proposition which
  required proof, but the proof of which would, if inserted in the
  particular place, break the thread of the demonstration: hence it
  was necessary either to prove it beforehand as a preliminary
  proposition or to postpone it to be proved afterwards (\greek{ὡς
  ἐξῆς δειχθήσεται}).  when, after the time of Geminus, the progress
  of original discovery in geometry was arrested, geometers occupied
  themselves with the study and elucidation of the works of the great
  mathematicians who had preceded them. This involved the
  investigation of propositions explicitly quoted or tacitly assumed
  in the great classical treatises; and naturally it was found that
  several such remained to be demonstrated, either because Lhe authors
  had omitted them as being easy enough to be left to the reader
  himself to prove, or because books in which they were proved had
  been lost in the meantime. Hence arose a class of complementary or
  auxiliary propositions which were called lemmas. Thus Pappus gives
  in his Book~\r7\ a collection of lemmas in elucidation of the
  treatises of Euclid and Apollonius included in the so-called
  ``Treasury of Analysis (\greek{τόπος ἀναλυόμενος}).  When Proclus
  goes on to distinguish three methods of discovering lemmas,
  \emph{analysis}, \emph{division}, and \emph{reductio ad absurdum},
  he seems to imply that the principal business of contemporary
  geometers was the investigation of these auxiliary propositions.

Nevertheless certain methods have been handed down. The finest is
the method which by means of analysis carries the thing sought up to
an acknowledged principle, a method which Plato, as they say, com-
municated to Leodamas 1 , and by which the latter, too, is said to have
discovered many things in geometry. The second is the method of
division*, which divides into its parts the genus proposed for con-
sideration and gives a starting-point for the demonstration by means
of the elimination of the other elements in the construction of what is
proposed, which method also Plato extolled as being of assistance to
all sciences. The third is that by means of the reductio ad absurdntn,
which does not show what is sought di recti yi but refutes its opposite
and discovers the truth incidentally.''

3. Case.

``The case* (irTmtTK),'' Proclus proceeds*, ``announces different ways
of construction and alteration of positions due to the transposition of
points or lines or planes or solids. And, in general, all its varieties
are seen in the figure, and this is why it is called case, being a trans-
position in the construction.''

4. Porism.

``The term porism is used also of certain problems such as the
Porisms written by Euclid. But it is specially used when from what
has been demonstrated some other theorem is revealed at the same
time without our propounding it, which theorem has on this very
account been called a porism (corollary) as being a sort of incidental
gain arising from the scientific demonstration .'' cf.\ the note on 1. 15,

1 This passage and another from Diogenes Laertius (\r3.~24, p.~74
  ed.\ Cobet) to the effect that ``He [Plato] explained
  (\greek{εἰσηγήσατο}) to Leodamas of Thasos the method of inquiry by
  analysis ``have been commonly understood as ascribing to Plato the
  \emph{invention} of the method of analysis; but Tannery points out
  forcibly (pp.~112, 113) bow difficult it is to explain in what
  Plato's discovery could have consisted if \emph{analysis} be taken
  in the sense attributed to it in Pappus, where we can see no more
  than a series of successive \emph{reductions} of a problem until it
  is finally reduced to a known problem. On the other hand, Proclus'
  words about carrying up the thing sought to ``an acknowledged
  principle'' suggest that what he had in mind was the process
  described at the end of Book~\r6\ of the \emph{Republic} by which
  the dialectician (unlike the mathematician) uses hypotheses as
  stepping-stones up to a principle which is not hypothetical, and
  then is able to descend step by step verifying every one of the
  hypotheses by which he ascended. This description does not of course
  refer to mathematical analysis, but it may have given rise to the
  idea that analysis was Plato's discovery, since \emph{analysis} and
  \emph{synthesis} following each other are related in the same way as
  the upward and the downward progression in the dialectician's
  intellectual method. And it may be that Plato's achievement was to
  observe the importance, from the point of view of logical rigour, of
  the confirmatory synthesis following analysis, and to regularise in
  this way and elevate into a completely irrefragable method the
  partial and uncertain analysis upon which the works of his
  predecessors depended.

2 Here again the successive bipartitions of genera into species such
  as we find in the \emph{Sophist} and \emph{Republic} have very
  little to say to geometry, and the very fact that they are here
  mentioned side by side with analysis suggests that Proclus confused
  the latter with the philosophical method of \emph{Rep.}~\r6.

3 Tannery rightly remarks (p.~152) that the subdivision of a theorem
  or problem into several cases is foreign to the really classic form;
  the ancients preferred, where necessary, to multiply
  enunciations. As, however, some omissions necessarily occurred, the
  writers of lemmas naturally added separate \emph{cases}, which in
  some instances found their way into the text.  A good example is
  Euclid \prop{1}{7}, the second case of which, as it appears in our
  text-books, was interpolated. On the commentary of Proclus on this
  proposition Th.~Taylor rightly remarks that ``Euclid everywhere
  avoids a multitude of cases.''

4 Proclus, p.~212, 5–11.

5 Tannery notes however that, so far from distinguishing his
  corollaries from the conclusions of his propositions, Euclid inserts
  them before the closing words ``(being) what it was required to do''
  or ``to prove.'' In fact the porism-corollary is with Euclid rather
  a modified form of the regular conclusion than a separate
  proposition.

5. Objection.

``The objection (eWrao-i?) obstructs the whole course of the argu-
ment by appearing as an obstacle (or crying ' halt,' d-n-avrao-a) either
to the construction or to the demonstration. There is this difference
between the objection and the case, that, whereas he who propounds
the case has to prove the proposition to be true of it, he who makes
the objection does not need to prove anything: on the contrary it is
necessary to destroy the objection and to show that its author is
saying what is false'.''

That is, in general the objection endeavours to make it appear that
the demonstration is not true in every case; and it is then necessary
to prove, in refutation of the objection, either that the supposed case
is impossible, or that the demonstration is true even for that case. A
good instance is afforded by EucL \prop{1}{7}, The text- books give a second
case which is not in the original text of Euclid. Proclus remarks on
the proposition as given by Euclid that the objection may conceivably
be raised that what Euclid declares to be impossible may after all be
possible in the event of one pair of stiaight lines falling completely
within the other pair. Proclus then refutes the objection by proving
the impossibility in that case also. His proof then came to be given
in the text-books as part of Euclid's proposition.

The objection is one of the technical terms in Aristotle's logic and
its nature is explained in the Prior Analytics''. ``An objection is a
proposition contrary to a proposition.,.. Objections are of two sorts,
general or partial.... For when it is maintained that an attribute
belongs to every (member of a class), we object either that it belongs
to none (of the class) or that there is some one (member of the class)
to which it does not belong.''

6. Reduction.

This is again an Aristotelian term, explained in the Prior
Analytics*. It is well described by Proclus in the following passage:

``Reduction (diraywyq) is a transition from one problem or theorem
to another, the solution or proof of which makes that which is pro-
pounded manifest also. For example, after the doubling of the cube
had been investigated, they transformed the investigation into another
upon which it follows, namely the finding of the two means; and from
that time forward they inquired how between two given straight lines
two mean proportionals could be discovered. And Tney say that the
first to effect the reduction of difficult constructions was Hippocrates of
Chios, who also squared a lune and discovered many other things in
geometry, being second to none in ingenuity as regards constructions'.''

1 Proclus, p.~212, 18–23.

2 \emph{Anal.\ prior.}\ \r2.~26, 69~a~37.

3 \ibid~\r2.~25, 69~a~20.

4 Proclus, pp.~212, 24–213, \r2. This passage has frequently been
  taken as crediting Hippocrates with the discovery of the method of
  geometrical reduction: cf.\ Taylor (Translation of Proclus, \r2.~
  p.~26), Allman (p.~41~\emph{n.}, 59), Gow (pp.~169, 170), As Tannery
  remarks (p.~110), if the particular reduction of the duplication
  problem to that of the two means is the first noted in history, it
  is difficult to suppose that it was really the first; for
  Hippocrates must have found instances of it in the Pythagorean
  geometry. Bretschneider, I think, comes nearer the truth when he
  boldly (p.~99) translates: ``This reduction \emph{of the aforesaid
  construction} is said to have been first given by Hippocrates.''
  The words are \greek{πρῶτον δέ φασι τῶν ἀποπουμένων διαγραμμάτων τὴν
  ἀπαγωγὴν ποιήσασθαι}, which must, literally, be translated as in the
  text above; but, when Proclus speaks vaguely of ``difficult
  constructions,'' he probably means to say simply that ``this first
  recorded instance of a reduction of a difficult construction is
  attributed to Hippocrates.''

7. Reductio ad absurdum.

This is variously called by Aristotle ``reductio ad absurdum'' ft) «*s
rr) aZvvarav airaycoyi})', ``proof per impossible ``(17 Bia toS dZvvarov
Setfw or a-n-oetfcy, ``proof leading to the impossible ``(r) Wc to
rtSufaroy ayova-a airoBeti is) 1 . It is part of ``proof (starting) from a
hypothesis' ``(«'f tiTrofeo-ew?). ``All (syllogisms) which reach the
conclusion per impossibile reason out a conclusion which is false, and
they prove the original contention (by the method starting) from a
hypothesis, when something impossible results from assuming the
contradictory of the original contention, as, for example, when it is
proved that the diagonal (of a square) is incommensurable because,
if it be assumed commensurable, it will follow that odd (numbers)
are equal to even (numbers) 1 .'' Or again, ``proof (leading) to the
impossible differs from the direct (£ec*r(«ri}t) in that it assumes what
it desires to destroy [namely the hypothesis of the falsity of the
conclusion] and then reduces it to something admittedly false, whereas
the direct proof starts from premisses admittedly true'.''

Proclus has the following description of the reductio ad absurdum.
``Proofs by reductio ad absurdum in every case reach a conclusion
manifestly impossible, a conclusion the contradictory of which is
admitted. In some cases the conclusions are found to conflict with
the common notions, or the postulates, or the hypotheses (from which
we started); in others they contradict propositions previously estab-
lished 7 ``. ..''Every reductio ad absurdum assumes what conflicts with
the desired result, then, using that as a basis, proceeds until it arrives
at an admitted absurdity, and, by thus destroying the hypothesis,
establishes the result originally desired. For it is necessary to under-
stand generally that all mathematical arguments either proceed from
the first principles or lead back to them, as Porphyry somewhere says.
And those which proceed from the first principles are again of two
kinds, for they start either from common notions and the clearness of
the self-evident alone, or from results previously proved; while those
which lead back to the principles are either by way of assuming the
principles or by way of destroying them. Those which assume the
principles are called analyses, and the opposite of these are syntheses —
for it is possible to start from the said principles and to proceed in
the regular order to the desired conclusion, and this process is syn-
thesis — while the arguments which would destroy the principles are
called reductiones ad absurdum. For it is the function of this method
to upset something admitted as clear 1 .''

1 Aristotle, \greek{Anal.\ prior.}\ \r1.~7, 29~b~5; \r1.~44, 50~a~30.

2 \ibid~\r1. 21, 39~b~32; \r1.~29, 45~a~35.

3 \emph{Anal.\ post.}\ \r1.~24, 85~a~16 etc.

4 \emh{Anal.\ prior.}\ \r1.~23, 40~b~25.

5 \emh{Anal.\ prior.}\ \r1.~23, 41~a~24.

6 \ibid~\r2.~14, 62~b~29.

7 Proclus, p.~254, 22–27.



8. Analysis and Synthesis.

It will be seen from the note on Eucl. xin. i that the mss. of the
Elements contain definitions of Analysis and Synthesis followed by
alternative proofs of xill. i — 5 aftet that method. The definitions and
alternative proofs are interpolated, but they have great historical
interest because of the possibility that they represent an ancient
method of dealing with these propositions, anterior to Euclid. The
propositions give properties of a line cut ``in extreme and mean ratio,''
and they are preliminary to the construction and comparison of the
five regular solids. Now Pappus, in the section of his Collection dealing
with the latter subject'', says that he will give the comparisons between
the five figures, the pyramid, cube, octahedron, dodecahedron and
icosahedron, which have equal surfaces, ``not by means of the so-called
analytical inquiry, by which some of the ancients worked out the proofs,
but by the synthetical method 1 ....'' The conjecture of Bretschneider
that the matter interpolated in Eucl. XIII. is a survival of investiga-
tions due to Eudoxus has at first sight much to commend it 4 . In the
first place, we are told by Proclus that Eudoxus ``greatly added to
the number of the theorems which Plato originated regarding the
sectim,nd employed in them the method of analysis'.'' It is obvious
that ``the section ``was some particular section which by the time of
Plato had assumed great importance; and the one section of which
this can safely be said is that which was called the ``golden section,''
namely, the division of a straight line in extreme and mean ratio
which appears in Eucl. \prop{2}{1} 1 and is therefore most probably Pytha-
gorean. Secondly, as Cantor points out 8 , Eudoxus was the founder
of the theory of proportions in the form in which we find it in Euclid
v., VI., and it was no doubt through meeting, in the course of his
investigations, with proportions not expressible by whole numbers
that he came to realise the necessity for a new theory of proportions
which should be applicable to incommensurable as well as commen-
surable magnitudes. The ``golden section'' would furnish such a case.
And it is even mentioned by Proclus in this connexion. He is
explaining' that it is only in arithmetic that all quantities bear
``rational'' ratios (pyros Xoyos) to one another, while in geometry there
are ``irrational ``ones (appqros) as well. ``Theorems about sections
like those in Euclid's second Book are common to both [arithmetic
and geometry] except that in which the straight line is cut in extreme
and mean ratio''.''

1 Proclus, p.~255, 8–26.

2 Pappus, \r5.~p.~410 sqq.

3 \ibid~pp.~410, 27–412, 2.

* Bretschneider, p.~168. See however Heiberg's recent suggestion
  (\emph{Paralipomena zu Euklid} in \emph{Hermes}, \r38., 1903) that
  the author was Heron. The suggestion is based on a comparison with
  the remarks on analysis and synthesis quoted from Heron by
  an-Nairīzī (ed.\ Curtze, p.~89) at the beginning of his commentary
  on Eucl.\ Book~\r2.  On the whole, this suggestion commends itself
  to me more than that of Bretschneider.

5 Proclus, p.~67, 6.

6 Cantor, \emph{Gesch.\ d.\ Math.}\ \r\tsub{3}, p.~241.

7 Proclus, p, 60, 7–9.

8 \ibid~p.~60, 16–19.

The definitions of Analysis and Synt Justs interpolated in Eucl.
XIII. are as follows (I adopt the reading of B and V, the only in-
telligible one, for the second).

``Analysis is an assumption of that which is sought as if it were
admitted < and the passage > through its consequences to something
admitted (to be) true.

``Synthesis is an assumption of that which is admitted < and the
passage > through its consequences to the finishing or attainment of
what is sought.''

The language is by no means clear and has, at the best, to be
filled out.

Pappus has a fuller account 1:

``The so-called ivakvfMPW (' Treasury of Analysis ``) is, to put it
shortly, a special body of doctrine provided for the use of those who,
after finishing the ordinary Elements, are desirous of acquiring the
power of solving problems which may be set them involving (the
construction of) lines, and it is useful for this alone. It is the work
of three men, Euclid the author of the Elements, Apollonius of Perga,
and Aristaeus the elder, and proceeds by way of analysis and synthesis.

``Analysis then takes that which is sought as if it were admitted
and passes from it through its successive consequences to something
which is admitted as the result of synthesis: for in analysis we assume
that which is sought as if it were (already) done (yeyovos), and we
inquire what it is from which this results, and again what is the ante-
cedent cause of the latter, and so on, until by so retracing our steps
we come upon something already known or belonging to the class of
first principles, and such a method we call analysis as being solution

backwards (deaira\tv \vtrtr).

``But in synthesis, reversing the process, we take as already done
that which was last arrived at in the analysis and, by arranging in
their natural order as consequences what were before antecedents,
and successively connecting them one with another, we arrive finally
at the construction of what was sought; and this we call synthesis.

``Now analysis is of two kinds, the one directed to searching for
the truth and called theoretical, the other directed to finding what we
are told to find and called problematical, (1) In the theoretical kind
we assume what is sought as if it were existent and true, after which
we pass through its successive consequences, as if they too were true
and established by virtue of our hypothesis, to something admitted:
then (a), if that something admitted is true, that which is sought will
also be true and the proof will correspond in the reverse order to the
analysis, but (d), if we come upon something admittedly false, that
which is sought will also be false. (2) In the problematical kind we
assume that which is propounded as if it were known, after which we
pass through its successive consequences, taking them as true, up to
something admitted: if then (a) what is admitted is possible and
obtainable, that is, what mathematicians call given, what was originally
proposed will also be possible, and the proof will again correspond in

1 Pappus, \r7.~pp.~634–6.

reverse order to the analysis, but if {) we come upon something
admittedly impossible, the problem will also be impossible.''

The ancient Analysis has been made the subject of careful studies
by several writers during the last half-century, the most complete
being those of Hankel, Duhamel and Zeuthen.; others by Ofterdinger
and Cantor should also be mentioned'.

The method is as follows. It is required, let us say, to prove that
a certain proposition A is true. We assume as a hypothesis that A
is true and, starting from this we find that, if A is true, a certain
other proposition B is true; if B is true, then C; and so on until
we arrive at a proposition K which is admittedly true. The object
of the method is to enable us to infer, in the reverse order, that, since
K is true, the proposition A originally assumed is true. Now
Aristotle had already made it clear that false hypotheses might lead
to a conclusion which is true. There is therefore a possibility of error
unless a certain precaution is taken. While, for example, B may be a
necessary consequence of A, it may happen that A is not a necessary
consequence of B, Thus, in order that the reverse inference from the
truth of K that A is true may be logically justified, it is necessary
that each step in the chain of inferences should be unconditionally
convertible. As a matter of fact, a very large number of theorems in
elementary geometry are unconditionally convertible, so that in practice
the difficulty in securing that the successive steps shall be convertible
is not so great as might be supposed. But care is always necessary.
For example, as Hankel says 1 , a proposition may not be uncon-
ditionally convertible in the form in which it is generally quoted.
Thus the proposition ``The vertices of all triangles having a common
base and constant vertical angle lie on a circle ``cannot be converted
into the proposition that ``All triangles with common base and vertices
lying on a circle have a constant vertical angle''; for this is only true
if the further conditions are satisfied (]) that the circle passes through
the extremities of the common base and {2) that only that part of the
circle is taken as the locus of the vertices which lies on one side of the
base. If these conditions are added, the proposition is unconditionally
convertible. Or again, as Zeuthen remarks 3 , K may be obtained by
a series of inferences in which A or some other proposition in the
series is only apparently used; this would be the case e.g.\ when the
method of modem algebra is being employed and the expressions on
each side of the sign of equality have been inadvertently multiplied
by some composite magnitude which is in reality equal to zero.

Although the above extract from Pappus does not make it clear
that each step in the chain of argument must be convertible in the
case taken, he almost implies this in the second part of the definition
jf Analysis where, instead of speaking of the consequences B, C...

1  Hankel, \emph{Zur Geschichte der Mathematik in Alterthum und
  Mittelalter}, 1874, pp.~137–150;
   Duhmamel, \emph{Des méthodes dans les sciences de raisonnement},
  Part~\r1, 3~ed., Paris, 1885, pp.~39–68;
   Zeuthen, \emph{Geschichte der Mathematik im Altertum und
  Mittelalter}, 1896, pp.~92–104;
   Ofterdinger, \emph{Beiträge zur Geschichte der griechischen
  Mathematik}, Ulm, 1860;
   Cantor, \emph{Geschichte der Mathematik}, \r1\tsub{3}, pp.~220–2.

1 Hankel, p.~139.

3 Zeuthen, p.~103.

successively following from A, he suddenly changes the expression
and says that we inquire what it is (B)frem which A follows (A being
thus the consequence of B, instead of the reverse), and then what
(viz.\ C) is the antecedent cause of B; and in practice the Greeks
secured what was wanted by always insisting on the analysis being
confirmed by subsequent synthesis, that is, they laboriously worked
backwards the whole way from K. to A, reversing the order of the
analysis, which process would undoubtedly bring to light any flaw
which had crept into the argument through the accidental neglect of
the necessary precautions.

Reductio ad absurdum a variety of analysis.

In the process of analysis starting from the hypothesis that a
proposition A is true and passing through B, C... as successive con-
sequences we may arrive at a proposition K which, instead of being
admittedly true, is either admittedly false or the contradictory of the
original hypothesis A or of some one or more of the propositions B, C...
intermediate between A and K. Now correct inference from a true
proposition cannot lead to a false proposition; and in this case there-
fore we may at once conclude, without any inquiry whether the
various steps in the argument are convertible or not, that the hypo-
thesis A is false, for, if it were true, all the consequences correctly
inferred from it would be true and no incompatibility could arise.
This method of proving that a given hypothesis is false furnishes an
indirect method of proving that a given hypothesis A is true, since we
have only to take the contradictory of A and to prove that it is false.
This is the method of reductio ad absurdum, which is therefore a variety
of analysis. The contradictory of A, or not-A, will generally include
more t han one case and, in order to prove its falsity, each of the cases
must be separately disposed of: e.g., if it is desired to prove that a
certain part of a figure is equal to some other part, we take separately
the hypotheses (i) that it is greater, (2) that it is less, and prove
that each of these hypotheses leads to a conclusion either admittedly
false or contradictory to the hypothesis itself or to some one of its
consequences.

Analysis as applied to problems.

It is in relation to problems that the ancient analysis has the
greatest significance, because it was the one general method which
the Greeks used for solving all ``the more abstruse problems'' (ra
dera<f>G(TTepa twv TrpopKufiaTwvy '.

We have, let us suppose, to construct a figure satisfying a certain
set of conditions If we are to proceed at all methodically and not
by mere guesswork, it is first necessary to ``analyse'' those conditions.
To enable this to be done we must get them clearly in our minds,
which is only possible by assuming all the conditions to be actually
fulfilled, in other words, by supposing the problem solved. Then v;e
have to transform those conditions, by all the means which practice in
such cases has taught us to employ, into other conditions which are
necessarily fulfilled if the original conditions are, and to continue this

1 Proclus, p.~242, 16, 17.

transformation until we at length arrive at conditions which we
are in a position to satisfy 1 . In other words, we must arrive at
some relation which enables us to construct a particular part of
the figure which, it is true, has been hypothetically assumed and
even drawn, but which nevertheless really requires to be found in
order that the problem may be solved. From that moment the
particular part of the figure becomes one of the data, and a fresh
relation has to be found which enables a fresh part of the figure
to be determined by means of the original data and the new one
together. When this is done, the second new part of the figure also
belongs to the data; and we proceed in this way until all the parts
of the required figure are found 1 . The first part of the analysis
down to the point of discovery of a relation which enables
us to say that a certain new part of the figure not belonging
to the original data is given, Hankel calls the transformation; the
second part, in which it is proved that all the remaining parts of
the figure are ``given,'' he calls the resolution. Then follows the
synthesis, which also consists of two parts, (1) the construction, in
the order in which it has to be actually carried out, and in general
following the course of the second part of the analysis, the resolution;
(2) the demonstration that the figure obtained does satisfy all the given
conditions, which follows the steps of the first part of the analysis,
the transformation, but in the reverse order. The second part of
the analysis, the resolution, would be much facilitated and shortened
by the existence of a systematic collection of Data such as Euclid's
book bearing that title, consisting of propositions proving that, if
in a figure certain parts or relations me. given, other parts or relations
are also given. As regards the first part of the analysis, the trans-
formation, the usual rule applies that every step in the chain must
be unconditionally convertible; and any failure to observe this
condition will be brought to light by the subsequent synthesis.
The second part, the resolution, can be directly turned into the
construction since that only is given which can be constructed by
the means provided in the Elements.

It would be difficult to find a better illustration of the above than
the example chosen by Hankel from Pappus*.

Given a circle ABC and two points D, E external to it, to draw
straight lines DB, KB from D, E to a point B on the circle such that,
if DB, KB produced meet the circle again in C, A, AC shall be parallel
A>DE.

Analysis.

Suppose the problem solved and the tangent at A drawn, meeting
ED produced in F.

(Part I. Transformation.)

Then, since AC is parallel to DE, the angle at C is equal to the
angle CDE.

But, since FA is a tangent, the angle at C is equal to the angle FAE.

Therefore the angle FAE is equal to the angle CDE, whence A,
B, D, Fact concyclic.

1 Zeuthen, p.~93.

2 Hankel, p.~141.

3 Pappus, \r7.~pp.~830–2.


Therefore the rectangle AE, EB is equal to the rectangle FE,
ED.

(Part II, Resolution.)

But the rectangle AE, EB is given,
because it is equal to the square on the
tangent from E.

Therefore the rectangle FE, ED is
given;

and, since ED is given, FE is given (in
length). [Data, 57.J

But FE is given in position also, so
that F is also given. [Data, 27.]

Now FA is the tangent from a given point F to a circle A BC
given in position;
therefore FA is given in position and magnitude. [Data, 90.]

And F is given; therefore A is given.

But £ is also given; therefore the straight line AE is given in
position, [Data, 26.]

And the circle ABC is given in position;
therefore the point B is also given. [Data, 25.]

But the points D, E are also given;
therefore the straight lines DB, BE are also given in position.

Synthesis.

(Part I. Construction)

Suppose the circle ABC and the points D, E given.

Take a rectangle contained by ED and by a certain straight
line EF equal to the square on the tangent to the circle from E.

From F draw FA touching the circle in A; join ABE and then
DB, producing DB to meet the circle at C. Join AC.

1 say then that A C is parallel to DE.

(Part II. Demonstration.)

Since, by hypothesis, the rectangle FE, ED is equal to the square
on the tangent from E, which again is equal to the rectangle AE, EB,
the rectangle AE, EB is equal to the rectangle FE, ED,

Therefore A, B, D, F are concyclic,
whence the angle FAE is equal to the angle BDE.

But the angle FAE is equal to the angle ACB in the alternate
segment;
therefore the angle A CB is equal to the angle BDE.

Therefore AC Is parallel to DEk

In cases where a iopt<rft6<; is necessary, i.e.\ where a solution is
only possible under certain conditions, the analysis will enable those
conditions to be ascertained. Sometimes the Stopw/ioe is stated and
proved at the end of the analysis, e.g.\ in Archimedes, On the Sphere
and Cylinder, \prop{2}{7}; sometimes it is stated in that place and the proof
postponed till after the end of the synthesis, e.g.\ in the solution of
the problem subsidiary to Oh the Sphere and Cylinder, \prop{2}{4}, preserved
in Eutocius' commentary on that proposition. The analysis should
also enable us to determine the number of solutions of which the
problem is susceptible.

\section{The Definitions}

General. ``Real'' and ``Nominal'' Definitions.

It is necessary, says Aristotlej whenever any one treats of any
whole subject, to divide the genus into its primary constituents, those
which are indivisible in species respectively: e.g.\ number must be
divided into triad and dyad; then an attempt must be made in this
way to obtain definitions, e.g.\ of a straight line, of a circle, and of
a right angle 1 .

The word for definition is 3po?. The original meaning of this
word seems to have been ``boundary,'' ``landmark.'' Then we have
it in Plato and Aristotle in the sense of standard or determining
principle (``id quo alicuius rei natura constituitur vel definitur,''
Index Aristotelicus) 1; and closely connected with this is the sense of
definition. Aristotle uses both JSpo? and 6pnrfia<t for definition, the
former occurring more frequently in the Topics, the latter in the
Metaphysics.

Let us now first be clear as to what a definition does not do.
There is nothing in connexion with definitions which Aristotle takes
more pains to emphasise than that a definition asserts nothing as to
the existence or non-existence of the thing defined. It is an answer
to the question what a thing is (ri 4cm), and does not say that it
is (5rt cart). The existence of the various things defined has to be
proved, except in the case of a few primary things in each science,
the existence of which is indemonstrable and must be assumed among
the first principles of each science; e.g.\ points and lines in geometry
must be assumed to exist, but the existence of everything else must
be proved This is stated clearly in the long passage quoted above
under First Principles'. It is reasserted in such passages as the
following. ``The (answer to the question) what is a man and the
fact that a man exists are different things'.'' ``It is clear that, even
according to the view of definitions now current, those who define
things do not prove that they exist 1 .'' ``We say that it is by
demonstration that we must show that everything exists, except
essence (tl fii) ova-la eiij). But the existence of a thing is never
essence; for the existent is not a genus. Therefore there must be
demonstration that a thing exists. Thus, what is meant by triangle
the geometer assumes, bat that it exists he has to prove 11 .'' ``Anterior
knowledge of two sorts is necessary: for it is necessary to presuppose,
with regard to some things, that they exist; in other cases it is
necessary to understand what the thing described is, and in other
cases it is necessary to do both. Thus, with the fact that one of two
contradictories must be true, we must know that it exists (is true);

1 \emph{Anal. post.}\ \r2.\ 13, 96~b~15.

2 cf.\ \emph{De anima}, \r1.~1, 404~a~9, where ``breathing'' is spoken
  of as the \greek{ὅρος} of ``life,'' and the many passages in the
  \emph{Politics} where the word is used to denote that which gives
  its \emph{special character} to the several forms of government
  (virtue being the \greek{ὄρος} of aristocracy, wealth of oligarchy,
  liberty of democracy, 1294~a~10), Plato, \emph{Republic},
  \r8,~551~\textsc{c}.

3 \emph{Anal.\ post.}\ \r1.~10, 76~a~31 sqq.

4 \ibid~\r2.~7, 92~b~10.

5 \ibid~92~b~19.

6 \ibid~92~b~12 sqq.

of the triangle we must know that it means such and such a thing; of
the unit we must know both what it means and that it exists 1 .'' What
is here so much insisted on is the very fact which Mill pointed out
in his discussion of earlier views of Definitions, where he says that
the so-called real definitions or definitions of things do not constitute
a different kind of definition from nominal definitions, or definitions
of names; the former is simply the latter plus something else, namely
a covert assertion that the thing defined exists, ``This covert assertion
is not a definition but a postulate. The definition is a mere identical
proposition which gives information only about the use of language,
and from which no conclusion affecting matters of fact can possibly
be drawn. The accompanying postulate, on the other hand, affirms
a fact which may lead to consequences of every degree of importance.
It affirms the actual or possible existence of Things possessing the
combination of attributes set forth in the definition: and this, if true,
may be foundation sufficient on which to build a whole fabric of
scientific truth 1 .'' This statement really adds nothing to Aristotle's
doctrine'; it has even the slight disadvantage, due to the use of
the word ``postulate'' to describe ``the covert assertion'' in all cases,
of not definitely pointing out that there are cases where existence
has to be proved as distinct from those where it must be assumed.
It is true that the existence of a definiend may have to be taken
for granted provisionally until the time comes for proving it; but,
so far as regards any case where existence must be proved sooner
or later, the provisional assumption would be for Aristotle, not a
postulate, but a hypothesis. In modern times, too, Mill's account of
the true distinction between real and nominal definitions had been
fully anticipated by Saccheri 1 , the editor of Euclides ab omni naevo
vindicatas (1733), famous in the history of non-Euclidean geometry.
In his Logica Demonstrativa (to which he also refers in his Euclid)
Saccheri lays down the clear distinction between what he calls de-
finitiones quid nominis or nominales, and definitiones quid rei or reales,
namely that the former are only intended to explain the meaning

1 \emph{Anal.\ post.}\ \r1.~1, 71~a~11 sqq.

2 Mill's \emph{System of Logic}, Bk.~\r1.~ch.~viii.

* It is true that it was in opposition to ``the ideas of most of the
  \emph{Aristotelian logicians}'' (rather than of Aristotle himself)
  that Mill laid such stress on his point of view.  Cf.\ his
  observation: ``We have already made, and shall often have to repeat,
  the remark, that the philosophers who overthrew Realism by no means
  got rid of the consequences of Realism, but retained long
  afterwards, in their own philosophy, numerous propositions which
  could only have a rational meaning as part of a Realistic system. It
  had been handed down from Aristotle, and probably from earlier
  times, as an obvious truth, that the science of geometry is deduced
  from definitions. This, so long as a definition was considered to be
  a proposition `unfolding the nature of the thing,' did well enough.
  But Hobbes followed and rejected utterly the notion that a
  definition declares the nature of the thing, or does anything but
  state the meaning of a name; yet he continued to affirm as broadly
  as any of his predecessors that the \greek{ἀρχαί}, \emph{principia},
  or original premisses of mathematics, and even of all science, are
  definitions; producing the singular paradox that systems of
  scientific truth, nay, all truths whatever at which we arrive by
  reasoning, are deduced from the arbitrary conventions of mankind
  concerning the signification of words.'' Aristotle was guilty of no
  such paradox; on the contrary, he exposed it as plainly as did Mill.

* This has been fully brought out in two papers by G.~Vailati,
  \emph{La teoria Aristotelica della definizione} (\emph{Rivista di
  Filosofia e scienze affini}. 1903}, and \emph{Di un' opera
  dimenticata del P.~Saccheri} (``Logica Demonstrativa,'' 1697) (in
  \emph{Rivista Filosofica}, 1903).

that is to be attached to a given term, whereas the latter, besides
declaring the meaning of a word, affirm at the same time the existence
of the thing defined or, in geometry, the possibility of constructing it.
The definitio quid nominis becomes a definitio quid rei ``by means of a
postulate, or when we come to the question whether the thing exists and
it is answered affirmatively 1 .'' Definitiones quid nominis are in them-
selves quite arbitrary, and neither require nor are capable of proof;
they are merely provisional and are only intended to be turned as
quickly as possible into definitiones quid rei, either (1) by means of
a postulate in which it is asserted or conceded that what is defined
exists or can be constructed, e.g.\ in the case of straight lines and
circles, to which Euclid's first three postulates refer, or (2) by
means of a demonstration reducing the construction of the figure
defined to the successive carrying-out of a certain number of those
elementary constructions, the possibility of which is postulated. Thus
definitiones quid rei are in general obtained as the result of a series of
demonstrations. Saccheri gives as an instance the construction of a
square in Euclid \prop{1}{46}. Suppose that it is objected that Euclid had
no right to define a square, as he does at the beginning of the Book,
when it was not certain that such a figure exists in nature; the
objection, he says, could only have force if, before proving and making
the construction, Euclid had assumed the aforesaid figure as given.
That Euclid is not guilty of this error is clear from the fact that
he never presupposes the existence of the square as defined until
after \prop{1}{46}.

Confusion between the nominal and the real definition as thus de-
scribed, i.e.\ the use of the former in demonstration before it has been
turned into the latter by the necessary proof that the thing defined
exists, is according to Saccheri one of the most fruitful sources of
illusory demonstration, and the danger is greater in proportion to
the ``complexity'' of the definition, i.e.\ the number and variety of
the attributes belonging to the thing defined. For the greater is the
possibility that there may be among the attributes some that are
incompatible, i.e.\ the simultaneous presence of which in a given figure
can be proved, by means of other postulates etc.\ forming part of the
basis of the science, to be impossible.

The same thought is expressed by Leibniz also. ``If,'' he says,
``we give any definition, and it is not clear from it that the idea, which
we ascribe to the thing, is possible, we cannot rely upon the demon-
strations which we have derived from that definition, because, if that
idea by chance involves a contradiction, it is possible that even con-
tradictories may be true of it at one and the same time, and thus our
demonstrations will be useless. Whence it is clear that definitions
are not arbitrary. And this is a secret which is hardly sufficiently
known'.'' Leibniz' favourite illustration was the ``regular polyhedron
with ten faces,'' the impossibility of which is not obvious at first sight.

1 ``Definitio \emph{quid nominis} nata est evadere definitio
  \emph{quid rei} per \emph{postulatum} vel dum venitur ad quaestionem
  \emph{an est} et respondetur affirmative.''

2. \emph{Opuscules et fragments inédits de Leibniz}, Paris, Alcan,
  1903, p.~431.  Quoted by Vailati.

It need hardly be added that, speaking generally, Euclid's defini-
tions, and his use of them, agree with the doctrine of Aristotle
that the definitions themselves say nothing as to the existence of the
things defined, but that the existence of each of them must be
proved or (in the case of the ``principles ``) assumed. In geometry,
says Aristotle, the existence of points and lines only must be as-
sumed, the existence of the rest being proved. Accordingly Euclid's
first three postulates declare the possibility of constructing straight
lines and circles (the only ``lines ``except straight lines used in the
Elements). Other things are defined and afterwards constructed and
proved to exist: e.g.\ in Book I., Def. 20, it is explained what is meant
by an equilateral triangle; then (1. 1 ) it is proposed to construct it,
and, when constructed, it is proved to agree with the definition.
When a square is defined (1. Def. 22), the question whether such a
thing really exists is left open until, in \prop{1}{46}, it is proposed to construct
it and, when constructed, it is proved to satisfy the definition 1 .
Similarly with the right angle (I. Def. 10, and \prop{1}{11}) and parallels
(I. Def. 23, and \prop{1}{27}—29). The greatest care is taken to exclude
mere presumption and imagination. The transition from the sub-
jective definition of names to the objective definition of things is
made, in geometry, by means of constructions (the first principles of
which are postulated), as in other sciences it is made by means of
experience 1 .

Aristotle's requirements in a definition.

We now come to the positive characteristics by which, according
to Aristotle, scientific definitions must be marked.

First, the different attributes in a definition, when taken separately,
cover more than the notion defined, but the combination of them
does not Aristotle illustrates this by the ``triad,'' into which enter
the several notions of number, odd and prime, and the last ``in both
its two senses (a) of not being measured by any (other) number (<as
firj fiiTpdadai dpiBfUii) and (£) of not being obtainable by adding
numbers together ``(<uv pA auytceiaBat, e£ aptQfiv), a unit not being a
number. Of these attributes some are present in all other odd
numbers as well, while the last [primeness in the second sense]
belongs also to the dyad, but in nothing but the triad are they all
present'.'' The fact can be equally well illustrated from geometry.
Thus, e.g.\ into the definition of a square (Eucl. I., Def. 22) there enter
the several notions of figure, four-sided, equilateral, and right-angled,
each of which covers more than the notion into which all enter as
attributes 4 .

Secondly, a definition must be expressed in terms of things which
are prior to, and better known than, the things defined'. This is

1 Trendelenburg, \emph{Elementa Logices Aristoteleae}, \S50,

2 Trendelenburg, \emph{Erläuterungen zu den Elementen der
  aristotelischen Logik}, 3~ed.\ p.~107.  On construction as proof of
  existence in ancient geometry cf.\ H.~G. Zeuthen, \emph{Die
  geometrische Construction als ``Existenzbeweis'' in der antiken
  Geometrie} (in \emph{Mathematische Annalen}, 4}. Band).

3 \emph{Anal.\ post.}\ \r2.~13, 96~a~33–b~1.

4 Trendelenburg, \emph{Erläuterungen}, p.~108.

5 \emph{Topics} \r6.~4, 141~a~26 sqq.

clear, since the object of a definition is to give us knowledge of the
thing defined, and it is by means of things prior and better known
that we acquire fresh knowledge, as in the course of demonstrations.
But the terms ``prior ``and ``better known ``are, as usual susceptible
of two meanings; they may mean (1 ) absolutely or logically prior and
better known, or (2) better known relatively to us. In the absolute
sense, or from the standpoint of reason, a point is better known than
a line, a line than a plane, and a plane than a solid, as also a unit is
better known than number (for the unit is prior to, and the first
principle of, any number). Similarly, in the absolute sense, a letter is
prior to a syllable. But the case is sometimes different relatively to
us; for example, a solid is more easily realised by the senses than a
plane, a plane than a line, and a line than a point. Hence, while it is
more scientific to begin with the absolutely prior, it may, perhaps, be
permissible, in case the learner is not capable of following the scientific
order, to explain things by means of what is more intelligible to him.
``Among the definitions framed on this principle are those of the
point, the line and the plane; all these explain what is prior by
means of what is posterior, for the point is described as the extremity
of a line, the line of a plane, the plane of a solid.'' But, if it is asserted
that such definitions by means of things which are more intelligible
relatively only to a particular individual are really definitions, it will
follow that there may be many definitions of the same thing, one for
each individual for whom a thing is being defined, and even different
definitions for one and the same individual at different times, since at
first sensible objects are more intelligible, while to a better trained
mind they become less so. It follows therefore that a thing should
be defined by means of the absolutely prior and not the relatively
prior, in order that there may be one sole and immutable definition.
This is further enforced by reference to the requirement that a good
definition must state the genus and the differentiae, for these are
among the things which are, in the absolute sense, better known than,
and prior to, the species (twv AttXs yvwptfMoTipwv teal irporpav tov
etBovs tffrtv). For to destroy the genus and the differentia .is to
destroy the species, so that the former are prior to the species; they
are also better known, for, when the species is known, the genus and
the differentia must necessarily be known also, e.g.\ he who knows
``man'' must also know ``animal'' and ``land-animal,'' but it does not
follow, when the genus and differentia are known, that the species is
known too, and hence the species is Jess known than they are 1 . It
may be frankly admitted that the scientific definition will require
superior mental powers for its apprehension; and the extent of its
use must be a matter of discretion. So far Aristotle; and we have
here the best possible explanation why Euclid supplemented his
definition of a point by the statement in 1. Def, 3 that the extremities of
a line are points and his definition of a surface by I. Def. 6 to the effect
that the extremities of a surface are lines. The supplementary expla-

1 \emph{Topics} \r6.~4, 141~b~25–34,

nations do in fact enable us to arrive at a better understanding of the
formal definitions of a point and a line respectively, as is well ex-
plained by Simson in his note on Def. I, Simson says, namely, that
we must consider a solid, that is, a magnitude which has length,
breadth and thickness, in order to understand aright the definitions of
a point, a line and a surface. Consider, for instance, the boundary
common to two solids which are contiguous or the boundary which
divides one solid into two contiguous parts; this boundary is a surface.
We can prove that it has no thickness by taking away either solid,
when it remains the boundary of the other; for, if it had thickness, the
thickness must either be a part of one solid or of the other, in which
case to take away one or other solid would take away the thickness
and therefore the boundary itself: which is impossible. Therefore
the boundary or the surface has no thickness. In exactly the same
way, regarding a line as the boundary of two contiguous surfaces, we
prove that the line has no breadth; and, lastly, regarding a point as
the common boundary or extremity of two lines, we prove that a
point has no length, breadth or thickness.

Aristotle on unscientific definitions.

Aristotle distinguishes three kinds of definition which are un-
scientific because founded on what is not prior (jtfy *W -n-poTipav). The
first is a definition of a thing by means of its opposite, e,g. of ``good ``
by means of ``bad ``; this is wrong because opposites are naturally
evolved together, and the knowledge of opposites is not uncommonly
regarded as one and the same, so that one of the two opposites
cannot be better known than the other. It is true that, in some
cases of opposites, it would appear that no other sort of definition is
possible: e.g.\ it would seem impossible to define double apart from the
half and, generally, this would be the case with things which in their
very nature {/cad* aura) are relative terms (w/109 Tt \eyerai), since one
cannot be known without the other, so that in the notion of either the
other must be comprised as well 1 . The second kind of definition
which is based on what is not prior is that in which there is a
complete circle through the unconscious use in the definition itself of
the notion to be defined though not of the name*. Trendelenburg
illustrates this by two current definitions, (i) that of magnitude as
that which can be increased or diminished, which is bad because the
positive and negative comparatives ``more ``and ``less ``presuppose
the notion of the positive ``great,'' (2) the famous Euclidean definition
of a straight line as that which ``lies evenly with the points on itself''
(££ Xaov to*s e<f>' lavTrjii <rijeiois Keirai), where ``lies evenly'' can only
be understood with the aid of the very notion of a straight line which is
to be defined'. The third kind of vicious definition from that which
is not prior is the definition of one of two coordinate species by means
of its coordinate (dirtiStypiiftivov), e.g.\ a definition of ``odd ``as that
which exceeds the even by a unit (the second alternative in Eucl. VII.
Def. 7); for ``odd ``and ``even ``are coordinates, being differentiae of
number'. This third kind is similar to the first. Thus, says Tren-
delenburg, it would be wrong to define a square as ``a rectangle
with equal sides.''

1 \emph{Topics} \r6.~4, 142~a~22–31.

2 \ibid~142~a~34–b~6.

3 Trendelenburg, \emph{Erläuterungen}, p.~115.

Aristotle's third requirement.

A third general observation of Aristotle which is specially relevant
to geometrical definitions is that ``to know what a thing is (ri itrTtv) is
the same as knowing why it is (foil ri ia-nv)'.'' ``What is an eclipse ?
A deprivation of light from the moon through the interposition of the
earth. Why does an eclipse take place? Or why is the moon
eclipsed ? Because the light fails through the earth obstructing it
What is harmony ? A ratio of numbers in high or low pitch. Why
does the high-pitched harmonise with the low-pitched? Because
the high and the low have a numerical ratio to one another*.'' ``We
seek the cause (ri Sto-rt) when we are already in possession of the
/act (rd ot(). Sometimes they both become evident at the same time,
but at al! events the cause cannot possibly be known [as a cause]
before the fact is known*.'' ``It is impossible to know what a thing is
if we do not know that it is'``Trendelenburg paraphrases: ``The
definition of the notion does not fulfil its purpose until it is made
genetic. It is the producing cause which first reveals the essence of

the thing The nominal definitions of geometry have only a

provisional significance and are superseded as soon as they are made
genetic by means of construction.'' e.g.\ the genetic definition of a
parallelogram is evolved from Eucl. \prop{1}{3} 1 (giving the construction for
parallels) and I, 33 about the lines joining corresponding ends of two
straight lines parallel and equal in length. Where existence is proved
by construction, the cause and the fact appear together 1 .

Again, ``it is not enough that the defining statement should set
forth the fact, as most definitions do; it should also contain and
present the cause; whereas in practice what is stated in the definition
is usually no more than a conclusion (o-vfiiripatrfta). For example,
what is quadrature ? The construction of an equilateral right-angled
figure equal to an oblong. But such a definition expresses merely the
conclusion. Whereas, if you say that quadrature is the discovery of a
mean proportional, then you state the reason'.'' This is better under-
stood if we compare the statement elsewhere that ``the cause is the
middle term, and this is what is sought in all cases','' and the illustra-
tion of this by the case of the proposition that the angle in a semi-
circle is a right angle. Here the middle term which it is sought to
establish by means of the figure is that the angle in the semi-circle is
equal to the half of two right angles. We have then the syllogism:
Whatever is half of two right angles is a right angle; the angle in a
semi-circle is the half of two right angles; therefore {conclusion) the
angle in a semi-circle is a right angle*. As with the demonstration, so

1 \emph{Topics}, \r6.~4, 142~b~7–10.

2 \emph{Anal.\ post.}\ \r2.~2, 90~a~31.

3 \emph{Anal.\ post.}\ \r2.~2, 90~a~15–21.

4 \ibid~\r2.~8, 93~a~17.

5 \ibid~93~a~10. 

6 Trendelenburg, \emph{Erläuterungen}, p.~110.

7 \emph{De anima} \r2.~2, 413~a~13–20.

8 \emph{Anal.\ post.}\ \r2.~2, 90~a~6.

9 \ibid~\r2.~11, 94~a~28.

it should be with the definition, A definition which is to show the
genesis of the thing defined should contain the middle term or cause;
otherwise it is a mere statement of a conclusion. Consider, for
instance, the definition of ``quadrature ``as ``making a square equal in
area to a rectangle with unequal sides,'' This gives no hint as to
whether a solution of the problem is possible or how it is solved: but,
if you add that to find the mean proportional between two given
straight lines gives another straight line such that the square on it is
equal to the rectangle contained by the first two straight lines, you
supply the necessary middle term or cause 1 .

Technical term a not defined by Euclid.

It will be observed that what is here defined, ``quadrature ``or
``squaring ``(TeTpaiyawio-oc), is not a geometrical figure, or an attribute
of such a figure or a part of a figure, but a technical term used to
describe a certain problem. Euclid does not define such things; but
the fact that Aristotle alludes to this particular definition as well as to
definitions of deflection (tce/ckd-a-Oat) and of verging {vevew) seems to
show that earlier text-books included among definitions explanations
of a number of technical terms, and that Euclid deliberately omitted
these explanations from his Elements as surplusage. Later the
tendency was again in the opposite direction, as we see from the much
expanded Definitions of Heron, which, for example, actually include
a definition of a deflected line (KettXaajUw) ypa/i,)'. Euclid uses the
passive of icTJiv occasionally*, but evidently considered it unnecessary
to explain such terms, which had come to bear a recognised meaning.

The mention too by Aristotle of a definition of verging (vcveiv)
suggests that the problems indicated by this term were not excluded
from elementary text-books before Euclid. The type of problem
(vev<ri<t) was that of placing a straight line across two lines, e.g.\ two
straight lines, or a straight line and a circle, so that it shall verge to a
given point (i.e.\ pass through -it if produced) and at the same time the
intercept on it made by the two given lines shall be of given length.

1 Other passages in Aristotle may be quoted to the like effect:
  e.g.\ \emph{Anal.\ post.}\ \r1.~2, 71~b~9 ``We consider that we know
  a particular thing in the absolute sense, as distinct from the
  sophistical and incidental sense, when we consider that we know the
  cause on account of which the thing is, in the sense of knowing that
  it is the cause of that thing and that it cannot be otherwise,''
  \ibid~\r1.~13, 79~a~2 ``For here to know the \emph{fact} is the
  function of those who are concerned with sensible things, to know
  the \emph{cause} is the function of the mathematician; it is he who
  possesses the proofs of the causes, and often he does not know the
  fact.'' In view of such passages it is difficult to see how Proclus
  came to write (p.~202, 11) that Aristotle was the originator
  (\greek{Ἀριστοτέλους κατάρξαντος}) of the idea of Amphinomus and
  others that geometry does not investigate the cause and the
  \emph{why} (\greek{τὸ διὰ τί}}. To this Geminus replied that the
  investigation of the cause does, on the contrary, appear in
  geometry. ``For how can it be maintained that it is not the business
  of the geometer to inquire for what reason, on the one hand, an
  infinite number of equilateral polygons are inscribed in a circle,
  but, on the other hand, it is not possible to inscribe in a sphere
  an infinite number of polyhedral figures, equilateral, equiangular,
  and made up of similar plane figures ?  Whose business is it to ask
  this question and find the answer to it if it is not that of the
  geometer? Now when geometers reason \emph{per impossibile} they are
  content to discover the property, but when they argue by direct
  proof, if such proof be only partial (\greek{ἐπὶ μέρους}), this does
  not suffice for showing the cause; if however it is general and
  applies to all like cases, the why (\greek{τὸ διὰ τί}) is at once
  and concurrently made evident.''

1 Heron, Def.~12 (vol.~\r4.\ Heib.\ pp.~22-24).

3 e.g.\ in \r2.~20 and in \emph{Data} 89,

In genera!, the use of conics is required for the theoretical solution of
these problems, or a mechanical contrivance for their practical
solution 1 . Zeuthen, following Oppermann, gives reasons for supposing,
not only that mechanical constructions were practically used by the
older Greek geometers for solving these problems, but that they were
theoretically recognised as a permissible means of solution when the
solution could not be effected by means of the straight line and circle,
and that it was only in later times that it was considered necessary to
use conics in every case where that was possible*. Heiberg' suggests
that the allusion of Aristotle to vtvo-em perhaps confirms this sup-
position, as Aristotle nowhere shows the slightest acquaintance with
conics. I doubt whether this is a safe inference, since the problems
of this type included in the elementary text-books might easily have
been limited to those which could be solved by ``plane ``methods (i.e.
by means of the straight line and circle). We know, e.g., from Pappus
that Apollonius wrote two Books on plane rewetf*. But one thing
is certain, namely that Euclid deliberately excluded this class of
problem, doubtless as not being essential in a book of Elements.

Definitions not afterwards used.

I-astly, Euclid has definitions of some terms which he never after-
wards uses, e.g.\ oblong (eTepo/ws), rhombus, rhomboid. The ``oblong''
occurs in Aristotle; and it is certain that all these definitions are
survivals from earlier books of Elements.

1 Cf.\ the chapter on \greek{νεύσεις} in \emph{The Works of
  Archimedes}, pp.~c–cxxii.

2 Zeuthen \emph{Die Lehre von den Kegelschnitten im Altertum}, ch.~12,
  p.~262.

3 Heiberg, \emph{Mathematisches zu Aristoteles}, p.~16.

4 Papps \r7.~pp.~670—2.

\end{comment}

\part{Book I}

\section*{Definitions}

\begin{enumerate}

\item A \textbf{point} is that which has no part.

\item A \textbf{line} is breadthless length,

\item The extremities of a line are points.

\item A \textbf{straight line} is a line which lies evenly with the
  points on itself.

\item A \textbf{surface} is that which has length and breadth only.

\item The extremities of a surface are lines.

\item A \textbf{plane surface} is a surface which lies evenly with the
  straight lines on itself.

\item A \textbf{plane angle} is the inclination to one another of two
  lines in a plane which meet one another and do not lie in a straight
  line.

\item And when the lines containing the angle are straight, the angle
  is called \textbf{rectilineal}.

\item When a straight line set up on a straight line makes the
  adjacent angles equal to one another, each of the equal angles is
  \textbf{right}, and the straight line standing on the other is
  called a \textbf{perpendicular} to that on which it stands.

\item An \textbf{obtuse angle} is an angle greater than a right angle.

\item An \textbf{acute angle} is an angle less than a right angle.

\item A \textbf{boundary} is that which is an extremity of anything.

\item A \textbf{figure} is that which is contained by any boundary or
  boundaries.

\item A \textbf{circle} is a plane figure contained by one line such
  that all the straight lines falling upon it from one point among
  those lying within the figure are equal to one another;

\item And the point is called the \textbf{centre} of the circle.

\item A \textbf{diameter} of the circle is any straight line drawn
  through the centre and terminated in both directions by the
  circumference of the circle, and such a straight line also bisects
  the circle.

\item A \textbf{semicircle} is the figure contained by the diameter
  and the circumference cut off by it.  And the centre of the
  semicircle is the same as that of the circle.

\item \textbf{Rectilineal figures} are those which are contained by
  straight lines, trilateral figures being those contained by three,
  quadrilateral those contained by four, and multilateral those
  contained by more than four straight lines.

\item Of trilateral figures, an \textbf{equilateral triangle} is that
  which has its three sides equal, an \textbf{isosceles triangle} that
  which has two of its sides alone equal, and a \textbf{scalene
    triangle} that which has its three sides unequal.

\item Further, of trilateral figures, a \textbf{right-angled triangle}
  is that which has a right angle, an \textbf{obtuse-angled triangle}
  that which has an obtuse angle, and an \textbf{acute-angled
    triangle} that which has its three angles acute.

\item Of quadrilateral figures, a \textbf{square} is that which is
  both equilateral and right-angled; an \textbf{oblong} that which is
  right-angled but not equilateral; a \textbf{rhombus} that which is
  equilateral but not right-angled; and a \textbf{rhomboid} that which
  has its opposite sides and angles equal to one another but is
  neither equilateral nor right-angled. And let quadrilaterals other
  than these be called \textbf{trapezia}.

\item \textbf{Parallel} straight lines are straight lines which, being
  in the same plane and being produced indefinitely in both
  directions, do not meet one another in either direction.

\end{enumerate}

\section*{Postulates}

Let the following be postulated:

\begin{enumerate}

\item\label{post:1} To draw a straight line from any point to any point.

\item\label{post:2} To produce a finite straight line continuously in a straight
  line.

\item\label{post:3} To describe a circle with any centre and distance.

\item\label{post:4} That all right angles are equal to one another.

\item\label{post:5} That, if a straight line falling on two straight
  lines make the interior angles on the same side less than two right
  angles, the two straight lines, if produced indefinitely, meet on
  that side on which are the angles less than the two right angles.

\end{enumerate}

\section*{Common Notions}

\begin{enumerate}

\item Things which are equal to the same thing are also equal to one
  another.

\item If equals be added to equals, the wholes are equal.

\item If equals be subtracted from equals, the remainders are equal.

\item\relax [7] Things which coincide with one another are equal to
  one another.

\item\relax [8] The whole is greater than the part

\end{enumerate}

\begin{comment}

Definition i.

Si)/i ttov ia-Ttv, uv [icpiK oi$iy.

A point is that which has no pari.

An exactly parallel use of fiepos (itrrt) in the singular is found in Aristotle,
Metaph. 1035 b 32 fitpos /11F ow tort (tat tqv mSovi, literally ``There is a
part even of the form ``; Borsitz translates as if the plural were used, ``Theile
giebt es,'' and the meaning is simply ``even the form is divisible (into parts).''
Accordingly it would be quite justifiable to translate in this case ``A point is
that which is indivisible into parts.''

Martianus Capella (5th c. a.d.) alone or almost alone translated differently,
``Punctum est cuius pars nihil est,'' ``a point is that a part of which is netting.''
Notwithstanding that Max Simon (Euclid vnd die sechs planimetrischen Sucker,
1 901) has adopted this translation (on grounds which I shall presently mention),
I cannot think that it gives any sense. If a part of a point is nothing, Euclid
might as well have said that a point is itself ``nothing,'' which of course he
does not do.

Pre -Euclidean definitions.

It would appear that this was not the definition given in earlier text-
books; for Aristotle (Topics vi. 4, 141 b 20), in speaking of ``the definitions''
of point, line, and surface, says that they alt define the prior by means of the
posterior, a point as an extremity of a line, a line of a surface, and a surface
of a solid

The first definition of a point of which we hear is that given by the
Pythagoreans (cf.\ Proclus, p.~95, 21), who defined it as a ``monad having
position'' or ``with position added'' (m°vo; irpoo-kafjowra 9i<rw). It is frequently
used by Aristotle, either in this exact form (cf.\ De anima 1. 4, 409 a 6) or its
equivalent: e.g.\ in Metaph. 1016 b 24 he says that that which is indivisible
every way in respect of magnitude and qu magnitude but has not position is
a monad, while that which is similarly indivisible and has position is a point.

Plato appears to have objected to this definition. Aristotle says (Metaph.
992 a 20) that he objected ``to this genus [that of points] as being a geometrical
fiction (ytinptTpiKav Soyita), and called a point the beginning of a line {px*l
ypaft/),), while again he frequently spoke of * indivisible lines. 1 ``To which
Aristotle replies that even ``indivisible lines ``must have extremities, so that
the same argument which proves the existence of lines can be used to prove
that points exist It would appear therefore that, when Aristotle objects to
the definition of a point as the extremity of a line (wipe.? ypap.p.ys) as un-
scientific {Topics vi. 4, 141 b 21), he is aiming at Plato. Heiberg conjectures
(Mathematisthes m Aristoteks, p.~8) that it was due to Plato's influence that
the word for ``point'' generally used by Aristotle {<my/iij) was replaced by
<n)iitiov (the regular term used by Euclid, Archimedes and later writers), the
latter term (-nota, a conventional mark) probably being considered more
suitable than trrtypy (a. puncture) which might appear to claim greater reality
for a point

Aristotle's conception of a point as that which is indivisible and has
position is further illustrated by such observations as that a point is not a
body (l)e caeh 11. 13, 196 a 17) and has no weight (\ibid~111. 1, 299 a 30);
again, we can make no distinction between a point and the place (jorm) where
it is (Physics iv. 1, 209 a n). He finds the usual difficulty in accounting for
the transition from the indivisible, or infinitely small, to the finite or divisible
magnitude. A point being indivisible, no accumulation of points, however far
it may be carried, can give us anything divisible, whereas of course a line is a
divisible magnitude. Hence he holds that points cannot make up anything
continuous like a line, point cannot be continuous with point (06 yap l<rrar
tofitvov tnjpMov trtjfitiov y ariy firj tniy/ofi, De gen. el corr. \prop{1}{2}, 317 a ro), and
a line is not made up of points (ou vy«€i™t in emy/uly, Physics iv, 8, a 1 5
b 19). A point, he says, is like the now in time: now is indivisible and is
not a. part of time, it b only the beginning or end, or a division, of time, and
similarly a point may be an extremity, beginning or division of a line, but is
not part of it or of magnitude (cf.\ De eaelo m. 1, 300 a 14, Physics IV. n,
220 a 1 — 21, vi. 1, 231 b 6 sqq.). It is only by motion that a point can
generate a line {De anima 1. 4, 409 a 4) and thus be the origin of magnitude.

Other ancient definitions.

According to an-Nairīzī (ed. Curtee, p.~3) one ``Herundes'' (not so far
identified) defined a point as ``the indivisible beginning of all magnitudes,''
and Position ius as ``an extremity which has no dimension, or an extremity of
a line.''

Criticisms by commentators.

Euclid's definition itself is of course practically the same as that which
Aristotle's frequent allusions show to have been then current, except that it
omits to say that the point must have position. Is it then sufficient, seeing
that there are other things which are without parts or indivisible, e.g.\ the now
in time, and the unit in number f Proclus answers {p.~93, 18) that the point
is the only thing in the subject-matter of geometry that is indivisible. Relatively
therefore to the particular science the definition is sufficient Secondly, the
definition has been over and over again criticised because it is purely negative.
Proclus' answer to this is (p.~94, 10) that negative descriptions are appropriate
to first principles, and he quotes Pa mien ides as having described his first and
last cause by means of negations merely. Aristotle too admits that it may
sometimes be necessary for one framing a definition to use negations, e.g.\ in
defining privative terms such as ``blind''; and he seems to accept as proper
the negative element in the definition of a point, since he says (De anima 111,6,
430 b 20) that ``the point and every division [e.g.\ in a length or in a period
of time], and that which is indivisible in this sense, ts exhibited as privation

Simplicius (quoted by an-Nairīzī) says that ``a point is the beginning of
magnitudes and that from which they grow; it is also the only thing which,
having position, is not divisible.'' He, like Aristotle, adds that it is by its
motion that a point can generate a magnitude: the particular magnitude can
only be ``of one dimension,'' viz.\ a line, since the point does not ``spread
itself'' (dimittat), Simplicius further observes that Euclid defined a point
negatively because it was arrived at by detaching surface from body, line from
surface, and finally point from line. ``Since then body has three dimensions
it follows that a point [arrived at after successively eliminating all three
dimensions] has none of the dimensions, and has no part,'' This of course
reappears in modern treatises (cf.\ Rausenberger, Eiementar'gtomctrit des
Punktes, der Geraden und der Ebene, 1887, p.~7).

An-Nairīzī adds an interesting observation. ``If any one seeks to know
the essence of a point, a thing more simple than a line, let him, in the sensible
world, think of the centre of the universe and the poles.'' But there is
nothing new under the sun: the same idea is mentioned, in an Aristotelian
treatise, in controverting those who imagine that the poles have some influence
in the motion of the sphere, ``when the poles have no magnitude but are
extremities and points ``(De motu animalium 3, 699a 11).

Modern views.

In the new geometry represented by the excellent treatises which start
from new systems of postulates or axioms, the result of the profound study of
the fundamental principles of geometry during recent years (I need only
mention the names of Pasch, Veronese, Enriques and Hilbert), points come
before lines, but the vain effort to define them a priori is not made; instead
of this, the nearest material things in nature are mentioned as illustrations,
with the remark that it is from them that we can get the abstract idea. Cf.
the full statement as regards the notion of a point in Weber and Wellstein,
Encyehpddie dtr eiementaren Mathematik, 11., 1905, p.~9. ``This notion is
evolved from the notion of tbe real or supposed material point by the process
of limits, i.e.\ by an act of the mind which sets a term to a series of presen-
tations in itself unlimited. Suppose a grain of sand or a mote in a sunbeam,
which continually becomes smaller and smaller. In this way vanishes more
and more the possibility of determining still smaller atoms in the grain of
sand, and there is evolved, so we say, with growing certainty, the presentation
of the point as a definite position in space which is one and is incapable of
further division. But this view is untenable; we have, it is true, some idea
how the grain of sand gets smaller and smaller, but only so long as it remains
just visible; after that we are completely in the dark, and we cannot see or
imagine the further diminution. That this procedure comes to an end is
unthinkable; that nevertheless there exists a term beyond which it cannot go,
we must believe or postulate without ever reaching it . . . It is a pure
act of will, not of the understanding.'' Max Simon observes similarly {Euclid,
p.~85) ``The notion 'point' belongs to the limit-notions (Grenzbegriffe, the
necessary conclusions of continued, and in themselves unlimited, senes of
presentations.'' He adds, ``The point is the limit of localisation; if this is
more and more energetically continued, it leads to the limit-notion 'point,'
better 'position,' which at the same time involves a change of notion. Content
of space vanishes, relative position remains. 'Point' then, according to our
interpretation of Euclid, is the extremest limit of that which we can still think
of (not observe) as a spatial presentation, and if we go further than that, not
only does extension cease but even relative place, and in this sense the 'part'
is nothing.'' I confess I think that even the meaning which Simon intends to
convey is better expressed by ``it has no part'' than by ``the part is nothing,''
since to take a ``part'' of a thing in Euclid's sense of the result of a simple
division, corresponding to an arithmetical fraction, would not be to change
the notion from that of the thing divided to an entirely different one.

Definition 2.

I 'pafip.!} St nrjiu>% a'wXaT«.

A line is breadthless length.

This definition may safely be attributed to the Platonic School, if not to
Plato himself. Aristotle (Topics vi. 6, 143 b 11) speaks of it as open to
objection because it ``divides the genus by negation,'' length being necessarily
either breadthless or possessed of breadth; it would seem however that the
objection was only taken in order to score a point against the Platonists, since
he says (\ibid~143 b 29) that the argument is ``of service only against those
who assert that the genus [sc. length] is one numerically, that ts, those who
assume ideas,'' e.g.\ the idea of length (amo u*o$) which they regard as a
genus: for if the genus, being one and self-existent, could be divided into
two species, one of which asserts what the other denies, it would be self-
contradictory (WaiU),

Proclus (pp.~96, 21—97, 3) observes that, whereas the definition of a point
is merely negative, the line introduces the first ``dimension,'' and so its
definition is to this extent positive, while it has also a negative element which
denies to it the other ``dimensions ``(Suur-rdtrtii). The negation of both
breadth and depth is involved in the single expression ``breadthless'' (atrXarti),
since everything that is without breadth is also destitute of depth, though the
converse is of course not tnie.

Alternative definitions.

The alternative definition alluded to by Proclus, fiiyiOtn iip' ty Suumjov
``magnitude in one dimension ``or, better perhaps, ``magnitude extended one
way ``(since Suuttoitk as used with reference to line, surface and solid scarcely
corresponds to our use of ``dimension ``when we speak of ``one,'' ``two,'' or
``three dimensions ``), is attributed by an-Nairlzs to ``Heromides,'' who must
presumably be the same as ``Herundes,'' to whom he attributes a certain
definition of a point. It appears however in substance in Aristotle, though
Aristotle does not use the adjective Sunmirw, nor does he apparently use
8«wrrao-« except of body as having three ``dimensions ``or ``having dimension
(or extension J a// ways (vavrg),'' the ``dimensions'' being in his view (1) up
and down, (2) before and behind, and (3) right and left, and ``up ``being the
principle or beginning of length, ``right ``of breadth, and ``before ``of depth
(De cat So 11. 2, 284 b 24). A line is, according to Aristotle, a magnitude
``divisible in one way only ``(jtattajrg iuLifttrov), in contrast to a magnitude
divisible in two ways (XB ZuxtpiTOv), or a surface, and a magnitude divisible
``in all or in three ways'' (iramg koX rpixfi jtnipcToV), or a body (Metaph,
1016 b 25 — 27); or it is a magnitude ``continuous one way (or in one
direction),'' as compared with magnitudes continuous tovo ways or three ways,
which curiously enough he describes as ``breadth ``and ``depth ``respectively

(jtiytdos Zi to p.iv lift tv uvvih firjitm, to 8' hrl Suo tAiitos, to 8' twi rpia (3a8o<;,
Metaph. 1020 a 11), though he immediately adds that ``length ``means a line,
``breadth ``a surface, and ``depth ``a body.

Proclus gives another alternative definition as ``flux of a point ``(/5ucr«
tnjfMuiv), i.e.\ the path of a point when moved. This idea is also alluded to in
Aristode (De anima 1. 4, 409 a 4 above quoted): ``they say that a line by its
motion produces a surface, and a point by its motion a line.'' ``This
definition,'' says Proclus (p.~97, 8 — r3), ``is a perfect one as showing the
essence of the line: he who called it the flux of a point seems to define it
from its genetic cause, and it is not every line that he sets before us, but only
the immaterial line , for it is this that is produced by the point, which, though
itself indivisible, is the cause of the existence of things divisible.''

Proclus (p.~r 00, 5 — 19) adds the useful remark, which, he says, was
current in the school of Apollonius, that we have the notion of a line when we
ask for the length of a road or a wall measured merely as length; for in that
case we mean something irrespective of breadth, viz.\ distance in one
*' dimension.'' Further we can obtain sensible perception of a line if we look
at the division between the light and the dark when a shadow is thrown on
the earth or the moon; for clearly the division is without breadth, but has
length.

Species of ``lines.''

After defining the ``line ``Euclid only mentions one species of line, the
straight line, although of course another species appears in the definition of a
circle later. He doubtless omitted all classification of lines as unnecessary for
his purpose, whereas, for example, Heron follows up his definition of a line by
a division of lines into (1) those which are ``straight ``and {2} those which are
not, and a further division of the latter into (a) ``circular circumferences,''
{) ``spiral-shaped'' (iiutottScw) lines and (r) ``curved'' (xajim-vAai) lines generally,
and then explains the four terms. Aristotle tells us {Metaph. 986 a 25) that
the Pythagoreans distinguished straight (iJW) and curved (ku/u t',W), and this
distinction appears in Plato (cf.\ Republic x. 602 c) and in Aristotle (cf.\ ``to a
line belong the attributes straight or curved,'' Anal. post. 1. 4, 73 b 19; ``as in
mathematics it is useful to know what is meant by the terms straight and
curved,'' De anima \prop{1}{1}, 402 b 19). But from the class of ``curved ``lines
Plato and Aristotle separate off the urtpiipijs or ``circular ``as a distinct
species often similarly contrasted with straight. Aristotle seems to recognise
broken lines forming an angle as one line: thus ``a line, if it be bent (xosafi-
pin}), but yet continuous, is called one'' (Me tap A. 1 01 6 a 2); ``the straight line
is more one than the bent line'' (\ibid~1016 a 12). cf.\ Heron, Def. 12, ``A
broken line (jeckW/u'vij y pappy) so-called is a line which, when produced,
does not meet itself.''

When Proclus says that both Plato and Aristotle divided lines into those
which are ``straight,'' ``circular'' (wtpujxpife) or ``a mixture of the two,'' adding,
as regards Plato, that he included in the last of these classes ``those which are
called helicoidal among plane (curves) and (curves) formed about solids, and
such species of curved lines as arise from sections of solids ``(p.~104, 1 — 5),
he appears to be not quite exact. The reference as regards Plato seems to be
to Parmenides 145 B: ``At that rate it would seem that the one must have
shape, either straight or round (arpoyyvKov) or some combination of the two'';
but this scarcely amounts to a formal classification of lines. As regards
Aristotle, Proclus seems to have in mind the passage (De each i. 2, 268 b 17)
where it is stated that ``all motion in space, which we call translation ($apd), is
(in) a straight line, a circle, or a combination of the two; for the first two ate
the only simple (motions).''

For completeness it is desirable to add the substance of Proclus' account
of the classification of lines, for which he quotes Geminus as his authority.

Geminus* first classification of lines.

This begins {p.~in, 1 — 9) with a division of lines into composite [avtSrrtK)
and incomposite (io-u'ec'rrov). The only illustration given of the composite
class is the ``broken line which forms an angle'' (9 KftXavjiivy) ™1 ywiW
ttoumaa); the subdivision of the incomposite class then follows (in the text as
it stands the word ``composite ``is clearly an error for ``incomposite ``). The
subdivisions of the incomposite class are repeated in a later passage (pp.~176,
27 — 177, 23) with some additional details. The following diagram reproduces
the effect of both versions as far as possible (all the illustrations mentioned by
Proclus being shown in brackets).

lines

composite
(broken line forming an angle)

incomposite

forming a figure

GXytHiTtrroiQvaa.i
or determinate

(circle, ellipse, eissoid)

not forming a figure

or

indeterminate

d6ptToi

and

extending without limit

i-r Arttpop iKfia.Wfin'a.t

(straight line, parabola, hyperbola, conchoid)

The additional details in the second version, which cannot easily be shown
in the diagram, are as follows:

(1) Of the lines which extend without limit, some do not form a figure at
all (viz.\ the straight line, the parabola and the hyperbola); but some first
``come together and form a figure'' (i.e.\ have a loop), ``and, for the rest,
extend without limit ``(p.~177, 8).

1 \emph{Notes pour l'histoire des lignes et surfaces courbes dans
  l'antiquité} in \emph{Bulletin des sciences mathém.\ et
  astronom.}\ 2 sér.\ \r8.\ (1884), pp.~108–9 (\emph{Mémoires
  scientifiques}, \r2.~p.~23).

As the only other curve, besides the parabola and the hyperbola, which
has been mentioned as proceeding to infinity is the conchoid (of Nicomedes),
we can hardly avoid the conclusion of Tannery 1 that the curve which has a
loop and then proceeds to infinity is a variety of the conchoid itself. As is
well known, the ordinary conchoid (which was used both for doubling the
cube and for trisecting the angle) is obtained in this way. Suppose any
number of rays passing through a fixed point (the pole) and intersecting a
fixed straight line; and suppose that points are taken on the rays, beyond the
fixed straight line, such that the portions of the rays intercepted between the
fixed straight line and the point are equal to a constant distance (hiacrnj/ia),
the locus of the points is a conchoid which has the fixed straight line for
asymptote. If the ``distance'' a is measured from the intersection of the ray
with the given straight line, not in the direction away from the pole, but
towards the pole, we obtain three other curves according as a is less than,
equal to, or greater than b, the distance of the pole from the fixed straight line,
which is an asymptote in each case. The case in which a  l> gives a curve
which forms a loop and then proceeds to infinity in the way Proclus describes.
Now we know both from Eutocius {Csmm. on Archimedes, ed. Heiberg, in.
p.~98) and Proclus (p.~272, 3 — 7) that Nicomedes wrote on conchoidr (in
the plural), and Pappus <iv. p.~244, 18) says that besides the ``first'' (used as
above stated) there were ``the second, the third and the fourth which are
useful for other theorems.''

(2) Proclus next observes (p.~177, 9) that, of the lines which extend
without limit, some are ``asymptotic'' (anummK namely ``those which
never meet, however they are produced,'' and some are ``symptotic'' namely
``those which will meet sometime ``; and, of the ``asymptotic ``class, some
are in one plane, and others not. Lastly, of the ``asymptotic'' lines in one
plane, some preserve always the same distance from one another, while others
continually ``lessen the distance, like the hyperbola with reference to the
straight line, and the conchoid with reference to the straight line.''

Geminus' second classification.

This (from Proclus, pp.~in, 9 — 20 and 112, 16 — 18) can be shown in a
diagram thus:

Incomposite lines
itiimBntx. ypafiftat

' ' ' ?

sirnple, drX$ mixed, part

making a figure indeterminate

tfXyW *oiou<ra A6p4TCt

(e.g.\ circle) (straight line}

lines in planes lines on solids

at ir Toft rrtpttiit

r; 1 ' 1

line meeting itself extending without limit

4 i* afr-tf tfvfi-Ktnwaix rr i-r Airtipar JKpaWofUnf
(e.g.\ cissoid)

lines formed by uctiom tines round solids

al Mri rii rtyjdi al rtjA ri mpti.

(e.g.\ conic sections, spirit curves) (e.g.\ kdix about 1 sphere or about a cone)

1; L - — 1

tumotomtrii Hot homfftomertc

{cylindrical helix) (all othera)

Notes on classes of ``lines'' and on particular curves.

We will now add the most interesting notes found in Proclus with
reference to the above classifications or the particular curves mentioned.

i. Homoeomeric lines.

By this term (ofUHOfLtpcU) are meant lines which are alike in all parts, so
that in any one such curve any part can be made to coincide with any other
part. Proclus observes that these lines are only three in number, two being
``simple'' and in a plane {the straight line and the circle), and the third
``mixed,'' {subsisting) ``about a solid,'' namely the cylindrical helix. The
latter curve was also called the coehlias or eochlion, and its homoeomeric
property was proved by Apollonius in his work n-cpt tov koXwu (Proclus,
p.~i«5i 5). The fact that there are only three homoeomeric lines was proved
by Geminus, ``who proved, as a preliminary proposition, that, if from a point
(ami rav trtjuuuv, but on p.~251, 4 $' ' v ° s <7ij/iei'ou) two straight lines be drawn
to a homoeomeric line making equal angles with it, the straight lines are
equal'' {pp.~112, 1—113, 3i cf - P- *5'> 3 ~ l 9)-

2. Mixed lines.

It might be supposed, says Proclus (p.~105, n), that the cylindrical helix,
being homoeomerie, like the straight line and the circle, must like them be
simple. He replies that it is not simple, but mixed, because it is generated by
two unlike motions. Two like motions, said Geminus, e.g.\ two motions at the
same speed in the directions of two adjoining sides of a square, produce a
simple line, namely a straight line (the diagonal); and again, if a straight line
moves with its extremities upon the two sides of a right angle respectively,
this same motion gives a simple curve (a circle) for the locus of the middle
point of the straight line, and a mixed curve (an ellipse) for the locus of any
Other point on it (p.~106, 3—15).

Geminus also explained that the term ``mixed,'' as applied to curves, and
as applied to surfaces, respectively, is used in different senses. As applied to
curves, ``mixing'' neither means simple ``putting together'' (<rw0«7«) nor
``blending'' (upturn). Thus the helix (or spiral) is a ``mixed ``line, but (1) it
is not ``mixed ``in the sense of ``putting together,'' as it would be if, say, part
of it were straight and part circular, and (2) it is not mixed in the sense of
``blending,'' because, if it is cut in any way, it does not present the appearance
of any simple lines {of which it might be supposed to be compounded, as it
were). The ``mixing ``in the case of lines is rather that in which the con-
stituents are destroyed so far as their own character is concerned, and are
replaced, as it were, by a chemical combination (feme h avrjj trwiapfiira to
ixpa, koI <rvyic€)(y)i.cra.). On the other hand ``mixed ``surfaces are mixed in
the sense of a sort of ``blending ``[xatci «™ Kpaa-iy). For take a cone gene-
rated by a straight line passing through a fixed point and passing always
through the circumference of a circle: if you cut this by a plane parallel to
that of the circle, you obtain a circular section, and if you cut it by a plane
through the vertex, you obtain a triangle, the ``mixed ``surface of the cone
being thus cut into simple lines (pp.~117, 12 — 118, 23).

3. Spiric curves.

These curves, classed with conics as being sections of solids, were dis-
covered by Perseus, according to an epigram of Perseus' own quoted by
Proclus (p.~ii2, t), which says that Perseus found ``three lines upon {or,
perhaps, in addition to) five sections ``(rpew ypn/xtta? hA vltrt Totals).
Proclus throws some light on these in the following passages:

``Of the spiric sections, one is interlaced, resembling the horse-fetter
(finroy irflh)); another is widened out in the middle and contracts on each
side (of the middle), a third is elongated and is narrower in the middle,
broadening out on each side of it'' (p.~11 a, 4 — 8).

``This is the case with the spiric surface; for it is conceived as generated
by the revolution of a circle remaining at right angles [to a plane] and turning
about a point which is not its centre [in other words, generated by the revo-
lution of a circle about a straight line in its plane not passing through the
centre]. Hence the spirt takes three forms, for the centre [of rotation] is
either on the circumference, or within it, or without it. And if the centre of
rotation is on the circumference, we have the continuous spire (vwixnt if
within, the interlaced {iii.wnrk(yu.evi)), and if without, the open (S«x>js). And
the spiric sections are three according to these three differences'' (p.~119,
8-r 7 ).

``When the hippopede, which is one of the spiric curves, forms an angle
with itself, this angle also is contained by mixed lines'' (p.~1.27, 1 — 3).

``Perseus showed for spirics what was their property {a-vinrriia) ``

(P- 356. ``)

Thus the spiric surface was what we call a tore, or (when open) an anchor-
ring. Heron (Def. 97) says it was called alternatively spire {uirflpa} or ring
(xpuur;); he calls the variety in which ``the circle cuts itself,'' not ``interlaced,''
but ``cross ing-ltse If'' (iTrakkaTTUwra).

Tannery 1 has discussed these passages, as also did Schiaparelli*. It is clear
that Prochis' remark that the difference in the three curves which he mentions
corresponds to the difference between the three surfaces is a slip, due perhaps
to too hurried transcribing from Geminus; all three arise from plane sections
of the open anchor-ring. If r is the radius of the revolving circle, a the
distance of its centre from the axis of rotation, d the distance of the plane
section (supposed to be parallel to the axis) from the axis, the three curves
described in the first extract correspond to the following cases:

(1) d=a~r. In this case the curve is the hippopede, of which the
lemniscate of Bernoulli is a particular case, namely that in which a = zr.

The name hippopede was doubtless adopted for this one of Perseus' curves
on the ground of its resemblance to the hippopede of Eudoxus, which seems to
have been the curve of intersection of a sphere with a cylinder touching it
internally.

(2) a + r>d>a. Here the curve is an oval,

(3) a > rf> a - r. The curve is now narrowest in the middle.
Tannery explains the ``three lines upon (in addition to) five sections ``

thus. He points out that with the open tore there are two other sections
corresponding to

(4) d= a: transition from (2) to (3).

(5) a- r>d> o, in which case the section consists of two symmetrical
ovals.

He then shows that the sections of the closed or continuous tore, corre-
sponding to a = r, give curves corresponding to (2), (3) and (4) only. Instead
of (t) and (5) we have only a section consisting of two equal circles touching
one another.

On the other hand, the third spire (the interlaced variety) gives three new
forms, which make a group of three in addition to the first group oifive sections.

1 \emph{Notes pour l'histoire des lignes et surfaces courbes dans
  l'antiquité} in \emph{Bulletin des sciences mathém.\ et
  astronom.}\ \r8.\ (1884), pp.~25–27 (\emph{Mémoires scientifiques},
  \r2.~p.~24–28).

2 \emph{Die homocentrischen Sphären des Eudoxus, des Kallippus und des
  Aristoteles} (\emph{Abhandlungen zur Gesch.\ der Math.}\ \r1.~Heft,
  1877, pp.~149–152).

The difficulty which I see in this interpretation is the fact that, just after
``three lines on five sections ``are mentioned, Proclus describes three curves
which were evidently the most important; but these three belong to three of
the five sections of the open tore, and are not separate from them,

4. The cissoid.

This curve is assumed to be the same as that by means of which, according
to Eutocius (Comm. on Archimedes, in. p.~66 sqq.), Diodes in his book ir«pi
Kvpiwv (On burning-glasses) solved the problem of doubling the cube. It is
the locus of points which he found by the following construction. Let AC,
BD be diameters at right angles in a circle with centre O.

Let E, Fbe points on the quadrants BC, BA respectively such that the
arcs BE, BE ait equal.

Draw EG, FH perpendicular to CA. D

Join AE, and let P be its intersection
with FH.

The cissoid is the locus of all the
points P corresponding to different posi-
tions of E on the quadrant BC and of F
at an equal distance from B along the arc
BA.

A ts the point on the curve correspond-
ing to the position C for the point E, and
B the point on the curve corresponding
to the position of E in which it coincides
with B.

It is easy to see that the curve extends
in the direction AB beyond B, and that
CK drawn perpendicular to CA is an
asymptote. It may be regarded also as
having a branch AD symmetrical with
AB, and) beyond D, approaching KC produced as asymptote.

If OA, 0£> are coordinate axes, the equation of the curve is obviously

/(« + *) = (``-*)*>
where a is the radius of the circle.

There is a cusp at A, and it agrees with this that Proclus should say
(p.~1 a 6, 34 that ``cissoidal lines converging to one point like the leaves of
ivy — for this is the origin of their name — form an angle.'' He makes the
slight correction (p.~1 28, 5) that it is not two farts of a curve, but one curve,
which in this case makes an angle.

But what is surprising is that Proclus seems to have no idea of the curve
passing outside the circle and having an asymptote, for he several times
speaks of it as a closed curve (forming a figure and including an area): cf.
p.~152, 7, ``the plane (area) cut off by the cissoidal line has one bounding
(line), but it has not in it a centre such that all (straight lines drawn to the
curve) from it are equal.'' It would appear as if Proclus regarded the cissoid
as formed by the/our symmetrical cissoidal arcs shown in the figure.

Even more peculiar is Proclus' view of the

5, ``Single-turn Spiral.''

This is really the spiral of Archimedes traced by a point starting from
the fixed extremity of a straight line and moving uniformly along it, while
simultaneously the straight line itself moves uniformly in a plane about its fixed
extremity. In Archimedes the spiral has of course any number of turns, the
straight line making the same number of complete revolutions. Yet I'roclus,
while giving the same account of the generation of the spiral (p.~180, 8 — 12),
regards the single-turn spiral us actually stopping short at the point reached
after one complete revolution of the straight line: ``it is necessary to know
that extending without limit is not a property of all lines; for it neither
belongs to the circle nor to the cissoid, nor in general to lines which form
figures; nor even to those which do not form figures. For even the single-
turn spiral does not extend without limit— -for it is constructed between two
pointsnor does any of the other lines so generated do so'' (p.~187, 19 — 25).
It is curious that Pappus (vm. p.~n 10 sqq. } uses the same term finvoVrpoins
\\ii to denote one turn, not of the spiral, but of the cylindrical helix.

Definition 3.

Tp*Hprj$ SI Tripara {TTjMld.

The extremities of a line art points.

It being unscientific, as Aristotle said, to define a point as the ``extremity
of a line ``(iripas ypa/iftjjs), thereby explaining the prior by the posterior,
Euclid defined a point differently; then, as it was necessary to connect a
point with a line,, he introduced this explanation after the definitions of both
had been given. This compromise is no doubt his own idea; the same
thing occurs with reference to a surface and a line as its extremity in Def. 6,
and with reference to a solid and a surface as its extremity in xt. Def. 2.

We miss a statement of the facts, equally requiring to be known, that a
``division ``(Suupio-ij) of a line, no less than its ``beginning ``or ``end,'' is a
point (this is brought out by Aristotle: cf, Metapk. 1060 b 15), and that
the intersection of two lines is also a point. If these additional explanations
had been given, Proclus would have been spared the difficulty which he finds
in the fact that some of the tines used in Euclid (namelv infinite straight lines
on the one hand, and circles on the other) have no ``extremities.'' So also
the ellipse, which Proclus calls by the, old name (foptfc (``shield ``). In the
case of the circle and ellipse we can, he observes (p.~105, 7), take a portion
bounded by points, and the definition applies to that portion. His rather
far-fetched distinction between two aspects of a circle or ellipse as a line and
as a closed figure (thus, while you are describing a circle, you have two extremi-
ties at any moment, but they disappear when it is finished) is an unnecessarily
elaborate attempt to establish the literal universality of the ``definition,''
which is really no more than an explanation that, if a line has extremities,
those extremities are points.

Definition 4.

Eu#eiu yfnififurj tarty, $rtf ci laov toIs i<f tavriji <r<i(fi(ot* xcirai.
A straight line is a line which lies evenly with the points on itself.

The only definition of a straight line authenticated as pre-Euclidean is
that of Plato, who defined it as ``that of which the middle covers the ends ``
(relatively, that is, to an eye placed at either end and looking along the
straight line). It appears in the Parmenidcs 137 k: ``suaight is whatever has
its middle in front of (i.e.\ so placed as to obstruct the view of) both its ends ``
(diflii yt ov iv to /iitror ap.tj>oir ratv iaxdrotv btiTrpottOfv p). Aristotle quotes it in
equivalent terms {Topics vi. n, 148 b 27), o5 to /h'o-oc r«rnotr#«I row inpao-ii';
and, as he does not mention the name of its author, but states it in combina-
tion with the definition of a line as the extremity of a surface, we may assume
that he used it as being well known. Proclus also quotes the definition as
Plato's in almost identical terms, fc to fU<ra row oupow is-tirpoo-fltt (p.~109, ji).
This definition is ingenious, but implicitly appeals to the sense of sight and
involves the postulate that the line of sight is straight. (cf.\ the Aristotelian
Problems 31, to, 950 a 39, where the question is why we can better observe
straightness in a row, say, of letters with one eye than with two.) As regards
the straightness of ``visual rays,'' oijrui, cf.\ Euclid's own Optics, Ueff. i, *,
assumed as hypotheses, in which he first speaks of the ``straight lines ``drawn
from the eye, avoiding the word ttyws, and then says that the figure contained
by the visual rays (oif/us) is a cone with its vertex in the eye.

As Aristotle mentions no definition of a straight line resembling Euclid's,
but gives only Plato's definition and the other explaining it as the ``extremity
of a surface,'' the latter being evidently the current definition in contemporary
textbooks, we may safely infer that Euclid's definition was a new departure of
his own.

Proclus on Euclid's definition.

Coming now to the interpretation of Euclid's definition, tW«a ypa/117
t<mv, ljTis i£ utou ToTi i$' (ttvnji otjiiiIok KctTat, we find any number of slightly
different versions, but none that can be described as quite satisfactory; some
authorities, e.g.\ Savile, have confessed that they could make nothing of it It
is natural to appeal to Proclus first; and we find that he does in fact give an
interpretation which at first sight seems plausible. He says {p.~109, 8 sq.) that
Euclid ``shows by means of this that the straight line alone [of all lines]
occupies a distance (mriiv $cacn;/ia) equal to that between the points on it.
For, as far as one of the points is distant from another, so great is the length
(p.fytSm) of the straight line of which they are the extremities; and this is the
meaning of lying i£ «™v to (or with) the points on it ``\i( urov being thus,
apparently, interpreted as ``at'' (or ``over'') ``an equal distance'']. ``But if
you take two points on the circumference (of a circle) or any other line, the
distance cut off between them along the line is greater than the interval
separating them. And this is the case with every line except the straight line.
Hence the ordinary remark, based on a common notion, that those who
journey in a straight line only travel the necessary distance, while those who
do not go straight travel more than the necessary distance.'' (cf.\ Aristotle,
De caclo 1. 4, 271 a r3, ``we always call the distance of anything the straight
line'' drawn to it.) Thus Proclus would interpret somewhat in this way: ``a
straight line is that which represents extension equal with (the distances
separating) the points on it.'' This explanation seems to be an attempt to
graft on to Euclid's definition the assumption (it is a Xunfiavoptvor, not a
definition) of Archimedes {On the sphere and cylinder 1. ad init.) that ``of all
the lines which have the same extremities the straight line is least.'' For this
purpose i£ «row has apparently to be taken as meaning ``at an equal distance,''
and again ``lying at an equal distance'' as equivalent to ``extending over (or
representing) an equal distance.'' This is difficult enough in itself, but is
seen to be an impossible interpretation when applied to the similar definition
of a plane by Euclid (Def. 7) as a surface ``which lies evenly with the straight
lines on itself.'' In that connexion Proclus tries to make the same words «£ Iirou
ciroi mean ``extends over an equal area with.'' He says namely (p.~117, 2)
that, ``if two straight lines are set out ``on the plane, the plane surface
``occupies a space equal to that between the straight lines.'' But two straight
lines do not determine by themselves any space at all j it would be necessary
to have a dosed figure with its boundaries in the plane before we could arrive
at the equivalent of the other assumption of Archimedes that ``of surfaces
which have the same extremities, if those extremities are in a plane, the plane is
the least [in area].'' This seems to be an impossible sense for i£ urov even on
the assumption that it means ``at an equal distance ``in the present definition.
The necessity therefore of interpreting i£ Urov similarly in hoth definitions
makes it impossible to regard it as referring to distance or length at all. It
should be added that Simplicius gave the same explanations as Proclus
(an-Nairtzl, p.~5).

The language and construction of the definition.

Let us now consider the actual wording and grammar of the phrase ns i£
Urov toU i<j> tauTtji <rrj(uioK xttTai. As regards the expression if urov we note
that Plato and Aristotle (whose use of it seems typical) commonly have it in
the sense of ``on a footing of equality'': cf.\ oi i( EVou in Plato's Laws 777 d,
gig d; Aristotle, Politics 1259 b g «£ ta-uv ttvat (fovkerai tt/v $va-w t ``tend to
be on an equality in nature,'' Eth. Nit. vm. 12, 1161 a 8 tVraWa ira>r« i(
Urov, ``there all are on a footing of equality.'' Slightly different are the uses
in Aristotle, Eth, Nie. x. 8, 1178 a 25 rur /ikr yap iwyimri'w xP tul *<'' i£ Urov
Itrrta, ``both need the necessaries of life to the same extent, Set us say''; Topics ix,
15, 174 a 32 i$ urov iroioWa rije tpiinjatv, ``asking the question indifferently''
{i.e.\ without showing any expectation of one answer being given rather than
another). The natural meaning would therefore appear to be ``evenly placed''
(or balanced), ``in equal measure,'' ``indifferently'' or ``without bias'' one way
or the other. Next, is the dative rots i<f>' io.vrr}t o''i)/u£ok constructed with i( urov
or with MttBt? In the first case the phrase must mean ``that which heevenfy
with (or in respect to) the points on it,'' in the second apparently ``that which,
in (or by) the points on it, lies (or is placed) evenly (or uniformly).'' Max Simon
takes the first construction to give the sense ``die Gerade liegt in gleicher
Weise wie ihre Punkte.'' If the last words mean ``in the same way as (or in
like manner as) its points,'' I cannot see that they tell us anything, although
Simon attaches to the words the notion of distance (Abstand) like Proclus.
The second construction he takes as giving ``die Gerade liegt fur (durch) ihre
Punkte gleichmassig,'' ``the straight line lies symmetrically for (or through) its
points''; or, if k«t<h is taken as the passive of Ti'Siff«, ``die Gerade ist durch
ihre Funkte gleichmassig gegeben worden,'' ``the straight line is symmetrically
determined by its points.'' He adds that the idea is here direction, and that
both direction and distance (as between two different given points simply)
would be to Euclid, as later to Bolzano {Betrachtungen iiber einige Gegenstdnde
der Elementargeometrie, 1804, quoted by Schotten, Inhalt and Metkode des
planimetrischtn Unttrrichts, ». p.~16), primary irreducible notions.

While the language is thus seen to be hopelessly obscure, we can safely
say that the sort of idea which Euclid wished to express was that of a line
which presents the same shape at and relatively to all points on it, without
any irregular or unsymmetrical feature distinguishing one part or side of it
from another. Any such irregularity could, as Saccheri points out (Engel and
Stackel, Die Theorie der Parallellinien von Euklidbis Gauss, 1895, p.~109), be
at once made perceptible by keeping the ends fixed and turning the line about
them right round; if any two positions were distinguishable, e.g.\ one being to
the left or right relatively to another, ``it would not lie in a uniform manner
between its points.''

A conjecture as to its origin and meaning.

The question arises, what- was the origin of Euclid's definition, or, how
was it suggested to him ? It seems to me that the basis of it was really
Plato's definition of a straight line as ``that line the middle of which covers
the ends.'' Euclid was a Platonist, and what more natural than that he
should have adopted Plato's definition in substance, while regarding it as
essential to change the form of words in order to make it independent of any
implied appeal to vision, which, as a physical fact, could not properly find a
place in a purely geometrical definition ? I believe therefore that Euclid's
definition is simply an attempt (albeit unsuccessful, from the nature of the
case) to express, in terms to which a geometer could not object as not being
part of geometrical subject-matter, the same thing as the Platonic definition.

The truth is that Euclid was attempting the impossible. As Pfieiderer
says (Scholia to Euclid), ``It seems as though the notion of a straight line,
owing to its simplicity , cannot be explained by any regular definition which
does not introduce words already containing in themselves, by implication,
the notion to be defined (such e.g.\ are direction, equality, uniformity or
evenness of position, unswerving course), and as though it were impossible, if
a person does not already know what the term straight here means, to teach
it to him unless by putting before him in some way a picture or a drawing of
it.'' This is accordingly done in such books as Veronese's Elementi ii
geometria (Part I., roo4, p.~ro): ``A stretched string, e.g.\ a plummet, a ray of
light entering by a small hole into a dark room, are rectilineal objects. Hie
image of them gives us the abstract idea of the limited line which is called a
rectilineal segment,''

Other definitions.

We will conclude this note with some other famous definitions of a straight
line. The following are given by Proclus (p.~no, 18 — a 3).

i, A line stretched to the utmost, bf anpov Ttrafdrr) ypafi./. This appears
in Heron also, with the words ``towards the ends ``(iwl ri urifiara) added.
(Heron, Def. 4).

2. Part of it cannot be in the assumed plane while part is in one higher up
(in fitTiiupoTtpui). This is a. proposition in Euclid (xi. 1).

3. All its parts fit on all {other parts) alike, Tavra avnjs ™ fii/nj wairiv
biioiun ty<W*o(f . Heron has this too (Def. 4), but instead of ``alike ``he
says ravroitos, ``in all ways,'' which is better as indicating that the applied part
may be applied one way or the reverse way, with the same result.

4. That line which, when its ends remain fixed, itself remains fixed, ij t«c
irtpdruiv ittvovrmv naX avr? /itvowra. Heron's addition to this, ``when it is, as
it were, turned round in the same plane ``(otur iv ry aw /n-iWo orpio/tcnf),
and his next variation, ``and about the same ends having always the same
position,'' show that thf; definition of a straight line as ``that which does
not change its position when it is turned about its extremities (or any two
points in it) as poles ``was no original discovery of Leibniz, or Saccheri, or
Krafft, or Gauss, but goes back at least to the beginning of the Christian era.
Gauss' form of this definition was: ``The line in which lie all points that,
during the revolution of a body (a part of space) about two fixed paints,
maintain their position unchanged is called a straight line.'' Schotten
(I- p.~315) maintains that the notion of a straight tine and its property of
being determined by two points are unconsciously assumed in this definition,
which is therefore a logical ``circle.''

5. That line which with one other of tkt same species cannot complete a
figure, y fA€TCi njs po<£$ov? /ita; crjjxa fty dirorfkowsa. This is an obvious
tcrripov-icponpov, since it assumes the notion of a figure.

Lastly Leibniz' definition should be mentioned: A straight line is one
which divides a plane into two halves identical in all but position. Apart from
the fact that this definition introduces the plane, it does not seem to have any
advantages over the definition last but one referred to.

Legendre uses the Archimedean property of a straight line as the shortest
distance between two points. Van Swinden observes (Elcmcnte der Geometric,
1834, p.~4), that to take this as the definition involves assuml/tgthe proposition
that any two sides of a triangle are greater than the third and proving that
straight lines which have two points in common coincide tnroughout their
length (cf.\ Legendre, Aliments de Giom/trie \prop{1}{3}, 8).

The above definitions all illustrate the observation of Unger {Die Geometric
dcs Euilid, 1833): ``Straight is a simple notion, and hence all definitions of
tt must fail.... But if the proper idea of a straight line has once been grasped,
it will be recognised in all the various definitions usually given of it; all
the definitions must therefore be regarded as explanations, and among them
that one is the best from which further inferences can immediately be drawn
as to the essence of the straight line.''

Definition 5.

Em<ftivtta Si itrroi, $ ft$KM *<'' wAaros jaocav Jvtt.

A surface it that which has length and breadth only.

The word Irudrua was used by Euclid and later writers to denote surfact
in general, while they appropriated the word twiirtSov for plane surface, thus
making tiriwtSor a species of the genus i-rttdyua. A solitary use of hrt$avtta
by Euclid when a plane is meant (xi. Def. 1 1 ) is probably due to the fact that
the particular definition came from an earlier textbook. Froclus (p.~116, 17)
remarks that the older philosophers, including Plato and Aristotle, used the
words brufrdvtia and imnov indifferently for any kind of surface. Aristotle
does indeed use both words for a surface, with perhaps a tendency to use
iVioKtio more than iV«r«W for a surface not plane. cf.\ Categories 6, 5 a 1 sq.,
where both words are used in one sentence: ``You can find a common
boundary at which the parts fit together, a point in the case of a line, and a line
in the case of a surface (ortaVtm); for the parts of the surface (imniov) do fit
together at some common boundary. Similarly also in the ease of a body you
can find a common boundary, a line or a surface (irujuLvftn), at which the
parts of the body fit together.'' Plato however does not use ivupdvtia at all in
the sense of surface, but only cjmrtSov for both surface and plane surface.
There is reason therefore for doubting the correctness of the notice in
Diogenes Laertius, 111. a 4, that Plato ``was the first philosopher to name,
among extremities, the//o«# surface'' («V»VtSo$ rnufmytui).

(Ttocuo of course means literally the feature of a body which is apparent
to the eye (iwiavrj*), namely the surface.

Aristotle tells us (De sensu 3, 439 a 31) that the Pythagoreans called a
surface xpoin, which seems to have meant skin as well as colour. Aristotle
explains the term with reference to colour (xpl*'') a* a thing inseparable from
the extremity (srepai) of a body.

Alternative definitions.

The definitions of a surface correspond to those of a line. As in Aristotle
a line is a magnitude ``(extended) one way, or in one ' dimension ' ``(i<j> h),
``continuous one way'' (i<j> tr trunx), or ``divisible in one way'' (iowiy
Statpfroi-), so a surface is a magnitude extended or continuous two ways (hi
Sw>), or divisible in two ways (Sixi)). As in Euclid a surface has ``length and
breadth ``only, so in Aristotle ``breadth ``is characteristic of the surface and is
once used as synonymous with it (Metaph. tojo a ix), and again ``lengths
are made up of long and short, surfaces of broad and narrow, and solids (oyi™)
of deep and shallow'' (Metaph. 1085 a 10).

Aristotle mentions the common remark that a line by its motion produces a
surface (De anima ]. 4, 409 a 4). He also gives the a posteriori description of
a surface as the ``extremity of a solid'' (Topics vi. 4, [41 b 21}, and as ``the
section (rofnf) or division {S«up«rii) of a body'' (Metaph. 1060 b 14).

Proclus remarks (p.~114, jo) that we get a notion of a surface when we
measure areas and mark their boundaries in the sense of length and breadth;
and we further get a sort of perception of it by looking at shadows, since
these have no depth {for they do not penetrate the earth) but only have length
and breadth.

Classification of surfaces.

Heron gives (Def. 74, p.~50, ed. Heiberg) two alternative divisions or
surfaces into two classes, corresponding to Gemirtus' alternative divisions of
lines, viz.\ into (1) incomposite and composite and (2) simple and mixed.

(1) Incomposite surfaces are ``those which, when produced, fall into (or
coalesce with) themselves'' (wtim iK0akXo/ityai, airat tiaff iavrwv Wwrowie),
i.e.\ are of continuous curvature, e.g.\ the sphere.

Composite surfaces are ``those which, when produced, cut one another.''
Of composite surfaces, again, some are (a) made up of non-homogeneous
(elements) (i$ dvofiotoyow) such as cones, cylinders and hemispheres, others
() made up of homogeneous (elements), namely the rectilineal (or polyhedral)
surfaces.

(2) Under the alternative division, simple surfaces are the plane and the
spherical surfaces, but no others; the mixed class includes all other surfaces
whatever and is therefore infinite in variety.

Heron specially mentions as belonging to the mixed class (a) the surface
of cones, cylinders and the like, which are a mixture of plane and circular
(fitKTal i£ tTrnrt'Sou Kal irepicepttas) and (h) spirie surfaces, which are ``a mixture
of two circumferences ``(by which he must mean a mixture of two circular
elements, namely the generating circle and its circular motion about an axis in
the same plane).

Proclus adds the remark that, curiously enough, mixed surfaces may arise
by the revolution either of simple curves, e.g.\ in the case of the spire, or of
mixed curves, e.g.\ the ``right-angled conoid'' from a parabola, ``another
conoid'' from the hyperbola, the ``oblong'' (krifinKts, in Archimedes iropa-
pa-xts) and ``flat ``(VwtjrAa-nJ) spheroids from an ellipse according as it revolves
about the major or minor axis respectively (pp.~119, 6 — 120, 2). The homoeo-
meric surfaces, namely those any part of which will coincide with any other
part, are two only (the plane and the spherical surface), not three as in the case
of lines (p.~no, 7).

Definition 6.

'Eirtavtias Si -ripara. ypappmi.

The extremities of a surface are lines.

It being unscientific, as Aristotle says, to define a line as the extremity of
a surface, Euclid avoids the error of defining the prior by means of the
posterior in this way, and gives a different definition not open to this
objection. Then, by way of compromise, and in order to show the connexion
between a line and a surface, he adds the equivalent of the definition of a line
previously current as an explanation.

As in the corresponding Def. 3 above, he omits to add what is made
clear by Aristotle (Metapk. 1060 b 15) that a ``division'' (Siaipco-is) or
``section ``(to/ii;) of a solid or body is also a surface, or that the common
boundary at which two parts of a solid fit together (Categories 6, 5 a a)
may be a surface.

Proclus discusses how the fact stated in Def. 6 can be said to be true of
surfaces like that of the sphere ``which is bounded (wtiripairrat), it is true, but
not by lines.'' His explanation (p.~116, 8 — 14) is that, ``if we take the surface
(of a sphere), so far as it is extended two ways (BtyjJ a<rr<iTi/), we shall find
that it is bounded by lines as to length and breadth; and if we consider the
spherical surface as possessing a form of its own and invested with a fresh
quality, we must regard it as having fitted end on to beginning and made
the two ends (or extremities) one, being thus one potentially only, and not in
actuality.''

Definition 7,

'Eir(jr«S« brttJMvtta itrrtv, iJtii t£ itrmi rati 1$ iavrijs ivBtiais wttrai.
A plane surface is a surface which ties evenly with the straight lines on
itself.

The Greek follows exactly the definition of a straight line mutatis mutandis ,
i.e.\ with Ta«...riS(t'ttit for tok. . .(rryuiW Proclus remarks that, in general,
all the definitions of a straight line can be adapted to the plane surface by
merely changing the genus. Thus, for instance, a plane surface is ``a surface
the middle of which covers the ends ``(this being the adaptation of Plato's
definition of a straight line). Whether Plato actually gave this as the defini-
tion of a plane surface or not, I believe that Euclid's definition of a plane
surface as lying evenly with the straight lines on itself was intended simply to
express the same idea without any implied appeal to vision (just as in the
corresponding case of the definition of a straight line).

As already noted under Def. 4, Proclus tries to read into Euclid's defini-
tion the Archimedean assumption that ``of surfaces which have the same
extremities, if those extremities are in a plane, the plane is the least.'' But,
as I have stated, his interpretation of the words seems impossible, although it
is adopted by Simplicius also (see an-Naiiizi).

Ancient alternatives.

The other ancient definitions recorded are as follows.

1. The surf act which is stretched to the utmost («V axpav t tropin}): a
definition which Proclus describes as equivalent to Euclid's definition (on
Proclus' own view of that definition). cf.\ Heron, Def. 9, ``(a surface) which
is right (and) stretched out ``(ip8i) ovtra airortrnjuoTj), words which he adds to
Euclid's definition.

z. The least surface among all those which have the same extremities.
Proclus is here (p.~1 1 7, 9) obviously quoting the Archimedean assumption.

3. A surface all the farts of which have the property of fitting on (each
ether) (Heron, Def. 9).

4. A surface such that a straight line fits on all parts of it (Proclus,
p.~117, 8), or such that the straight line fits on it all ways, i.e.\ however placed
(Proclus, p.~117, jo).

With this should be compared:

5. ``(A plane surface is) such that, if a straight line pass through two
points on it, the line coincides wholly with it at every spot, all ways,'' i.e.\ however
placed (one way or the reverse, no matter how), 7c ixtiSiiv vo trqutuiiv aijnjTai
ritdtla, koX okrj aurp Kara irarra roirctv miPTOtfd? i$apfi4£*Tat (Heron, Def. 9).

This appears, with the words «<rra mivra Torov iraiToMot omitted, in Theon of
Smyrna (p.~112, 5, ed. Hiller), so that it goes back at least as far as the
1st c. a.d. It is of course the same as the definition commonly attributed to
Robert Simson, and very widely adopted as a substitute for Euclid's.

This same definition appears also in an-Nairīzī (ed. Curtze, p.~10) who,
after quoting Simplicius' explanation (on the same lines as Proclus') of the
meaning of Euclid's definition, goes on to say that ``others defined the plane
surface as that in which it is possible to draw a straight tine from any point to
any other.''

pifficuitiea in ordinary definitions.

Gauss observed in a letter to Bessel that the definition of a plane surface
as a surface such that, if any him points in it be tahcn, the straight line joining
them lies wAolfy in the surface (which, for short, we will call ``Simson's''
definition) contains more than is necessary, in that a plane can be obtained by
simply projecting a straight line lying in it from a point outside the line but also
lying on the plane; in fact the definition includes a theorem, or postulate, as
well. The same is true of Euclid's definition of a plane as the surface which
``lies evenly with (all) the straight lines on itself,'' because it is sufficient for a
definition of a plane if the surface ``lies evenly ``with those lines only which
pass through a fixed point on it and each of the several points of a straight line
also lying in it but not passing through the point But from Euclid's point
of view it is immaterial whether a definition contains more than the necessary
minimum provided that the existence of a thing possessing all the attributes
contained in the definition is afterwards proved. This however is not done
in regard to the plane. No proposition about the nature of a plane as such
appears before Book XI., although its existence is presupposed in all the
geometrical Books 1.— iv. and vi.; nor in Book xi. is there any attempt to
prove, _e.g.\ by construction, the existence of a surface conforming to the
definition. The explanation may be that the existence of the plane as defined
was deliberately assumed from the beginning like that of points and lines, the
existence of which, according to Aristotle, must be assumed as principles
unproved, while the existence of everything else must be proved; and tt may
well be that Aristotle would have included plane surfaces with points and
lines in this statement had it not been that he generally took his illustrations
from plane geometry (excluding solid).

But, whatever definition of a plane is taken, the evolution of its essential
properties is extraordinarily difficult. Crelle, who wrote an elaborate article
Zur Theorie der Ebene (read in the Academie der Wissenschaften in 1834) of
which account must be taken in any full history of the subject, observes that,
since the plane is the field, as it were, of almost all the rest of geometry, while
a proper conception of it is necessary to enable Eucl. 1. 1 to be understood,
it might have been expected that the theory of the plane would have been the
subject of at least the same amount of attention as, say, that of parallels. This
however was far from being the case, perhaps because the subject of parallels
(which, for the rest, presuppose the notion of a plane) is much taster than that
of the plane. The nature of the difficulties as regards the plane have also
been pointed out recently by Mr Frank land {The First Book of Euclid's
Elements, Cambridge, 1905); it would appear that, whatever definition is
taken, whether the simplest (as containing the minimum necessary to deter-
mine a plane) or the more complex, e.g.\ Simson's, some postulate has to be
assumed in addition before the fundamental properties, or the truth of the
other definitions, can be established. Crelle notes the same thing as regards
Simson's definition, containing mare than is necessary. Suppose a plane in
which lies the triangle ABC. Let AD join the vertex A
to any point D on EC, and BE the vertex B to any
point E on CA. Then, according to the definition, AD
lies wholly in the plane of the triangle; so does BE.
But, if both AD and BE are to lie wholly in the one
plane, AD, BE must intersect, say at F: if they did not,
there would be two planes in question, not one. But the fact that the lines
intersect and that, say, AD does not pass above or below BE, is by no
means self-evident

Mr Frankland points out the similar difficulty as regards the simpler
definition of a plane as the surface generated by a straight
line passing always through a fixed point and always
intersecting a fixed straight line. Let OFF, OQQ
drawn from intersect the straight line X at P, Q
respectively. Let E be any third point on X: then it
needs to be proved that OR intersects P'Q in some
point, say R'. Without some postulate, however, it is
not easy to see how to prove this, or even to prove that P'Q intersects X.

Crelle'a essay. Definitions by Fourier, Deahna, Becker,
Crelle takes as the standard of a good definition that it shall be, not only as
simple as possible, but also the best adapted for deducing, with the aid of the
simplest possible principles, further properties belonging to the thing defined.
He was much attracted by a very lucid definition, due, he says, to Fourier,
according to which a flam is formed by the aggregate of ail. the straight lints
which, passing through one point on a straight line in space, are perpendicular
to that straight line. (This is really no more than an adaptation from Euclid's
proposition Xi. 5, to the effect that, if one of four concurrent straight lines be
at right angles to each of the other three, those three are in one plane, which
proposition is also used in Aristotle, Meteorohgica m. 3, 373 a 13.) But
Crelle confesses that he had not been able to deduce the necessary properties
from this and had had to substitute the definition, already mentioned, of a
plane as the surface containing, throughout their whole length, all the straight
lines passing through a fixed point and also intersecting a straight line in space;
and he only claims to have proved, after a long series of propositions, that the
``Fourier ``- or ``perpendicular ``-surface and the plane of the other definition
just given are identical, after which the properties of the ``Fourier ``-surface
can be used along with those of the plane. The advantage of the Fourier
definition is that it leads easily, by means of the two propositions that
triangles are equal in all respects (i) when two sides and the included angle
are respectively equal and (j) when all three sides are respectively equal, to the
property expressed in Simsou's definition. But Crelle uses to establish these
two congruence-theorems a number of propositions about equal angles, supple-
mentary angles, right angles, greater and less angles; and it is difficult to
question the soundness of Schotten's criticism that these notions in themselves
really presuppose that of a plane. The difficulty due to Fourier's use of
the word ``perpendicular,'' if that were all, could no doubt be got over. Thus
Deahna in a dissertation (Marburg, 1837) constructed a plane as follows.
Presupposing the notions of a straight line and a sphere, he observes that, if a
sphere revolve about a diameter, all the points of its surface which move
describe closed curves (circles). Each of these circles, during the revolution,
moves along itself, and one of them divides the surface of the sphere into two
congruent parts. The aggregate then of the lines joining the centre to the
points of this circle forms thepfojie. Again, J. K. Becker (Die Elemente der
Geometric, 1877) pointed out that the revolution of a right angle about one
side of it produces a conical surface which differs from all other conical
surfaces generated by the revolution of other angles in the fact that the
particular cone coincides with the cone vertically opposite to it: this characteristic
might therefore be taken in order to get rid of the use of the right angle.

W. Bolyai and Lobachewsky.

Very similar to Deahna's equivalent for Fourier's definition is the device
of W. Bolyai and Lobachewsky (described by Frischauf, Elementt der
absnlulen Geometric, 1876). They worked upon a fundamental idea first
suggested, apparently, by Leibniz, Briefly stated, their way of evolving a
plane and a straight line was as follows. Conceive an infinite number of
pairs of concentric spheres described about two fixed points in space, O, 0,
as centres, and with equal radii, gradually increasing: these pairs of equal
spherical surfaces intersect respectively in homogeneous curves (circles), and
the ``InbegrifiT'' or aggregate of these curves of intersection forms a plane.
If A be a point on one of these circles {k say), suppose points M, M' to start
simultaneously from A and to move in opposite directions at the same speed
till they meet at B, say; B then is ``opposite'' to A, and A, .ff divide the
circumference into two equal halves. If the points A, B be held fast and the
whole system be turned about them until O takes the place of 0, and of
O, the circle k will occupy the same position as before (though turned a
different way). Two opposite points, P, Q say, of each of the other circles
will remain stationary during the motion as well as A, B: the ``InbegrifT'' or
aggregate of all such points which remain stationary forms a straight line. It
is next observed that the plane as defined can be generated by the revolution
of the straight line about 0(7, and this suggests the following construction
for a plane. Let a circle as one of the curves of intersection of the pairs of
spherical surfaces be divided as before into two equal halves tA,B. Let the
arc ADB be similarly bisected at D, and let C be the
middle point of AB. This determines a straight line CD
which is then defined as ``perpendicular'' to AB. The revo-
lution of CD about AB generates a plane. The property
stated in Simson's definition is then proved by means of the
congruence-theorems proved in Eucl. 1. 8 and 1. 4. The
first is taken as proved, practically by considerations of
symmetry and homogeneity. If two spherical surfaces, not necessarily equal,
with centres O, intersect, A and its ``opposite'' point B are taken as
before on the curve of intersection (a circle) and, relatively to 00', the point
A is taken to be convertible with B or any other point on the homogeneous
curve. The second (that of Eucl. i. 4) is established by simple application.
Rausenberger objects to these proofs on the grounds that the first assumes
that the two spherical surfaces intersect in one single curve, not in several,
and that the second compares angles: a comparison which, be says, is possible
only in a plane, so that a plane is really presupposed. Perhaps as regards
the particular comparison of angles Rausenberger is hypercritical; but it is
difficult to regard the supposed proof of the theorem of Eucl. 1. 8 as sufficiently
rigorous (quite apart from the use of the uniform motion of points for the
purpose of bisecting lines).

Simson's property is proved from the two congruence-theorems thus.
Suppose that AB is ``perpendicular ``(as defined by Bolyai) to two generators
CM, CN of a plane, or suppose CM, CN respectively to make with AB two
angles congruent with one another. It is enough to prove that, if P be any
point on the straight line MN, then CP, just as
much as CM, CN respectively, makes with AB two
angles congruent with one another and is therefore
a generator. We prove successively the congruence
of the following pairs of triangles:

ACM, BCM

ACN, BCN

AMN, BMN
AMP, BMP
ACP, BCP,

whence the angles A CP, BCP are congruent.

Other views.

Enriques and Amaldi (Ekmenti di geometria, Bologna, 1905), Veronese
(in his Elementi) and Hilbert all assume as a postulate the property stated in
Simson's definition. But G, Ingrami {Elementi di geometria, Bologna, 1904)
proves it tn the course of a remarkable series of closely argued proposition
based upon a much less comprehensive postulate. He evolves the theory of
the plane from that of a triangle, beginning with a triangle as a mere three-side
(trilatero), i.e.\ a frame, as it were. His postulate relates to the three-side and
is to the effect that each ``(rectilineal) segment ``joining a vertex to a point of
the opposite side meets every segment similarly joining each of the other two
vertices to the points of the sides opposite to them respectively, and, con-
versely, if a point be taken on a segment joining a vertex to a point of the
opposite side, and if a straight line be drawn from another vertex to the point
on the segment so taken, it will if produced meet the opposite side. A
triangle is then defined as the figure formed by the aggregate of all the
segments joining the respective vertices of a three-side to points on the
opposite sides. After a series of propositions, Ingrami evolves a plane as the
figure formed by the ``half straight-lines ``which project from an internal point
of the triangle the points of the perimeter, and then, after two more theorems,
proves that a plane is determined by any three of its points which are not in
a straight line, and that a straight line which has two points in a plane has all
its points in it.

The argument by which Bolyai and I>obachewsky evolved the plane is
of course equivalent to the definition of a plane as the locus of all points
equidistant from two fixed points in space.

Leibniz in a letter to Giordano defined a plane as thai surface which
divides space into turn congruent farts. Adverting to Giordano's criticism that
you could conceive of surfaces and lines which divided space or a plane into
two congruent parts without being plane or straight respectively, Beez ( liber
Euklidischc und Nicht-Euklidische Geometric, 1888) pointed out that what was
wanted to complete the definition was the further condition that the two
congruent spaces could be slid along each other without the surfaces ceasing
to coincide, and claimed priority for his completion of the definition in this
way. But the idea of all the parts of a plane fitting exactly on all other parts
is ancient, appearing, as we have seen, in Heron, Def. 9.

Definitions 8, 9.

8. '£mV«Sos i yiMivla iariv y iv cwnriw Svo ypafijjLv OriTTOfUvtav a\kijktitv
jcat i*t] tV iiOua.% Ktifiivtiiv TTfm aAAijXuf ruv ypu/Hav xAurtt.

9. ``Orav * al Trfptixowai ttjv yutvcW ypaftfXal tv4itu umtiv, tvuypaftftos
jraA.«T<u )J ymria,

8. A plane angle is the inclination to one another of two lines in a plane
which meet one another and do not lie in a straight line,

9. And when the lines containing the angle are straight, the angle is called
rectilineal.

The phrase ``not in a straight line ``is strange, seeing that the definition
purports to apply to angles formed by curves as well as straight lines. We
should rather have expected continuous (tTwtxfc) with one another; and
Heron takes this to be the meaning, since he at once adds an explanation as
to what is meant by lines not being continuous (oi <rw<x<»)- It looks as though
Euclid really intended to define a rectilineal angle, but on second thoughts,
as a concession to the then common recognition of curvilineal angles, altered
``straight lines ``into ``lines ``and separated the definition into two.

I think all our evidence suggests that Euclid's definition of an angle as
inclination (*A£«s) was a new departure. The word does not occur in
Aristotle; and we should gather from him that the idea generally associated
with an angle in his time was rather deflection or breaking of lines (kKiIit): cf.
his common use of xtK\a<r6at and other parts of the verb nASf, and also his
reference to one bent line forming an angle (tt)i/ ntKapjiivipr vol lyaotrav ftriav 1
Metaph. 1016 a 13)

Proclus has a long and elaborate note on this definition, much of which
(pp.~i2i t 12 — 126, 6) is apparently taken direct from a work by his master
Syrian us (6 iJfiiTtpo! mvjnpimrL Two criticisms contained in the note need
occasion no difficulty. One of these asks how, if an angle be an inclination,
one inclination can produce two angles. The other (p.~1 28, 2) is to the effect
that the definition seems to exclude an angle formed by one and the same
curve with itself, e.g.\ the complete cissoid [at what we call the ``cusp ``1 or the
curve known as the hippopede (horse-fetter) [shaped like a lemniscate]. But
such an ``angle'' as this belongs to higher geometry, which Euclid may well
be excused for leaving out of account in any case.

Other ancient definitions: Apollonius, Plutarch, Carpus.

Proclus' note records other definitions of great interest. Apollonius
defined an angle as a contracting of a surface or a solid at one point under a
broken line Or surface ((rvmyuyq iftBWMM 4 <7T«p«o5 vpot Iri oij/uiy viti
<t((cAa<rtv); ypawji y iruf>arflf), where again an angle is supposed to be
formed by one broken line or surface. Still more interesting, perhaps, is the
definition by ``those who say that the first distance under the point (ri •wprar
uiim)iia thro to mffuiw) is the angle. Among these is Plutarch, who insists
that Apollonius meant the same thing; for, he says, there must be some first
distance under the breaking (or deflection) of the including lines or surfaces,
though, the distance under the point being continuous, it is impossible to
obtain the actual first, since every distance is divisible without limit'' (**•'
ajrttpov). There is some vagueness in the use of the word ``distance'' (Suwrnj/ia);
thus it was objected that ``if we anyhow separate off the first ``(distance being
apparently the word understood) ``and draw a straight line through it, we get
a triangle and not one angle.'' In spite of the objection, I cannot but see in
the idea of Plutarch and the others the germ of a valuable conception in
infinitesimals, an attempt (though partial and imperfect) to get at the rate
of divergence between the lines at their point of meeting as a measure of the
angle between them.

A third view of an angle was that of Carpus of Antioch, who said ``that
the angle was a quantity (too-ov), namely a distance (SuMmf/m) between the
lines or surfaces containing it This means that it would be a distance (or
divergence) in one sense (i<fi tv Siurrwc), although the angle is not on that
account a straight line. For it is not everything extended in one sense (to 1$ tv
Biao-raToV) that is a line.'' This very phrase ``extended one way'' being held
to define a line, it is natural that Carpus' idea should have been described as
the greatest possible paradox (wdrrw TrupoSoorarov). The difficulty seems to
have been caused by the want of a different technical term to express a new
idea; for Carpus seems undoubtedly to have been anticipating the more
modern idea of an angle as representing divergence rather than distance, and to
have meant by <+' %v in one sense (rotationatly) as distinct from one way or in
one dimension (linearly).

To what category does an angle belong ?

There was much debate among philosophers as to the particular category
(according to the Aristotelian scheme) in which an angle should be placed;
is it, namely, a quantum (voaov), quale (iroioV) or relation (irpos ti)?

I. Those who put it in the category of quantity argued from the fact that
a plane angle is divided by a line and a solid angle by a surface. Since, then,
it is a surface which is divided by a line, and a solid which is divided by
a surface, they felt obliged to conclude that an angle is a surface or a solid, and
therefore a magnitude. But homogeneous finite magnitudes, e.g.\ plane
angles, must bear a ratio to one another, or one must be capable of being
multiplied until it exceeds the other. This is, however, not the case with a
rectilineal angle and the horn-like angle (jiton™*), by which tatter is meant
the ``angle'' between a circle and a tangent to it, since (Eucl. in. 16) the
latter ``angle'' is less than any rectilineal angle whatever. The objection, it
will be observed, assumes that the two sorts of angl<£ are homogeneous.
Plutarch and Carpus are classed among those who, in one way or other, placed
an angle among magnitudes; and, as above noted, Plutarch claimed Apollonius
as a supporter of his view, although the word contraction (of a surface or solid)
used by the latter does not in itself suggest magnitude much more than Euclid's
inclination. It was this last consideration which doubtless led ``Aganis,'' the
``friend ``(socius) apparently of Simplicius, to substitute for Apollonius'
wording ``a quantify which has dimensions and the extremities of which arrive
at one point'' (an-NairUE, p.~13).

3. Eudemus the Peripatetic, who wrote a whole work on the angle, main-
tained that it belonged to the category of quality. Aristotle had given as his
fourth variety of quality ``figure and the shape subsisting in each thing, and,
besides these, straight ness, curvature, and the like ``(Categories 8, 10 a 11).
He says that each individual thing is spoken of as quale in respect of its form,
and he instances a triangle and a square, using them again later on (\ibid~1 1 a 5)
to show that it is not all qualities which are susceptible of more and less; again,
in Physics 1. 5, 188 a 25 angle, straight, circular are called kinds of figure.
Aristotle would no doubt have regarded deflection (xtukairQai) as belonging to
the same category with straightness and curvature (KOf«rvA<mft). At all events,
Eudemus took up an angle as having its origin in the breaking or deflection
(«Aao-«) of lines: deflection, he argued, was quality if straightness was, and that
which has its origin in quality is itself quality. Objectors to this view argued
thus. If an angle be a quality (n-oionjs) like heat or cold, how can it be bisected,
say? It can in fact be divided; and, if things of which divisibility is an
essential attribute are varieties of quantum and not qualities, an angle cannot
be a quality. Further, the more and the less are the appropriate attributes of
quality, not the equal and the unequal; if therefore an angle were a quality,
we should have to say of angles, not that one is greater and another smaller,
but that one is more an angle and another less an angle, and that two angles
are not unequal but dissimilar (ouo'/ioum). As a matter of fact, we are told by
Simplicius, 538, at, on Arist De caelo that those who brought the angle under
the category of quale did call equal angles similar angles; and Aristotle
himself speaks of similar angles in this sense in De caelo 296020, 311 b 34.

3. Euclid and all who called an angle an inclination are held by Sy nanus
to have classed it as a relation (irpoi ti). Yet Euclid certainty regarded angles
as magnitudes; this is clear both from the earliest propositions dealing
specifically with angles, e.g.\ 1. 9, 13, and also (though in another way) from
his describing an angle in the very next definition and always as contained
(wtpiixo/Uvr]') by the two lines forming it (Simon, Euclid, p.~28}.

Proclus (i.e.\ in this case Sy nanus) adds that the truth lies between these
three views. The angle partakes in fact of all those categories: it needs the
quantity involved in magnitude, thereby becoming susceptible of equality,
inequality and .the like; it needs the qualify given it by its form, and lastly
the relation subsisting between the lines or planes bounding it.

Ancient classification of ``angles.''

An elaborate classification of angles given by Proclus (pp.~126, 7 — 127, 16)
may safely be attributed to Ge minus. In order to show it by a diagram it

Angles

on surfaces in solidt
{I* OTcptoil)

, I ' ~ — ``i

on simple surfaces on mixed surfaces

(e.g.\ cones, cylinders)

on planes on spherical surfaces

1 ' 1 ' 1

made by simple lines made by ``mixed'' lines by one of each

e. g. the angle made by a (e.g.\ the angle fanned by an
curve, such as the cissoid ellipse and its axis of by

and hippppede, with itself) an ellipse and a circle)

— ' — ``1 T''* '. ]

ine line.circumf. cLrcumf.-circumf.

line- eon vex line -concave convex-convex concave-concave mixed, or

(e.g.\ angle of a e.g.\ korH-likt (dpfcupw) {alraAwl convex-concave

semicircle) {Kcflaroeidfa) or ``scraper-lilce'' (e.g.\ those of

(fwrpotiftii) tunes)

will be necessary to make a convention about terms. Angles are to be under-
stood under each class, ``line-circumference ``means an angle contained by a
straight line and an arc of a circle, ``line-convex ``an angle contained by a
straight line and a circular arc with convexity outwards, and so on in every
case.

Definitions of angle classified.

As for the point, straight line, and plane, so foi the angle, Schotten gives
a valuable summary, classification and criticism of the different modern views
up to date (Inhalt und Methode des planitnetrisehen Unterrichts, 11., 1893,
pp.~94 — 183}; and for later developments represented by Veronese reference
may be made to the third article (by Amaldi) in Questioni riguardanti le
matematiche elementari, t, (Bologna, 19 12).

With one or two exceptions, says Schotten, the definitions of an angle may
be classed in three groups representing generally the following views:

1. The angle is the difference of direction between two straight lines. (With
this group may be compared Euclid's definition of an angle as an inclination.)

2. The angle is the quantity or amount (or the measure) of the rotation
necessary to bring one of its sides from its own position to thai of the other s:de
without its moving out of the plane containing both.

3. The angle is the portion of a plane included between two straight tines in
the plane which meet in a point (or two rays issuing from the point).

It is remarkable however that nearly all of the text-books which give
definitions different from those in group 2 add to them something pointing to
a connexion between an angle and rotation: a striking indication that the
essential nature of an angle is closely connected with rotation, and that a good
definition must take account of that connexion.

The definitions in the first group must be admitted to be (autologous, or
circular, inasmuch as they really presuppose some conception of an angle.
Direction (as between tow given points) may no doubt be regarded as a primary
notion; and it may be defined as ``the immediate relation of two points which
the ray enables us to realise'' (Schotten). But ``a direction is no intensive
magnitude, and therefore two directions cannot have any quantitative
difference ``(Biirklen). Nor is direction susceptible of differences such as
those between qualities, e.g.\ colours. Direction is a singular entity: there
cannot be different sorts or degrees of direction. If we speak of ``a different
direction,'' we use the word equivocally; what we mean is simply ``another ``
direction. The fact is that these definitions of an angle as a difference of
direction unconsciously appeal to something outside the notion of direction
altogether, to some conception equivalent to that of the angle itself.

Recent Italian views.

The second group of definitions are (says Amaldi) based on the idea of the
rotation of a straight line or ray in a plane about a point: an idea which,
logically formulated, may lead to a convenient method of introducing the
angle. But it must be made independent of metric conceptions, or of the
conception of congruence, so as to bring out first the notion of an angle, and
afterwards the notion of equal angles.

The third group of definitions satisfy the condition of not including metric
conceptions; but they do not entirely correspond to our intuitive conception
of an angle, to which we attribute the character of an entity in one dimension
(as Veronese says) with respect to the ray as element, or an entity in two
dimensions with reference to feints as elements, which may be called an angular
sector. The defect is however easily remedied by considering the angle as
``the aggregate of the rays issuing from the vertex and comprised in the angular
sector,''

Proceeding to consider the principal methods of arriving at the logical
formulation of the first superficial properties of the plant from which a
definition of the angle may emerge, Amaldi distinguishes two points of view
(l) the genetic, (z) the actual.

(i) From the first point of view we consider the cluster of straight lines
or rays (the aggregate of all the straight lines in a plane passing through a
point, or of all the rays with their extremities in that point) as generated by
the movement of a straight line or ray in the plane, about a point. This leads
to the post ulation of a closed order, or circular disposition, of the straight lines
or rays in a cluster. Next comes the connexion subsisting between the
disposition of any two clusters whatever in one, plane, and so on.

(2) Starting from the point of view of the actual, we lay the foundation
of the definition of an angle in the division of the plane into two parts (half-
planes) by the straight line. Next, two straight lines (a, b) in the plane, inter-
secting at a point O, divide the plane into four regions which ate called
angular sectors (convex); and finally the angle (ab) or (6a) may lie defined as
the aggregate of the rays issuing from O and belonging to the angular sector
which has a and b for sides.

Veronese's procedure (in his Elementi) is as follows. He begins with the
first properties of the plane introduced by the following definition.

The figure given by all the straight lines joining the points of a straight
line r to a point P outside it and by
the parallel to r through P is called a
cluster of straight lines, a cluster of rays,
or a plane, according as we consider
the element of the figure itself to be the
straight line, the ray terminated at P,
or a foinl.

[It will be observed that this method of producing a plane involves using
the parallel to r. This presents no difficulty to Veronese because he has
previously defined parallels, without reference to the plane, by means of reflex
or opposite figures, with respect to a point O: ``two straight lines are called
parallel, if one of them contains two points opposite to (or the reflex of) two
points of the other with respect to the middle point of a common transversal
(of the two lines).'' He proves by means of a postulate that the parallel r
does belong to the plane Pr, Ingrami avoids the use of the parallel by
defining a plane as ``the figure formed by the half straight lines which project
from an internal point of a triangle (i.e.\ a point on a line joining any vertex of
a three-side to a point of the opposite side) the points of its perimeter,'' and
then defining a cluster of rays as ``the aggregate of the half straight lines in a
plane starting from a given point of the plane and passing through the points
of the perimeter of a triangle containing the point'']

Veronese goes on to the definition of an angle. ``We call an angle a part
of a cluster of rays, bounded by two rays (as the segment is a part of a straight
line bounded hy two points).

``An angle of the cluster, the bounding rays, of which are opposite, is called a
flat angle.''

Then, after a postulate corresponding to postulates which he lays down for

i. Deff. 9-1 a] NOTES ON DEFINITIONS 9—12 181

a rectilineal segment and for a straight line, Veronese proves that ail flat angles
are equal to one another.

a v e

Hence he concludes that ``the cluster of rays is a homogeneous linear
system in which the element is the ray instead of the point. The cluster
being a homogeneous linear system, all the propositions deduced from
[Veronese's] Post. 1 for the straight line apply to it, e.g.\ that relative to
the sum and difference of the segments: it is only necessary to substitute
the ray for the point, and the angle for the segment.''

Definitions 10, ti, 12.

id. T Orak Si cvdfui hr ivtlav trra$tura tu? tfj>e£y<> yoivta? urac cIAA'Xacs
TOtp, Api) itcaripa i£r law ywetuie fart, Hal ij l$t<rri}m/ta tv#«a itdOtrm KoXiirai,

Ijt TjV ilflilTTTJKfV,

1 1 . A/i/JAeiJi yavia itrriv )J ft«'£o»' ipftjs.

1 2 . '0£tia Si 1} tXAirmr ipftjs.

10. When a straight line set up on a straight line makes the adjacent angles
equal is one another, each of the equal angles is right, and the straight line
standing on the other is called a perpendicular to that on which it stands.

11. An obtuse angle is an angle greater than a right angle.
1 2. An acute angle is an angle less than a right angle.

i4>t<s is the regular term for adjacent angles, meaning literally ``(next) in
order.'' I do not find the term used in Aristotle of angles, but he explains its
meaning in such passages as Physics \prop{6}{1}, 131 b 8: ``those things are (next)
in order which have nothing of the same kind (tjvyyh) between them.''

KaSrrtn, perpendicular, means literally let fall: the full expression is perpen-
dicular straight line, as we see from the enunciation of Eucl. 1. 11, and the
notion is that of a straight line let fall upon the surface of the earth, s. plumb-
line. Proclus (p.~283, 9) tells us that in ancient times the perpendicular was
called gnomon-wise (Kara yecJporo), because the gnomon (an upright stick) was
set up at right angles to the horizon.

The three kinds of angles are among the things which according to the
Platonic Socrates {Republic vi. 510 c) the geometer assumes and argues from,
declining to give any account of them because they are obvious. Aristotle
discusses the priority of the right angle in comparison with the acute (Metaph.
1084 b;): in one way the right angle is prior, i.e.\ in being defined (on
(Jptirrat) and by its notion (t< Aoyijt), in another way the acute is prior, i.e.\ as
being a part, and because the right angle is divided into acute angles; the
acute angle is prior as matter, the right angle in respect of form; cf.\ also
Metaph. 1035 b 6, ``the notion of the right angle is not divided into
that of an acute angle, but the reverse; for, when denning an acute angle,
you make use of the right angle.'' Proclus {p.~133, 15) observes that it is by
the perpendicular that we measure the heights of figures, and that it is by
reference to the right angle that we distinguish the other rectilineal angles,
which are otherwise undistinguished the one from the other.

The Aristotelian Problems {16, 4, 013 b 36) contain an expression perhaps
worth quoting. The question discussed is why things which fall on the
ground and rebound make ``similar'' angles with the surface on both sides of
the point of impact; and it is observed that ``the right angle is the limit
(opm) of the opposite angles,'' where however ``opposite ``seems to mean, not
``supplementary ``(or acute and obtuse), but the equal angles made with the
surface on opposite sides of the perpendicular.

Proclus, after his manner, remarks that the statement that an angle less
than a right angle is acute is not true without qualification, for ( 1 ) the horn-like
angle (between the circumference of a circle and a tangent) is less than a
right angle, since it is less tnan an acute angle, but is not an acute angle, while
(2) the ``angle of a semicircle'' (between the arc and a diameter) is also less
than a right angle, but is not an acute angle.

The existence of the right angle is of course proved in 1. n.

Definition 13.

*Opos iariv, tivos lati iripas.

A boundary is that which is an extremity of anything.

Aristotle also uses the words Spot and W/kk as synonymous. cf.\ De gen.
animal, it. 6, 745 a 6, 9, where in the expression '* limit of magnitude ``first
one and then the other word is used.

Proclus {p, 136, 8) remarks that the word boundary is appropriate to the
origin of geometry, which began from the measurement of areas of ground
and involved the marking of boundaries.

Definition 14.

%-rjyni Itrrt to vtt6 tlvo* ij WW Qpwv Trtpttofitvot:
A figure is that which is contained Ay any boundary or boundaries.
Plato in the Meno observes that roundness (o-rpoyyvkarrfi) or the round is a
``figure,'' and that the straight and many other things are so too j he then
inquires what there is common to all of them, in virtue of which we apply the
term ``figure'' to them. His answer is (76 a): ``with reference to every
figure I say that that in which the solid terminates (tovto, tU S to <rrtp€oy
mpaivft) is a figure, or, to put it briefly, a figure is an extremity of a solid.''
The first observation is similar to Aristotle's in the Physics 1, 5, 188 a 25,
where angle, straight, and circular are mentioned as genera of figure. In the
Categories 8, 10 a 11, ``figure'' is placed with straightness and curved ness in
the category of quality. Here however ``figure ``appears to mean shape
(ttopifnj) rather than ``figure ``in our sense. Coming nearer to ''figure' 1 in our
sense, Aristotle admits that figure is ``a sort of magnitude'' {fit anima ill. 1,
425 a i£), and he distinguishes plane figures of two kinds, in language not
unlike Euclid's, as contained by straight and circular lines respectively ! ``every
plane figure is either rectilineal or formed by circular lines (xtpufitp6ypa.ii.por),
and the rectilineal figure is contained by several lines, the circular by one
line'' (De caelo 11. 4, 286 b rj). He is careful to explain that a plane is not a
figure, nor a figure a plane, but that a plane figure constitutes one notion and
is a spaiti of the genus figure {Anal, pott. 11. 3, 00 b 37). Aristotle does not
attempt to define figure in general, in fact he says it would be useless: ``From
this it is clear that there is one definition of soul in the same way as there is
one definition of figure; for in the one case there is no figure except the
triangle, quadrilateral, and so on, nor is there any soul other than those above
mentioned. A definition might be constructed which should apply to all
figures but not specially to any particular figure, and similarly with the
species of soul referred to. [But such a general definition would serve no
purpose.] Hence it is absurd here as elsewhere to seek a general definition
which wjll not be properly a definition of anything in existence and will not
be applicable to the particular irreducible species before us, to the neglect of
the definition which is so applicable'' (De anima ir. 3, 414 b 20 — 28}.

Comparing Euclid's definition with the above, we observe that by intro-
ducing boundary (opm) he at once excludes the straight which Aristotle classed
as figure; he doubtless excluded angle also, as we may judge by (1) Heron's
statement that ``neither one nor two straight lines can complete a figure,''
(a) the alternative definition of a straight line as ``that which cannot with
another line of the same species form a figure,'' (3) Ge minus' distinction
between the line which forms a figure (o-xij/urrosrotowa) and the line which
extends indefinitely (tr axttpav J*,9oAA<i/iiifi), which latter term includes a
hyperbola and a parabola. Instead of calling figure an extremity as
Plato did in the expression ``extremity (or limit) of a solid,'' Euclid
describes a figure as that -which has a boundary or boundaries. And lastly,
in spite of Aristotle's objection, he does attempt a general definition to
cover all kinds of figure, solid and plane. It appears certain therefore that
Euclid's definition is entirely his own.

Another view of a figure, recalling that of Plato in Mens 76 a, is attributed
by Proclus (p.~143, 8) to Posidonius. The latter regarded the figure as the
confining extremity or limit {ripat avyiAaor), ``separating the notion of figure
from quantity (or magnitude) and making it the cause of definition, limitation,
and inclusion (rot lipitrSai not TrtrtpatrBtu not ti irtpioy). . . Posidonius thus
seems to have in view only the boundary placed round from outside, Euclid
the whole content, so that Euclid will speak of the circle as a figure in
respect of its whole plane (surface) and of its inclusion (from) without, whereas
Posidonius (makes it a figure) in respect of its circumference... Posidonius
wished to explain the notion of figure as itself limit trig and confining magnitude.''

Proclus observes that a logical and refining critic might object to Euclid's
definition as defining the genus from the species, since that which is enclosed
by one boundary and that which is enclosed by several are both species of
figure. The best answer to this seems to be supplied by the passage of
Aristotle's De anima quoted above.

Definitions 15, 16.

15. KtikXof Itrrl irjfijfMi t(b(8w iuro pJas ypa/i/is Trtpt(xo/t*fov [ij Kakturat
rtpitip€LtA t irpo9 %v ££' £fot trrjfitlov rvv crros rov oyrnfjuiTos KttfAtvwv iratrat at
irpc*T7ri777ovai tvOtlai (Vpos -ryv tqv kvkXov Tr*pt$ipiiav] urtu dXAirfXaif titriv.

16. Kivrpov Si rot kvk\ov to trijfAfioy xaAfirat.

15. A circle is a plane figure contained by one line such that all the straight
lines falling upon it from one point among those lying -within the figure are equal
to one another;

16. And the point is called the centre of the circle.

The words ij KaXtirm rtpiiptia, ``which is called the circumference,'' and
Tpot ttjv toS kvkXov wtpuftipttay, ``to the circumference of the circle,'' are
bracketed by Heiberg because, although the mss. have them, they are
omitted in other ancient sources, viz.\ Proclus, Taurus, Sextus Empiricus and
Boethius, and Heron also omits the second gloss. The recently discovered
papyrus Hercuianensis No. 1061 also quotes the definition without the words
in question, confirming Heiberg's rejection of them (see Heiberg in Hermes
xxxviii., 1903, p.~47), The words were doubtless added in view of the
occurrence of the word ``circumference'' in Deff. 17, 18 immediately
following, without any explanation. But no explanation was needed. Though
the word irtpiiptia does not occur in Flato, Aristotle uses it several times
(1) in the general sense of con tour without any special mathematical signification,
{2) mathematically, with reference to the rainbow and the circumference, as
well as an arc, of a circle. Hence Euclid was perfectly justified in employing
the word in Deff. 17, 18 and elsewhere, but leaving it undefined as being a
word universally understood and not involving in itself any mathematical
conception. It may be added that an-Nairīzī had not the bracketed words
in his text; for he comments on and tries to explain Euclid's omission to
define the circumference.

The definition itself contained nothing new in substance. Plato {Parme-
nides 137 e) says: ``Round is, I take it, that the extremes of which are every
way equally distant from the middle ``(aTpoyyvkor yi irov Am touto, oi v to
loara irayra airo rov pitrnv Ixrov lirritj). In Aristotle we find the following
expressions: ``the circular (v4pi<i>fp6ypajifu>v) plane figure Dounded by one
line'' {De taeh n. 4, 286 b 13 — 16); ``the plane equal (i.e.\ extending equally
all ways) from the middle ``(hrUtfov to i*. tov p.iao\s urov), meaning a
circle (Rhetoric 111. 6, 1407 b 27); he also contrasts with the circle ``any
other figure which has not the lines from the middle equal, as for example an
egg-shaped figure'' {De eaelo n. 4, 287 a 19). The word ``centre'' {nivrpov)
was also regularly used: cf.\ Produs' quotation from the ``oracles ``(Wyto),
``the centre from which all (lines extending) as far as the rim are equal.''

The definition as it stands has no genetic character. It says nothing as to
the existence or non-existence of the thing defined or as to the method of
constructing it. It simply explains what is meant by the word ``circle,'' and
is a provisional definition which cannot be used until the existence of circles
is proved or assumed. Generally, in such a case, existence is proved by
actual construction; but here the possibility of constructing the circle as
defined, and consequently its existence, are postulated (Postulate 3). A genetic
definition might state that a circle is the figure described when a straight line,
always remaining in one plane, moves about one extremity as a fixed point
until it returns to its first position (so Heron, Def. ay).

Simplicius indeed, who points out that the distance between the feet of a
pair of compasses is a straight line from the centre to the circumference, will
have it that Euclid intended by this definition to show how to construct a
circle by the revolution of a straight line about one end as centre; and an-
Nairlzi points to this as the explanation (r) of Euclid's definition of a circle
as a plane figure, meaning the whole surface bounded by the circumference,
and not the circumference itself, and (2) of his omission to mention the
``circumference,'' since with this construction the circumference is not drawn
separately as a line. But it is not necessary to suppose that Euclid himself
did more than follow the traditional view; for the same conception of the
circle as a plane figure appears, as we have seen, in Aristotle. While, however,
Euclid is generally careful to say the ``circumference of a circle ``when he means
the circumference, or an arc, only, there are cases where ``circle'' means
``circumference of a circle,'' e.g.\ in ill. 10 1 ``A circle does not cut a circle
in more points than two.''

Heron, Proclus and Simplicius are all careful to point out that the centre
is not the only point which is equidistant from all points of the circumference.
The centre is the only point in the plane of the circle (``lying within the figure,''
as Euclid says) of which this is true; any point not in the same plane which
is equidistant from all points of the circumference is a pole. If you set up a
``gnomon ``(an upright stick) at the centre of a circle (i.e.\ a line through the
centre perpendicular to the plane of the circle), its upper extremity is a pole
(Proclus, p.~153, 3); the perpendicular is the locus of all such poles.

Definition 17.

ktAfitrpof Si rov kvkKuv ttrrlv tvOttOr tls ia tw Ktvrpov Tjyfjvrj k<u irtpaiav-
p-ivT) ($ tKfXTfpa tq pip7j uiro tt|s ro? kvkXov ircpMrpfiav, Tfrc? *a>. Stya Ti/ivfi tov
kvx\qv.

A diameter of the circle is any straight line drawn through the centre and
terminated in both directions by the circumference of the circle, and such a straight
line also bisects the circle.

The last words, literally ``which (straight line) also bisects the circle,''
are omitted by Simson and the editors who followed him. But they are
necessary even though they do not ``belong to the definition ``but only
express a property of the diameter as defined. For, without this explanation,
Euclid would not have been justified in describing as a .fcrjw-circle a portion
of a circle bounded by a diameter and the circumference cut off by it.

Simplicius observes that the diameter is so called because it passes through
the whole surface of a circle as if measuring it, and also because it divides the
circle into two equal parts. He might however have added that, in general, it
is a line passing through a figure where it is widest, as well as dividing it
equally: thus in Aristotle ri nara Sia/ttrpw KtCptva, ``things diametrically
situated ``in space, are at their maximum distance apart. Diameter was the
regular word in Euclid and elsewhere for the diameter of a square, and also
of a parallelogram; diagonal (Stayuptot) was a later term, defined by Heron
(Def. by] as the straight line drawn from an angle to an angle.

Proclus (p.~157, 10) says that Thales was the first to prove that a circle is
bisected by its diameter; but we are not told how he proved it. Proclus gives
as the reason of the property ``the undeviating course of the straight line
through the centre ``(a simple appeal to symmetry), but adds that, if it is
desired to prove it mathematically, it is only necessary to imagine the diameter
drawn and one part of the circle applied to the other; it is then clear that
they must coincide, for, if they did not, and one fell inside or outside the
other, the straight lines from the centre to the circumference would not all be
equal: which is absurd.

Saccheri's proof is worth quoting. It depends on three ``Lemmas ``
immediately preceding, (1) that two straight lines cannot enclose a space,
(z) that two straight lines cannot have one and the same segment common,
(3) that, if two straight lines meet at a point, they do not touch, but cut one
another, at it.

``Let MDHNKM be a circle, A its centre, MN a diameter. Suppose
the portion MNKM of the circle turned about the fixed points M, N, so
that it ultimately comes near to or coincides with the remaining portion
MNHDM.

``Then (i) the whole diameter MAN', with all
its points, clearly remains in the 3ame position,
since otherwise two straight lines would enclose a
space (contrary to the first Lemma).

``(ii) Clearly no point K of the circumference
NKM falls within or outside the surface enclosed
by the diameter MANand the other part, NHDM,
of the circumference, since otherwise, contrary to
the nature of the circle, a radius as AK would be
less or greater than another radius as All.

``(iii) Any radius MA can clearly be rectilineally produced only along a
single other radius AN, since otherwise (contrary to the second Lemma) two
lines assumed straight, e.g.\ MAN, MAH, would have one and the same
common segment.

``(iv) All diameters of the circle obviously cut one another in the centre
(Lemma 3 preceding), and they bisect one another there, by the general
properties of the circle.

``From all this it is manifest that the diameter MAN divides its circle
and the circumference of it just exactly into two equal parts, and the same
may be generally asserted for every diameter whatsoever of the same circle;
which was to be proved.''

Simson observes that the property is easily deduced from tit. 31 and 24;
for it follows from lit. 31 that the two parts of the circle are ``similar
segments'' of a circle (segments containing equal angles, in. Def. 11), and
from tti. 24 that they are equal to one another.

Definition 18.

H/iuruKXiov hi itTTt to Trtpttofitvov tr)(fffia. inrd re t?/s BiafUTpov Kat Trj%
Ivokafiawo/Ltv iiir avre vtpitpiia. nivTfujv Si tov ijfiiKvtt\{oii to avrd, 

Ktll TOV KVtikrtv CffTtV.

A semicircle if the figure contained by the diameter and the circumference cut
off by it. And the centre of the semicircle is the same as that of the circle.

The last words, ``And the centre of the semicircle is the same as that
of the circle,'' are added from Proclus to the definition as it appears in the
\textsc{mss.}\  Scarburgh remarks that a semicircle has no centre, properly speaking,
and thinks that the words are not Euclid's, but only a note by Proclus. I am
however inclined to think that they are genuine, if only because of the very
futility of an observation added by Proclus. He explains, namely, that the
semicircle is the only plane figure that has its centre on its perimeter (!), ``so
that you may conclude that the centre has three positions, since it may be
within the figure, as in the case of a circle, or on the perimeter, as with the
semicircle, or outside, as with some conic lines (the single-branch hyperbola
presumably)'' !

Proclus and Simplicius point out that, in the order adopted by Euclid for
these definitions of figures, the first figure taken is that bounded by on* line
(the circle), then follows that bounded by two lines (the semicircle), then the
triangle, bounded by three lines, and so on. Proclus, as usual, distinguishes
different kinds of figures bounded by two lines (pp.~159, 14 — 160, 9). Thus
they may be formed

(1) by circumference and circumference, e.g.\ (a) those forming angles, as
a tunc (to pijvottSn) and the figure included by two arcs with convexities
outward, and (b) the angle-less (iytinov), as the figure included between two
concentric circles (the coronal);

(2) by circumference and straight line, e.g.\ the semicircle or segments of
circles (tty£S<t is a name given to those less than a semicircle);

(3) by ``mixed ``line and ``mixed ``line, e.g.\ two ellipses cutting one
another;

(4) by ``mixed ``line and circumference, e.g.\ intersecting ellipse and
circle;

(5) by ``mixed'' line and straight line, e.g.\ half an ellipse.

Following Def. 18 in the mss. is a definition of a segment of a circle which
was obviously interpolated from in. Def. 6. Proclus, Martianus Capella and
Boethius do not give it in this place, and it is therefore properly omitted.

Definitions 19, 20, 21.

19. SxfOTo tiSvypa/ifia i<m ra wro tifituSv vepnoptva, rptirXtvpa fiir
ra viro rpcwv, TtrpawXtvpa Si Ttt vro Tftrtrapwv, Trokvirktvpa Si to Wd ir\<toi*iii' 9
Ttwdpwv (vfyii'vi- irtpttxoprtva,

20. Tw Si Tparktvpttiv tTxypATwv Uroirktvpov piy tprMMW cart to rat rptts
uras tX 01 ' irAtvpas, to-oo-«€ Acs Si to ras Svo povas tcras %\ov irXtvpai, tTKttX.i/fvov Si
TO rac rptls dvurow; t\oc vktvpos.

21. *Ett Si Tali' Tpnrktvpwv )ri]pf£-rwv 6p$oytovtav ptv rpiytovov itrri to l)(ov
6p$ijv yomav, dp.ftkvytii'tQV Si to tyov dfiflkiiav ytortav, ovycuviO*' Si to Tat Tpits
o£cia¥ i%Qv ytin-tas.

19. Rectilineal figures are those which are contained by straight lines,
trilateral figures being those contained by three, quadrilateral those contained by

four, and multilateral those contained by more than four straight lines.

20. Of trilateral figures, an equilateral triangle is that which has its three
sides equal, an isosceles triangle that which has two of its sides alone equal, and
a scalene triangle that which has Us three sides unequal.

x 1 . Further, of trilateral figures, a right-angled triangle it that which has
a right angle, an obtuse-angled triangle that which has an obtuse angle, and an
acute-angled triangle that which has its three angles acute.

1 9.

The latter part of this definition, distinguishing three-sided, four-sided and
many-sided figures, is probably due to Euclid himself, since the words
TplirXtvpov, TtrpairKivpov and ircAvTrAtvpop do not appear in Plato or Aristotle
(only in one passage of the Mechanics and of the Problems respectively does
even rtrpdirktvpov, quadrilateral, occur). By his use of TtTpa'irAtupoK,
quadrilateral, Euclid seems practically to have put an end to any ambiguity
in the use by mathematicians of the word Terpdyiiivov, literally ``four-angled
(figure),'' and to have got it restricted to the square. cf.\ note on Def. 22,

20.

Isosceles (io-octmX, with equal legs) is used by Plato as well as Aristotle.
ScaJene (o-praAijvos, with the variant o-mtA-irtTje) is used by Aristotle of a triangle
with no two sides equal: cf.\ also Tim. Locr. 98 b. Plato, Euthyphro 1 2 rj,

i88 BOOK I [i. Deff. 20, 21

applies the term ``scalene ``to an odd number in contrast to ``isosceles ``used
of an even number, l'roclus (p.~168, 24) seems to connect it with tntatu, to
limp; others make it akin to a-ioAto's, crooked, aslant. Apollonius uses the
same word ``scalene ``of an oblique circular cone.

Triangles are classified, first with reference to their sides, and then with
reference to their angles. l'roclus points out that seven distinct species of
triangles emerge: (1) the equilateral triangle, (2) three species of isosceles
triangles, the right-angled, the obtuse-angled and the acute-angled, (3} the
same three varieties of scalene triangles.

Proclus gives an odd reason for the dual classification according to sides
and angles, namely that Euclid was mindful of the fact that it is not every
triangle that is trilateral also. He explains this statement by reference
(p.~165, 2j) to a figure which some called barb-like («SmiSiji) while
Zenodovus called it hollow-angled ((toiAoywi'w). Proclus mentions it again
in his note on 1. 22 (p.~328, 21 sqq.) as one of the paradoxes of geometry,
observing that it is seen in the figure of that proposition. This ``triangle ``is
merely a quadrilateral with a re-entrant angle; and the idea that
it has only three angles is due to the non-recognition of the
fourth angle (which is greater than two right angles) as being an
angle at all. Since Proclus speaks of the four-sided triangle as
``one of the paradoxes in geometry,'' it is perhaps not safe to
assume that the misconception underlying the expression existed
in the mind of Proclus alone; but there does not seem to be any evidence
that Zenodorus called the figure in question a triangle (cf.\ Pappus, ed.
Hultsch, pp.~1 154, 1206).

Definition 22.

Twv Si TrrpawXcvpwv trj(jjfjidTmv Ttrpdymvov p\kv i<mv, S itroTrXtvpoy t4 lart
Kai 6p$oy<wtov, irtpofjiTjitts £i, u 6pBoywViov firv, ovx urowktvpov S«, pojiftot S(. S
UronXtvpov piv, oIk p$<rywvutr Si, pop.fio*i£is Bi to ts i.Trtvarrioir vktvpdx rt xal
yinvia icac oAAij'Aats X ov > * our* i<jQTr\cvpov i<rriv out* ApGoyvviov To 04 Trapa
TauTa TfrpdirXtvpa TpaW£ia naKttaw.

0/ quadrilateral figures, a square is that which is both equilateral and right-
angled; an oblong thai which is right-angled but not equilateral; a rhombus
that which is equilateral but not right-angled; and a rhomboid that which has
its opposite sides and angles equal te one another but is neither equilateral nor
right-angled. And let quadrilaterals ether than these be called trapezia.

Ttrpayiiivov was already a square with the Pythagoreans (cf, Aristotle,
Metapk. 986 a 26), and it is so most commonly in Aristotle; but in Dt anima
11. 3, 414 t> 31 it seems to be a quadrilateral, and in Metapk. 1054 b 2,
``equal and equiangular Ttrpdyaiva,'' it cannot be anything else but quadri-
lateral if ``equiangular'' is to have any sense. Though, by introducing
Tcrpdir\tvpav for any quadrilateral, Euclid enabled ambiguity to be avoided,
there seem to be traces of the older vague use of Ttrpdytowr in much later
writers. Thus Heron (Def. 100) speaks of a cube as ``contained by six equi-
lateral and equiangular Ttrpdywra'' and Proclus (p.~166, 10) adds to his
remark about the ``four-sided triangle ``that ``you might have rtrpaybiva with
more than the four sides,'' where rtrpayvsva can hardly mean squares.

tTtpofiyKn, oblong (with sides of different length), is also a Pythagorean term.

The word right-angled {ipBoyiiytor) as here applied to quadrilaterals
must mean rectangular (i.e., practically, having all its angles right angles);
for, although it is tempting to take the word in the same sense for a
square as for a triangle (i.e.\ ``having one right angle ``), this will not do in the
case of the oblong, which, unless it were stated that three of its angles are
right angles, would not be sufficiently defined.

If it be objected, as it was by Todhunter for example, that the definition
of a square assumes more than is necessary, since it is sufficient that, being
equilateral, it should have one right angle, the answer is that, as in other cases,
the superfluity does not matter from Euclid's point of view; on the contrary,
the more of the essential attributes of a thing that could be included in its
definition the better, provided that the existence of the thing defined and its
possession of all those attributes is proved before the definition is. actually
used; and Euclid does this in the case of the square by construction in 1. 46,
making no use of the definition before that proposition.

The word rhombus (po/t/Jot) is apparently derived from fiipfioi, to turn
round and round, and meant among other things a spinning-top.~Archimedes
uses the term solid rhombus to denote a solid figure made up of two right
cones with a common circular base and vertices turned in opposite directions.
We can of course easily imagine this solid generated by spinning; and, if the
cones were equal, the section through the common axis would be a plane
rhombus, which would also be the apparent form of the spinning solid to the
eye. The difficulty in the way of supposing the plane figure to have been
named after the solid figure is that in Archimedes the cones forming the solid
are not necessarily equal. It is however possible that the solid to which the
name was originally given was made up of two equal cones, that the plane
rhombus then received its name from that solid, and that Archimedes, in
taking up the old name again, extended its signification (cf.\ J. H. T. Miiller,
Beitrdge zur Terminologie der griechisehen Mathematiker, i860, p.~20).
Proclus, while he speaks of a rhombus as being like a shaken, i.e.\ deformed,
square, and of a rhomboid as an oblong that has been moved, tries to explain
the rhombus by reference to the appearance of a spinning square {rsrpaymvov
poftfioiittvor).

It is true that the definition of a rhomboid says more than is necessary in
describing it as having its opposite sides and angles equal to one another.
The answer to the objection is the same as the answer to the similar objection
to the definition of a square.

Euclid makes no use in the Elements of the oblong, the rhombus and
the rhomboid. The explanation of his inclusion of definitions of these
figures is no doubt that they were taken from earlier text- books. From
the words ``let quadrilaterals other than these be called trapezia'' we may
perhaps infer that trapezium was a new name or a new application of an old
name.

As Euclid has not yet defined parallel lines and does not anywhere
define a parallelogram, he is not in a position to make the more elaborate
classification of quadrilaterals attributed by Proclus to Posidonius and
appearing also in Heron's Definitions. It may be shown by the following
diagram, distinguishing seven species of quadrilaterals.

Quadrilaterals

parallelograms non- parallelograms

rectangular non-rectangular two sides parallel no sides parallel

[traptxtum] {traprtt)

r-U

squart oblong rhombus rhomboid i iGscdts trapezium scaitnt trapezium

It will be observed that, while Euclid in the above definition classes as
trapezia all quadrilaterals other than squares, oblongs, r ho in hi, and rhomboids,
the word is in this classification restricted to quadrilaterals having two sides
(only) parallel, and trapezoid is used to denote the rest Euclid appears to
have used trapezium in the restricted sense of a quadrilateral with two sides
parallel in his book, vtpi Suuptatav (on divisions of figures). Archimedes
uses it in the same sense, but in one place describes it more precisely as a
trapezium with its two sides parallel.

Definition 23.

IlapaX7]\ai turiv ciStitu, a*riMf iv to) uurw hrtwio ovat mil «*/3aXAo/j.*i'ai
tit irnpov itji' imiTtpa. Ta /Mpf in ftrfiirtpa tni/iviirTOtxTiv XXykiuf.

Parallel straight tines are straight lines which, being in the same plane and
being produced indefinitely in both directions, do not meet one another in either
direction.

Wap6XKvj\\K (alongside one another) written in one word does not appear
in Plato; but with Aristotle it was already a familiar term.

«(? aire tpok cannot be translated ``to infinity ``because these words might
seem to suggest a region or place infinitely distant, whereas •« airttpoy, which
seems to be used indifferently with «jt' Zirnpuv, is adverbial, meaning ``without
limit,'' i.e.\ ``indefinitely.'' Thus the expression is used of a magnitude being
``infinitely divisible,'' or of a series of terms extending without limit

in both directions, i<ft' tKattpa ™ fitpift literally ``towards both the parts''
where ``parts'' must be used in the sense of ``regions'' (cf Thuc. 11. 96).

It is clear that with Aristotle the general notion of parallels was that of
straight lines which do not meet, as in Euclid: thus Aristotle discusses the
question whether to think that parallels do meet should be called a
geometrical or an ungeometrical error (Anal. post. 1. 12, 77 b 22), and (more
interesting still in relation to Euclid) he observes that there is nothing
surprising in different hypotheses leading to the same error, as one might
conclude that parallels meet by starting from the assumption, either (a) that
the interior (angle) is greater than the exterior, or (b) that the angles of a
triangle make up more than two right angles (Anal, prior, u. 17, 66 a 11).

Another definition is attributed by Proclus to Posidonius, who said that
``parallel tines are those which, (being) m one plane, neither converge nor diverge,
but have all the perpendiculars equal which are drawn from the points oj one
line to tlit other, while such (straight lines) as make the perpendiculars less and
less continually do converge to one another; for the perpendicular is enough
to define (opt'f %%v jvrarcu) the heights of areas and the distances between lines.
For this reason, when the perpendiculars are equal, the distances between the
straight lines are equal, but when they become greater and less, the interval is
lessened, and the straight lines converge to one another in the direction in
which the less perpendiculars are ``(Proclus, p.~176, 6 — 17).

Posidonius' definition, with the explanation as to distances between straight
lines, their convergence and divergence, amounts to the definition quoted by
Simplicius (an-Nairlzi, p.~25, ed. Curt/.e) which described straight lines as
parallel if, when they are produced indefinitely both ways, the distance between
them, or the perpendicular drawn from either of them to the other, is always
equal and not different. To the objection that it should be proved that the
distance between two parallel lines is the perpendicular to them Simplicius
replies that the definition will do equally well if all mention of the perpen-
dicular be omitted and it be merely stated that the distance remains equal,
although ``for proving the matter in question it is necessary to say that one
straight line is perpendicular to both'' (an-NairizI, ed. Besthorn-Heiberg, p.~9).
He then quotes the definition of ``the philosopher Aganis'': ``Parallel
straight tines are straight lints, situated in the same plane, the distance between
which, if they are predated indefinitely in both directions at the same time, is
everywhere the same,'' (This definition forms the basis of the attempt of
``Aganis'' to prove the Postulate of Parallels.) On the definition Simplicius
remarks that the words ``situated in the same plane'' are perhaps unnecessary,
since, if the distance between the lines is everywhere the same, and one does
not incline at all towards the other, they must for that reason be in the same
plane. He adds that the ``distance'' referred to in the definition is the
shortest tine which joins things disjoined. Thus, between point and point,
the distance is the straight line joining them; between a point and a straight
line or between a point and a plane it is the perpendicular drawn from the point
to the tine or plane; ``as regards the distance between two lines, that distance
is, if the lines are parallel, one and the same, equal to itself at al! places on
the lines, it is the shortest distance and, at all places on the lines, perpendicular
to both'' {\ibid~p, 10).

The same idea occurs in a quotation by Proclus (p, 177, n) from
Geminus. As part of a classification of lines which do not meet he observes:
``Of lines which do not meet, some are in one plane with one another, others
not. Of those which meet and are in one plane, some are always the same
distance from one another, others lessen the distance continually, as the hyper-
bola (approaches) the straight line, and the conchoid the straight line (i.e.\ the
asymptote in each case). For these, while the distance is being continually
lessened, are continually (in the position of) not meeting, though they converge
to one another; they never converge entirely, and this is the most paradoxical
theorem in geometry, since it shows that the convergence of some lines is non-
convergent. But of lines which are always an equal distance apart, those
which are straight and never make the (distance) between them smaller, and
which are in one plane, are parallel.''

Thus the equidistance-theary of parallels (to which we shall return) is very
fully represented in antiquity. I seem also to see traces in Greek writers of a
conception equivalent to the vicious direction-theory which has been adopted
in so many modem text-books. Aristotle has an interesting, though obscure,
allusion in Anal, prior, ti. 16, 65 a 4 to a petit io principii committed by ``those
who think that they draw parallels ``(or ``establish the theory of parallels,''
which is a possible translation of ts «-apaAA»J*ovs ypdifttw): ``for they un-
consciously assume such things as it is not possible to demonstrate if parallels
do not exist'' It is clear from this that there was a vicious circle in the then
current theory of parallels; something which depended for its truth on the
properties of parallels was assumed in the actual proof of those properties,
e.g.\ that the three angles of a triangle make up two right angles. This is not
the case in Euclid, and the passage makes it clear that it was Euclid himself
who got rid of the petitio principii in earlier text-books by formulating and
premising before 1. 19 the famous Postulate 5, which must ever be regarded
as among the most epoch-making achievements in the domain of geometry.
But one of the commentators on Aristotle, Philoponus, has a note on the
above passage purporting to give the specific character of the petitio principii
alluded to; and it is here that a direction-theory of parallels may be hinted at,
whether Philoponus is or is not right in supposing that this was what Aristotle
had in mind. Philoponus says: ``The same thing is done by those who draw
parallels, namely begging the original question; for they will have it that it is
possible to draw parallel straight lines from the meridian circle, and they
assume a point, so to say, falling on the plane of that circle and thus they
draw the straight lines. And what was sought is thereby assumed; for he
who does not admit the genesis of the parallels will not admit the point
referred to either.'' What is meant is, I think, somewhat as follows. Given
a straight line and a point through which a parallel to it is to be drawn, we
are to suppose the given straight line placed in the plane of the meridian.
Then we are told to draw through the given point another straight line in the
plane of the meridian (strictly speaking it should be drawn in a plane parallel
to the plane of the meridian, but the idea is that, compared with the sue of
the meridian circle, the distance between the point and the straight line is
negligible); and this, as I read Philoponus, is supposed to be equivalent to
assuming a very distant point in the meridian plane and joining the given
point to it. But obviously no ruler would stretch to such a point, and the
objector would say that we cannot really direct a straight line to the assumed
distant point except by drawing it, without more ado, parallel to the given
straight line. And herein is the pditio principii. I am confirmed in seeing
in Philoponus an allusion to a direction-theory by a remark of Schotten on a
similar reference to the meridian plane supposed to be used by advocates of
that theory. Schotten is arguing that direction is not in itself a conception
such that you can predicate one direction of two different lines. ``If any one
should reply that nevertheless many lines can be conceived which all have the
direction from north to south,'' he replies that this represents only a nominal,
not a real, identity of direction.

Coming now to modern times, we may classify under three groups
practically all the different definitions that have been given of parallels
(Schotten, op.~cit. it p.~188 sqq.).

(i) Parallel straight lines have no point common, under which general
conception the following varieties of statement may be included:

(a) they do not cut one another,

(6) they meet at infinity, or

(c ) they have a common point at infinity.

(a) Parallel straight lines have the same, or like, direction or directions,
under which class of definitions must be included all those which introduce
transversals and say that the parallels make equal angles with a transversal.

(3) Parallel straight lines have the distance between them constant;
with which group we may connect the attempt to explain a parallel as the
geometrical focus of all points which are equidistant from a straight line.

But the three points of view have a good deal in common; some of them
lead easily to the others. Thus the idea of the lines having no point common
led to the notion of their having a common point at infinity, through the
influence of modem geometry seeking to embrace different cases under one
conception; and then again the idea of the lines having a common point at
infinity might suggest their having the same direction. The ``non-secant ``
idea would also naturally lead to that of equidistance (3}, since our
observation shows that it is things which come nearer to one another that
tend to meet, and hence, if lines are not to meet, the obvious thing is to see
that they shall not conic nearer, i.e.\ shall remain the same distance apart.

We will now take the three groups in order.

(1) The first observation of Schotten is that the varieties of this group
which regard parallels as (a) meeting at infinity or (6) having a common
point at infinity (first mentioned apparendy by Kepler, 1604, as a ``facpn de
parler ``and then used by Desargues, 1630) are at least unsuitable definitions
for elementary text-books. How do we know that the lines cut or meet at
infinity ? We are not entitled to assume either that they do or that they do
not, because ``infinity'' is outside our field of observation and we cannot verify
either. As Gauss says (letter to Schumacher), ``Finite man cannot claim to
be able to regard the infinite as something to be grasped by means of ordinary
methods of observation.'' Steiner, in speaking of the rays passing through a
point and successive paints of a straight tine, observes that as the point of
intersection gets further away the ray moves continually in one and the same
direction (``nach einer und derselben Richtung hin ``); only in one position,
that in which it is parallel to the straight line, ``there is no real cutting''
between the ray and the straight line; what we have to say is that the ray is
``directed towards the infinitely distant point on the straight line.'' It is true
that higher geometry has to assume that the lines do meet at infinity: whether
such lines exist in nature or not does not matter (just as we deal with ``straight
lines ``although there is no such thing as a straight line). But if two lines do
not cut at any finite distance, may not the same thing be true at infinity also?
Are lines conceivable which would not cut even at infinity but always remain
at the same distance from one another even there? Take the case of a line
of railway. Must the two rails meet at infinity so that a train could not stand
on them there (whether we could see it or not makes no difference)? It
seems best therefore to leave to higher geometry the conception of infinitely
distant points on a line and of two straight lines meeting at infinity, like
imaginary points of intersection, and, for the purposes of elementary geometry,
to rely on the plain distinction between ``parallel ``and ``cutting ``which
average human intelligence can readily grasp.~This is the method adopted
by Euclid in his definition, which of course belongs to the group (1) of
definitions regarding parallels as non-secant.

It is significant, I think, that such authorities as Ingram! {Elementi di
geometria, 1904) and Enriques and Amaldi {EUmenti di geometria, 1905),
after all the discussion of principles that has taken place of late years, give
definitions of parallels equivalent to Euclid's: ``those straight lines in a plane
which have not any point in common are called parallels.'' Hilbert adopts
the same point of view, Veronese, it is true, takes a different line. In his
great work Eondamenti di geometria, 1891, he had taken a ray to be parallel to
another when a point at infinity on the second is situated on the first; but he
appears to have come to the conclusion that this definition was unsuitable for
his Element i. He avoids however giving the Euclidean definition of parallels
as ``straight lines in a plane which, though produced indefinitely, never meet,''
because no one has ever seen two straight lines of this sort,'' and because
the postulate generally used in connexion with this definition is not evident in
the way that, in the field of our experience, it is evident that only one straight
line can pass through two points. Hence he gives a different definition, for
which he claims the advantage that it is independent of the plane. It is
based on a definition of figures ``opposite to one another with respect to a
point'' (or rtflex figures). ``Two figures are opposite to one another with
respect to a point O, e.g.\ the figures ABC ... and A'ffC .,., if to every point
of the one there corresponds one sole point of the other, and if the segments
OA, OB, OC, ... joining the points of one figure to O are respectively equal
and opposite to the segments OA\ OB, OC'',... joining to O the corresponding
points of the second ``: then, a transversal of two straight lines being any
segment having as its extremities one point of one line and one point of the
other, ``two straight lines are called parallel if one of them contains two points
opposite to two paints of the other with respect to the middle point of a common
transversal'' It is true, as Veronese says, that the parallels so defined and the
parallels of Euclid are in substance the same; but it can hardly be said that
the definition gives as good an idea of the essential nature of parallels as does
Euclid's. Veronese has to prove, of course, that his parallels have no point in
common, and his ``Postulate of Parallels'' can hardly be called more evident
than Euclid's: ``If two straight lines are parallel, they are figures opposite to
one another with respect to the middle points of all their transversal segments.''

(2) The direction-theory.

The fallacy of this theory has nowhere been more completely exposed
than by C. L. Dodgson (Euclid and his modern Rivals, 1879). According to
Killing (JEinfiihrung in die Grundlagen der Geometrie, 1. p.~5) it would appear
to have originated with no less a person than Leibniz. In the text-books
which employ this method the notion of direction appears to be regarded as a
primary, not a derivative notion, since no definition is given. But we ought
at least to know how the same direction or like directions can be recognised
when two different straight lines are in question. But no answer to this
question is forthcoming. The fact is that the whole idea as applied to non-
coincident straight lines is derived from knowledge of the properties of
parallels; it is a case of explaining a thing by itself. The idea of parallels
being in the same direction perhaps arose from the conception of an angle as
a difference of direction (the hoi low n ess of which has already been exposed);
sameness of direction for parallels follows from the same ``difference of
direction ``which both exhibit relatively to a third line. But this is not
enough. As Gauss said ( Werke, iv. p.~365), ``If it [identity of direction] is
recognised by the equality of the angles formed with one third straight line,
we do not yet know without an antecedent proof whether this same equality
will also be found in the angles formed with a fourth straight line ``{and any
number of other transversals); and in order to make this theory of parallels
valid, so far from getting rid of axioms such as Euclid's, you would have to
assume as an axiom what is much less axiomatic, namely that ``straight lines
which make equal corresponding angles with a certain transversal do so with
any transversal ``(Dodgson, p.~101).

(3) In modern times the conception of parallels as equidistant straight
lines was practically adopted by Clavius (the editor of Euclid, born at
Bamberg, 1537) and (according to Saccheri) by Borelli (Euctides restitutus,
1658) although they do not seem to have defined parallels in this way.
Saccheri points out that, before such a definition can be used, it has to
be proved that ``the geometrical locus of points equidistant from a straight
line is a straight line.'' To do him justice, Clavius saw this and tried to
prove it: he makes out that the locus is a straight line according to the
definition of Euclid, because ``it lies evenly with respect to all the points
on it''; but there is a confusion here, because such ``evenness'' as the locus
has is with respect to the straight line from which its points are equidistant,
and there is nothing to show that it possesses this property with respect
to itself. In fact the theorem cannot be proved without a postulate.

Postulate i.

'Hmjffn aire jravres aijfuiou irl woe (njpMOV tvSiiav ypaptpTJv dyaytiv.
Let the following be postulated: to draw a straight line from any point to
any point.

From any point to any point. In general statements of this kind
the Greeks did not say, as we do, ``any point,'' ``any triangle ``etc., but
``every point,'' ``every triangle ``and the like. Thus the words are here
literally ``from every point to every point.'' Similarly the first words of
Postulate 3 are ``with every centre and distance,'' and the enunciation, e.g., of
1. 18 is ``In every triangle the greater side subtends the greater angle.''

It will be remembered that, according to Aristotle, the geometer must in
general assume what a thing is, or its definition, but must prove that it is,
i.e.\ the existence of the thing corresponding to the definition: only in the case
of the two most primary things, points and lines, does he assume, without
proof, both the definition and the existence of the thing defined. Euclid has
indeed no separate assumption affirming the existence oi points such as we find
nowadays in text-books like those of Veronese, Ingrami, Enriques, ``there exist
distinct points'' or ``there exist an infinite number of points.'' But, as re-
gards the only lines dealt with in the Elements, straight lines and circles,
existence is asserted in Postulates 1 and 3 respectively. Postulate 1 however
does much more than (1) postulate the existence of straight lines. It is
(2) an answer to a possible objector who should say that you cannot, with the
imperfect instruments at your disposal, draw a mathematical straight line at all,
and consequently (in the words of Aristotle, A/ml. post. 1. to, 76 b 41) that
the geometer uses false hypotheses, since he calls a line a foot long when it is
not or straight when it is not straight. It would seem (if Gherard's translation
is right) that an-Nairlsd saw that one purpose of the Postulate was to refute
this criticism: ``the utility of the first three postulates is (to ensure) that the
weakness of our equipment shall not prevent (scientific) demonstration ``
(ed. CurUe, p.~30). The fact is, as Aristotle says, that the geometer's demon-
stration is not concerned with the particular imperfect straight line which he
has drawn, but with the ideal straight line of which it is the imperfect
representation. Simplidus too indicates that the object of the Postulate is
rather to enable the drawing of a mathematical straight line to be imagined
than to assert that it can actually be realised in practice: ``he would be a
rash person who, taking things as they actually are, should postulate the
drawing of a straight line from Aries to Libra.''

There is still something more that must be inferred from the Postulate
combined with the definition of a straight line, namely {3) that the straight
line joining two points is unique: in other words that, if two straight lines
(``rectilineal segments,'' as Veronese would call them) have the same extremities,
they must coincide throughout their length. The omission of Euclid to state
this in so many words, though he assumes it in 1. 4, is no doubt answerable for
the interpolation in the text of the equivalent assumption that two straight
lines cannot enclose a space, which has constantly appeared in mss. and editions
of Euclid, either among Axioms or Postulates. That Postulate 1 included it,
by conscious implication, is even clear from Proclus' words in his note on 1. 4
(p.~139, 16): ``therefore two straight lines do not enclose a space, and it was
with knowledge of this fact that the writer of the Elements said in the first of
his Postulates, to draw a straight line from any point to any point, implying
that it is one straight line which would always join the two points, not two.''

Proclus attempts in the same note (p.~339) to prove that two straight lines
cannot enclose a space, using as his basis the definition of the diameter of a
circle and the theorem, stated in it, that any diameter divides the circle into
two equal parts.

Suppose, he says, ACS, ADB to be two straight lines enclosing a space.
Produce them (beyond B) indefinitely. With centre £
and distance AB describe a circle, cutting the lines so
produced in F, E respectively.

Then, since ACBF, ADBE are both diameters
cutting off semi-circles, the arcs AE, AEF are equal:
which is impossible. Therefore etc

It will be observed, however, that the straight lines
produced are assumed to meet the circle given in two
different points E, F, whereas, for anything we know,
E, F might coincide and the straight lines have three common points. The
proof is therefore delusive.

Saccheri gives a different proof. From Euclid's definition of a straight
line as that which lies evenly with its points he infers that, when
such a line is turned about its two extremities, which remain fixed,
all the points on it must remain throughout in the same position, and
cannot take up different positions as the revolution proceeds. ``In
this view of the straight line the truth of the assertion that two
straight lines do not enclose a space is obviously involved. In fact,
if two lines are given which enclose a space, and of which the two
points A and X are the common extremities, it is easily shown that
neither, or else only one, of the two lines is straight.''

It is however better to assume as a postulate the (act, inseparably
connected with the idea of a straight line, that there exists only one straight
line containing two given points, or, if two straight lines have two points in
common, they coincide throughout.

Postulate 2.

Kai vertpao-iiinpr ivOtlav Kara to minuet <V tiStiai infiaXtuf.

To produce a finite straight line continuously in a straight line.

I translate ir«r tpaapi vyv by finite, because that is the received equivalent,
and because any alternative word such as limited, terminated, if applied to a
straight line, would equally fail to express what modem Italian geometers aptly
call a rectilineal segment, that is, a straight line having two extremities.

Just as Post. 1 asserting the possibility of drawing a straight line from any
one point to another must be held to declare at the same time that the
straight line so drawn is unique, so Post. 2 maintaining the possibility of
producing a finite straight line (a ``rectilineal segment ``) continuously in a
straight line must also be held to assert that the straight line can only be
produced in one way at either end, or that the produced part in either
direction is unique; in other words, that two straight lines cannot have a
common segment, This latter assumption is not expressly appealed to by
Euclid until XL t. But it is needed at the very beginning of Book 1. Proclus
(p.~114, 18) says that Zeno of Sidon, an Epicurean, maintained that the very
Erst proposition 1. 1 requires it to be admitted that ``two straight lines cannot
have the same segments ``; otherwise AC, BC might meet before they arrive
at C and have the rest of their length common, in which case the actual
triangle formed by them and A B would not be equilateral. The assumption
that two straight lines cannot have a common segment is certainly necessary
in 1. 4, where one side of one triangle is placed on that side of the other
triangle which is equal to it, and it is inferred that the two coincide throughout

their length: this would by no means follow if two straight lines could have a

common segment. Proclus (p.~315, 24), while observing that Post. 2 clearly

indicates that the produced portion must be one, attempts to prove it, but

unsuccessfully. Both he and Simplicius practically

use the same argument. Suppose, says Proclus,

that the straight lines AC, AD have .« as 1

common segment With centre B and radius BA

describe a circle (Post. 3) meeting AC, AD in

C, D.- Then, since ABC is a straight line through

the centre, AEC is a semi-circle. Similarly, ABD

being a straight line through the centre, A ED is a

semi-circle. Therefore AEC is equal to AED:

which is impossible.

Proclus observes that Zeno would object to this proof as really depending
on the assumption that ``two circumferences (of circles) cannot have one
portion common ``; for this, he would say, is assumed in the common proof
by superposition of the fact that a circle is bisected by a diameter, since that
proof takes it for granted that, if one part of the circumference cut off by the
diameter, when applied to the other, does not coincide with it, it must neces-
sarily fall either entirely outside or entirely inside it, whereas there is nothing
to prevent their coinciding, not altogether, but in part only; and, until you
really prove the bisection of a circle by its diameter, the above proof is not
valid. Posidonius is represented as having derided Zeno for not seeing that
the proof of the bisection of a circle by its diameter goes on just as well if the
circumferences fail to coincide in fart only. But the true objection to the
proof above given is that the proof of the bisection of a circle by any diameter
itself assumes that two straight lines cannot have a common segment; for, if
we wish to draw the diameter of a circle which has its extremity at a given point
of the circumference we have to join the latter point to the centre (Post. 1) and
then to produce the straight line so drawn till it meets the circle again (Post, a),
and it is necessary for the proof that the produced part shall be unique.

Saccheri adopted the proper order when he gave, first the proposition that
two straight lines cannot have a common segment, and after that the
proposition that any diameter of a circle bisects the circle and its circumference.

Saccheri's proof of the former is very interesting as showing the thorough-
ness of his method, if not at the end entirely convincing. It is in five stages
which I shall indicate shortly, giving the full argument of the first only.

Suppose, if possible, that AX is a common segment of both the straight
lines AXB, AXC, in one plane, produced beyond
X. Then describe about X as centre, with radius
XB or XC, the arc BMC, and draw through X to
any point on it the straight line XM.

(i) I maintain that, with the assumption
made, the lint AXM is also a straight line which
is drawn from the point A to the point X and pro-
duced beyond X.

For, if this line were not straight, we could draw
another straight line AM which for its part would
be straight. This straight line will either (a) cut one
of the two straight lines XB, XC in a certain point
A'' or (6) enclose one of them, for instance XB, in
the area bounded by AX, XM and APLM.

But the first alternative (a) obviously contradicts the foregoing lemma [that
two straight lines cannot enclose a space], since in that case the two lines
AXK, ATK, which by hypothesis are straight, would enclose a space.

The second possibility (b) is at once seen to involve a similar absurdity.
For the straight Jine X B must, when produced beyond 3, ultimately meet
APLM in a point L. Consequently the two lines AXBL, APL, which by
hypothesis are straight, would again enclose a space. If however we were to
assume that the straight line XB produced beyond B will ultimately meet
either the straight line XM or the straight line XA in another point, we should
in the same way arrive at a contradiction.

From this it obviously follows that, on the assumption made, the line
AXM is itself the straight line which was drawn from the point A to the point
M\ and that is what was maintained.

The remaining stages are in substance these.

(ii) If the straight line AXB, regarded as rigid, revolves about AX as axis,
it cannot assume two more positions in the same plane, so that, for example, in
one position XB should coincide with XC, and in the other -with XM,

[This is proved by considerations of symmetry. AXB cannot be altogether
``similar or equal to ``AXC, if viewed from the same side {left or right) of
both: otherwise they would coincide, which by hypothesis they do not. But
there is nothing to prevent AXB viewed from one side (say the left) being
``similar or equal to ``AXC viewed from the other side (i.e.\ the right), so that
AXB can, without any change, be brought into the position AXC.

AXB cannot however take the position of the other straight line AXM
well- If they were like on one side, they would coincide; if they were like on
opposite sides, AXM, AXC would be like on the same side and therefore
coincide.]

(iii) The other positions of AXB during the revolution must be above or
below the original plane.

(iv) It is next maintained that there is a point D on the are BC such that, if
XD is drawn, AXD is not only a straight line but is such that viewed from the left
side it is exactly ``similar or equal'' to what it is when vieioedfrom the right side.

[First, it is proved that points M, F can be found on the arc, corresponding
in the same way as B, C do, but nearer together, and of course AXM, AXF
are both straight lines.

Secondly, similar corresponding points can be found still nearer together,
and so on continually, until either (a) we come to one point D such that AXD
is exactly like itself when the right and left sides are compared, or (b) there are
two ultimate points of this sort M, F, so that both AXM, AXF have this
property.

Thirdly, {) is ruled out by reference to the definition of a straight line.

Hence (a) only is true, and there is only one point D such as described.]

(v) Lastly, Saccheri concludes that the straight line AXD so determined
``is alone a straight line, and the immediate prolongation from A beyond X to
D,'' relying again on the definition of a straight line as ``lying evenly.''

Simson deduced the proposition that two straight lines cannot have a
common segment as a corollary from 1, 11; but his argument is a complete
petitio principii, as shown by Tod hunter in his note on that proposition.

Proclus (p.~317, 10) records an ancient proof also based on the proposition
1. 11. Zeno, he says, propounded this proof and then criticised it.

Suppose that two straight tines AC, AD have a common segment AB, and
let BE be drawn at right angles to AC.

Then the angle EBC is right.

If then the angle EBD is also right, the two
angles will be equal; which is impossible.

If the angle EBD is not right, draw BE at right
angles to AD; therefore the angle FBA is right.

But the angle EBA is right.

Therefore the angles EBA, FBA are equal:
which is impossible.

Zeno objected to this, says Proclus, because it assumed the later pro-
position 1. 1 1 for its proof. Posidonius said that there was no trace of such
a proof to be found in the text-books of Elements, and that it was only invented
by Zeno for the purpose of slandering contemporary geometers. Posidonius
maintains further that even this proof has something to be said for it. There
must be some straight line at right angles to each of the two straight lines A C,
AD (the very definition of right angles assumes this): ``suppose then it happens
to be the straight line we have set up.'' Here then we have an ancient instance
of a defence of hypothetical construction, but in such apologetic terms (``it is
possible to say something even for this proof'') that we may conclude that in
general it would not have been accepted by geometers of that time as a
legitimate means of proving a proposition.

Todhunter proposed to deduce that ftvo straight lines cannot have a
common segment from i. 13. But this will not serve either, since, as before
mentioned, the assumption is really required for i. 4.

It is best to make it a postulate.

Postulate 3.

Kol itavTi Wirpcii ttal 8ta<mj/iaTL kvuXov ypafaaat.

To describe a circle with any centre and distance.

In this case Euclid's text has the passive of the verb: ``a circle can be
drawn ``; Proclus however has the active {yputyu) as Euclid has in the first
two Postulates.

Distance, Swo-nJfiaTt. This word, meaning '' distance ``quite generally (cf.
Arist. Metaph. 1055 a 9 ``it is between extremities that distance is greatest,''
\ibid~1056 a 36 ``things which have something between them, that is, a certain
distance ``), and also ``distance ``in the sense of ``dimension ``(as in ``space
has three dimensions, length, breadth and depth,'' Arist. Physics iv. 1, 209 a 4),
was the regular word used for describing a circle with a certain radius, the
idea being that each point of the circumference was at that distance from the
centre (cf.\ Arist Afetearologica ill. 5, 376 b 8: ``if a circle be drawn... with
distance Mil ``). The Greeks had no word corresponding to radius: if they
had to express it, they said ``(straight lines) drawn from the centre'' («u « tou
nfrrpov, Eucl. lit. Def. 1 and Prop.~26 \ Mtteorologica 11.5,362 b! has the full
phrase oi in toS kotob ayo/MW'' ypauftai).

Mr Frankland observes that it would be remarkable if, unlike Postulates 1
and 2, this Postulate implied merely what tt says, that a circle can be drawn
with any centre and distance. We may regard it, if we please, as helping to the
complete delineation of the Space which Euclid's geometry is to investigate
formally. The Postulate has the effect of removing any restriction upon the
siie of the circle. It may (1) be indefinitely small, and this implies that space
is continuous, not discrete, with an irreducible minimum distance between
contiguous points in it (a) The circle may be indefinitely large, which
implies the fundamental hypothesis of infinitude of space. This last assumed
characteristic of space is essential to the proof of 1. 16, a theorem not
universally valid in a space which is unbounded in extent but finite in size. It
would however be unsafe to suppose that Euclid foresaw the use to which his
Postulate might thus be put, or formulated it with such an intention.

Postulate 4.

That all fight angles are equal to one another.

While this Postulate asserts the essential truth that a right angle is a
determinate magnitude so that it really serves as an invariable standard by
which other (acute and obtuse) angles may be measured, much more than
this is implied, as will easily be seen from the following consideration. If the
statement is to be proved, it can only be proved by the method of applying one
pair of right angles to another and so arguing their equality. But this method
would not be valid unless on the assumption of the invariability of figures,
which would therefore have to be asserted as an antecedent postulate. Euclid
preferred to assert as a postulate, directly, the fact that all right angles are
equal; and hence his postulate must be taken as equivalent to the principle of
invariability of figures or its equivalent, the homogeneity of space.

According to Proclus, Geminus held that this Postulate should not be
classed as a postulate but as an axiom, since it does not, like the first three
Postulates, assert the possibility of some construction but expresses an essential
property of right angles. Proclus further observes (p.~i88, 8} that it is not a
postulate in Aristotle's sense either. (In this I think he is wrong, as explained
above.) Proclus himself, while regarding the assumption as axiomatic (``the
equality of right angles suggests itself even hy virtue of our common notions''),
is prepared with a proof, if such is asked for.

Let ABC, DEF be two right
angles.

If they are not equal, one of them
must be the greater, say ABC.

Then, if we apply DE to AB, EF
will fall within ABC, as BG.

Produce CB to H. Then, since
ABC is a right angle, so is ABU, and the two angles are equal (a right angle
being by definition equal to its adjacent angle).

Therefore the angle ABH is greater than the angle ABG.

Producing GB to K, we have similarly the two angles ABK, ABG both
right and equal to one another; whence the angle ABH is less than the angle
ABG.

But it is also greater: which is impossible.

Therefore etc.

A defect in this proof is the assumption that CB t GB can each be
produced only in one way, and that BK falls outside the angle ABH.

Saccheri's proof is more careful in that he premises a third lemma in
addition to those asserting <i) that two straight lines
cannot enclose a space and (*) that two straight lines
cannot have a common segment. The third lemma is:
If two straight lines AB, CXD meet one another at an
intermediate point X, they do not touch at that point, but
cut one another.

Suppose now that DA standing on BAC makes the two angles DAB,
DAC equal, so that each is a right angle by the definition.

Similarly, let LHioTm. with the straight line FHM the right angles LHF,
LHM.

Let DA, HL be equal; and sup-
pose the whole of the second figure
so laid upon the first that the point
H falls on A, and L on D.

Then the straight line FHM 'will
(by the third lemma) not touch the
straight line BC at A; it will either

(a) coincide exactly with BC, or

(0) cut it so that one of its extremities, as F, will fall above [BC] and the
other, M, below it.

If the alternative (a) is true, we have already proved the exact equality of
all rectilineal right angles.

Under alternative (e) we prove that the angle I.HF, being equal to the
angle DAF, is less than the angle DAB or DAC, and a fortiori less than the
angle DAM ox LHM; which is contrary to the hypothesis.

[Hence (a) is the only possible alternative, so that all right angles are
equal.]

Saicheri adds that it makes no difference if the. angle DAF diverges
infinitely lift It from the angle DAB. This would equally lead to a conclusion
contradicting the hypothesis.

It will be observed that Saccheri speaks of ``the exact equality of all
rectilineal right angles.'' He may have had in mind the remark of Pappus,
quoted by Proclus (p.~t8a, 1 1), that the converse of
this postulate, namely that an angle which is equal
to a right angle is also right, is not necessarily true,
unless the former angle is rectilineal. Suppose two
equal straight lines BA, BCsX right angles to one
another, and semi-circles described on BA, BC
respectively as AEB, BDC in the figure. Then,
since the semi-circles are equal, they coincide .if
applied to one another. Hence the ``angles''
BBA, DBC are equal. Add to each the ``angle ``
ABD; and it follows that the iunular angle EBD is equal to the right angle
ABC. (Similarly, if BA, BCbt inclined at an acute or obtuse angle, instead
of at a right angle, we find a Iunular angle equal to an acute or obtuse angle.)
This is one of the curiosities which Greek commentators delighted in.

Veronese, Ingrami, and Enriques and Amaldi deduce the fact that all
right angles are equal from the equivalent fact that all flat angles are equal,
which is either itself assumed as a postulate or immediately deduced from some
other postulate.

Hilbert takes quite a different line. He considers that Euclid did wrong
in placing Post 4 among ``axioms.'' He himself, after his Group in. of
Axioms containing six relating to congruence, proves several theorems about
the congruence of triangles and angles, and then deduces our Postulate.

As to the raison fHre and the place of Post. 4 one thing is quite certain.
It was essential from Euclid's point of view that it should come before Post. 5,
since the condition in the latter that a certain pair of angles are together less
than two right angles would be useless unless it were first made clear that
right angles are angles of determinate and invariable magnitude.

Postulate 5.

K HI cav cfc Sue tv@twS t\$<ltt fynriVrovtra T4V VTOf Kcii tVt TO. ctvrii ft* 1*7) ytavia s
Silo upCuS* Aa<r<rovas iroip, (uaAAo/itfas To! Sw> tWtiM i»' aiMipw <ri>/ijrtnTMif,
c£ a ,r< i,ij (tcrtv ai toJv Buo opuv eX<£crf/in'€!.

TXrt/, if a straight line falling on two straight lints make the interior angles
on the same side less than two right angles, the two straight lines, if produeed
indefinitely, meet on that side on whieh are the angles less than the tit<o right
angles.

Although Aristotle gives a clear idea of what he understood by a postulate,
he does not give any instances from geometry; still less has he any allusion
recalling the particular postulates found in Euclid. We naturally infer that
the formulation of these postulates was Euclid's own work. There is a more
positive indication of the originality of Postulate 5, since in the passage {A rial
prior. 11. 16, 65 a 4) quoted above in the note on the definition of parallels he
alludes to some petitio principii involved in the theory of parallels 'current in
his time. This reproach was removed by Euclid when he laid down this
epoch-making Postulate. When we consider the countless successive attempts
made through more than twenty centuries to prove the Postulate, many of
them by geometers of ability, we cannot but admire the genius of the man
who concluded that such a hypothesis, which he found necessary to the
validity of his whole system of geometry, was really indemonstrable.

From the very beginning, as we know from Proclus, the Postulate was
attacked as such, and attempts were made to prove it as a theorem or to get
rid of it by adopting some other definition of parallels; while in modem times
the literature of the subject is enormous. Riccardi (Saggio di una bibliografia
Euclidea, Part iv,, Bologna, 1890) has twenty quarto pages of titles of mono-
graphs relating to Post. 5 between the dates 1607 and 1887. Max Simon
{Ueber die Entwickit/ng der Elementar-geometrie im XIX. Jakrhundert, 1906)
notes that he has seen three new attempts, as late as 1891 (a century after
Gauss laid the foundation of non-Euclidean geometry), to prove the theory of
parallels independently of the Postulate. Max Simon himself (pp.~53 — 61)
gives a large number of references to books or articles on the subject and
refers to the copious information, as to contents as well as names, con-
tained in Schotten's Inhalt mid Methode des planimetrisehen Unterrichts, 11.

PP '83— 33 2 -

This note will include some account of or allusion to a few of the most
noteworthy attempts to prove the Postulate. Only those of ancient times, as
being less generally accessible, will be described at any length; shorter
references must suffice in the case of the modern geometers who have made
the most important contributions to the discussion of the Postulate and have
thereby, in particular, contributed most towards the foundation of the non-
Euclidean geometries, and here I shall make use principally of the valuable
Article 8, Sulla teoria delle paraliele e suite geometric non-euclidee (by Roberto
Bonola), in Quesiioni riguardanti le matematiehe elementari, I. pp.~147 — 363.

Proclus (p.~191, xi sqq.) states very clearly the nature of the first objec-
tions taken to the Postulate.

``This ought even to be struck out of the Postulates altogether; for it is a
theorem involving many difficulties, which Ptolemy, in a certain book, set
himself to solve, and it requires for the demonstration of it a number
of definitions as well as theorems. And the converse of it is actually
proved by Euclid himself as a theorem. It may be that some would be
deceived and would think it proper to place even the assumption in question
among the postulates as affording, in the lessening of the two right angles,
ground for an instantaneous belief that the straight lines converge and meet.
To such as these Geminus correctly replied that we have learned from the
very pioneers of this science not to have any regard to mere plausible imagin-
ings when it is a question of the reasonings to be included in our geometrical
doctrine. For Aristotle says that it is as justifiable Co ask scientific proofs of
a rhetorician as to accept mere plausibilities from a geometer; and Sim mi as is
made by Plato to say that he recognises as quacks those who fashion for
themselves proofs from probabilities. So in this case the fact that, when the
right angles are lessened, the straight lines converge is true and necessary;
but the statement that, since they converge more and more as they are pro-
duced, they will sometime meet is plausible but not necessary, in the absence
of some argument showing that this is true in the case of straight tines. For
the fact that some lines exist which approach indefinitely, but yet remain
non-secant (itrvpirrtirroi), although it seems improbable and paradoxical, is
nevertheless true and fully ascertained with regard to other species of lines.
May not then the same thing be possible in the case of straight lines which
happens in the case of the lines referred to ? Indeed, until the statement in
the Postulate is clinched by proof, the facts shown in the case of other lines
may direct our imagination the opposite way. And, though the controversial
arguments against the meeting of the straight lines should contain much that
is surprising, is there not all the more reason why we should expel from our
body of doctrine this merely plausible and unreasoned (hypothesis) ?

``It is then clear from this that we must seek a proof of the present
theorem, and that it is alien to the special character of postulates. But how
it should be proved, and by what sort of arguments the objections taken to
it should be removed, we must explain at the point where the writer of the
Elements is actually about to recall it and use it as obvious. It will be
necessary at that stage to show that its obvious character does not appear
independently of proof, but is turned by proof into matter of knowledge.''

Before passing to the attempts of Ptolemy and Proclus to prove the
Postulate, I should note here that Simplicius says (in an-Nairlzi, ed. Besthorn-
Heiberg, p.~119, ed. Curtze, p.~65) that this Postulate is by no means manifest,
but requires proof, and accordingly ``Abthiniathus ``and Diodorus had
already proved it by means of many different propositions, while Ptolemy also
had explained and proved it, using for the purpose Eucl. t. 13, 15 and 16 (or
18). The Diodorus here mentioned may be the author of the Analtmma on
which Pappus wrote a commentary. It is difficult even to frame a conjecture
as to who ``Abthiniathus'' is. In one place in the Arabic text the name
appears to be written ``Anthisathus ``(H. Suter in Zeztstkrifi fiir Math, und
Phystk, xxxviii,, hist. litt. Abth. p.~194). It has occurred to me whether he
might be Peithon, a friend of Serenus of Antinoeia (Antinoupolis) who was
long known as Serenus of Antissa. Serenus says (De sittione cyiindri, ed.
Heiberg, p.~96): ``Peithon the geometer, explaining parallels in a work of his,
was not satisfied with what Euclid said, but showed their nature more cleverly
by an example; for he says that parallel straight lines are such a thing as we
see on walls or on the ground in the shadows of pillars which are made when
either a torch or a lamp is burning behind them. And, although this has only
been matter of merriment to every one, I at least must not deride it, for the
respect I have for the author, who is my friend.'' If Peithon was known as
``of Antinoeia ``or ``of Antissa,'' the two forms of the mysterious name might
perhaps be an attempt at an equivalent; but this is no more than a guess.

Simplicity adds in full and word for word the attempt of his ``friend ``or
his ``master Aganis ``to prove the Postulate.

Proclus returns to the subject (p, 365, 5) in his note on Eucl. 1. 29. He
says that before his time a certain number of geometers had classed as a
theorem this Euclidean postulate and thought it matter for proof, and he then
proceeds to give an account of Ptolemy's argument.

Noteworthy attempts to prove the Postulate.

Ptolemy.

We learn from Proclus (p.~365, 7 — -11) that Ptolemy wrote a book on the
proposition that ``straight lines drawn from angles less than two right angles
meet if produced,'' and that he used in his ``proof'' many of the theorems in
Euclid preceding 1. 19. Proclus excuses himself from reproducing the early
part of Ptolemy's argument, only mentioning as one of the propositions
proved in it the theorem of Eucl. 1. 28 that, if two straight lines meeting a
transversal make the two interior angles on the same side equal to two right
angles, the straight lines do not meet, however far produced.

I. From Proclus' note on \prop{1}{28} (p.~362, 14 sq.) we know that Ptolemy
proved this somewhat as follows.

Suppose that there are two straight lines A B, • CD, and that EFGff,
meeting them, makes the angles BFG, FGD equal to two right angles.
I say that AB, CD are parallel, that is, they
are non-secant.

For, if possible, let FB, GD meet at K.'

Now, since the angles BFG, FGD are
equal to two right angles, while the four
angles AFG, BFG, FGD, FGC are together
equal to four right angles,

the angles AFG, FGC are equal to two
right angles.

``If therefore FB, GD, when the interior angles are equal to two right
angles, meet at K, the straight lines FA, GC will also meet if produced ',- for the
angles AFG, CGFare also equal to two right angles.

``Therefore the straight lines will either meet in both directions or in
neither direction, if the two pairs of interior angles are both equal to two right
angles.

``Let, then, FA, GC meet at L.

``Therefore the straight lines LABK, LCDK enclose a space; which is
impossible.

``Therefore it is not possible for two straight lines to meet when the
interior angles are equal to two right angles. Therefore they are parallel.''

[The argument in the words italicised would be clearer if it'' had been
shown that the two interior angles on one side of EH are severally equal to the
two interior angles on the other, namely BFG to CGF and FGD to AFG;
whence, assuming FB, GD to meet in K, we can take the triangle KFG and
place it (e.g.\ by rotating it in the plane about O the middle point of FG) so
that FG falls where GFis in the figure and GD falls on FA, in which case
FB must also fall on GC; hence, since FB, GD meet at K, GC and FA
must meet at a corresponding point L, Or, as Mr Frank land does, we may
substitute for FG a straight line MN through the middle point of FG
drawn perpendicular to one of the parallels, say AB. Then, since the two
triangles OMF, ONG have two angles equal respectively, namely FOM to
GON(i. 15) and OFM to 0GJV; and one side OF equal to one side OG, the
triangles are congruent, the angle ONG is a right angle, and MN is perpen-
dicular to both AB and CD. Then, by the same method of application,
MA, NC are shown to form with MN a triangle MALCN congruent with
the triangle NDKBM, and MA, NC meet at a point L corresponding to K.
Thus the two straight lines would meet at the two points K, L, This is what
happens under the Riemann hypothesis, where the axiom that two straight
lines cannot enclose a space does not hold, but all straight lines meeting in
one point have another point common also, and e,g. in the particular figure
just used K, L are points common to all perpendiculars to MN. If we
suppose that K, L are not distinct points, but one point, the axiom that two
straight lines cannot enclose a space is not contradicted.]

II. Ptolemy now tries to prove 1. 29 without using our Postulate, and
then deduces the Postulate from it (Proclus, pp.~365, 14 — 367, 37).

The argument to prove 1. 29 is as follows.

The straight line which cuts the parallels must make the sum of the
interior angles on the same side equal to, greater
than, or less than, two right angles. b £ P

``Let AB, CD be parallel, and let FG meet
them. I say (1) that FG does not make the
interior angles on the same side greater than two £—
right angles.

``For, if the angles AFG, CGF are greater than two right angles, the
remaining angles BFG, DGF are less than two right angles.

``But the same two angles are also greater than two right angles; for AF,
CG are no more parallel than FB, GD, so that, if the straight line falling on
AF, CG makes the interior angles greater than two right angles, the straight line
falling on FB, GD will also make the interior angles greater than two right
angles.

``But the same angles are also less than two right angles; for the four
angles AFG, CGF, BFG, DGF are equal to four right angles:
which fa impossible

*' Similarly (2) we can show that the straight line falling on the parallels
does not make the interior angles on the same side less than two right angles.

``But (3), if it makes them neither greater nor less than two right angles,
it can only make the interior angles on the same side equal to two right
angles.''

III, Ptolemy deduces Post 5 thus:

Suppose that the straight lines making angles with a transversa) less than
two right angles do not meet on the side on which those angles are.

Then, a fortiori, they will not meet on the other side on which are the
angles greater than two right angles.

Hence /he straight lines will not meet in either direction; they are there-
fore parallel.

But, if so, the angles made by them with the transversal are equal to two
right angles, by the preceding proposition (= 1. 29).

Therefore the same angles will be both equal to and less than two right
angles:
which is impossible.

Hence the straight lines will meet.

IV. Ptolemy lastly enforces his conclusion that the straight lines will
meet on the side on which are the angles less than two right angles by recurring
to the a fortiori step in the foregoing proof.

Let the angles AFG, CGF in the accompanying figure be together less
than two right angles.

Therefore the angles BFG, DGF are greater
than two right angles.

We have proved that the straight lines are not
non-secant.

If they meet, they must meet either towards
A, C, or towards B, D.

(i) Suppose they meet towards B, £>, at K.

Then, since the angles AFG, CGFsze. less than
two right angles, and the angles AFG, GFB are
equal to two right angles, take away the common angle AFG, and

the angle CGF is less than the angle BFG;

that is, the exterior angle of the triangle KFG is less than the interior arid
opposite angle BFG:
which is impossible.

Therefore AB, CD do not meet towards B, D.

(v) But they do meet, and therefore they must meet in one direction or
the other:

therefore they meet towards A, B, that is, on the side where are the
angles less than two right angles.

The flaw in Ptolemy's argument is of course in the part of his proof of
i. 2 9 which I have italicised. As Proclus says, he is not entitled to assume
that, if AB, CD are parallel, whatever is true of the interior angles on one
side of FG (i.e.\ that they are together equal to, greater than, or less than, two
right angles} is necessarily true at the same time of the interior angles on the
other side. Ptolemy justifies this by saying that FA, GC are no more parallel
in one direction than FB, GD are in the other: which is equivalent to the
assumption that through any point only one parallel can be drawn to a given
straight line. That is, he assumes an equivalent of the very Postulate he is
endeavouring to prove.

Proclus.

Before passing to his own attempt at a proof, Proclus (p.~368, 26 sqq.)
examines an ingenious argument (recalling somewhat the famous one about
Achilles and the tortoise) which appeared to show that it was impossible for
the lines described in the Postulate to meet.

Let AB, CD make with AC the angles BAC, ACD together less than
two right angles.

Bisect AC a.t E and along AB, CD
respectively measure AF, CG so that each
is equal to AE. gl \ |h

Bisect FG at K and mark off FK,
GL each equal to FH; and so on.

Then AF, CG will not meet at any
point on FG; for, if that were the case, two sides of a triangle would be
together equal to the third: which is impossible.

Similarly, AB, CD will not meet at any point on KL; and ``proceeding
like this indefinitely, joining the non-coincident points, bisecting the lines so
drawn, and cutting off from the straight lines portions equal to the half of
these, they say they thereby prove that the straight lines AB, CD will not
meet anywhere.''

It is not surprising that Proclus does not succeed in exposing the fallacy
here (the fact being thai the process will indeed be endless, and yet the straight
lines will intersect within a finite distance). But Proclus' criticism contains
nevertheless something of value. He says that the argument will prove too
much, since we have only to join A G in order to see that straight lines making
svme angles which are together less than two right angles do in fact meet,
namely AG, CG. ``Therefore it is not possible to assert, without some definite
limitation, that the straight lines produced from angles less than two right
angles do not meet. On the contrary, it is manifest that MM straight lines,
when produced from angles less than two right angles, do meet, although the
argument seems to require it to be proved that this property belongs to all
such straight lines. For one might say that, the lessening of the two right
angles being subject to no limitation, with such and such an amount of
lessening the straight tines remain non-secant, but with an amount of lessening
in excess of this they meet (p.~371, 2 — 10).''

[Here then we have the germ of such an idea as that worked out by
Lobachewsky, namely that the straight lines issuing from a point in a plane
can be divided with reference to a straight line lying in that plane into two
classes, ``secant'' and ``non-secant,'' and that we may define as parallel the
two straight lines which divide the secant from the non-secant class.]

Proclus goes on (p.~371, io) to base his own argument upon ``an axiom
such as Aristotle too used in arguing that the universe is finite. For, if from
one point two straight lines forming an angle be produced indefinitely, the distance
(StaiTTairit, Arist. htaimjjta) between the said straight tines produced indefinitely
will exceed any finite magnitude. Aristotle at all events showed that, if the
Straight lines drawn from the centre to the circumference are infinite, the
interval between them is infinite. For, if it is finite, it is impossible to
increase the distance, so that the straight lines {the radii) are not infinite.
Hence the straight lines, when produced indefinitely, will be at a distance from
one another greater than any assumed finite magnitude.''

This is a fair representation of Aristotle's argument in De caelo ]. 5, 271
b 28, although of course it is not a proof of what Proclus assumes as an
axiom.

This being premised, Proclus proceeds (p.~371, 24):

I. ``I say that, if any straight line cuts one of two parallels, it will cut
the other also.

``For let AB, CD be parallel, and let EFG cut AB; I say that it will cut
CD also.

``For, since BF, FG are two straight lines from .E

one point F, they have, when produced indefinitely, ft X B

a distance greater than any magnitude, so that it will \

also be greater than the interval between the parallels. ``

Whenever therefore they are at a distance from one

another greater than the distance between the parallels,
FG will cut CD.

``Therefore etc.''

II. ``Having proved this, we shall prove, as a deduction from it, the
theorem in question.

``For let AB, CD be two straight lines, and let EF falling on them make
the angles BEF, DFE less than two right angles.

``I say that the straight lines will meet on that
side on which are the angles less than two right
angles.

``For, since the angles BEF, DFE are less
than two right angles, let the angle HEB be equal
to the excess of two right angles (over them), and let HE be produced to K.

``Since then EF falls on KH, CD and makes the two interior angles
HEF, DFE equal to two right angles,

the straight lines MX, CD are parallel.

``And AB cuts KH\ therefore it will also cut CD, by what was before
shown.

``Therefore AB, CD will meet on that side on which are the angles less
than two right angles.

``Hence the theorem is proved.''

Clavius criticised this proof on the ground that the axiom from which
it starts, taken from Aristotle, itself requires proof. He points out that, just
as you cannot assume that two lines which continually approach one another
will meet (witness the hyperbola and its asymptote), so you cannot assume
that two lines which continually diverge will ultimately be so far apart that a
perpendicular from a point on one let fall on the other will be greater than
any assigned distance; and he refers to the conchoid of Nkomedes, which
continually approaches its asymptote, and therefore continually gets farther
away from the tangent at the vertex • yet the perpendicular from any point on
the curve to that tangent will always be less than the distance between the
tangent and the asymptote. Saccheri supports the objection.

Proclus' first proposition is open to the objection that it assumes that two
``parallels'' (in the Euclidean sense) or, as we may say, two straight liius
which have a common perpendicular, are (not necessarily equidistant, but)
so related that, when they are produced indefinitely, the perpendicular from a
point of one upon the other remains finite.

This last assumption is incorrect on the hyperbolic hypothesis; the
``axiom'' taken from Aristotle does not hold on the elliptic hypothesis,

Nasiraddln at.-TusI.

The Persian-bom editor of Euclid, whose date is 1201— 1274, has three
lemmas leading up to the final proposition. Their content is substantially as
follows, the first lemma being apparently assumed as evident

I. (o) If AB, CD be two straight lines such that successive perpen-
diculars, as EF, GH, KL, from points on AB to CD always make with A3
unequal angles, which are always acute on the side towards B and always
obtuse on the side towards A, then the lines AB,
CD, so long as they do not cut, approach continually
nearer in the direction of the acute angles and diverge
continually in the direction of the obtuse angles, and
the perpendiculars diminish towards B, D, and in-
crease towards A, C.

(6) Conversely, if the perpendiculars so drawn

L H F C

continually become shorter in the direction of B, D, and longer in the
direction of A, C, the straight lines AB, CD approach continually nearer in
the direction of B, D and diverge continually in the other direction; also
each perpendicular will make with AB two angles one of which is acute and
the other is obtuse, and all the acute angles will lie in the direction towards
B, D, and the obtuse angles in the opposite direction.

[Saccheri points out that even the first part {a) requires proof. As
regards the converse (b) he asks, why should not the successive acute angles
made by the perpendiculars with AB, while remaining acute, become greater
and greater as the perpendiculars become smaller until we arrive at last at a
perpendicular which is a common perpendicular to both lines? If that happens,
alt the author's efforts are in vain. And, if you are to assume the truth of the
statement in the lemma without proof, would it not, as Wall is said, be as
easy to assume as axiomatic the statement in Post. 5 without more ado?]

II. AC, BI) be drawn from the extremities of AB at right angles to it
and on the same side, and if AC, EDfc made equal to one another and CD be
joined, each of the angles ACD, BDC will be right, and
CD will be equal to AB,

The first part of this lemma is proved by redwtio ad
absurdum from the preceding lemma. If, e.g., the angle
A CD is not right, it must either be acute or obtuse.

Suppose it is acute; then, by lemma 1, A C is greater
than BD, which '\s contrary to the hypothesis. And so on.

The angles ACD, BDC being proved to be right angles, it is easy to
prove that AB, CD are equal.

[It is of course assumed in this ``proof'' that, if the angle ACD is acute,
the angle BDC is obtuse, and vice versa.]

III. /// any triangle the three angles are together equal to two right angles.
This is proved for a right-angled triangle by means of the foregoing lemma,

the four angles of the quadrilateral ABCD of that lemma being all right angles.
The proposition is then true for any triangle, since any triangle can be divided
into two right-angled triangles

IV. Here we have the final ``proof ``of Post. 5. Three cases are
distinguished, 'but it is enough to show the case where one of the interior
angles is right and the other acute.

Suppose AB, CD to be two straight lines met by FCE making the angle
ECD a right angle and the angle CEB
an acute angle.

'lake any point G on EB, and draw
GH perpendicular to EC.

Since the angle CEG is acute, the
perpendicular GH will fall on the side of
E towards D, and will either coincide
with CD or not coincide with it. In the
former case the proposition is proved.

If GH does not coincide with CD
but falls on the side of it towards F, CD, being within the triangle formed by
the perpendicular and by CE, EG, must cut EG. [An axiom is here used,
namely that, if CD be produced far enough, it must pass outside the triangle
and therefore cut some side, which must be EB, since it cannot be the
perpendicular (1. 27), or CE.)

Lasdy, let C.fffall on the side of CD towards E.

aro BOOK I [i. Post. 5

Along HC set off HK, KL etc., each equal to EH, until we get the first
point of division, as M, beyond C.

Along GB set off GN, NO etc, each equal to EG, until EP is the same
multiple of EG that EM is of EH.

Then we can prove that the perpendiculars from N, 0, P on EC fall on
the points K, L, M respectively.

For take the first perpendicular, that from N, and call it NS.

Draw EQaa right angles to EH and equal to GH, and set off SR along
SiValso equal to GH, Join QG, GR,

Then (second lemma) the angles EQG, QGHaie right, and QG = EH.

Similarly the angles SRG, RGHwe right, and RG-SH

Thus RGQ is one straight line, and the vertically opposite angles NGR,
EGQ are equal. The angles NRG, EQG are both right, and NG = GE, by
construction.

Therefore (1. 26'') EG = GQ;

whence SH= HE = KH, and S coincides with K.

We may proceed similarly with the other perpendiculars.

Thus PM is perpendicular to EE. Hence CD, being parallel to MP and
within the triangle PME, must cut EP, if produced far enough.

John Wallis.

As is well known, the argument of Wallis (1616—1703) assumed as a
postulate that, given a figure, another figure is possible which is similar to (he
given one and of any sine whatever. In fact Wallis assumed this for triangles
only. He first proved (1) that, if a finite straight line is placed on an infinite
straight line, and is then moved in its own direction as far as we please,
it will always lie on the same infinite straight line, (2) that, if an angle be
moved so that one leg always slides along an infinite straight line, the angle
will remain the same, or equal, (3) that, if two straight lines, cut by a third,
make the interior angles on the same side less than two right angles, each
of the exterior angles is greater than the opposite
interior angle (proved by means of 1. 13). p x

(4) UAB, CD make, with AC, the interior
angles less than two right angles, suppose AC

(with AB rigidly attached to it) to move along _  \- g

AF to the position ay, such that a coincides

with C. If AB then takes the position aft, o£ lies entirely outside CD (proved
by means of {3) above).

(5) With the same hypotheses, the straight line aft, sr AB, during Us
motion, and before a reaches C, must cut the straight tine CD.

S6) Here is enunciated the postulate stated above.
7-) Postulate S is now proved thus.

Let AB, CD be the straight lines which make, with the infinite straight
line ACF meeting them, the interior angles
BA C, DC A together less than two right angles.

Suppose AC (with AB rigidly attached to
it) to move along ACF until AB takes the
position of aft cutting CD in it.

Then, «CVr being a triangle, we can, by
the above postulate, suppose a triangle drawn
on the base CA similar to the triangle aCVr.

Let it be ACF.

[Wallis here interposes a defence of the hypothetical construction.]

Thus CP and AP meet at P; and, as by the definition of similar figures
the angles of the triangles PCA, rCa are respectively equal, the angle PCA
being equal to the angle rCa and the angle PAC to the angle miCor BAC,
it follows that CP, APMe on CD, A3 produced respectively.

Hence AB, CD meet on the side on which are the angles less than two
right angles.

[The whole gist of this proof lies in the assumed postulate as to the
existence of similar figures; and, as Saccheri points out, this is equivalent to
unconditionally assuming the ``hypothesis of the right angle,'' and consequently
Euclid's Postulate 5.]

Gerolamo Saccheri.

The book Euclides ab omni naevo vindicatus (1733) by GeTolamo Saccheri
{1667 — 1733), a Jesuit, and professor at the University of Pavia, is now
accessible (1) edited in German by Engel and Stackel, Die Theorie dtr
Parallellinien von Euhtid bis auf Gauss, 1895, pp.~41 — 136, and (2) in an
Italian version, abridged but annotated, L'Euclide emendato del p.~Gerolamo
Saccheri, by G. Boccardini (Hoepli, Milan, 1904}. It is of much greater
importance than all the earlier attempts to prove Post. 5 because Saccheri
was the first to contemplate the possibility of hypotheses other than that of
Euclid, and to work out a number of consequences of those hypotheses.
He was therefore a true precursor of Legendie and of Lobachewsky, as
Beltrami called him (1889), and, it might be added, of Riemann also. For,
as Veronese observes {Fondamenti di geometria, p, 570), Saccheri obtained
a glimpse of the theory of parallels in all its generality, while Legendre,
Lobachewsky and G, Bolyai excluded a priori, without knowing it, the ``hypo-
thesis of the obtuse angle,'' or the Riemann hypothesis. Saccheri, however,
was the victim of the preconceived notion of his time that the sole possible
geometry was the Euclidean, and he presents the curious spectacle of a man
laboriously erecting a structure upon new foundations for the very purpose of
demolishing it afterwards; he sought for contradictions in the heart of the
systems which he constructed, in order to prove thereby the falsity of his
hypotheses.

For the purpose of formulating his hypotheses he takes a plane quadri-
lateral ABDC, two opposite sides of which, A C, BD,
are equal and perpendicular to a third AB. Then the
angles at C and D are easily proved to be equal. On
the Euclidean hypothesis they are both right angles;
but apart from this hypothesis they might be both
obtuse or both acute. To the three possibilities, whicfc
Saccheri distinguishes by the names (1) the hypothesis of
the right angle, (i) the hypothesis of the obtuse angle and
(3) the hypothesis of the acute angle respectively, there corresponds a certain
group of theorems; and Saccheri's point of view is that the Postulate will
be completely proved if the consequences which follow from the last two
hypotheses comprise results inconsistent with one another.

Among the most important of his propositions are the following:

(1) If the hypothesis of the right angle, or of the obtuse angle, or of the acute
angle is proved true in a single case, it is true in every other case. (Props, v.,

VI., VII.)

(2) According as the hypothesis of the right angle, the obtuse angle, or the
acute angle is true, the sum of the thru angles of a triangle is equal to, greater
than, or less than two right angles. (Prop.~i\. )

lit BOOK I [i. Post. 5

(3) From the existence of a single triangle in which the sum of the angles is
equal to, greater than, or less than two right angles (lie truth of the hypothesis
of the right angle, obtuse angle, or acute angle respectively follows. (Prop, xv.)

These propositions involve the following: If in a single triangle the sum
of the angles is equal to, greater than, or less than two right angles, then any
triangle has the sum of its angles equal to, greater than, or less than tlt'O right
angles respectively, which was proved about a century later by Legendre for
the two cases only where the sum is equal to or less than two right angles.

The proofs are not free from imperfections, as when, in the proofs of
Prop.~xii. and the part of Prop.~xm. relating to the hypothesis of the obtuse
angle, Saccheri uses Eucl. 1. 18 depending on 1. 16, a proposition which is
only valid on the assumption that straight lines are infinite in length; for this
assumption itself does not hold under the hypothesis of the obtuse angle
(the Riemann hypothesis).

The hypothesis of the acute angle takes Saccheri much longer to dispose
of, and this part of the book is less satisfactory; but it contains the following
propositions afterwards established anew by Lobachewsky and Bolyai, viz.:

(4) Two straight lines in a plane {even on the hypothesis of the acute
angle) either have a common perpendicular, or must, if produced in one and the
same direction, either intersect once at a finite distance or at least continually
approach one another. {Prop, xxin.)

(5) In a cluster of rays issuing from a point there exist always (on the
hypothesis of the acute angle) two determinate straight lines which separate the
straight lines which intersect a fixed straight line from those which do not
intersect it, ending with and including the straight line which has a common
perpendicular with the fixed straight line. (Props. XXX., xxxc, xxxii.)

Lambert.

A dissertation by G.S. Kliigel, Conatuum praecipuorum tlteoriamparallelarnm
demonstrandi recensio (1 763), contained an examination of some thirty ``demon-
strations'' of Post. 5 and is remarkable for its conclusion expressing, apparently
for the first time, doubt as to its demenstrability and observing that the
certainty which we have in us of the truth of the Euclidean hypothesis is
not the result of a series of rigorous deductions but rather of experimental
observations. It also had the greater merit that it called the attention of
Johann Heinrich Lambert (1728—1777) to the theory of parallels. His
Theory of Parallels was written in 1766 and published after his death by
G. Bernoulli and C. F. Hindenburg; it is reproduced by Engel and Stackel
(op.~sit. pp.~151 — 208).

The third part of Lambert's tract is devoted to the discussion of the same
three hypotheses as Saccheri's, the hypothesis of the right angle being for
Lambert the first, that of the obtuse angle the second, and that of the acute
angle the third, hypothesis; and, with reference to a quadrilateral with three
right angles from which Lambert starts (that is, one of the halves into which
the median divides Saccheri's quadrilateral), the three hypotheses are the
assumptions that the fourth angle is a right angle, an obtuse angle, or an
acute angle respectively.

Lambert goes much further than Saccheri in the deduction of new
propositions from the second and third hypotheses. The most remarkable is
the following.

The area of a plane triangle, under the second and third hypotheses, is
proportional to the difference between the sum vf the three angles and two right
angles.

Thus the numerical expression for the area of a triangle is, under the
third hypothesis

.=;A( r -A-£-C) (1),

and under the second hypothesis

. = *(A + £+C-*) (a),

where A is a positive constant

A remarkable observation is appended (5 82): ``In connexion with this it
seems to be remarkable that the second hypothesis holds if spherical instead of
plane triangles are taken, because in the former also the sum of the angles is
greater than two right angles, and the excess is proportional to the area of the
triangle.

``It appears still more remarkable that what I here assert of spherical
triangles can be proved independently of the difficulty of parallels.

This discovery that the second hypothesis is realised on the surface of a
sphere is important in view of the development, later, of the Riemann
hypothesis (1854).

Still more remarkable is the following prophetic sentence: ``I am almost
inclined to draw the conclusion that the third hypothesis arises with an imaginary
spherical surface'' (cf.\ Lobachewsky's Gcome'trie imaginaire, 1837).

No doubt Lambert was confirmed in this by the fact that, in the formula
(1) above, which, for h = r 1 , represents the area of a spherical triangle, if
r V- 1 is substituted for r, and r 1 = k, we obtain the formula (1).

Legend re.

No account of our present subject would be complete without a full
reference to what is of permanent value in the investigations of Adrien Marie
Legendre {1752 — 1833) relating to the theory of parallels, which extended over
the space of a generation. His different attempts to prove the Euclidean
hypothesis appeared in the successive editions of his aliments de Giomklrie
from the first {1794) to the twelfth {1823), which last may be said to contain
his last word on the subject. Later, in 1833, he published, in the Afhnoires
de I'Acadimie Royals des Sciences, xn. p.~367 sqq., a collection of his different
proofs under the title Reflexions sur dffirentes maniires de dhnontrer la thiorie
des paralteles. His exposition brought out clearly, as Saccheri had done, and
kept steadily in view, the essential connexion between the theory of parallels
and the sum of the angles of a triangle. In the first edition of the Elements
the proposition that the sum of the angles of a triangle is equal to two right
angles was proved analytically on the basis of the assumption that the choice
of a unit of length does not affect the correctness of the proposition to be
proved, which is of course equivalent to Wallis' assumption of the existence of
similar figures. A similar analytical proof is given in thj notes to the twelfth
edition. In his second edition Legendre proved Postulate 5 by means of the
assumption that, given three points not in a straight line, there exists a circle
passing through all three. In the third edition (1800) he gave the proposition
that the sum of the angles of a triangle is not greater than two right angles;
this proof, which was geometrical, was replaced later by another, the best
known, depending on a construction like that of Euclid 1. 16, the continued
application of which enables any number of successive triangles to be evolved
in which, while the sum of the angles in each remains always equal to the
sum of the angles of the original triangle, one of the angles increases and the
sum of the other two diminishes continually. But Legendre found the proof
of the equally necessary proposition that the sum of the angles of a triangle is
not less than two right angles to present great difficulties. He first observed
that, as in the case of spherical triangles (in which the sum of the angles is
greater than two right angles) the excess of the sura of the angles over two
right angles is proportional to the area of the triangle, so in the case of
rectilineal triangles, if the sum of the angles is less than two right angles by a
Mrtain deficit, the deficit will be proportional to the area of the triangle.
Hence if, starting from a given triangle, we could construct another triangle
in which the original triangle is contained at least m times, the deficit of this
new triangle will be equal to at least m times that of the original triangle, so
that the sum of the angles of the greater triangle will diminish progressively
as m increases, until it becomes zero or negative: which is absurd. The
whole difficulty was thus reduced to that of the construction of a triangle
containing the given triangle at least twice; but the solution of even this
simple problem requires it to be assumed (or proved) that through a given
point within a given angle less than two-thirds of a right angle we can always
draw a straight tine which shall meet both sides of the angle. This is however
really equivalent to Euclid's Postulate. The proof in the course of which the
necessity for the assumption appeared is as follows.

It is required to prove that the sum of the angles of a triangle cannot be
less than two right angles.

Suppose A is the least of the three angles of a triangle ABC. Apply to
the opposite side 2?C a triangle DBC, equal to
the triangle ACB, and such that the angle
DBC is equal to the angle ACB, and the angle
DCB to the angle ABC; and draw any straight
line through D cutting AB, AC produced in
E, F.

If now the sum of the angles of the triangle
ABC is less than two right angles, being equal
to aB-i say, the sum of the angles of the triangle DBC, equal to the
triangte ABC, is also 2-8.

Since the sum of the three angles of the remaining triangles DEB, FDC
respectively cannot at all events be greater than two right angles [for I>egendre's
proofs of this see below], the sum of the twelve angles of the four triangles in
the figure cannot be greater than

4B + {2B - ) + (2B - ty, i.e.\ %R-al.

Now the sum of the three angles at each of the points B, C, D is iR.

Subtracting these nine angles, we have the result that the three angles of
the triangle AEF cannot be greater than 2R - 28.

Hence, if the sum of the angles of the triangle ABC is less than two right
angles by £, the sum of the angles of the- larger triangle AEF is less than two
right angles by at least 28.

We can continue the construction, making a still larger triangle from AEF,
and so on.

But, however small 8 is, we can arrive at a multiple 2*$ which shall exceed
any given angle and therefore tR itself; so that the sum of the three angles
of a triangle sufficiently large would be zero or even less than zero: which is
absurd.

Therefore etc.

The difficulty caused by the necessity of making the above-mentioned
assumption made Legendre abandon, in his ninth edition, the method of the
editions from the third to the eighth and return to Euclid's method pure and
simple.

But again, in the twelfth, he returned to the plan of constructing any
number of successive triangles such that the sum of the three angles in all of
them remains equal to the sum of the three angles of the original triangle,
but two of the angles of the new triangles become smaller and smaller, while
the third becomes larger and larger; and this time he claims to prove in one
proposition that the sum of the three angles of the original triangle is equal to
two right angles by continuing the construction of new triangles indefinitely
and compressing the two smaller angles of the ultimate triangle into nothing,
while the third angle is made to become a flat angle at the same time. The
construction and attempted proof are as follows.

Let ABC be the given triangle; let A B be the greatest side and BC the
least; therefore C is the greatest angle and A the least.

From A draw AD to the middle point of BC, and produce AD to C,
making AC equal to AB,

Produce AB to B 1 , making AB equal to twice AD.

The triangle ABC is then such that the sum of its three angles is equal
to the sum of the three angles of the triangle ABC.

For take AK along AB equal to AD, and join C'K.

Then the triangles ABD, ACK havt two sides and the included angles
respectively equal, and are therefore equal in all respects; and C'K is equal to
BD or DC.

Next, in the triangles BCK, A CD, the angles BKC, ADC are equal,
being respectively supplementary to the equal angles AKC, ADB; and the
two sides about the equal angles are respectively equal;

therefore the triangles BC'K, A CD are equal in all respects.

Thus the angle AC'B is the sum of two angles respectively equal to the
angles B, C of the original triangle; and the angle A in the original triangle
is the sum of two angles respectively equal to the angles at A and B' in the
triangle ABC.

It follows that the sum of the three angles of the new triangle ABC is
equal to the sum of the angles of the triangle ABC.

Moreover, the side AC, being equal to AB, and therefore greater than
AC, is greater than BC which is equal to AC.

Hence the angle C'AB'w less than the angle ABC; so that the angle
CAB is less than \A, where A denotes the angle CAB of the original
triangle.

[It will be observed that the triangle ABC is really the same triangle as
the triangle ABB obtained by the construction of Eucl. 1. 16, but differently
placed so that the longest side lies along AB.]

By taking the middle point D of the side BC and repeating the same
construction, we obtain a triangle AB'C'' such that (1) the sum of its three
angles is equal to the sum of the three angles of ABC, (a) the sum of the
two angles CAB'', AB''C'' is equal to the angle CAB in the preceding
triangle, and is therefore less than \A, and (3) the angle CAB' is less than
half the angle CAB, and therefore less than \A.

Continuing in this way, we shall obtain a triangle Abe such that the sum of

two angles, those at A and i, is less than — A, and the angle at c is greater

than the corresponding angle in the preceding triangle.

If, Legendre argues, the construction be continued indefinitely so that

- n A becomes smaller than any assigned angle, the point c ultimately ties on

Alt, and the sum of the three angles of the triangle (which is equal to the sum
of the three angles of the original triangle) becomes identical with the angle
at c, which is then a.Jlat angle, and therefore equal to two right angles.

This proof was however shown to be unsound (in respect of the final
inference) by J. p.~W. Stein in Gergonne's Annaln de Mathimatiques XV.,
1824, pp.~77—79.

We will now reproduce shortly the substance of the theorems of Legendre
which are of the most permanent value as not depending on a particular
hypothesis as regards parallels.

I. The sum of the three angles of a triangle cannot it greater than two
right angles.

This Legendre proved in two ways.

(r) Mr si proof (in the third edition of the Aliments).

Let ABC be the given triangle, and ACf a straight line.

Make CE equal to AC, the angle DCE equal to the angle BAC, and DC
equal to AB. Join DE.

Then the triangle DCE is equal to the triangle BAC in all respects.

If then the sum of the three angles of the triangle ABC is greater than

%R, the said sum must be greater than the sum of the angles BCA, BCD,
DCE, which sum is equal to 2H.

Subtracting the equal angles on both sides, we have the result that

the angle ABC is greater than the angle BCD.

But the two sides AB, BC of the triangle ABC are respectively equal to
the two sides DC, CB of the triangle BCD.

Therefore the base AC is greater than the base BD (Eucl. 1. 14).

Next, make the triangle BEG {by the same construction) equal in all
respects to the triangle BAC or DCE; and we prove in the same way that
CE (or AC) is greater than DE.

And, at the same time, BD is equal to DE, because the angles BCD,
DEE are equal.

Continuing the construction of further triangles, however small the
difference between AC and BD is, we shall ultimately reach some multiple
of this difference, represented in the figure by (say) the difference between
the straight line AJ and the composite line BDFHK, which will be greater
than any assigned length, and greater therefore than the sum of AB and JJC.

Hence, on the assumption that the sum of the angles of the triangle ABC
is greater than 2R, the broken line ABDFHKJ may be less than the straight
Hne AJ: which is impossible.

Therefore etc.

(2) Proof substituted later.

If possible, let 2 + a be the sum of the three angles of the triangle ABC,
of which A is not greater than either of the
others.

Bisect BC at H, and produce AH to D,
making HD equal to AH; join BD.

Then the triangles AHC, DHB are equal in
all respects (l. 4); and the angles CAH,ACHaie
respectively equal to the angles BDH, DBH.

It follows that the sum of the angles of the
triangle ABD is equal to the sum of the angles of the original triangle, i.e.
to tR + a.

And one of the angles DAB, ADB is either equal to or less than half the
angle. CAB.

Continuing the same construction with the triangle ADB, we find a third
triangle in which the sum of the angles is still zR + a, while one of them is
equal to or less than J L CAB)jj\.

Proceeding in this way, we arrive at a triangle in which the sum of the
angles is 2R + a, and one of them is not greater than ( L CAB)J2 H .

And, if n is sufficiently large, this will be less than a.; in which case we
should have a triangle in which two angles are together greater than two right
angles: which is absurd.

Therefore a is equal to or less than zero.

(It will be noted that in both these proofs, as in Eucl. 1. 16,'' it is taken for
granted that a straight line is infinite in length and does not return into itself,
which is not true under the Riemann hypothesis.)

II. On the assumption that the sum of the angles of a triangle is less
than two right angles, if a triangle is made up of two others, the ``deficit'' of ' tht,
former is equal to the sum of the ``deficits ``of the others.

In fact, if the sums of the angles of the component triangles are 2R -a,
2R - fi respectively, the sum of the angles of the whole triangle is

(zj¥-a) + (2JP-£)-3.ff = 7Je-fa + P).

III. If the sum of the three angles of a triangle is equal to two right
angles, the same is true of all triangles obtained by subdividing it by straight
lines drawn from a vertex to meet the opposite side.

Since the sum of the angles of the triangle ABC is equal to 2R, if the
sum of the angles of the triangle ABD were 2R - a, it
would follow that the sum of the angles of the triangle A

ADC must be »R + q, which is absurd (by I. above).

IV. If in a triangle the sum of the three angles is
equal to two right angles, a quadrilateral can always be

constructed with four right angles and four equal sides B £2 io

exceeding in length any assigned rectilineal segment.

Let ABC be a triangle in which the sum of the angles is equal to two
right angles. We can assume ABC to be an isosceles right-angled triangle
because we can reduce the case to this by making subdivisions of ABC by
straight lines through vertices (as in Prop.~III. above).

Taking two equal triangles of this kind and placing their hypotenuses
together, we obtain a quadrilateral with four right angles and four equal
sides.

Putting four of these quadrilaterals together, we obtain a new quadrilateral
0/ the same kind but with its sides double of those of the first quadrilateral.

After n such operations we have a quadrilateral with four right angles and
four equal sides, each being equal to 3'' times the side AH.

The diagonal of this quadrilateral divides it into two equal isosceles right-
angled triangles in each of which the sum of the angles is equal to two right
angles.

Consequently, from the existence ot one triangle in which the sum of the
three angles is equal to two right angles it follows that there exists an isosceles
right-angled triangle with sides greater than any assigned rectilineal segment
and such that the sum of its three angles is also equal to two right angles.

V. If the sum of the three angles of one triangle is equal to two right
angles, the sum of the three angles of any other triangle is also equal to two
right angles.

It is enough to prove this for a right-angled triangle, since any triangle can
be divided into two right-angled triangles.

Let ABC be any right-angled triangle.

If then the sum of the angles of any one
triangle is equal to two right angles, we can
construct (by the preceding Prop.) an isosceles
right-angled triangle with the same property and
with its perpendicular sides, greater than those of
ABC.

Let A'BC' be such a triangle, and let it be
applied to ABC, as in the figure.

Applying then Prop.~111. above, we deduce
first that the sum of the three angles of the
triangle ABC is equal to two right angles, and
next, for the same reason, that the sum of the three angles of the original
triangle A BC is equal to two right angles.

VI. If in any one triangle the sum of the three angles is less than two
right angles, the sum of the three angles of any other triangle is also less than
two right angles.

This follows from the preceding theorem.

(It will be observed that the last two theorems are included among those
of Saccheri, which contain however in addition the corresponding theorem
touching the case where the sum of the angles is greater than two right
angles.)

We come now to the bearing of these propositions upon Euclid's Postulate
5; and the next theorem is

VII. If the sum of the three angles of a triangle is equal to two right
angles, through any point in a plane there (an only be drawn one parallel to a
given straight line.

For the proof of this we require the following

Lemma. // is always possible, through a faint P, to draw a straight line
which shall make, with a ghten straight line (r), an angle less than any assigned
angle.

Let Q be the foot of the perpendicular from /'upon r.

Let a segment QR be taken on r,
on either side of Q, such that QR is
equal to PQ.

Join PR, and mark off the segment
RR' equal to PR; join PR'.

If tu represents the angle QPR or
the angle QRP, each of the equal
angles RPR', RR'P is not greater
than 10/ 1.

Continuing the construction, we obtain, after the requisite number of
operations, a triangle PR,-, R n in which each of the equal angles is equal to
or less than <o/*''.

Hence we shall arrive at a straight line PR, which, starting from /'and
meeting r, makes with r an angle as small as we please.

To return now to the Proposition. Draw from P the straight line s
perpendicular to PQ.

Then any straight line drawn from P which meets r in R will form equal
angles with r and s, since, by hypothesis, the sum of the angles of the triangle
PQR is equal to two right angles.

And since, by the Lemma, it is always possible to draw through P straight
lines which form with r angles as small as we please, it follows that all the
straight lines through /', except s, will meet r. Hence s is the only parallel
to r that can be drawn through P.

The history of the attempts to prove Postulate 5 or something equivalent
has now been brought down to the parting of the ways. The further
developments on lines independent of the Postulate, beginning with
Schweikart {1780 — 1857), Taurinus {1794 — 1874), Gauss {1777 — 1855),
Lobacbewsky {1793 — 1850), J. Bolyai (180* — 1860), Riemann (1826— 1866),
belong to the history of non-Euclidean geometry, which is outside the scope
of this work. I may refer the reader to the full article Sulla teoria delle
par allele e suite geometric non-euelidec by R. Bonola in Questioni riguardanti
le matematiche elementari, I., of which I have made considerable use in the
above, to the same author's La geometria non-euclidea, Bologna, 1906, to the
first volume of Killing's Einfuhrutig in die Grundlagen der Geometric,
Paderborn, 1893, to p.~Mansion's Premiers princifes de mitagomitrie, and
p.~Barbarin's La giomitrie ncn-Euclidicnne, Paris, 1902, to the historical
summary in Veronese's Fondamenti di geometria, 1891, p.~565 sqq., and (for
original sources) to Engel and Stackel, Die Theorie der Paralleltinicn von
Eukltd bis auf Gauss, 1895, and Urkunden zur Gcschichte der nickt-Euklidischtn
Geometric, 1. (Lobachewsky), 1899, and ti. (Wolfgang und Johann Bolyai).
I will only add that it was Gauss who first expressed a conviction that the
Postulate could never be proved; he indicated this in reviews in the Gottin-
gische gelckrfe Anzeigen, 20 Apr. 18 16 and 18 Oct. 1822, and affirmed it in a
letter to Bessei of 27 January, 1829. The actual indemonstrability of the Pos-
tulate was proved by Beltrami (1868) and by Houel {Note sur rimpessibilite'' de
dimontrer par une construction plane leprincipe de la thiorie des paralletes dit Pos-
tulatum (f.EwA'AmBattaglini's Giornale di matematiehe,\ lit., 1870, pp.84 — 89).

Alternatives for Postulate 5.

It may be convenient to collect here a few of the more noteworthy
substitutes which have from time to time been formally suggested or tacitly
assumed.

{ i ) Through a given point only one parallel can be drawn to a given
straight line or, Two straight lines which intersect one another cannot both be
parallel to one and the same straight line.

This is commonly known as ``Playfair's Axiom,'' but it was of course not
a new discovery. It is distinctly stated in Proclus' note to Eucl. \prop{1}{31}.

(r a) If a straight tine intersect one of two parallels, it will intersect the
other also (Proclus).

(1 b) Straight lines parallel to the same straight line are parallel to one
another.

The forms (1 a) and (1 b) are exactly equivalent to (i).

{2) There exist straight lines everywhere equidistant from one another
(Posidonius and Geminus); with which may be compared Proclus' tacit
assumption that Parallels remain, throughout their length, at a finite distance
from one another.

(3) There exists a triangle in which the sum of the three angles is equal to
two right angles (Legendre).

(4) Given any figure, there exists a figure similar to it of any size toe please
(Wallis, Carnot, Laplace).

Saccheri points out that it is not necessary to assume so much, and that it
is enough to postulate that there exist two unequal triangles with equal angles.

(5) Through any point within an angle less than two-thirds of a right angle
a straight tine can always be drawn which meets both sides of the angle
{Legendre).

With this may be compared the similar axiom of Lorenz (Grundriss der
r einen und angewandten Mathematik, 1791): Every straight line through a
point within an angle must meet one of the sides of the angle.

(6) Given any three points not in a straight line, there exists a circle passing
through them {Legendre, W, Bolyai).

(7) ``If I could prove that a rectilineal triangle is possible the content of
which is greater than any given area, I am in a position to prove perfectly
rigorously the whole of geometry'' (Gauss, in a letter to W. Bolyai, 1 799).

cf.\ the proposition of Legendre numbered iv. above, and the axiom of
Worpitzky: There exists no triangle in which every angle is as small as we
please.

(8) If in a quadrilateral three angles are right angles, the fourth angle is
a right angle also (Clairaut, 1741).

(9) If two straight lines are parallel, they arc figures opposite to (or the
reflex of) one another with respect to the middle points of ail their transversal
segments (Veronese, Elements, 1904).

Or, Two parallel straight lines intercept, on every transiiersal which passes
through the middle point of a segment included between them, another segment
the middle point of which is the middle point of the first (Ingrami, Etementi,
1904).

Veronese and Ingrami deduce immediately Playfair's Axiom,

NOTES ON THE COMMON NOTIONS 221

AXIOMS OR COMMON NOTIONS.

In a paper Sur Pauthenticite des axiomes d'Euclide in the Bulletin des sciences
math, tt astron. 1884, p.~i6z sq. {Mimoires scientifigues, ir., pp, 48—63), Paul
Tannery maintained that the Common Notions (including the first three) were
not in Euclid's work but were interpolated later. The following are his main
arguments. (1) If Euclid had set about distinguishing between indemon-
strable principles (a) common to all demonstrative sciences and (b) peculiar
to geometry, he would, says Tannery, certainly not have placed the common
principles second and the special principles (the Postulates) first. (2) If the
Common Notions are Euclid's, this designation of them must be his too; for he
must have used some name to distinguish them from the Postulates and, if he
had used another name, such as Axioms, it is impossible to imagine why that
name was changed afterwards for a less suitable one. The word iwom
{nation), says Tannery, never signified a notion in the sense of a proposition,
but a notion of some object; nor is it found in any technical sense in Plato
and Aristotle. (3) Tannery's own view was that the formulation of the
Common Notions dates from the time of Apollonius, and that it was inspired
by his work relating to the Elements (we know from Proclus that Apollonius
tried to prove the Common Notions), This idea, Tannery thought, was
confirmed by a ``fortunate coincidence ``furnished by the occurrence of the
word hrovt. {notion) in a quotation by Proclus (p.~100, 6): ``we shall agree
with Apollonius when he says that we have a notion {iwoaui) of a line when
we order the lengths, only, of roads or walls to be measured.''

In reply to argument (r) that it is an unnatural order to place the purely
geometrical Postulates first, and the Common Notions, which are not peculiar
to geometry, last, it may be pointed out that it would surely have been a still
more awkward arrangement to give the Definitions first and then to separate
from them, by the interposition of the Common Notions, the Postulates, which
are so closely connected with the Definitions in that they proceed to postulate
the existence of certain of the things defined, namely straight lines and circles.

(2) Though it is true that ii™« in Plato and Aristotle is generally a
notion of an object, not of a fact or proposition, there are instances in Aristotle
where it does mean a notion of a fact: thus in the Eth. Nic. ix. 11, 1 1 7 1 a 32
he speaks of ``the notion (or consciousness) that friends sympathise ``(1} iwom
rod m>ya\yth Tok tt>£\uvs) and again, b 14, of ``the notion (or consciousness)
that they are pleased at his good fortune.'' It is true that Plato and Aristotle
do not use the word in a technical sense; but neither was there apparently in
Aristotle's time any fixed technical term for what we call ``axioms,'' since he
speaks of them variously as ``the so-called axioms in mathematics,'' ``the so-
called common axioms,'' ``the common (things) ``(to. jcotpd), and even ``the
common opinions ``(kmvoi bofru). I see therefore no reason why Euclid should
not himself have given a technical sense to ``Common Notions,'' which is at
least a distinct improvement upon ``common opinions.''

(3) The use of fcroux in Proclus' quotation from Apollonius seems to me
to be an unfortunate, rather than a fortunate, coincidence from Tannery's point
of view, for it is there used precisely in the old sense of the notion of an
object (in that case a line).

No doubt it is difficult to feel certain that Euclid did himself use the term
Common Notions, seeing that Proclus' commentary generally speaks of Axioms,
But even Proclus (p.~194, 8), after explaining the meaning of the word
``axiom,'' first as used by the Stoics, and secondly as used by ``Aristotle and
the geometers,'' goes on to say: ``For in their view (that of Aristotle and the
geometers) axiom and common notion are the same thing.'' This, as it seems
to me, may be a sort of apology for using the word ``axiom ``exclusively in
what has gone before, as if Proclus had suddenly bethought himself that he
had described both Aristotle and the geometers as using the one term
``axiom,'' whereas he should have said that Aristotle spoke of ``axioms,'' while
``the geometers'' (in fact Euclid), though meaning the same thing, called them
Common Notions, It may be for a like reason that in another passage (p.~76,
16), after quoting Aristotle's view of an ``axiom,'' as distinct from a postulate
and a hypothesis, he proceeds: ``For it is not by virtue of a common notion
that, without being taught, we preconceive the circle to be such and such a
figure.'' If this view of the two passages just quoted is correct, it would
strengthen rather than weaken the case for the genuineness of Common Notions
as the Euclidean term.

Again, it is clear from Aristotle's allusions to the ``common axioms in
mathematics ``that more than one axiom of this kind had a place in the text-
books of his day; and as he constantly quotes the particular axiom that, if
equals be taken from equals, the remainders are equals which is Euclid's Common
Notion 3, it would seem that at least the first three Common Notions were
adopted by Euclid from earlier textbooks. It is ? besides, scarcely credible
that, if the Common Notions which Apollonius tried to prove had not been
introduced earlier (e.g.\ by Euclid), they would then have been interpolated as
axioms and not as propositions to be proved. The line taken by Apollonius
is much better explained on the assumption that he was directly attacking
axioms which he found already admitted into the Elements.

Proclus, who recognised the five Common Notions given in the text, warns
us, not only against the error of unnecessarily multiplying the axioms, but
against the contrary error of reducing their number unduly (p.~196, 15), ``as
Heron does in enunciating three only; for it is also an axiom that the whole is
greater than the part, and indeed the geometer employs this in many places for
his demonstrations, and again that things which coincide are equal.''

Thus Heron recognised the first three of the Common Notions; and this
fact, together with Aristotle's allusions to ``common axioms'' (in the plural),
and in particular to our Common Notion 3, may satisfy us that at least the first
three Common Notions were contained in the Elements as they left Euclid's
hands.

Common Notion i.

Things which are equal to the same thing are also equal to one another.

Aristotle throughout emphasises the fact that axioms are self-evident truths,
which it is impossible to demonstrate. If, he says, any one should attempt to
prove them, it could only be through ignorance. Aristotle therefore would
undoubtedly have agreed in Proclus' strictures on Apollonius for attempting
to prove the axioms. Proclus gives (p.~194, 25), as a specimen
of these attempted proofs by Apollonius, that of the first of the
Common Notions. ``Let A be equal to B, and the latter to C;
I say that A is also equal to C. For, since A is equal to JB, it A B
occupies the same space with it; and since B is equal to C, it
occupies the same space with it.

Therefore A also occupies the same space with C.''

Proclus rightly remarks (p.~194, 23) that ``the middle term is no more
intelligible {better known, yvmpifuar€pov} than the conclusion, if it is not
actually more disputable.'' Again (p.~195, 6), the proof assumes two things,
(1) that things which ``occupy the same space'' <tot<k) are equal to one
another, and (2) that things which occupy the same space with one and the
same thing occupy the same space with one another; which is to explain the
obvious by something much more obscure, for space is an entity more
unknown to us than the things which exist in space.

Aristotle would also have objected to the proof that it is partial and not
general {xatfoXav), since it refers only to things which can be supposed to
occupy a space (or take up room), whereas the axiom is, as Froclus says
(p.~196, 1), true of numbers, speeds, and periods of time as well, though of
course each science uses axioms in relation to its own subject-matter only.

Common Notions 2, 3.

2, Kttt lav iiTiia «ra nrpo<rr*9j7, ™ oka JVfw urn.

3. Kai iav otto four lira dtfxHpffljj, ta narnAtHTO/itni fUTtk Ifftt.

7, JftquaU be added to equals, the wholes are equal.

3. //equals he subtracted from equals, the remainders are equal.

These two Common Notions are recognised by Heron and Proclus as
genuine. The latter is the axiom which is so favourite an illustration, with
Aristotle.

Following them in the Mss. and editions there came four others of the
same type as 1 — 3. Three of these are given by Heiberg in brackets; the
fourth he omits altogether.

The three are:

(«) If equals be ad/led to unequals, the wholes are unequal,

(b) Things which are double a/ the same thing are equal to oHe another.

{c) Things which are halves of the same thing are equal to one another.

The fourth, which was placed between (a) and {), was:

(d) If equals be subtracted front unequals, the remainders are unequal.

Proclus, in observing that axioms ought not to be multiplied, indicates
that all should be rejected which follow from the five admitted by him and
appearing in the text above {p.~155). He mentions the second of those just
quoted (/') as one of those to be excluded, since it follows from Common
Notion 1. Proclus does not mention (a), (c) or (rf); an-Naiml gives (a), (d), {b)
and (f), in that order, as Euclid's, adding a note of Simplicius that ``three
axioms (sentenriae acceptae) only are extant in the ancient manuscripts, but
the number was increased in the more recent.''

(a) stands self-condemned because ``unequal ``tells us nothing. It is easy
to see what is wanted if we refer to 1. 17, where the same angle is added to a
greater and a less, and it is inferred that the first sum is greater than the second.
So far however as the wording of (a) is concerned, the addition of equal to
greater and less might be supposed to produce less and greater respectively. If
therefore such an axiom were given at all, it should be divided into two.
Heiberg conjectures that this axiom may have been taken from the commentary
of Pappus, who had the axiom about equals added to unequais quoted below
(e); if so, it can only be an unskilful adaptation of some remark of Pappus, for
his axiom (e) has some point, whereas (a) is useless.

As regards (b), I agree with Tannery in seeing no sufficient reason why, if
we reject it (as we certainly must), the words in \prop{1}{47} ``But things which are
double of equals are equal to one another ``should be condemned as an
interpolation. If they were interpolated, we should have expected to find the
same interpolation in ). 42, where the axiom is tacitly assumed. I think
it quite possible that Euclid may have inserted such words in one case and
left them out in another, without necessarily implying either that he was
quoting a formal Common Notion of his own or that he had wt included
among his Common Notions the particular fact stated as obvious.

The corresponding axiom (c) about the halves of equals can hardly be
genuine if {b) is not, and Proclus does not mention it. Tannery acutely
observes however that, when Heiberg, in 1. 37, 38, brackets words stating that
``the halves of equal things are equal to one another'' on the ground that
axiom (e) was interpolated (although before Theon's time), and explains that
Euclid used Common Notion 3 in making his inference, he is clearly mistaken.
For, while axiom (b) is an obvious inference from Common Notion 2, axiom (c)
is not an inference from Common Notion 3. Tannery says, in a note, that (c)
would have to be established by rtdudio ad absurdum with the help of axiom
(6), that is to say, of Common Notion 2. But, as the hypothesis in the reductw
ad absurdum would be that one of the halves is greater than the other, and it
would therefore be necessary to prove that the one whole is greater than the
other, while axiom (b) or Common Notion 2 only refers to equals, a little
argument would be necessary in addition to the reference to Common Notion 2.
I think Euclid would not have gone through this process in order to prove (c),
but would have assumed it as equally obvious with (b).

Proclus (pp.~197, 6 — 198, 5) definitely rejects two other axioms of the
above kind given by Pappus, observing that, as they follow from the genuine
axioms, they are rightly omitted in most copies, although Pappus said that
they were ``on record ``with the others (<rvvara.ypiufn<T$at):

(e) If unequa/s be added to equals, the difference between the wholes is equal
to the difference between the added parts; and

(/) //equals be added to unequals, the difference between the wholes is equal
to the difference bttween (he original unequals.

Proclus and Simplicius (in an-Nairīzī) give proofs of both. The proof of
the former, as given by Simplicius, is as follows:

Let AB, CD be equal magnitudes; and let EB, FD be £
added to them respectively, EB being greater than FD. q

I say that AE exceeds CF by the same difference as that by
which BE exceeds DF.

Cut off from BE the magnitude BG equal to DF.

Then, since AE exceeds AG by GE, and AG is equal to CF
and BG to DF,

AE exceeds CF by the same difference as that by which BE
exceeds DF.

Common Notion 4.

Kcu to. t<f>upyjj(,uvra tir oAAipXa ura aAAr/Xoi? cotcV.

Things which coincide with one another are equal to one anotJur.

The word itjmp/Aiiltiv, as a geometrical term, has a different meaning
according as it is used in the active or in the passive. In the passive,
<£af>/u>£«rltu, it means ``to be applied to'' without any implication that the
applied figure will exactly fit, or coincide with, the figure to which it is applied;
on the other hand the active 'vaptw is used intransitively and means ``to
lit exactly,'' ``to coincide with.'' In Euclid and Archimedes t<#>a/)o{<«' is
constructed with «ri and the accusative, in Pappus with the dative.

On Common Notion 4 Tannery observes that it is incontestably geometrical
in character, and should therefore have been excluded from the Common
Notions; again, it is difficult to see why it is not accompanied by its converse,
at all events for straight lines (and, it might be added, angles also), which
Euclid makes use of in \prop{1}{4}. As it is, says Tannery, we have here a definition
of geometrical equality more or less sufficient, but not a real axiom.

It is true that Froclus seems to recognise this Common Notion and the next
as proper axioms in the passage (p.~196, 15 — 21) where he says that we should
not cut down the axioms to the minimum, as Heron does in giving only three
axioms; but the statement seems to rest, not upon authority, but upon an
assumption that Euclid would state explicitly at the beginning all axioms
subsequently used and not reducible to others unquestionably included. Now
in 1. 4 this Common Notion is not quoted; it is simply inferred that ``the base
BC will coincide with EF, and will be equal to it.'' The position is therefore
the same as it is in regard to the statement in the same proposition that, ``if...
the base BC does not coincide with EF, two straight lines will enclose a spate:
which is impossible ``; and, if we do not admit that Euclid had the axiom that
``two straight lines cannot enclose a space,'' neither need we infer that he had
Common Notion 4. I am therefore inclined to think that the latter is more
likely than not to be an interpolation.

It seems clear that the Common Notion, as here formulated, is intended
to assert that superposition is a legitimate way of proving the equality of two
figures which have the necessary parts respectively equal, or, in other words,
to serve as an axiom of congruence.

The phraseology of the propositions, e.g.\ 1. 4 and 1. 3, in which Euclid
employs the method indicated, leaves no room for doubt that he regarded one
figure as actually moved and placed upon the other. Thus in 1. 4 he says,
``The triangle ABC being applied (t<£apfu>{o(ia>ov) to the triangle DEF, and
the point A being placed {rJltjitvov) upon the point D, and the straight line
AB on DE, the point B will also coincide with E because AB is equal to
DE''; and in 1. 8, ``If the sides BA, AC do not coincide with ED, DF, but
fall beside them (take a different position, TapaAAafowrtv), then ``etc.\ At the
same time, it is clear that Euclid disliked the method and avoided it wherever
he could, e.g.\ in 1. 26, where he proves the equality of two triangles which have
two angles respectively equal to two angles and one side of the one equal to
the corresponding side of the other. It looks as though he found the method
handed down by tradition (we can hardly suppose that, if Thales proved that
the diameter of a circle divides it into two equal parts, he would do so by any
other method than that of superposition), and followed it, in -the few cases
where he does so, only because he had not been able to see his way to a
satisfactory substitute. But seeing how much of the Elements depends on 1. 4,
directly or indirectly, the method can hardly be regarded as being, in Euclid,
of only subordinate importance , on the contrary, it is fundamental. Nor, as
a matter of fact, do we find in the ancient geometers any expression of doubt
as to the legitimacy of the method. Archimedes uses it to prove that any
spheroidal figure cut by a plane through the centre is divided into two equal
parts in respect of both its surface and its volume; he also postulates in
Equilibrium of Planes 1. that ``when equal and similar plane figures coincide
if applied to one another, their centres of gravity coincide also.''

Killing {Einfuhrung in die Grundlagen der Geometric, 11. pp.~4, 5)
contrasts the attitude of the Greek geometers with that of the philosophers,
who, he says, appear to have agreed in banishing motion from geometry
altogether. In support of this he refers to the view frequently expressed by
Aristotle that mathematics has to do with immovable objects {wivr/rd), and that
only where astronomy is admitted as part of mathematical science is motion
mentioned as a subject for mathematics. cf.\ Mttaph. 989 b 32 ``For mathe-
matical objects are among things which exist apart from motion, except such
as relate to astronomy''; Metapk. rofi4 a 30 ``Physics deals with things
which have in themselves the principle of motion; mathematics is a theoretical
science and one concerned with things which are stationary (mfovto) but not
separable'' (sc. from matter); in Pkysies if. z, 193 b 34 he speaks of the
subjects of mathematics as ``in thought separable from motion.''

But I doubt whether in Aristotle's use of the words ``immovable,'' ``with-
out motion ``etc.\ as applied to the subjects of mathematics there is any
implication such as Killing supposes. We arrive at mathematical concepts
by abstraction from material objects; and just as we, in thought, eliminate
the matter, so according to Aristotle we eliminate the attributes of matter as
such, e.g.\ qualitative change and motion. It does not appear to me that the
use of ``immovable ``in the passages referred to means more than this. I do
not think that Aristotle would have regarded it as illegitimate to move a
geometrical figure from one position to another; and I infer this from a
passage in De caelo lit. 1 where he is criticising ``those who make up every
body that has an origin by putting together plants, and resolve it again into
planes.'' The reference must be to the Timaeus (54 b sqq.) where Plato
evolves the four elements in this way. He begins with a right-angled triangle
in which the hypotenuse is double of the smaller side; six of these put together
in the proper way produce one equilateral triangle. Making solid angles with
{a) three, (6) four, and (c) five of these equilateral triangles respectively, and
taking the requisite number of these solid angles, namely four of (a), six of ()
and twelve of (c) respectively, and putting them together so as to form regular
solids, he obtains (a) a tetrahedron, (/J) an octahedron, (v) an icosahedron
respectively. For the fourth element (earth), four isosceles right-angled triangles
are first put together so as to form a square, and then six of these squares are
put together to form a cube. Now, says Aristotle (299 b 23), ``it is absurd that
planes should only admit of being put together so as to touch in a line; for just
as a line and a line are put together in both ways, lengthwise and breadthwise,
so must a plane and a plane. A line can be combined with a line in the sense
of being a line superposed, and not added''; the inference being that a plane can
be superposed on plane. Now this is precisely the sort of motion in question
here; and Aristotle, so far from denying its permissibility, seems to blame
Plato for not using it. cf.\ also Physits v. 4, 228 b 25, where Aristotle speaks
of ``the spiral or other magnitude in which any part will not coincide with
any other part,'' an where superposition is obviously contemplated.

Motion without deformation.

It is well known that Helmholtz maintained that geometry requires us to
assume the actual existence of rigid bodies and their free mobility in space,
whence he inferred that geometry is dependent on mechanics.

Veronese exposed the fallacy in this {Fondamenti ii geometria, pp.~xxxv —
xxxvi, 239 — 240 note, 615 — 7), his argument being as follows. Since geometry
is concerned with empty space, which is immovable, it would be at least strange
if it was necessary to have recourse to the real motion of bodies for a definition,
and for the proof of the properties, of immovable space. We must distinguish
the intuitive principle of motion in itself from that of motion without deforma-
tion. Every point of a figure which moves is transferred to another point in
space. ``Without deformation ``means that the mutual relations between the
points of the figure do not change, but the relations between them and other
figures do change (for if they did not, the figure could not move). Now
consider what we mean by saying that, when the figure A has moved from
the position A, to the position A it the relations between the points of A in
the position A, are unaltered from what they were in the position A u are the
same in fact as if A had not moved but remained at A t . We can only say
that, judging of the figure (or the body with its physical qualities eliminated)
by the impressions it produces in us during its movement, the impressions
produced in us in the two different positions (which are in time distinct)
are equal. In fact, we are making use of the notion of equality between two
distinct figures. Thus, if we say that two bodies are equal when they
can be superposed by means of movement without deformation, we are com-
mitting a petitio principii. The notion of the equality of spaces is really prior
to that of rigid bodies or of motion without deformation. HelmholU supported
his view by reference to the process of measurement in which the measure
must be, at least approximately, a rigid body, but the existence of a rigid body
as a standard to measure by, and the question how we discover two equal
spaces to be equal, are matters of no concern to the geometer. The method
of superposition, depending on motion without deformation, is only of use as
a practical test; it has nothing to do with the theory of geometry.

Compare an acute observation of Schopenhauer {Die Welt als (Vi/le, 2 ed-
1844, 11. p.~130) which was a criticism in advance of Helmholtz' theory: ``I
am surprised that, instead of the eleventh axiom [the Parallel-Postulate], the
eighth is not rather attacked: ' Figures which coincide (sich decken) are
equal to one another.' For coincidence (das Sichdecken) is either mere
tautology, or something entirely empirical, which belongs, not to pure intuition
(Anschauung), but to external sensuous experience. It presupposes in fact
the mobility of figures; but that which is movable in space is matter and
nothing else. Thus this appeal to coincidence means leaving pure space, the
sole element of geometry, in order to pass over to the material and empirical.''

Mr Bertrand Russell observes {Encyclopaedia Britannica, Suppl. Vol. 4,
1Q02, Art. ``Geometry, non-Euclidean ``) that the apparent use of motion here
is deceptive; what in geometry is called a motion is merely the transference
of our attention from one figure to another. Actual superposition, which is
nominally employed by Euclid, is not required; all that is required is the
transference of our attention from the original figure to a new one defined by
the position of some of its elements and by certain properties which it shares
with the original figure.

If the method of superposition is given up as a means of defining theoreti-
cally the equality of two figures, some other definition of equality is necessary.
But such a definition can be evolved out of empirical or practical observation
of the result of superposing two material representations of figures. This is
done by Veronese {Elementi di geometria, 1904) and Ingrami {Element i di
geometria, 1904). Ingrami says, namely (p.~66);

``If a sheet of paper be folded double, and a triangle be drawn Upon it
and then cut out, we obtain two triangles superposed which we in practice call
equal. If points A, B, C, D ... be marked on one of the triangles, then,
when we place this triangle upon the other (so as to coincide with it), we see
chat each of the particular points taken on the first is superposed on one
particular point of the second in such a way that the segments AB, AC, AD,
BC, BD, CD, ... ace respectively superposed on as many segments in the
second triangle and are therefore equal to them respectively. In this way we
justify the following

``Definition of equality.

``Any two figures whatever will be called equal when to the points of one
the points of the other can be made to correspond univocally [i.e.\ every one
point in one to one distinct point in the other and vice versa] in such a way
that the segments which join the points, two and two, in one figure are
respectively equal to the segments which join, two and two, the corresponding
points in the other.''

Ingram! has of course previously postulated as known the signification of
the phrase equal (reciiUneat) segments, of which we get a practical notion when
we can place one upon the other or can place a third movable segment
successively on both.

New systems of Congruence-Postulates.

In the fourth Article of Questioni riguardanti ie matematiehe etementari, I.,
pp.~93 — 122, a review is given of three different systems: (i) that of Pasch in
Vorlesungen titer neuere Geometrie, 1882, p.~101 sqq., (3) that of Veronese
according to the Fondamenti di geometria, 1891, and the Ekmcnti taken
together, (3) that of Hilbert (see Grundlagen der Geometric, 1903, pp.~7—15).

These systems differ in the particular conceptions taken by the three
authors as primary, (t) Pasch considers as primary the notion of congruence
or equality between any figures which are made up of a finite number 0/ points
only. The definitions of congruent segments and of congruent angles have to
be deduced in the way shown on pp.~102 — 103 of the Article referred to, after
which Eucl. 1. 4 follows immediately, and Eucl. 1. 26 (1) and 1. 8 by a method
recalling that in Eucl. 1. 7, 8.

(2) Veronese takes as primary the conception of congruence between
segments (rectilineal). The transition to congruent angles, and thence to
triangles is made by means of the following postulate:

``Let AB, Cand AB, A'C be two pairs of straight lines intersecting
at A, A', and let there be determined upon them the congruent segments
AB, A'ff and the congruent segments AC, A'C;

then, if BC, BC are congTuent, the two pairs of straight lines are con-
gruent.''

<i) Hilbert takes as primary the notions of congruence between both
segments and angles.

It is observed in the Article referred to that, from the theoretical stand-
point, Veronese's system is an advance upon that of Pasch, since the idea of
congruence between segments is more simple than that of congruence between
any figures; but, didactically, the development of the theory is more compli-
cated when we start from Veronese's system than when we start from that of
Pasch.

The system of Hilbert offers advantages over both the others from the
point of view of the teaching of geometry, and I shall therefore give a short
account of his system only, following the Artiole above quoted.

Hilbcrt's system.

The following are substantially the Postulates laid down,

(1) If one segment is congruent toith another, the second is also congruent
with the first.

(2) If an angle is congruent with another angle, the second angle is also
congruent with the first.

(3) Two segments congruent with a third are congruent with one another.

(4) Two angles congruent with a third are congruent with one another.

(5) Any segment AB is congruent with itself, independently of its sense.
This we may express symbolically thus:

AB=AB= BA.

(6) Any angle (ab) is congruent with itself, independently of its sense.
This we may express symbolically thus:

(ab) = (ab) = (6a).

(7) On any straight line x\ starting from any one of its points A', and on
each side of it respectively, there exists one and only one segment congruent with a
segment AB belonging to the straight tine 1.

() Given a ray a, issuing from a paint O, in any plane which contains it
and on each of the two sides of it, there exists one and only one ray b issuing
from O such that the angle (ab) is congruent with a given angle (a'b').

(9) If AB, BC are two consecutive segments of the same straight line r
(segments, that is, having an extremity and no other point common), and A'B',
B'C two consecutive segments on another straight line r'', and if AB E A'B',
BC 3 B'C, then

AC=A'C.

{10) If (ah), (be) are two consecutive angles in the same planer (angles,
that is, having the vertex and one side common), and (a'b'), (b'c') two consecu-
tive angles in another plane *', and if (ab) £ (a'b'), (be) = (b'c'), then

(ac) = (a'c').

(n) If two triangles have two sides and the included angles respectively
congruent, they have also their third sides congruent as well as the angles
opposite to the congruent sides respectively.

As a matter of fact, Hilbert's postulate corresponding to (n) does not
assert the equality of the third sides in each, but only the equality of the two
remaining angles in one triangle to the two remaining angles in the other
respectively. He proves the equality of the third sides (thereby completing
the theorem of Eucl. L 4) by reductio

ad absurdum thus. Let ABC, A'B'C A A'

be the two triangles which have the S\ jr\

sides AB, AC respectively congruent r \ jr jl

with the sides A B', A'C and the j \ y; \

included angle at A congruent with — £ gf- £-jy

the included angle at A'.

Then, by Hilbert's own postulate, the angles ABC, A'B'C are congruent,
as also the angles ACB, A'C'ff.

If BC is not congruent with B'C, let D be taken on BC such that BC,
BD are congruent and join A'D.

Then the two triangles ABC, A'ffD have two sides and the included
angles congruent respectively; therefore, by the same postulate, the angles
BAC, SA'D are congruent

But the angles BAC, B'A'C' are congruent; therefore, by (4) above, the
angles B'A'C 1 , BAD are congruent: which is impossible, since it contradicts
(8) above.

Hence BC, B'C cannot but be congruent.

Eucl. 1. 4 is thus proved; but it seems to be as well to include all of that
theorem in the postulate, as is done in (11) above, since the two parts of it are
equally suggested by empirical observation of the result of one superposition.

A proof similar to that just given immediately establishes Eucl. 1. 26 (1),
and Hilbert next proves that

If two angles ABC, A' B'C' are congruent with one another, their supple-
mentary angles CBD, C'B'D' are also congruent with one another.

We choose A, D on one of the straight lines forming the first angle, and
A',  on one of those forming the second angle, and again C, C on the other

* Vi 0'

straight lines forming the angles, so that A'B is congruent with AB, C'ff
with CB, and Uff with DB.

The triangles ABC, A' B'C are congruent, by (11) above; and AC is
congruent with A'C, and the angle CAB with the angle C'A'B 1 .

Thus, AD, A'D' being congruent, by (9), the triangles CAD, C'A'D are
also congruent, by ( 1 1 );

whence CD is congruent with CD?, and the angle ADC with the angle
ADC.

Lastly, hy (n), the triangles CDS, CDS are congruent, and the angles
CBD, C'B'D are thus congruent.

Hilbert's next proposition is that

Given that the angle (h, k) in the plane a is congruent with the angle (h', k')
in the plane a', and that 1 is a half-ray in the plane a. starting from the vertex
of the angle (h, k) and lying within that angle, there always exists a half-ray V
in the second plane a', starting from the vertex of the angle {h', k') and lying
within that angle, such that

(h,l)s(h',l'), and(k, l) = (k',I').

If O, C are the vertices, we choose points A, B on h, k, and points A, ff
on H, k' respectively, such that OA, OA are congruent and also OB, Off.

The triangles OAB, OAff are then congruent; and, if / meets AB in C,
: can determine C on A'B such that AC is congruent with AC.
Then f drawn from through C is the half-ray required.

The congruence of the angles (h, I), (A', /) follows from (11) directly, and
that of {k, I) and {X, t) follows in the same way after we have inferred by
means of (9) that, AB, AC being respectively congruent with A'B 1 , A'C, the
difference BC is congruent with the difference B C'.

It is by means of the two propositions just given that Hilbert proves that
All right angles are congruent with one another.

Let the angle BAD be congruent with its adjacent angle CAD, and
likewise the angle BA'D congruent with its adjacent angle CA'D All four
angles are then right angles.

A'

C

If the angle B'A'D is not congruent with the angle BAD, let the angle
with AB for one side and congruent with the angle B'A'D be the angle
BAD', so that AD' falls either within the angle BAD or within the angle
DAC. Suppose the former.

By the last proposition but one (about adjacent angles), the angles
BAD, BAD' being congruent, the angles CA'D, CAD' are congruent.

Hence, by the hypothesis and postulate (4) above, the angles BAD' ,
CAD'' are alio congruent.

And, since the angles BAD, CAD are congruent, we can find within the
ingle CAD a half-ray CAD''' such that the angles BAD', CAD'' are
congruent, and likewise the angles DAD', DAD'' (by the last proposition).

But the angles BAD', CAD'' were congruent (see above); and it
follows, by (4), that the angles CAD', CAD'' are congruent: which is
impossible, since it contradicts postulate (8),

Therefore etc.

Euclid 1. s follows directly by applying the postulate (n) above to ABC,
ACB as distinct triangles.

Postulates (9), (10) above give in substance the proposition that ``the
sums or differences of segments, or of
angles, respectively equal, are equal.''

Lastly, Hilbert proves Eucl. 1. 8 by
means of the theorem of Eucl. 1. 5 and
the proposition just stated as applied to
angles.

ABC, A'BC being the given triangles
with three sides respectively congruent,
we suppose an angle CBA'' to be deter-
mined, on the side of BC opposite to A,
congruent with the angle A'BC, and we make BA'' equal to A'B.

The proof is obvious, being equivalent to the alternative proof often given
in our text-books for Eucl. \prop{1}{8}.

Common Notion 5.

koI to oAov tou ttipavs fLttfcav [f tTTtv],
J>fe icAoA w greater than the part.

Proclus includes this ``axiom ``on the same ground as the preceding one.
I think however there is force in the objection which Tannery takes to it,
namely that it replaces a different expression in Eucl. f. 6, where it is stated
that ``the triangle DBC will be equal to the triangle A CB, the /ess to the
greater: which is absurd.'' The axiom appears to be an abstraction or
generalisation substituted for an immediate inference from a geometrical
figure, but it takes the form of a sort of definition of whole and part. The
probabilities seem to be against its being genuine, notwithstanding Proclus'
approval of it

Clavius added the axiom that the whole is the equal to the sum of its parts.

Other Axioms introduced after Euclid's time.

[9] Two straight lines do not enclose (or can tain) a space.
Proclus {p.~196, 21) mentions this in illustration of the undue multiplication
of axioms, and he points out, as an objection to it, that it belongs to the
subject matter of geometry, whereas axioms are of a general character, and
not peculiar to any one science. The real objection to the axiom is that it is
unnecessary, since i.he fact which it states is included in the meaning of
Postulate i. It was nr> doubt taken from the passage in 1. 4, ``if. the base
BC does not coincide with the base EF, two straight tines wilt enclose a space:
which is impossible''; and we must certainly regard it as an interpolation,
notwithstanding that two of the best mss. have it after Postulate 5, and one
gives it as Common Notion 9.

Pappus added some others which Proclus objects to (p.~198, 5) because
they are either anticipated in the definitions or follow from them.

(g) All the parts of a plane, or of a straight line, coincide with one another.
\K) A point divides a line, a line a surface, and a surface a solid; on which
Proclus remarks that everything is divided by the same things as those by
which it is bounded.

An-Nairīzī {ed. Besthorn-Heiberg, p.~31, ed. Curtze, p.~38) in his version
of this axiom, which be also attributes to Pappus, omits the reference to
solids, but mentions planes as a particular case of surfaces.
``(a) A Surface cuts a surface in a line;
fj8) If two surfaces which cut one another are plane, they cut one another

in a straight line;

(y) A line cuts a line in a point (this last we need in the first proposition}.''

(A) Magnitudes are susceptible of the infinite {or unlimited) both by way of

addition and by way of successive diminution, but in both cases potentially only

(to mrttpov fv rots fnyi$ttrw iirriv got Tp xpoaGitrtt Kal rp iiittai0atpio , u, oW ifLtt

Si imLrtpav).

An-Nairīzī's version of this refers to straight lines and plane surfaces only:
``as regards the straight line and the plane surface, in consequence of their
evenness, it is possible to produce them indefinitely.

This ``axiom'' of Pappus, as quoted by Proclus, seems to be taken directly
from the discussion of to avtipov in Aristotle, Physics ill. 5 — 8, even to the
wording, for, while Aristotle uses the term division {huaiptatsi) most frequently
as the antithesis of addition (avrBttrn), he occasionally speaks of subtraction
(AAJHupttrn) and diminution (icaflaipHrii). Hankel (Zur Geschichte der Mathe-
matik im Alterthum und Mittelalter, 1874, pp.~119 — 120) gave an admirable

1. Axx.] ADDITIONAL AXIOMS 233

summary of Aristotle's views on this subject; and they are stated in greater
detail in Gorland, AristoteUs und die Mathematik, Marburg, 1899, pp.~157—
183. The infinite or unlimited (airttpov) only exists potentially {Sura/i«), not
in actuality (Ivtpytiq.). The infinite is so in virtue of its endlessly changing
into something else, like day or the Olympic Games {Phys. m. 6, 206 a 15 — 2 S)-
The infinite is manifested in different forms in time, in Man, and in the
division of magnitudes. For, in general, the infinite consists in something new
being continually taken, that something being itself always finite but always
different. Therefore the infinite must not be regarded as a particular thing
(toS« t»), as man, house, but as being always in course of becoming or decay,
and, though finite at any moment, always different from moment to moment.
But there is the distinction between the forms above referred to that, whereas
in the case of magnitudes what is once taken remains, in the case of time and
Man it passes or is destroyed but the succession is unbroken. The case of
addition is in a sense the same as that of division; in the finite magnitude the
former takes place in the converse way to the latter; for, as we see the finite
magnitude divided ad infinitum, so we shall find that addition gives a sum
tending to a definite limit. I mean that, in the case of a finite magnitude,
you may take a definite fraction of it and add to it (continually) in the same
ratio; if now the successive added terms do not include one and the same
magnitude whatever it is [i.e.\ if the successive terms diminish in geometrical
progression], you will not come to the end of the finite magnitude, but, if the
ratio is increased so that each term does include one and the same magnitude
whatever it is, you will come to the end of the finite magnitude, for every
finite magnitude is exhausted by continually taking from it any definite
fraction whatever. Thus in no other sense does the infinite exist, but only
in the sense just mentioned, that is, potentially and by way of diminution
(106 a 25 — b 13). And in this sense you may have potentially infinite
addition, the process being, as we say, in a manner, the same as with division
ad infinitum: for in the case of addition you will always be able to find some*
thing outside the total for the time being, but the total will never exceed every
definite (or assigned) magnitude in the way that, in the direction of division,
the result will pass every definite magnitude, that is, by becoming smaller
than it. The infinite therefore cannot exist even potentially in the sense of
exceeding every finite magnitude as the result of successive addition (206 b
16 — 22). It follows that the correct view of the infinite is the opposite of
that commonly held: it is not that which has nothing outside it, but that
which always has something outside it (206 b 33 — 207 a r).

Contrasting the case of number and magnitude, Aristotle points out that
(1) in number there is a limit in the direction of smallness, namely unity, but
none in the other direction: a number may exceed any assigned number
however great; but (2) with magnitude the contrary is the case: you can find
a magnitude smaller than any assigned magnitude, but in the other direction
there is no such thing as an infinite magnitude (207 b r — 5). The tatter
assertion he justified by the following argument. However large a thing can
be potentially, it can be as large actually. But there is no magnitude
perceptible to sense that is infinite. Therefore excess over every assigned
magnitude is an impossibility j otherwise there would be something larger
than the universe (oipavot) {207 b 17—21).

Aristotle is aware that it is essentially of physical magnitudes that he is
speaking. He had observed in an earlier passage (PAys, in, 5, 204 a 34) that
it ts perhaps a more general inquiry that would be necessary to determine
whether the infinite is possible in mathematics, and in the domain of thought
and of things which have no magnitude; but he excuses himself from entering
upon this inquiry on the ground that his subject is physics and sensible
objects. He returns however to the bearing of his conclusions on mathematics
in m. 7, 207 b 2j: ``my argument does not even rob mathematicians of their
study, although it denies the existence of the infinite in the sense of actual
existence as something increased to such an extent that it cannot be gone
through (dSwfiTifrov); for, as it is, they do not even need the infinite or use
it, but only require that the finite (straight line) shall be as long as they please;
and another magnitude of any size whatever can be cut in the same ratio as
the greatest magnitude. Hence it will make no difference to them for the
purpose of demonstration.''

Iastly, if it should be urged that the infinite exists in thought, Aristotle
replies that this does not involve its existence in fact. A thing is not greater
than a certain size because it is conceived to be so, but because it is; and
magnitude is not infinite in virtue of increase in thought (ao8 a 16 — zz).

Hankel and Gorland do not quote the passage about an infinite series of
magnitudes {206 b 3—13) included in the above paraphrase; but I have
thought that mathematicians would be interested in the distinct expression of
Aristotle's view that the existence of an infinite series the terms of which are
magnitudes is impossible unless it is convergent, and (with reference to
Riemann's developments) in the statement that it does not matter to geometry
if the straight line is not infinite in length, provided that it is as long as we
please.

Aristotle's denial of even the potential existence of a sum of magnitudes
which shall exceed every definite magnitude was, as he himself implies, in
conflict with the lemma or assumption used by Eudoxus (as we infer from
Archimedes) to prove the theorem about the volume of a pyramid. The
lemma is thus stated by Archimedes (Quadrature of a parabola, preface):
``The excess by which the greater of two unequal areas exceeds the less can,
if it be continually added to itself, be made to exceed any assigned finite
area.'' We can therefore well understand why, a century later, Archimedes
felt it necessary to justify his own use of the lemma as he does in the same
preface; ``The earlier geometers too have used this lemma: for it is by its
help that they have proved that circles have to one another the duplicate
ratio of their diameters, that spheres have to one another the triplicate ratio
of their diameters, and so on. And, in the result, each of the said theorems
has been accepted no less than those proved without the aid of this lemma.''

Principle of continuity.

The use of actual construction as a method of proving the existence ot
figures having certain properties is one of the characteristics of the Elements.
Now constructions are effected by means of straight lines and circles drawn
in accordance with Postulates 1 — 3; the essence of them is that such straight
lines and circles determine by their intersections other points in addition to
those given, and these points again are used to determine new lines, and so on.
This being so, the existence of such points of intersection must be postulated
or proved in the same way as that of the lines which determine them. Yet
there is no postulate of thfs character expressed in Euclid except Post J.
This postulate asserts that two straight lines meet if they satisfy a certain
condition. The condition is of the nature of a Stop«r>«5s (discrimination, or
condition of possibility) in a problem; and, if the existence of the point of
intersection were not granted, the solutions of' problems in which the points of
intersection of straight lines are used would not in general furnish the required
proofs of the existence of the figures to be constructed.

But, equally with the intersections of straight lines, the intersections of
circle with straight line, and of circle with circle, are used in constructions.
Hence, in addition to Postulate 5, we require postulates asserting the actual
existence of points of intersection of circle with straight line and of circle
with circle. In the very first proposition the vertex of the required equilateral
triangle is determined as one of the intersections of two circles, and we need
therefore to be assured that the circles will intersect. Euclid seems to assume
it as obvious, although it is not so; and he makes a similar assumption in
1. 2 2. It is true that in the latter case Euclid adds to the enunciation that
two of the given straight lines must be together greater than the third; but
there is nothing to show that, if this condition is satisfied, the construction is
always possible. In 1. 12, in order to be sure that the circle with a given
centre will intersect a given straight line, Euclid makes the circle pass through
a point on the side of the line opposite to that where the centre is. It appears
therefore as if, in this case, he based his inference in some way upon the
definition of a circle combined with the fact that the point within it called
the centre is on one side of the straight line and one point of the circumference
on the other, and, in the case of two intersecting circles, upon similar con-
siderations. But not even in Book hi., where there are several propositions
about the relative positions of two circles, do we find any discussion of the
conditions under which two circles have two, one, or no point common.

The deficiency can only be made good by the Principle of Continuity.

Killing {Einfiihrung in die Grundlagen der Geometric, 11. p 43) gives the
following forms as sufficient for most purposes

(a) Suppose a line belongs entirely to a figure which is divided into two
parts; then, if the line has at least one point common with each part, it must
also meet the boundary between the parts; or

{b) If a point moves in a figure which is divided into two parts, and if it
belongs at the beginning of the motion to one part and at the end of the
motion to the other part, it must during the motion arrive at the boundary
between the two parts.

I n the Questioni riguardanti le matematuhe elemeniari, l.,Art.s,pp.~123—143,
the principle of continuity is discussed with special reference to the Postulate
of Dedektnd, and it is shown, first, how the Postulate may be led up to and,
secondly, how it may be applied for the purposes of elementary geometry.

Suppose that in a segment A B of a straight line a point C determines
two segments AC, CB. If we consider the point Cas belonging to only one
of the two segments A C, CB, we have a division of the segment AB into two
parts with the following properties.

1. Every point of the segment AB belongs to one of the two parts.

2. The point A belongs to one of the two parts (which we will call the
first) and the point B to the other; the point C may belong indifferently to

one or the other of the two parts according as we choose to premise.

3. Every point of the first part precedes every point of the second in the
order AB of the segment.

(For generality we may also suppose the case in which the point C falls at
A or at B. Considering C, in these cases respectively, as belonging to the
first or second part, we still have a division into parts which have the
properties above enunciated, one part being then a single point A or B.)

Now, considering carefully the inverse of the above proposition, we see
that it agrees with the idea which we have of the continuity of the straight
line. Consequently we are induced to admit as a postulate the following.

If a segment of a straight lint AB is divided into two parts so that
(i ) every point of the segment AB belongs to one of the parts,

(2) the extremity A belongs to the first part and B to the second, and

(3) any point whatever of the first part precedes any point whatever of the
second part, in the order AB of the segment,

there exists a point C of the segment AB {which may belong either to one
part or to the other) such that every point of AB that precedes C belongs to the
first part, and every point of AB that follows C belongs to the second part in
the division originally assumed.

(If one of the two parts consists of the single point A or B, the point C
is the said extremity A or B of the segment.)

This is the Postulate of Dedekind, which was enunciated by Dedekind
himself in the following slightly different form (Stetigkcit unci irrationale Zahlen,
187s, new edition 1905, p.~11).

``If all points of a straight line fall into hvo classes such that every point of
the first class lies to the left of every point of the second class, there exists one and
only one point which produces this division of all the points into two daises, this
division of the straight line into two parts.''

The above enunciation may be said to correspond to the intuitive notion
which we have that, if in a segment of a straight line two points start from
the ends and describe the segment in opposite senses, they meet in a point.
The point of meeting might be regarded as belonging to both parts, but for
the present purpose we must regard it as belonging to one only and subtracted
from the other part.

Application of Dedckind's postulate to angles.

If we consider an angle less than two right angles bounded by two rays
a, b, and draw the straight line connecting A, a point on a, with B, a point
on b, we see that all points on the finite segment AB correspond univocally to
all the rays of the angle, the point corresponding to any ray being the point
in which the ray cuts the segment AB; and if a ray be supposed to move
about the vertex of the angle from the position a to the position b, the
corresponding points of the segment AB are seen to follow in the same
order as the corresponding rays of the angle (ab).

Consequently, if the angle (ab) is divided into two parts so that
(1) each ray of the angle (ab) belongs to one of the two parts,
(t) the outside ray a belongs to the first part and the ray b to the second,
(3) any ray whatever of the first part precedes any ray whatever of the
second part,
the corresponding points of the segment AB determine tteo parts of the
segments such that

(1) every point of the segment AB belongs to one of the two parts,

(2) the extremity A belongs to the first part and B to the second,

(3) any point whatever of the first part precedes any point whatever of
the second.

But in that case there exists a point C of AB {which may belong to one
or the other of the two parts) such that every point of AB that precedes C
belongs to the first part and every point of AB that follows C belongs to the
second part.

Thus exactly the same thing holds of c, the ray corresponding to C, with
reference to the division of the angle {ad) into two parts.

It is not difficult to extend this to an angle (ai) which is either flat or
greater than two right angles; this is done (Vitali, op.~cit. pp.~126—127) by
supposing the angle to be divided into two, {ad), {di>), each less than two
right angles, and considering the three cases in which

(1) the ray d is such that all the rays that precede it belong to the first
patt and those which follow it to the second part,

(2) the ray d is followed by some rays of the first part,

(3) the ray d is preceded by some rays of the second part.

Application to circular arcs.

If we consider an arc AB of a circle with centre O, the points of the arc
correspond uni vocally, and in the same order, to the rays from the point O
passing through those points respectively, and the same argument by which
we passed from a segment of a straight line to an angle can be used to make
the transition from an angle to an arc.

Intersections of a straight line with a circle.

It is possible to use the Postulate of Dedekind to prove that

If a straight line has one point inside and one point outside a circle, it has
two points common with the circle.

For this purpose it is necessary to assume (1) the proposition with reference
to the perpendicular and obliques drawn from a given point to a given straight
line, namely that of all straight lines drawn from a given point to a given
straight line the perpendicular is the shortest, and of the rest (the obliques)
that is the longer which has the longer projection upon the straight line, while
those are equal the projections of which are equal, so that for any given
length of projection there are two equal obliques and two only, one on each
side of the perpendicular, and (2) the proposition that any side of a triangle
is less than the sum of the other two.

Consider the circle {(7) with centre 0, and a straight line {r) with one
point A inside and one point B outside the
circle.

By the definition of the circle, if R is
the radius,

OA<R, OB>R.

Draw OP perpendicular to the straight
line r.

Then OP< OA, so that OP is always
less than R, and P is therefore within the
circle C.

Now let us fix our attention on the finite segment AB of the straight
line r. It can be divided into two parts, (1) that containing all the points H
for which OIf< R (i.e.\ points inside C), and (2) that containing all the
points K for which OK £ R (points outside C or on the circumference of C).

Thus, remembering that, of two obliques from a given point to a given
straight line, that is greater the projection of which is greater, we can assert
that all the points of the segment PB which precede a point inside C are
inside C, and those which follow a point on the circumference of C or outside
C are outside C.

Hence, by the Postulate of Dedekind, there exists on the segment PB a
point M such that all the points which precede it belong to the first part and
those which follow it to the second part.

I say that M is common to the straight line r and the circle C, or

OM=R.

For suppose, e.g., that OM <.R.

There will then exist a segment (or length) a less than the difference
between R and OM.

Consider the point Af, one of those which follow M, such that MAT is
equal to a.

Then, because any side of a triangle is less than the sum of the Other two,
OM' < OM+ MM'.

But OM+ MM' = OM+<r<R,

whence OM' < R,

which is absurd.

A similar absurdity would follow if we suppose that OM > R.

Therefore OM mist be equal to R.

It is immediately obvious that, corresponding to the point Mor\ the segment
PS which is common to r and C, there is another point on r which has the
same property, namely that which is symmetrical to M with respect to P.

And the proposition is proved.

Intersections of two circles.

We can likewise use the Postulate of Dedekind to prove that

If in a given plane a circle C has one paint X inside and one point Y outside
another circle C ', the two circles intersect in two points.

We must first prove the following

Lemma.

If O, 0' are the centres of two circles C, C, and R, R' their radii
respectively, the straight line 00' meets the circle C in two points A, S, one
of which is inside C'' and the other outside it.

Now one of these points must fall (r) on the prolongation of 00
beyond O or {2) on 00 itself or {3) on the
prolongation of OO' beyond 0.

(1) First, suppose A to lie on OO pro-
duced.

Then A0=AO+ 00' = JP+ OO'' (a).

But, in the triangle OO Y,

0Y<OY+O0,
and, since 0Y>R\ OY=R,
R<R+ Off.
It follows from (a) that A0>R; and
therefore lies outside C''.

(2) Secondly, suppose A to lie on
00.

Then 00 = OA + A0 = R + A0 ...($).
From the triangle 00 X we have

O0<OX+0X,
and, since OX-R, 0X<R, it follows
that

00'<R + R,
whence, by {£), A0 < R, so that A lies inside C

Consider

{3) Thirdly, suppose A to lie on 00 produced.

Then R = OA = 00 + O'A (7).

And, in the triangle 00 X,

ox<oa+0x, w

that is R < 00 + OX, B f

whence, by (y),

00 + O'A < Off + 0X,
or O'A < 0X,

and A lies inside C.

It is to be observed that one of the two points A, B is in the position of
case (1) and the other in the position of either case {2) or case (3): whence
we must conclude that one of the two points A, B is inside and the other
outside the circle C.

Proof of theorem.

The circle C is divided by the points A, B into two semicircles,
one of them, and suppose it to be
described by a point moving from A
to B.

Take two separate points P, Q
on it and, to fix our ideas, suppose
that P precedes Q.

Comparing the triangles O0P,
OaQ, we observe that one side 00
is common, OP is equal to OQ, and
the angle PO0 is less than the angle

QO0.

Therefore 0P< 0Q.

Now, considering the semicircle APQB as divided into two parts, so that
the points of the first part are inside the circle C, and those of the second
part on the circumference of C or outside it, we have the conditions necessary
for the applicability of the Postulate of Dedekind (which is true for arcs of
circles as for straight lines); whence there exists a point M separating the two
parts.

I say that 0M=R''.

For, if not, suppose 0M < R 1 .

If then a signifies the difference between R' and 0M, suppose a point if,
which follows M, taken on the semicircle such that the chord MM' is not
greater than cr (for a way of doing this see below).

Then, in the triangle 0MM',

0AT < OM+ MM' < 0M+ <r,
and therefore 0M' < R.

It follows that M', a point on the arc MB, is inside the circle C'':
which is absurd.

Similarly it may be proved that 0M is not greater than R.

Hence 0M=R.

[To find a point M' such that the chord MM' is not greater than a, we
may proceed thus.

Draw from M a straight line MP distinct from OM, and cut off MP on it
equal to er/a.

a4o BOOK I [i. Axx.

Join OP, and draw another radius OQ such that the angle POQ is equal
to the angle MOp.~q

The intersection, M\ of 0@ with the
circle satisfies the required condition.

For MM' meets OP at right angles
in S.

Therefore, in the right-angled triangle
MSP, MS is not greater than MP (it is
less, unless MP coincides with MS, when
it is equal).

Therefore MS is not greater than <rjt, so that MM' is not greater than <r.\

\end{comment}

\part{BOOK I. PROPOSITIONS}

\begin{proposition}
\label{prop:I1}

\begin{statement}
On a given finite straight line to construct an equilateral triangle.
\end{statement}

\begin{proof}

Let $AB$ be the given finite straight line.

\begin{wrapfigure}{r}{0pt}
\includegraphics{bookI_1}
\end{wrapfigure}

Thus it is required to construct an equilateral triangle on the
straight line $AB$.

With centre $A$ and distance $AB$ let the circle $BCD$ be described;
\since{Post.~\ref{post:3}}

again, with centre $B$ and distance $BA$ let the circle $ACE$ be
described;\since{Post.~\ref{post:3}}

and from the point $C$, in which the circles cut one another, to
the points $A$, $B$ let the straight lines $CA$, $CB$ be joined.
\since{Post.~\ref{post:1}}

Now, since the point $A$ is the centre of the circle $CDB$,
$AC$ is equal to $AB$.\since{Def.~15}

Again, since the point $B$ is the centre of the circle $CAB$,
$BC$ is equal to $BA$.\since{Def.~15}

But $CA$ was also proved equal to $AB$;

therefore each of the straight lines $CA$, $CB$ is equal to $AB$.

And things which are equal to the same thing are also
equal to one another;\since{\emph{C. N.}~1}

therefore $CA$ is also equal to $CB$.

Therefore the three straight lines $CA$, $AB$, $BC$ are aj equal to
one another.

Therefore the triangle $ABC$ is equilateral; and it has been
constructed on the given finite straight line $AB$.

\begin{flushright}
(Being) what it was required to do.
\end{flushright}

\end{proof}

\end{proposition}

\begin{notes}

1. \textbf{On a given finite straight line.} The Greek usage differs
from ours in that the definite article is employed in such a phrase as
this where we have the indefinite. \greek{ἐπὶ της δοθείσης εὐθείας
  πεπερασμένης}, ``on \emph{the} given finite straight line,''
i.e.\ the finite straight line which we choose to take.

3. \textbf{Let $AB$ be the given finite straight line.} To be strictly
literal we should have to translate in the reverse order ``let the
given finite straight line be the (straight line) $AB$''; but this
order is inconvenient in other cases where there is more than one
datum, e.g.\ in the sitting-out of I.~2, ``let the given point be $A$,
and the given straight Line $BC$,'' the awkwardness arising from the
omission of the verb in the second clause.  Hence I have, for
clearness' sake, adopted the other order throughout the book.

8. \textbf{let the circle $BCD$ be described.} Two things are here to
be noted, (1)~the elegant and practically universal use of the perfect
passive imperative in constructions, \greek{γεγράφθω} meaning of
course ``let it \emph{have been} described'' or ``suppose it
described,'' (2)~the impossibility of expressing shortly in a
translation the force of the words in their original order.
\greek{κύκλος γεγράφθω ὁ ΒΓΔ} means literally ``let a circle have been
described, the (circle, namely, which I denote by) $BCD$.'' Similarly
we have lower down ``let straight lines, (namely) the (straight Lines)
$CA$, $CB$, be joined,'' \greek{ἐπεζεύχθωσαν εὐθεῖαι αἰ ΓΑ, ΓΒ}. There
seems to be no practicable alternative, in English, but to translate
as I have done in the text.

13. \textbf{from the point $C$\dots.} Euclid is careful to adhere to the
phraseology of Postulate~\ref{post:1} except that he speaks of
``joining'' (\greek{ἐπεζεύχθωσαν}) instead of ``drawing
(\greek{γράφειν}). He does not allow himself to use the shortened
expression ``let the straight line $FC$ be joined'' (without mention
of the points $F$, $C$) until \prop{1}{5}.

20. \textbf{each of the straight lines $CA$, $CB$, \greek{ἐκατέρα τῶν
    ΓΑ, ΓΒ}} and 24.~\textbf{the three straight lines $CA$, $AB$,
  $BC$, \greek{αἱ τρεῖς αἱ ΓΑ, ΑΒ, ΒΓ}}.  I have, here and in all
  similar expressions, inserted the words ``straight lines'' which are
  not in the Greek. The possession of the inflected definite article
  enables the Greek to omit the words, but this it not possible in
  English, and it would scarcely be English to write ``each of $CA$,
  $CB$'' or ``the three $CA$, $AB$, $BC$.''

\end{notes}

\begin{commentary}

It is a commonplace that Euclid has no right to assume, without
premising some postulate, that the two circles \emph{will} meet in a
point~$C$. To supply what is wanted we must invoke the Principle of
Continuity (see note thereon above, p.~235\?). It is sufficient for
the purpose of this proposition and of \prop{1}{21}, where there is a
similar tacit assumption, to use the form of postulate suggested by
Killing. ``\emph{If a line} [in this case e.g.\ the circumference
  $ACE$] \emph{belongs entirely to a figure} [in this case a plane]
\emph{which is divided into two parts} [namely the part enclosed
  within the circumference of the circle $BCD$ and the part outside
  that circle], \emph{and if the line has at least one point common
  with each part, it must also meet the boundary between the parts}
     [i.e.\ the circumference $ACE$ must meet the circumference
       $BCD$].''

Zeno's remark that the problem is not solved unless it is taken for
granted that two straight lines cannot have a common segment has
already been mentioned (note on Post.~\ref{post:2}, p.~100\?). Thus,
if $AC$, $BC$ meet at $F$ before reaching~$C$, and have the part $EC$
common, the triangle obtained, namely $FAB$, will not be equilateral,
but $FA$, $FB$ will each be less than~$AB$. But Post.~\ref{post:2} has
already laid it down that two straight lines cannot have a common
segment.

Proclus devotes considerable space to this part of Zeno's criticism,
but satisfies himself with the bare mention of the other part, to the
effect that it is also necessary to assume that two
\emph{circumferences} (with different centres) cannot have a common
part. That is, for anything we know, there may be any number of points
$C$ common to the two circumferences $ACE$, $BCD$. It is not until
\prop{3}{10} that it is proved that two circles cannot intersect in
more points than two, so that we are not entitled to assume it
here. The most we can say is that it is enough for the purpose of this
proposition if \emph{one} equilateral triangle can be found with the
given base; that the construction only gives \emph{two} such triangles
has to be left over to be proved subsequently.  And indeed we have not
long to wait; for \prop{1}{7} clearly shows that on either side of the
base $AB$ only \emph{one} equilateral triangle can be described. Thus
\prop{1}{7} gives us the \emph{number of solutions} of which the
present problem is susceptible, and it supplies the same want in
\prop{1}{22} where a triangle has to be described with three sides of
given length; that is, \prop{1}{7} furnishes us, in both cases, with
one of the essential parts of a complete \greek{}, which includes not
only the determination of the conditions of possibility but also the
number of solutions (\greek{}, Proclus, p.~202,~5). This view of
\prop{1}{7} as supplying an equivalent for \prop{3}{10} absolutely
needed in \prop{1}{1} and \prop{1}{22} should serve to correct the
idea so common among writers of text-books that \prop{1}{7} is merely
of use as a lemma to Euclid's proof of \prop{1}{8}, and therefore may
be left out if an alternative proof of that proposition is adopted.

\begin{wrapfigure}{r}{0pt}
\includegraphics{bookI_1a}
\end{wrapfigure}

Agreeably to his notion that it is from \prop{1}{1} that we must
satisfy ourselves that isosceles and scalene triangles actually exist,
as well as equilateral triangles, Proclus shows how to draw, first a
particular isosceles triangle, and then a scalene triangle, by means
of the figure of the proposition. To make an isosceles triangle he
produces $AB$ in both directions to meet the respective circles in
$D$, $E$, and then describes circles with $A$, $B$ as centres and
$AE$, $BD$ as radii respectively. The result is an isosceles triangle
with each of two sides double of the third side. To make an isosceles
triangle in which the equal sides are not so related to the third side
but have any given length would require the use of \prop{1}{3}; and
there is no object in treating the question at all in advance of
\prop{1}{22}. An easier way of satisfying ourselves of the existence
of some isosceles triangles would surely be to conceive any two radii
of a circle drawn and their extremities joined.

\begin{wrapfigure}{r}{0pt}
\includegraphics{bookI_1b}
\end{wrapfigure}

There is more point in Proclus' construction of a \emph{scalene}
triangle, Suppose $AC$ to be a radius of one of the two circles, and
$D$ a point on $AC$ lying in that portion of the circle with
centre~$A$ which is outside the circle with centre~$B$.  Then, joining
$BD$, as in the figure, we have a triangle which obviously has all its
sides unequal, that is, a scalene triangle.

The above two constructions appear in al-Nairīzī's commentary under
the name of Heron; Proclus does not mention his source.

In addition to the above construction for a scalene triangle
(producing a triangle in which the ``given'' side is greater than one
and less than the other of the two remaining sides), Heron has two
others showing the other two possible cases, in which the ``given''
side is (1)~less than, (2)~greater than, either of the other two
sides.

\end{commentary}

\end{document}

Proposition 2.

To place at a given point (as an extremity) a straight line
equal to a given straight line.

Let A be the given point, and BC the given straight line.
Thus it is required to place at the point A (as an extremity)
l a straight line equal to the given
straight line BC.

From the point A to the point B
let the straight line AB be joined;

[Post 1]
and on it let the equilateral triangle
10 DAB be constructed. [1. i\

Let the straight lines A£, BF be
produced in a straight line with DA,
DB; [Post a]

with centre B and distance BC let the
i circle CGH be described; [Post 3]

and again, with centre D and distance DG let the circle GKL
be described, [Post. 3]

Then, since the point B is the centre of the circle CGH,
BC is equal to BG.
20 Again, since the point D is the centre of the circle GKL,
DL is equal to DG.
And in these DA is equal to DB;

therefore the remainder AL is equal to the remainder
BG. [C.N. 3]

2 S But BC was also proved equal to BG;

therefore each of the straight lines AL, BC is equal
XoBG.
And things which are equal to the same thing are also
equal to one another; [CJK 1]

p therefore AL is also equal to BC.

Therefore at the given point A the straight line AL is
placed equal to the given straight line BC.

(Being) what it was required to do.

I . (as an extremity}. I have inserted these words because ``to place a straight line
at a given point ``(rftht t£ Bo$4m ffijff) is not quite clear enough, at least in English.

II. Let the straight lines AB, BF be produced.,.. It will be observed that in this
first application of Postulate a, and again in \prop{1}{5}, Euclid speaks of the conJinuaiiffft of the
straight line as that which is produced in such cases, inlikftXIjawaw and TpoaiKfttffMir#trtar
meaning little more than dram'nr straight lines ``in a straight line with ``the given straight
lines. The first place in which Euclid uses phraseology exactly corresponding to ours when
speaking of a straight Line being produced ia in I. l6: ``let one side of it, BC t be produced
to D ``(rp<mt;ep\1iir8u airoC >t(a irXtupA ij Br * W to A).

13. the remainder AL...the remainder BG. Tbe Greek expressions ate X«r 
A A and XorviJ r£ BH> and the literal translation would be 'Mi (or BG) rtmmning™
bat tbe shade of meaning conveyed by the position of the definite article can hardly be
expressed in English.

This proposition gives Proclus an opportunity, such as the Greek
commentators revelled in, of distinguishing a multitude of cases. After
explaining that those theorems and problems are said to have eases which
have the same force, though admitting of a number of different figures, and
preserve the same method of demonstration while admitting variations of
position, and that cases reveal themselves in the etms/rue/tbn, he proceeds to
distinguish the cases in this problem arising from the different positions
which the given point may occupy relatively to the given straight line. It may
be (he says) either (i) outside the line or (a) on the line, and, if (1), it may be
either (a) on the line produced or (6) situated obliquely with regard to it; if
(a), it may be either (a) one of the extremities of the line or (A) an intermediate
point on it. It will be seen that Proclus' anxiety to subdivide leads him to
give a ``case,'' (a) (a), which is useless, since in that ``case'' we are given
what we are required to rind, and there is really no problem to solve. As
Savile says, ``qui quaerit ad punctum ponere rectam aequaiem rjj fiy rectae,
quaerit quod datum est, quod nemo faceret nisi forte msaniat,''

Proclus gives the construction for (a) (i) following Euclid's way of taking
G as the point in .which the circle with centre B intersects DB produced, and
then proceeds to ``cases,'' of which there are still more, which result from the
different ways of drawing the equilateral triangle and of producing its sides.

This last class of ``cases'' he subdivides into three according as AB is
(1) equal to, {a) greater than or (3) less than BC, Here again ``case ``(i) serves
no purpose, since, if AB is equal to BC, the problem is already solved. But
Proclus' figures for the other two cases are worth giving, because in one of
them the point G is on BD produced beyond D, and in the other it lies on
BD itself and there is no need to produce any side of the equilateral triangle.

A glance at these figures will show that, if they were used in the proposition,
each of them would require a slight modification in the wording (r) of the
construction, since BD is in one case produced beyond D instead of B and
in the other case not produced at all, (a) of the proof, since BG, instead of
being the difference between DG and DB, is in one case the sum of DG and
DB and in the other the difference between DB and DG.

Modern editors generally seem to classify the cases according to the
possible variations in the construction rather than according to differences in
the data. Thus Lardner, Potts, and Todhunter distinguish eight cases due
to the three possible alternatives, (1) that the given point may be joined to
either end of the given straight line, {2) that the equilateral triangle may then
be described on either side of the joining line, and {3) that the side of the
equilateral triangle which is produced may be produced in either direction.
(But it should have been observed that, where AB is greater than BC, the
third alternative is between producing DB and not producing it at all.) Potts
adds that, when the given point lies either on the line or on the line produced,
the distinction which arises from joining the two ends of the line with the
given point no longer exists, and there are only four cases of the problem
(I think he should rather have said solutions).

To distinguish a number of cases in this way was foreign to the really
classical manner. Thus, as we shall see, Euclid's method is to give one case
only, for choice the most difficult, leaving the reader to supply the rest for
himself. Where there was a real distinction between cases, sufficient to
necessitate a substantial difference in the proof, the practice was to give
separate enunciations and proofs altogether as we may see, e.g., from the
Cenia and the De section* rationis of Apollonius.

Proclus alludes, in conclusion, to the error of those who proposed to solve
1. 2 by describing a circle with the given point as centre and with a distance
equal to BC, which, as he says, is a petitio principii. De Morgan puts the
matter very clearly {Supplementary Remarks on the first six Books* of Euclid's
Elements in the Companion to the Almanac, 1849, p.~6). We should ``insist,''
he says, ``here upon the restrictions imposed by the first three postulates,
which do not allow a circle to be drawn with a compass-carried distance;
suppose the compasses to dose of themselves the moment they cease to touch
the paper. These two propositions [1. 2, 3] extend the power of construction
to what it would have been if all the usual power of the compasses had been
assumed; they are mysterious to all who do not see that postulate iii does
not ask for every use of the compasses.''

Proposition 3.

Given two unequal straight lines, to cut off from the
greater a straight line equal to the
less. c

Let AB, C be the two given un-
equal straight lines, and let AB be
the greater of them.

Thus it is required to cut off from
AB the greater a straight line equal
to C the less.

At the point A let AD be placed
equal to the straight line C; [1. 2]
and with centre A and distance AD let the circle DEF be
described. [Post 3]

Now, since the point A is the centre of the circle DEF,

AE is equal to AD. [Def. 15]

But C is also equal to AD.

Therefore each of the straight lines AE, C is equal to
AD; so that A E is also equal to C. [C.N. 1]

Therefore, given the two straight lines AB, C, from AB
the greater AE has been cut off equal to C the less.

(Being) what it was required to do.

P roc his contrives to make a number of ``cases'' out of this proposition
also, and gives as many as eight figures. But he only produces this variety by
practically incorporating the construction of the preceding proposition, instead
of assuming it as we are entitled to do. If Prop, 2 is assumed, there is really
only one ``case ``of the present proposition, for Potts distinction between two
cases according to the particular extremity of the straight line from which the
given length has to be cut off scarcely seems to be worth making.

Proposition 4.

If two triangles have the two sides equal to two sides
respectively, and have the angles contained by the equal straight
lines equal, they will also have the base equal to the base, the
triangle will be equal to the triangle, and the remaining angles
s will be equal to the remaining angles respectively, namely those
which the equal sides subtend.

Let ABC, DEF be two triangles having the two sides
AB, AC equal to the two sides DE, DF respectively, namely
AB to DE and AC to DF, and the angle BA C equal to the
10 angle EDF.

I say that the base BC is also equal to the base EF, the
triangle ABC will be equal to the triangle DEF, and the
remaining angles will be equal to the remaining angles
respectively, namely those which the equal sides subtend, that
is is, the angle ABC to the angle DEF, and the angle ACB
to the angle DEE.

For, if the triangle ABC be
applied to the triangle DEF,
and if the point A be placed
ao On the point D

and the straight line AB
on DE,
then the point B will also coincide with E, because AB is
equal to DE.

»j Again, AB coinciding with DE,

the straight line AC will also coincide with DF, because the

angle SAC is equal to the angle EDF;

hence the point C will also coincide with the point F,

because AC is again equal to DF.
30 But B also coincided with E;

hence the base BC will coincide with the base EF.

[For if, when B coincides with E and C with F, the base

BC does not coincide with the base EF, two straight lines

will enclose a space: which is impossible.

35 Therefore the base BC will coincide with

EF~\ and will be equal to it. \C.N. 4]

Thus the whole triangle ABC will coincide with the
whole triangle DEF,

and will be equal to it.

40 And the remaining angles will also coincide with the
remaining angles and will be equal to them,

the angle ABC to the angle DEF,

and the angle ACB to the angle DFE,

Therefore etc.
*S (Being) what it was required to prove.

1 — 3. It is a fact that Euclid's enunciations not infrequently leave something to be
desired in point of clearness and precision. Here he speaks of the triangles having *' the
angle equal to the angle, namely the angle contained by the equal straight lines ``[rrtr yuriar
r£ ywif tar}* §xv T h r v* ruv law iijdtiutv wtpi.txt>M yr l v )i only one of the two angles being
described in the latter expression (in the accusative], and a similar expression in the dative
being left to be understood of the other angle. It is curious too that, after mentioning two
``sides'' he speaks of the angles contained by the equal ``straight lints'' not ``sides. It
may be that he wished to adhere scrupulously, at the outset, to the phraseology of the
definitions, where the angle is the inclination to one another of two lines or straight lints.
Similarly in the enunciation of I: £ he speaks of producing the equal ``straight lines'' as if to
keep strictly to the wording of Postulate 1.

t. respectively. I agree with Mr H. M. Taylor {Euclid, p.~ix) that it is best to
abandon the traditional translation of [ ' each to each,'' which would naturally seem to imply
that all the four magnitudes are equal rather than (as the Greek itaripa. 1 tariff does) that
one is equal to one and the other to the other.

3. the base. Here we have the word bast used for the first time in the Elements.
Proclus explains it (p.~136, 13 — tj) as meaning (1), when no side of a triangle has been
mentioned before, the side ``which is on a level with the sight ``(rV rpbi tq ftipet ntifiirjjr),
and (1), when two sides have already been mentioned, the third side. Proclus thus avoids
the mistake made by some modern editors who explain the term exclusively with reference to
the case where two sides have been mentioned before. That this is an error is proved (t) by
the occurrence of .the term in the enunciations of 1. 37 etc.\ about triangles on the same base
and equal bases, (1) by the application of the same term to the bases of parallelograms in
h 35 etc.\ The truth is that the use of the term must have been suggested by the practice of
drawing the particular side horizontally, as it were, and the rest of the figure above it. The
bait of a figure was therefore spoken of, primarily, in the same sense as the base of anything
else. e.g.\ of a pedestal or column; but 'vhen, as in ]. 5. two triangles were compared
occupying other thun the norma! positions which gave rise to the name, and when two side'
had been previously mentioned , the base was as Proclus says, necessarily the third side.
6. subtend. Owvrttteuf br6, ``to stretch under,'' with accusative*
9. the angle BAC. The full Greek expression would be it irro rwp BA, AT vtpuxorn
yaivtu, ``the angle contained by the (straight lines) BA, AC.'' But it was a common practice
of Greek geometers, e.g.\ of Archimedes and Apollonius (and Euclid too in Books x.— xiij.), to
use the abbreviation at BAl' for at BA, AT, ``the (straight lines} BA, AC.'' Thus, on
TtpttX'tfni being dropped, the expression would become first  iiri rat BAr -yafta, then
i M BAP yttrln, and finally i brb BAr, without ywla, as we regularly find it in Euclid.

17. if the triangle be applied to..., 13. coincide. The difference between the
technical use of the passive t>apjt4f»T0ai ``to be applitd (to),'' and of the active t£npf«>
``to coincide (with} has been noticed above (note on Common tfetien 4, pp.~114 — j).

J j. [For if, when B coincides... j'>. coincide with EF]. Heiberg (ParaHpumcna i»
lid in Hermts, xxxvin., 1003, p.~56] has pointed out, as a conclusive reason for regarding
these words as an early interpolation, that the text of an-NairM [Codex Ltidtnsis 300, 1, ed.
Besthom- Heiberg, p.~55) does not give the words in this place but after the conclusion q.e.d.,
which shows that they constitute a scholium only. They were doubt less added b? some
commentator who thought it necessary to explain the immediate inference that, since B
coincides with E and C with F. the straight line BC coincides with the straight line EF,
an inference which really follows from the definition of a straight tine and Post. 1; and no
doubt the Postulate that ``Two straight lines cannot enclose a space'' (afterwards placed
among the Common Notions) was interpolated at the same time.

44. Therefore etc.\ Where (as here) Euclid's conclusion merely repeats the enunciation
word for word, I shall avoid the repetition and write ``Therefore etc'' simply.

In the note on Common Notion 4 I have already mentioned that Euclid
obviously used the method of superposition with reluctance, and I have given,
after Veronese for the most part, the reason for holding that that method is
not admissible as a theoretical means of proving equality, although it may be
of use as a practical test, and may thus furnish an empirical basis on which to
found a postulate. Mr Bertrand Russell observes {Principles of Mathematics
I. p.~405) that Euclid would have done better to assume 1. 4 as an axiom, as
is practically done by Hilbert (Grundtagen der Geometric, p.~9). It may be
that Euclid himself was as well aware of the objections to the method as are
his modem critics; but at all events those objections were stated, with almost
equal clearness, as early as the middle of the 16th century. Peletarius
(Jacques Peletier) has a long note on this proposition (Jn Eudidis Elementa
gtometrica demonttratwnunt libri sex, 1557), in which he observes that, if
superposition of lines and figures could be assumed as a method of proof, the
whole of geometry would be full of such proofs, that it could equally well have
been used in 1. 2, 3 (thus in 1. * we could simply have supposed the line taken
up and placed at the point), and that in short it is obvious how far removed the
method is from the dignity of geometry. The theorem, he adds, is obvious in
itself and does not require proof; although it is introduced as a theorem, it
would seem that Euclid intended it rather as a definition than a theorem, ``for
I cannot think that two angles are equal unless I have a conception of what
equality of angles is.'' Why then did Euclid include the proposition among
theorems, instead of placing it among the axioms ? Peletarius makes the best
excuse he can, but concludes thus: ``Huius itaque propositionis veritatem non
aliunde quam a communi iudirio petemus: cogitabimusque figuras figuris
superponere, Mechantcum quippiam esse: intelligere verb, id demum esse
Mathematicum.''

Expressed in terms of the modern systems of Congruence-Axioms referred
to in the note on Common Notion 4, what Euclid really assumes amounts to
the following:

(1) On the line DE, there is a point E, on either side of D, such that AB
is equal to DE.

(2) On either side of the ray DE there is a ray DF such that the angle
EDF is equal to the angle BAC-

It now follows that on DF there is a point  such that DF is equal
toC.

And lastly (3), we require an axiom from which to infer that the two
remaining angles of the triangles are respectively equal and that the bases are
equal.

I have shown above (pp.~229 — 230) that Hilbert has an axiom stating the
equality of the remaining angles simply, but proves the equality of the bases.

Another alternative is that of Pasch ( VorUsungen titer neuert Geometric,
p.~109) who has the following ``Gnmdsatz'':

If two figures AB and FGH are given (FGH not being contained in a
straight length), and AB, FG are congruent, and if a plane surface be laid
through A and B, we can specify in this plane surface, produced if necessary,
two points C, D, neither more nor less, such that the figures ABC and ABD
are congruent with the figure FGH t and the straight line CD has with the
straight line AB or with AB produced one point common.

I pass to two points of detail in Euclid's proof:

(1) The inference that, since B coincides with E, and C with F, the
bases of the triangles are wholly coincident rests, as expressly stated, on the
impossibility of two straight tines enclosing a space, and therefore presents no
difficulty.

But (2) most editors seem to have failed to observe that at the very
beginning of the proof a much more serious assumption is made without any
explanation whatever, namely that, if A be placed on D, and AB on DE, the
point B will coincide with £, because AB is equal to DE. That is, the
converse of Common Notion 4 is assumed for straight lines. Proem s merely
observes, with regard to the converse of this Common Notion, that it is only
true in the case of things ``of the same form ``(i/toeiBi}), which he explains as
meaning straight lines, arcs of one and the same circle, and angles ``contained
by lines similar and similarly situated'' (p.~241, 3 — 8.

Savile however saw the difficulty and grappled with it in his note on the
Common Notion. After stating that all straight lines with two points common
are congruent between them (for otherwise two straight lines would enclose a
space), he argues thus. Let there be two straight lines AB, DE, and let A be
placed on D, and AB on DE. Then B will coincide with E. For, if not,
let B fall somewhere short of E or beyond E; and in either case it will follow
that the less is equal to the greater, which is impossible.

Savile seems to assume (and so apparently does Lardner who gives the
same proof) that, if the straight lines be ``applied,'' B will fall somewhere on
DE or DE produced. But the ground for this assumption should surely be
stated; and it seems to me that it is necessary to use, not Postulate 1 alone,
nor Postulate 2 alone, but both, for this purpose (in other words to assume,
not only that two straight lines cannot enclose a space, but also that two straight
lines cannot have a common segment). For the only safe course is to place A
upon D and then turn AB about D until some point on AB intermediate
between A and B coincides with some point on DE. In this position AB and
DE have two points common. Then Postulate 1 enables us to infer that the
straight lines coincide between the two common points, and Postulate 2 that
they coincide beyond the second common point towards B and E. Thus the
straight lines coincide throughout so far as both extend; and Savile's argument
then proves that B coincides with E.

Proposition 5.

In isosceles triangles the angles at the base are equal to one
another, and, if the equal straight lines be produced further,
the angles under the base will be equal to one another.

Let ABC be an isosceles triangle having the side AB
5 equal to the side AC; '

and let the straight lines BD, CE be produced further in a
straight line with AB, AC. [Post. 2}

1 say that the angle ABC is equal to the angle ACS, and
the angle CBD to the angle BCE.
10 Let a point F be taken at random
on BD;

from AE the greater let AG be cut off
equal to AF the less; [1. 3]

and let the straight lines FC, GB be joined.

[Post 1]

is Then, since AF is equal to AG and
AB to AC,

the two sides FA, AC are equal to the
two sides GA, AB, respectively;
and they contain a common angle, the angle FAG.

10 Therefore the base FC is equal to the base GB,
and the triangle AFC is equal to the triangle AGB,

and the remaining angles will be equal to the remaining angles
respectively, namely those which the equal sides subtend,
that is, the angle ACF to the angle ABG,

11 and the angle AFC to the angle AGB. [l 4]
And, since the whole AF is equal to the whole AG,

and in these AB is equal to AC,
the remainder BF is equal to the remainder CG.
But FC was also proved equal to GB;
v> therefore the two sides BF, FC are equal to the two sides
CG, GB respectively;

and the angle BFC is equal to the angle CGB,

while the base BC is common to them;
therefore the triangle BFC is also equal to the triangle CGB,
(S and the remaining angles will be equal to the remaining
angles respectively, namely those which the equal sides
subtend;

therefore the angle FBC is equal to the angle GCB,

and the angle BCF to the angle CBG.

4° Accordingly, since the whole angle ABG was proved
equal to the angle ACF,

and in these the angle CBG is equal to the angle BCF,
the remaining angle ABC is equal to the remaining angle
ACB;
45 and they are at the base of the triangle ABC.

But the angle FBC was also proved equal to the angle GCB;
and they are under the base.
Therefore etc.\ Q. e. d.

1. the equal straight lines (meaning the equal sida). cf.\ note on the similar
expression in Prop.~4, lines 1, 3.

10, Let a point F be taken at random on BD, ettrjtfu irl rQt BA Tv%6r miitur t Z,
where rvxhr a-rjfiof means ``a chance point/'

17, the two sides FA, AC are equal to the two aides OA, AB respectively, l(So
at ZA, AT 0url toTi HA, AB loat flair iuaripa tzartpf. Here, and in numberless later
passages, I have inserted the word ``sides'' for the reason given in the note on 1- r, line 10.
It would have been permissible to supply either ``straight lines'' or ``sides''; but on the
whole M sides ``seems to be more in accordance with the phraseology of \prop{1}{4},

33. the base BC is common to them, i.e., apparently, common to the origin, as
the arW-ur in pdtrti arww ttotrij can only refer to yuyLa and yuAa preceding. Simson wrote
``and the base BC is common to the two triangles BFC, CGB , Todhunter left out these
words as being of no use and tending to perplex a beginner. But Euclid evidently chose
to quote the conclusion of 1. 4 exactly I the first phrase of that conclusion is that the bases
(of the two triangles) are equal, and, as the equal bases are here the same base, Euclid
naturally substitutes the worn ``common'' for '* equal.''

48. As ``(Being) what it was required to prove ``{or ``do ``) is somewhat long, 1 shall
henceforth write the time-honoured ``Q. e. D. and ``<}. F_ F.'' for irtp titt ititfit and ortp
Wet rotfyrat.

According to Proclus (p.~250, 20) the discoverer of the fact that in any
isosceles triangle the angles at the base are equal was Thales, who however
is said to have spoken of the angles as being similar, and not as being equal.
(cf.\ ArisL De caelo iv. 4, 311 b 34 n-pos ifiolas ymvlat dxuVcrtu dxpoVcvov where
equal angles are meant.)

A pre-Euclidean proof of \prop{1}{5}.

One of the most interesting of the passages in Aristotle indicating differences
between Euclid's proofs and those with which Aristotle was familiar, in other
words, those of the text-books immediately preceding Euclid's, has reference to
the theorem of 1. 5. The passage {Anal. Prior. 1. 34, 41 b 13 — 21) is so
important that I must quote it in Kill. Aristotle is illustrating the fact that in
any syllogism one of the propositions must be affirmative and universal
{uttftJAou}. ``This,'' he says, ``is better shown in the case of geometrical
propositions ``{b> toii StaypapiMurtv), e.g.\ the proposition that the angles at Ike
bast of an isosceles triangle are equal.

``For let A, B be drawn [i.e.\ joined] to the centre.

``If, then, we assumed (i) that the angle AC [i.e.\ A + C] is equal to the
angle BD [i.e.\ B + D\ without asserting generally
that the angles of semicircles are equal, and again
(a) that the angle C is equal to the angle D without
making the further assumption that the two angles of
all segments art equal, and if we then inferred, lastly,
that, since the whole angles are equal, and equal
angles are subtracted from them, the angles which
remain, namely E, F, are equal, we should commit
a petitio principii, unless we assumed [generally j
that, when equals art subtracted from equals, tht
remainders are equal.''

The language is noteworthy in some respects.

(i ) A, B are said to be drawn (iyy/«Viii) to the centre (of the circle of
which the two equal sides are radii) as if A, B were not the angular points but
the sides or the radii themselves. (There is a parallel for this in Eucl. iv. 4.)

(2) ``The angle AC'' is the angle which is the sum of A and C, and A
means here the angle at A of the isosceles triangle shown in the figure, and
afterwards spoken of by Aristotle as E, while C is the ``mixed ``angle between
AB and the circumference of the smaller segment cut off by it.

(3) The ``angle of a. semicircle'' (i.e.\ the ``angle'' between the diameter
and the circumference, at the extremity of the diameter) and the ``angle of a
segment'' appear in Euclid tn. 16 and 111. Def. 7 respectively, obviously as
survivals from earlier text-books.

But the most significant facts to be gathered from the extract are that in
the text-books which preceded Euclid's ``mixed ``angles played a much more
important part than they do with Euclid, and, in particular, that at least two
propositions concerning such angles appeared quite at the beginning, namely
the propositions that the (mixed) angles of semicircles art equal and that the two
(mixed) angles of any segment of a circle art equal. The wording of the first
of the two propositions is vague, but it does not necessarily mean more than
that the two (mixed) angles in one semicircle are equal, and I know of no
evidence going to show that it asserts that the angle of any one semicircle is
equal to the angle of any other semicircle (of different size). It is quoted in
the same form, ``because the angles of semicircles are equal,'' in the Latin
translation from the Arabic of Heron's Catopirica, Prop.~9 (Heron, Vol. 11.,
Teubner, p.~334), but it is only inferred that the different radii of one circle
make equal ``angles'' with the circumference; and in the similar proposition
of the Pseudo-Euclidean Catoptriea (Euclid, Vol. vn., p.~394) angles of the
same sort in one circle are said to be equal ``because they are (angles) of
a semicircle.'' Therefore the first of the two propositions may be only a
particular case of the second.

But it is remarkable enough that the second proposition (that the two
``angles of'' any segment of a circle art equal) should, in earlier text-books, have
been placed before the theorem of Eucl. 1. 5. We can hardly suppose it to
have been proved otherwise than by the superposition of the semicircles into
which the circle is divided by the diameter which bisects at right angles the
base of the segment; and no doubt the proof would be closely connected with
that of Thales' other proposition that any diameter of a circle bisects it, which
must also (as Proclus indicates) have been proved by superposing one of the
two parts upon the other.

It is a natural inference from the passage of Aristotle that Euclid's proof of

i. 5 was his own, and it would thus appear that his innovations as regards
order of propositions and methods of proof began at the very threshold of the
subject.

Proof without producing the sides.
In this proof, given by Proclus (pp.~148, 21—249, J 9)>  an  E are ta en
on AS, AC, instead of on AB, AC produced, so that AD, AEte equal. The
method of proof is of course exactly like Euclid's, but it does not establish the
equality of the angles beyond the base as well.

Pappus* proof.

Proclus (pp.~249, 20 — 250, 1 2) says that Pappus proved the theorem in a
still shorter manner without the help of any construction whatever.

This very interesting proof is given as follows:

``Let ABC be an isosceles triangle, and AB equal to
AC.

Let us conceive this one triangle as two triangles, and let
us argue in this way.

Since AB is equal to AC, and AC to AB,
the two sides AB, AC are equal to the two sides AC, AB.

And the angle BA C is equal to the angle CAB, for it is
the same.

Therefore all the corresponding parts (in the triangles) are equal, namely

BC to BC,

the triangle ABC to the triangle ABC (i.e.\ ACB),

the angle ABC to the angle ACB,

and the angle ACB to the angle ABC,

(for these are the angles subtended by the equal sides AB, A C.

Therefore in isosceles triangles the angles at the base are equal.''

This will no doubt be recognised as the foundation of the alternative
proof frequently given by modern editors, though they do not refer to Pappus.
But they state the proof in a different form, the common method being to
suppose the triangle to be taken up, turned over, and placed again upon itself,
after which the same considerations of congruence as those used by Euclid in
1. 4 are used over again. There is the obvious difficulty that it supposes the
triangle to be taken up and at the same time to remain where it is, (Cf.
Dodgson's humorous remark upon this, Euclid and Ail modern Rivals, p.~47.)
Whatever we may say in justification of the proceeding (e.g.\ that the triangle
may be supposed to leave a tract), it is really equivalent to assuming the
construction (hypothetical, if you will) of another triangle equal in all respects
to the given triangle; and such an assumption is not in accordance with
Euclid's principles and practice.

It seems to me that the form given to the proof by Pappus himself is by far
the best, for the reasons (i) that it assumes no construction of a second
triangle, real or hypothetical, (2) that it avoids the distinct awkwardness
involved by a proof which, instead of merely quoting and applying the result
of a previous proposition, repeats, with reference to a new set of data, the
process by which that result was established. If it is asked how we are to
realise Pappus' idea of two triangles, surely we may answer that we keep to one
triangle and merely view it in two aspects. If it were a question of helping a
beginner to understand this, we might say that one triangle is the triangle
looked at in front and that the other triangle is the same triangle looked at
from behind; but even this is not really necessary.

Pappus' proof, of course, does not include the proof of the second part of
the proposition about the angles under the base, and we should still have to
establish this much in the same way as Euclid does.

Purpose of the second part of the theorem.

An interesting question arises as to the reason for Euclid's insertion of the
second part, to which, it will be observed, the converse proposition 1. 6 has
nothing corresponding. As a matter of fact, it is not necessary for any
subsequent demonstration that is to be found in the original text of Euclid,
but only for the interpolated second case of 1. 7; and it was perhaps not
unnatural that the undoubted genuineness of the second part of 1. 5 convinced
many editors that the second case of 1. 7 must necessarily be Euclid's also.
Proctus' explanation, which must apparently be the right one, is that the
second part of 1. 5 was inserted for the purpose of fore-arming the learner
against a possible objection (frimturif), as it was technically called, which might
be raised to 1. 7 as given in the text, with one case only. The objection would,
as we have seen, take the specific ground that, as demonstrated, the theorem
was not conclusive, since it did not cover all possible cases. From this point
of view, the second part of 1. 5 is useful not only for 1. 7 but, according to
Proclus, for 1. 9 also. Simson does not seem to have grasped Proclus'
meaning, for he says: ``And Proclus acknowledges, that the second part of
Prop.~5 was added upon account of Prop.~7 but gives a ridiculous reason for
it, 'that it might afford an answer to objections made against the 7th,' as if the
case of the 7th which is left out were, as he expressly makes it, an objection
against the proposition itself.''

Proposition 6.

If in a triangle two angles be equal to one another, the
sides which subtend the equal angles will also be equal to one
another.

Let ABC be a triangle having the angle ABC equal to
the angle ACS',

I say that the side AB is also equal to the
side AC.

For, if AB is unequal to AC, one of them is
greater.

Let AB be greater; and from AB the
greater let DB be cut off equal to AC the less;

let DC be joined.

Then, since DB is equal to AC,
and BC is common,

the two sides DB, BC are equal to the two sides AC,
CB respectively;

*$6 BOOK I [i. 6

and the angle DBC is equal to the angle ACB;

therefore the base DC is equal to the base AB,
and the triangle DBC will be equal to the triangle ACB,

the less to the greater:
which is absurd.

Therefore AB is not unequal to AC;
it is therefore equal to it.
Therefore etc

Q. E. D.

Euclid assumes that, because D is between A and B, the triangle DBC
is less than the triangle ABC. Some postulate is necessary to justify this
tacit assumption; considering an angle less than two right angles, say the
angle ACB in the figure of the proposition, as a cluster of rays issuing from
C and bounded by the rays CA f CB, and joining AB (where A, B are any
two points on CA, CB respectively), we see that to each successive ray taken
in the direction from CA to CB there corresponds one point on AB in which
the said ray intersects AB, and that all the points on AB taken in order from
A to B correspond uni vocally to all the rays taken in order from CA to
CB, each point namely to the ray intersecting AB in the point.

We have here used, for the first time in the Elements, the method of
redact in ad absurdum, as to which I would refer to the section above (pp.~136,
140) dealing with this among other technical terms.

This proposition also, being the converse of the preceding proposition,
brings us to the subject of

Geometrical Conversion.

This must of course be distinguished from the logical conversion of a
proposition. Thus, from the proposition that alt isosceles triangles have the
angles opposite to the equal sides equal, logical conversion would only enable
us to conclude that some triangles with two angles equal are isosceles. Thus
1. 6 is the geometrical, but not the logical, converse of 1. 5. On the other
hand, as De Morgan points out (Companion to the Almanac, 1849, p.~7), 1. 6 is
a purely logical deduction from t. 5 and 1. 18 taken together, as is 1. 19 also.
For the general argument see the note on 1. 19. For the present proposition
it is enough to state the matter thus. Let X denote the class of triangles
which have the two sides other than the base equal, Y the class of triangles
which have the base angles equal; then we may call aan-X the class of
triangles having the sides other than the base unequal non- Y the class of
triangles having the base angles unequal.

Thus we have

aii x is r, [1. 5]

All aon-X is non-K; [1. 18)
and it is a purely logical deduction that

All Y is X. [1. 6]

According to Proclus (p.~252, 5 sqq.) two forms of geometrical conversion
were distinguished.

(1) The leading form {rpoyovji.iyin). the conversion par excellence (rj xvpum
awKrrpoitf), is the complete or simple conversion in which the hypothesis
and the conclusion of a theorem change places exactly, the conclusion of the
theorem being the hypothesis of the converse theorem, which again establishes,
as its conclusion, the hypothesis of the original theorem. The relation between
the first part of 1. 5 and 1. 6 is of this character. In the former the hypothesis
is that two sides of a triangle are equal and the conclusion is that the angles
at the base are equal, while the converse {1. 6) starts from the hypothesis that
two angles are equal and proves that the sides subtending them are equal.

(2) The other form of conversion, which we may call partial, is seen
in cases where a theorem starts from two or more hypotheses combined into
one enunciation and leads to a certain conclusion, after which the converse
theorem takes this conclusion in substitution for one of the hypotheses of
the original theorem and from the said conclusion along with the rest of the
original hypotheses obtains, as its conclusion, the omitted hypothesis of the
original theorem, r, 8 is in this sense a converse proposition to 1. 4; for 1. 4
takes as hypotheses ( 1 ) that two sides in two triangles are respectively equal,
(a) that the included angles are equal, and proves (3) that the bases are equal,
while 1. 8 takes {1) and (3) as hypotheses and proves (2) as its conclusion. It
is clear that a conversion of the leading type must be unique, while there
may be many partial conversions of a theorem according to the number of
hypotheses from which it starts.

Further, of convertible theorems, those which took as their hypothesis
the genus and proved a, property were distinguished as the leading theorems
(rponfyovptva), while those which started from the property as hypothesis
and described, as the conclusion, the genus possessing that property were the
converse theorems. 1. 5 is thus the leading theorem and [. 6 its converse,
since the genus is in this case taken to be the isosceles triangle.

Converse of second part of \prop{1}{5}.

Why, asks Proclus, did not Euclid convert the second part of \prop{1}{5} as well ?
He suggests, properly enough, two reasons: (1) that the second part of 1. 5
itself is not wanted for any proof occurring in the original text, but is only put
in to enable objections to the existing form of later propositions to be met,
whereas the converse is not even wanted for this purpose; (2) that the converse
could be deduced from t. 6, if wanted, at any time after we have passed 1. 13,
which can be used to prove that, if the angles formed by producing two sides
of a triangle beyond the base are equal, the base angles themselves are equal.

Proclus adds a proof of the converse of the second part of 1. 5. i.e.\ of the
proposition that, if the angles formed by producing two
sides of_ a triangle beyond the base are equal, the triangle a

is isosceles; but it runs to some length and then only „ A

effects a reduction to the theorem of 1. 6 as we have it. AA

As the result of this should hardly be assumed, a better / \\

proof would be an independent one adapting Euclid's / 

own method in 1. 6. Thus, with the construction of 1. 5, l''~~~\
we first prove by means of 1. 4 that the triangles BFC, /-''*~\

CGB are equal in all respects, and therefore that FC is y~ ``

equal to GB, and the angle BFC equal to the angle CGB. D g

Then we have to prove that AF, AG are equal. If they
are not, let AF be the greater, and from FA cut off FH equal to GA.
Join CH.

Then we have, in the two triangles HFC, AGB,

two sides HF, FC equal to two sides AG, GB
and the angle HFC equal to the angle AGB.

Therefore (l 4) the triangles HFC, AGB are equal. But the triangles
BFC, CGB are also equal

Therefore {if we take away these equals respectively) the triangles HBC,
ACB are equal: which is impossible.

Therefore AF, AG are not unequal.

Hence AF'm equal to AG and, if we subtract the equals BF, CG respec-
tively, AB is equal to A C.

This proof is found in the commentary of an-Nairlzi (ed. Besthom-Heiberg,
p.~61; ed. Curtze, p.~50).

Alternative proofs of \prop{1}{6}.

Todhunter points out that \prop{1}{6}, not being wanted till \prop{2}{4}, could be
postponed till later and proved by means of i. 26. Bisect the angle BAC
by a straight line meeting the base at D. Then the triangles ABD, A CD
are equal in all respects.

Another method depending on 1. 26 is given by an-Nairlzi after that
proposition.

Measure equal lengths BD, CE along the sides BA, CA.
Join BE, CD.

Then [1. 4] the triangles DBC, ECB are equal in all
respects;

therefore EB, DC are equal, and the angles BEC, CDB
are equal.

The supplements of the latter angles are equal [1. 13],
and hence the triangles ABE, A CD have two angles equal respectively and
the side BE equal to the side CD.

Therefore [1. 26] AB is equal to AC.

P HO POSITION 7.

Given two straight lines constructed on a straight line
[from its extremities) and meeting in a point, there cannot be
constructed on the same straight line {from its extremities),
and on the same side of it, two other straight lines meeting in

5 another point and equal to the former two respectively, namely
each to that which has the same extremity with it.

For, if possible, given two straight lines AC, CB con-
structed on the straight line AB and meeting
at the point C, let two other straight lines

10 AD, DB be constructed on the same straight
line AB, on the same side of it, meeting in
another point D and equal to the former two
respectively, namely each to that which has
the same extremity with it, so that CA is

15 equal to DA which has the same extremity A with it, and
CB to DB which has the same extremity B with it; and let
CD be joined.

Then, since AC is equal to AD,

the angle A CD is also equal to the angle ADC; [1. 5]
20 therefore the angle ADC is greater than the angle DCB;
therefore the angle CDB is much greater than the angle
DCB.

Again, since CB is equal to DB,

the angle CDB is also equal to the angle DCB.
25 But it was also proved much greater than it:
which is impossible.
Therefore etc.\ Q. E. D.

1 — 6. In an English translation of the enunciation of this proposition it is absolutely
necessary, in order to make it intelligible, to insert some words which are not in the Greek.
The reason is partly that the Greek enunciation is itself very elliptical, and partly that some
words used in it conveyed more meaning than the corresponding words in English do.
Particularly is this the case with oC evaratrboorrat iri ``there shall not be constructed upon,''
since evrUraaSat. is the regular word for constructing a triangle in particular. Tbus a Greek
would easily understand avaraB-fyrarrat iri as meaning the construction of two lines forming
a triangle on a given straight line as base; whereaa.ro ``construct two straight lines on a
straight line ``is not in English sufficiently definite unless we explain that they are drawn
from the ends of the straight line to meet at a point. I have had the less hesitation in putting
in the words ``from its extremities'' because they are actually used by Euclid in the somewhat
similar enunciation of 1. at.

How impossible a literal translation into English is, if it is to convey the meaning of the
enunciation intelligibly, will be clear from the following attempt to render literally: ``On the
same straight line there shall not be constructed two other straight lines equal, each to each,
to the same two straight lines, (terminating) at different points on the same side, having the
same extremities as the original straight lines ``(irl riji a*rj)i eiBctai S60 raft airaii tiStlmt
AXXor ivo evSttat focu inar4pa ixarlpa ov ewTaBfaairrai vpdr dWtj rtai AXXI OTjfuU i-wl f A ovrd
pift r« a*''k ripara (x'oeai roll it Apx*)' tiltlaa).

The reason why Euclid allowed himself to use, in this enunciation, language apparently
so obscure is no doubt that the phraseology was traditional and therefore, vague as it was,
had a conventional meaning which the contemporary geometer well understood. This is
proved, I think, by the occurrence in Aristotle (Meteorologica 111, 5, 376 a j sqq.) of the very
same, evidently technical, expressions. Aristotle is there alluding to the theorem given by
Eutocius from Apollonius' Plane Loci to the effect that, if H, K be two fixed points and M
such a variable point that the ratio of MH to MK is a given ratio (not one of equality), the
locus of M is a circle. (For an account of this theorem see note on vt. 3 below.) Now
Aristotle says ``The lines drawn up from H, K in this ratio cannot be constructed to two
different points of the semicircle A ``(of ttr iri tCh HK inayb/urai ypawial tr roirip rif
\t>yy uLr ffvffTadaovTat rov £q> y A iuxurXfou rpbt AWo teal A\\o a-rjfLtiov).

If a paraphrase is allowed instead of a translation adhering as closely as possible to the
original, Simson's is the best that could be found, since the fact that the straight lines form
triangles on the same base is really conveyed in the Greek. Simson's enunciation is, Upon
the same base, and on the same side of i(, there cannot lie two triangles line have their sides
which are terminated in one extremity of the iate equal to one another, and liiewue these
which art terminated at the other extremity. Th. Taylor (the translator of Proclus) attacks
Simson's alteration as ``indiscreet'' and as detracting from the beauty and accuracy of
Euclid's enunciation which are enlarged upon by Proclus in his commentary. Yet, when
Taylor says ``Whatever difficulty learners may find in conceiving this proposition abstractedly
is easily removed by its exposition in the figure,'' he really gives his case away. The fact is
that Taylor, always enthusiastic over his author, was nettled by Simson's slighting remarks
on Proclus' comments on the proposition. Simson had said, with reference to Proclus'
explanation of the bearing of the second part of 1. % on 1. 7, that it was not "worth while
to relate his [rifles at full length,'' to which Taylor retorts ``But Mr Simson was no
philosopher; and therefore the greatest part of these Commentaries must be considered by
him as trifles, from the want of a philosophic genius to comprehend their meaning, and
a taste superior to that of a mere mathematician, to discover their beauty and elegance.''

10. It would be natural to insert here the step ``but the angle ACD is greater than the
angle BCD. [C. N. 5.3''

tl, much greater, literally ''greater by much'' (roXXi fielfav). Simson and those who
follow him translate: ``much mart then is the angle BDC greater than the angle BCD,''
but the Greek for this would have to be toXXv (or iroXii) jiaX \i r tm...JU(JW>. troXX aXXof,
however, though used by Apotlonius, is not, apparently, found in Euclid or Archimedes.

Just as in \prop{1}{6} we need a Postulate to justify theoretically the statement that
CD falls within the angle ACB, so that the triangle DBC is less than the
triangle ABC, so here we need Postulates which shall satisfy us as to the
relative positions of CA, CB, CD on the one hand and of DC, DA, DB
on the other, in order that we may be able to infer that the angle BDC is
greater than the angle ADC, and the angle ACD greater than the angle BCD,

De Morgan {sp.~cit. p.~7) observes that 1. 7 would be made easy to
beginners if they were first familiarised, as a common notion, with ``if two
magnitudes be equal, any magnitude greater than the one is greater than any
magnitude less than the other.'' I doubt however whether a beginner would
follow this easily; perhaps it would be more easily apprehended tn the form
``if any magnitude A is greater than a magnitude B, the magnitude A is
greater than any magnitude equal to B, and (a fortiori) greater than any
magnitude less than B.''

It has been mentioned already (note on 1. 5) that the second case of 1. 7
given by Simson and in our text-books generally is not in the original text
(the omission being in accordance with Euclid's general practice of giving
only one case, and that the most difficult, and leaving the others to be worked
out by the reader for himself). The second case is given by Proclus as the
answer to a possible objection to Euclid's proposition, which should assert that
the proposition is not proved to be universally true, since the proof given does
not cover all possible cases. Here the objector is supposed to contend that
what Euclid declares to be impossible may still be possible if one pair of lines
lie wholly within the other pair of lines; and the second part of 1. 5 enables
the objection to be refuted.

If possible, let AD, DB be entirely within the triangle formed by AC,
CB with AB, and let AC be equal to AD and BC
to BD.

Join CD, and produce AC, AD to E and F.

Then, since AC is equal to AD, o(

the triangle ACD is isosceles,
and the angles ECD, FDC under the base are equal.
But the angle ECD is greater than the angle BCD ,

therefore the angle FDC is also greater than the angle

BCD.

Therefore the angle BDC is greater by far than the angle BCD.

Again, since DB is equal to CB,
the angles at the base of the triangle BDC are equal, [1. 5]

that is, the angle BDC is equal to the angle BCD.

Therefore the same angle BDC is both greater than and equal to the angle
BCD: which is impossible.

The case in which D falls on AC or BC does not require proof,

I have already referred (note on 1. r) to the mistake made by those
editors who regard \prop{1}{7} as being of no use except to prove 1. 8. What 1. 7
proves is that if, in addition to the base of a triangle, the length of the side
terminating at each extremity of the base is given, only one triangle satisfying
these conditions can be constructed on one and the same side of the given
base. Hence not only does 1. 7 enable us to prove 1. 8, but it supplements
1. 1 and 1. 22 by showing that the constructions of those propositions give one
triangle only on one and the same side of the base. But for [. 7 this could
not be proved except by anticipating in, 10, of which therefore 1. 7 is the
equivalent for Book 1. purposes. Dodgson (Etttfid and his modern Rivals,
pp.~194 — 5) puts it in another way. ``It [l. 7] shows that, of airplane figures
that can be made by hingeing rods together, the /Aree-sided ones (and these
only) are rigid (which is another way of stating the fact that there cannot be
two such figures on the same base). This is analogous to the fact, in relation
to solids contained by plane surfaces hinged together, that any such solid is
rigid, there being no maximum number of sides. And there is a close analogy
between 1. 7, 8 and in. 23, 24. These analogies give to geometry much of its
beauty, and I think that they ought not to be lost sight of.'' It will therefore
be apparent how ill-advised are those editors who eliminate 1. 7 altogether and
rely on Philo's proof for 1. 8.

Proclus, it may be added, gives (pp.~2 68, 19 — 269, 10) another explanation
of the retention of \prop{1}{7}, notwithstanding that it was apparently only required
for 1. 8. It was said that astronomers used it to prove that three successive
eclipses could not occur at equal intervals of time, i.e.\ that the third could not
follow the second at the same interval as the second followed the first; and it
was argued that Euclid had an eye to this astronomical application of the
proposition. But, as we have seen, there are other grounds for retaining the
proposition which are quite sufficient of themselves.

Proposition 8.

If two triangles have the two sides equal to two sides
respectively, and have also the base equal to the base, they will
also have the angles equal which are contained by the equal
straight lines.
s Let ABC, DBF be two triangles having the two sides
AB, AC equal to the two sides
DE, DF respectively, namely
AB to DE, and AC to DF\ and
let them have the base BC equal
10 to the base EF;

I say that the angle BAC is
also equal to the angle EDF.

For, if the triangle ABC be
applied to the triangle DEF, and if the point B be placed on
15 the point E and the straight line BC on EF,
the point C will also coincide with F,
because BC is equal to EF.

Then, BC coinciding with EF,

BA, AC will also coincide with ED, DF;
ao for, if the base BC coincides with the base EF, and the sides
BA, AC do not coincide with ED, DF but fall beside them
as EG, GF,

then, given two straight lines constructed on a straight

line (from its extremities) and meeting in a point, there will

»5 have been constructed on the same straight line (from its

extremities), and on the same side of It, two other straight

lines meeting in another point and equal to the former

two respectively, namely each to that which has the same

extremity with it

30 But they cannot be so constructed. [1. 7}

Therefore it is not possible that, if the base BC be applied

to the base EF, the sides BA, AC should not coincide with

ED, DF;

they will therefore coincide,
35 so that the angle BAC will also coincide with the angle
EDF t and will be equal to it.

If therefore etc.\ q. e. d.

19. BA, AC. The text has here ``BA, CA.''

SI, fall be aide them. The Greek has the future, i-apoXXdfoLvf. TapaXkm* means
``to pass by without touching,'' ``to miss'' or ``to deviate.''

As pointed out above (p.~157) 1, 3 is a par tied converse of t. 4.

It is to be observed that in \prop{1}{8} Euclid is satisfied with proving the equality
of the vertical angles and does not, as in 1. 4, add that the triangles are equal,
and the remaining angles are equal respectively. The reason is no doubt (as
pointed out by Proclus and by Savile after him) that, when once the vertical
angles are proved equal, the rest follows from 1. 4, and there is no object in
proving again what has been proved already.

Aristotle has an allusion to the theorem of this proposition in Meorokgka
in. 3, 373 a 5 — 16. He is speaking of the rainbow and observes that, if equal
rays be reflected from one and the same point to one and the same point, the
points at which reflection takes place are on the circumference of a circle.
``For let the broken lines ACB, AFB, ADB be all reflected from the point
A to the point B (in such a way that) AC, AF, AD are all equal to one
another, and the lines {terminating) at B, i.e.\ CB, FB, DB, are likewise all
equal; and let AEB be joined. It follows that the triangles art equal; for
they are upon the equal (base) AEB.''

Heiberg {Mathtmatisehes tu Aristolelts, p.~18) thinks that the form of the
conclusion quoted is an indication that in the corresponding proposition tc.
Eucl. 1. 8, as it lay before Aristotle, it was maintained that the triangles were
equal, and not only the angles, and ``we see here therefore, in a clear example,
how the stones of the ancient fabric were recut for the rigid structure of his

Elements. ``I do not, however, think that this inference from Aristotle's
language as to the form of the pre- Euclidean proposition is safe. Thus if we,
nowadays, were arguing from the data in the passage of Aristotle, we should
doubtless infer directly that the triangles are equal in all respects, quoting I 8
alone. Besides, Aristotle's language is rather careless, as the next sentences
of the same passage show. ``Let perpendiculars,''
he says, ``be drawn to AEB from the angles, CE
from C, FE from and DE from D. These, then,
are equal; for they are all in equal triangles, and
in one plane; for all of them are perpendicular
to AEB, and they meet at one point E. There-
fore the (line) drawn (through C, F, D) will be a
circle, and its centre (will be) E.'' Aristotle should

obviously have proved that the three perpendiculars will meet at one point E
cm AEB before he spoke of drawing the perpendiculars CE, FE, DE.
This of course follows from their being ``in equal triangles'' {by means of
EucL i. 26); and then, from the fact that tbe perpendiculars meet at one
point on AB, it can be inferred that all three are in one plane.

Philo's proof of \prop{1}{8}.

This alternative proof avoids the use of 1. 7, and it is elegant; but it is
inconvenient in one respect, since three cases have to be distinguished.
Proctus gives the proof in the following order (pp.~266, 15 — 168, 14).

I-et ABC, DEF be two triangles having the sides AB, A C equal to the
sides DE, DE respectively, and the base BC equal to the base EF.

Let the triangle ABC be applied to the triangle DEF, so that B is placed
on E and BC on EF, but so that A falls on the opposite side of EF from D,
taking the position G. Then C will coincide with F, since BC is equal to
EF.

Now FG will either be in a straight line with DF, or make an angle with
it, and in the tatter case the angle will either be interior (™ro to Ivtos) to the
figure or exterior (no.™ to Item).

I. Let FG be in a straight line with
DF.

Then, since DE is equal to EG-, and
DFG is a straight line,

DEG is an isosceles triangle, and the
angle at D is equal to the angle at G.

[' 5].

II. Let DF, FG form an angle interior to the figure.
Let DG be joined.
Then, since DE, EG are equal,

the angle EDG is equal to the angle
EGD.

Again, since DF is equal to FG,
the angle FDG is equal to the angle
FGD.

Therefore, by addition,
the whole angle EDF is equal to the
whole angle EGF.

III. Let DF, FG form an angle ex/trier to the figure.

Let DG be joined.

The proof proceeds as in the last case,
except that subtraction takes the place of
addition, and

the remaining angle EDF is equal to the
remaining angle ECF.

Therefore in all three cases the angle
EDF is equal to the angle EGF, that is,
to the angle BAC.

It wrill be observed that, in accordance with the practice of the Greek
geometers in not recognising as an ``angle'' any angle not less than two right
angles, the re-entrant angle of the quadrilateral JDEGF'm. ignored and the angle
DFG is said to be outside the figure.

Proposition 9.

To bisect a given rectilineal angle.

Let the angle BAC be the given rectilineal angle.

Thus it is required to bisect it.

Let a point D be taken at random on AB;
let AE be cut off from AC equal to AD; [1. 3]
let DE be joined, and on DE let the equilateral
triangle DEF be constructed;
let AF be joined.

I say that the angle BAC has been bisected by the
straight line AF.

For, since AD is equal to AE,
and AF is common,

the two sides DA, AF are equal to the two sides
EA, AF respectively.

And the base DF is equal to the base EF;

therefore the angle DAF is equal to the angle EAF.

[.. 8]

Therefore the given rectilineal angle BAC has been
bisected by the straight line AF. q, e. f.

It will be observed from the translation of this proposition that Euclid
does not say, in his description of the construction, that the equilateral triangle
should be constructed on the side of HE opposite to A; he leaves this to be
inferred from his figure. There is no particular value in Proclus' explanation
as to how we should proceed in case any one should assert that he could not
recognise the existence of any space below DE. He supposes, then, the
equilateral triangle described on the side of DE towards A, and hence has to
consider three cases according as the vertex of the equilateral triangle falls
on A, above A or below it. The second and third cases do not d'ffer
substantially from Euclid's. In the first case, where ADE is the. equilateral
triangle constructed on DE, take any point F or\ AD, and from AE cut off
AG equal to AF. Join DG, EF meeting in H\ and
join AH. Then AH is the bisector required.

Proclus also answers the possible objection that
might be raised to Euclid's proof on the ground that
it assumes that, if the equilateral triangle be described
on the side of DE opposite to A, its vertex .f will lie
within the angle BAC. The objector is supposed to
argue that this is not necessary, but that F might fall
either on one of the lines forming the angle or outside
it altogether. The two cases are disposed of thus.

Suppose Fxx> fall as shown in the two figures below respectively.

Then, since FD is equal to FE,
the angle FDE is equal to the angle FED.

Therefore the angle CED is greater than the angle FDE; and, in the
second figure, a fortiori, the angle CED is greater than the angle BDE.

But, since ADE is an isosceles triangle, and the equal sides are produced,

the angles under the base are equal,

i.e., the angle CED is equal to the angle BDE.

But the angle CED was proved greater: which is impossible.

Here then is the second case in which, in Proclus' view, the second part
of i. 5 is useful for refuting objections.

On this proposition Proclus takes occasion (p.~27 r, rj — 19) to emphasize
the fact that the given angle must be rectilineal, since the bisection of any sort
of angle (including angles made by curves with one another or with straight
lines) is not matter for an elementary treatise, besides which it is questionable
whether such bisection is always possible. ``Thus it is difficult to say
whether it is possible to bisect the so-called horn-like angle ``(formed by the
circumference of a circle and a tangent to it).

Trisection of an angle.

Further it is here that Proclus gives us his valuable historical note about
the trisection of any acute angle, which (as well as the division of an angle in
any given ratio) requires resort to other curves than circles, i.e.\ curves of the
species which, after Geminus, he calls ``mixed.'' ``This,'' he says (p.~372,
1 — 12), ``is shown by those who have set themselves the task of trisecting such
a given rectilineal angle. For Nicomedes trisected any rectilineal angle by
means of the conchoidal lines, the origin, order, and properties of which he
has handed down to us, being himself the discoverer of their peculiarity.
Others have done the same thing by means of the quadratrices of Hippias
and Nicomedes, thereby again using 'mixed' curves. But others, starting
from the Archimedean spirals, cut a given rectilineal angle in a given ratio.''

(a) T riscction by means of the conchoid.

I have already spoken of the conchoid of Nicomedes {note on Def. t,
pp.~160 — i); it remains to show how it could be used for trisecting an
angle. Pappus explains this (iv. pp.~274 — 5) as follows.

Let ABC be the given acute angle, and from any point A in AB draw
A C perpendicular to BC.

B O

Complete the parallelogram FBCA and produce FA to a point E such
that, if BE be joined, BE intercepts between AC and AE a length DE equal
to twice AB.

I say that the angle EBC is one-third of the angle ABC.

For, joining A to G, the middle point of DE, we have the three straight
lines AG, DG, EG equal, and the angle AGO is double of the angle A ED
or EBC.

But DE is double of AB;
therefore AG, which is equal to DG, is equal to AB.

Hence the angle AGD is equal to the angle ABG.

Therefore the angle ABD is also double of the angle EBC;
so that the angle EBC is one-third of the angle ABC.

So far Pappus, who reduces the construction to the drawing of BE so
that DE shall be equal to twice AB.

This is what the conchoid constructed with B as pole, AC 'as directrix, and
distance equal to twice AB enables us to do; for that conchoid cuts AE in
the required point E.

(6) Use of the quadratrix.

The plural quadratrices in the above passage is a Hellenism for the
singular quadratrix, which was a curve discovered by Hippias of El is about
420 B.C. According to Proclus (p.~356, 11) Hippias proved its properties;
and we are told (1) in the passage quoted above that Nicomedes also
investigated it and that it was used for trisecting an angle, and (2) by Pappus
(iv. pp.~350, 33— 353, 4) that it was used by Dinostratus and Nicomedes and
some more recent writers for squaring the circle, whence its name. It is
described thus (Pappus iv. p.~352).

Suppose that ABCD is a square and BED a quadrant of a circle with
centre A.

Suppose (1) that a radius of the circle moves
uniformly about A from the position AB to the
position AD, and (*) that in the same time the
line BC moves uniformly, always parallel to itself,
and with its extremity B moving along BA, from
the position BC to the position AD.

Then the radius AE and the moving line BC
determine at any instant by their intersection a
point F,

The locus of F'\s the quadratrix.

The property of the curve is that, if F is any point, the arc BED is
to the arc ED as AB is to FH.

In other words, if  is the angle FAD, p the radius vector AFa.nd a the
side of the square,

(p sin $)/« = $/Jt.

Now the angle EAD can not only be trisected but divided in any given
ratio by means of the quadratrix (Pappus iv. p.~386).

For let FJfbe divided at JCin the given ratio.

Draw KL parallel to AD, meeting the curve in L; join AL and produce
it to meet the circle in N.

Then the angles EAN, NAD are in the ratio of FK to KH, as is easily
proved.

(e) Use of the spiral of Archimedes.

The trisect ion of an angle, or the division of an angle in any ratio, by
means or the spiral of Archimedes is of course an equally simple matter.
Suppose any angle included between the two radii vectores OA and OB of the
spiral, and let it be required to cut the angle AOB in a given ratio. Since
the radius vector increases proportionally with the angle described by the
vector which generates the curve (reckoned from the original position of the
vector coinciding with the initial line to the particular position assumed), we
have only to take the radius vector OB (the greater of the two OA, OB),
mark off OC along it equal to OA, cut CB in the given ratio (at D say}, and
then draw the circle with centre and radius OD cutting the spiral in E.
Then OE will divide the angle AOB in the required manner.

Proposition 10.
To bisect a given finite straight line.

Let AB be the given finite straight line.

Thus it is required to bisect the finite straight line AB.

Let the equilateral triangle ABC be
constructed on it, [1. r]

and let the angle ACB be bisected by the
straight line CD; ft, 9]

I say that the straight line AB has
been bisected at the point D. .

For, since AC is equal to CB,
and CD is common,

the two sides A C, CD are equal to the two sides BC,
CD respectively;

and the angle A CD is equal to the angle BCD;

therefore the base AD is equal to the base BD. [1. 4]
Therefore the given finite straight line AB has been
bisected at D. q. e, f.

A poll on i us, we are told (P rod us, pp.~279, 16—280, 4), bisected a straight
line AB by a construction tike that of 1. 1.
With centres A, B, and radii AB, BA respec-
tively, two circles are described, intersecting in
C, p.~Joining CD, AC, CB, AD, DB, Apoi-
lonius proves in two steps that CD bisects AB.

(1) Since, in the triangles A CD, BCD,
two sides AC, CD are equal to two sides
BC, CD,
and the bases AD, BD are equal,
the angle A CD is equal to the angle
BCD.

[1.8]

(2) The latter angles being equal, and AC being equal to CB, while CE
is common,

the equality of AE, EB follows by \prop{1}{4}.

The objection to this proof is that, instead of assuming the bisection of
the angle ACB, as already effected by 1. 9, Apollo nius goes a step further
back and embodies a construction for bisecting the angle. That is, he
unnecessarily does over again what has been done before, which is open to
objection from a theoretical point of view.

Proclus (pp.~277, 25 — 279, 4) warns us against being moved by this
proposition to conclude that geometers assumed, as a preliminary hypothesis,
that a line is not made up of indivisible parts (1$ AjttpAr). This might be
argued thus. If a line is made up of indivisibles, there must be in a finite
line either an odd or an even number of them. If the number were odd,
it would be necessary in order to bisect the line to bisect an indivisible (the
odd one). In that case therefore it would not be possible to bisect a straight
line, if it is a magnitude made up of indivisibles. But, if it is not so made
up, the straight line can be divided ad infinitum or without limit (hf irtipov
faatpttTai). Hence it was argued (acrtV), says Proclus, that the divisibility
of magnitudes without limit was admitted and assumed as a geometrical
principle. To this he replies, following Geminus, that geometers did indeed
assume, by way of a common notion, that a continuous magnitude, i.e.\ a
magnitude consisting of parts connected together (owrjfijtow), is divisible
(SuuptTov). But infinite divisibility was not assumed by them; it was proved
by means of the first principles applicable to the case. ``For when,'' he
says, ``they prove that the incommensurable exists among magnitudes, and
that it is not all things that are commensurable with one another, what
else will any one say that they prove but that every magnitude can be
divided for ever, and that we shall never arrive at the indivisible, that
is, the least common measure of the magnitudes? This then is matter of
demonstration, whereas it is an axiom that everything continuous is divisible,
so that a finite continuous line is divisible. The writer of the Elements
bisects a finite straight line, starting from the latter notion, and not from any
assumption that it is divisible without limit'' Proclus adds that the proposition
may also serve to refute Xenocrates' theory of indivisible lines (di-ojiu* ypapjiat).
The argument given by Proclus to disprove the existence of indivisible lines
is substantially that used by Aristotle as regards magnitudes generally (cf
Physkt vi. 1, 231 a 21 sqq. and especially vl. 2, 133 b 15 — 32).

Proposition ii.

To draw a straight line at right angles to a given straight
line from a given point on it.

Let AB be the given straight line, and C the given point
on it.

s Thus it is required to draw from the point C a straight
line at right angles to the straight
line AB.

Let a point D be taken at ran-
dom on AC;
10 let CE be made equal to CD; [1. 3]
on DE let the equilateral triangle
FDE be constructed, [1. i]

and let FC be joined;

I say that the straight line FC has been drawn at right
is angles to the given straight line AB from C the given point
on it.

For, since DC is equal to CE,
and CF is common,

the two sides DC, CF are equal to the two sides EC,
20 CF respectively;

and the base DF is equal to the base FE;

therefore the angle DCF is equal to the angle ECF\

M]
and they are adjacent angles.

But, when a straight line set up on a straight line makes

»s the adjacent angles equal to one another, each of the equal

angles is right; [Def. 10]

therefore each of the angles DCF, FCE is right.

Therefore the straight line CF has been drawn at right

angles to the given straight line AB from the given point

30 C on it.

Q. E. F.

10. let CB be made equal to CD. The verb is mbrSu which, as welt as the othet
parts of Ktitiai, a constant iy used lor the passive of rWiftu ``lo plats ``; and the latter word
it constantly used in the sense of making, e.g., one straight line equal to another straight line.

1 )e Morgan remarks that this proposition, which is ``to bisect the angle
made by a straight line and its continuation ``[i.e.\ a flat angle], should be a
particular case of 1. 9, the constructions being the same. Thjs is certainly
worth noting, though I doubt the advantage of rearranging the propositions
in consequence.

Apollonius gave a construction for this proposition (see P rod us, p.~282, 8)
differing from Euclid's in much the same way as his construction for bisecting
a straight line differed from that of 1. 10. Instead of assuming an equilateral
triangle drawn without repeating the process of 1. 1, Apollonius takes D and
E equidistant from C as in Euclid, and then draws circles in the manner of

1. 1 meeting at F This necessitates proving again that DF\s equal to FE\
whereas Euclid's assumption of the construction of 1. 1 in the words ``let the
equilateral triangle FDE be constructed ``enables him to dispense with the
drawing of circles and with the proof that DF is equal to FE at the same
time. While however the substitution of Apollonius' constructions for 1. 10
and 1 1 would show faulty arrangement in a theoretical treatise like Euclid's,
they are entirely suitable for what we call practical geometry, and such may
have been Apollonius' object in these constructions and in his alternative for

'- »3- . . . .

Proclus gives a construction for drawing a straight line at right angles to
another straight line but from one end of it, instead of from an intermediate
point on it, it being supposed (for the sake of argument) that we are not
permitted to produce the straight line. In the commentary of an-Nairīzī (ed.
Besthorn-Heiberg, pp.~73 — 4; ed. Curtze, pp.~54 — 5) this construction is
attributed to Heron.

Let it be required to draw from A a straight line at right angles to AB.

On AB take any point C, and in the manner of the proposition draw CE
at right angles to AB.

from CE cut off CD equal to AC, bisect the
angle ACE by the straight line CF\ [1. 9]

and draw DF at right angles to CE meeting CF
in F. Join FA.

Then the angle FAC will be a right angle.

For, since, in the triangles ACF, DCF, the
two sides AC, CF are equal to the two sides
DC, CF respectively, and the included angles
ACF, DCFare equal,

the triangles are equal in all respects. [1, 4]

Therefore the angle at A is equal to the angle at D, and is accordingly a
right angle.

E

F

D

A (

a

PROPOSITION 12.

To a given infinite straight tine, from a given point
which is not on it, to draw a perpendicular straight line.

Let AB be the given infinite straight line, and C the
given point which is not on it;
Sthus it is required to draw to the given infinite straight
line AB, from the given point
C which is not on it, a per-
pendicular straight line.

For let a point D be taken

10 at random on the other side of

the straight line AB, and with

centre C and distance CD let

the circle EFG be described;

[Post. 3]

let the straight line EG

ij be bisected at H, [1. 10]

and let the straight lines CG, CH, CE be joined.

[Post. 1]
I say that CH has been drawn perpendicular to the given
infinite straight line AB from the given point C which is
not on it.
jo For, since GH is equal to HE,
and HC is common,

the two sides GH, HC are equal to the two sides
EH, HC respectively;
and the base CG is equal to the base CE;
35 therefore the angle CHG is equal to the angle EHC.

[1. 8]

And they are adjacent angles.

But, when a straight line set up on a straight line makes

the adjacent angles equal to one another, each of the equal

angles is right, and the straight line standing on the other is

jo called a perpendicular to that on which it stands. [Def. 10]

Therefore CH has been drawn perpendicular to the given

infinite straight line AB from the given point C which is

not on it

Q. E. F.

1. a perpendicular straight line, tiitrer ti$ttv ypapph'- This is the full expression
for a ptrpendkutar, KiBtrm meaning >tt f<il( or Ut down, so that the expression corresponds
10 our plumb-lint, if jcdrar is however constantly used atone for a perpendicular, ypafi
being understood.

10. on the other side of the straight line AB, literally ``towards the other parts of
the straight line AB,'' irl ri Irtpa ftfpif rt,s AB. cf.\ ``on the same side'' (6rl ri au-ri
pif''!) in Post, s *»d ``in both directions'' (ty inirtpa ri nipt)) in Def. 13.

``This problem,'' says Proctus (p.~183, 7 — 10), ``was first investigated
by Oenopides [5 th cent b,c], who thought it useful for astronomy. He
however calls the perpendicular, in the archaic manner, (a line drawn)
gnomon-wise (koto yvu/iova), because the gnomon is also at right angles to the
horizon.'' In this earlier sense the gnomon was a staff placed in a vertical
position for the purpose of casting shadows and so serving as a means of
measuring time (Cantor, Gezchichle der Mathemattk, i s , p.~161). The later
meanings of the word as used in Eucl. Book n. and elsewhere will be
explained in the note on Book n. Def. 2,

Proclus says that two kinds of perpendicular were distinguished, the ``plane''
(iiriwtSos) and the ``solid'' (err«pea), the former being the perpendicular
dropped on a line is a plane and the latter the perpendicular dropped on a
plane. The term ``solid perpendicular'' is sufficiently curious, but it may
perhaps be compared with the Greek term ``solid locus ``applied to a conic
section, apparently on the ground that it has its origin in the section of a
solid, namely a cone.

Attention is called by most editors to the assumption in this proposition
that, if only D be taken on the side of AB remote from C, the circle described
with CD as radius must necessarily cut AB in two points. To satisfy us of
this we need, as in I, 1, some postulate of continuity, e.g.\ something like that
suggested by Killing (see note on the Principle of Continuity above, p.~235):
``If a point [here the point describing the circle] moves in a figure which is
divided into two parts [by the straight line J, and if it belongs at the beginning
of the motion to one part and at another stage of the motion to the other
part, it must during the motion cut the boundary between the two parts,'' and
this of course applies to the motion in two directions from D.

But the editors have not, as a rule, noticed a possible objection to the
Euclidean statement of this problem which is much more difficult to dispose
of at this stage, i.e.\ without employing any proposition later than this in
Euclid's order. How do we know, says the supposed critic, that the circle
does not cut AB in three or more points, in which case there would be not
one perpendicular but three or more? Proclus (pp.~186, 12—489, *>) tries to
refute this objection, and it is interesting to follow his argument, though it
will easily be seen to be inconclusive. He takes in order three possible
suppositions.

1. May not the circle meet AB in a third point K between the middle
point of GE and either extremity of it, taking the form drawn in the figure
appended ?

Suppose this possible. Bisect GE in H. Join CH, and produce it to
meet the circle in L. Join CG, CK, CE.

Then, since CG is equal to CE, and
CH is common, while the base GH is
equal to the base HE,

the angles CHG, CHE are equal and,
since they are adjacent, they are both right.

Again, since CG is equal to CE,
the angles at G and E are equal.

Lastly, since CK is equal to CG and
also to CE, the angles CGK, CKG are
equal, as also are the angles CKE, CEK.

Since the angles CGK, CEK are equal, it follows that

the angles CKG, CKE are equal and therefore both right.

Therefore the angle CKH'\% equal to the angle CHK,
and CH is equal to CK.

But CK is equal to CL, by the definition of the circle; therefore CH is
equal to CL: which is impossible.

Thus Proclus; but why should not the circle meet AB in H as well as .X?

2. May not the circle meet AB in // the middle point of GE and take
the form shown in the second figure?

In that case, says Proclus, join CG, CH, CE as before. Then bisect ME
at K, join CK and produce it to meet
the circumference at L.

Now, since HK is equal to KE, CK
is common, and the base CH is equal to
the base CE,

the angles at K are equal and therefore
both right angles.

Therefore the angle CHK is equal to
the angle CKH, whence CK is equal to CH
and therefore to CL: which is impossible.

So Proclus; but why should not the circle meet AB in A* as well as Hf

3, May not the circle meet AB in two points besides G, E and pass,
between those two points, to the side of A3 towards C, as in the next figure ?

Here again, by the same method, Proclus proves that, K, L being the
other two points in which the circle cuts
AB,

CK is equal to CH,

and, since the circle cuts CH'm M,

CM is equal to CK and therefore to
CH: which is impossible.

But, again, why should the circle not
cut AB in the point H a well?

In fact, Proclus' cases are not mutually
exclusive, and his method of proof only enables us to show that, if the circle
meets AB in one more point besides G, E, it must meet it in more points
still. We can always find a new point of intersection by bisecting the distance
separating any two points of intersection, and so, applying the method ad
infinitum, we should have to conclude ultimately that the circle with radius
CH (or CG) coincides with AB. It would follow that a circle with centre
C and radius greater than CH would not meet AB at all. Also, since all
straight lines from C to points on AB would be equal in length, there would
be an infinite number of perpendiculars from C on AB.

Is this under any circumstances possible ? It is not possible in Euclidean
space, but it is possible, under the Riemann hypothesis (where a straight line
is a ``closed series ``and returns on itself), in the case where C is the pole of
the straight line AB.

It is natural therefore that, for a proof that in Euclidean space there is
only one perpendicular from a point to a straight line, we have to wait until
1. 16, the precise proposition which under the Riemann hypothesis is only valid
with a certain restriction and not universally. There is no difficulty involved
by waiting until 1. 16, since t. 12 is not used before that proposition is reached;
and we are only in the same position as when, in order to satisfy ourselves of
the number of possible solutions of 1. r, we have to wait till 1. 7.

But if we wish, after all, to prove the truth of the assumption without
recourse to any later proposition than 1. 12, we can do so by means of this
same invaluable 1. 7.

If the circle intersects AB as before in G, E, let H be the middle point of
GE, and suppose, if possible, that the
circle also intersects AB in any other point
K on AH.

From H, on the side of AB opposite to
C, draw HL at right angles to AB, and
make HL equal to HC.

Join CG, EG, CK, LK.

Now, in the triangles CHG, EHG,
CHis equal to /.//, and HG is common.

Also the angles CHG, EHG, being
both right, are equal.

Therefore the base CG is equal to the base EG.

Similarly we prove that CK is equal to LK.

But, by hypothesis, since K is on the circle,

CK is equal to CG.

Therefore CG, CK, LG, LK are all equal.

Now the next proposition, i. 13, will tell us that CH, HL are in a straight
line; but we will not assume this. Join CE,

Then on the same base CE and on the same side of it we have two pairs
of straight lines drawn from C, L to G and K such that CG is equal to CK
and EG to LK

But this is impossible [1. 7 \

Therefore the circle cannot cut BA or BA produced in any point other
than G on that side of CL on which G is.

Similarly it cannot cut AB or AB produced at any point other than E
on the other side of CL.

The only possibility le/t therefore is that the circle might cut AB in the
same point as that in which CL cuts it. But this is shown to be impossible
by an adaptation of the proof of \prop{1}{7},

For the assumption is that there may be some point M on CL such thai
CM is equal to CG and LM to LG.

If possible, let this be the case, and produce CG
to N.

Then, since CM is equal to CG,
the angle NGM is equal to the angle GML [1. 5, part 2].

Therefore the angle GML is greater than the angle
MGL.

Again, since LG k equal to LM,
the angle GML is equal to the angle MGL.

But it was also greater: which is impossible.

Hence the circle in the original figure cannot cut AB in the point in
which CL cuts it.

Therefore the circle cannot cut AB in any point whatever except G and E.

[This proof of course does not prove that CK is less than CG, but only
that it is not equal to it. The proposition that, of the obliques drawn
from C to AB, that is less the foot of which is nearer to H can only be proved
later. The proof by 1. 7 also fails, under the Riemann hypothesis, if C, L are
the poles of the straight line AB, since the broken lines CGL, CKL etc.
become equal straight lines, all perpendicular to AB.]

Proclus rightly adds (p.~z8q, 18 sqq.) that it is not mcessary to take D on
the side of AB away from A if an objector ``says that there is no space on
that side.'' If it is not desired to trespass on that side of AB, we can take D
anywhere on AB and describe the arc of a circle between D and the point
where it meets AB again, drawing the arc on the side of AB on which C is.
If it should happen that the selected point D is such that the circle only meets
AB in oik point (D itself), we have only to describe the circle with CD as
radius, then, if E be a point on this circle, take Fa point further from C than
E is, and describe with CF as radius the circular arc meeting AB in two
points.

Proposition 13.

If a straight line set up on a straight line make angles, it
wilt make either two right angles or angles equal to two right
angles.

For let any straight line AB set up on the straight line
s CD make the angles CBA, ABD j

I say that the angles CBA, ABD
are either two right angles or equal to
two right angles.

Now, if the angle CBA is equal to
10 the angle ABD,

they are two right angles. [Def. 10]

But, if not, let BE be drawn from the point B at right

angles to CD; [1. n]

therefore the angles CBE, EBD are two right angles.

is Then, since the angle CBE is equal to the two angles

CBA, ABE,

let the angle EBD be added to each;
therefore the angles CBE, EBD are equal to the three
angles CBA, ABE, EBD. [C X a]

M Again, since the angle DBA is equal to the two angles
DBE, EBA,

let the angle ABC be added to each;
therefore the angles DBA. ABC are equal to the three
angles DBE, EBA, ABC. [ax*]

ij But the angles CBE, EBD were also proved equal to
the same three angles;

and things which are equal to the same thing are also
equal to one another; [C. X 1]

therefore the angles CBE, EBD are also equal to the
3a angles DBA, ABC.

But the angles CBE, EBD are two right angles;

therefore the angles DBA, ABC are also equal to two

right angles.

Therefore etc.

Q. E, D,

17. let the angle EBD be added to each, literally ``let the angle EBD be added
(so as to be) common, ``*a*fy rpoandeBu % irrh EBA. Similarly tow)) dpjsrtfw is used of
subtracting a straight line or angle from each of two others. ``Let the common angle EBD
be added is clearly an inaccurate translation, for the angle is not common before it 13 added,
i.e.\ the muri is proleptie. ``Let the common angle be tuitraclid'' as a translation of rar
d<t?W?l<j8u would be less unsatisfactory, it is true, but, as it is desirable to use corresponding
words when translating the two expressions, it seems hopeless to attempt to keep the word
``common,'' and I have therefore said ``to each'' and ``from each. ``simply.

Proposition 14.

If with any straight line, and at a point on it, two straight
lines not lying on the same side make the adjacent angles equal
to two right angles, the two straight lines will be in a straight
line with one another.

5 For with any straight line AB, and at the point B on it,
let the two straight lines BC, BD not lying on the same side
make the adjacent angles ABC, ABD equal to two right
angles;

I say that BD is in a straight line with CB.

w For, if BD is not in a straight line
with BC, let BE be in a straight line
with CB.

Then, since the straight line AB 5 —
stands on the straight line CBE,

15 the angles ABC, ABE are equal to two right angles.

[< n]

But the angles ABC, ABD are also equal to two right angles;
therefore the angles CBA, ABE are equal to the angles
CBA, ABD. [Post 4 and C. N. t]

Let the angle CBA be subtracted from each;
20 therefore the remaining angle ABE is equal to the remaining
angle ABD, [C. N. 3]

the less to the greater: which is impossible.
Therefore BE is not in a straight line with CB.
Similarly we can prove that neither is any other straight
aj line except BD.

Therefore CB is in a straight line with BD.
Therefore etc.

Q. E. D.

i. If with any straight line... There is no greater difficulty in translating the works
of the Greek geometers than that of accurately giving the force of prepositions, *pbi, for
instance, is used in all sorts of expressions with various shades of meaning. The present
enunciation begins 'EAr Tpt Tttfi ciffcta rai tQ rpAi avrv eTtfttitp, and it is really necessary in
this one sentence to translate rp6i by three different words, wit A, at, and en. The first rp6t
must be translated by with because two straight lines ``make'' an angle with one another. On
the other hand, where [he similar expression rpot rp bofalfy tdcta occurs 1 in \prop{1}{33}, but it is
a question of M constructing ™ an angle (riPcrrffaffPat), we have to say ``to construct on a
given straight line.'' Againtt would perhaps be the English word coming nearest to
expressing all these meanings of irp£t, but it would be intolerable as a translation.

17. Todhunter points out tbat for the inference in this line Post. 4, that all right angles
are equal, is necessary as well as the Common Notion that things which are equal to the same
thing [or rather, here, to squat things) are equal. A similar remark applies to steps in the
proofs of \prop{1}{15} and \prop{1}{18}.

34. we can prove. The Greek expresses this by the future of the verb, *t%o/ur,
``we shall prove,'' which however would perhaps be misleading in English.

P rocki 5 observes (p.~297) that two straight lines on the same side of another
straight line and meeting it in one and the same
point may make with one and the same portion
of the straight line terminated at the point two
angles which are together equal to two right angles,
in which case however the two straight lines would
not be in a straight line with one another. And
he quotes from Porphyry a construction for two
such straight lines in the particular case where they
form with the given straight line angles equal
respectively to half a right angle and one and a
half right angles. There is no particular value in
the construction, which will be gathered from the annexed figure where CB,
CF are drawn at the prescribed inclinations to CD.

Proposition 15.

If two straight lines cut one another, they make the vertical
angles equal to one another.

For let the straight lines AB, CD cut one another at the
point E;
5 I say that the angle AEC is equal to i
the angle DEB, _\I_

and the angle CEB to the angle D
AED.

For, since the straight line AE stands
10 on the straight line CD, making the angles CEA, AED,
the angles CEA, AED are equal to two right angles

L'- '3]

Again, since the straight line DE stands on the straight
line AB, making the angles AED, DEB,

the angles AED, DEB are equal to two right angles.

[« '3]

«s But the angles CEA, A ED were also proved equal to
two right angles;

therefore the angles CEA, A ED are equal to the

angles AED DEE. [Post 4 and C. If. 1]

Let the angle AED be subtracted from each;

*> therefore the remaining angle CEA is equal to the

remaining angle BED. [C. If. 3]

Similarly it can be proved that the angles CEB, DEA

are also equal.

Therefore etc.\ Q. E, D.

*5 [Porism. From this it is manifest that, if two straight
lines cut one another, they will make the angles at the point
of section equal to four right angles.]

1. the vertical angles. The difference between adjacent angles (ai ifeffit fut/ku) and
vertical fugles (td Kara xopv<pfyv yuvttu) is t hi lis explained by Proclus (p.~398, 14—34). The
first term describes the angles made by two straight lines when one only it divided by the
other, i.e.\ when one straight Line meets another at a point which is not either of its extremi-
ties, but is not itself produced beyond the point of meeting. When the first straight line is
produced, so that the lines cross at the point, they make two pairs of vertical angles (which
are more clearly described as vertically opposite angles), and which are so called because their
convergence is from opposite directions to one point (the intersection of the lines) as vertex

16. at the point of section, literally ``at the section,'' rpii tjj 7-05,

This theorem, according to Eudemus, was first discovered by Thales, but
found its scientific demonstration in Euclid (Proc)us, p.~2991, 3 — 6).

Proclus gives a converse theorem which may be stated thus. If a straight
line is met at one and the same point intermediate in its length by two other
straight lines on different sides of it and such as to make the vertical angles
equal, the latter straight lines are in a straight line with one another. The
proof need not be given, since it is almost self-evident, whether (1) it is direct,
by means of 1. 13, 14, or (2) indirect, by reductio ad abmrdum depending
on 1. 15.

The balance of ms. authority seems to be against the genuineness of this
Porism, but Proclus and Psellus both have it. The word is not here used, as it
is in the title of Euclid's lost Potisms, to signify a particular class of independent
propositions which Proclus describes as being in some sort intermediate between
theorems and problems (requiring us, not to bring a thing into existence, but
to find something which we know to exist). Porism has here (and wherever
the term is used in the Elements) its second meaning; it is what we call a
corollary, i.e.\ an incidental result springing from the proof of a theorem or the
solution of a problem, a result not directly sought but appearing as it were by
chance without any additional labour, and constituting, as Proclus says, a sort
of windfall (ippaiov) and bonus (ntpSos). These Porisms appear in both the
geometrical and arithmetical Books of the Elements, and may either result
from theorems or problems. Here the Porism is geometrical, and springs out
of a theorem; vit. 1 affords an instance of an arithmetical Porism. As an
instance of a Porism to a problem Proclus cites ``that which is found in the
second Book'' (ri b t Stvtipif jStAup mimfov); but as to this see notes on
11. 4 and iv. 15.

The present Porism, says Proclus, formed the basis of ``that paradoxical
theorem which proves that only the following three (regular) polygons can fill
up the whole space surrounding one point, the equilateral triangle, the square,
and the equilateral and equiangular hexagon.'' We can in fact place round a
point in this manner six equilateral triangles, three regular hexagons, or four
squares. ``But only the angles of these regular figures, to the number specified,
can make up four right angles: a theorem due to the Pythagoreans.''

Proclus further adds that it results from the Porism that, if any number of
straight lines intersect one another at one point, the sum of all the angles so
formed will still be equal to four right angles. This is of course what is
generally given in the text-boots as Corollary 1.

Proposition 16,

In any triangle, if one of the sides be produced, the exterior
angle is greater than either of the interior and opposite angles.
Let ABC be a triangle, and let one side of it BC be
produced to D;
l I say that the exterior angle ACD is greater than either
of the interior and opposite angles
CBA, BAC.

Let AC be bisected at E [t 10],
and let BE be joined and produced
10 in a straight line to F;

let EFbe made equal to BE[t. 3],
let EC be joined [Post. i],and let AC
be drawn through to G [Post. a].
Then, since AE is equal to EC,
15 and BE to EF,

the two sides AE, EB are equal to the two sides CE,
EF respectively;

and the angle AEB is equal to the angle FEC,

for they are vertical angles. [1. 15]

» Therefore the base AB is equal to the base FC,

and the triangle ABE is equal to the triangle CFE,

and the remaining angles are equal to the remaining angles

respectively, namely those which the equal sides subtend; [1. 4]

therefore the angle BAE is equal to the angle ECF.

i But the angle BCD is greater than the angle ECF;

[C.N. 5]

therefore the angle A CD is greater than the angle BAB.

Similarly also, if BC be bisected, the angle BCG, that is,

the angle ACD [i. is], can be proved greater than the angle

ABC as well.

Therefore etc.\ Q. E. D.

i. the exterior angle, literally ``the outside angle,'' <\ l«rdt yurla.
i. the interior and opposite angles, rut inrit Kai iritarrior yurir.
11. let AC be drawn through to G. The word is 5iiJx* u p a variation on Ihe more
usual 4K$ffi\-fiadw. ``let it be praduHd. ``
at. CFE, in the text ``F£C,''

As is well known, this proposition is not universally true under the
Riemann hypothesis of a space endless in extent but not infinite in size. On
this hypothesis a straight line is a ``closed series'' and returns on itself; and
two straight lines which have one point of intersection have another point of
intersection also, which bisects the whole length of the straight line measured
from the first point on it to the same point again; thus the axiom of Euclidean
geometry that two straight lines do not enclose a space does not hold. If 4 A
denotes the finite length of a straight line measured from any point once
round to the same point again, 2A is the distance between the two intersections
of two straight lines which meet. Two points A, B do not determine one
sole straight line unless the distance between them is different from 2 A. In
order that there may only be one perpendicular from a point C to a straight
line AS, C must not be one of the two ``poles ``of the straight line.

Now, in order that the proof of the present proposition may be universally
valid, it is necessary that CF should always fall within the angle ACD so that,
the angle A CF may be less than the angle ACD. But this will not always be
so on the Riemann hypothesis. For, (1) if BE is equal to A, so that BF is
equal to 2 A, .Fwill be the second point in which BE and BD intersect; Le.
F will lie on CD, and the angle ACF will be equal to the angle ACD. In
this case the exterior angle ACD will be equal to the interior angle BAC.
(2) If BE is greater than A and less than 2 A, so that BF is greater than a A
and less than 4A, the angle ACF will be greater than the angle ACD, and
therefore the angle A CD will be less than the interior angle BAC. Thus, e.g.,
in the particular case of a right-angled triangle, the angles other than the right
angle may be (1) both acute, (2) one acute and one obtuse, or (3) both obtuse
according as the perpendicular sides are (1) both less than A, (2) one less and
the other greater than A, (3) both greater than A.

Proclus tells us (p.~307, 1 — 12) that some combined this theorem with the
next in one enunciation thus: In any triangle, if one side be produced, the
exterior angle of the triangle is greater than either of the inierior and opposite
angles, and any two of the interior angles are less than two right angles, the
combination having been suggested by the similar enunciation of Euclid 1. 32,
In any triangle, if one of the suits be produced, the exterior angle is equal to the
two inierior and opposite angles, and the three interior angles of the triangle are
equal to two right angles.

The present proposition enables Proclus to prove what he did not succeed
in establishing conclusively in his note on 1. 12, namely that from one point-
there cannot be drawn to the same straight line three straight lines equal in length.

For, if possible, let AB, AC, AD be all equal, B, C, D being in a
straight line.

Then, since AB, AC are equal, the angles

ABC, ACB are equal.
Similarly, since AB, AD are equal, the angles

ABD, ADB are equal.
Therefore the angle ACB is equal to the angle

ADC, i.e.\ the exterior angle to the interior and
opposite angle: which is impossible.

Proclus next {p.~308, 14 sqq.) undertakes to prove by means of 1. 16 that,
if a straight line falling on two straight lines make the exterior angle equal to
the interior and opposite angle, the two straight lines will not form a triangle or
meet, for in that case the same angle would be both greater and equal.

The proof is really equivalent to that of Eucl. 1. 17, If BE falls on the
two straight lines AB, CD in such a way that the angle
CDE is equal to the interior and opposite angle ABD,
AB and CD cannot form a triangle or meet. For, if
they did, then (by 1. 16) the angle CDE would be
greater than the angle ABD, while by the hypothesis
it is at the same time equal to it.

Hence, says Proclus, in order that BA, DC may
form a triangle it is necessary for them to approaeh one
another in the sense of being turned round one pair of
corresponding extremities, e.g.\ B, D, so that the other extremities A, C come
nearer. This may be brought about in one of three ways: (t) AB may
remain fixed and CD be turned about D so that the angle CDE increases;
(2) CD may remain fixed and AB be turned about B so that the angle ABD
b?r jmes smaller; (3) both AB and CD may move so as to make the angle
ABD smaller and the angle CDE larger at the same time. The reason, then,
of the straight lines AB, CD coming to form a triangle or to meet is (says
Proclus) the movement of the straight lines.

Though he does not mention it here, Proclus does in another passage
(p.~371, 2 — ro, quoted on p.~207 above) hint at the possibility that, while 1. 16
may remain universally true, either of the straight lines BA, DC (or both
together) may be turned through any angle not greater than a certain finite
angle and yet may not meet (the Bolyai-Lobachewsky hypothesis).

Proposition 17.

In any triangle two angles taken together in any manner
are less than two right angles.
Let ABC be a triangle;

I say that two angles of the triangle ABC taken together in

any manner are less than two right angles.

For let BC be produced to D. [Post. 2]

Then, since the angle ACD is an exterior angle of the

triangle ABC,

it is greater than the interior and opposite angle ABC.

Let the angle ACB be added to each;
therefore the angles A CD, ACB are greater than the angles
ABC, BCA.

But the angles A CD, ACB are equal to two right angles.

['; '3]
Therefore the angles ABC, BCA are less than two right
angles.

Similarly we can prove that the angles BAC, ACB are
also less than two right angles, and so are the angles CAB,
ABC as well.
Therefore etc.

Q. E. D.

1. taken together in any manner, rirrg pcraXop0artyui'eu, Le. an; pail added
together-

As in his note on the previous proposition, Proclus tries to state the cause
of the property. He takes the case of two straight lines forming right angles
with a transversal and observes that it is the convergence of the straight lines
towards one another (<n!™«rn tv tvBuuf), the lessening of the two right angles,
which produces the triangle. He will not have it that the fact of the exterior
angle being greater than the interior and opposite angle is the cause of the
property, for the odd reason that ``it is not necessary that a side should be
produced, or that there should be any exterior angle constructed. ..and how can
what is not necessary be the cause of what is necessary ?'' (p.~311, 17 — 21).

Agreeably to this view, Proclus then sets himself to prove the theorem
without producing a side of the triangle.

Let ABC be a triangle. Take any point D on
BC, and join AD.

Then the exterior angle A DC of the triangle ABD
is greater than the interior and opposite angle ABD.

Similarly the exterior angle ADB of the triangle
ADC is greater than the interior and opposite angle
A CD.

Therefore, by addition, the angles ADB, ADC are together greater than
the angles ABC, ACB.

But the angles ADB, ADC are equal to two right angles; therefore the
angles ABC, ACB are less than two right angles.

Lastly, Proclus proves (what is obvious from this proposition) that there
cannot be more than one perpendicular to a straight line from a point without
it. For, if this were possible, two of such perpendiculars would form a triangle
in which two angles would be right angles: which is impossible, since any two
angles of a triangle are together less than two right angles.

Proposition 18.
In any triangle the greater side subtends the greater angle.

For let ABC be a triangle having the side AC greater
than AB;

I say that the angle ABC is also greater than the angle
BCA.

For, since AC is greater than AB, let AD be made equal
to AB [i. 3], and let BD be joined.

Then, since the angle ADB
is an exterior angle of the triangle
BCD,

it is greater than the interior
and opposite angle DCB. [1. 16]

But the angle ADB is equal
to the angle ABD,

since the side AB is equal to AD;

therefore the angle ABD is also greater than the angle
ACB\

therefore the angle ABC is much greater than the angle
ACB.

Therefore etc.

Q. E. D.

In the enunciation of this proposition wc have inrarttvay (*'' subtend ``} used with the
simple accusative instead of the more usual inti with accusative. The latter construction
is used in the enunciation of I. ig» which otherwise only diners from that of [• 18 in the order
of the words. The point to remember in order to distinguish the two is that the datum
comes first and the fuanititm second, the datum being in this proposition the greater sidt
and in the next the greater angle. Thns the enunciations are (i. 18) i-a»n6i i-ftydtov j) pclpgr
J-XrupA tjj* fitlfrtrvL ywflsw uwordvti and (l. 19) rarrbt rpvywou vxb rV pd{ot>a -yuvLar Jj
licifuw xXcupi. iinoTtltti. In order to keep the proper order in English we must use the
passive of the verb in \prop{1}{19}. Aristotle quotes the result of 1. to, using the exact wording,
vk fkp t)jv ttflfw ywrtay bwoTttt/ci {Mtteorologica tit. 5, 376 a 11).

``In order to assist the student in remembering which of these two
propositions [1. 18, 19] is demonstrated directly and
which indirectly, it may be observed that the order is
similar to that in \prop{1}{5} and 1. 6'' (Todhunter).

An alternative proof of L 18 given by Porpnyry
(see Ptoclus, pp.~315, \\ — 316, 13} is interesting. It
starts by supposing a length equal to AB cut off from
the other end of AC; that is, CD and not AD is
made equal to AB.

Produce AB to E so that BE is equal to AD, and
join EC.

Then, since AB is equal to CD, and BE to AD,
AE is equal to AC,

Therefore the angle A EC is equal to the angle ACE.

Now the angle ABC is greater than the angle A EC, [t. 16]

and therefore greater than the angle ACE.
Hence, a fortiori, the angle ABC is greater than the angle ACB.

Proposition 19.

In any triangle the greater angle is subtended by the
greater side.

Let ABC be a triangle having the angle ABC greater
than the angle BCA;

I say that the side AC is also greater than the side AB.

For, if not, AC is either equal to AB or less.

Now AC is not equal to AB;
for then the angle ABC would also have been
equal to the angle A CB; [u 5]

but it is not;

therefore AC is not equal to AB.

Neither is AC less than AB,
for then the angle ABC would also have been less than the
angle ACB; [t 18]

but it is not;

therefore AC is not less than AB.

And it was proved that it is not equal either.
Therefore AC is greater than AB,

Therefore etc.\ q. e. d.

This proposition, like t. 6, can be proved by merely logical deduction from
1. 5 and 1. 18 taken together, as pointed out by De Morgan. The general
form of the argument used by De Morgan is given in his Formal Logic (1847),
p.~25, thus:

``Hypothesis. Let there be any number of propositions or assertions —
three for instance, X, Y and Z — of which it is the property that one or the
other must be true, and one only. Let there be three other propositions
P, Q and P of which it is also the property that one, and one only, must be
true. Let it be a connexion of those assertions that:

when X is true, P is true,

when Kis true, Q is true,

when Z is true, R is true.
Constquenu: then it follows that,

when P is true, X is true,

when Q is true, Y is true,

when JR is true, Z is true''

To apply this to the case before us, let us denote the sides of the triangle
ABC by a, 6, c, and the angles opposite to these sides by A, B, C respectively,
and suppose that a is the base.

Then we have the three propositions,

when b is equal to c, B is equal to C,
when b is greater than c, B is greater than C, 1
when 6 is less than c, B is less than C, f

and it follows logically that,

when B is equal to C, b is equal to c,

when B is greater than C, b is greater than c, \

when i? is less than C, b is less than c. J

Reductio ad absurdum by exhaustion.

Here, says Proclus (p.~318, 16 — 33), Euclid proves the impossibility ``by
means of division'' (« fkaipurtan). This means simply the separation of
different hypotheses, each of which is inconsistent with the truth of the
theorem to be proved, and which therefore must be successively shown to be
impossible. If a straight line is not greater than a straight tine, it must be
either equal to it or less; thus in a reductio ad absurdum intended to prove
such a theorem as 1. 19 it is necessary to dispose successively of two hypotheses
inconsistent with the truth of the theorem.

Alternative (direct) proof.

Proclus gives a direct proof (pp.~319—321) which an-NairizI also has and
attributes to Heron. It requires a lemma and is consequently open to the
slight objection of separating a theorem from its converse. But the lemma
and proof are worth giving.

Lemma.

If an angle of a triangle be bisected and the straight line bisecting it meet the
base and divide it into unequal parts, the sides containing the angle will be
unequal, and the greater will be that which meets the greater segment of the base,
and the less that which meets the lest.

Let AD, the bisector of the angle A of the triangle ABC, meet BC in D,
making CD greater than BD.

I say that AC is greater than AB.
Produce AD to £ so that DE is equal to
AD. And, since DC is greater than BD, cut
off DF equal to BD.

Join BFsmd produce it to G.
Then, since the two sides AD, DB are
equal to the two sides ED, DF, and the
vertical angles at D are equal,

AB is equal to EF,
and the angle DEF to the angle BAD,

i.e.\ to the angle DAG (by hypothesis).
Therefore AG is equal to EG,

and therefore greater than EF, or AB,
Hence, a fortiori, AC is greater than AB.

Proof of \prop{1}{19}.

Let ABC be a triangle in which the angle ABC is greater than the angle
ACB. *

Bisect BC at D, join AD, and produce it to B so that DE is equal to
Z>. Join BE.

Then the iwo sides BD t DE are equal to the two
sides CD, DA, and the vertical angles at D are equal;

therefore BE is equal to AC,

and the angle DBE to the angle at C.

But the angle at C is less than the angle ABC;

therefore the angle DBE is less than the angle
ABD.

Hence, if BF bisect the angle ABE, BF meets
AE between A and D. Therefore EF is greater
than FA.

It follows, by the lemma, that BE is greater than
BA,

that is, AC is greater than >4.#.

Proposition 20.

/« a«y triangle twu sides taken together in any manner
are greater than the remaining one.

For let ABC be a triangle;
I say that in the triangle ABC two sides taken together in
any manner are greater than the remaining one, namely
BA, AC greater than BC,
AB, BC greater than A C,
BC, CA greater than AB.
For let BA be drawn through to the point D s
let DA be made equal to CA, and let DC be
joined.

Then, since DA is equal to AC,

the angle ADC is also equal to the angle

ACD; [1. 5 ]

therefore the angle BCD is greater than

the angle ADC. [C. J\prop{5}{5}]

And, since DCB is a triangle having the angle BCD
greater than the angle BDC,

and the greater angle is subtended by the greater side,

[l 19]
therefore DB is greater than BC.

ButZM is equal to AC;

therefore BA, AC are greater than BC.
Similarly we can prove that AB, BC are also greater
than CA, and BC, CA than AB.
Therefore etc.

Q. E. D,

It was the habit of the Epicureans, says Proclus (p.~322), to ridicule this
theorem as being evident even to an ass and requiring no proof, and their
allegation that the theorem was ``known'' (yywpijiov) even to an ass was based
on the fact that, if fodder is placed at one angular point and the ass at another,
he does not, in order to get to his food, traverse the two sides of the triangle
but only the one side separating them (an argument which makes Savile exclaim
that its authors were ``digni ipsi, qui cum Asino foenum essent,'' p.~78).
Proclus replies truly that a mere perception of the truth of the theorem is a
different thing from a scientific proof of it and a knowledge of the reason why
it is true. Moreover, as Sim son says, the number of axioms should not be
increased without necessity.

Alternative Proofs.

Heron and Porphyry, we are told (Proclus, pp.~323 — 6), proved this
theorem in different ways as follows, without producing one of the sides.

First proof.

Let ABC be the triangle, and let it be required to prove that the sides
BA, AC are greater than BC.

Bisect the angle BAC by AD meeting BC inD.

Then, in the triangle ABD,

the exterior angle ADC is greater than the
interior and opposite angle BAD, [1. 16]

that is, greater than the angle DAC.

Therefore the side AC is greater than the side
CD, [1. 19]

Similarly we can prove that AB is greater than BD.

Hence, by addition, BA, AC are greater than BC.

Second proof.

This, like the first proof, is direct. There are several cases to be considered.

(1) If the triangle is equilateral, the truth of the proposition is obvious.

(3) If the triangle is isosceles, the proposition needs no proof in the case
(a) where each of the equal sides is greater than the base.

(#) If the base is greater than either of the other sides, we have to prove
that the sum of the two equal sides is greater than
the base. Let BC be the base in such a triangle.

Cut off from BC a length BD equal to AB, and
join AD.

Then, in the triangle ADB, the exterior angle
ADC is greater than the interior and opposite angle
BAD. [1. 1 61

Similarly, in the triangle ADC, the exterior angle ADB is greater than the
interior and opposite angle CAD.

*88 BOOK I [i. to

By addition, the tyro angles BDA, ADC are together greater than the
two angles BA D, DA C (or the whole angle BA C).

Subtracting the equal angles BDA, BAD, we have the angle ADC
greater than the angle CAD.

It follows that AC is greater than CD; [i. 19]

and, adding the equals AB, BD respectively, we have BA, AC together
greater than BC.

(3) If the triangle be scalene, we can arrange the sides in order of length.
Suppose BC is the greatest, AB the intermediate and AC the least side.
Then it is obvious that AB, BC are together greater than AC, and BC, CA
together greater than AB.

It only remains therefore to prove that CA, AB are together greater
than BC.

We cut off from BC a length BD equal to the adjacent side, join AD, and
proceed exactly as in the above case of the isosceles triangle.

Thirdprwf.

This proof is by reduetio ad ahsurdum.

Suppose that BC is the greatest side and, as before, we have to prove that
BA, AC are greater than BC.

If they are not, they must be either equal to A

or less than BC.

(1) Suppose BA, AC ire together equal
to BC.

From BC cut off BD equal to BA, and
join AD.

It follows from the hypothesis that DC is equal to AC.

Then, since BA is equal to BD,
the angle BDA is equal to the angle BAD.

Similarly, since AC is equal to CD,
the angle CDA is equal to the angle CAD.

By addition, the angles BDA, ADC are together equal to the whole angle
BAC.

That is, the angle BAC is equal to two right angles: which is impossible.

{2) Suppose BA, AC ate together less than BC.

From BC cut off BD equal to BA, and from CB cut off CE equal to
CA. Join AD, AE.

In this case, we prove in the same way that
the angle BDA is equal to the angle BAD, and
the angle CEA to the angle CAE.

By addition, the sum of the angles BDA,
AEC is equal to the sum of the angles BAD,
CAE.

Now, by 1. 16, the angle BDA is greater than the angle DAC, and
therefore, a fortiori, greater than the angle EA C.

Similarly the angle AEC is greater than the angle BAD.

Hence the sum of the angles BDA, AEC is greater than the sum of the
angles BAD, EAC.

But the former sum was also equal to the latter: which is impossible,

Proposition 21.

If on one of the sides of a triangle, from its extremities,

there be constructed two straight lines meeting within the

triangle, the straight lines so constructed will be less than the

remaining two sides of the triangle, but will contain a greater

s angle.

On BC, one of the sides of the triangle ABC, from its
extremities B, C, let the two straight lines BD, DC be con-
structed meeting within the triangle;

I say that BD, DC are less than the remaining two sides
10 of the triangle BA, AC, but contain an angle BDC greater
than the angle BAC.

For let BD be drawn through to E.

Then, since in any triangle two

sides are greater than the remaining

15 one, [1. zo]

therefore, in the triangle ABE, the

two sides A B, AE are greater than BE.

Let EC be added to each;

therefore BA, AC art greater than BE, EC.
20 Again, since, in the triangle CED,

the two sides CE, ED are greater than CD,
let DB be added to each;

therefore CE, EB are greater than CD, DB.
But BA, AC were proved greater than BE, EC;
2$ therefore BA, AC are much greater than BD, DC.

Again, since in any triangle the exterior angle is greater
than the interior and opposite angle, [1- 16]

therefore, in the triangle CDE,

the exterior angle BDC is greater than the angle CED.
30 For the same reason, moreover, in the triangle ABE also,
the exterior angle CEB is greater than the angle BAC.
But the angle BDC was proved greater than the angle CEB;
therefore the angle BDC is much greater than the angle
BAC.
3 5 Therefore etc.\ q. e. d,

1* be con strutted. ..meeting within the triangle. The word n meeting'' is not in
the Greek, where the words are irrht rwmrtrty. evrLrrwiBtu is the word used of con-
structing two straight lines to a point (cf- [ 7) or so as to form a triangle; but it is necessary
in English to indicate that they mat.

3. the straight lines so constructed. Observe the elegant brevity of the Creek al

The editors generally call attention to the fact that the lines drawn within
the triangle in this proposition must be drawn,
as the enunciation says, from the ends of the
side; otherwise it is not necessary that their
sum should be less than that of the remaining
sides of the triangle. Proclus (p.~3*7, II sqq.)
gives a simple illustration.

Let ABC be a right-angled triangle. Take
any point D on BC, join DA, and cut off
from it DE equal to AB. Bisect AE at F,
and join FC.

Then shall CF, FD be together greater than CA, AB.
For CF, FE are equal to CF, FA,
and therefore greater than CA.

Add the equals ED, AB respectively;

therefore CF, FD are together greater than CA, AB.

Pappus gives the same proposition as that just proved, but follows it up
by a number of others more elaborate in character, selected apparently from
``the so-called paradoxes ``of one Erycinus (Pappus, m. p.~106 sqq.). Thus
he proves the following:

1. In any triangle, except an equilateral triangle or an isosceles triangle
with base less than one of the other sides, it is possible to construct on the
base and within the triangle two straight lines the sum of which is equal to
the sum of the other two sides of the triangle.

2. In any triangle in which it is possible to construct two straight lines on
the base which are equal to the sum of the other two sides of the triangle it is
also possible to construct two others the sum of which is greater than that sum.

3. Under the same conditions, if the base is greater than either of the
other two sides, two straight tines can be constructed in the manner described
which are respectively greater than the other two sides of the triangle; and the
lines may be constructed so as to be respectively equal to the two sides, if one
of those two sides is less than the other and each of them less than the base.

4. The lines may be so constructed that their sum will bear to the sum
of the two sides of the triangle any ratio less than 2:1.

As a specimen of the proofs we will give that of the proposition which has

been numbered (1) for the case where the triangle is isosceles (Pappus, in.
pp.~108 — 110)1

Let ABC be an isosceles triangle in which the base AC is greater than
either of the equal sides .<4.5, BC.

With centre v4 and radius AB describe a circle meeting j4Cin D.

Draw any radius AEFsuch that it meets BC in a point F outside the circle.

Take any point G on EF, and through it draw GZf parallel to AC. Take
any point .AT on GH, and draw KL parallel to FA meeting AC in L.

From jSCcut off BN equal to EG.

Thus AG, or LK, is equal to the sum of AB, BN, and CWis less than LK.

Now GF, Fffare together greater than GH,
and CH, UK together greater than CK,

Therefore, by addition,
CF, FG, HK are together greater than CK, HG.

Subtracting HK from each side, we see that
CF, FG are together greater than CK, KG;
therefore, if we add AG to each,

AF, FCaie together greater than AG, GK, KC.

And AB, BC are together greater than AF, EC. [1. 31]

Therefore AB, BC are together greater than A G, GK, KC.

But, by construction, AB, BN are together equal to AG;
therefore, by subtraction, NC is greater than GK, KC,
and a fortiori greater than KC.

Take on KC produced a point .A/' such that KM is equal to NC;
with centre K and radius KM describe a circle meeting CL in 0, and join KO.

Then shall LK, KO be equal to AB, BC.

For, by construction, LK is equal to the sum of AB, BN, and KO is
equal to NC;

therefore LK, KO are together equal to AB, BC.

It is after 1. at that (as remarked by De Morgan) the important
proposition about the perpendicular and obliques drawn from a point to a
straight line of unlimited length is best introduced:

Of all straight lines that can be drawn to a given straight line of unlimited
length from a given point without it:

(a) the perpendicular is the shortest;

(b) of the obliques, that is the greater the fool of which is further from the
perpendicular;

(e) given one oblique, only one other can be found of the same length, namely
that the foot of which is equally distant with the foot of the given one from the
perpendicular, but on the other side of it.

Let A be the given point, BC the given straight line; let AD be
the perpendicular from A on BC,
and AE, AF any two obliques of
which AF makes the greater angle
with AD.

Produce AD to A', making A'D
equal to AD, and join A'E, A F.

Then the triangles ADE, A'DE
are equal in all respects; and so are
the triangles ADF, A'DF.

Now (1) in the triangle AEA' the
two sides AE, EA' are-greater than AA' [1. 20I, that is, twice AE is greater
than twice AD.

Therefore AE is greater than AD.
(a) Since AE, A'E are drawn to E, a point within the triangle A FA',
AE, EA' are together greater than AE, EA\ [l *i]

or twice AE is greater than twice AE..
Therefore AE is greater than AE.

(3) Along DB measure off DG equal to DF, and join AG.
The triangles AGD, AFD are then equal in all respects, so that the
angles GAD, FAD are equal, and AG is equal to AF.

Proposition 22.

Out of three straight lines, which are equal to three given
straight lines, to construct a triangle: thus it is necessary that
two of the straight lines taken together in any manner should
be greater than the remaining one. [1. *o]

Let the three given straight lines be A, B, C, and of these
let two taken together in any manner be greater than the
remaining one,

namely A, B greater than C,

A, C greater than B,
and B, C greater than A;

thus it is required to construct a triangle out of straight lines
equal to A, B, C.

a-

8-

c-

Let there be set out a straight line DB, terminated at D
but of infinite length in the direction of B,
and let DF be made equal to A, FG equal to B, and GH
equal to C. [1. 3]

With centre F and distance FD let the circle DKL be
described;

again, with centre G and distance GH let the circle KLH be
described;
and let KF, KG be joined;

I say that the triangle KFG has been constructed out of
three straight lines, equal to A, B, C.

For, since the point F is the centre of the circle DKS,,

FD is equal to FK.

But FD is equal to A;

therefore KF is also equal to A.

Again, since the point G is the centre of the circle LKH,

GH is equal to GK.

But GH is equal to C;

therefore KG is also equal to C.

And FG is also equal to B;

therefore the three straight lines KF, FG, GK are equal to

the three straight lines A, B, C.

Therefore out of the three straight lines KF, FG, GK,

which are equal to the three given straight lines A, B, C, the

triangle KFG has been constructed.

6 q. e. F.

i — *> This is the first cast in the Elements of a Satparfiii to a problem in the sense of a
statement of the conditions or limits of the possibility of a solution. The criterion is of
course supplied by the preceding proposition.

j. thu» It is necessary. This is usually translated (e.g.\ by Williamson and Simson)
``But it is necessary,'' which is however inaccurate, since the Greek is not Sti W but ri ti).
The words are the same as those used to introduce the ttopurpit in the other sense of the
``definition ``or ``particular statement ``of a construction to be effected. Hence, as in the
latter case we say ``thus it is required ``(e.g.\ to bisect the finite straight line AS, \prop{1}{10}}, we
should here translate ``thus it is necessary.

4* To this enunciation alt the M5S. and Bocthius add, after the Sioptefifa, the words
``because in any triangle two sides taken together in any manner are greater than the
remaining one.'' But this explanation has the appearance of a gloss, and it is omitted hy
Proclus and Campanus. Moreover there is no corresponding addition to the Supwitit
of vi, 18.

It was early observed that Euclid assumes, without giving any reason, that
the circles drawn as described will meet if the condition that any two of the
straight lines A, S, C are together greater than the third be fulfilled. Proclus
(p.~33 r, S sqq.) argues the matter by means of redudio ad abrurdum, but
does not exhaust the possible hypotheses inconsistent with the contention.
He says the circles must do one of three things, (1) cut one another, (1) touch
one another, (3} stand apart {«rrarai) from one another. He then considers
the hypotheses (a) of their touching externally, (t>) of their being separated
from one another by a space. He should have considered also the hypothesis
(r) of one circle touching the other internally or lying entirely within the
other without touching. These three hypotheses being successively disproved,
it follows that the circles must meet (this is the line taken by Camerer and
Todhunter).

Simson says: ``Some authors blame Euclid because he does not
demonstrate that the two circles made use of in the construction of this
problem must cut one another: but this is very plain from the determination
he has given, namely, that any two of the straight lines DF, FG, GH must
be greater than the third, For who is so dull, though only beginning to
learn the Elements, as not to perceive that the circle described from the
centre F, at the distance FD, must meet FH betwixt F and H, because FD
is less than FH; and that, for the like reason, the circle described from the
centre G at the distance GH must meet DG betwixt D and G; and that
these circles must meet one another, because FD and GH are together
greater than FG.''

We have in fact only to satisfy ourselves that one of the circles, e.g.\ that
with centre G, has at least one point of its circumference outside the other
circle and also at least one point of its circumference inside the same circle;
and this is best shown with reference to the points in which the first circle
cuts the straight line DE. For (i) FH, being equal to the sum of B and C,
is greater than A t i.e.\ than the radius of the circle with centre F, and therefore
His outside that circle. (2) If GAf be measured along GF equal to GH
or C, then, since GM is either (a) less or {p} greater than GF, Jfwill fall
either {a) between G and F or (6) beyond F towards D; in the first case
(a) the sum of FM and C is equal to FG and therefore less than the sum
of A and C, so that FM is less than A or FD; in the second case {/>) the
sum of MF and FG, i.e.\ the sum of MF and B, is equal to GAf or C, and
therefore less than the sum of  and B, so that MF is less than A or FD;
hence in either case M falls within the circle with centre F.

It being now proved that the circumference of the circle with centre G
has at least one point outside, and at least one point inside, the circle with
centre F, we have only to invoke the Principle of Continuity, as we have to
do in 1. 1 (cf.\ the note on that proposition, p.~242, where the necessary
postulate is stated in the form suggested by Killing).

That the construction of the proposition gives only two points of
intersection between the circles, and therefore only two triangles satisfying
the condition, one on each side of FG, is clear from \prop{1}{7}, which, as before
pointed out, takes the place, in Book 1., of 111. 10 proving that two circles
cannot intersect in more points than two.

Proposition 23.

On a given straight line and at a point on it to construct a
rectilineal angle equal to a given rectilineal angle.

Let AB be the ``given straight line, A the point on it, and
the angle DCE the given rectilineal angle;

thus it is required to construct on the given straight line
AB, and at the point A on it, a rectilineal angle equal to the
given rectilineal angle DCE.

On the straight lines CD, CE respectively let the points
D, E be taken at random;
let DE be joined,
and out of three straight lines which are equal to the three
straight lines CD, DE, CE let the triangle AFG he. con-
structed in such a way that CD is equal to AF, CE to AG,
and further DE to FG. [1. 22]

Then, since the two sides DC, CE are equal to the two
sides FA., AG respectively,

and the base DE is equal to the base FG,

the angle DCE is equal to the angle FAG. [1. 8]

Therefore on the given straight line A3, and at the point
A on it, the rectilinealangle FAG has been constructed equal
to the given rectilineal angle DCE.

e Q. E. F.

This problem was, according to Eudemus (see Proclus, p.~333, 5), ``rather
the discovery of Oenopides,'' from which we must apparently infer, not that
Oenopides was the first to find any solution of it, but that it was he who dis-
covered the particular solution given by Euclid. (cf.\ Bretschneider, p.~65.)

The editors do not seem to have noticed the fact that the construction of
the triangle assumed in this proposition is not exactly the construction given
in 1. 22. We have here to construct a triangle on a certain finite straight line
AG as base; in 1. 11 we have only to construct a triangle with sides of given
length without any restriction as to how it is to be placed. Thus in 1. • we
set out any tine whatever and measure successively three lengths along it
beginning from the given extremity, and what we must regard as the base is the
intermediate length, not the length beginning at the given extremity, of the
straight line arbitrarily set out. Here the base is a given straight line abutting
at a given point Thus the construction has to be modified somewhat from

h B

that of the preceding proposition. We must measure AG along AB so that
AG is equal to CE (or CD), and GH along GB equal to DE; and then we
must produce BA, in the opposite direction, to F, so that AF'is equal to CD
(or CE, if AG has been made equal to CD).

Then, by drawing circles (1) with centre A and radius AF, (2) with centre
G and radius GH, we determine K, one of their points of intersection, and we
prove that the triangle KAG is equal in all respects to the triangle DCE, and
then that the angle at A is equal to the angle DCE.

I think that Proclus must (though he does not say so) have felt the same
difficulty with regard to the use in 1. 33 of the result of 1. 22, and that this is
probably the reason why he gives over again the construction which I have
given above, with the remark (p.~334, 6) that ``you may obtain the construction
of the triangle in a more instructive manner (StSatricaA.iKioTioi') as follows,''

Proclus objects to the procedure of Apollonius in constructing an angle
under the same conditions, and certainly, if he quotes Apollonius correctly, the
tatter's exposition must have been somewhat slipshod.

``He takes an angle CDE at random,'' says Proclus (p.~335> '9 s Vl-)> ' ,and
a straight line AB, and with centre D and distance
CD describes the circumference CE, and in the same
way with centre A and distance AB the circumference
FB. Then, cutting off FB equal to CE, he joins AF.
And he declares that the angles A, D standing on
equal circumferences are equal.''

In the first place, as Proclus remarks, it should be
premised that AB is equal to CD in order that the
circles may be equal; and the use of Book lit. for
such an elementary construction is objectionable.
The omission to state that AB must be taken equal
to CD was no doubt a slip, if it occurred. And, as
regards the equal angles ``standing on equal circum-
ferences,'' it would seem possible that Apollonius said
this in explanation, for the sake of brevity, rather than by way of proof. It
seems to me probable that his construction was only given from the point of
view of practical, not theoretical, geometry. It really comes to the same thing
as Euclid's except that DC is taken equal to DE. For cutting off the arc BF
equal to the arc CE can only be meant in the sense of measuring the chord
CE, say, with a pair of compasses, and then drawing a circle with centre B
and radius equal to the chord CE. Apollonius' direction was therefore
probably intended as a practical short cut, avoiding the actual drawing of the
chords CE, BF, which, as well as a proof of the equality in all respects of the
triangles CDE, BAF, would be required to establish theoretically the correct-
ness of the construction.

Proposition 24.

If two triangles have the two sides equal to two sides
respectively, but have the one of the angles contained by the equal
straight lines greater than the other, they will also have the
base greater than the base.

5 Let ABC, DEF be two triangles having the two sides

AB, A C equal to the two sides DE, DF respectively, namely

AB to DE, and A C to DF, and let the angle at A be greater

than the angle at D;

I say that the base BC is also greater than the base £F,
10 For, since the angle BAC

is greater than the angle EDF,

let there be constructed, on the

straight line DE, and at the

point D on it, the angle EDG
jj equal to the angle BAC; [1. 13]

let DG be made equal to either

of the two straight lines AC,

DF, and let EG, FG be joined.

Then, since AB is equal to DE, and AC to DG,
*>the two sides BA, AC are equal to the two sides ED, DG,
respectively;

and the angle BAC is equal to the angle EDG;
therefore the base BC is equal to the base EG. [1. 4]
Again, since DF is equal to DG,
25 the angle DGF is also equal to the angle DFG; [l 5]

therefore the angle DFG is greater than the angle EGF.
Therefore the angle EFG is much greater than the angle
EGF.

And, since EFG is a triangle having the angle EFG
30 greater than the angle EGF,

and the greater angle is subtended by the greater side,

[*. 19]
the side EG is also greater than EF.
But EG is equal to BC.

Therefore BC is also greater than EF.
3S Therefore etc.

Q. E. D.

10. I have naturally left out the well-known words added by Simson in
order to avoid the necessity of considering three cases: ``Of the two sides
DE, DF let DE be the side which is not greater than the other.'' I doubt
whether Euclid could have been induced to insert the words himself, even if
it had been represented to him that their omission meant leaving two possible
cases out of consideration. His habit and that of the great Greek geometers
was, not to set out all possible cases, but to give as a rule one case, generally
the most difficult, as here, and to leave the others to the reader to work out for
himself. We have already seen one instance in 1. 7.

Proclus of course gives the other
two cases which arise if we do not
first provide that DE is not greater
than DF.

(1) In the first case G may fall
on EF produced, and it is then
obvious that EG is greater than EF.

(2) In the second case EG may
fall below EF.

If so, by 1. 21, DF, FE are
together less than DG, GE.

But DF is equal to DG; there-
fore EF is less than EG, i.e.\ than
BC.

These two cases are therefore
decidedly simpler than the case taken
by Euclid as typical, and could well be left to the ingenuity of the learner.

If however after all we prefer to insert Simson's words and avoid the latter
two cases, the proof is not complete unless we show that, with his assumption,
/''must, in the figure of the proposition, fall below EG.

De Morgan would make the following proposition precede: Every straight
tine drawn from the vertex of a triangle to the base is less than the greater of the
two sides, or than either if they are equal, and he would prove it by means of
the proposition relating to perpendicular and obliques given above, p.~291.

But it is easy to prove directly that F falls below EG, if
DE is not greater than DG, by the method employed by
Pfleiderer, Lardner, and Todhunter.

Let DF, produced if necessary, meet EG in H.
Then the angle DUG is greater than the angle DEG\

[``• >6]
and the angle DEG is not less than the angle DGE;

[.. .8]
therefore the angle DUG is greater than the angle DGH.
Hence DH is less than DG, [1. 19]

and therefore DH is less than DF.

Alternative proof.

Lastly, the modern alternative proof is worth giving.

A

Let DHhisect the angle FDG (after the triangle DEG has been made
equal in all respects to the triangle A BC t as in the proposition), and let DH
meet EG in H. Join HF.

Then, in the triangles FDH, GDH,

the two sides FD, DH are equal to the two sides GD, DH,
and the included angles FDH, GDH at equal;
therefore the base HF is equal to the base HG
Accordingly EG is equal to the sum of EH, HF;

and EH, HF are together greater than EF; [1. 20]

therefore EG, or BC, is greater than EF.
Proclus (p.~339, 1 1 sqq.) answers by anticipation the possible question that
might occur to any one on this proposition, viz.\ why does Euclid not compare
the areas of the triangles as he does in 1. 4 ? He observes that inequality of
the areas does not follow from the inequality of the angles contained by the
equal sides, and that Euclid leaves out all reference to the question both for
this reason and because the areas cannot be compared without the help of the
theory of parallels. ``But if,'' says Proclus, ``we must anticipate what is to
come and make our comparison of the areas at once, we assert that (1) if
the angles A, D — supposing that our argument proceeds with reference to the
figure in the proposition — are {together) equal to two right angles, the triangles
a« primed equal, {2) if greater than tins right angles, that triangle which has
the greater angle is less, and (3) </ they are less, greater,'' Proclus then gives
the proof, but without any reference to the source from which he quoted
the proposition. Now an-NairizI adds a similar proposition to 1. 38, but
definitely attributes it to Heron. I shall accordingly give it in the place
where Heron put it

Proposition 25.

If two triangles have the two sides equal to two sides
respectively, but have the base greater than the base, they will
also have the one of the angles contained by the equal straight
lines greater than the other.

Let ABC, DEF be two triangles having the two sides
AB, AC equal to the two sides DE, DF respectively, namely
AB to DE, and A C to DF; and let the base BC be greater
than the base FF;

I say that the angle BAC is also greater than the angle
EDF

For, if not, it is either equal to it or less.
Now the angle BAC is not equal to the angle EDF;
for then the base BC would also have been equal to the base
EF r> 4]

but it is not;

therefore the angle BA C is not equal to the angle EDF.

Neither again is the angle BA C less than the angle EDF;

for then the base BC would also have been less than the base

EF, [ft 24]

but it is not;
therefore the angle BA C is not less than the angle EDF.
But it was proved that it is not equal either;

therefore the angle BAC is greater than the angle EDF.
Therefore etc.

Q. E. D.

De Morgan points out that this proposition (as also i. 8) is a purely logical
consequence of i. 4 and 1. 14 in the same way as 1. r 9 and 1. 6 are purely
logical consequences of 1. 18 and 1. 5. If d, 6, e denote the sides, A, B, C the
angles opposite to them in a triangle ABC, and a', b', /, A', E, C the sides
and opposite angles respectively in a triangle A'HC, 1. 4 and \prop{1}{14} tell us
that, i, e being respectively equal to i\ /,

( 1 ) if A is equal to A\ then a is equal to a',

(z) if A is less than A', then a is less than «',

(3) if A is greater than A', then a is greater than a';
and it follows logically that,

(1) if a is equal to a, the angle A is equal to the angle A', [1. 8]

(3) if a is less than a, A is less than A', \

(3) if a is greater than d, A is greater than A'. } I * a 5J

Two alternative proofs of this theorem are given by Proclus (pp.~345 — 7),
and they are both interesting. Moreover both are direct.

I. Proof by Menelaus of Alexandria.

Let ABC, DEB'' be two triangles having the two sides BA, AC equal to
the two sides ED, DF, but the base BC greater than the base EF.

Then shall the angle at A be greater than the angle at D.
From BC cut off BG equal to EF. At B, on the straight line BC, make
the angle GBH (on the side of BG remote from A) equal to the angle FED.
Make BH equal to DE; join HG, and produce it to meet A C in K.
Join AH.

Then, since the two sides GB, BH are equal to the two sides FE, ED
respectively,

and the angles contained by them are equal,
HG is equal to ZVor AC,
and the angle BHG is equal to the angle EDF.
Now HK is greater than HG or AC,

and a fortiori greater than AK;

therefore the angle KAH\% greater than the angle KHA.

And, since AB is equal to BH,

the angle BAH is equal to the angle BHA.

Therefore, by addition,

the whole angle BA C is greater than the whole angle BHG,
that is, greater than the angle EDF.

i. «5, a6] PROPOSITIONS 35, 36 301

II, Heron's proof.

Let the triangles be given as before.

Since BC is greater than EF, produce EF to G so that EG is equal to
EC.

Produce ED to If so that DH is equal to DF. The circle with centre
D and radius DFwiW then pass through H. Let it be described, as FKH.

Now, since BA, AC are together greater than £C,

and Atf, C are equal to ,££>, Z>Zf respectively,

while JC is equal to EG,

EH is greater than EG.

Therefore the circle with centre E and radius EG will cut Elf, and

therefore will cut the circle already drawn. Let it cut that circle in K, and

join DK, KB.

Then, since D is the centre of the circle FKH,

DK is equal to Z>or AC
Similarly, since £ is the centre of the circle KG,
EK is equal to EG or BC,
And DE is equal to A B.

Therefore the two sides BA, AC are equal to the two sides ED, DK
respectively;

and the base BC is equal to the base EK;
therefore the angle BAC is equal to the angle EDK
Therefore the angle BA C is greater than the angle EDF.

Proposition 26,

If two triangles have the two angles equal to two angles
respectively, and one side equal to one side, namely, either the
side adjoining the equal angles, or that subtending erne 0/ the
equal angles, they wilt also have the remaining sides equal to
s the remaining sides and the remaining angle to the remaining
angle.

Let ABC, DEF be two triangles having the two angles
ABC, BCA equal to the two angles DEF, EFD respectively,
namely the angle ABC to the angle DEF, and the angle

10 BCA to the angle EFD; and let them also have one side
equal to one side, first that adjoining the equal angles, namely
BC to EF;

I say that they will also have the remaining sides equal
to the remaining sides respectively, namely AB to DE and

>s AC to DF, and the remaining angle to the remaining angle,
namely the angle BA C to the angle EDF.

For, if AB is unequal to DE, one of them Is greater.
Let AB be greater, and let BG be made equal to DE;
and let GC be joined.
20 Then, since BG is equal to DE, and BC to EF,
the two sides GB, BC are equal to the two sides DE, EF
respectively;
and the angle GBC is equal to the angle DEF;

therefore the base GC is equal to the base DF,
25 and the triangle GBC is equal to the triangle DEF,
and the remaining angles will be equal to the remaining angles,
namely those which the equal sides subtend; [1. 4]

therefore the angle GCB is equal to the angle DFE.
But the angle DFE is by hypothesis equal to the angle BCA;
3 o therefore the angle BCG is equal to the angle BCA,

the less to the greater: which is impossible.
Therefore AB is not unequal to DE,
and is therefore equal to it.
But BC is also equal to EF;
35 therefore the two sides AB, BC are equal to the two

sides DE, EF respectively,
and the angle ABC is equal to the angle DEF;

therefore the base A C is equal to the base DF,
and the remaining angle BAC is equal to the remaining
40 angle EDF. [l 4]

Again, let sides subtending equal angles be equal, as AB
to DE;

I say again that the remaining sides will be equal to the
remaining sides, namely AC to DF and BC to EF, and
45 further the remaining angle BAC is equal to the remaining
angle EDF.

For, if BC is unequal to EF, one of them is greater.
Let BC be greater, if possible, and let BH be made equal
to EF; let AH be joined.
50 Then, since BH is equal to EF, and AB to DE,
the two sides AB, BH are equal to the two sides DE, EF
respectively, and they contain equal angles;

therefore the base AH is equal to the base DF,
and the triangle ABH is equal to the triangle DEF,
ss and the remaining angles will be equal to the remaining angles,
namely those which the equal sides subtend; [1. 4]

therefore the angle BHA is equal to the angle EFD.
But the angle EFD is equal to the angle BCA;
therefore, in the triangle AHC, the exterior angle BHA is
60 equal to the interior and opposite angle BCA:

which is impossible. [1. 16]

Therefore BC is not unequal to EF,

and is therefore equal to it.
But AB is also equal to DE;
65 therefore the two sides AB, BC are equal to the two sides
DE, EF respectively, and they contain equal angles;
therefore the base AC is equal to the base DF,
the triangle ABC equal to the triangle DEF,
and the remaining angle BAC equal to the remaining angle
r>EDF, [1.4]

Therefore etc.

Q, E. D.

1 — 3. the aide adjoining the equal angles, rhtupir Hjr rpot roTi but ywrtati.

39. la by hypothesis equal. vr6Ktir at tnj, according to the elegant Greek idiom.
ifbtettiat is used for the passive of irrvrlffitfn, as Keit*eu is used for the passive of riBrtfu, and
to with the other compounds. cf.\ Trpxruiitidat, ``to be added.''

The alternative method of proving this proposition, viz.\ by applying one
triangle to the other, was very early discovered, at least so far as regards the
case where the equal sides are adjacent to the equal angles in each. An-Nairīzī
gives it for this case, observing that the proof is one which he had found, but
of which he did not know the author.

Proclus has the following interesting note {p.~35*, 13 — 18): ``Eudemus
in his geometrical history refers this theorem to Thales. For he says that, in
the method by which they say that Thales proved the distance of ships in the
sea, it was necessary to make use this theorem.'' As, unfortunately, this
information is not sufficient of itself to enable us to determine how Thales
solved this problem, there is considerable room for conjecture as to bis
method.

The suggestions of Bretschneider and Cantor agree in the assumption
that the necessary observations were probably made from the top of some
tower or structure of known height, and that a right-angled triangle was used in
which the tower was the perpendicular, and the line connecting the bottom of
the tower and the ship was the base, as in the annexed figure, where AB is the
tower and C the ship.~Bretschneider (Die Geometrie and die Geometer vor
Eukleides, § 30) says that it was only necessary for
the observer to observe the angle CAB, and then
the triangle would be completely determined by
means of this angle and the known length AB.
As Bretschneider says that the result would be
obtained ``in a moment ``by this method, it is not
clear in what sense he supposes Thales to have
``observed'' the angle SAC. Cantor is more
definite (GescA. d. Math. i ( , p.~145), for he says that
the problem was nearly related to that of finding the
Seqt from given sides. By the Seqt in the Papyrus Rhind is meant the
ratio to one another of certain lines in pyramids ot obelisks. Eisenlohr and
Cantor took the one word to be equivalent, sometimes to the cosine of the
angle made by the edge of the pyramid with the coterminous diagonal of the
base, sometimes to the tangent of the angle of slope of the faces of the pyramid.
Ft is now certain that it meant one thing, viz.\ the ratio of half the side of
the base to the height of the pyramid, i.e.\ the cotangent of the angle of
slope. The calculation of the Seqt thus implying a sort of theory of simi-
larity, or even of trigonometry, the suggestion of Cantor is apparently that
the Seqt in this case would be found from a smalt right-angled triangle ADE
having a common angle A with ABC as shown in the figure, and that the
ascertained value of the Stqt with the length AB would determine BC. This
amounts to the use of the property of similar triangles; and Bretschneider's
suggestion must apparently come to the same thing, since, even if Thales
measured the angle in our sense (e.g.\ by its ratio to a right angle), he would,
in the absence of something corresponding to a table of trigonometrical ratios,
have gained nothing and would have had to work out the proportions all the
same.

Max C P, Schmidt also (Kulturhistorische Beiirage zur Kenntnis des
griechisehcn und romischen Alter turns, 1906, p.~32) similarly supposes Thales to
have had a right angle made of wood or bronze with the legs graduated, to
have placed it in the position ADE (A being the position of his eye}, and
then to have read off the lengths AD, DE respectively, and worked out the
length of BC by the rule of three.

How then does the supposed use of similar triangles and their property
square with Eudemus' remark about 1. 26 ? As it stands, it asserts the
equality of two triangles which have two angles and one side respectively
equal, and the theorem can only be brought into relation with the above
explanations by taking it as asserting that, if two angles and one side of one
triangle are given, the triangle is completely determined. But, if Thales
practically used proportions, as supposed, 1. 26 is surely not at all the theorem
which this procedure would naturally suggest as underlying it and being
``necessarily used''; the use of proportions or of similar but not equal
triangles would surely have taken attention altogether away from 1. 26 and
fixed it on vi. 4.

For this reason I think Tannery is on the right road when he tries to find
a solution using 1. 26 as it stands, and withal as primitive as any recorded
solution of such a problem. His suggestion (La Gtemitrit gretqut, pp.~90—1)
is based on the fiuminis varatio of the Roman agrimensor Marcus Junius
Nipsus and is as follows.

To find the distance from a point A to an inaccessible point B. From A
measure along a straight line at right angles to AB a
length AC and bisect it at D. From C draw C£ at right
angles to CA on the side of it remote from B, and let £
be the point on it which is in a straight line with B and D.

Then, by 1. 26, CE is obviously equal to AB.

As regards the equality of angles, it is to be observed
that those at D are equal because they are vertically
opposite, and, curiously enough, Thales is expressly
credited with the discovery of the equality of such angles.

The only objection which I can see to Tannery's
solution is that it would require, in the case of the ship, a
certain extent of free and level ground for the construction
and measurements.

I suggest therefore that the following may have been
Thales' method. Assuming that he was on the top of a
tower, he had only to use a rough instrument made of a straight stick and a
cross-piece fastened to it so as to be capable of turning about the fastening
(say a nail) so that it could form any angle with the stick and would remain
where it was put. Then the natural thing would be to fix the stick upright
(by means of a plumb-line) and direct the cross-piece towards the ship.
Next, leaving the cross-piece at the angle so found, the stick could be turned
round, still remaining vertical, until the cross-piece pointed to some visible
object on the shore, when the object could be mentally noted and the distance
from the bottom of the tower to it could be subsequently measured. This
would, by \prop{1}{36}, give the distance from the bottom of the tower to the ship.
This solution has the advantage of corresponding better to the simpler and
more probable version of Thales' method of measuring the height of the
pyramids; Diogenes Laertius says namely (i. 27, p.~6, ed. Cobet) on the
authority of Hieronymus of Rhodes (b.c 293 — 230), that he waited for this
purpose until the moment when our shadows are of thesamt length as ourselves.

Recapitulation of congruence theorems.

Proclus, like other commentators, gives at this point (p.~347, 20 sqq.) a
summary of the cases in which the equality of two triangles in all respects can
be established. We may, he says, seek the conditions of such equality by
successively considering as hypotheses the equality (1) of sides alone, (2) of
angles alone, (3) of sides and angles combined. Taking (1) first, we can only
establish the equality of the triangles in all respects if all three sides are
respectively equal; we cannot establish the equality of the triangles by any
hypothesis of class (2), not even the hypothesis that all the three angles are
respectively equal; among the hypotheses of class (3), the equality of one
side and one angle in each triangle is not enough, the equality (a) of one side
and all three angles is more than enough, as is also the equality {) of two
sides and two or three angles, and (c) of three sides and one or two angles.

The only hypotheses therefore to be examined from this point of view are
the equality of

(a) three sides [Eucl. 1. 8].

(jB) two sides and one angle [1. 4 proves one case of this, where the angle
is that contained by the sides which are by hypothesis equal].

(y) one side and two angles [1. *6 covers all cases].

It is curious that Proclus makes no allusion to what we call the ambiguous
case, that case namely of (£') in which it is an angle opposite to one of the
two specified sides in one triangle which is equal to the angle opposite to the
equal side in the other triangle. Camerer indeed attributes to Proclus the
observation that in this case the equality of the triangles cannot, unless some
other condition is added, be asserted generally; but it would appear that
Camerer was probably misled by a figure {Proclus, p.~351) which looks like a
figure of the ambiguous case but is really only used to show that in 1. 26 the
equal sides most be corresponding sides, i.e., they must be either adjacent to the
equal angles in each triangle, or opposite to corresponding equal angles, and
that, e.g., one of the equal sides must not be adjacent to the two angles in
one triangle, while the side equal to it in the other triangle is opposite to one
of the two corresponding angles in that triangle.

The ambiguous case.

If (wo triangles have two sides equal to two sides respectively, and if /he
angles opposite to one pair of equal sides be also equal, then wilt the angles
opposite the other pair of equal sides be either equal or supplementary; and, in
the former ease, the triangles will be equal in all respects.

Let ABC, DEFbt two triangles such that AB is equal to DE, and AC
to DF, while the angle ABC is equal to the angle DEF
it is required to prove that the angles ACB, DFE are either equal or
supplementary.

A

A ?

Now (1), if the angle BAC be equal to the angle EDF, it follows, since
the two sides AB, A Cute equal to the two sides DE, DF respectively, that
the triangles ABC, DEFaxe equal in all respects, [l. 4]

and the angles A CB, DFE are equal.

(a) If the angles BAC, EDF be not equal, make the angle EDO {on
the same side of ED as the angle EDF) equal to the angle BA C.

Let EF, produced if necessary, meet DG in G.

Then, in the triangles ABC, DEG,
the two angles BAC, ABC are equal to the two angles EDG, DEG
respectively,
and the side AB is equal to the side DE;
therefore the triangles ABC, DEG ate equal in all respects, [1. a 6]
so that the side AC is equal to the side DG,
and the angle ACS is equal to the angle DGE.
Again, since AC is equal to DFs well as to DG,
DF'vl equal to DG,
and therefore the angles DFG, DGFaie equal.

But the angle DFE is supplementary to the angle DFG; and the angle
DGF was proved equal to the angle ACS;

therefore the angle DFE is supplementary to the angle ACB.

If it is desired to avoid the ambiguity and secure that the triangles may
be congruent, we can introduce the necessary conditions into trie enunciation,
on the analogy of Eucl. vi. 7.

If two triangles have two sides of the one equal to two sides of the other
respectively, and the angles opposite to a pair of equal sides equal, then, if tie
angles opposite to the other pair of equal sides are both acute, or both obtuse, or if
one of them is a right angle, the two triangles art equal in all respects.

The proof of the three cases (by reductio ad absurdum) was given by
Todhunter-

Proposition 27.

If a straight line failing on two straight lines make the
alternate angles equal to one another, the straight lines will be
parallel to one another.

For let the straight line EF failing on the two straight
s lines AB, CD make the alternate angles AEF, EFD equal
to one another;

I say that AB is parallel to CD.

For, if not, AB, CD when pro-
duced will meet either in the direction
10 of B, D or towards A, C.

Let them be produced and meet,
in the direction of B, D, at G.

Then, in the triangle GEF,

the exterior angle A EF is equal to the interior and opposite
is angle EFG:
which is impossible. [1. 16'

Therefore AB, CD when produced will not meet in the
direction of B, D.

Similarly it can be proved that neither will they meet
10 towards A, C.

But straight lines which do not meet in either direction
are parallel; [Def. 23]

therefore AB is parallel to CD. .
Therefore etc.

Q. E. D.

1. filling on two straight lines, ds tfa tiSiUit t/irtrrvuita, the phrase being the s
u that used in Post, 5, meaning a transversa!.

1. the alternate angles, a! rfrsXXif yteriai. Produs (p.~357, 9) explains that Euclid
uses the word alternate (or, more exactly, alternattfy, kaXXdf) in two conneiions, (1) of a
certain transformation of a proportion, as in Book V, and the arithmetical Books, (1} as here,
of certain of the angles formed by parallels with a straight line crossing them, Attentate
angles are, according to Euclid as interpreted by Produs, those which are not on the same
side of the transversal, and are not adjacent, but are separated by the transversal, both being
within the parallels but one ``above'' and the other ``below. - ' The meaning is natural
enough if we imagine the four internal angles to be taken in cyclic order and alternate angles
to be any two of them not successive but separated by one angle of the four.

9. in the direction of B, D or towards A, C, literally ``towards the parts B, D or
towards A, C,'' iri rk B, > Mm 4 M *i A, T.

With this proposition begins the second section of the first Book. Up
to this point Euclid has dealt mainly with triangles, their construction
and their properties in the sense of the relation of their parts, the sides and
angles, to one another, and the comparison of different triangles in respect of
their parts, and of their area in the particular cases where they are congruent.

The second section leads up Co the third, in which we pass to relations
between the areas of triangles, parallelograms and squares, the special feature
being a new conception of equality of areas, equality not dependent on
amgrvenu. This whole subject requires the use of parallels. Consequently
the second section beginning at 1. 27 establishes the theory of parallels,
introduces the cognate matter of the equality of the sum of the angles of a
triangle to two right angles (1. 32), and ends with two propositions forming the
transition to the third section, namely 1, 33, 34, which introduce the parallelo-
gram for the first time,

Aristotle on parallels.

We have already seen reason to believe that Euclid's personal contribution
to the subject was nothing less than the formulation of the famous Postulate
5 (see the notes on that Postulate and on Def. 13), since Aristotle indicates
that the then current theory of parallels contained a petitio principii, and
presumably therefore it was Euclid who saw the defect and proposed the
remedy.

But it is clear that the propositions 1. 27, 18 were contained in earlier
text-books. They were familiar to Aristotle, as we may judge from two
interesting passages.

(1) In Anal. Pott. 1. 5 he is explaining that a scientific demonstration
must not only prove a fact of every individual of a class (mri rarro?) but
must' prove it primarily and generally true (rpaTur xaBokov) of the whok of
the class as one; it will not do to prove it first of one part, then of another
part, and so on, until the class is exhausted. He illustrates this {74 a 13 — 16)
by a reference to parallels: ``If then one were to show that right (angles) do
not meet, the proof of this might be thought to depend on the fact that this
is true of all (pairs of actual) right angles. But this is not so, inasmuch as
the result does not follow because (the two angles are) equal (to two right
angles) in the particular way [i.e.\ because each is a right angle], but by virtue
of their being equal {to two right angles) in any way whatever [i.e.\ because
the sum only needs to be equal to two right angles, and the angles themselves
may vary as much as we please subject to this],''

(2) The second passage has already been quoted in the note on Def. 23:
``there is nothing surprising in different hypotheses leading to the same false
conclusion; e.g.\ the conclusion that parallels meet might equally be drawn
from either of the assumptions (a) that the interior (angle) is greater than the
exterior or (b) that the sum of the angles of a mangle is greater than two
right angles'' (Anal, Prior, n. 17, 66 a n — 15).

I do not quite concur in the interpretation which Heiberg places upon
these passages (Mathematisehes su Aristoteks, pp.~18 — 19), He says, first,
that the allusion to the ``interior angle'' being ``greater than the exterior'' in
the second passage shows that the reference in the first passage must be to
Eucl. 1. 28 and not to 1. 27, and he therefore takes the words Sri wSl lv<u in
the first passage {which I have translated ``because the two angles are equal
to two right angles in the particular way ``) as meaning ``because the angles,
viz.\ the exterior and the interior, are equal in the particular way.'' He also
takes at ipBai oi oTiiiTrCirrmxri (which I have translated ``right angles do not
meet,'' an expression quite in Aristotle's manner) to mean ``perpendicular
straight lines do not meet ``; this is very awkward, especially as he is obliged
to supply angles with Urai in the next sentence.

But I think that the first passage certainly refers to 1. 28, although I do
not think that the alternative (a) tn the second passage suggests it. This
alternative may, I think, equally with the alternative (b) refer to 1. 27. That
proposition is proved by reduetio ad absurdum based on the fact that, if the
straight lines do meet, they must form a triangle, in which case the exterior
angle must be greater than the interior (while according to the hypothesis
these angles are equal). It is true that Aristotle speaks of the hypothesis
that the interior angle is greater than the exterior; but after all Aristotle had
only to state some incorrect hypothesis. It is of course only in connexion
with straight lines meeting, as the hypothesis in 1. 27 makes them, that the
alternative (b) about the sum of the angles of a triangle could come in, and
alternative (a) implies alternative (b).

It seems clear then from Aristotle that \prop{1}{27}, 28 at least are pre-Euclidean,
and that it was only in 1. 29 that Euclid made a change by using his Postulate.

De Morgan observes that 1. 27 is a logical equivalent of I- 16. Thus, if A
means ``straight lines forming a triangle with a transversal,'' B ``straight lines
making angles with a transversal on the same side which are together less
than two right angles,'' we have

All A is 3,
and it follows logically that

All not-i? is not- A.

Proposition 28.

If a straight line falling on two straight lines make the
exterior angle equal to the interior and opposite angle on the
same side, or the interior angles on the same side equal to two
right angles, the straight lines will be parallel to one another.

For let the straight line EF falling on the two straight
lines AB, CD make the exterior angle EGB equal to the
interior and opposite angle GHD, or the interior angles on
the same side, namely BGH, GHD, equal to two right angles;

I say that AB is parallel to CD.

For, since the angle EGB is
equal to the angle GHD,
while the angle EGB is equal to the
angle AGH, [t 15]

the angle A GH is also equal to the
angle GHD;
and they are alternate;

therefore AB is parallel to CD. [1. 97]

Again, since the angles BGH, GHD are equal to two
right angles, and the angles AGH, BGH are also equal to
two right angles, [l 13]

the angles A GH, BGH are equal to the angles BGH, GHD.

Let the angle BGH be subtracted from each;
therefore the remaining angle AGH Is equal to the remaining
angle GHD;
and they are alternate;

therefore AB is parallel to CD. [1. aj]

Therefore etc.

Q. E. D.

One criterion of parallelism, the equality of alternate angles, is given in
t. 37; here we have two more, each of which is easily reducible, and is actually
reduced, to the other.

Proclus observes (pp.~358—9) that Euclid could have stated six criteria as
well as three, by using, in addition, other pairs of angles
in the figure (not adjacent) of which it could be predi-
cated that the two angles are equal or that their sum
is equal to two right angles, A natural division is to
consider, first the pairs which are on the same side of
the transversal, and secondly the pairs which are on
different sides of it.

Taking (1) the possible pairs on the same side, we
may have a pair consisting of

(it) two internal angles, viz.\ the pairs (BGH,
G£f£>) nd (AG ff.GIfC);

(b) two external angles, viz.\ the pairs (EGB, DHJF) and (EGA, CHF);

(e) one external and one internal angle, viz.\ the pairs (EGB, GffD\
(FHD, HGB), (EGA, GHC) and (EMC, HGA).

i. 28, i<>] PROPOSITIONS 28, 29 311

And (2) the possible pairs on different sides of the transversal may consist
respectively of

(a) two internal angles, viz.\ the pairs (AGH, GHD) and {CHG, HGB);

{b) two external angles, vi*. the pairs {AGE, DHF) and {EGB, CHF);

if) one external and one internal, viz.\ the pairs {AGE, GHD), {EGB,
GHC), (FHC, HGB) aird {FHD, HGA).

The angles are equal in the pairs (i) {e), {2) (a) and (2) (/>), and the sum
is equal to two right angles in the case of the pairs (1 ) (a), (1) {6) and (2) {c\
For his criteria Euclid selects the cases (2) (o) [i. 27] and (1) {(), {i)(jn) [1. 28],
leaving out the other three, which are of course equivalent but are not quite
so easily expressed.

From Proclus' note on 1. 28 (p.~361) we iearn that one Aigeias {? Aineias)
of Hierapolis wrote an epitome or abridgment of the Elements. This seems
to be the only mention of this editor and his work; and they are only
mentioned as having combined Eucl. 1. 27, 28 into one proposition. To do
this, or to make the three hypotheses the subject of three separate theorems,
would, Proclus thinks, have been more natural than to deal with them, as
Euclid does, in two propositions. Proclus has no suggestion for explaining
Euclid's arrangement unless the ground were that 1. 27 deals with angles on
different sides, 1. 28 with angles on one and the same side, of the transversal.
But may not the reason have been one of convenience, namely that the
criterion of 1. 27 is that actually used to prove parallelism, and is moreover
the basis of the construction of parallels in \prop{1}{31}, while 1. 28 only reduces the
other two hypotheses to that of 1. 27, so that precision of reference, as well as
clearness of exposition, is better secured by the arrangement adopted ?

Proposition 29.

A straight line falling on parallel straight lines makes
the alternate angles equal to one another, the exterior angle
equal to the interior and opposite angle, and the interior angles
on the same side equal to two right angles.
5 For let the straight line EF fall on the parallel straight
lines AB, CD;

I say that it makes the alternate angles AGH, GHD
equal, the exterior angle EGB equal to the interior and
opposite angle GHD, and the interior angles on the same
10 side, namely BGH, GHD, equal to two right angles.

For, if the angle AGH is unequal
to the angle GHD, one of them is
greater.

Let the angle AGH be greater.
is .Let the angle BGH be added to
each;

therefore the angles A GH, BGH
are greater than the angles BGH,
GHD.

*> But the angles A GH, BGH are equal to two right angles;

p.~13]

therefore the angles BGH, GHD are less than two
right angles.

But straight lines produced indefinitely from angles less

than two right angles meet; [Post. 5J

35 therefore AB, CD, if produced indefinitely, will meet;

but they do not meet, because they are by hypothesis parallel.

Therefore the angle AGH is not unequal to the angle

GHD,

and is therefore equal to it.
30 Again, the angle AGH is equal to the angle EGB; [* 15]
therefore the angle EGB is also equal to the angle
GHD. [C H. 1]

Let the angle BGH be added to each;

therefore the angles EGB, BGH are equal to the

3S angles BGH, GHD. [C. N. j]

But the angles EGB, BGH are equal to two right angles;

therefore the angles BGH, GHD are also equal to
two right angles.

Therefore etc q. e. d.

13. straight lines produced indefinitely from angles less than two right angles,
of H aV' {\avirbrwy ff Sio ofifltvy fKaWliuf *a. til drttpv vvilvIttww, a variation from the
more explicit language of Postulate j. A good deal is left to be understood, namely that the
straight lines begin from points at which they meet a transversal, and make with it internal
angles on the same aide the sum of which is less than two right angles.

16. because they are by hypothesis parallel, literally ``because they are supposed
parallel,'' d t6 rapoXKXavt afrai itTrtmtiaBai.

Proof by ``Playfair's'' axiom.

If, instead of Postulate 5, it is preferred to use ``Playfair's ``axiom in the
proof of this proposition, we proceed thus.

To prove that the alternate angles AGH,
GHD are equal.

If they are not equal, draw another straight
line KL through G making the angle KGH
equal to the angle GHD.

Then, since the angles KGH, GHDaie equal,

KL is parallel to CD. [1. 17]

Therefore two straight lines KL, AB intersecting at G are bath parallel to
the straight line CD:

which is impossible {by the axiom).

Therefore the angle AGH cannot but be equal to the angle GHD.
The rest of the proposition follows as in Euclid.

Proof of Euclid's Postulate 5 from ``PI ay fair's'' axiom.

Let AB, CD make with the transversal EF the angles AEF, EFC
:ogether less than two right angles.

To prove that AS, CD meet towards A, C. _ B

Through E draw GH making with -E-Fthe angle Q
GEF equal (and alternate) to the angle EFD. \

Thus GH is parallel to CD, [1. 27] \

Then (1) AB must meet CZ>in one direction or c F

the other.

For, if it does not, AB must be parallel to CD; hence we have two
straight lines AB, GH intersecting at E and both parallel to CD:
which is impossible.

Therefore AB, CD must meet.

(2) Since AB, CD meet, they must- form a triangle with EF.

But in any triangle any two angles are together less than two right angles.

Therefore the angles AEF, EFC (which are less than two right angles),
and not the angles BEF, EFD (which are together greater than two right
angles, by 1. 13), are the angles of the triangle;

that is, EA, FC meet in the direction of A, C, or on the side of EF on
which are the angles together less than two right angles.

The usual course in modern text-books which use ``Play fair's ``axiom in
lieu of Euclid's Postulate is apparently to prove i a 9 by means of the axiom,
and then Euclid's Postulate by means of 1. 29,

De Morgan would introduce the proof of Postulate 5 by means of
``Playfair's ``axiom before 1. 29, and would therefore apparently prove I. ag as
Euclid does, without any change.

As between Euclid's Postulate 5 and ``Playfair's ``axiom, it would appear
that the, tendency in modern text-books is rather in favour of the latter.
Thus, to take a few noteworthy foreign writers, we find that Rausenberger
stands almost alone in using Euclid's Postulate, while Hilbert, Henrict and
Trentlein, Rouch and De Comberousse, Enriques and Amaldi all use
``Playfair's ``axiom.

Yet the case for preferring Euclid's Postulate is argued with some force by
Dodgson (Euclid and his modem Rivals, pp.~44 — fi). He maintains (i) that
``Playfair's ``axiom in fact involves Euclid's Postulate, but at the same time
involves more than the latter, so that, to that extent, it is a needless strain on
the faith of the learner. This is shown as follows.

Given AB, CD making with EF l\\z angles AEF, EFC together less than
two right angles, draw GH through E so that the angles GEF, EFC are
together equal to two right angles.

Then, by 1, 28, GH, CD are ``separational,''

We see then that any lines which have the property (a) that they make
with a transversal angles less than two right angles have also the property (£)
that one of them intersects a straight line which is ``separational'' from
the other.

Now Playfair's axiom asserts that the lines which have property (p) meet
if produced: for, if they did not, we should have two intersecting straight
lines both ``separational ``from a third, which is impossible.

We then argue that lines having property (a) meet because lines having
property (a) are lines having property (ji). But we do not know, until we
have proved 1. 29, that all pairs of lines having property (fi) have also property

(a). For anything we know to the contrary, class (0) may be greater than
class (a). Hence, if you assert anything of class (fi), the logical effect is more
extensive than if you assert it of class (o); for you assert it, not only of that
portion of class 0) which is known to be included in class (a), but also of the
unknown (but possibly existing) portion which is not so included.

(a) Euclid's Postulate puts before the beginner clear and positive con-
ceptions, a pair of straight lines, a transversal, and two angles together less
than two right angles, whereas ``PlayfairV' axiom requires him to realise a
pair of straight lines which never meet though produced to infinity: a negative
conception which does not convey to the mind any clear notion of the relative
position of the lines. And (p, 68) Euclid's Postulate gives a direct criterion
for judging that two straight lines meet, a criterion which is constantly required,
e.g.\ in \prop{1}{44}. It is true that the Postulate can be deduced from ``Playfair's ``
axiom, but editors frequently omit to deduce it, and then tacitly assume it
afterwards: which is the least justifiable course of all.

Proposition 30.

Straight lines parallel to the same straight line are also
parallel to one another.

Let each of the straight lines AB, CD be parallel to EF\
I say that AB is also parallel to CD.
5 For let the straight line GK fall upon
them.

Then, since the straight line GK
has fallen on the parallel straight lines
AB, EF,
to the angle AGK is equal to the

angle GHF. [1. *a]

Again, since the straight line GK has fallen on the parallel
straight lines EF, CD,

the angle GHF is equal to the angle GKD. [i- 29]
1 j But the angle A GK was also proved equal to the angle
GHF;

therefore the angle AGK is also equal to the angle
GKD; \C.N.i\

and they are alternate.

30 Therefore AB is parallel to CD.

Q. E. D.

JO. The usual comlusiim in general terms (``Therefore etc.'') repenting the oouociaiion
ts, curiously enough, wanting at the end of this proposition.

The proposition is, as De Morgan points out, the iogieai equivalent of
``PlayfaiiV' axiom. Thus, if X denote ``pairs of straight lines intersecting one
another,'' Y'' pairs of straight lines parallel to one and the same straight line,''
we have

No X is Y,
and it follows logically that

No Y is X.

De Morgan adds that a proposition is much wanted about parallels (or
perpendiculars) to two straight lines respectively making the same angles with
one another as the latter do. The proposition may be enunciated thus:

If the sides of one angle be respectively (1) parallel or (2) perpendicular to
the sides of another angle, the two angles are either
equal Or supplementary.

(1) Let DE be parallel to AB and GEF parallel
toBC.

To prove that the angles ABC, DEG are equal
and the angles ABC, DEF supplementary.

Produce DE to meet BC in H.

Then [1. 29] the angle DEG is equal to the angle
DMC,

and the angle ABC is equal to the angle DHC.

Therefore the angle DEG is equal to the angle ABC; whence also the
angle DEF is supplementary to the angle ABC-
it) Let ED be perpendicular to AB, and GEF perpendicular to BC.

To prove that the angles ABC, DEG are
equal, and the angles ABC, DEF supplementary.

Draw ED' at right angles to ED on the side
of it opposite to B, and draw EG' at right angles
to EFoa the side of it opposite to B.

Then, since the angles BDE, DED, being
right angles, are equal,

ED is parallel to BA. [1. 27]
Similarly EG' is parallel to BC.

Therefore [Part (1)] the angle DEG' is equal to the angle ABC.
But, the right angle DED being equal to the right angle GEG, if the
common angle GED be subtracted,

the angle DEG is equal to the angle D''EG '.
Therefore the angle DEG is equal to the angle ABC; and hence the
angle DEF is supplementary to the angle ABC.

Proposition 31.

Through a given point to draw a straight line parallel to a
given straight line.

Let A be the given point, and BC the given straight
line;

thus it is required to draw through the point A a straight
line parallel to the straight line BC.

Let a point D be taken at random on BC, and let AD be
joined; on the straight line DA,

and at the point A on it, let the g £

angle DAE be constructed equal /

to the angle W/?C [i. 23]; and let the /

straight line AF be produced in a e d c

straight line with EA.

Then, since the straight line AD falling on the two
straight lines BC, EF has made the alternate angles EAD,
ADC equal to one another,

therefore EAF is parallel to BC. [i. 17]

Therefore through the given point A the straight line
EAF has been drawn parallel to the given straight line BC.

Q. E. F.

Proclus rightly remarks (p.~376, 14 — 20) that, as it is implied in 1. iz
that only one perpendicular can be drawn to a straight line from an external
point, so here it is implied that only one straight line can be drawn through a
point parallel to a given straight line. The construction, be il observed,
depends only upon \prop{1}{27}, and might therefore have come directly after that
proposition. Why then did Euclid postpone it until after 1, 29 and 1. 30?
Presumably because he considered it necessary, before giving the construction,
to place beyond all doubt the fact that only one such parallel can be drawn.
Proclus infers this fact from 1. 30; for, he says, if two straight lines could be
drawn through one and the same point parallel to the same straight line, the two
straight lines would be parallel, though intersecting at the given point: which
is impossible. I think it is a fair inference that Euclid would have considered
it necessary to justify the assumption that only one parallel can be drawn
by some such argument, and that he deliberately determined that his own
assumption was more appropriate to be made the subject of a Postulate
than the assumption of the uniqueness of the parallel.

Proposition 32.

In any triangle, if one of the sides be produced, the exterior
angle is equal to the two interior and opposite angles, and the
three interior angles of the triangle are equal to two right
angles.

Let ABC be a triangle, and let one side of it BC be
produced to D;

I say that the exterior angle A CD is equal to the two
interior and opposite angles CAB, ABC, and the three
interior angles of the triangle ABC, BCA, CAB are equal
to two right angles.

For let CE be drawn through the point C parallel to the
straight line AB. [i- 3']

Then, since AB is parallel to CE,
and AC has fallen upon them,

the alternate angles EAC, ACE are
equal to one another. [1. 19]

Again, since AB is parallel to
CE,

and the straight line BD has fallen upon them,
the exterior angle ECD is equal to the interior and opposite
angle ABC. [1. 29]

But the angle ACE was also proved equal to the angle
BAC;

therefore the whole angle A CD is equal to the two
interior and opposite angles EAC, ABC.

Let the angle A CB be added to each;

therefore the angles A CD, ACE are equal to the three
angles ABC, BCA, CAB.

But the angles A CD, ACE are equal to two right angles;

, D- 13]
therefore the angles ABC, BCA t CAB are also equal

to two right angles.

Therefore etc.

q. E. D.

This theorem was discovered in the very early stages of Greek geometry.
What we know of the history of it is gathered from three allusions found in
Eutocius, Proclus and Diogenes Laertius respectively.

1. Eutocius at the beginning of his commentary on the Conics of
Apollonius (ed. Heiberg, Vol. u- p.~r7o) quotes Geminus as saying that ``the
ancients (oE Juum) investigated the theorem of the two right angles in each
individual species of triangle, first in the equilateral, again in the isosceles,
and afterwards in the scalene triangle, and later geometers demonstrated the
general theorem to the effect that in any triangle the three interior angles are
equal to two right angles.''

a. Now, according to Proclus (p.~379, 2 — 5), Eudemus the Peripatetic
refers the discovery of this theorem to the Pythagoreans and gives what he
affirms to be their demonstration of it This demonstration will be given
below, but it should be remarked that it is general, and therefore that the
``later geometers'' spoken of by Geminus were presumably the Pythagoreans,
whence it appears that the ``ancients'' contrasted with them must have
belonged to .the time of Thales, if they were not his Egyptian instructors.

3. That the truth of the theorem was known to Thales might also
be inferred from the statement of Pamphile (quoted by Diogenes Laertius,
1. 34 — Si p 6, ed. Cobet) that ``he, having leamt geometry from the
Egyptians, was the first to inscribe a right-angled triangle in a circle and
sacrificed an ox'' (on the strength of it); in other words, he discovered that
the angle in a semicircle is a right angle. No doubt, when this fact was once
discovered (empirically, say), the consideration of the two isosceles triangles
having the centre for vertex and the sides of the right angle for bases
respectively, with the help of the theorem of Eucl. t. 5, also known to
Thales, would easily lead to the conclusion that the sum of the angles of
a right-angled triangle is equal to two right angles, and it could be readily
inferred that the angles of any triangle were likewise equal to two right angles
(by resolving it into two right-angled triangles). But it is not easy to see how
the property of the angle in a semicircle could he pruned except (in the reverse
order) by means of the equality of the sum of the angles of a right-angled
triangle to two right angles; and hence it is most natural to suppose, with
Cantor, that Thales proved it (if he did prove it) practically as Euclid does
in lit. 31, i.e.\ by means of 1. 31 as applied to right-angled triangles at all events.
If the theorem of 1. 32 was proved before Thales' time, or by Thales
himself, by the stages indicated in the note of Geminus, we may be satisfied
that the reconstruction of the argument of the older proof by Hankel
(pp.~96 — -7) and Cantor (i,, pp.~143 — 4) is not far wrong. First, it must have
been observed that six angles equal to an angle of an equilateral triangle would,
if placed adjacent to one another round a common vertex, fill up the whole
space round that vertex. It is true that Proclus attributes to the Pythagoreans
the general theorem that only three kinds of regular polygons, the equilateral
triangle, the square and the regular hexagon, can fill up the entire space round
a point, but the practical knowledge that equilateral triangles have this
property could hardly have escaped the Egyptians, whether they made floors
with tiles in the form of equilateral triangles or regular hexagons (Allman,
Greek Geometry from Tkalef to Euclid, p.~11) or joined the ends of adjacent
radii of a figure like the six-spoked wheel, which was their common form of
wheel from the time of Ramses II. of the nineteenth Dynasty, say 1 300 B.C.
(Cantor, i lt p, 109). It would then be clear that six angles equal to an angle
of an equilateral triangle are equal to four right angles, and therefore that the
three angles of an equilateral triangle are equal to two right angles. (It would
be as clear or clearer, from observation of a square divided into two triangles
by a diagonal, that an isosceles right-angled triangle has each of its equal
angles equal to half a right angle, so that an isosceles right-angled triangle
must have the sum of its angles equal to two right angles.) Next, with regard
to the equilateral triangle, it could not fail to be observed
that, if AD were drawn from the vertex A perpendicular
to the base BC, each of the two right-angled triangles so
formed would have the sum of its angles equal to two right j

angles; and this would be confirmed by completing the J

rectangle ADCE, when it would be seen that the rectangle /

(with its angles equal to four right angles) was divided by /

its diagonal into two equal triangles, each of which had B
the sum of its angles equal to two right angles. Next it
would be inferred, as the result of drawing the diagonal of any rectangle and
observing the equality of the triangles forming the two halves, that the sum of
the angles of any right-angled triangle is equal to two right angles, and henqe
(the two congruent right-angled triangles being then placed so as to form one
isosceles triangle) that the same is true of any isosceles triangle. Only the
last step remained, namely that of observing that any triangle could be
regarded as the half of a rectangle (drawn as indicated in the next figure), or
simply that any triangle could be divided into two right-angled triangles,
whence it would be inferred that in general the

sum of the angles of any triangle is equal to two P yT >s sL

right angles. i / I TS. j

Such would be the probabilities if we could j /  j

absolutely rely upon the statements attributed to '•,/_ i vi

Pamphile and Geminus respectively. But in fact
there is considerable ground for doubt in both cases.

1. Pamphiie's stoty of the sacrifice of an ox by Thales for joy at his
discovery that the angle in a semicircle is a right angle is too suspiciously like
the similar story told with reference to Pythagoras and his discovery of the
theorem of Eucl. 1. 47 (Proclus, p.~426, 6—9). And, as if this were not
enough, Diogenes Laertius immediately adds that ''others, among whom is
Apollodorus the calculator (0 \oyum«K), say it was Pythagoras ``(sc. who
``inscribed the right-angled triangle in a circle ``). Now Pamphile lived in the
reign of Nero (a.d. 54 — 68) and therefore some 700 years after the birth of
Thales (about 640 B.C.). I do not know on what Max Schmidt bases his
statement {Kullurhistorische Beitrage tur Kcnntnis des grieehischm uttd romiuhen
AUertums, 190C, p.~31) that ``other, much older, sources name Pythagoras as
the discoverer of the said proposition,'' because nothing more seems to be
known of Apollodorus than what is stated here by Diogenes Laertius. But it
would at least appear that Apollodorus was only one of several authorities
who attributed the proposition to Pythagoras while Parrfphile is alone
mentioned as referring it to Thales. Again, the connexion of Pythagoras with
the investigation of the right-angled triangle makes it a priori more likely
that it would be he who would discover its relation to a semicircle. On
the whole, therefore, the attribution to Thales would seem to be more than
doubtful.

2. As regards Geminus' account of the three stages through which the
proof of the theorem of 1. 32 passed, we note, first, that it is certainly not
confirmed by Eudemus, who referred to the Pythagoreans the discovery of the
theorem that the sum of the angles of any triangle is equal to two right
angles and says nothing about any gradual stages by which it was proved.
Secondly, it must be admitted, I think, that in the evolution of the proof as
reconstructed by Hank el the middle stage is rather artificial and unnecessary,
since, once it is proved that any right-angled triangle has the sum of its angles
equal to two right angles, it is just as easy to pass at once to any scalene
triangle (which is decomposable into two unequal right-angled triangles) as to
the isosceles triangle made up of two congruent right-angled triangles. Thirdly,
as Heiberg has recently pointed out (Mathcmatisehes su Aristoteles, p.~ao), it
is quite possible that the statement of Geminus from beginning to end is
simply due to a misapprehension of a passage of Aristotle (Anal. Post. 1. 5,
74 a 25). Aristotle is illustrating his contention that a property is not
scientifically proved to belong to a class of things unless it is proved to belong
primarily (xpwrav) and generally (kuAEXm) to the whole of the class. His first
illustration relates to parallels making with a transversal angles on the same
side together equal to two right angles, and has been quoted above in the note
on 1. 27 (pp.~308 — 9). His second illustration refers to the transformation of
a proportion aliernando, which (he says) ``used at one time to be proved
separately ``for numbers, lines, solids, and times, although it admits of being
proved of all at once by one demonstration. The third illustration is: ``For
the same reason, even if one should prove (0O8' ok t« ti(g) with reference to
each (sort of) triangle, the equilateral, scalene and isosceles, separately, that
each has its angles equal to two right angles, either by one proof 01 by different
proofs, he does not yet know that the triangle, i.e.\ the triangle in general, has
its angles equal to two right angles, except in a sophistical sense, even though
there exists no triangle other than triangles of the kinds mentioned For he
knows it, not qua\ triangle, nor of tvtty triangle, except in a numerical sense
(not apiBy£v)\ he does not know it nationally (nwr tlot) of every triangle, even
though there be actually no triangle which he does not know.''

The difference between the phrase ``used at one time to be proved ``in
the second illustration and ``if any one should prove ``in the third appears to
indicate that, while the former referred to a historical fact, the latter does not;
the reference to a person who should prove the theorem of 1. 3* for the three
kinds of triangle separately, and then claim that he had proved it generally,
states a purely hypothetical case, a mere illustration. Vet, coming after the
historical fact stated in the preceding illustration, it might not unnaturally give
the impression, at first sight, that it was historical too.

On the whole, therefore, it would seem that we cannot safely go behind
the dictum of Eudemus that the discovery and proof of the theorem of i. 32
in all its generality were Pythagorean. This does not however preclude its
having been discovered by stages such as those above set out after Hankel
and Cantor. Nor need it be doubted that Thales and even his Egyptian
instructors had advanced some way on the same road, so far at all events as
to see that in an equilateral triangle, and in an isosceles right-angled triangle,
the sum of the angles is equal to two right angles.

The Pythagorean proof.

This proof, handed down by Eudemus (Proclus, p.~379,-2 — 15), is no less
elegant than that given by Euclid, and is a natural
development from the last figure in the recon-
structed argument of Hankel. It would be seen,
after the theory of parallels was added to geometry,
that the actual drawing of the perpendicular and
the complete rectangle on BC as base was
unnecessary, and that the parallel to BC through
A was all that was required.

Let ABC be a triangle, and through A draw DE parallel to BC. [1. 31]

Then, since BC, DE are parallel,
the alternate angles DAB, ABC are equal, [1. 39]

and so are the alternate angles EAC, ACB also.

Therefore the angles ABC, ACS are together equal to the angles
DAB, EAC.

Add to each the angle BAC;
therefore the sum of the angles ABC, ACB, BAC is equal to the sum of the
angles DAB, BAC, CAE, that is, to two right angles.

Euclid's proof pre- Euclidean.

The theorem of 1. 32 is Aristotle's favourite illustration when he wishes to
refer to some truth generally acknowledged, and so often does it occur that
it is often indicated by two or three words in themselves hardly intelligible,
e.g.\ to Suo-If opSatt (Anal. Post. 1. 24, 85 b 5) and tijnipjjfi Tcayri rpiytivip to !o

(\ibid~85 b 11).

One passage (MttapA. 1051 a 14) makes it clear, as Heiberg {op.~at.
p.~19) acutely observes, that in the proof as Aristotle knew it Euclid's
construction was used. ``Why does the triangle make up two right angles ?
Because the angles about one point are equal to two right angles. If then the
parallel to the side had been drawn up (civkto), the fact would at once have
been clear from merely looking at the figure.'' The words ``the angles about
one point ``would equally fit the Pythagorean construction, but ``drawn
upwards ``applied to the parallel to a side can only indicate Euclid's.

Attempts at proof independently of parallels.

The most indefatigable worker on these lines was Legendre, and a sketch
of his work has been given in the note on Postulate 5 above.

One other attempted proof needs to be mentioned here because it has
found much favour. I allude to

Thibaut's method.

This appeared in Thibaut's Grundriss der reinen Mathematik, Gottingen
{2 ed. 1809, 3 ed. 1818), and is to the following effect.

Suppose CB produced to D, and let BD (produced to any necessary extent
either way) revolve in one direction (say
clockwise) first about B into the position
BA, then about A into the position of AC
produced both ways, and lastly about C
into the position CB produced both ways.

The argument then is that the straight
line BD has revolved through the sum of
the three exterior angles of the triangle.
But, since it has at the end of the revolution

assumed a position in the same straight line with its original position, it must
have revolved through four wight angles.

Therefore the sum of the three exterior angles is equal to four right
angles;

from which it follows that the sum of the three angles of the triangle is equal
to two right angles.

But it is to be observed that the straight line BD revolves about different
points in it, so that there is translation combined with rotatory motion, and it
is necessary to assume as an axiom that the two motions are independent, and
therefore that the translation may be neglected.

Schumacher (letter to Gauss of 3 May, 1831) tried to represent the
rotatory motion graphically in a second figure as mere motion round a point;
but Gauss (letter of 17 May, 1831) pointed out in reply that he really
assumed, without proving it, a proposition to the effect that ``If two straight
tines (1) and (2) which cut one another make angles A , A'' with  straight
line (3) cutting both of them, and if a straight line {4) in the same plane is
likewise cut by (1) at an angle A', then {4) will be cut by (2) at the angle A''.
But this proposition not only needs proof, but we may say that it is, in
essence, the very proposition to be proved ``(see Engel and Stackel, Die
Theorit der ParalkUinien von Euktid it's auf Gauss, 1895, p.~230).

How easy it is to be deluded in this way is plainly shown by Proclus'
attempt on the same lines. He says (p.~384, 13 — 21) that the truth of the
theorem is borne in upon us by the help of ``common notions ``only. For,
if we conceive a straight line with two perpendiculars drawn to it at its ex-
tremities, and if we then suppose the perpendiculars to (revolve about
their feet and) approach one another, so as to form a triangle, we see that,
to the extent to which they converge, they diminish the right, angles which
they made with the straight line, so that the amount taken from the right
angles is also the amount added to the vertical angle of the triangle, and the
three angles are necessarily made equal to two right angles,'' But a moment's
reflection shows that, so far from being founded on mere ``common notions,''
the supposed proof assumes, to begin with, that, if the perpendiculars ap-
proach one another ever so little, they will then form a triangle immediately,
LO., it assumes Postulate 5 itself; and the fact about the vertical angle can only
be seen by means of the equality of the alternate angles exhibited by drawing
a perpendicular from the vertex of the triangle to the base, i.e.\ % parallel to
either of the original perpendiculars.

Extension to polygons.

The two important corollaries added to 1. 32 in Simson's edition are given
by Proclus; but Proclus' proof of the first is different from, and perhaps
somewhat simpler than, Simson's.

1. The turn of the interior angles of a convex rectilineal figure is equal to
twice as many right angles as the figure has sides,
less four.

For let one angular point A be joined to all
the other angular points with which it is not con-
nected already.

The figure is then divided into triangles, and
mere inspection shows

(t) that the number of triangles is two less
than the number of sides in the figure,

(3) that the sura of the angles of all the
triangles is equal to the sum of all the interior angles of the figure.

Since then the sum of the angles of each triangle is equal to two nght angles
the sum of the interior angles of the figure is equal to i(n — 2) right angles,
i.e.\ (zn - 4) right angles, where n is the number of sides in the figure.

3. The exterior angles of any convex rectilineal
figure an together equal to four right angles.

For the interior and exterior angles together are
equal to 27i right angles, where n is the number of sides.

And the interior angles are together equal to
(a«-4> right angles.

Therefore the exterior angles are together equal to
four right angles.

This last property is already quoted by Aristotle
as true of all rectilineal figures in two passages (Anal.
Post. 1. 24, 85 b 38 and 11. 17, 99 a 10).

Proposition 33.
The straight lines joining equal and parallel straight
lines {at the extremities which are) in the same directions
{respectively) are themselves also equal and parallel.

Let AB, CD be equal and parallel, and let the straight
5 lines AC, BD join them (at the extremities which are) in the
same directions (respectively);

I say that AC, BD are also equal and parallel.
Let BC be joined.
Then, since AB is parallel to CD,
10 and BC has fallen upon them,

the alternate angles ABC, BCD
are equal to one another. [1. 29]

And, since AB is equal to CD,
and BC is common,
15 the two sides AB, BC are equal to the two sides DC, CB;
and the angle ABC is equal to the angle BCD;

therefore the base AC is equal to the base BD,
and the triangle ABC is equal to the triangle DCB,
and the remaining angles will be equal to the remaining angles
20 respectively, namely those which the equal sides subtend; [1. 4]
therefore the angle ACB is equal to the angle CBD.
And, since the straight line BC falling on the two straight
lines AC, BD has made the alternate angles equal to one
another,
J S AC is parallel to BD. * [1. 27]

And it was also proved equal to it.
Therefore etc.\ Q. e. d.

1. joining. ..{at the extremities which are) In the same directions (respectively).
I have for clearness' sake inserted the words irv brackets though they are not in the original
Greek, which has ``joining., .in the same directions'' or ``on the same sides,'' ivl t4 o£ra pip/i
tT,fcuy*(i6v<rair The expression ``towards the same parts, 11 though usage has sanctioned it,
is perhaps not quite satisfactory.

ij. DC, CB and 18. DCB. The Greek has ``BC, CD ``and ``BCD'' in these places
respectively. Euclid is not always careful to write in corresponding order the Letters denoting
corresponding points in congruent figures. On the contrary, he evidently prefers the alpha-
betical order, and seems to disdain to alter it for the sake of beginners or others who might
be confused by it. In the case of angles alteration is perhaps unnecessary; but in the case
of triangles and pairs of corresponding sides I have ventured to alter the order to that which
the mathematician of to-day expects.

This proposition is, as Proclus says (p.~385, j), the connecting link between
the exposition of the theory of parallels and the investigation of parallelograms.
For, while it only speaks of equal and parallel straight lines connecting those
ends of equal and parallel straight lines which are in the same directions, it
gives, without expressing the fact, the construction or origin of the parallelogram,
so that in the next proposition Euclid is able to speak of ``parallelogram mic
areas'' without any further explanation.

Proposition 34.

In parallelogrammic areas the opposite sides and angles
are equal to one another, and the diameter bisects the areas.

Let ACDB be a parallelogrammic area, and BC its
diameter;

s I say that the opposite sides and angles of the parallelogram
ACDB are equal to one another, and the diameter BC
bisects it.

For, since AB is parallel to CD,
and the straight line BC has fallen
I0 upon them,

the alternate angles ABC, BCD
are equal to one another. [i. 29]

Again, since AC 'is parallel to BD,
and BC has fallen upon them,

15 the alternate angles ACB, CBD are equal to one

another. [1. 29]

Therefore ABC, DCB are two triangles having the two
angles ABC, BCA equal to the two angles DCB, CBD
respectively, and one side equal to one side, namely that
to adjoining the equal angles and common to both of them, BC;
therefore they will also have the remaining sides equal
to the remaining sides respectively, and the remaining angle
to the remaining angle; [1. 26]

therefore the side AB is equal to CD,
*5 and Cto BD,

and further the angle BA C is equal to the angle CDB.
And, since the angle ABC is equal to the angle BCD,
and the angle CBD to the angle ACB,

the whole angle ABD is equal to the whole angle A CD.

[CM 2]

30 And the angle BACvtas also proved equal to the angle CDB.
Therefore in parallelogram mic areas the opposite sides
and angles are equal to one another.

I say, next, that the diameter also bisects the areas.
For, since AB is equal to CD,
35 and BC is common,
the two sides AB, BC are equal to the two sides DC, CB
respectively;

and the angle ABC is equal to the angle BCD;
therefore the base AC is also equal to DB,
+0 and the triangle ABC is equal to the triangle DCB. [1. 4]
Therefore the diameter BC bisects the parallelogram
ACDB. Q. E. D.

i. It is to be observed that, when parallelograms hare to be mentioned for the first time,
Euclid calls them ``parallelogram mic areas 17 or, mare exactly, ``parallelogram'' areas
{rapa\\Tt6ypa/jifta wpfa). The meaning is simply areas bounded by parallel straight lines
whh the further limitation placed upon the term by Euclid that ojAyfottr-sidtd figures arc so
called, although of course there are certain regular polygons which have opposite sides
parallel, and which therefore might be said to be areas bounded by parallel straight lines, We
gather from Froclus {p.~303) that the word ''parallelogram M was first introduced by Euclid,
that its use was suggested by J- 33, and that the formation of the word raaX\i;Xypa>ior
(parallel- lined) was analogous to that of ib6trypap.im {straight- lined or rectilineal).

17, 18, +0. DCB and 36. DC, CB. The Greek has in these places ``BCD'' and
%% CD, EC respectively. cf.\ note on \prop{1}{33}, lines 15, 18.

After specifying the particular kinds of parallelograms (squares and rhombi)
in which the diagonals bisect the angles which they join, as well as the areas,
and those (rectangles and rhomboids) in which the diagonals do not bisect
the angles, Proclus proceeds {pp.~390 sqq.) to analyse this proposition with
reference to the distinction in Aristotle's Anal, Post. (1. 4, 5, 73 a 21 — 74 b 4)
between attributes which are only predicable of every individual thing (iitb
mrm) in a class and those which are true of it primarily {tovtou wpiorov) and
generally (ko$6X.ov), We are apt, says Aristotle, to mistake a proof Kara
warm for a proof tovtou irpwrou iraSoXov because it is either impossible to
find a higher generality to comprehend all the particulars of which the
predicate is true, or to find a name for it. {Part of this passage of Aristotle
has been quoted above in the note on 1. 32, pp.~319 — 320.)

Now, says Proclus, adapting Aristotle's distinction to theorems, the present
proposition exhibits the distinction between theorems which are general and
theorems which are not general. According to Proclus, the first part of
the proposition stating that the opposite sides and angles of a parallelogram
are equal is general because the property is only true of parallelograms; but
the second part which asserts that the diameter bisects the area is not general
because it does not include all the figures of which this property is true, e.g.
circles and ellipses. Indeed, says Proclus, the first attempts upon problems
seem usually to have been of this partial character (fupiKiirtpat), and generality
was only attained by degrees. Thus ``the ancients, after investigating the
fact that the diameter bisects an ellipse, a circle, and a parallelogram
respectively, proceeded to investigate what was common to these cases,''
though ``it is difficult to show what is common to an ellipse, a circle and a
parallelogram.''

I doubt whether the supposed distinction between the two parts of the
proposition, in point of ``generality,'' can be sustained. Proclus himself admits
that it is presupposed that the subject of the proposition is a quadrilateral,
because there are other figures (e.g.\ regular polygons of an even number of
sides) besides parallelograms which have their opposite sides and angles
equal; therefore the second part of the theorem is, in this respect, no more
general than the other, and, if we are entitled to the tacit limitation of the
theorem to quadrilaterals in one part, we are equally entitled to it in the other.

It would almost appear as though Proclus had drawn the distinction for
the mere purpose of alluding to investigations by Greek geometers on the
general subject of diameters of all sorts of figures; and it may have been these
which brought the subject to the point at which Apollonius could say in the
first definitions at the beginning of his Conics that ``In any bent line, such as
is in one plane, I give the name diameter to any straight line which, being
drawn from the bent line, bisects all the straight lines (chords) drawn in the
line parallel to any straight line.'' The term bent line (capuvAi; ypafi/*n')
includes, e.g.\ in Archimedes, not only curves, but any composite line made
up of straight lines and curves joined together in any manner. It is of course
clear that either diagonal of a parallelogram bisects all lines drawn within the
parallelogram parallel to the other diagonal.

An-Nairīzī gives after 1.31a neat construction for dividing a straight line
into any number of equal parts (ed. Curtze, p.~74, ed. Besthorn-Heiberg,
pp.~141 — 3) which requires only one measurement repeated, together with the
properties of parallel lines including 1. 33, 34. As 1. 33, 34 are assumed, I
place the problem here. The particular case taken is the problem of dividing
a straight line into tkret equal parts.

Let AB be the given straight line. Draw AC, BD at right angles to it
on opposite sides.

An-Nairīzī takes AC, BD of the same
length and then bisects AC at E and BD
at F. But of course it is even simpler to
measure AE, EC along one' perpendicular
equal and of any length, and BF, FD along
the other also equal and of the same length.

Join ED, CF meeting AB in G, H
respectively.

Then shall AG, GH, HB all be equal.

Draw HK parallel to AC, or at right
angles to AB.

Since now EC, FD are equal and parallel,
ED, CFaxe equal and parallel. [1. 33]

And HK was drawn parallel to AC.

Therefore ECHK is a parallelogram; whence KH is equal as well as
parallel to EC, and therefore to EA.

The triangles EAG, KHG have now two angles respectively equal and the
sides AE, HK equal

Thus the triangles are equal in all respects, and
AG is equal to GH.

Similarly the triangles KHG, FBH are equal in all respects, and
GH is equal to SB.

If now we wish to extend the problem to the case where AB is to be
divided into » parts, we have only to measure («-i) successive equal lengths
along AC and {n— 1) successive lengths, each equal to the others, along BD.
Then join the first point arrived at on AC to the last point on BD, the
second on AC to the last but one on BD, and so on; and the joining lines
cut AB in points dividing it into n equal parts.

Proposition 35.

Parallelograms -which are on the same base and in the
same parallels are equal to one another.

Let A BCD, EBCF be parallelograms on the same base
BC and in the same parallels AF, BC;
s I say that ABCD is equal to the parallelogram EBCF,
For, since ABCD is a parallelogram,

AD is equal to BC. [1. 34]

For the same reason also

EF is equal to BC,

10 so that AD is also equal to EF; \C. N. 1]

and DE is common;

therefore the whole AE is equal to the whole DF.

[C. M a]

But AB is also equal to DC; [t 34J

therefore the two sides EA, AB are equal to the two sides

is FD t DC respectively,

and the angle FDC is equal to

the angle EAB, the exterior to the

interior; [1. 39]

therefore the base EB is equal

30 to the base FC,

and the triangle EAB will be equal to the triangle FDC.

[tfl
Let DGE be subtracted from each;

therefore the trapezium ABGD which remains is equal to
the trapezium EGCF which remains, [C. N. 3]

»s Let the triangle l75C be added to each;

therefore the whole parallelogram A BCD is equal to the whole
parallelogram EBCF.\ [C. N. 2]

Therefore etc.

Q. E. D.

11. FDC. The text has ``DFC.''

11. Lei DOE be subtract ed. Euclid speaks of the triangle DGE without any
explanation that, in the cue which he takes (where AD, EF have no point in common),
BE, CD must meet at a point G between the two parallels. He allow* this to appear from
the figure simply.

Equality in a new sense.

It is important to observe that we are in this proposition introduced for
the first time to a new conception of equality between .figures. Hitherto we
have had equality in the sense of congruence only, as applied to straight lines,
angles, and even triangles (cf.\ t. 4). Now, without any explicit reference to
any change in the meaning of the term, figures are inferred to be equal which
are equal in area or in content but need not be of the same form. No
definition of equality is anywhere given by Euclid; we are left to infer its
meaning from the few axioms about ``equal things.'' It will be observed that
in the above proof the ``equality ``of two parallelograms on the same base
and between the same parallels is inferred by the successive steps (1) of
subtracting one and the same area (the triangle DGE) from two areas equal
in the sense of congruence (the triangles ABB, DFC), and inferring that the
remainders (the trapezia ABGD, EGCF) are ``equal''; (2) of adding one and
the same area {the triangle GBC) to each of the latter ``equal ``trapezia, and
inferring the equality of the respective sums (the two given parallelograms).

As is well known, Simson (after Clairaut) slightly altered the proof in order
to make it applicable to all the three possible cases. The alteration
substituted one step of subtracting congruent areas (the triangles AE3, DEC)
froth one and the same area (the trapezium ABCF) for the two steps above
shown of first subtracting and then adding a certain area.

While, in either case, nothing more is explicitly used than the axioms that,
if equals be added to equals, the wholes are equal and that, if equals be subtracted
from equals, the remainders are equal, there is the further tacit assumption that
it b indifferent to what part or from what part of the same or equal areas the
same or equal areas are added or subtracted. De Morgan observes that the
postulate ``an area taken from an area leaves the same area from whatever
part it may be taken ``is particularly important as the key to equality of non-
rectiiineal areas which could not be cut into coincidence geometrically.

Legendre introduced the word equivalent to express this wider sense of
equality, restricting the term equal to things equal in the sense of congruent;
and this distinction has been found convenient.

I do not think it necessary, nor have I the space, to give any account of
the recent developments of the theory of equivalence on new lines represented
by the researches of W. Bolyai, Duhamel, De Zolt, Stolz, Schur, Veronese,
Hilbert and others, and must refer the reader to Ugo Amaldi's article Sulla
teoria dell' cquivatenza in Quest ioni riguardanti k matematiche elementari, L
(Bologna, 1912), pp.~145 — 198, and to Max Simon, liber die Entwichlung der
Elcmcntar-geometrie im XIX. fahrhundert (Leiprig, 1906), pp.~115 — no, with
their full references to the literature of the subject. I may however refer to
the suggestive distinction of phraseology used by Hilbert (Grundlagen der
Geometrie, pp.~39, 40):

(ij ``Two polygons are called divisibly-equal (zertegungsgich) if they cap
be divided into a finite number of triangles which are congruent two and two.''

(z) ``Two polygons are called equal in content (inhaltsgletch) or of equal
content if it is possible to add divisibly-equal polygons to them in such a way
that the two combined polygons are divisibly-equal.''

(Amaldi suggests as alternatives for the terms in (1) and (2) the expressions
equivalent by sum and equivalent by difference respectively.)

From these definitions it follows that ``by combining divisibly-equal
polygons we again arrive at divisibly-equal polygons; and, if we subtract
divisibly-equal polygons from divisibly-equal polygons, the polygons remaining
are equal in content.''

rhe proposition also follows without difficulty that, ``if two polygons are
divisibly-equal to a third polygon, they are also divisibly-equal to one another;
and, if two polygons are equal in content to a third polygon, they are equal in
content to one another.''

The different cases.

As usual, Proclus (pp.~399—400), observing that Euclid has given only the
most difficult of the three possible cases, adds the other two with separate
proofs. In the case where E in the figure of the proposition falls between A
and D, he adds the congruent triangles ABE t DCF respectively to the
smaller trapezium EBCD, instead of subtracting them (as Simson does) from
the larger trapezium ABCF.

An ancient ``Budget of Paradoxes.''

Proclus observes (p.~396, 12 sqq.) that the present theorem and the
similar one relating to triangles are among the so-called paradoxical theorems
of mathematics, since the un in strutted might well regard it as impossible that
the area of the parallelograms should remain the same while the length of the
sides other than the base and the side opposite to it may increase indefinitely.
He adds that mathematicians had made a collection of such paradoxes, the
so-called treasury of paradoxes (b wapdSofos tottm) — cf.\ the similar expressions
Tojrot ivaXvoinytK (treasury of analysis) and tottos arpovonttu/ixvos — in the same
way as the Stoics with their illustrations (<*<rrtp ol dwb 15 %roai Iwl ™v
S«y/MiTw). It may be that this treasury of paradoxes was the work of
Erycinus quoted by Pappus (in. p.~107, 8) and mentioned above (note on
\prop{1}{21}, p.~ago).

Locus-theorems and loci in Greek geometry.

The proposition 1. 35 is, says Proclus (pp, 394 — 6), the first locus-theorem
tjojriKQy BiapriiLa) given by Euclid Accordingly it is in his note on this
proposition that Proclus gives us his view of the nature of a locus-theorem
and of the meaning of the word locus (toVot); and great importance attaches
to his words because he is one of the three writers (Pappus and Eutocius
being the two others) upon whom we have to rely for all that is known of the
Greek conception of geometrical loci.

Proclus' explanation (pp.~394, 15 — 395, 2} is as follows. ``I call those
(theorems) locus-theorems (tottiko) in which the same property is found to exist
on the whole of some locus (irpos oAui nrl Ton#), and (I call) a locus a position
of a line or a surface producing one and the same property (ypa/ifiv* 4 **''-
tfravtvK SiViv -suvowrav tv xal raifov (rv/nrriu/ui). For, of locus-theorems, some
are constructed on lines and others on surfaces (rwv yap ToiriKwy t piv 1<tti
Trpor ypajLfiLS truvKntifttva, rk Si itph r .rnaveiais). And, since 30me lines are

plane {tTtriSot) and others solid (on-tpW)— those being plane which are simply
conceived of in a plane (w in ivmihtf djrXi; jj raijo-ie), and those solid the
origin of which is revealed from some section of a solid figure, as the cylin-
drical helix and the conic lines (us 7-9 iruAivSpuriJi tAurot xai raw jcuwmSi'
ypafifiutt) — I should say ($a(i)v dv) further that, of locus-theorems on lines,
some give a plane locus and others a solid locus.''

Leaving out of sight for the moment the class of bet on surfaces, we find
that the distinction between plane and solid loci, or plant and solid lints, was
similarly understood by Eutocius, who says (Apollonius, ed. Heiberg, 11.
p.~184) that ``solid loci have obtained their name from the fact that the lines
used in the solution of problems regarding them have their origin in the
section of solids, for example the sections of the cone and several others.''
Similarly we gather from Pappus that plant loci were straight lines and circles,
and solid loci were conics. Thus he tells us (vii. p.~672, 20) that Aristaeus
wrote five books of Solid Loci ``supplementary to (literally, continuous withl
the conics''; and, though Hultsch brackets the passage (vn. p.~662, to — 15)
which says plainly that plant loci are straight lines and circles, while solid loci
are sections of cones, i.e.\ parabolas, ellipses and hyperbolas, we have the
exactly corresponding distinction drawn by Pappus (hi. p.~54, 7 — 16) between
plant and solid problems, plane problems being those solved by means of
straight lines and circumferences of circles, and solid problems those solved
by means of one or more of the sections of the cone. But, whereas Proclus
and Eutocius speak of other solid loci besides conics, there is nothing in
Pappus to support the wider application of the term. According to Pappus
(ill. p.~J4, 1 6— ai) problems which could not be solved by means of straight
lines, circles, or conics were linear (ypaji/uKa) because they used for their
construction lines having a more complicated and unnatural origin than those
mentioned, namely such curves as quadratrices, conchoids and cissoids.
Similarly, in the passage supposed to be interpolated, linear led are distin-
guished as those which are neither straight lines nor circles nor any of the
conic sections (vii. p.~66a, 13—15). Thus the classification given by Proclus
and Eutocius is less precise than that which we find in Pappus; and the
inclusion by Proclus of the cylindrical helix among solid loci, on the ground
that it arises from a section of a solid figure, would seem to be, in any case,
due to some misapprehension.

Comparing these passages and the hints in Pappus about lod on surfaces
(iwoi irpos brufravtfy.) with special reference to Euclid's two books under that
title, Heiberg concludes that lad on lines and loci an surfaces in Proclus'
explanation are loci which are lines and loci which are surfaces respectively.
Bui some qualification is necessary as regards Proclus' conception of loci on
lines, because he goes on to say (p.~395, 5), with reference to this proposition,
that, while the locus is a loots on lines and moreover plane, it is ``the whole
space between the parallels ``which is the locus of the various parallelograms
on the same base proved to be equal in area. Similarly, when he quotes
in. 21 about the equality of the angles in the same segment and in. 31 about
the right angle in a semicircle as cases where a circumference of a circle
takes the place of a straight line in a plane locus-theorem, he appears to
imply that it is the segment or semicircle as an area which is regarded as the
locus of an infinite number of triangles with the same base and equal vertical
angles, rather than that it is the drcumferetice which is the locus of the angular
points. Likewise he gives the equality of parallelograms inscribed in ``the
asymptotes and the hyperbola ``as an example of a solid locus-theorem, as if
the area included between the curve and its asymptotes was regarded as the
locus of the equal parallelograms. However this may be, it is clear that the
locus in the present proposition can only be either (r) a line-locus of a line,
not a point, or (2) an area-locus of an area, not a point or a line; and we
seem to be thus brought to another and different classification of loci
corresponding to that quoted by Pappus (TO. p.~660, 18 sqq.) from the pre-
liminary exposition given by Apollonius in his Plane tod. According to this,
loci in general are of three kinds: (1) ifaKTutoL, holding-in, in which sense
the locus of a point is a point, of a line a line, of a surface a surface, and of a
solid a solid, (2) u£oxai, moving along, a line being in this sense a locus of a
point, a surface of a line and a solid of a surface, {3) aeaorpoucot, where a
surface is a locus of a point and a solid of a line. Thus the locus in this
proposition, whether it is the space between the two parallels regarded as the
locus of the equal parallelograms, or the line parallel to the base regarded as
the locus of the sides opposite to the base, would seem to be of the first class
(<£««««); and, as Proclus takes the former view of it, a locus on lines is
apparently not merely a locus which is a line but a locus bounded by lines
also, the locus being plane in the particular case because it is bounded by
straight lines, or, in the case of in. 21, 31, by straight lines and circles, but
not by any higher curves.

Proclus notes lastly (p.~395, r3 — 21) that, according to Gem bus,
``Chrysippus likened locus-theorems to the ideas. For, as the ideas confine
the genesis of unlimited (particulars) within defined limits, so in such theorems
the unlimited (particular figures) are confined within defined places or loci
(tovdi). And it is this boundary which is the cause of the equality; for the
height of the parallels, which remains the same, while an infinite number of
parallelograms are conceived on the same base, is what makes them ail equal
to one another.''

Proposition 36.

Parallelograms which are on equal banes and in the same
parallels are equal to one another.

Let ABCD, EFGH be parallelograms which are on
equal bases BC, FG and in the same parallels AH, BG;

I say that the parallelogram ABCD is equal to EFGH.

For let BE, CH be joined.

Then, since BC is equal to FG
while FG is equal to EH,

BC is also equal to EH. [C. \prop{10}{1}]

But they are also parallel.

And EB, HC join them;
but straight lines joining equal and parallel straight lines (at
the extremities which are) in the same directions (respectively)
are equal and parallel. [1. 33]

Therefore EBCH is a parallelogram. [1. 34]

And it is equal to ABCD;
for it has the same base BC with it, and is in the same
parallels BC, AHmth it [1. 35]

For the same reason also EFGH is equal to the same
EBCH; [..35]

so that the parallelogram ABCD is also equal to EFGH.

[ax 1]

Therefore etc.

q. e, o.

Proposition 37.

Triangles which are on the same base and in the same
parallels are equal to one another.

Let ABC, DBC be triangles on the same base BC and
in the same parallels AD, BC;
5 I say that the triangle ABC is equal to the triangle DBC

Let AD be produced in both
directions to E, F; E a d

through B let BE be drawn parallel
to CA, [1. 31]

10 and through C let CF be drawn
parallel to BD. [1. 31]

Then each of the figures
EBCA, DBCF is a parallelogram;

and they are equal,
5 for they are on the same base BC and in the same
parallels BC, EF. [1. as]

Moreover the triangle ABC is half of the parallelogram
EBCA; for the diameter AB bisects it. [1. 34]

And the triangle DBC is half of the parallelogram DBCF;
to for the diameter DC bisects it [t 34]

[But the halves of equal things are equal to one another.]

Therefore the triangle ABC is equal to the triangle DBC.

Therefore etc.

Q. E. D.

Si. Here and in the next proposition t lei berg brackets the words ``But the halves of
equal things are equal to one another'' on the ground that, since the Common Notion
which asserted this fact was interpolated at a very early date (before the time of Theon),
it is probable that the words here were interpolated at the same time. cf.\ note above
(p.~J34) on the interpolated Common Notion.

There is a lacuna in the text of Proclus' notes to 1. 36 and 1. 37.
Apparently the end of the former and the beginning of the latter are missing,
the mss. and the editio princeps showing no separate note for i. 37 and no
lacuna, but going straight on without regard to sense. Proclus had evidently
remarked again in the missing passage that, in the case of both parallelograms
and triangles between the same parallels, the two sides which stretch from one
parallel to the other may increase in length to any extent, while the area
remains the same. Thus the perimeter in parallelograms or triangles is of
itself no criterion as to their area. Misconception on this subject was rife
among non-mathematicians; and Proclus {p.~403, 5 sqq.) tells us (1) of
describers of countries (xtapoypafoi) who drew conclusions cegarding the size
of cities from their perimeters, and (a) of certain members of communistic
societies in his own time who cheated their fellow members by giving them
land of greater perimeter but less area than they took themselves, so that, on
the one hand, they got a reputation for greater honesty while, on the other, they
took more than their share of produce. Cantor (Gesck. d. Ma/A. i„ p, 172)
quotes several remarks of ancient authors which show the prevalence of the
same misconception. Thus Thucydides estimates the size of Sicily according
to the time required for circumnavigating it. About 130 B.C. Polybius said
that there were people who could not understand that camps of the same
periphery might have different capacities. Quintilian has a similar remark,
and Cantor thinks he may have had in his mind the calculations of Pliny, who
compares the size of different parts of the earth by adding their length to their
breadth.

The comparison however of the areas of different figures of equal contour
had not been neglected by mathematicians. Theon of Alexandria, in his
commentary on Book 1. of Ptolemy's Syntaxis, has preserved a number of
propositions on the subject taken from a treatise by Zenodorus mpl taofiirpw
(rxriiidtiiiv {reproduced in Latin on pp.~1190 — tan of Hultsch's edition of
Pappus) which was written at some date between, say, 200 b.c. -«.d 90 a.d.,
and probably not long after the former date, Pappus too has at the beginning
of Book v. of bis Collection {pp.~308 sqq.) the same propositions, in which he
appears to have followed Zenodorus pretty closely while making some changes
in detail. The propositions proved by Zenodorus and Pappus include the
following: (1) that, of all polygons of the same number of sides and equal
perimeter, the equilateral and equiangular polygon is the greatest in area,
{a) that, of regular polygons of equal perimeter, that is the greatest in area
which has the most angles, {3} that a circle is greater than any regular polygon
of equal contour, (4) that, of all circular segments in which the arcs are equal in
length, the semicircle is the greatest. The treatise of Zenodorus was not con-
fined to propositions about plane figures, hut gave also the theorem that, of
alt solid figures the surfaces of which are equal, the sphere is the greatest in
volume.

Proposition 38.

Triangles which are on equal bases and in the same
Parallels are equal to one another.

Let ABC, DEF be triangles on equal bases B C, EF and
in the same parallels BF, AD;

I say that the triangle ABC is q a_ d h

equal to the triangle DEF.

For let AD be produced in
both directions to G, H\
through B let BG be drawn
parallel to CA, (l 31]

and through F let FH be drawn parallel to DE.

Then each of the figures GBCA, DEFH is a parallelo-
gram;
and GBCA is equal to DEFH;
for they are on equal bases BC, EF and in the same
parallels BF, GH. [1. 36]

Moreover the triangle ABC is half of the parallelogram
GBCA; for the diameter AB bisects it. [i. 34]

And the triangle FED Is half of the parallelogram DEFH\
for the diameter DF bisects it, [i- 34]

[But the halves of equal things are equal to one another.]

Therefore the triangle ABC is equal to the triangle DEF.

Therefore etc.

Q. E. D.

On this proposition Proclus remarks (pp.~405 — 6) that Euclid seems to
him to have given in vi. 1 one proof including all the four theorems from
I- 35 to 1. 38, and that most people had failed to notice this. When Euclid,
he says, proves that triangles and parallelograms of the same altitude have to
one another the same ratio as their bases, he simply proves all these
propositions more generally by ( he use of proportion; for of course to be of
the same altitude is equivalent to being in the same parallels. It is true that
vi. 1 generalises these propositions, but it must be observed that it does not
prove the propositions themselves, as Proclus seems to imply; they ate in fact
assumed in order to prove \prop{6}{1}.

Comparison of areas of triangles of \prop{1}{34}.

The theorem already mentioned as given by Proclus on 1. 34 {pp.~340 — 4)
is placed here by Heron, who also enunciates it more clearly (an-Nairtzi, ed.
Besthom-Heiberg, pp.~155 — 161, ed. Curtze, pp.~75 — 8).

If in two triangles two sides of the one be equal to two sides of the other
respectively, and the angle of the one be greater than the angle of the other,
namely the angles contained by the equal sides, then, (t) if the sum of the tovo
angles contained by the equal sides is equal to two right angles, the two triangles
are equal to site another; (2) if less than two right angles, the triangle which
has the greater angle is also itself greater than the other; (3) if greater than two
right angles, the triangle which has the less angle is greater than the other
triangle.

D

Let two triangles ABC, DEF have the sides AB, AC respectively equal
to DE, DF.

(1) First, suppose that the angles at A and D in the triangles ABC,
DEF axe together equal to two right angles.

Heron's construction is now as follows.

Make the angle EDG equal to the angle BAC.

Draw /KT parallel to ED meeting DG in H.

lwa.EH.

Then, since the angles BAC, EDF are equal to two right angles, the
angles EDH, ED Pax*, equal to two right angles.

But so are the angles EDH, DHF,

Therefore the angles EDF, DHF axe equal.

And the alternate angles EDF, DFB are equal. [1. 39]

Therefore the angles DHF, DFHaxe equal,

and DF'v> equal to DH. [1. 6]

Hence the two sides ED, DHaxe equal to the two sides BA, AC\ and
the included angles are equal.

Therefore the triangles ABC, DEH axe equal in all respects.

And the triangles DEF, DEH between the same parallels are equal.

['• 37]

Therefore the triangles ABC, DEF axe equal.

[Proclus takes the construction of Eucl. 1. 24, i.e., he makes DH equal to
DF and then proves that ED, Fffare parallel.]

(2) Suppose the angles BAC, EDF together less than two right angles.

As before, make the angle EDG equal to the angle BAC, draw FH
parallel to ED, and join EH.

In this case the angles EDH, EDF are together less than two right
angles, while the angles EDH, DHF are equal to two right angles. [1. 39]

Hence the angle EDF, and therefore the angle DFH, is less than the
angle DHF.

Therefore DH is less than DF. [1. 19]

Produce DH to G so that DGis equal to Z>or AC, and join EG.

Then the triangle DEG, which is equal to the triangle ABC, is greater
than the triangle DEH, and therefore greater than the triangle DEF.

(3) Suppose the angles BAC, EDF together greater than two right
angles.

A J?

We make the same construction in this case, and we prove in like manner
that the angle DHF is less than the angle DFH,

whence DH is greater than DFot AC.

Make DG equal to A C, and join EG.

It then follows that the triangle DEF is greater than the triangle ABC.

[In the second and third cases again Proclus starts from the construction
in 1. 34, and proves, in the second case, that the parallel, FH, to ED cuts
DG and, in the third case, that it cuts DG produced]

There is no necessity for Heron to take account of the position of F in
relation to the side opposite D. For in the first and third cases F must fall

in the position in which Euclid draws it in 1. 24, whatever be the relative
lengths of A B, AC. In the second case the figure may be as annexed, but the
proof is the same, or rather the case needs no proof at all.

Proposition 39.

Equal triangles which are on the same base and on the
same side are also in the same parallels.

Let ABC, DBC be equal triangles which are on the same
base BC and on the same side of it;
s [I say that they are also in the same parallels.]
And [For] let AD be joined;
I say that AD is parallel to BC.

For, if not, let AE be drawn through
the point A parallel to the straight line
10 BC, [1. 31]

and let EC be joined.

Therefore the triangle ABC is equal
to the triangle EBC;

for it is on the same base BC with it and in the same
15 parallels. [1. 37]

But ABC is equal to DBC;
therefore DBC is also equal to EBC, [C. N. 1]

the greater to the less: which is impossible.
Therefore AE is not parallel to BC.
«> Similarly we can prove that neither is any other straight
line except AD;

therefore AD is parallel to BC.
Therefore etc.

Q. E. D.

5. [I say that they are also in the same parallels.] Heiberg has proved \Heruies t
xxxviji., [993, p.~50} from a recently discovered papyrus-fragment [Fayiim towns and their
papyri, p.~go, No. IX.) that these words are an interpolation by some one who did not observe
that the words ``And let AD be joined ``are part of the stlting-eut (IxStnt), but took them
as belonging to the construction (KaTaSKtirff) and consequently thought that a Sioptrfioi or
``definition {of the thing to be proved) should precede* The interpolator then altered
11 And ``into ``For ``in the next sentence.

This theorem is of course the partial converse of t. 37. In 1. 37 we have
triangles which are (1) on the same base, {2) in the same parallels, and the
theorem proves {3) that the triangles are equal. Here the hypothesis (1) and
the conclusion (3) are combined as hypotheses, and the conclusion is the
hypothesis (1) of 1. 37, that the triangles are in the same parallels. The
additional qualification in this proposition that the triangles must be on the
same side of the base is necessary because it is not, as in 1. 37, involved in the
other hypotheses.

Proclus (p.~407, 4 — 17) remarks that Euclid only converts \prop{1}{37} and 1. 38
relative to triangles, and omits the converses of 1. 35, 36 about parallelograms
as unnecessary because it is easy to see that the method would be the same,
and therefore the reader may properly be left to prove them for himself.

The proof is, as Proclus points out (p.~408, 5 — 21), equally easy on the
supposition that the assumed parallel AE meets BD or CD produced
beyond £>.

[Proposition 40.

Equal triangles which are on equal bases and on the same
side are also in the same parallels.

Let ABC, CDE be equal triangles on equal bases BC,
CE and on the same side.

I say that they are also in the same parallels.

For let AD be joined;
I say that AD is parallel to BE. 

For, if not, let AF be drawn through f\ — pp

A parallel to BE [1. 31], and let FE be
joined.

Therefore the triangle ABC is equal b
to the triangle FCE;

for they are on equal bases BC, CE and in the same parallels
BE, AF. [1. 38]

But the triangle ABC is equal to the triangle DCS;
therefore the triangle DCE is also equal to the triangle
FCE, [C. N. 1]

the greater to the less: which is impossible.
Therefore AF is not parallel to BE.

Similarly we can prove that neither is any other straight
line except AD;

therefore AD is parallel to BE.
Therefore etc.

Q, E. D.]

Heiberg has proved by means of the papyrus-fragment mentioned in the
last note that this proposition is an interpolation by some one who thought
that there should be a proposition following 1. 39 and related to it in the same
way as 1. 38 is related to 1. 37, and 1. 36 to t. 35.

Proposition 41.

If a parallelogram have the same base milk a triangle and
be in the same parallels, the parallelogram is double of the
triangle.

For let the parallelogram ABCD have the same base BC
with the triangle BBC, and let it be in the same parallels
BC, AE;

I say that the parallelogram ABCD is double of the
triangle BEC.

For let AC be joined.

Then the triangle ABC is equal to
the triangle EBC;

for it is on the same base BC with it
and in the same parallels BC, AE.

['• 37]
But the parallelogram ABCD is double of the triangle
ABC;

for the diameter AC bisects it; [1. 34]

so that the parallelogram ABCD is also double of the triangle
EBC,

Therefore etc.

Q. E. D.

On this proposition Proctus {pp.~414, 15 — 415, 16), ``by way of practice''
(yiyiriKri'iK Iffica), considers the area of a trapezium (a quadrilateral with only
one pair of opposite sides parallel) in comparison with that of the triangles
in the same parallels and having the greater and less of the parallel sides of
the trapezium for bases respectively, and proves that the trapezium is less
than double of the former triangle and more than double of the latter.
He next (pp.~415, 22—416, 14) proves the proposition that,
If a triangle be formed by joining the middle point of either of the non-
parallel sides to the extremities of the opposite side, the area of the trapezium is
always double of that of the triangle.

Let ABCD be a trapezium in which AD, BC are the parallel sides, and
E the middle point of one of the non-parallel sides,
say DC.

Job EA, EB and produce BE to meet AD
produced in F.

Then the triangles BEC, FED have two angles
equal respectively, and one side CE equal to one
side DE;

therefore the triangles are equal in atl respects.

Add to each the quadrilateral ABED;
therefore the trapezium ABCD is equal to the triangle ASF,

that is, to twice the triangle AEB, since BE is equal to EF. [l. 38]

The three properties proved by Proclus may be combined in one enuncia-
tion thus:

If a triangle be formed by joining the middle point of one side of a trapezium
to the extremities of the opposite side, the area of the trapezium is {1) greater
than, (1) equal to, or (3) less than, double the area of the triangle according as
the side the middle point of which is taken is ( 1 ) the greater of the parallel sides,
(a) either of the non-parallel sides, or (3) the lesser of the parallel sides.

Proposition 42.

To construct, in a given rectilineal angle, a parallelogram
equal to a given triangle.

Let ABC be the given triangle, and D the given recti-
lineal angle;

thus it is required to construct in the rectilineal angle D a
parallelogram equal to the
triangle ABC.

Let BC be bisected at E,
and let AE be joined;
on the straight line EC, and
at the point E on it, let the
angle CEF be constructed
equal to the angle D; [1. 23]

through A let AG be drawn parallel to EC, and [1. 31]

through C let CG be drawn parallel to EF,

Then FECG is a parallelogram.

And, since BE is equal to EC,

the triangle ABE is also equal to the triangle AEC,
for they are on equal bases BE, EC and in the same parallels
BC,AG; [..38]

therefore the triangle ABC is double of the triangle
AEC.

But the parallelogram FECG is also double of the triangle
A EC, for it has the same base with it and is in the same
parallels with it; [1. 41]

therefore the parallelogram FECG is equal to the
triangle ABC.
And it has the angle CEF equal to the given angle D.
Therefore the parallelogram FECG has been constructed
equal to the given triangle ABC, in the angle CEF which is
equal to D. Q. E. F.

Proposition 43.

In any parallelogram tke complements of the parallelograms
about tke diameter are equal to one another.

Let ABCD be a parallelogram, and AC its diameter;
and about AC let EH, FG be parallelograms, and BK, KD
S the so-called complements;

I say that the complement BK is equal to the complement
KD.

For, since ABCD is a parallelogram, and AC its diameter,
the triangle ABC is equal to
to the triangle A CD. [1. 34]

Again, since EH is a parallelo-
gram, and AK is its diameter,

the triangle AEK is equal to
the triangle AHK.
15 For the same reason

the triangle KFC is also equal to KGC.
Now, since the triangle AEK is equal to the triangle
AHK,

and KFC to KGC,
*o the triangle AEK together with KGC is equal to the triangle
AHK together with KFC. [C. M 1]

And the whole triangle ABC is also equal to the whole
ADC;

therefore the complement BK which remains is equal to the
a j complement KD which remains. [C. N. 3]

Therefore etc.

Q. E. D.

1. complement*, iraparXi)pv>umi, the figures put in to fill up (interstices).

4. and About AC..., Euclid's phraseology here and in the next proposition implies
that the complements as well at the other parallelograms are ``about ``the diagonal. Toe
words are here rtpl Si iiy/ AT wapaWTjXfr'pawui \ikv fara t E9, ZH, ri Si XcyttA
TapaT\ypJittaTa ra BK, KA. The expression ``the so-called complements'' indicates that
this technical use of xopaiXijjaTo was not new, though it might not be universally known.

In the text of Proclus' commentary as we have it, the end of the note on
i. 41, the whole of that on 1. 4a, and the beginning of that on 1. 43 are
missing.

Proclus remarks (p.~418, 15 — 30) that Euclid did not need to give a
formal definition of complement because the name was simply suggested by the
facts; when once we have the two ``parallelograms about the diameter,''
the complements are necessarily the areas remain-
ing over on each side of the diameter, which fill
up the complete parallelogram. Thus (p.~417,
1 sqq.) the complements need not be parallelo-
grams. They are so if the two ``parallelograms
about the diameter'' are formed by straight lines
drawn through one point of the diameter parallel
to the sides of the original parallelogram, but not
otherwise. If, as in the first of the accompanying figures, the parallelograms
have no common point, the complements are five-sided figures as shown.
When the parallelograms overlap, as in the second figure, Proclus regards
the complements as being the small parallelo-
grams FG, EH But, if complements are strictly
the areas required to fill up the original parallelo-
gram, Proclus is inaccurate in describing FG, EH
as the complements. The complements are really
(1) the parallelogram FG minus the triangle LM N,
and (2) the parallelogram EH minus the triangle
KMN, respectively; the possibility that the re-
spective differences may be negative merely means the possibility that the
sum of the two parallelograms about the diameter may be together greater
than the original parallelogram.

In all the cases it is easy to show, as Proclus does, that the complements
are still equal.

Proposition 44.

To a given straight line to apply, in a given rectilineal
angle, a parallelogram equal to a given triangle.

Let AB be the given straight line, C the given triangle
and D the given rectilineal angle;
5 thus it is required to apply to the given straight line AB, in
an angle equal to the angle D, a parallelogram equal to the
given triangle C.

Let the parallelogram BEFG be constructed equal to

the triangle C, in the angle EBG which is equal to D [1. 42];

10 let it be placed so that BE is in a straight line with AB; let

FG be drawn through to H, and let AH he drawn through
A parallel to either BG or EF. [j. 3*]

Let HB be joined.

Then, since the straight line HF falls upon the parallels
is AH, EF,

the angles AHF, HFE are equal to two right angles.

[t 39]
Therefore the angles BHG, GFE are less than two right
angles;

and straight lines produced indefinitely from angles less than
*> two right angles meet; [Post. 5]

therefore HB, FE, when produced, wilt meet.
Let them be produced and meet at K; through the point
K let KL be drawn parallel to either EA or FH, [1. 31]

and let HA, GB be produced to the points L, M.
n Then HLKF is a parallelogram,
HK is its diameter, and AG, ME are parallelograms, and
LB, BF the so-called complements, about HK;

therefore LB is equal to BF. [1. 43]

But BF is equal to the triangle C;
3° therefore LB is also equal to C. [C. N. 1]

And, since the angle GBE is equal to the angle ABM,

[1 'Si
while the angle GBE is equal to D,

the angle ABM is also equal to the angle D.
Therefore the parallelogram LB equal to the given triangle
35 C has been applied to the given straight line AB, in the angle
ABM which is equal to D.

Q. E. F.

14, since the straight line HP falls.... The verb is in the wrist (triitviy) here ftnd
in similar expressions in the following propositions.

This proposition wilt always remain one of the most impressive in all
geometry when account is taken (i) of the great importance of the result

obtained, the transformation of a parallelogram of any shape into another
with the same angle and of equal area but with one side of any given
length, e.g.\ a unit length, and (a) of the simplicity of the means employed,
namely the mere application of the property that the complements of the
``parallelograms about the diameter'' of a parallelogram are equal. The
marvellous ingenuity of the solution is indeed worthy of the ``godlike men of
old,'' as Proclus calls the discoverers of the method of ``application of areas'';
and there would seem to be no reason to doubt that the particular solution,
like the whole theory, was Pythagorean, and not a new solution due to Euclid
himself.

Application of areas.

On this proposition Proclus gives (pp.~419, 15 — 420, 23) a valuable note
on the method of ``application of areas ``here introduced, which was one of
the most powerful methods on which Greek geometry relied. The note runs
as follows:

``These things, says Eudemus (ol irrpl rov Evwwe), are ancient and are
discoveries of the Muse of the Pythagoreans, I mean the application of anas
(rapaflokij twi> x M P''' 8, ')i their exceeding (inrtpfjoXij) and their falling-short
(£AA««). It was from the Pythagoreans that later geometers [i.e.\ Apollonius]
took the names, which they again transferred to the so-called conk lines,
designating one of these a parabola (application), another a hyperbola
(exceeding) and another an ellipse (falling-short), whereas those godlike men
of old saw the things signified by these names in the construction, in a plane,
of areas upon a finite straight line. For, when you have a straight line set
out and lay the given area exactly alongside the whole of the straight line, then
they say that you apply (irupailaXkur) the said area; when however you
make the length of the area greater than the straight line itself, it is said to
exceed (vrcpfiukkuv), and when you make it less, in which case, after the area
has been drawn, there is some part of the straight line extending beyond it,
it is said to fall short (MAcia-w). Euclid too, in the sixth book, speaks in
this way both of exceeding and falling-short; but in this place he needed the
application simply, as he sought to apply to a given straight line an area equal
to a given triangle in order that we might have in our power, not only the
construction (crwmurw) of a parallelogram equal to a given triangle, but also
the application of it to a finite straight line. For example, given a triangle
with an area of 12 feet, and a straight line set out the length of which is
4 feet, we apply to the straight line the area equal 10 the triangle if we take
the whole length of 4 feet and find how many feet the breadth must be in
order that the parallelogram may be equal to the triangle. In the particular
case, if we find a breadth of 3 feet and multiply the length into the breadth,
supposing that the angle set out is a right angle, we shall have the area. Such
then is the application handed down from early times by the Pythagoreans.''

Other passages to a similar effect are quoted from Plutarch, (1) ``Pytha-
goras sacrificed an ox on the strength of his proposition (Siaypa/i/u) as
Apollodotus (?-rus) says... whether it was the theorem of the hypotenuse, viz.
that the square on it is equal to the squares on the sides containing the
right angle, or the problem about the application of an area'' (JVon posse
suauiter vivi secundum Epieurum, c, n.) (2) ``Among the most geometrical
theorems, or rather problems, is the following: given two figures; to apply a
third equal to the one and similar to the other, on the strength of which
discovery they say moreover that Pythagoras sacrificed. This is indeed
unquestionably more subtle and more scientific than the theorem which
demonstrated that the square on the hypotenuse is equal to the squares on
the sides about the right angle ``{Symp.~vm. 2,4).

The story of the sacrifice must (as noted by Bretschneider and Hankel)
be given up as inconsistent with Pythagorean ritual, which forbade such
sacrifices; but there is no reason to doubt that the first distinct formulation
and introduction into Greek geometry of the method of application of areas
was due to the Pythagoreans. The complete exposition of the application of
areas, their exceeding and their falling-short, and of the construction of a
rectilineal figure equal to one given figure and similar to another, takes us
into the sixth Book of Euclid; but it will be convenient to note here the
general features of the theory of application, exceeding and falling-short.

The simple application of a parallelogram of given area to a given
straight line as one of its sides is what we have in 1. 44 and 45; the general
form of the problem with regard to exceeding and falling-short may be stated
thus:

``To apply to a given straight line a rectangle (or, more generally, a
parallelogram) equal to a given rectilineal figure and (i) exceeding or
(2) falling-short by a square {or, in the more general case, a parallelogram
similar to a given parallelogram).''

What is meant by saying that the applied parallelogram (1) exceeds or
(2) falls short is that, while its base coincides and is coterminous at one end
with the straight line, the said base (1) overlaps or (2) falls short of the
straight line at the other end, and the portion by which the applied
parallelogram exceeds a parallelogram of the same angle and height on the
given straight line (exactly) as base is a parallelogram similar to a given
parallelogram (or, in particular cases, a square). In the case where the
parallelogram is to fall short, a Siofmr/to's is necessary to express the condition
of possibility of solution.

We shall have occasion to see, when we come to the relative propositions
in the second and sixth Books, that the general problem here stated is
equivalent to that of solving geometrically a mixed quadratic equation. We
shall see that, even by means of 11. 5 and 6, we can solve geometrically the
equations

ax± x? = P,
—ax — P;

but in vi. 28, 29 Euclid gives the equivalent of the solution of the general
equations

* _, C

— e m

We are now in a position to understand the application of the terms

parabola (application), hyperbola (exceeding) and ellipse (falling-short) to

conic sections. These names were first so applied by A polio ni us as expressing

in each case the fundamental property of the curves as stated by him. This

fundamental property is the geometrical equivalent of the Cartesian equation

referred to any diameter of the conic and the tangent at its extremity as (in

general, oblique) axes. If the parameter of the ordinates from the several

points of the conic drawn to the given diameter be denoted by p (p being

d'' 1
accordingly, in the case of the hyperbola and ellipse, equal to -3 , where d is

the length of the given diameter and d' that of its conjugate), Apollonius gives
the properties of the three conics in the following form.

(1) For the parabola, the square on the ordinate at any point is equal to
a rectangle applied to / as base with altitude equal to the corresponding
abscissa. That is to say, with the usual notation,

(3) For the hyperbola and ellipse, the square on the ordinate is equal to
the rectangle applied to p having as its width the abscissa and exceeding (for
the hyperbola) or falling-short (for the ellipse) by a figure similar and similarly
situated to the rectangle contained by the given diameter and p.

That is, in the hyperbola jr =px + -3 pd,
or y* -px +  *'' i

and in the ellipse y'=px — -.x*.

d

The form of these equations will be seen to be exactly the same as that of
the general equations above given, and thus Apollonius' nomenclature followed
exactly the traditional theory of application, exceeding, a.nA.faliing*short.

Proposition 45.

To construct, in a given rectilineal angle, a parallelogram
equal to a given rectilineal figure.

Let ABCD be the given rectilineal figure and E the given
rectilineal angle;
5 thus it is required to construct, in the given angle E, a
parallelogram equal to the rectilineal figure ABCD.

- - 

Let DB be joined, and let the parallelogram FH be
constructed equal to the triangle ABD, in the angle HKF
which is equal to E; [1. 4a]

10 let the parallelogram GM equal to the triangle DBC be
applied to the straight line GH, in the angle GHM which is
equal to /:'. [1. 44]

Then, since the angle E is equal to each of the angles
HKF, GHM,
i S the angle HKF is also equal to the angle GHM. [C. N. 1]

Let the angle KHG be added to each;
therefore the angles FKH, KHG are equal to the angles
KHG, GHM,

But the angles FKH, KHG are equal to two right angles;

ft. 3 9 ]

*> therefore the angles KHG, GHM are also equal to two right
angles.

Thus, with a straight line GH, and at the point H on it,

two straight lines KH, HM not lying on the same side make

the adjacent angles equal to two right angles;

as therefore KH is in a straight line with HM. [i. 14]

And, since the straight line HG falls upon the parallels

KM, FG, the alternate angles MHG, HGF are equal to one

another. [1, 39]

Let the angle HGL be added to each;

jo therefore the angles MHG, HGL are equal to the angles

HGF, HGL. [C N. 2]

But the angies MHG, HGL are equal to two right angles;

% *9]

therefore the angles HGF, HGL are also equal to two right

angles. [C. N. 1]

35 Therefore FG-is in a straight line with GL. [t. 14]

And, since FK is equal and parallel to HG, [1. 34]

and HG to ML also,

KF is also equal and parallel to ML; [C. N. 1; 1. 30]

and the straight lines KM, FL join them (at their extremities);

40 therefore KM, FL are also equal and parallel. [1. 33]

Therefore KFLM is a parallelogram.

And, since the triangle ABD is equal to the parallelogram

FM > and DBC to GM,

45 the whole rectilineal figure ABCD is equal to the whole
parallelogram KFLM.

Therefore the parallelogram KFLM has been constructed
equal to the given rectilineal figure ABCD, in the angle FKM
which is equal to the given angle E. q. e. f.

?, 3, £, 45. 48- rectilineal figure, in the Greek ``rectilineal'' simply, without ``figure,''
fitit-ypaixo* being here used as a substantive, like the similarly formed rapdXXifiypafitttim.

Transformation of areas.

We can now take stock of how far the propositions 1. 43 — 45 bring us in
the matter of transformation 0/ areas, which constitutes so important a part of
what has been fitly called the geometrical algebra of the Greeks. We have
now learnt how to represent any rectilineal area, which can of course be
resolved into triangles, by a single parallelogram having one side equal to any
given straight line and one angle equal to any given rectilineal angle. Most
important of all such parallelograms is the rectangle, which is one of the simplest
forms in which an area can be shown. Since a rectangle corresponds to the
product of two magnitudes in algebra, we see that application to a given
straight line of a rectangle equal to a given area is the geometrical equivalent
of algebraical division of the product of two quantities by a third. Further
than this, it enables us to add or subtract any rectilineal areas and to represent
the sum or difference by one rectangle with one side of any given length, the
process being the equivalent of obtaining a common factor. But one step
still remains, the finding of a square equal to a given rectangle, i.e.\ to a
given rectilineal figure; and this step is not taken till 11. 14. In general,
the transformation of combinations of rectangles and squares into other
combinations of rectangles and squares is the subject-matter of Book 11., with
the exception of the expression of the sum of two squares as a single square
which appears earlier in the other Pythagorean theorem 1. 47. Thus the
transformation of rectilineal areas is made complete in one direction, i.e.\ in the
direction of their simplest expression in terms of rectangles and squares, by the
end of Book 11. The reverse process of transforming the simpler rectangular
area into an equal area which shall be similar to any rectilineal figure requires,
of course, the use of proportions, and therefore does not appear till vi. 25.

Proclus adds to his note on this proposition the remark (pp.~411, 24 —
413, 6): ``I conceive that it was in consequence of this problem that the
ancient geometers were ted to investigate the squaring of the circle as well.
For, if a parallelogram can be found equal to any rectilineal figure, it is worth
inquiring whether it be not also possibte to prove rectilineal figures equal to
circular. And Archimedes actually proved that any circle is equal to the
right-angled triangle which has one of its sides about the right angle [the
perpendicular] equal to the radius of the circle and its base equal to the
pen meter of the circle. But of this elsewhere.''

Proposition 46.

On a given straight line to describe a square.

Let AB be the given straight line;

thus it is required to describe a square

on the straight line AB.

5 Let AC be drawn at right angles to

the straight line AB from the point A

on it [j. 11], and let AD be made equal

through the point D let DE be drawn
10 parallel to AB,

and through the point B let BE be drawn parallel to AD.

[1. 31]

Therefore ADEB is a parallelogram;

therefore AB is equal to DE, and AD to BE. [1. 34]
But AB is equal to AD;
•5 therefore the four straight lines BA, AD, DE, EB

are equal to one another;
therefore the parallelogram ADEB is equilateral.
I say next that it is also right-angled.
For, since the straight line AD falls upon the parallels
*>AB, DE,

the angles BAD, ADE are equal to two right angles.

[1. a 9 ]
But the angle BAD is right;

therefore the angle ADE is also right.
And in parallelogrammic areas the opposite sides and
as angles are equal to one another; [1. 34]

therefore each of the opposite angles ABE, BED is also
right

Therefore ADEB is right-angled.
And it was also proved equilateral.
30 Therefore it is a square; and it is described on the straight
line AB.

Q. E. F.

i (p.~4 13, iSsqq.) notes the difference between the ward constntet
I by Euclid to the construction of a triangle (and! he might have added,

I, 3, 30. Proclus
{ffveripatdiu) ap

of an angle) and the words describe en {dvtypiipt .r iri) used of drawing » square on a given
straight line as one side. The triangle (or angk) is, so to say, pieced together, while the
describing of a square on a given straight line is the making of a figure ``from ``one side,
and corresponds to the multiplication of the number representing the side by itself.

Proclus (pp.~424 — s) proves that, if squares are described on equal straight
lines, the squares are equal; and, conversely that,
if two squares are equal, the straight lines are
equal on which they are described. The first
proposition is immediately obvious if we divide
tbe squares into two triangles by drawing a
diagonal in each. The converse is proved as
follows.

Place the two equal squares AF, CG so
that AB, BC are in a straight line. Then,
since the angles are right, FB, BG will also
be in a straight line. Join AF, FC, CG, GA.

Now, since the squares are equal, the
triangles ABF, CBG are equal.

Add to each the triangle FBC; therefore the triangles AFC, GFC are
equal, and hence they must be in the same parallels.

Therefore AG, CFaie parallel.

Also, since each of the alternate angles AFG, FGC is half a right angle,
AF, CG are parallel.

Hence AFCG is a parallelogram; and AF, CG are equal.

Thus the triangles ABF, CBG have two angles and one side respectively
equal;
therefore AB is equal to BC, and BF to BG.

Proposition 47.

In right-angled triangles the square on the side subtending
the right angle is equal to the squares on the sides containing
the right angle.

Let ABC be a right-angled triangle having the angle
%BAC right;

I say that the square on BC is equal to the squares on
BA, AC.

For let there be described
on BC the square BDEC,
10 and on BA, AC the squares
GB,HC\ [1.46]

through A let AL be drawn
parallel to either BD or CE,
and let AD, FC be joined.
is Then, since each of the
angles BAC, BAG is right,
it follows that with a straight
line BA, and at the point A
on it, the two straight lines
20 AC, AG not lying on the
same side make the adjacent
angles equal to two right
angles;

therefore CA is in a straight line with AG.
For the same reason

BA is also in a straight line with AH.
And, since the angle DBC is equal to the angle FBA: for
each is right:

let the angle ABC be added to each;
jo therefore the whole angle DBA is equal to the whole
angle FBC. [C N. *]

And, since DB is equal to BC, and FB to BA,
the two sides AB, BD are equal to the two sides FB, BC
respectively ,
35 and the angle ABD is equal to the angle FBC;
therefore the base AD is equal to the base FC,
and the triangle ABD is equal to the triangle FBC. [i. 4]
Now the parallelogram BL is double of the triangle ABD,
for they have the same base BD and are in the same parallels
4° BD, AL, [1. 4 i]

And the square GB is double of the triangle FBC,
for they again have the same base FB and are in the same
parallels FB, GC. [1. 41]

[But the doubles of equals are equal to one another,]
45 Therefore the parallelogram BL is also equal to the

square GB.
Similarly, if AE, BK be joined,
the parallelogram CL can also be proved equal to the square
HC;
jo therefore the whole square BDEC is equal to the two

squares GB, HC. [C. N. a]

And the square BDEC is described on BC,

and the squares GB, HC on BA, AC.
Therefore the square on the side BC is equal to the
55 squares on the sides BA, AC.

Therefore etc.\ Q. E. D.

1. the square on, rb farb...TtTpy<onir l the word droypaQir or AnytypattfiJrott being
understood.

subtending the right angle. Here trtTtuiiiiinn, ``subtending,'' is used with the
simple accusative (Hi* bpSIp ywrtor) instead of being followed by rb and the accusative,
which seems to be the original and more orthodox construction. CI. I. i St, note.

33. the two sides AB, BD.... Euclid actually writes ``DB, BA,'' and therefore the
equal sides in the two triangles are not mentioned in corresponding order, though be adheres
to the words ttaripa Ixwtpa Ll respectively.'' Here DB is equal to BC and BA to FB.

44. [But the doubles of equals are equal to one another.] Heiberg brackets
these words as an interpolation, since it quotes a Common Nsiion which is itself interpolated.
Cf, notes on 1. 37, p.~337, and on interpolated Common Notions, pp.~313 — 4.

*' If we listen,'' says Proclus {p.~426, 6 sqq.), ``to those who wish to
recount ancient history, we may find some of them referring this theorem to
Pythagoras and saying that he sacrificed an ox in honour of his discovery.
But for my part, while I admire those who first observed the truth of this
theorem, I marvel more at the writer of the Elements, not only because he
made it fast («a.T($i}o-aTo) by a most lucid demonstration, but because he
compelled assent to the still more general theorem by the irrefragable
arguments of science in the sixth Book. For in that Book he proves
generally that, in right-angled triangles, the figure on the side subtending
the right angle is equal to the similar and similarly situated figures described
on the sides about the right angle.''

In addition, Plutarch (in the passages quoted above in the note on 1. 44),
Diogenes Laertius (yin. iz) and Athenaeus (x. 13) agree in attributing this
proposition to Pythagoras. It is easy to point out, as does G. Junge (``Wann
haben die Griechen das Irrationale entdeckt?'' in Novae Symbolae Jbachimicac,
Halle a. S., 1907, pp.~221—264), that these are late witnesses, and that the
Greek literature which we possess belonging to the first five centuries after
Pythagoras contains no statement specifying this or any other particular great
geometrical discovery as due to him. Yet the distich of Apollodorus the
``calculator,'' whose date (though it cannot be fixed) is at least earlier than
that of Plutarch and presumably of Cicero, is quite definite as to the existence
of one ``famous proposition ``discovered by Pythagoras, whatever it was. Nor
does Cicero, in commenting apparently on the verses IDe nat. dear, 111. c. 36,
§ 88), seem to dispute the fact of the geometrical discovery, but only the story
of the sacrifice, junge naturally emphasises the apparent uncertainty in the
statements of Plutarch and Proclus. But, as I read the passages of Plutarch,
I see nothing in them inconsistent with the supposition that Plutarch un-
hesitatingly accepted as discoveries of Pythagoras both the thtorem of the
square of the hypotenuse and the problem of the application of an area, and
the only doubt he felt was as to which of the two discoveries was the mqre
appropriate occasion for the supposed sacrifice. There is also other evidence
not without bearing on the question. The theorem is closely connected with
the whole of the matter of Eucl. Book 11., in which one of the most prominent
features is the use of the gnonwn. Now the gnomon was a well-understood
term with the Pythagoreans (cf.\ the fragment of Philolaus quoted on p.~141 of
Boeckh's Philolaos des PytAagorecrs Lehren, 1819). Aristotle also {Physics
111. 4, 203 a 10 — 15) clearly attributes to the Pythagoreans the placing of odd
numbers as gnomons round successive squares beginning with 1, thereby
forming new squares, while in another place (Caieg. 14, 15 a 30) the word
gnomon occurs in the same (obviously familiar) sense: ``e.g.\ a square, when a
gnomon is placed round it, is increased in size but is not altered in form.''
The inference must therefore be that practically the whole doctrine of Book it.
is Pythagorean. Again Heron (? 3rd cent a.d.), like Proclus, credits Pythagoras
with a general rule for forming right-angled triangles with rational whole
numbers for sides. Lastly, the ``summary'' of Proclus appears to credit
Pythagoras with tbe discovery of the theory, or study, of irrationals Mp M*
aXoytav -wpayftaTtiav). But it is now more or less agreed that the reading here
should be, not tuv oXoywv, but tw dpoAoyuK, or rather t£v dva Xoyov (``of
proportionals''), and that the author intended to attribute to Pythagoras a
theory of proportion, i.e.\ the (arithmetical) theory of proportion applicable
only to commensurable magnitudes, as distinct from the theory of Eucl.
Book v., which was due to Eudoxus. It is not however disputed that the
Pythagoreans discovered the irrational {cf.\ the scholium No. 1 to Book x. ).
Now everything goes to show that this discovery of the irrational was made
with reference to J 2, the ratio of the diagonal of a square to its side. It is
clear that this presupposes the knowledge that \prop{1}{47} is true of an isosceles
right-angled triangle; and the fact that some triangles of which it had been
discovered to be true were rational right-angled triangles was doubtless
what suggested the inquiry whether the ratio between the lengths of the
diagonal and the side of a square could also be expressed in whole numbers.
On the whole, therefore, I see no sufficient reason to question the tradition
that, so far as Greek geometry is concerned (the possible priority of the
discovery of the same proposition in India will be considered later), Pythagoras
was the first to introduce the theorem of i. 47 and to give a general proof
of it.

On this assumption, how was Pythagoras led to this discovery? It has
been suggested and commonly assumed that the Egyptians were aware that a
triangle with its sides in the ratio 3, 4, 5 was right-angled. Cantor inferred
this from the fact that this was precisely the triangle with which Pythagoras
began, if we may accept the testimony of Vitruvius {ix. 2) that Pythagoras
taught how to make a right angle by means of three lengths measured by the
numbers 3, 4, 5. If then he took from the Egyptians the triangle 3, 4, 5, he
presumably learnt its property from them also. Now the Egyptians must
certainly be credited from a period at least as far back as 2000 b.c. with the
knowledge that 4' + 3* = 5 ! . Cantor finds proof of this in a fragment of
papyrus belonging to the time of the izth Dynasty newly discovered at
Kahun. In this papyrus we have extractions of square roots: e.g.\ that of 16
is 4, that of i-fc is rj, that of 6J is 2$, and the following equations can be
traced:

'' + <!>' =(!«

8*+ 6'' =io'

l6 J + 12* = 20 1 .

It will be seen that 4 1 + 3* = 5* can be derived from each of these by
multiplying, or dividing out, by one and the same factor. We may therefore
admit that the Egyptians knew that 3* + 4*= 5'. But there seems to be no
evidence that they knew that the triangle {3, 4, 5) is right-angled; indeed,
according to the latest authority {T. Eric Peet, The Rhind Mathematical
Papyrus, 1923)1 nothing in Egyptian mathematics suggests that the Egyptians
were acquainted with this or any special cases of the Pythagorean theorem.

How then did Pythagoras discover the general theorem ? Observing that
3, 4, 5 was a right-angled triangle, while 3 + 4 1 = $', he was probably led to

consider whether a similar relation was true of the sides of right-angled
triangles other than the particular one. The simplest case (geometrically) to
investigate was that of the isosceles right-angled triangle; and the truth of the
theorem in this particular case would easily appear from the mere construction
of a figure. Cantor {i„ p.~185) and Allman {Greek Geometry from Thales to
Euclid, p, 29) illustrate by a figure in which the squares are drawn outwards,
as in 1. 47, and divided by diagonals into equal triangles; but I think that the
truth was morejikely to befirst observed from a figure of the kind suggested
by Biirk (Das ApastambaSulba-Sutra in Zeilschrift der deuts ken morgentand.
Gesellschaft, lv., 1901, p.~557) to explain how the Indians arrived at the
same thing. The two figures are as shown above. When the geometrical
consideration of the figure had shown that the isosceles right-angled triangle
had the property in question, the investigation of the same fact from the
arithmetical point of view would ultimately lead to the other momentous
discovery of the irrationality of the length of the diagonal of a square expressed
in terms of its side.

The irrational will come up for discussion later; and our next question
is: Assuming that Pythagoras had observed the geometrical truth of the
theorem in the case of the two particular triangles, and doubtless of Other
rational right-angled triangles, how did he establish it generally ?

There is no positive evidence on this point. Two possible lines are
however marked out. (1) Tannery says (La Gtomitrie greegue, p.~105) that
the geometry of Pythagoras was sufficiently advanced to make it possible
for him to prove the theorem by similar triangles. He does not say in
what particular manner similar triangles would be used, but their use must
apparently have involved the use of proportions, and, in order that the proof
should be conclusive, of the theory of proportions in its complete form
applicable to incommensurable as well as commensurable magnitudes. Now
Eudoxus was the first to make the theory of proportion independent of the
hypothesis of commensurability; and as, before Eudoxus' time, this had not
been done, any proof of the general theorem by means of proportions given
by Pythagoras must at least have been inconclusive. But this does not
constitute any objection to the supposition that the truth of the general
theorem may have been discovered in such a manner; on the contrary, the
supposition that Pythagoras proved it by means of an imperfect theory of
proportions would better than anything else account for the fact that Euclid
had to devise an entirely new proof, as Proclus says he did in 1. 47. This
proof had to be independent of the theory of proportion even in its rigorous
form, because the plan of the Elements postponed that theory to Books v.
and vi. , while the Pythagorean theorem was required as early as Book U.
On the other hand, if the Pythagorean proof had been based on the doctrine
of Books 1. and 11. only, it would scarcely have been necessary for Euclid to
supply a new proof.

The possible proofs by means of proportion would seem to be practically
limited to two.

(a) One method is to prove, from the similarity of the triangles ABC,
DBA, that the rectangle CB, BD is equal to the
square on BA, and, from the similarity of the
triangles ABC, DAC, that the rectangle BC, CD
is equal to the square on CA; whence the result
follows by addition.

It will be observed that this proof is in substance
identical with that of Euclid, the only difference
being that the equality of the two smaller squares

to the respective rectangles is inferred by the method of Book vi. instead
of from the relation between the areas of parallelograms and triangles on the
same base and between the same parallels established in Book 1. It occurred
to me whether, if Pythagoras' proof had come, even in substance, so near to
Euclid's, Proclus would have emphasised so much as he does the originality
of Euclid's, or would have gone so far as to say that he marvelled more at
that proof than at the original discovery of the theorem. But on the whole
I see no difficulty; for there can be little doubt that the proof by proportion
is what suggested to Euclid the method of 1. 47, and the transformation of
the method of proportions into one based on Book i. only, effected by a
construction and proof so extraordinarily ingenious, is a veritable tour de
font which compels admiration, notwithstanding the ignorant strictures of
Schopenhauer, who wanted something as obvious as the second figure in
the case of the isosceles right-angled triangle (p.~353), and accordingly
(Siimmtlithr Wtrkt, 111. § 39 and 1. § 15) calls Euclid's proof ``a mouse-trap
proof'' and ``a proof walking on stilts, nay, a mean, underhand, proof'' (``Des
Eukleides stelzbeiniger, ja, hi nterlis tiger Beweis ``).

{b) The other possible method is this. As it would be seen that the
triangles into which the original triangle is divided by the perpendicular from
the right angle on the hypotenuse are similar to one another and to the whole
triangle, while in these three triangles the two sides about the right angle in the
original triangle, and the hypotenuse of the original triangle, are corresponding
sides, and that the sum of the two former similar triangles is identically equal
to the simitar triangle on the hypotenuse, it might be inferred that the same
would also be true of squares described on the corresponding three sides
respectively, because squares as well as similar triangles are to one another in
the duplicate ratio of corresponding sides. But the same thing is equally true
of any similar rectilineal figures, so that this proof would practically establish
the extended theorem of Eucl. vi. 31, which theorem, however, Proclus
appears to regard as being entirely Euclid's discovery.

On the whole, the most probable supposition seems to me to be that
Pythagoras used the first method (a) of proof by means of the theory of
proportion as he knew it, i.e.\ in the defective form which was in use up to the
date of Eudoxus.

(1) I have pointed out the difficulty in the way of the supposition that
Pythagoras' proof depended upon the principles of Eucl. Books 1. and 11. only.

,,„ I , » ,

5 ' A \

\
\
\
\

Were it not for this difficulty, the conjecture of Bretschneider (p.~81), followed
by Hankel (p.~98), would be the most tempting hypothesis. According to this
suggestion, we are to suppose a figure like that of Eucl. it. 4 in which a, b are
the sides of the two inner squares respectively, and a + b is the side of the
complete square. Then, if the two complements, which are equal, are divided
by their two diagonals into four equal triangles of sides a, b, t, we can place
these triangles round another square of the same size as the whole square, in the
manner shown in the second figure, so that the sides a, b of sutmsnx triangles
make up one of the sides of the square and are arranged in cyclic order. It
readily follows that the remainder of the square when the four triangles are
deducted is, in the one case, a square whose side is e, and in the other the sum of
two squares whose sides are a, b respectively. Therefore the.square on t is equal
to the sum of the squares on a, b. All that can be said against this con-
jectural proof is that it has no specifically Greek colouring
but rather recalls the Indian method. Thus Bhaskara
(bom 1 1 14 a.d.; see Cantor, 1,, p.~656) simply draws
four right-angled triangles equal to the original one in-
wards, one on each side of the squaie on the hypotenuse,
and says ``see ! ``, without even adding that inspection
shows that

c> =  + {a-i,y--

a> + o*.

Though, for the reason given, there is difficulty in supposing that
Pythagoras used a general proof of this kind, which applies of course to right-
angled triangles with sides incommensurable as well as commensurable, there
is no objection, I think, to supposing that the truth of the proposition in the
case of the first rational right-angled triangles discovered, e.g.\ 3, 4, 5, was
proved by a method of this sort. Where the sides are commensurable in this
way, the squares can be divided up into small (unit) squares, which would
much facilitate the comparison between them. That this subdivision was in
fact resorted to in adding and subtracting squares is made probable by
Aristotle's allusion to odd numbers as gnomons placed round unity to form
successive squares in Physics lit. 4; this must mean that the squares were
represented by dots arranged in the form of a square and a gnomon formed of
dots put round, or that (if the given square was drawn in the usual way) the
gnomon was divided up into unit squares. Zeuthen has shown (``Thiorime
de Pythagore'' Origine de la Giomctrie scUntifique in Comptes rendus du
//'``Congres international de Philosophic, Geneve, 1904), how easily the
proposition could be proved by a method of this kind for the triangle 3, 4, 5.
To admit of the two smaller squares being shown side by side, take a square
on a line containing 7 units of length (4 + 3), and divide it up into 49
small squares. It would be obvious that the
whole square could be exhibited as containing
four rectangles of sides 4, 3 cyclically arranged
round the figure with one unit sqire in the
middle. (This same figure is given by Cantor, t„
p.~680. to illustrate the method given in. the
Chinese ``Ch6u-pel ``,) It would be seen that

(i) the. whole square (7*) is made up of two
squares 3* and 4*, and two rectangles 3, 4;

(ii) the same square is made up of the square
EFGH and the halves of four of the same rect-
angles 3t 4) whence the square EFGH, being equal

to the sum of the squares 3* and 4', must contain 25 unit squares and its side,
or the diagonal of one of the rectangles, must contain 5 units of length.

Or the result might equally be seen by observing that

(i) the square EFGH on the diagonal of one of the rectangles is made
up of the halves of four rectangles and the unit square in the middle, while

(ii) the squares 3* and 4* placed at adjacent comers of the large square
make up two rectangles 3, 4 with the unit square in the middle.

The procedure would be equally easy for any rational right-angled triangle,
and would be a natural method of trying to prove the property when it had
once been empirically observed that triangles like 3, 4, 5 did in fact contain a
right angle.

Zeuthen has, in the same paper, shown in a most ingenious way how the
property of the triangle 3, 4, 5 could be verified by a sort of combination of
the second possible method by similar triangles,
(t) on p.~354 above, with subdivision of rectangles
into simitar smalt rectangles. 1 give the method on
account of its interest, although it is no doubt too
advanced to have been used by those who first
proved the property of the particular triangle.

Let ABC be a triangle right-angled at A, and
such that the lengths of the sides A B, A C are 4 and
3 units respectively.

Draw the perpendicular AD, divide up AB, AC
into unit lengths, complete the rectangle on B C as

base and with AD as altitude, and subdivide this rectangle into small
rectangles by drawing parallels to BC, AD through the points of division of
AB, AC.

Now, since the diagonals of the small rectangles are all equal, each being
of unit length, it follows by similar triangles that the small rectangles are all
equal. And the rectangle with AB for diagonal contains ifi of the small
rectangles, while the rectangle with diagonal A C contains 9 of them.

But the sum of the triangles ABD, ADC is equal to the triangle ABC.

Hence the rectangle with BC as diagonal contains 9+16 or 15 of the
small rectangles;
and therefore BC- 5.

Rational right-angled triangles from the arithmetical stand-
point.

Pythagoras investigated the arithmetical problem of finding rational
numbers which could be made the sides of right-angled triangles, or of finding
square numbers which are the sum of two squares; and herein we find the
beginning of the indeterminate analysis which reached so high a stage of
development in Diophantus. Fortunately Proclus has preserved Pythagoras'
method of solution in the following passage (pp.~418, 7 — 419, 8). ``Certain
methods for the discovery of triangles of this kind are handed down, one of
which they refer to Plato, and another to Pythagoras. [The latter] starts from
odd numbers. For it makes the odd number the smaller of the sides about
the right angle; then it takes the square of it, subtracts unity, and makes
half the difference the greater of the sides about the right angle; lastly it adds
unjty to this and so forms the remaining side, the hypotenuse. For example,
taking 3, squaring it, and subtracting unity from the 9, the method takes naif
of the 8, namely 4; then, adding unity to it again, it makes 5, and a right-
angled triangle has been found with one side 3, another 4 and another 5. But
the method of Plato argues from even numbers. For it takes the given even
number and makes it one of the sides about the right angle; then, bisecting
this number and squaring the half, it add si unity to the square to form the
hypotenuse, and subtracts unity from the square to form the other side about
the right angle. For example, taking 4, the method squares half of this, or
2, and so makes 4; then, subtracting unity; it produces 3, and adding unity
it produces 5. Thus it has formed the same triangle as that which was
obtained by the other method.''

The formula of Pythagoras amounts, if fn be an odd number, to

the sides of the right-angled triangle being m, — , . Cantor

2 2

(ii, pp.~185—6), taking up an idea of Roth (Geschuhte der abendldndisehen
Phitosophie, 11. 517), gives the following as a possible explanation of the way in
which Pythagoras arrived at his formula. It £ - a* + 1?, it follows that

d> = t'-b* = {e + b)(e-b).

Numbers can be found satisfying the first equation if (i) c + b and c — b are
either both even or both odd, and if further (2) e + b and t — b are such
numbers as, when multiplied together, produce a square number. The first
condition is necessary because, in order that e and b may both be whole
numbers, the sum and difference of c + b and c — b must both be even. The
second condition is satisfied if e + b and e—b are what were called similar
numbers (o/i«<u dpid/iot); and that such numbers were most probably known
in the time before Plato may be inferred from their appearing in Theon of
Smyrna (Exposiiio rerum mathemalicarum ad legendum Platonem utitsum, ed.
Hiller, p.~36, 1 2), who says that similar plane numbers are, first, all square
numbers and, secondly, such oblong numbers as have the sides which contain
them proportion al. Thus 6 is an oblong number with length 3 and breadth a;
24 is another with length 6 and breadth 4. Since therefore 6 is to 3 as 4 is
to 2, the numbers 6 and 24 are similar.

Now the simplest case of two similar numbers is that of 1 and a\ and,
since 1 is odd, the condition (1) requires that a\ and therefore a, is also odd.
That is, we may take r and (2/1 + 1 f and equate them respectively to t— b and
c +6, whence we have

2

(a« + i) 1 - 1

e = ' + 1 ,

2

while u = a« + 1.

As Cantor remarks, the form in which c and b appear correspond sufficiently

closely to the description in the text of Proclus.

Another obvious possibility would be, instead of equating c - b to unity, to

put (~b~i, in which case the similar number e + b must be equated to

double of some square, i.e.\ to a number of the form in 1 , or to the half of an

fan)'
even square number, say * — - . This would gjve

a- an,
b = n*-i,
(~n*+ 1,

which is Plato's solution, as given by Proclus.

The two solutions supplement each other. It is interesting to observe that
the method suggested by Roth and Cantor is very like that of Euci. x.
(Lemma 1 following Prop.~28). We shall come to this later, but it may be
mentioned here that the problem is to find two square numbers such that their
sum is also a square. Euclid there uses the property of u. 6 to the effect that,
if AB is bisected at C and produced to D,

AD.DB + BC'CZT.
We may write this uv = <?-,

where u = c + b, v = c-b.

In order that uv may be a square, Euclid points out that u and v must be
similar numbers, and further that u and v must be either both odd or both
even in order that b may be a whole number. We may then put for the
similar numbers, say, aB* and ay', whence (if aj8°, ay 1 are either both odd or
both even) we obtain the solution

, oy > + pfcag y = ( ``ffw y

But I think a serious, and even fatal, objection to the conjecture of Cantor
and Roth is the very fact that the method enables both the Pythagorean and
the Platonic series of triangles to be deduced with equal ease. If this had
been the case with the method used by Pythagoras, it would not, I think, hate
been left to Plato to discover the second series of such triangles. It seems to
me therefore that Pythagoras must have used some method which would
produce his rule only; and further it would be some less recondite method,
suggested by direct obstrvation rather than by argument from general
principles.

One solution satisfying these conditions is that of Bretschneider (p.~83),
who suggests the following simple method Pythagoras was certainly aware
that the successive odd numbers are gnomons, or the differences between
successive square numbers. It was then a simple matter to write down in
three rows (a) the natural numbers, (/>) their squares, (V) the successive odd
numbers constituting the differences between the successive squares in (b), thus:
r 2 3 4 5 6 7 8 9 10 n 12 13 14
1 4 9 16 35 36 49 64 81 100 121 144 169 196
1357 9 11 13 15 17 19 21 *3 25 27
Pythagoras had then only to pick out the numbers in the third row which are
squares, and his rule would be obtained by finding the formula connecting the
square in the third line with the two adjacent squares in the second line. But
even this would require some little argument; and I think a still better
suggestion, because making pure observation play a greater part, is that of
p.~Treutlein {Zeiisckrift fiir Mathematik and Physik, xxviil, 1883, Hist.-litt.
Abtheilung, pp.~209 sqq.).

We have the best evidence (e.g.\ in Theon of Smyrna) of the practice of
representing square numbers and other figured numbers, e.g.\ oblong, triangular,
hexagonal, by dots or signs arranged in the shape of the particular figure.
(cf.\ Aristotle, Mttaph. 1092 b 12). Thus, says Treutlein, it would be easily
seen that any square number can be turned into the next higher square
by putting a single row of dots round two adjacent sides, in the form of a
gnomon (see figures on next page).

If a is the side of a particular square, the gnomon round it is shown by
simple inspection to contain 2a + 1 dots or units. Now, in order that za + 1
may itself be a square, let us suppose

a» + 1 = n\
whence a = J (« a - 1 ),

and a + 1 = § (n* + 1).

In order that a and a + t may be integral, n must be odd, and we have at
once the Pythagorean formula

I think Treutlein's hypothesis is shown to be the conect one by the passuge
in Aristotle's Physics already quoted, where the reference is undoubtedly to the
Pythagoreans, and odd numbers are clearly identified with gnomons ``placed
round i.'' But the ancient commentaries on the passage make the matter
clearer still. Philoponus says: ``As a proof... the Pythagoreans refer to what

13

3

M3

happens with the addition of numbers; for when the odd numbers are
successively added to a square number they keep it square and equilateral....
Odd numbers are accordingly called gnomons because, when added to what are
already squares, they preserve the square form,... Alexander has excellently
said in explanation that the phrase ' when gnomons are placed round ' means
making a figure with the odd numbers (rip/ xara tow s-tptrroin apfunn
<rxi} r viToyfMiu')...{or it is the practice with the Pythagoreans to represent
things in figures (trxTJiMTttyptufnu/).''

The next question is; assuming this explanation of the Pythagorean
formula, what are we to say of the origin of Plato's ? It could of course be
obtained as a particular case of the general formula of EucL x. already
referred to; but there are two simple alternative explanations in this case also,
(i) Bretschneider observes that, to obtain Plato's formula, we have only to
double the sides of the squares in the Pythagorean formula,
for <**)« + (*-!)• = (*+,)',

where however « is not necessarily odd.

(a) T re u tie in would explain by means of an extension of the gnomon idea.
As, he says, the Pythagorean formula was obtained by placing a gnomon
consisting of a single row of dots round two adjacent sides of a square, it
would be natural to try whether another solution could not
be found by placing round the square a gnomon consisting of
a double row of dots. Such a gnomon would equally turn the
square into a larger square; and the question would be
whether the double-row gnomon itself could be a square. If . .j. « .
the side of the original square was a, it would easily be seen
that the number of units in the double-row gnomon would be 4« + 4, and we
have only to put

40 + 4 = 4**,

Construct on EF the square EG, and produce DH to A'' so that DK
may be equal to AC.

It is then proved that, in the triangles
BAC, CFG, KHG, BDK,
the sides BA, CF, KH, BD are all equal,
and
the sides AC, FG, HG, DKare all equal.

The angles included by the equal sides
are all right angles; hence the four triangles
are equal in all respects. [l. 4]

Hence BC, CG, GK, KB are all equal.

Further the angles DBK, ABC are equal;
hence, if we add to each the angle DBC,
the angle KBC is equal to the angle ABD
and is therefore a right angle.

In the same way the angle CGK is right;
therefore BCGK is a square, i.e.\ the square on BC.

Now the sum of the quadrilateral GCLH and the triangle LDB together
with two of the equal triangles make the squares on AB, AC, and together
with the other two make the square on BC.

Therefore etc.

II. Another proof is easily arrived at by taking the particular case of
Pappus' more general proposition given below in which the given triangle
is right-angled and the parallelograms on the sides containing the right angles
are squares. If the figure is drawn, it will he seen that, with no more than
one additional line inserted, it contains Thsbit's figure, so that T habit's proof
may have been practically derived from that of Pappus.

III. The most interesting of the remaining proofs seems to be that
shown in the accompanying figure.

It is given by J. W. Miiller, Systema-
tistht Zusammensttllung der wiehtigstm
bisher bekannttn Btweist des Pythag.
Likrsatzis (Numbers, 1819), and in
the second edition (Mainz, 182 1) of
Ign. Hoffmann, Der Pythag. Lehr-
tah mil 31 thdls bekannttn theih
ruuett Btweisen [3 more in second
edition]. It appears to come from
one of the scientific papers of Lion-
ardo da Vinci 452— 1519).

The triangle HKL is constructed
on the base KH with the side KL
equal to BC and the side LH equal
to AB.

Then the triangle HLK is equal in all respects to the triangle ABC,
and to the triangle EBF.

Now D£, BG, which bisect the angles ABE, CBF respectively, are
in a straight line. Join BL.

It is easily proved that the four quadrilaterals ADGC, EDGF, ABLK,
HLBC are all equal.

Hence the hexagons ADEFGC, ABCHLK are equal.
Subtracting from the former the two triangles ABC, EBF, and from the
Utter the two equal triangles ABC, HLK, we prove that

the square CK is equal to the sum of the squares AE, CF.

Pappus' extension of \prop{1}{47}.

In this elegant extension the triangle may be any triangle (not necessarily
right-angled), and any parallelograms take the place of squares on two of the
sides.

Pappus (iv. p.~177) enunciates the theorem as follows:

If ABC be a triangle, and any parallelograms whatever ABED, BCFG
be described on AB, BC, and if DE, FG be
produced to H, and HB be joined, the
parallelograms ABED, BCFG are equal
to the parallelogram contained by AC,
HB in an angle which is equal to the
sum of the angles BAC, DHB.

Produce MB to K; through A, C
draw AL, CM parallel to HK, and join
LM.

Then, since ALHB is a parallelo-
gram, AL, HB are equal and parallel.
Similarly MC, HB are equal and parallel.

Therefore AL, MC are equal and
parallel;
whence LM, A C are also equal and parallel,

and ALMC is a parallelogram.

Further, the angle LAC al this parallelogram is equal to the sum of the
angles BAC, DHB, since the angle DHB is equal to the angle LAB.

Now, since the parallelogram DABE is equal to the parallelogram LABH
(for they are on the same base AB and in the same parallels AB, DM),
and likewise LABH is equal to LAKN (for they are on the same base LA
and in the same parallels LA, HK),

the parallelogram DABE is equal to the parallelogram LAKH.

For the same reason,

the parallelogram BGFC is equal to the parallelogram NKCM.

Therefore the sum of the parallelograms DABE, BGFC is equal to the
parallelogram LA CM, that is, to the parallelogram which is contained by AC,
HB in an angle LAC which is equal to the sum of the angles BAC, BHD.

``And this is far more general than what is proved in the Elements about
squares in the case of right-angled (triangles).''

Heron's proof that AL, BK, CF in Euclid's figure meet in
a point.

The final words of Proclus' note on 1. 47 (p.~429, 9 — 15) are historically
interesting. He says: ``The demonstration by the writer of the Elements being
clear, I consider that it is unnecessary to add anything further, and that we may
be satisfied with what has been written, since, in fact those who have added
anything more, like Pappus and Heron, were obliged to draw upon what is
proved in the sixth Book, for no really useful object.'' These words cannot
of course refer to the extension of 1, 47 given by Pappus; but the key to
them, so far as Heron is concerned, is to be found in the commentary of
an-NairizI (pp.~175— 185, ed. Besthorn-Heiberg; pp.~78 — 84, ed. Curtze) on
1. 47, wherein he gives Heron's proof that the lines AL, FC, BK in Euclid's
figure meet in a point. Heron proved this by means of three lemmas which
would most naturally be proved from the principle of similitude as laid down
in Book vi., but which Heron, as a tour it force, proved on the principles of
Book 1. only. The first lemma is to the following effect

If, in a triangle ABC, DE be drawn parallel to the base BC, and if AF be
drawn from the vertex A to the middle point F of BC, then AF wilt also
bisect DE.

This is proved by drawing HK through A parallel _, Q'

to DE or BC, and HDL, KEM through D, E re \ ";"" ? D

spectively parallel to AGE, and lastly joining DE, EF. \;' ..•''

Then the triangles ABE, AFC are equal (being
on equal bases), and the triangles DBF, EFC are also
equal (being on equal bases and between the same
parallels).

Therefore, by subtraction, the triangles ADF, AEF
are equal, and hence the parallelograms AL, AM are
equal.

These parallelograms are between the same parallels JLM, HK '; therefore
LF, FM ait equal, whence DG, GE are also equal.

The second lemma is an extension of this to the case where DE meets
BA, CA produced beyond A.

The third lemma proves the converse of Euclid 1. 43, that, If a paral-
lelogram AB it cat into four others ADGE, DF, FGCB, CE, so that DF,
CE are equal, the common vertex G will be on the diagonal AB.

Heron produces AG till it meets CF in H. Then, if we join HB, we
have to prove that AHB is one straight line. The
proof is as follows. Since the areas DE, EC are
equal, the triangles DGF, ECG are equal.

If we add to each the triangle GCF,

the triangles ECF, DCF are equal;
therefore MD, CF are parallel.

Now it follows from 1, 34, 20 and 26 that the
triangles AKE, GKD are equal in all respects;

therefore EK is equal to KD.

Hence, by the second lemma,

CH is equal to HE.

Therefore, in the triangles FHB, CHG,
the two sides BF, FH are equal to the two sides G C, CH,

and the angle BFHh equal to the angle GCH;
hence the triangles are equal in all respects,
and the angle B HE is equal to the angle GHC.

Adding to each the angle GHF, we find that the angles BHF, FHG are
equal to the angles CHG, GHF,

and therefore to two right angles.

Therefore AHB is a straight line.

Heron now proceeds to prove the proposition that, in the accompanying
figure, if AKL perpendicular to BC meet
EC in M, and if BM, MG be joined,
BM, MG are in one straight line.
Parallelograms are completed as shown
in the figure, and the diagonals OA, FH
of the parallelogram FH arc drawn.

Then the triangles FAH, BAC are
clearly equal in all respects;

therefore the angle UFA is equal to

the angle ABC, and therefore to the angle

CAK (since AK is perpendicular to BC).

But, the diagonals of the rectangle

FH cutting one another in Y,

FY is equal to YA,

and the angle HFA is equal to the

angle OAF.
Therefore the angles OAF, CAK are
equal, and accordingly

OA, AK are in a straight line.
Hence OM is the diagonal of SQ;

therefore AS is equal to AQ,
and, if we add AM to each,

FM is equal to MH.
But, since EC is the diagonal of the parallelogram FN,

FM \s equal to MN.
Therefore MH is equal to MN;
and, by the third lemma, BM, MG are in a straight line.

Proposition 48.

If in a triangle the square on one of the sides be equal to
the squares on the remaining two sides of the triangle, the
angle contained by the remaining two sides of the triangle is
right.

For in the triangle ABC let the square on one side BC
be equal to the squares on the sides BA, A C;

I say that the angle BA C is right.

For let AD be drawn from the point A at
right angles to the straight line AC, let AD
be made equal to BA, and let DC be joined.

Since DA is equal to AB,
the square on DA is also equal to the square
on AB.

Let the square on A C be added to each;
therefore the squares on DA, AC are equal to the squares
on BA, AC.

But the square on DC is equal to the squares on DA,
AC, for the angle DAC is right; [i. 47]

and the square on BC is equal to the squares on BA, AC, for
this is the hypothesis;

therefore the square on DC is equal to the square on BC,
so that the side DC is also equal to BC

And, since DA is equal to AB,
and AC is common,

the two sides DA, AC are equal to the two sides BA,
AC;
and the base DC is equal to the base BC;

therefore the angle DAC is equal to the angle BAC. [1, 8]
But the angle DA C is right;

therefore the angle BAC is also right.

Therefore etc, q, e. d

P roc! us' note (p.~430) on this proposition, though it does not mention
Heron's name, gives an alternative proof, which is the same as that definitely
attributed by an-NairizT to Heron, the only difference being that Proclus
demonstrates two cases in full, while Heron dismisses the second with a
``similarly.'' The alternative proof is another instance of the use of 1. 7 as a
means of answering objections. If, says Proclus, it be not admitted that the
perpendicular AD may be drawn on the opposite side of A C from B, we may
draw it on the same side as AB, in which case it is impossible that it should
not coincide with AB. Proclus takes two cases,
first supposing that the perpendicular falls, as AD,
within the angle CAB, and secondly that it falls,
as AE, outside that angle. In either case the
absurdity results that, on the same straight line AC
and on the same side of it, AD, DC must be re-
spectively equal to AB, BC, which contradicts 1. 7.

Much to the same effect is the note of De Morgan that there is here ``an
appearance of avoiding indirect demonstration by drawing the triangles on
different sides of the base and appealing to \prop{1}{8}, because drawing them on the
same side would make the appeal to 1. 7 (on which, however, 1. 8 is founded).''

BOOK II.

DEFINITIONS.

i . Any rectangular parallelogram is said to be contained
by the two straight lines containing the right angle.

2. And in any parallelogrammic area let any one whatever
of the parallelograms about its diameter with the two comple-
ments be called a gnomon.

Definition i.

II aV mpak\Ti\Bypa.fLfiQv 6p$ayutvtov Trfpicx«rdat Xfytrai vrro Svo ruv ryy
6p9y ymvlav TTcptcxovo-cuy ttf}uv>

As the full expression in Greek for ``the angle BAC'' is ``the angle
contained by the (straight lines) BA, AC,'' ij ujto tuv BA, Ar irtpitpftirj]
ywrta., so the full expression for ``the rectangle contained by BA, AC''
is rd vro riav BA, Ar wipLttfptyov 6pSoyuvu>ir. In this case too BA, Ar is
commonly abbreviated by the Greek geometers into BAI\ Thus in Archi-
medes and Apollonius to vwh BAT or to vh-o tup BAT means the rectangle
BA, AC, just as tJ vtto BAr means the angle BAC; the gender of the article
shows which is meant in each case. In the early Books Euclid uses the full
expression to vwo t£v BA, Ar; but the shorter form to vtto w BAr is found
from Book x. onwards. cf.\ xn. u, where to (jfixjpaTa) titl t<Zv ©OE, EQZ,
ZPH, H2® means the segments on the eight straight lines ®0, OE, En, nz,
ZF, PH, HS, se.

Definition 2.

Ilavrot ot TraaaXA.T/Xoypafjiui) jfupiw tuk irtpl rrjv Sta/tcrpor avrov TapaXXif-
A.aypdjLfititv tv inrotavovv trvv tdk Swri Trapa.7rKijpwfjunri yvtvfiwf KaAcurir.

Meaning literally a thing enabling something to be known, observed or
verified, a teller or marker, as we might say, the word gnomon (yvwpur) was
first used in the sense (1) in which it appears in a passage of Herodotus (n. 109)
stating that ``the Greeks learnt the t™'Aot, the gnomon and the twelve parts of
the day from the Babylonians.'' According to Suidas, it was Anaximander
(6ti — 545 B.C.) who introduced the gnomon into Greece. Whatever may be
the details of the construction of the two instruments called the wdAcw and
the gnomon, so much is certain, that the gnomon had to do with the
measurement of time by shadows thrown by the sun, and that the word
signified the placing of a staff perpendicular to the horizon. This is home
out by the statement of Proclus that Oenopides of Chios, who first investigated
the problem (End. 1. 1 2) of drawing a perpendicular from an external point
to a given straight line, called the perpendicular a straight line drawn
``gnomon-wist ``(xaro y™fM*a). Then (1) we find the
term used of a mechanical instrument for drawing right
angles, as shown in the figure annexed. This seems to be
the meaning in Theognis 805, where it is said that the
envoy sent to consult the oracle at Delphi should be
``straighter (tBmtps) than the Toprtvt, the (mW/iti and the
gnomon,'' and all three words evidently denote appliances,
the ropvot being an instrument for drawing a circle
(probably a string stretched between a fixed and a moving point), and the
<rra.6ji.tf a plumb-line. Next (3) it was natural that the gnomon, owing to its
shape, should become the figure which remained of a square when a smaller
square was cut out of one corner (or the figure, as Aristotle says, which when
added to a square increases its size but does not alter its form). We have
seen (note on 1.47, p.~351) that the Pythagoreans used the term in this sense, and
further applied it, by analogy, to the series of odd numbers as having the same
property in relation to square numbers. The earliest evidence for this is the
fragment of Philolaus (<-. 460 B-c.) already mentioned (see Boeckh, Philolaos
des Pythagoreers Lehren, p.~141) where he says that ``number makes all things
knowable and mutually agreeing (jcaratyopo. SXkakovi) in the way characteristic of
the gnomon ``(mra yiwfiovos Qvaw). As Boeckh says (p.~144), it would appear
from the fragment that the connexion between the gnomon and the square to
which it is added was regarded as symbolical of union and agreement, and that
Philolaus used the idea to explain the knowledge of things, making the
knowing embrace and grasp the known as the gnomon does the square. Cf.
Scholium 11. No. 1 1 (Euclid, ed. Heiberg, Vol. v. p.~115), which says ``It is
to be noted that the gnomon was discovered by geometers with a view to
brevity, while the name came from its incidental property, namely that from
tt the whole is known, whether of the whole area or of the remainder, when it
is either placed round or taken away. In sundials too its sole function is to
make the actual time of day known.

The geometrical meaning of the word is extended in the definition of
gnomon given by Euclid, where (4) the gnomon has
the same relation to any parallelogram as it before
had to a square. From the fact that Euclid says
``l /et h the figure described ``be coiled a gnomon ``we
may infer that he was using the word in the wider
sense for the first time. Later still (5) we find
Heron of Alexandria defining a gnomon in general
as any figure which, when added to any figure
whatever, makes rhe whole figure similar to that to which it is added. In
this definition of Heron (Def. 58) Hultsch brackets the words which make it
apply to any number as well; but'Theon of Smyrna, who explains that plane,
triangular, square, solid and Other kinds of numbers tire so called after the
likeness of the areas which they measure, does make the term in its most
general sense apply to numbers. ``All the successive numbers which [by
being successively added] produce triangles or squares or polygons are called
jpiomons''(p.~37, 11— i3,ed. Hilier). Thus the successive odd numbers added
together make square numbers; the gnomons in the case of triangular
numbers are the successive numbers i, 2, 3, 4...; those for pentagonal
numbers are the series 1, 4, 7, 10... (the common difference being 3), and so
on. In general, the successive gnemonic numbers tor any polygonal number,
say of n sides, have n — i for their common difference (Theon of Smyrna,
P- 34. '3— IS)-

Geometrical Algebra,

We have already seen (cf, part of the note on 1. 47 and the above note on
the gnomon) how the Pythagoreans and later Greek mathematicians exhibited
different kinds of numbers as forming different geometrical figures. Thus,
says Theon of Smyrna (p.~36, 6 — 11), ``plane numbers, triangular, square
and solid numbers, and the rest, are not so called independently (niptwt) but
in virtue of their similarity to the areas which they measure; for 4, since it
measures a square area, is called square by adaptation from it, and 6 is called
obiong for the same reason,'' A ``plane number ``is similarly described as a
number obtained by multiplying two numbers together, which two numbers
are sometimes spoken of as ``sides,'' sometimes as the ``length ``and
``breadth ``respectively, of the number which is their product.

The product of two numbers was thus represented geometrically by the
rectangle contained by the straight lines representing the two numbers
respectively. It only needed the discovery of incommensurable or irrational
straight lines in order to represent geometrically by a rectangle the product of
any two quantities whatever, rational or irrational; and it was possible to ad-
vance from a geometrical arithmetic to a geometrical algebra, which indeed by
Euclid's time (and probably long before) had reached such a stage of develop-
ment that it could solve the same problems as our algebra so far as they do
not involve the manipulation of expressions of a degree higher than the
second. In order to make the geometrical algebra so generally effective, the
theory of proportions was essential. Thus, suppose that x, y, t etc.\ are
quantities which can be represented by straight tines, while a, ft y etc.\ are
coefficients which can be expressed by ratios between straight lines. We can
then by means or Book vt. find a single straight line J such that

ax + /}/ + yt + ... ``d.
To solve the simple equation in its general form

ax + a = b,
where a represents any ratio between straight lines also requires recourse to
the sixth Book, though, e.g., if a is J or J or any submultiple of unity, or if a is
3, 4 or any power of 2, we should not require anything beyond Book 1. for
solving the aquation. Similarly the general form of a quadratic equation
requires Book vi. for its geometrical solution, though particular quadratic
equations may be so solved by means of Book 11. alone.

Besides enabling us to solve geometrically these particular quadratic
equations, Book 11. gives the geometrical proofs of a number of algebraical
formulae. Thus the first ten propositions give the equivalent of the several
identities

t. a(b + e + rf+ ...) = ab'+ ac + ad + ...,

2. (a4-fi)a-t-(a + b)b = (a + b)*,

3. {a * i>) a = ab + a*,

4. {a + bf - a* + b* + tab.

. (a + b ,\ a (a + 4*

or (a + £) (a - £) + 0* = a*,

6. (za + *)* + a* = (,i + 4)*,
or <«+#{/»-«} +#-#

7. {a + b)' + a 1 = a (a + 6) a + P,

8. 4(a + b)a + b* = t(a4i) + e\\

or(<t + /J)> + {a-0)*=j(a« + jS'),
10. (2a + b) 1 + ** = x la' + (a + *)«},
or (a + £)* + (0 - a)* = 3 (a' + 0*).

f he form of these identities may of course be varied according to the different
symbols which we may use to denote particular portions of the lines given in
Euclid's figures. They are, for the most part, simple identities, but there is no
reason to suppose that these were the only applications of the geometrical
algebra that Euclid and his predecessors had been able to make. We may
infer the very contrary from the fact that Apollonius in his Comes frequently
states without proof much more complicated propositions of the kind.

It is important however to bear in mind that the whole procedure of
Book it. is geometrical; rectangles and squares are shown in the figures, and
the equality of certain combinations to other combinations is proved by those
figures. We gather that this was the classical or standard method of proving
such propositions, and that the algebraical method of proving them, with no
figure except a tine with points marked thereon, was a later introduction.
Accordingly Eutocius 1 method of proving certain lemmas assumed by
Apollonius {ConUs, 11. 23 and in. 29) probably represents more nearly than
Pappus' proof of the same the point of view from which Apollonius regarded
them.

It would appear that Heron was the first to adopt the algebraical method
of demonstrating the propositions of Book 11., beginning from the second,
without figures, as consequences of the first proposition corresponding to

a(b + c + d) = ab + ae + ad,

According to an-Nairiii (ed. Curtze, p.~89), Heron explains that it is not
possible to prove 11. 1 without drawing a number of lines (i.e.\ without actually
drawing the rectangles), but that the following propositions up to 11. 10
inclusive can be proved by merely drawing one line. He distinguishes two
varieties of the method, one by dissolutio, the other by composition by which he
seems to mean splitting-up of rectangles and squares, and combination of them
into others. But in his proofs he sometimes combines the two varieties.

When he comes to 11. 11, he says that it is not possible to do without a
figure because the proposition is a problem, which accordingly requires an
operation and therefore the drawing of a figure.

The algebraical method has been preferred to Euclid's by some English
editors; but it should not find favour with those who wish to preserve the
essential features of Greek geometry as presented by its greatest exponents, or
to appreciate their point of view.

It may not be out of place to add a word with reference to the geometrical
equivalent of the algebraical operations. The addition and subtraction of
quantities represented in the geometrical algebra by lines is of course effected
by producing the line to the required extent or cutting off a portion of it. The
equivalent of multiplication is the construction of the rectangle of which the
given lines are adjacent sides. The equivalent of the division of one quantity
represented by a line by another quantity represented by a line is simply the
statement of a ratio between lines on the principles of Books v. and vi. The
division of a product of two quantities by a third is represented in the
geometrical algebra by the finding of a rectangle with one side of a given
length and equal to a given rectangle or square. This is the problem of
application of areas solved tn \prop{1}{44}, 45. The addition and subtraction of
products is, in the geometrical algebra, the addition and subtraction of
rectangles or squares; the sum or difference can be transformed into a single
rectangle by means of the application of areas to any line of given length,
corresponding to the algebraical process of finding a common measure. Lastly,
the extraction of the square root is, in the geometrical algebra, the finding of a
square equal to a given rectangle, which is done in ll. 14 with the help of 1. 47.

BOOK II. PROPOSITIONS.

Proposition i.

If there be two straight lines, and one of them be cut into
any number of segments whatever, the rectangle contained by
the two straight lines is equal to the rectangles contained by the
uncut straight line and each of the segments.

s Let A, BC be two straight lines, and let BC be cut at
random at the points D, E;

I say that the rectangle contained by A, BC is equal to the
rectangle contained by A, BD,
that contained by A, DE and
10 that contained by A, EC.

For let BF be drawn from B
at right angles to BC; [i. it]

let BG be made equal to A, [t. 3]
through G let GH be drawn
is parallel to BC, [u 31]

and through D, E, C let DK,
EL, CH be drawn parallel to
BG.

Then BH is equal to BK, DL, EH.
ao Now BH is the rectangle A, BC, for it is contained by
GB, BC, and BG is equal to A;

BK is the rectangle A, BD, for it is contained by GB,
BD, and BG is equal to A;

and DL is the rectangle A, DE, far DK, that is BG [1. 34],
5 is equal to A.

Similarly also EH is the rectangle A, EC.

Therefore the rectangle A, BC is equal to the rectangle
A, BD, the rectangle A, DE and the rectangle A, EC.

Therefore etc.

Q. E. D.

so. the rectangle A, BC. From this point onward I shall Iran slate thus in cases where
Euclid leaves out the word contained {irtpitxwtrw}- Though the word ``rectangle'' is also
omitted in the Greek (the neuter article l>eing sufficient to show that the rectangle is
meant), it cannot he dispensed with in English. De Morgan advises the use of the expres-
sion ``the rectangle uiuitr two lines.'' This does not seem to me a very good expression,
and, if used in a translation from the Greek, it might suggest that irwi in ri iri meant
uxdrr, which it does not.

This proposition, the geometrical equivalent of the algebraical formula
a(b + e + d+ ...)=-ab + M+-ad+ ...,
can, of course, easily be extended so as to correspond to the more general
algebraical proposition that the product of an expression consisting of any
number of terms added together and another expression also consisting of
any number of terms added together is equal to the sum of all the products
obtained by multiplying each term of one expression by all the terms of the
other expression, one after another. The geometrical proof of the more
general proposition would be effected by means of a figure showing all the
rectangles corresponding to the partial products, in the same way as they are
shown in the simpler case of II. i; the difference would be that a series or
parallels to DC would have to be drawn as well as the series of parallels

to bjk

Proposition 2.

If a straight line be cut at random, the rectangle contained
by the whole and both of the segments is equal to the square on
the whole.

For let the straight line AB be cut at random at the
point C;

I say that the rectangle contained by AB, BC together with
the rectangle contained by BA, AC is equal
to the square on AB.

For let the square ADEB be described
on AB [1. 46], and let CF be drawn through
C parallel to either AD or BE. [1. 31]

Then AE is equal to AF, CE.

Now AE is the square on AB;

AF is the rectangle contained by BA,
AC, for it is contained by DA, AC, and
AD is equal to AB;

and CE is the rectangle AB, BC, for BE is equal to
AB.

Therefore the rectangle BA, AC together with the rect-
angle AB, BC is equal to the square on AB.

Therefore etc.

Q. E, D.

A C B

OF I!

The fact asserted in the enunciation of this proposition has already been
used in the proof of 1. 47; but there was no occasion in that proof to observe
that the two rectangles BL, CL making up the square on BC are the
rectangles contained by BC and the two parts, respectively, into which it is
divided by the perpendicular from A on BC. It is this fact which it is
necessary to state in this proposition, in accordance with the plan of Book IT.

The second and third propositions are of course particular cases of the
first They were no doubt separately enunciated by Euclid in order that they
might be immediately available for use hereafter, instead of having to be
deduced for the particular occasion from 11. 1. For, if they had not been thus
separately stated, it would scarcely have been practicable to quote them later
without explaining at the same time that they are included in 11. 1 as particular
cases. And, though the propositions are not used by Euclid in the later
propositions of Book 11., they are used afterwards in MIL 10 and \prop{9}{15}
respectively; and they are of extreme importance for geometry generally,
being constantly used by Pappus, for example, who frequently quotes the
third proposition by the Book and number.

Attention has been called to the fact that 11. t is never used by Euclid;
and this may seem no less remarkable than the fact that 11. 2, 3 are not again
used in Book 11. But it is important, I think, to observe that the proof) of
all the first ten propositions of Book 11. are practically independent of each
other, though the results are really so interwoven that they can often be
deduced from each other in a variety of ways. What then was Euclid's
intention, first in inserting some propositions not immediately required, and
secondly in making the proofs of the first ten practically independent of
each other? Surely the object was to show the power of the method of
geometrical algebra as much as to arrive at results. From the point of view
of illustrating the method, there can be no doubt that Euclid's procedure is
far more instructive than the semi-algebraical substitutes which seem to rind
a good deal of favour; practically it means that, instead of relying on our
memory of a few standard formulae, we can use the machinery given us by
Euclid's method to prove immediately ab initio any of the propositions taken
at random.

Let us contrast with Euclid's plan the semi-algebraical alternative. One
editor, for example, thinks that, as \prop{2}{1} is not used by Euclid afterwards, it
seems more logical to deduce from it those of the subsequent propositions
which can be readily so deduced. Putting this idea into practice, he proves
11. 2 and 3 by quoting 11. 1, then proves 11. 4 by means of 11. 1 and 3, 11. 5 and
6 by means of 11. 1, 3 and 4, and so on. The result is ultimately to deduce
the whole of the first ten propositions from 11. 1, which Euclid does not use at
all; and this is to give an importance to 11. 1 which is altogether dispro-
portionate and, by starting with such a narrow foundation, to make the whole
structure of Book 11. top-heavy.

Editors have of course been much influenced by a desire to make the
proofs of the propositions of Book 11. easier, as they think, for schoolboys.
But, even from this point of view, is it an improvement to deduce 11. 2 and 3
from 11. 1 as corollaries P I doubt it. For, in the first place, Euclid's figures
visua/ist the results and so make it easier to grasp their meaning; the truth
of the propositions is made clear even to the eye. Then, in the matter of
brevity, to which such an exaggerated importance is attached, Euclid's proof
positively has the advantage. Counting a capital letter or a collocation of such
as one word, I find, e.g., that Mr H. M. Taylor's proof of 11. 2 contains
1 20 words, of which 8 represent the construction. Euclid's as above trans-
lated has 126 words, of which 22 are descriptive of the construction; therefore
the actual proof by Euclid has 8 words fewer than Mr Taylor's, and the extra
words due to the construction in Euclid are much more than atoned for by
the advantage of picturing the result in the figure.

The advantages then which Euclid's method may claim are, I think, these:
in the case of \prop{2}{2}, 3 it produces the result more easily and clearly than does
the alternative proof by means of 11. 1, and, in its general application, it is
more powerful in that it makes us independent of any recollection of results.

Proposition 3.

If a straight line be cut at random, the rectangle contained
by the whole and one of the segments is equal to the rectangle
contained by the segments and the square on the aforesaid
segment.

For let the straight line AB be cut at random at C;
I say that the rectangle contained by AB, BC is equal to the
rectangle contained by AC, CB together
with the square on BC.

For let the square CDEB be de-
scribed on CB; [1. 46]
let ED be drawn through to F,
and through A let AF be drawn parallel
to either CD or BE. [1. 31]

Then AE is equal to AD, CE.

Now AE is the rectangle contained by AB, BC, for it is
contained by AB, BE, and BE is equal to BC;

AD is the rectangle AC, CB, for DC is equal to CB;

and DB is the square on CB.
Therefore the rectangle contained by AB, BC is equal to
the rectangle contained by AC, CB together with the square
on BC.

Therefore etc.

Q. E. D.

If we leave out of account the contents of Book 11. itself and merely look
to the applicability of propositions to general use, this proposition and the
preceding are, as already indicated, of great importance, and particularly so to
the semi-algebraical method just described, which seems to have found its first
exponents in Heron and Pappus. Thus the proposition that the difference of
the squares on two straight Urns is equal to the rectangle contained by the sum
and the differtna of the straight lines, which is generally given as equivalent to
ii. 5, 6, can be proved by means of ii. i, z, 3, as shown

by Lardner. For suppose the given straight lines are ft C P

A B, BC, the latter being measured along BA.

, Then, by n. 2, the square on AB is equal to. the sum of the rectangles
AB, BC and AB, AC.

By 11. 3, the rectangle AB, BC is equal to the sum of the square on BC
and the rectangle AC, CB.

Therefore the square on AB is equal to the square BC together with the
sum of the rectangles AC, AB and AC, CB.

But, by 11. 1, the sum of the latter rectangles is equal to the rectangle
contained by AC and the sum of A3, BC, i.e.\ the rectangle contained by the
sum and difference of AB, BC.

Hence the square en AB is equal to the square on BC and the rectangle
contained by the sum and difference of AB, BC\

that is, the difference of the squares on AB, BCis equal to the rectangle
contained by the sum and difference of AB, BC.

Proposition 4.

If a straight tine be cut at random, the square on the whole
is equal to the squares on the segments and twice the rectangle
contained by the segments.

For let the straight line AB be cut at random at C;

s I say that the square on AB is equal to the squares on A C,
CB and twice the rectangle contained
by AC, CB.

For let the square ADEB be de-
scribed on AB, [1. 46]

10 let BD be joined;
through C let CF be drawn parallel to
either AD or EB,

and through G let HK be drawn parallel
to either AB or DE. [1. 31]

is Then, since CF is parallel to AD,
and BD has fallen on them,

the exterior angle CGB is equal to the interior and opposite
angle ADB. [t 19]

But the angle ADB is equal to the angle ABD,

io since the side BA is also equal to AD; [f. 5]

therefore the angle CGB is also equal to the angle GBC,
so that the side BC is also equal to the side CG. [1. 6]

But CB is equal to GK, and CG to KB; [i. 34]

therefore GK is also equal to KB;
*s therefore CGKB is equilateral.

I say next that it is also right-angled.
For, since CG is parallel to BK,

the angles KBC, GCB are equal to two right angles.

[l. 2 9 ]

But the angle KBC is right;
3° therefore the angle BCG is also right,

so that the opposite angles CGK, GKB are also right.

b- 34]

Therefore CGKB is right-angled;
and it was also proved equilateral;

therefore it is a square;
35 and it is described on CB.
For the same reason

HF is also a square;
and it is described on HG, that is AC. [1. 34]

Therefore the squares HF, AX'' are the squares on AC, CB.
4° Now, since A G is equal to GE,
and A G is the rectangle A C, CB, for GC is equal to CB,
therefore GE is also equal to the rectangle AC, CB.
Therefore AG, GE are, equal to twice the rectangle AC,
CB.
+s But the squares HF, CK are also the squares on AC, CB;
therefore the four areas HF, CK, AG, GE are equal to
the squares on AC, CB and twice the rectangle contained by
AC, CB.

But HF, CK, AG, GE are the whole A DEB,
50 which is the square on AB.

Therefore the square on AB is equal to the squares on
AC, CB and twice the rectangle contained by AC, CB.
Therefore etc.\ Q. E. D-

1. twice the rectangle contained by the segments. By a carious idiom ihjs is in
Greek ``the rectangle Mi contained by the segments.'' Similarly ``twice the rectangle
contained by AC, CB'' is expressed as ``the rectangle twin contained by AC, CB'' (rilli
vri>* Ar, TB ittpaxbfAtirw Apfhy4vtQi'),

35, 38. described, jo, 43. the squares (before ``on''). These words are not in the
Greek, which limply says that the squares ``are on ``{tlalr iri) their respective sides.

46. areas. It is necessary to supply some substantive (the Greek leaves it to be under-
stood); and I prefer ``areas ``to ``figures.''

The editions of the Greek text which preceded that of E. F. August
(Berlin, 1826 — 9) give a second proof of this proposition introduced by the
usual word dXAun or ``otherwise thus.'' Heiberg follows August in omitting
this proof, which is attributed to Theon, and which is indeed not worth
reproducing, since it only differs from the genuine proof in that portion of it
which proves that CGKB is a square. The proof that CGKB is equilateral
is rather longer than Euclid's, and the only interesting point to notice is that,
whereas Euclid still, as in 1. 46, seems to regard it as necessary to prove that
all the angles of CGKB are right angles before he concludes that it is right-
angled, Theon says simply ``And it also has the angle CBK right; therefore
CK is a square.'' The shorter form indicates a legitimate abbreviation of the
genuine proof; because there can be no need to repeat exactly that part of the
proof of 1. 46 which shows that all the angles of the figure there constructed
are right when one is.

There is also In the Greek text a Porism which is undoubtedly interpolated:
``From this it is manifest that in square areas the parallelograms about the
diameter are squares.'' Heiberg doubted its genuineness when preparing his
edition, and conjectured that it too may have been added by Theon; but the
matter is placed beyond doubt by a papyrus-fragment referred to already (see
Heiberg, Paralipomena zu Euklid, in Hermes xxxvm., 1903, p.~48) in which
the Porism was evidently wanting. It is the only Porism in Book 11., but
does not correspond to Proclus* remark (p.~304, 2) that ``the Porism found in
the second book belongs to a problem.'' Heiberg regards these words as
referring to the Porism to iv. 15, the correct reading having probably been not

itmifnf but 8', i.e.\ TwapTiii.

The semi -algebraical proof of tins proposition is very easy, and is of course
old enough, being found in Clavius and in most later editions. It proceeds
thus.

By 11, 1, the square on AB is equal to the sum of the rectangles AB, AC
and AB, CB.

But, by 11. 3, the rectangle AB, AC is equal to the sum of the square on
AC and the rectangle AC, CB;

while, by 11, 3, the rectangle AB, CB is equal to the sum of the square on
BC and the rectangle AC, CB.

Therefore the square on AB is equal to the sum of the squares on
AC, CB and twice the rectangle AC, CB.

The figure of the proposition also helps to visualise, in the orthodox
manner, the proof of the theorem deduced above from 11. 1 — 3, viz.\ that the
difference of the squares on two given straight lines is equal to the rectangle
contained by the sum and the difference of the lines.

For, if the lines be AB, BC respectively, the shorter of the lines being
measured along BA, the figure shows that

the square AE is equal to the sum of the square CK and the rectangles
AF,FK,

that is, the square on A B is equal to the sum of the square on BC and
the rectangles AB, AC and AC, BC.

But the rectangles AB, AC and BC, AC ait, by 11. 1, together equal to
the rectangle contained by Cand the sum of AB, BC,
i.e to the rectangle contained by the sum and difference of AB, BC.

Whence the result follows as before.

The proposition 11. 4 can also be extended to the case where a straight
line is divided into any number of segments; for the figure will show in like
manner that the square on the whole line is equal to the sum of the squares
on all the parts together with twice the rectangles contained by every pair of
the parts.

Proposition 5.

If a straight line be cut into equal and unequal segments,
the rectangle contained by the unequal segments of the whole
together with the square on the straight line between the
points of section is equal to the square on the half

For let a straight line AB be cut into equal segments
at C and into unequal segments at D;

I say that the rectangle contained by AD, DB together with
the square on CD is equal to the square on CB.

For let the square CEFB be described on CB, [1. 46]

and let BE be joined;

through D let DG be drawn parallel to either CE or BF,
through H again let KM be drawn parallel to either AB or
EF t

and again through A let AK be drawn parallel to either CL
or BM. [1.31]

Then, since the complement CH is equal to the comple-
ment HF % [1. 43]
let DM be added to each;

therefore the whole CM is equal to the whole DF.

But CM is equal to AL,

since AC is also equal to CB; • [1. 3 6 ]

therefore AL is also equal to DF.
Let CH be added to each;

therefore the whole AH is equal to the gnomon NOP.

But AH is the rectangle AD, DB, for DH is equal to
DB,

therefore the gnomon NOP is also equal to the rectangle
AD, DB.

Let LG, which is equal to the square on CD, be added to
each;

therefore the gnomon NOP and LG are equal to the
rectangle contained by AD, DB and the square on CD.

But the gnomon NOP and LG are the whole square
CEFB, which is described on CB;

therefore the rectangle contained by AD, DB together
with the square on CD is equal to the square on CB.

Therefore etc.\ q. e. d.

3. between the points of section, literally ``between the icct/mr,'' ihc word being
the same (rofii) is that used of  conic irriim.

It will be observed that the gnomon is indicated in the figure by three separate letters
and a dotted carve. This is no doubt a clearer way of showing what exactly the gnomon is
than the method usual in our text -books. In this particular case the figure of the ttss. has
hiv M's in it, the gnomon being MN2. I have corrected the lettering to avoid confusion.

It is easily seen that this proposition and [he next give exactly the
theorem already alluded to under the last propositions, namely that the
difference of the squares on two straight lines is equal te the rectangle contained
by their sum and difference. The two given lines are, in 11. 5, the lines CB
and CD, and their sum and difference are respectively equal to AD and DB.
To show that 11. 6 gives the same theorem we have only to make CD the
greater line and CB the less, i.e.\ to
draw CD' equal to CB, measure . cob

CB along it equal to CD, and then ' '

produce B C to A', making A'C equal a| gj p' O *

to BC\ whence it is immediately clear

that A' D' on the second line is equal

to AD on the first, while DB is also equal to DB, so that the rectangles

AD, DB and A'D'', DB are equal, while the difference of the squares on

CB, CD is equal to the difference of the squares on CD, CB.

Perhaps the most important fact about 11. 5, 6 is however their bearing on
the

Geometrical solution of a quadratic equation.

Suppose, in the figure of 11. 5, that AB = a, DB = x;
then «*-*= the rectangle AH

= the gnomon NOP.

Thus, if the area of the gnomon is given (=*', say), and if a is given
(= AB), the problem of solving the equation

is, in the language of geometry, To a given straight lint (a) to apply a rectangle
which shall Si equal to a given square (c5*) and shall fall snort by a square figure,
Le. to construct the rectangle AH at the gnomon NOP,

Now we are told by Proclus (on 1. 44) that ``these propositions are ancient
and the discoveries of the Muse of the Pythagoreans, the application of
areas, their exceeding and their falling-short'' We can therefore hardly
avoid crediting the Pythagoreans with the geometrical solution, based upon
ti. 5, 6, of the problems corresponding to the quadratic equations which
are directly obtainable from them. It is certain that the Pythagoreans solved
the problem in n. 1 1, which corresponds to the quadratic equation

a (a — *) = ar 1 ,

and Simson has suggested the following easy solution of the equation now in
question,

ax~x*=P,
on exactly similar lines.

Draw CO perpendicular to AB and equal to b; produce OC to A 7 so
that ON= CB (or \a); and with O as centre
and radius ON describe a circle cutting CB
in D.

Then DB (or x) is found, and therefore
the required rectangle AH.

For the rectangle AD, DB together with
the square on CD is equal to the square on
CB, [n. S ]

i.e.\ to the square on OD,
i.e.\ to the squares on OC, CD; [l. 47]
whence the rectangle AD, DB is equal to the square on OC,
or ax - x* = b*.

It is of course a necessary condition of the possibility of a real solution
that P must not be greater that (Ja)'. This condition itself can easily be
obtained from Euclid's proposition; for, since the sum of the rectangle AD,
DB and the square on CD is equal to the square on CB, which is constant,
it follows that, as CD diminishes, i.e.\ as D moves nearer to C, the rectangle
AD, DB increases and, when D actually coincides with C, so that CD
vanishes, the rectangle AD, DB becomes the rectangle AC, CB, i.e.\ the
square on CB, and is a maximum. It wilt be seen also that the geometrical
solution of the quadratic equation derived from Euclid does not differ from
our practice of solving a quadratic by completing the square on the side
containing the terms in x* and x.

But, while in this case there are two geometrically real solutions (because
the circle described with ON as radius will not only cut CB in D but will
also cut AC in another point E), Euclid's fig a re corresponds to one only of
the two solutions. Not that there is any doubt that Euclid was aware that the
method of solving the quadratic gives two solutions; he could not fail to see
that x = BE satisfies the equation as well as x = BD. If however he hud
actually given us the solution of the equation, he would probably have
omitted to specify the solution * = BE because the rectangle found by means
of it, which would be a rectangle on the base AE (equal to BD) and with
altitude EB (equal to AD), is really an equal rectangle to that corresponding
to the other solution x = BD; there is therefore no real object in distinguishing
two solutions. This is easily understood when we regard the equation as a
statement of the problem of finding two magnitudes when their sum (a) and
product (b*) are given, i.e.\ as equivalent to the simultaneous equations

x+y = a,
xy = b*.

These symmetrical equations have really only one solution, as the two apparent
solutions are simply the result of interchanging the values of x and y. This
form of the problem was known to Euclid, as appears from the Data, Prop.
85, which states that, If two straight lines contain a parallelogram given in
magnitude in a given angle, and if the sum of them be given, then shall each
of them be given.

This proposition then enables us to solve the problem of finding a
rectangle the area and perimeter of which are both given; and it also enables
us to infer that, of all rectangles of given perimeter, the square has the
greatest area, while, the more unequal the sides are, the less is the area.

If in the figure of 11. 5 we suppose that AD=a, BD=b, we find that
CB = (« + *)/ 2 and CD = (a—b)li, and we may state the result of the
proposition in the following algebraical form

ffl-ffi-*

This way of stating it (which could hardly have escaped the Pythagoreans)
gives a ready means of obtaining the two rales, respectively attributed to the
Pythagoreans and Plato, for finding integral square numbers which are the
sum of two other integral square numbers. We have only to make ab a
perfect square in the above formula. The simplest way in which this can be
done is to put a = n', b=i, whence we have

and in order that the first two squares may be integral a 1 , and therefore n,
must be odd Hence the Pythagorean rule.
Suppose next that a = in*, b-%, and we have
(«*+.)'-(«*-i)' = 4A\
whence Plato's rale starting from an even number in.

Proposition 6.

If a straight line be bisected and a straight line be added
to ii in a straight line, the rectangle contained by the whole
with the added straight line and the added straight line together
with the square on the half is equal to the square on the
straight line made up of the half and the added straight
line.

For let a straight line AB be bisected at the point C, and
let a straight line BD be added to it in a straight line;

I say that the rectangle contained by AD, DB together
with the square on CB is equal to the square on CD.

For let the square CEFD be described on CD, [1. 46]

and let DE be joined;

through the point B let BG be drawn parallel to either EC or
DF,
through the point H let KM be drawn parallel to either A 3
or EF,

and further through A let y4A*
be drawn parallel to either CL
or DM. [i. 31]

Then, since 4C is equal
toC#,

j4Z is also equal to CH. [1. 36]
But CH is equal to HF Ql 43]
Therefore AL is also equal
toJKE
Let CM be added to each;

therefore the whole AM is equal to the gnomon NOP.
But /f4f is the rectangle AD, DB,

for DM is equal to Z?/?;

therefore the gnomon NOP is also equal to the rectangle
AD, DB.

Let LG, which is equal to the square on BC, be added
to each;

therefore the rectangle contained by AD, DB together
with the square on CB is equal to the gnomon NOP and LG.

But the gnomon NOP and LG are the whole square
CEFD, which is described on CD;

therefore the rectangle contained by AD, DB together
with the square on CB is equal to the square on CD.

Therefore etc.

Q. E, D.

In this case the rectangle AD, DB is ``a rectangle applied to a given
straight line (AB) but exceeding by a square (the side of which is equal to
BD) ``; and the problem suggested by 11. 6 is to rind a rectangle of this
description equal to a given area, which we will, for convenience, suppose to
be a square; Le., in the language of geometry, to apply to a given straight
line a rectangle which shall be equal to a given square and shall exceed by a
square figure.

We suppose that in Euclid's figure AB = a, BD=x; then, if the given
square be F, the problem is to solve geometrically the equation

ax + jc* = #•.
The solution of a problem theoretically equivalent to the solution of a
quadratic equation of this kind is presupposed in the fragment of Hippocrates'
Quadrature of lunes preserved in a quotation by Simplicius (Comment, in
Aristot. Phys. pp.~61 — 68, ed. Diels) from Eudemus' History of Geometry, In
this fragment Hippocrates (5th cent. b,c.) assumes the following construction.

AB being the diameter and O the centre of a semicircle, and C being the
middle point of OB and CD at right
angles to AB, a straight line of length
such that its square is \\ times the square
on the radius (i.e.\ of length aj$, where
a is the radius) is to be so placed, as EF,
between CD and the circumference AD
Jiat it ``verges towards B,'' that is, EF
when produced passes through B.

Now the right-angled triangles BFC,
BAE are similar, so that

BF:BC=BA .BE,

and therefore the rectangle BE, BF= rect BA, BC

= sq. on BO.

In other words, EF ( = o ,/f) being given in length, BF ( = x, say) has
to be found such that

(7t « + *)#=**;
or the quadratic equation

,/£ ax + ar* = a''
has to be solved.

A straight line of length ajl would easily be constructed, for, in the
figure, CD*=AC. CB = \a\ or CD=\aJs, and aj\ is the diagonal of
a square of which CD is t,he side.

There is no doubt that Hippocrates could have solved the equation by
the geometrical construction given below, but he may have contemplated, on
this occasion, the merely mcehanital process of placing the straight line of the
length required between CD and the circumference AD and moving it until
E, F, B were in a straight line. Zeuthen {Die Lehre von den Kegelschnittm
im Altertum, pp.~370, 27 r) thinks this probable because, curiously enough,
the fragment speaks immediately afterwards of ``joining B to F.''

To solve the equation

we have to find the rectangle AH, or the
gnomon NOP, which is equal in area to £* and
has one of the sides containing the inner right
angle equal to CB or \a. Thus we know
(Ja)* and £*, and we have to find, by \prop{1}{47},
a square equal to the sum of two given
squares.

To do this Simson draws BQ at right
angles to AB and equal to b, joins CQ and,
with centre C and radius CQ, describes a
circle cutting AB produced in D. Thus
BD, or x, is found.

Now the rectangle AD, DB together with the square on CB
is equal to the square on CD,
i.e.\ to the square on CQ,
i.e.\ to the squares on CB, BQ.

Therefore the rectangle AD, DB is equal to the square on BQ, that is,

jx + x* — fi.

From Euclid's point of view there would only be one solution in this case.
This proposition enables us also to solve the equation
x* — ax-
in a similar manner.

We have only to suppose that AB = a, and AD (instead of BD) = x; then
.» x*—ax = the gnomon.

To find the gnomon we have its area (P) and the area, CB 1 or (|o)*, by
which the gnomon differs from CD 1 . Thus we can find D (and therefore
AD or x) by the same construction as that just given.

Converse propositions to 11. 5, 6 are given by Pappus (vii. pp.~948—950)
among his lemmas to the Conits of Apollonius to the effect that,
(1) if D be a point dividing AB unequally, and C another point on AB
such that the rectangle AD, DB together with the square on CD is
equal to the square on AC, then

Cis equal to CB;

(a) if D be a point on AB produced, and C a point on AB such that the
rectangle AD, DB together with the square on CB is equal to the
square on CD, then

AC is equal to CB.

Proposition 7.

If a straight line be cut at random, the square on the
whole and that on one of the segments both together are equal
to twice the rectangle contained by the whole and the said
segment and the square on the remaining segment.

For let a straight line AB be cut at random at the point C;

I say that the squares on AB, BC are equal to twice the
rectangle contained by AB, BC and the
square on CA.

For let the square ADEB be
described on AB, [1. 46]

and let the figure be drawn.

Then, since AG is equal to GE, [i. 43]
let CF be added to each;

therefore the whole AF is equal to
the whole CE,

Therefore AF, CE are double of
AF.

But AF, CF are the gnomon KLM and the square CF;
therefore the gnomon KLM and the square CF are double
of AF.

But twice the rectangle AB, BC is also double of AF;
for BF is equal to BC;

therefore the gnomon KLM and the square CF are equal to
twice the rectangle AB, BC.

Let DG, which is the square on AC, be added to each;
therefore the gnomon KLM and the squares BG, GD are
equal to twice the rectangle contained by AB, BC and the
square on AC.

But the gnomon KLM and the squares BG, GD are the
whole ADEB and CF,

which are squares described on AB, BC;
therefore the squares on AB, BC are equal to twice the
rectangle contained by AB, BC together with the square on
AC.

Therefore etc.

q. £. D.

An interesting variation of the form of this proposition may be obtained by
regarding AB, BC as two given straight lines of which AS is the greater, and
AC as the difference between the two straight lines. Thus the proposition
shows that the squares on two straight lines are together equal to twice the
rectangle contained by them and the square on their difference. That is, the
square en the different of two straight lines is equal to the sum of the squares on
the straight lines diminished by twite the rectangle contained by them. In other
words, just as 11. 4 is the geometrical equivalent of the identity

(a + b) , d , + b t + 2ab,
so 11. 7 proves that

(a -t) , = a* + P-tab.
The addition and subtraction of these formulae give the algebraical equivalent
of the propositions it. 9, 10 and 11. 8 respectively; and we have accordingly
a suggestion of alternative methods of proving those propositions.

Proposition 8.

If a straight line be cut at random, four times the rectangle
contained by the whole and one of the segments together with
the square on the remaining segment is equal to the square
described on the whole and the aforesaid segment as on one
straight line.

For let a straight line AB be cut at random at the point C;

1 say that four times the rectangle contained by AB, BC
together with the square on AC is equal to the square
described on AB, BC as on one straight line.

For let [the straight line] .5Z> be produced in a straight
line [with AB\, and let BD be
made equal to CB;
let the square A FFD be described
on AD, and let the figure be
drawn double.

Then, since CB is equal to BD,
while CB is equal to GK, and
BD to AW,
therefore <7T is also equal to KN.

For the same reason
QR is also equal to RP.

And, since BC is equal to BD, and 6l#'' to KN,
therefore CK is also equal to KD, and GA to RN [i. 36]

But CK is equal to AjV, for they are complements of the
parallelogram CP; [1. 43]

therefore KD is also equal to GR \

therefore the four areas DK, CK, GR, RN are equal to one
another.

Therefore the four are quadruple of CK.

Again, since CB is equal to BD,
while BD is equal to BK, that is CG,
and CB is equal to GK, that is GQ,

therefore CG is also equal to GQ.

And, since CG is equal to GQ, and QR to RP,

AG'xs also equal to MQ, and £>Z. to RF. [1. 36]

But J/0 is equal to QL, for they are complements of the
parallelogram ML; [1. 43)

therefore AG is also equal to RF;
therefore the four areas AG, MQ, QL, RF are equal to one
another.

Therefore the four are quadruple of AG.
But the four areas CK, KD, GR, RN were proved to be
quadruple of CK;

therefore the eight areas, which contain the gnomon
STU, are quadruple of AK,

Now, since AK is the rectangle AB, BD, for BK is equal
to BD,

therefore four times the rectangle AB, BD is quadruple of
AK.

But the gnomon STU was also proved to be quadruple
otAK;

therefore four times the rectangle AB, BD is equal to the
gnomon STU.

Let OH, which is equal to the square on AC, be added
to each;

therefore four times the rectangle AB, BD together with
the square on AC is equal to the gnomon STU and OH.

But the gnomon STU and OH are the whole square
AEFD,

which is described on AD •
therefore four times the rectangle AB, BD together with
the square on AC is equal to the square on AD

But BD is equal to BC;
therefore four times the rectangle contained by AB, BC
together with the square on AC is equal to the square on
AD, that is to the square described on AB and BC as on
one straight line.

Therefore etc.

This proposition is quoted by Pappus {p.~418, ed. Hultsch) and is used
also by Euclid himself in the Data, Prop.~86. Further, it is of decided use
in proving the fundamental property of a parabola.

Two alternative proofs are worth giving.

The first is that suggested hy the consideration mentioned in the last
note, though the proof is old enough, being given by Clavius and others. It
is of the semi-algebraical type.

Produce AB to D (in the figure of the pro-
position), so that BD is equal to BC.

By 11. 4, the square on AD is equal to the
squares on AB, BD and twice the rectangle AB,
BD, i.e.\ to the squares on AB, BC and twice
the rectangle AB, BC.

By 11. 7, the squares on AB, BC are equal to
twice the rectangle AB, BC together with the
square on AC

Therefore the square on AD is equal to four
times the rectangle AB, BC together with the ' ``

square on AC.

The second proof is after the manner of Euclid but with a difference.
Produce BA to D so that AD is equal to BC On BD construct the square
BEFD.

Take BG, Elf, FK each equal to BC or AD, and draw ALP, HNM
parallel to BE and GML, KPW parallel to BD.

Then it can be shown that each of the rectangles BL, AK, FN, EM is
equal to the rectangle AB, BC, and that PM is equal to the square on AC.

Therefore the square on BD is equal to four times the rectangle AB,
BC together with the square on AC.

Proposition 9.

If a straight line be cut into equal and unequal segments,
the squares on the unequal segments of the whole are double
of the square on the half and of the square on the straight line
between the points of section.

For let a straight line AB be cut into equal segments
at C, and into unequal segments at D\

I say that the squares on AD, DB are double of the
squares on AC, CD.

For let CE be drawn from
C at right angles to AB,
and let it be made equal to
either AC at CB;
let EA, EB be joined,
let DF be drawn through D
parallel to EC,

and FG through F parallel to
AB,
and let AF be joined.

Then, since AC is equal to CE,
the angle EAC is also equal to the angle A EC.

And, since the angle at C is right,

the remaining angles EAC, AEC are equal to one
right angle. ['• 3*]

And they are equal;

therefore each of the angles CEA, CAE is half a right
angle.

For the same reason

each of the angles CEB, EBC is also half a right angle;

therefore the whole angle AEB is right
And, since the angle GEF is half a right angle.

and the angle EGF is right, for it is equal to the interior and
opposite angle ECB, [l ag]

the remaining angle EFG is half a right angle; [1. 3 2 ]
therefore the angle GEF is equal to the angle EFG,

so that the side EG is also equal to GF. [1. 6]

Again, since the angle at B is half a right angle,

and the angle FDB is right, for it is again equal to the interior

and opposite angle ECB, [i- *9]

the remaining angle BFD is half a right angle; [1. 3*]

therefore the angle at B is equal to the angle DFB,

so that the side FD is also equal to the side DB. [1. 6]
Now, since AC is equal to CE,

the square on AC is also equal to the square on CE;
therefore the squares on AC, CE are double of the square
on AC.

But the square on EA is equal to the squares on AC, CE t
for the angle A CE is right; [1. 47]

therefore the square on EA is double of the square on A C.
Again, since EG is equal to GF,
the square on EG is also equal to the square on GF;

therefore the squares on EG, GF are double of the square on
GF.

But the square on EF is equal to the squares on EG, GF;
therefore the square on EF is double of the square on GF.

But GF is equal to CD; [j. 34)

therefore the square on EF is double of the square on CD.

But the square on EA is also double of the square on AC;

therefore the squares on AE, EFre double of the squares
on AC, CD.

And the square on AF is equal to the squares on AE, EF,
for the angle AEF is right; [1. 47]

therefore the square on AF is double of the squares on AC,
CD.

But the squares on AD, DF are equal to the square on
AF, for the angle at D is right; [1. 47]

therefore the squares on AD, DF are double of the squares
on AC, CD.

And DF is equal to DB;
therefore the squares on AD, DB are double of the squares
on AC, CD.

Therefore etc.

Q. E. D.

It is noteworthy that, while the first eight propositions of Book it. are
proved independently of the Pythagorean theorem i. 47, all the remaining
propositions beginning with the 9th are proved by means of it. Also the 9th
and 10th propositions mark a new departure in another respect; the method
of demonstration by showing in the figures the various rectangles and squares
to which the theorems relate is here abandoned.

The 9th and 10th propositions are related to one another in the same way
as the 5th and 6th; they really prove the same result which can, as in the
earlier case, be comprised in a single enunciation thus: The sum of the squares
on the sum and difference of two given straight lines is equal to twice the sum of
the squares on the lines.

The semi-algebraical proof of Prop, 9 is that suggested by the remark on
the algebraical formulae given at the end of the note on 11. 7. It applies
with a very slight modification to both u. 9 and 11. 10. We will put in
brackets the variations belonging to 11. 10.

The first of the annexed lines is the figure  COB

for 11. 9 and the second for ti. 10. ' '

By 11. 4, the square on AD is equal to a C a D

the squares on AC, CD and twice the ¥ >

rectangle AC, CD.

By 11. 7, the squares on CB, CD {CD, C£) are equal to

twice the rectangle CB, CD together with the square on BD.

By addition of these equals crosswise,
the squares on AD, DB together with twice the rectangle CB, CD are
equal to the squares on AC, CD, CB, CD together with twice
the rectangle AC, CD.

But AC, CB are equal, and therefore the rectangles AC, CD and CB,
CD are equal.

Taking away the equals, we see that

the squares on AD t DB are equal to the squares on AC, CD, CB, CD,

i.e.\ to twice the squares on AC, CD.
To show also that the method of geometrical algebra illustrated by
11. 1 — 8 is still effective for the purpose of
proving 11. 9, 10, we will now prove 11. 9 in
that manner.

Draw squares on AD, DB respectively
as shown in the figure. Measure DH along
DE equal to CD, and HL along HE also
equal to CD.

Draw HK, LNO parallel to EF, and
CNM parallel to DE.

Measure NP along NO equal to CD, F ``Q M E

and draw PQ parallel to DB.

Now, since AD, CD are respectively equal to DE, DH,
HE is equal to AC or CB;
and, since HL is equal to CD, LE is equal to DB.

Similarly, since each of the segments EM, MQ is equal to CD,
EQ is equal to EL or BD.

Therefore OQ is equal to the square on DB.

We have to prove that the squares on AD, DB are equal to twice the
squares on AC, CD.

Now the square on AD includes KM (the square on AC) and CH, HN
(that is, twice the square on CD).

Therefore we have to prove that what is left over of the square on AD
together with the square on DB is equal to the square on AC.

The parts left over are the rectangles CK and NE, which are equal to
KJV, PM respectively.

But the latter with the square on DB are equal to the rectangles KN,
BMand the square OQ,

i.e.\ to the square KM, or the square on AC.

Hence the required result follows.

Proposition 10.

If a straight line be bisected, and a straight line be added
to it in a straight line, the square on the whole with the added
straight line and the square $n the added straight line both
together are double of the square on the half and of the square
described on the straight line made up of the half and the
added straight line as on one straight line.

For let a straight line AB be bisected at C, and let a
straight line BD be added to it in a straight line;

I say that the squares on AD, DB are double of the
squares on AC, CD.

For let CE be drawn from
the point C at right angles to
AB [1. 11], and let it be made
equal to either AC ox CB [1. 3];

let EA, EB be joined;

through E let EF be drawn
parallel to AD,

and through D let FD be drawn
parallel to CE. [1. 31]

Then, since a straight line EF falls on the parallel straight
lines EC, FD,

the angles CEF, EFD are equal to two right angles; [i. a$]
therefore the angles FEB, EFD are less than two right
angles.

But straight lines produced from angles less than two
right angles meet; [i. Post 5]

therefore EB, FD, if produced in the direction B, D, will
meet.

Let them be produced and meet at G,
and let AG be joined.

Then, since A C is equal to CE,
the angle EA C is also equal to the angle AEC; [1. 5]

and the angle at C is right;

therefore each of the angles EAC, AEC is half a right
angle. [1. 32]

For the same reason

each of the angles CEB, EBC is also half a right angle;
therefore the angle AEB is right

And, since the angle EBC is half a right angle,
the angle DBG is also half a right angle. [1. 15]

Rut the angle BDG is also right,
for it is equal to the angle DCE, they being alternate; [1. 19]

therefore the remaining angle DGB is half a right angle;

['• 3'']
therefore the angle DGB is equal to the angle DBG,

so that the side BD is also equal to the side GD, [1. 6]

Again, since the angle EGF is half a right angle,
and the angle at F is right, for it is equal to the opposite
angle, the angle at C, [1. 34]

the remaining angle FEG is half a right angle; [1. 3*]

therefore the angle EGF is equal to the angle FEG,

so that the side GF is also equal to the side EF. [1. 6]

Now, since the square on EC is equal to the square on
CA,
the squares on EC, CA are double of the square on CA.

But the square on EA is equal to the squares on EC, CA;

b- «]

therefore the square on EA is double of the square on A C.

[a k 1]

Again, since FG is equal to EF,
the square on FG is also equal to the square on FE;
therefore the squares on GF, FE are double of the square on
EF

But the square on EG is equal to the squares on GF, FE;

[>• 47]
therefore the square on EG is double of the square on EF.

And EF is equal to CD; [i- 34]

therefore the square on EG is double of the square on CD.
But the square on EA was also proved double of the square
on AC;

therefore the squares on AE, EG are double of the squares
on AC, CD.

And the square on AG is equal to the squares on AE,
EG; [i. 47]

therefore the square on AG is double of the squares on AC,
CD.

But the squares on AD, DG are equal to the square on AG;

[•47]

therefore the squares on AD, DG are double of the squares
on AC, CD.

And DG is equal to DB;
therefore the squares on AD, DB are double of the squares
on AC, CD.

Therefore etc.

Q. E. D.

The alternative proof of this proposition by means of the principles
exhibited in n. i — 8 follows the lines of that
which I have given for the preceding proposition.

It is at once obvious from the figure that the
square on AD includes within it twice the square
on AC together with once the square on CD.
What is left over is the sum of the rectangles AH,
KE. These, which are equivalent to BH, GK,
make up the square on CD less the square on
BD. Adding therefore the square BG to each
side, we have the required result.

Another alternative proof of the theorem which
includes both n. 9 and 10 is worth giving. The
theorem states that the sum of the squares on the

sum and difference of two given straight lines is equal to twite the sum of the
squares on the lines.

Let AD, DB be the two given straight lines (of which AD is the greater),
placed so as to be in one straight line. Make AC equal to DB and com-
pJete the figure as shown, each of the segments CG
and DH being equal to AC or DB. AC B

Now, AD, DB being the given straight lines, AB
is their sum and CD is equal to their difference.

Also AD is equal to BC.

And AE is the square on AB, GK is equal to
the square on CD, AK or Fffia the square on AD,
and BL the square on CB, while each of the small
squares AG, BH, EK, FL is equal to the square on
ACazDB.

We have to prove that twice the squares on AD,
DB are equal to the squares on AB, CD.

Now twice the square on AD is the sum of the squares on AD, CB,
which is equal to the sum of the squares BL, FH , and the figure shows
these to be equal to twice the inner square GK and once the remainder of
the large square AE excluding the two squares AG, KE, which latter squares
are equal to twice the square on AC 01 DB.

Therefore twice the squares on AD, DB are equal to twice the inner
square GK together with once the remainder of the large square AE, that is,
to the sum of the squares AE, GK, which are the squares on AB, CD.

``Side'' and ``diagonal ``numbers giving successive approxi-
mations to J2.

Zeuthen pointed out {Dii Lthre von den Kegehcknitten im Alia- turn, rS86,
pp.~37, 38) that 11. 9, 10 have great interest

in connexion with a problem of indeterminate: g g B

analysis which received much attention from

the ancient Greeks. If we take the straight line AB divided at C and D as

in 11. 9, and if we put CD = x, DB=y, the result obtained by Euclid, namely:

AEP + DB* lAC + i CD*,

or AD t ~iAC = iCD'-DB',

becomes the formula

(tx+y)* — a(x*yf = tx*-jf.

If therefore x, y be numbers which satisfy one of the two equations

2x* —y* — ± 1,

the formula gives us two higher numbers, x+y and 2x +y, which satisfy the
other of the two equations.

Euclid's propositions thus give a general proof of the very formula used
for the formation of the succession of what were called ``side ``and ``diagonal
numbers.''

As is well known, Theon of Smyrna (pp.~43, 44, ed. Hiller) describes this
system of numbers. The unit, being the beginning of all things, must be
potentially both a side and a diameter. Consequently we begin with two units,
the one being the first side and the other the first diameter, and (a) from the
sum of them, (£) from the sum of twice the first unit and once the second, we
form two new numbers

1.1 + 1 = *, 1.1 + 1 = 3.

Of these new numbers the first is a side- and the second a diagonal- umber,
or (as we may say)

flj=2, d,= 3.

In the same way as these numbers were formed from a I = 1, d, = 1, successive
pairs of numbers are formed from o,, d % , and so on, according to the formula

o«+i = o* + Jn rf»+i = **.. + <?»•
Thus a,= a + 3 = 5, d, = 8.3 + 3 = 7,

«*=5 + 7 = ``. rf,= 3. 5 + 7 = 17,
and so on.

Theon states, with reference to these numbers, the general proposition that
d n *=2a n i ± I,
and he observes (i) that the signs alternate as successive d's and n's are taken,
d* - as,' being equal to — 1, /4* - 't equal to + r, df - ia, s equal to I, and
so on, (2) that the sum of the squares of all the d's will be double of the sum
of the squares of all the a's. [If the number of successive terms in each
series is finite, it is of course necessary that the number should be even.]
The proof, no doubt omitted because it was well known, may be put
algebraically thus

rf„* - io,* = (ao,., + <C)' ``* ( a -> + rf «-0*

'= + tJ - an-A m 'ike manner,
ana so on, while d,* - 30,'  - i. Thus the theorem is established.

Euclid's propositions enable us to establish the theorem geometrically;
and this fact might well be thought to confirm the conjecture that the
investigation of the indeterminate equation 2-y'-±i in the manner
explained by Theon was no new thing but began at a period long before
Euclid's time. No one familiar with the truth of the proposition stated by
Theon could have failed to observe that, as the corresponding side- and
diagonal-numbers were successively formed, the value of rf„*/ a »'' would
approach more and more nearly to 3, and consequently that the successive
fractions dja n would give nearer and nearer approximations to the value of

/, mill II il

It is fairly clear that in the famous passage of Plato's Republk (546 c)
about the ``geometrical number'' some such system of approximations is
hinted at Plato there contrasts the ``rational diameter of five' (/Jijti) StafMrpot
tt}<; vtftwdSfK) with the ``irrational ``(diameter). This was certainly taken
from the Pythagorean theory of numbers (cf.\ the expression immediately
preceding, 546 B, C ntn xptxnjyopa no! /Sirra irpw aXkijka dviifnprav, with the
phrase wavta yv<iwra mi rordyopa dWdkois diripytifcrrat in the fragment of

Philolaus). The reference of Plato is to the fol lowin g consideration. If the
square of side 5 be taken, the diagonal is -J 2. 35 or V50. This is the
Pythagorean ``irrati onal d iameter'' of 5; and the ``rational diameter'' was
the approximation V50- 1, or 7.

But the conjecture of Zeuthen, and the attribution of the whole theory of
side- and (amtaAnumbers to the Pythagoreans, have now been fully confirmed
by the publication of Kroll's edition of Prodi Diadochi in Platonis rempuilieam
commentarii (Teubner), Vol. 11., 1901. The passages (cc. 33 and 37, pp.~34,
15 and 2 j — 39) which there saw the light for the first time describe the same
system of forming side- and diagen a /-numbers and definitely attribute it, as
well as the distinction between the ``rational ``and ``irrational diameter,'' to
the Pythagoreans. Procl us further says (p.~27, 16 — 22) that the property of the
side- and diagonal-naxabtts ``is proved graphically (ypafifitxmt) in the second
book of the Elements by 'Aim' (aV iiciirov). For, if a straight line be bisected
and a straight line be added to it, the square on the whole line including the
added straight line and the square on the latter atone are double of the square on
the half of the original straight line and of the square on the straight line made
up of the half and of the added straight line.'' And this is simply Eucl. 11. 10.
Proclus then goes on to show specifically how this proposition was used to
prove that, with the notation above used, the diameter corresponding to the
side a -t-dixa + d. Let AB be a side and BC equal to it, while CD is the
diameter corresponding to AB, i.e.\ a straight line such that the square on it is
double of the square on AB. (I use the figure supplied by Hultsch on p.~397
of KroU's Vol 11.)

Then, by the theorem of Eucl. 11. 10, the squares on AD, DC are double
of the squares on AB, BD,

But the square on DC (i.e.\ BE) is double of the square on AB; therefore,
by subtraction, the square on AD Is double of the square on BD.

And the square on DF, the diagonal corresponding to the side BD, is
double the square of BD.

Therefore the square on DP is equal to the square on AD, so that jPis
equal to AD.

That is, while the side BD is, with our notation, a + d, the corresponding
diagonal, being equal to AD, is 1a + d.

In the above reference by Proclus to 11. 10 dw' imivav ``by him'' must
apparently mean w BinXxiSov, ``by Euclid,'' although Euclid's name has not
been mentioned in the chapter; the phrase would be equivalent to saying
``in the second Book of the famous Elements.'' But, when Proclus says ``this
is firwed in the second Book of the Elements,'' he does not imply that it had
not been proved before; on the contrary, it is clear that the theorem had
been proved by the Pythagoreans, and we have therefore here a confirmation
of the inference from the part played by the gnomon and by 1. 47 in Book 11.
that the whole of the substance of that Book was Pythagorean. For further
detailed explanation of the passages of Proclus reference should be made to
Hultsch's note in KroU's Vol. 11. pp.~393 — 400, and to the separate article,
also by Hultsch, in the Bibliotheca Mathematica I,, 1900, pp.~8 — 12.

p.~Bergh has an ingenious suggestion (see Zeitschrift fur Math. u. Physik


xxxi. Hist-titL Abt. p.~135, and Cantor, Geschithte der Mathematik, i„ p.~437)

as to the way in which the formation of the successive

side- and diagonal-numbers may have been discovered,

namely by observation from a very simple geometrical

figure. Let ABC be an isosceles triangle, right-angled at

A, with sides o._i, «„-,, d n _, respectively. If now the

two sides AS, AC about the right angle be lengthened

by adding </„_, to each, and the extremities D, E be

joined, it is easily seen by means of the figure (in which

BF, CG are perpendicular to DE) that the new diagonal

d m a equal to aa„- y + f/„_, , while the equal sides a m are, by construction, equal

to a,-! + rf,-!.

Important deductions from II. g, 10.

I. Pappus (vii. pp.~856 — 8) uses 11. 9, 10 for proving tne well-known
theorem that

The sum of the squares on two sides of a triangle is equal to twice the square
oh half the base together with twice the square on the straight lint Joining the
Middle point of the base to the opposite vertex.

Let ABC be the given triangle and D the middle point of the base BC.
Join AD, and draw AE perpendicular to BC (produced if necessary).

C E

Now, by 11. 9, 10,
the squares on BE, EC are equal to twice the squares on BD, DE.
Add to each twice the square on AE.
Then, remembering that

the squares on BE, EA are equal to the square on BA,
the squares on AE, EC are equal to the square on A C,
and the squares on AE, ED are equal to the square on AD,
we find that

the squares on BA, AC axe equal to twice the squares on AD, BD,
The proposition is generally proved by means of 11. 12, 13, but not, I
think, so conveniently as by the method of Pappus.

II. The inference was early made by Gregory of St. Vincent (1584-166;)
and Vivian i (1622-1703) that In any parallelogram the squares on the diagonals
are together equal to the squares on the sides, or to twice the squares on adjacent
sides.

III. It appears that Leonhard Euler (1 707-83) was the first to discover
the coiresponding theorem with reference to any quadrilateral, namely that
In any quadrilateral the sum of the squares on the sides is equal to the sum of the
squares on the diagonals and four limes the square on the tine Joining the middle
points of the diagonals. Euler seems however to have proved the property
from the corresponding theorem for parallelograms just quoted (cf.\ Camerer's
Euclid, Vol. I. pp.~468, 469) and not from the property of the triangle, though
the latter brings out the result more easily.

Proposition i i.

To cut a given straight line so that the rectangle contained
by the whole and one of the segments is equal to the square on
the remaining segment.

Let AB be the given straight line;

thus it is required to cut AB so that the rectangle contained

by the whole and one of the segments is

equal to the square on the remaining ft

segment.

For let the square ABDC be described
on AB; [1. 46]

let AC be bisected at the point E, and let
BE be joined;

let CA be drawn through to F, and let EF
be made equal to BE;

let the square FH be described on AF, and
let Gbe drawn through to K.

I say that AB has been cut at H so as to make the
rectangle contained by AB, BH equal to the square on AH.

For, since the straight line AC has been bisected at E,
and FA is added to it,

the rectangle contained by CF, FA together with the
square on AE is equal to the square on EF. [n. 6]

But EF is equal to EB;

therefore the rectangle CF, FA together with the square
on AE is equal to the square on EB.

But the squares on BA, AE are equal to the square on
EB, for the angle at A is right: [1. 47]

therefore the rectangle CF, FA together with the square
on AE is equal to the squares on BA, AE.

Let the square on AE be subtracted from each;

therefore the rectangle CF, FA which remains is equal to
the square on AB,

Now the rectangle CF, FA is FK, for AF is equal to
FG\
and the square on AB is AD;

therefore FK is equal to AD,

Let AKbe subtracted from each;

therefore FH which remains is equal to HD.

And HD is the rectangle AB, BH, for AB is equal to
BD;
and /!# is the square on AH;

therefore the rectangle contained by AB, BH is equal
to the square on HA.

therefore the given straight line AB has been cut at H
so as to make the rectangle contained by AB, BH equal to
the square on HA.

Q. E. F,

As the solution of this problem is necessary to that of inscribing a regular
pentagon in a circle (Eucl. iv. 10, n), we must necessarily conclude that it
was solved by the Pythagoreans, or, in other words, that they discovered the
geometrical solution of the quadratic equation

*(«-*) = *.
or * J + ax = «*.

The solution in 11. 11, too, exactly corresponds to the solution of the more
general equation

x* + ax = £*,
which, as shown above (pp.~387 — 8), Simson based upon 11. 6. Only Sijnson's
solution, if applied here, gives us the point F cm CA produced and does not
directly find the point //. It takes £ the middle point of CA, draws AB at
right angles to CA and of length equal to CA, and then describes a circle
with EB as radius cutting EA produced in F. The only difference between
the solution in this case and in the more general case is that AB is here equal
to CA instead of being equal to another given straight line b.

As in the more general case, there is, from Euclid's point of view, only one
solution.

The construction shows that CF is also divided at A in the manner
described in the enunciation, since the rectangle CF, FA is equal to the
square on CA.

The problem in 11. 11 reappears in vi. 30 in the form of cutting a given
straight tint in extreme and mean ratio.

Proposition 12.

In obtuse-angled triangles the square on the side subtending
the obtuse angle is greater than the squares on the sides con-
taining ike obtuse angle by twice the rectangle contained by one
of the sides about the obtuse angle, namely that on which the
perpendicular falls, and the straight line cut off outside by the
perpendicular towards the obtuse angle.

Let ABC be an obtuse-angled triangle having the angle
JiAC obtuse, and let BD be drawn from the point B per-
pendicular to CA produced;

I say that the square on BC is greater than the squares
on BA, AC by twice ihe rectangle con-
tained by CA, AD.

For, since the straight line CD has
been cut at random at the point A,
the square on DC is equal to the
squares on CA, AD and twice the rect-
angle contained by CA, AD. [«. 4]

Let the square on DB be added to
each;

therefore the squares on CD, DB are equal to the squares on
CA, AD, DB and twice the rectangle CA, AD.

But the square on CB is equal to the squares on CD, DB,
for the angle at D is right; [1. 47]

and the square on AB is equal to the squares on AD,
DB; [1.47]

therefore the square on CB is equal to the squares on CA, AB
and twice the rectangle contained by CA,-AD;

so that the square on CB is greater than the squares on
CA, AB by twice the rectangle contained by CA. AD.

Therefore etc.\ q. e. d.

Since in this proposition and the next we have to do with the squares on
the sides of triangles, the particular form of graphic representation of areas
which we have had in Book n. up to this point does not help us to visualise
the results of the propositions in the same way, and only two lines of proof
axe possible, (1) by means of the results of certain earlier propositions in
Book il combined with the result of i. 47 and (2) by means of the procedure
in Euclid's proof of 1. 47 itself. The alternative proofs of ii. 12, 13 after the
manner of Euclid's proof of 1. 47 are therefore alone worth giving.

These proofs appear in certain modern text-books (e.g.\ Mehler, Henrict and
Treutlein, H. M. Taylor, Smith and Bryant). Smith and Bryant are not
correct in saying (p.~142) that they cannot be traced further back than
Lardner's Euclid (i8a8); they are to be found in Gregory of St Vincent's
work (published in 1647) Opus geemttricum quadraturtu circuK ei sectionum
eoni, Book 1. Pt 2, Props. 44, 45 (pp.~3r, 31).

To prove 11. 12, take an obtuse-angled triangle ABC in which the angle at
A is the obtuse angle

Describe squares on BC, CA, AB, as BCED, CAGF, ABKH.

Draw AL, BM, CN, perpendicular to BC, CA, AB (produced if neces-
sary), and produce them to meet the further
sides of the squares on them in P, Q, R re-
spectively.

Join AD, CK.

Then, as in t. 47, the triangles KBC, ABD
are equal in all respects;
therefore their doubles, the parallelograms in
the same parallels' respectively, are equal;

that is, the rectangle BP is equal to the
rectangle BR.

Similarly the rectangle CP is equal to the
rectangle CQ.

Also, if BG, CH be joined, we see that
the triangles BAG, HAC are equal in
all respects;
therefore their doubles, the rectangles AQ, AR, are equal.

Now the square on BCis equal to the sum of the rectangles BP, CP,

ie. to the sum of the rectangles BR, CQ,

i.e.\ to the sum of the squares BH, CG and

the rectangles AR, A Q.

But the rectangles AR, AQ are equal, and they are respectively the
rectangle contained by BA, ANacaA the rectangle contained by CA, AM.

Therefore the square on BCis equal to the squares on BA, AC together
with twice the rectangle BA, AN~ar CA, AM.

Incidentally this proof shows that the rectangle BA, AN is equal to, the
rectangle CA, AM: a result which will be seen later on to be a particular
case of the theorem in in, 35.

Heron (in an-NairlsI, ed. Curtze, p.~109) gives a ``converse'' of il u
related to it as 1. 48 is related to 1. 47.

In any triangle, if the square on one of the sides is greater than the squares
on the other two sides, the angle contained by the latter is obtuse.

Let ABC be a triangle such that the square on BC is greater than the
squares on BA, AC.

Draw AD at right angles to AC and
of length equal to AB.

Join DC.

Then, since DAC is a right angle,
the square on DC is equal to the squares
on DA, AC, [1. 47]

i.e.\ to the squares on BA, A C.

But the square on BC is greater than
the squares on BA, AC; therefore the square on BC is greater than the
square on DC.

Therefore BC is greater than DC.

Thus, in the triangles BAC, DAC, *
the two sides BA, AC axe equal to the two sides DA, A C respectively,
but the base BC h greater than the base DC.

Therefore the angle BAC is greater than the angle DAC; [1. »s]

that is, the angle BA C is obtuse.

Proposition 13.

In acute-angled triangles the square on the side subtending
the acute angle is less than the squares on the sides containing
the acute angle by twice the rectangle contained by one of the
sides about the acute angle, namely that on which the per-
pendicular falls, and the straight line cut off within by the
perpendicular towards the acute angle.

Let ABC be an acute- angled triangle having the angle
at B acute, and let AD be drawn from the point A perpen-
dicular to BC;

I say that the square on AC is less than the squares on
CB, BA by twice the rectangle contained
by CB, BD,

For, since the straight line CB has
been cut at random at D,

the squares on CB, BD are equal to
twice the rectangle contained by CB, BD
and the square on DC. [». 7]

Let the square on DA be added to
each;

therefore the squares on CB, BD, DA are equal to twice
the rectangle contained by CB, BD and the squares on AD,
DC.

But the square on AB is equal to the squares on BD,
DA, for the angle at D is right; [i. 47]

and the square on AC is equal to the squares on AD, DC;
therefore the squares on CB, BA are equal to the square on
A C and twice the rectangle CB, BD,

so that the square on AC alone is less than the squares
on CB, BA by twice the rectangle contained by CB, BD.

Therefore etc.

q. e. d.

As the text stands, this proposition is unequivocally enunciated of aeutt-
angttd triangles; and, as if to obviate any doubt as to whether the restriction
was fully intended, the enunciation speaks of the rectangle contained by one
of the sides containing the acute angle and the straight line intercepted
within by the perpendicular towards the acute angle. On the other hand, it
is curious that it speaks of tbe square on the side subtending the acute angle;
and again the setting-out begins ``let ABC lie an acute-angled triangle having
the angle at B acute,'' though the last words have no point if all the angles of
the triangle are necessarily acute.

It was however very early noticed, not only by Isaacus Monachus,
Cam pan us, Peletarius, Clavius, Commandinus and the rest, but by the Greek
scholiast (Heiberg, VoL v, p.~253), that the relation between the sides of a
triangle established by this theorem is true of the side opposite to, and the
sides about, an acute angle respectively in any sort of triangle whether acute-
angled, right-angled or obtuse-angled. The scholiast tries to explain away the
word ``acute-angled'' in the enunciation: ``Since in the definitions he calls
acute-angled the triangle which has three acute angles, you must know that he
does not mean that here, but calls all triangles acute-angled because all have
an acute angle, one at least, if not all The enunciation therefore is: ' In any
triangle the square on the side subtending the acute angle is less than the
squares on the sides containing the acute angle by twice the rectangle, etc' ``

We may judge too by Heron's enunciation of his ``converse'' of the
proposition that he would have left the word ``acute-angled ``out of the
enunciation. His converse is: In any triangle in which the square on one of
the sides is less than the squares on the other two sides, the angle contained by the
latter sides is acute.

If the triangle that we take is a right-angled triangle, and the perpendicular
is drawn, not from the right angle, but from the acute angle
not referred to in the enunciation, the proposition reduces
to 1. 47, and this case need not detain us.

The other cases can be proved, like 11. is, after the
manner of 1. 47.

Let us take first the case where all the angles of the
triangle are acute.

e p

As before, if we draw ALP, BMQ, CNR perpendicular to BC, CA, AB
and meeting the further sides of the squares on BC, CA, AB in P, Q, R, and
if we join KC, AD, we have

the triangles KBC, ABD equal in all respects,
and consequently the rectangles BP, BR equal to one another.
Similarly the rectangles CP, CQ are equal to one another.

Next, by joining BG, CH, we prove in like manner that the rectangles AR,
AQaie equal.

Now the square on BC is equal to the sum of the rectangles BF, CP,

i.e.\ to the sum of the rectangles BR, CQ,

i.e.\ to the sum of the squares BH, CG diminished by the rectangles
AR, AQ.
But the rectangles AR, AQ are equal, and they are respectively the
rectangles contained by BA, AN and by CA, AM.

Therefore the square on JC is less than the squares on BA, AC by
twice the rectangle BA, AN or CA, AM.

Next suppose that we have to prove the theorem in the case where the
triangle has an obtuse angle at A.

Take B as the acute angle under considera-
tion, so that AC is the side opposite to it.

Now the square on CA is equal to the
difference of the rectangles CQ, AQ,

i.e.\ to the difference between CP and

AQ,
Le. to the difference between the square
BE and the sum of the rectangles
BP, AQ,
i.e.\ to the difference between the square
BE and the sum of the rectangles
BP, AR,

i.e.\ to the difference between the sum of
the squares BE, BH and the sum
of the rectangles BP, BR

(since AR is the difference between BR and BH).

But BP, BR are equal, and they are respectively the rectangles CB, BL
and A B, BN.

Therefore the square on CA is less than the squares on AB, BC by twice
the rectangle CB, BL or AB. BN.

Heron's proof of his converse proposition (an-NairlzI, ed, Curtze, p.~1 10),
which is also given by the Greek scholiast above quoted,
is of course simple. For let ABC be a triangle in which
the square on AC is less than the squares on AB, BC.

Draw BD at right angles to BC and of length equal
to BA.

Join DC.

Then, since the angle CBD is right,
the square on DC is equal to the squares on DB, BC,
Le. to the squares on AB, BC. [1. 47}

But the square on AC is less than the squares on
AB, BC.

Therefore the square on AC is less than the square on DC.

Therefore AC is less, than DC.

Hence in the two triangles DBC, ABC the sides about the angles DBC,

ABC are respectively equal, but the base DC is greater than the base AC.

Therefore the angle DBC {a right angle) is greater than the angle ABC
[l. 25], which latter is therefore acute.

It may be noted, lastly, that 11. 13, 13 are supplementary to 1. 47 and
complete the theory of the relations between the squares on the sides of any
triangle, whether right-angled or not.

Proposition 14.

To construct a square equal to a given rectilineal figure.

Let A be the given rectilineal figure;
thus it is required to construct a square equal to the rectilineal
figure A.

s For let there be constructed the rectangular parallelogram
BD equal to the rectilineal figure A. [1. 45]

Then, if BE is equal to ED, that which was enjoined
will have been done; for a square BD has been constructed
equal to the rectilineal figure A.
10 But, if not, one of the straight lines BE, ED is greater.
Let BE be greater, and let it be produced to F\
let EF be made equal to ED, and let BF be bisected at G.

With centre G and distance one of the straight lines GB,
GF let the semicircle BHF be described; let DE be produced
is to H, and let GH be joined.

Then, since the straight line BF has been cut into equal
segments at G, and into unequal segments at E,

the rectangle contained by BE, EF together with the
square on EG is equal to the square on GF. [u. 5]

m But GF is equal to GH;
therefore the rectangle BE, EF together with the square on
GE is equal to the square on GH.

But the squares on HE, EG are equal to the square on
GH; [1. 47]

as therefore the rectangle BE, EF together with the square on
GE is equal to the squares on HE, EG.

Let the square on GE be subtracted from each;
therefore the rectangle contained by BE, EF which
remains is equal to the square on EH.
30 But the rectangle BE, EF is BD, for EF is equal to ED;

therefore the parallelogram BD is equal to the square on
HE.

And BD is equal to the rectilineal figure A.

Therefore the rectilineal figure A is also equal to the square
3j which can be described on EH.

Therefore a square, namely that which can be described
on EH, has been constructed equal to the given rectilineal
figure A. q. e. f.

7. that which wan enjoined will have been done, literally ``would have been
done,'' >Fy»it t> ttj} tA iinTaxfr.

35, 36- which can be described, expressed by the future passive participle, iyvypaij.

Heiberg {Mathematisches zu AristoteUs, p.~»o) quotes as bearing on this
proposition Aristotle's remark (De ant ma 11. 2, 413 a 19: cf.\ Metaph. 996 b zi)
that ``squaring ``(Terpavuii'Krftoe) is better defined as the ``finding of the mean
(proportional) ``than as ``the making of an equilateral rectangle equal to a
riven oblong,'' because the former definition states the cause, the latter the
:ondusion only. This, Heiberg thinks, implies that in the text-books which were
in Aristotle's hands the problem of 11. 14 was solved by means of proportions.
As a matter of fact, the actual construction is the same in 11. 14 as in vi. 13;
and the change made by Euclid must have been confined to substituting in
the proof of the correctness of the construction an argument based on the
principles of Books 1. and 11. instead of Book vi.

As n. t2, 13 are supplementary to 1. 47, so 11. 14 completes the theory of
transformation of areas so far as it can be carried without the use of proportions.
As we have seen, the propositions 1. 42, 44, 45 enable us to construct a
parallelogram having a given side and angle, and equal to any given rectilineal
figure. The parallelogram can also be transformed into an equal triangle with
the same given side and angle by making the other side about the angle twice
the length. Thus we can, as a particular case, construct a rectangle on a
given base (or a right-angled triangle with one of the sides about the right
angle of given length) equal to a given square. Further, 1. 47 enables us
to make a square equal to the sum of any number of squares or to the
difference between any two squares. The problem still remaining unsolved is
to transform any rectangle (as representing an area equal to that of any
rectilineal figure) into a square of equal area. The solution of this problem,
given in 11. 14, is of course the equivalent of the extraction of the square root,
or of the solution of the pure quadratic equation

x* = ai.

Simson pointed out that, in the construction given by Euclid in this case,
it was not necessary to put in the words ``Let BE be greater,'' since the
construction is not affected by the question whether BE or ED is the greater.
This is true, but after all the words do little harm, and perhaps Euclid may
have regarded it as conducive to clearness to have the points B, G, E, F in
the same relative positions as the corresponding points A, C, D, B in the
figure of 11. 5 which he quotes in the proof.

\backmatter

\appendix

EXCURSUS I.

PYTHAGORAS AND THE PYTHAGOREANS.

The problem of determining how much of the Pythagorean discoveries in
mathematics can be attributed to Pythagoras himself is not only difficult; it
may be said to be insoluble. Tradition on the subject is very meagre and
uncertain, and further doubt is thrown upon it by the well-known tendency of
the later Pythagoreans to ascribe everything to the Master himself {airot iifta,
Ipse dixit). Pythagoras himself left no written exposition of his doctrines, nor
did any of his immediate successors, not even Hippasus, about whom the
different stories ran ( t ) that he was expelled from the school because he pub-
lished doctrines of Pythagoras, and (2) that he was drowned at sea for revealing
the construction of the dodecahedron in the sphere and claiming it as his own,
or (as others have it) for making known the discovery of the irrational or in-
commensurable. Nor is the absence of any written record of Pythagorean
doctrines down to the time of Philolaus to be put down to a pledge of secrecy
binding the school; at all events this did not apply to their mathematics or
their physics; and it may be that the supposed secrecy was invented to
account for the absence of documents. The fact seems to be that oral com-
munication was the tradition of the school, while their doctrines would in the
main be too abstruse to be understood by the generality of people outside.
Even Aristotle felt the difficulty; he evidently knew nothing for certain about
any ethical or physical doctrines going back to Pythagoras himself; when he
speaks of the Pythagorean system, he always refers it to ``the Pythagoreans.''
sometimes even to ``the so-called Pythagoreans.''

Since my note on Eucl. \prop{1}{47} was originally written the part of Pythagoras
in the Pythagorean mathematical discoveries has been further discussed and
every scrap of evidence closely, and even meticulously, examined in two long
articles by Heinrich Vogt, ``Die Geometrie des Pythagoras ``{BibUotheca
Maihemafica ix„ 1908/9, pp.~15 — 54) and ``Die Entstehungsgeschichte des
Irrationalen nach Plato und anderen Quelien des 4. Jahrhunderts ``(Siblie-
theax Mathematita x„ 1910, pp.~97 — 155). These papers would not indeed
have enabled me to modify greatly what I have written regarding the supposed
discoveries of Pythagoras and the early Pythagoreans, because I have through-
out been careful to give the traditions on the subject for what they are worth
and no more, and not to build too much upon them. It is right however to
give, in a separate note, a few details of Vogt's arguments.

G. Junge had, in his paper ``Wann haben die Griechen das Irrationale
entdeckt?'' mentioned above (p.~351), tried to prove that Pythagoras himself
could not have discovered the irrational; and the object of Vogt s papers is to
go further on the same lines and to show (1) that it was only the later
Pythagoreans who (before 410 B.C.) recognised the incommensurability of
the diagonal with the side of a square, (2) that the theory of the irrational was
first discovered by Theodorus, to whom Plato refers ( Theaetttvs 141 d), and
(3) that Pythagoras himself could not have been the discoverer of any one of
the things specifically attributed to him, namely (a) the theorem of Eucl. 1. 47,
(d) the construction of the five regular solids in the sense in which they are
respectively constructed in Eucl. xm, (c) the application of an area in its
widest sense, equivalent to the solution of a quadratic equation in its most
general form.

Vogt's main argument as regards (a) the theorem of 1. 47 is based on a
new translation which he gives of the well-known passage of Proclus' note on
the proposition (p.~426, 6 — -9), Ttijk fLtv Itrroptiv ra dp%aia fiavkopivw oxovovTa?
to Qiwpij/ta toSto tit Tlvtiayopar avairtjairwruii' larlv tvpilv xal /$ov8vrt)v \ty6vra»>
avrir iwl rj} cupttrtt Vogt translates this as follows: ``Unter den en, welche
das Altertum erforschen wollen, kann man einige tin den, welche denen Gehor
geben, die dieses Theorem auf Pythagoras zuruckfiihren und ihn als Stier-
opferer bei dieser Gelegenheit bezeichnen,'' ``Among those who have a taste
for research into antiquity, we can find some who give ear to those who refer
this theorem to Pythagoras and describe him as sacrificing an ox on the
strength of the discovery.'' According to this version the words t£v...
ftovkopJvwv and the words dwnvfpTrovTviv.,,Kal.,,\iy6i>T<o)' refer respectively to
two different sets of persons, in fact two different generations; tlje latter are
older authorities who are supposed to be cited by the former; the former are
a later generation, perhaps contemporaries of Proclus, some of whom accepted
the view of the older authorities while others did not. But this would have
required the article rwv before avajK/nron-wv, or some such expression as
aWwv Tii'Lt- ot aiij.Tri.TTi-iv'Tir instead of ayaircim-drrwy, Vogt's interpretation is
therefore quite inadmissible. The persons denoted by draTrtpirorTtav are some
of the persons denoted by run fjovkoittvuiv; hence Tannery's translation, to
which mine (p.~350 above) is equivalent, is the only possible one, namely
``Si Ton ecoute ceux qui veulent raconter 1'histoire des anciens temps, on
peut en trouver qui attribuent ce theoreme  Pythagore et lut font sacrifier un
boeuf apres sa decouverte ``{La Giotnttrit grecoue, p.~103). tsaSomat agrees
with the assumed subject of tvpilv; dviartfiviyrmv and ktyovrior should, strictly
speaking, have been dvairipTrovrat and Xcyovro.? agreeing with i-ivas (the direct
object of tvptly) understood, but are simply attracted into the case of jSov-
kdfiivia/; the construction is quite intelligible. I agree with Vogt that
Kudo mus' history contained nothing attributing the theorem to Pythagoras.
The words of Proclus imply this; but I do not think that they imply (as
Vogt maintains) any pronouncement by Proclus hi ruse If against such attribution.
In my opinion, Proclus is simply determined not to commit himself to any
view; his way of evading a decision is the sentence following, iyii Si Sav/idlm
f±*y Ktti roils irpwTovs 1 Trccrrafras ry touSc to£ 8f u>prjpa.TO<; aXfo t'rj , p.fiQ'>vw<i Si ayaat
TOf OTotx<miTi|V,..; the plural rots irpuirous /jrio-Tai'Tai is, I hold, used for the
very purpose of making the statement as vague as possible; he will not even
allow it to be inferred that he attributed the discovery to any single person.
Returning to 1$ rar dkoyiav irpaypanla. {Proclus, p.~65, rii), we may now
concede (following Diels) that we should read tuv dra 6yov (``proportionals ``)
instead of ™ d6y<av (``irrationals'') and that the author intended to attribute
to Pythagoras a theory of proportion (the arithmetical theory applicable to
commensurable magnitudes only) rather than the theory of irrationals. But
I do not agree in Vogt's contention that the theory of the irrational was first
discovered by Theodorus. It seems to me that we have evidence to the
contrary in the very passage of_Plato referred to. Plato {Thtattetus 147 d)
mentions J 3, 5, ... up to ``A 7 as dealt with by Theodorus, but omits Ji.
This fact, along with Plato's allusions elsewhere to the irrationality of J 2, and
to approximations to it, in the expressions ilppijrw and prfnj Sta/ttTpoi ttji
wifnraas, as if those expressions had a well-known signification, implies that
the discovery of the irrationality of J2 had been made before the time of
Theodorus. The words 7 rur aXayiav irpayfiariia might well be used even
if the reference is only to 2, because the first step would be the most
difficult, and irpayftartut need not mean the establishment of a complete theory
of anything more than ``investigation ``of a subject.

Junge and Vogt hold that the theory of the irrational was not discovered
by the early Pythagoreans any more than Pythagoras because, if it had been
so discovered, an impossibly long period would intervene between the investi-
gation of the particular case of J 2 and the extension of the theory by
Theodorus to the cases of 3, J$ etc But might not this well be due to the
fact that in the meantime the minds of geometers were engrossed by other
problems of importance, namely the quadrature of the circle {Hippocrates of
Chios and his quadratures of lunes), the tri section of any angle (Hippias of El is
and his curve, afterwards known as the quadratrix\ and the doubling of the
cube (reduced by Hippocrates to the problem of finding two mean pro-
portionals in continued proportion between two given straight lines), the last
of which problems, which meant finding geometrically the equivalent of $?,
would naturally follow the investigation of J 2 ? Now Hippias was probably
bom about 460 b.c., while Hippocrates seems to have been in Athens during
a considerable portion of the second half of the fifth century, perhaps from
about 450 to 430 n.c. Moreover Vogt has to get over the fact that Democritus
(born 470/469 B.C.) wrote a book mpl AXyiov ypa/ipuir ko.1 vatrrun; On irrational
lines and solids (or atoms). This difficulty he seeks to overcome by maintaining
that oAo'yw does not here mean ``irrational ``at all, but ``without ratio ``
(``verhaltnislos ``), in the sense that any two straight lines are ``without ratio ``
because they both contain an infinite number of the indivisible (or atomic)
lines, and therefore their ratio, being of the form «>/«>, is indeterminate.
But, if these were so, all lines (including commensurable lines) would be
``without a ratio'' to one another, whereas the title of Democritus' work
clearly implies that SXoyoi -ypau/uu are a class or classes of lines distinguished
from other lines. The fact is that Democritus was too good a mathematician
to have anything to do with ``indivisible lines.'' This is confirmed by a
scholium to Aristode's lie eaclo (p.~469 b 14, Brandis) which implicitly
denies to Democritus any theory of indivisible lines: ``of those who have
maintained the existence of indivisibles, some, as for example Leucippus and
Democritus, believe in indivisible bodies, others, like Xenocrates, in indivisible
lines.'' Moreover Simpiicius tells us that, according to Democritus himself,
even the atoms were, in a mathematical sense, divisible further and in fact
ad infinitum.

Coming now to {b) the construction of the cosmic figures, 9 rmr KoajUKv
iT)(r)tta-ra>v uvrrnuTii (Proclus, p.~65, 20), I agree with Vogt to the following
extent. It is unlikely that Pythagoras or even the early Pythagoreans ``con-
structed'' the five regular solids in the sense of a complete theoretical con-
struction such as we find, say, in Eucl. xm.; and it is possible that Theaetetus
was the first to give these constructions, whether ty paft in Suidas' notice,
jrpurov Si to «it< KaXov/uva artpta. typrnjit, means ``constructed ``or ``wrote
upon.'' But cruoracri! in the above phrase of Froclus may well mean something
less than the theoretical constructions and proofs of Eucl. xm.; it may mean,
as Vogt says, simply the ``putting together ``of the figures in the same way as
Plato puts them together in the JYmaeus, i.e.\ by bringing a certain number of
angles of equilateral triangles and of regular pentagons together at one point.
There is no reason why the early Pythagoreans should r.ot have. ``constructed ``
the five regular solids in this sense; in fact the supposition that they did so
agrees well with what we know of their having put angles of certain regular
figures together round a point (in connexion with the theorem of Eucl. \prop{1}{32}) and
shown that only three kinds of such angles would fill up the space in one plane
round the point. But I do not agree in the apparent refusal of Vogt to credit
the Pythagoreans with the knowledge of the theoretical construction of the
regular pentagon as we find it in Eucl. iv. 10, 1 1. I do not know of any
reason for rejecting the evidence of the Scholia iv, Nos. 2 and 4 which say
categorically that ``this Book ``{Book iv.) and ``the whole of the theorems ``
in it {including therefore Props. 10, n) are discoveries of the Pythagoreans.
And the division of a straight line in extreme and mean ratio, on which the
construction of the regular pentagon depends, comes in Eucl. Book n.
(Prop, n), while we have sufficient grounds for regarding the whole of the
substance of Book a. as Pythagorean.

I will permit myself one more criticism on Vogt's first paper. I think he
bases too much on the fact that it was left for Oenopides (in the period from,
say, 470 to 450 B.C.) to discover two elementary constructions (with ruler and
compasses only), namely that of a perpendicular to a straight line from an
external point (Eucl. 1. 1 2), and that of an angle equal to a given rectilineal
angle (Eucl. 1. 23). Vogt infers that geometry must have been in a very
rudimentary condition at the time. I do not think this follows; the explana-
tion would seem to be rather that, the restriction of the instruments used
in constructions to the ruler and compasses not having been definitely estab-
lished before the time when Oenopides wrote, it had not previously occurred to
anyone to substitute new constructions based on that principle for others
previously in vogue. In the case of the perpendicular, for example, the con-
struction would no doubt, in earlier days, have been made by means of a set
square.

EXCURSUS II.

POPULAR NAMES FOR EUCLIDEAN PROPOSITIONS.

Although some of these time-honoured names are famijiar to most educated
people, it seems to be impossible to trace them to their original sources, or to
say who applied them for the first time respectively. It may be that they were
handed down by oral tradition for long periods in each case before they found
their way into written documents.

We begin with

I. S .

i. This proposition is in this country universally known as the Pons
Asinorum, ``Asses' Bridge.'' Even in this case opinion is not unanimous as
to the exact implication of the term. Perhaps the more general view is that
taken in the Stanford Dictionary of Anglicistd Words and Phrases (by
C. A. M. Fennell) where the description is: ``Name of the fifth proposition
of the first Book of Euclid, suggested by the figure and the difficulties which
poor geometricians find in mastering it.'' This is certainly the equivalent of
what I gathered, in my early days at school, from a former Fellow of St John's,
the Reverend Anthony Bower, who was a high Wrangler in 1846 and a friend
of Todhunter's. The ``ass ``on this interpretation is a synonym for ``fool.''
But there is another view (as I have learnt lately) which is more complimentary
to the ass. It is that, the figure of the proposition being like that of a trestle-
bridge, with a ramp at each end which is the more practicable the flatter the
figure is drawn, the bridge is such that, while a horse could not surmount the
ramp, an ass could; in other words, the term is meant to refer to the s u re footed -
ness of the ass rather than to any want of intelligence on his part. (I may
perhaps mention that Sir George Green hill is a strong supporter of this view.)

An epigram of 1780 is the earliest reference to the term in Murray's
English Dictionary:

``If this be rightly called the bridge of asses,
He's not the fool that sticks but he that passes.''

The writer's own view is not too clear. He seems to imply that, while the
inventor of the name msant that only the fool finds the bridge difficult to
pass, the more proper view would be that, since the ass can get over, and
``ass ``is synonymous with ``fool,'' therefore it must be the fool who can get
over; in other words, he seems to object to the phrase as being a contradic-
tion in terms.

But we have also to take account of the fact that the French apply the
term to t. 47. Now in Euclid's figure for 1. 47 there is no suggestion of a
bridge, and the reference can only be to the nature of the theorem, its diffi-
culty or otherwise. It is curious that the French dictionaries give two different
explanations of Pont aux ana, Littre makes it ``ce que person ne ne doit ni
ne peut ignorer; ce qui est si facile que tout le monde doit y reussir.'' Now
no intelligent person could have applied the name to Eucl. 1. 47 for this
reason, namely that it was so easy that even a fool could not help knowing it.
Larousse is better informed; there we find ``Pont aux ants, certaine difficulte,
certainc question qui n'arrete que les ignorant s, et qui sert de criterium
pour juger l'intelligence de quelqu'un, et partieulierement d'un ecolier. C'est
ainsi que, dans Its classes de mathematiques, on ne manque jamais dt dire
que le carre - de I'hypotenuse est le pant aux dries de la geometric La plupatt
des dictionnaires entendent par ce mot une chose si -simple, si facile, que
personne ne doit l'ignorer: c'est une erreur e''vidente.'' Larousse is clearly right.
But it will be observed that, so far as it goes, Larousse's interpretation rather
supports the first of the two alternative explanations of the meaning of ``Asses'
Bridge'' as applied to \prop{1}{5}, namely that it is difficult for the fool (= ``ass'') to
master.

In the Stanford Die tiff nary it is added that ``in logic the term was in the
16 c. applied to the conversion of propositions by the aid of a difficult
diagram for finding middle terms''; and if the mathematicians borrowed the
term from logic, this again would be rather in favour of the first explanation
of its use for 1. 5.

If it is permitted deufiere in loco, I would add for the benefit of future
generations (in the hope that they will still be able to appreciate the joke or,
in the alternative, will be tempted to discuss learnedly what could possibly
have been meant) a very topical allusion in a recent Punch (14 Oct. 1935):
``When they film Euclid, as is suggested, we shall no doubt see a very
thrilling rescue over the burning Pons Asinorum.'' — And yet it is safe to
prophesy that the ``Asses' Bridge ``will outlive the ``film ``I

2. Elefuga.

This name for Eucl. 1. 5 is mentioned by Roger Bacon (about 1250), wno
also gives an explanation of it (Opus Tertium, c. vi). He observes that in his
day people in general, finding no utility in any science such as geometry, for
example, recoiled from the idea of studying it unless they were boys farced to
it by the rod, so that they would hardly learn so much as three or four pro-
positions. Hence it is, he says, that the fifth proposition is called ``Elefuga,
id est, fuga miserorum; elegia enim Graece dicitur, Latine miseria; et elegi
sunt miseri.'' That is, according to Roger Bacon, Elefuga is ``flight of the
miserable.'' This explanation no doubt accounts for the verses about Dul-
carnon in Chaucer's Troilus and Criseyde, hi, 11. 933-5:
``Dulcarnon called is * fltminge of wrecches';
It secmeth hard, for wrecches wol not lere
For venay slouthe or othere wilful tecches'';

since ``fleminge of wrecches,'' ``banishment of the miserable,'' is a translation
of ``fuga miserorum.'' Only Dulcarnon is there wrongly taken to be the same
proposition as Elefuga, i.e.\ 1. 5, whereas, as we shall see, Dulcarnon was really
the name for the Pythagorean theorem 1. 47.

Ety mo logically, Roger Bacon's explanation leaves something to be de-
sired. The word would really seem to be an attempt to compound the two
Greek words iXtos, pity (or the object of pity), and <j>vyrj, flight (cf.\ note
ad loc. in Skeat's edition of Chaucer). Notwithstanding the confusion of
tongues, the object seems to be a play upon the two words Elementa and
(A<ot, which both begin with the same three letters, and the implication is that
``escape from the Elements ``(which normally came when Prop.~5 was reached)
was equivalent to ``escape from misery'' or ``trouble.'' A better form for the
word would perhaps be Eleufuga; and this form actually occurs in Alanus'
Anfielaudianus, 111, c 6 (cited by Du Cange, Glossariutn, sv,). The word
also occurs, according to Skeat's note, in Richard of Bury's Philobiblm,
c. xiii, where it was somewhat oddly translated by J. B. Inglis in 1S32 ``How
many scholars has the H el le flight of Euclid repelled ! ``

I.47.

The Pythagorean proposition about the square on the hypotenuse has
taken even a deeper hold of the minds of men, and has been known by
a number of names,

I. The Theorem of the Bride {Stiiprfjui. 1-99 nJ/tijt).

This name is found in a \textsc{ms.}\  of Georgius Pachymeres (1 242-1310) in the
Bibliotheque Nationale at Paris; there is a note to this effect by Tannery
(La Giomitrie grecque, p.~105), but, as he says nothing more, it is probable
that the passage gives the mere name without any explanation of it. We
have, however, much earlier evidence of the supposed connexion of the pro-
position with marriage. Plutarch (born about 46 a.d.) says (De hide et
Osiride 56, p.~373 f) ``We may imagine the Egyptians (thinking of) the most
beautiful of triangles (and) likening the nature of the All to this triangle most
particularly, for it is this same triangle which Plato is thought to have
employed in the Republic, when he put together the Nuptial Figure (ya/uJAiov
SuJypa/ifta)'' — SuLypafi/ia, though literally meaning ``diagram ``or ``figure,'' was
commonly used in the sense of ``proposition ``— ``and in that triangle the per-
pendicular side is 3, the base 4, and the hypotenuse, the square on which is
equal to the sum of the squares on the sides containing (the right angle), 5.
We must, then, liken the perpendicular to the male, the base to the female and
the hypotenuse to the offspring of both... .For 3 is the first odd number and
is perfect, 4 is the square on an even side, », while the 5 partly resembles the
father and partly the mother, being the sum of 3 and 3.''

Plato used the three numbers 3, 4, 5 of the Pythagorean triangle in the
formation of his famous Geometrical Number; but Plato himself does not call
the triangle the Nuptial Triangle nor the number the Nuptial Number. It is
later writers, Plutarch, Nicomachus and Iamblichus, who connect the passage
about the Geometrical Number with marriage; Nicomachus (Introd. Ar., 11,
24, n) merely alludes to ``the passage in the Republic connected with the
so-called Marriage,'' while Iamblichus (In Nieom,, p.~82, to Pistelli) only
speaks of ``the Nuptial Number tn the Republic''

It would appear, then, that the name ``Nuptial Figure'' or ``Theorem of
the Bride ``was originally used of one particular right-angled triangle, namely
(3, 4, 5). A late Arabian writer Beha-ad-din (1547-1622) seems to have
applied the term ``Figure of the Bride ``to the same triangle; the Arabs
therefore seemingly followed the Greeks. The idea underlying the use of the
term, first for the triangle (3, 4, 5), and then for the general theorem of 1. 47,
seems to be roughly that of the two parties to a marriage becoming one, just
as the two squares on the sides containing the right angle become the one
square on the hypotenuse in the said theorem.

a. The ``Bride's Chair''

The origin of this name is more obscure. It must presumably have been
suggested by a supposed resemblance between the figure of the proposition
and such a chair. D. E. Smith {History of Mathematics, n, pp.~289-90)
remarks that the ``Bride's Chair ``may be so-called ``because the Euclid
figure is not unlike the chair which a slave carries on his back and in which
the Eastern bride is sometimes transported to the ceremony,'' and he cites a
note from Edouard Lucas' R/criations Afathimatiques, [I, p.~130: ``La demon-
stration que nous verions de donner du threme de Pythagore sur le carre
de Thypotenuse ne differ « pas essentia lement de la demonstration hindoue,
connue sous le nom de la Chaise de la petite mariie, que Ton rencontre dans

1 ouvrage de Bhascara (Bija-Ganita, § 146)-'' The figure of Bhkara is not
that of Euclid but that shown at the top of p.~355 above; I have however
not been able to find the name ``Bride's Chair ``in Colebrooke's translation
of the work of Bhaskara.

Notwithstanding the apparent frivolity of the setting, I venture to suggest
that Hght may be thrown on the question by a very modem version of the
``Bride's Chair'' which appeared during or since the War in La Vie Parinenne.
The illustration represents Euclid's figure for 1. 47 and, drawn over it, as on
a frame, a peiiu in full fighting kit carrying on his back his bride and his house-
hold belongings. Roughly speaking, the soldier is standing (or rather walking)
in the middle of the large square, his head and shoulders are bending to the
right within the contour of one of the small squares, while the lady, with
mirror and powder-puff in action, is sitting with her back to him in the right
angle between the two smaller squares (HAG in the figure on p.~349 above) 1 .
I am informed by Sir George Green hill that there was also an earlier version
``showing the chair as it is in use to-day in Cairo and Egypt, the earliest
version of a taxi -chair, a pattern as early as Euclid and suggesting the nick-
name of the proposition.'' This recalls to my mind the remark of a friend to
whom I mentioned the subject and showed the figure of the proposition; he
observed at once on seeing it ``But I should have said it was more like a sedan
chair,'' the large square suggesting to him the actual chair and the two smaller
squares the two bearers,

3. Dukarnon.

This name for 1. 47 appears, as above mentioned, in Chaucer's Troilus
and Criseyde, in, \prop{2}{930}-3, where Criseyde says:

' I am, til God me bettre minde sende,
At dulcarnon, right at my wiltes ende.'
Quod Pandirus, 'ye, nece, wol ye here?
Dulcarnon called is ``Aeminge of wrecches.'' '

Billingsley, too, in his edition of Euclid {1570) observes of 1. 47 that ``it hath
bene commonly called of barbarous writers of the latter time Dulcarnon.''

Dulcarnon (see Skeat's note ad loc.) seems to represent the Persian and
Arabic du 'Ikarnayn, lit. tivo-horned, from Pers. du, two, and karn, horn. The
name was applied to t. 47 because the two smaller squares stick up like two
horns and, as the proposition is difficult, the word here takes the sense of
``puzzle ``; hence Criseyde was ``at dulcarnon ``because she was perplexed
and at her wit's end.

4. Frantisd tunita = ``Franciskaner Kutte,' 1 ``Franciscan's cowl.''

This name is quoted by Wei ssen born (Die Uebcrsetsungen des Euklid
durck Campane uttd Zamberti, p.~42) as given in a Geometric by one Kunze.
The name is quite appropriate, one of the squares representing the hood
thrown back.

in. 7, 8.

I have already mentioned the names ``Goose* s Foot'' (Fes inserts) and
``Peacock's Tail'' (Cauda favonis) applied, suitably enough, to these pro*
positions respectively. They come from Luca Faciuolo's edition of Euclid
published in 1509 (vide Weissenbom, ibid,).

1 Old Cambridge men will recall a picture in some respects not unlike* though less
artistic than, the cartoon in La VU Pariienm t I mean the painting of ``The Man Loaded
with Mischief 11 which used to be over the door of the former inn of that name on the
St Keots Road, a short distance from Cambridge.

\end{document}

\backmatter

\chapter*{Index of Greek Words and Forms}

dyruir, angle-fas (figure) [87

drnro?: 17 dt ri AZ. -traytayjf t 4) * rev dS +

Stlit, if tls tit dfl, 070 wa dr66tttt 136
drffouftfti, barb-tike 188
dXo-yoi, irrational: xepi dhoywr -/pauP koX

vaffrQp, work by Democritus 413
d0A«a (-yuria), obtuse (angle) l8l
tytfJKvy thrust, obtuse>angled [S7
Afupfa, indivisible 41, 168
<kfii)Utjt\oz (of lut vil ineal angles) 1 7 8
iUi'pros 178
dira?pd£cu' d*-d to describe on contrasted with

to construct {trvenfjeao6<u.) 3 48
droXuAfFH (r4ir«), Treasury of Analysis 8,

to, il, 138
djraoTjHurdt (species of locus) 330
ttfojioio/tfpjjs, non-uniform 40, 1D1-3
dTrfr/Hnwrjiffl 256—7: hading variety,

 TpaifynufLiitTf or  jn/picuj, id/rf.
djukra/urrot, non-existent IM
difwcrror, indeterminate: {of lines or curves)

1 60: (of problems) 1 29
draywy, reduction 13c: fit ri iiWaTo* 136
diTEor, infinite: -4 ir dr. iKfinWoyAvTi of

line or curve extending without limit and

not ``forming a figure 160-1; £r' d»\ or

#fi a jt. adverbial 190: eV dr, duu/Kurfat

166; Aristotle on tq dx*i/w* 232-4
drXtfOt, simple: (of lines or curves) i6t-»:

{of surfaces) 170
drifcifif, proof (one of necessary divisions of

a proposition) 119, 130
ttrretftfai, to iw*rf (occasionally touch) 57
Appffroit irrational: of X 670* 1371 of diameter

(diagonal) 399
dVtfp0arej, incompatible 139
dtnwTcin-DT, not- meeting, non-secant, asymp-
totic 40 1 161, 303
dtffrfnw, incomposite; (of lines) 160, 161;

(of surfaces) 170
drorrot, unordered** (of problems) 128: (of

irrationals) 11 j
dropw ypavkfuat, ``indivisible lines*' 166
afrfo #£a, Ipse dixit* referring to Pythagoras

4''
difr = segment of circle less than semicircle 187
fiQai, depth 158-9
fttrts, base 148—9
yvtprffkior itidypafifott ``Figure of Marriage''

(Plutarch), term for Pythagorean triangle

<3- 4. 5). 4'7
y<ypip$w 34a
yvu)AW, see gnomon
ypaw§ t line {or curve) 0JB*
ypawtiKt f graphically 400
faflojiAffl* given, different senses 13-3:

Euclid's dtfofUra or Da/a, q.v.
btlyprtL, illustrations* of Stoics 319
Sei di), ``thus it is required,'' introducing

5toptffu.it 293
BtyfxLfifui. = proposition (Aristotle) 25?
dtaipOTt: /,:>;.>: 0/ division (Aristotle) 165,

170, 17 j: method of division (exhaustion)

185: Euclid's TtplStaipttrtiiHtR.i), 18,87,00
duurdrtif, almost = dimensions 157, 158
ouMTaTi* extended., i$' tv one may, irl Stio

two ways, £rl rpta, three ways (of lines,
surfaces and solids respectively) r jB, 170

auL<rnitia t distance 166, 167, 307: (of radius of
circle) 199: (of an angle) = divergence 176-7

it£o%iKQt (of a class of loci) 330

Bujxflu, ''let it be drawn through'' { = pro-
duced) 2 So

toptcffi.6s~{i} particular statement or defini-
1 ton, one of the formal divisions of a pro-
position 129: (a) statement of condition of
possibility 128, 1 20, 130, 131, 234, 243, 393

tieayvyy) apfiovtirif. Introduction to Harmony,
by Cleonides 17

ixnripa rarlfti, meaning respectively 148, 350

£xfsefi\r)<i8tMX?iLk, use of, 244

fofrti = Euclid 400

fxefftt, setting-out, one of formal divisions of
proposition 129; sometimes omissible [30

itcrds, Jtard rd {of an exterior angle in sense
of re-entrant) $63: % txtbt yv*ta t the
exterior angle 280

i\ixoei5T)t, spiral-shaped 159

tXketijut, falling-short (with reference to
application of areas) 36, 343-5, 383-4

cXXli-jt wp6p\-rftia, a deficient ( = indeter-
minate) problem 119

fraXXdf p alternately or (adjectivally) alternate

308

titpota, notion, use of, 211

eVaraittf, objection 135

*t£t, *ar to Or  frrdr (-yuwla) of an interior

angle 263, ?8o: t) irrbt rot dratarrfor

7upfa, the interior and opposite angle 2S0
lve$tvX<#f<w {irifctiyrvui, join) I4I
{rlrtov, plane in Euclid, used for surface

also in Plato and Aristotle 169
iwnrpocfftVi <tTL-rpM0cv efrat, to Stand in

front of (hiding from view) 165, 166
ivupdpeta, surface (Euchd) 169
irtpofiijKtt, oblong 151, 188
citdv, t6, the Straight 159: eveta [yptififf),

straight line 165-9
rdBtiypKuuin, rectilineal 187: neuter as sab-

stantive 346
iqid.irrnrBait to touch 57
iKLpfiitiv, to Coincide i$a.ppbfr{r8au, to be

applied to 1 68. 234-5, 2 49
ttfr*KTLKtn (of a class of loci) 330
ift-, ``in order'' 18 1: of adjacent angles

18 278
ANtaHMg theorem jv,:  lita rt ijirr

``Theorem of the Bride,'' = Eucl. h 47, 417
dapebi (shield) = ellipse 165
Irrov TriTf (horse-fetter), name for a certain

curve 1 6-1-3., l l
tooperpwr xnudray, -rtpl, on isometric figures

(Zenodorus) 26 17, 333
jcdtfrri* fi>(?«a ypa/Af, perpendicular iftf-9,

371: ''plane' and ``solid'' 172

ftaAfYtrnJt 1o

jcavXot, curved (of lines) 1 50

KaTOOKtw/j, construction, or machinery, one of

divisions of a proposition 129; sometimes

unnecessary 130
KaraTQv.il xnvbvot, Sectio canonis, attributed to

Euclid 17

icebrtw, ``let it be made'' 169
KttitpLfiiti), bent (of lines) 159, 176
Ktrrpar, centre 183, 184, 199: rr *v roO xirrpou

= radius 199
«-we«t*;i (yiarta), hem-like (angle) [77,

17S, 181
rXor, to inflect or dejlett, nexXduOat, kck\h*t-

pJrt), itXimr. 1 1 8, 1 jo, 159, 176, 178
ftXXrcf, inclination, 1J0
Koi\trfiiivioy t hollow-angled figure (Zenodoras)

17, 188
jcwai human, Common Notions (=axioms)

131-1: called also tA jcoo'd, rural $6£m

(Aristotle) no, 111
xowi) rpoOKthw, AQpfyrfkt 176
topurj, vertex: caraffopi* vertical (angles)

178
Mpiitot, ring (Heron) 1(13
\fyma, lemma (``something assumed, Xs>t-

fiari)i«i«r) 133-4
XwrrVr: X«r 4 A A Xo<*- jj r£ BII frig iarlv 145
prfpr, parts ( = direction) 190, 308, 313:

(=siae) 171
ufam, length, 158-9
/swrontfo, lune-ltkt (of angle) 3 6, 201: tA

ita-nt-lit (rrrt/w), lune 1B7
/tun-it, ``mixed (of lines 01 curves) ifji, 161;

(of surfaces) 170
(jorii rpotrXnoDira SAn>>, definition of a /tain/

55

tiarbrrptxpos f\i% "single- turn spiral" 113-
3»., 164-3: in Pappus = cylindrical helii

rrvrrett, inclinations, a class or problems
150-1: mfccr, to verge 118, 150

IwrrptinJ-fr, temper-tike (of angle) 178

«>iwi*t)t, ``of the same form'' ijo

5>UHOT t c1 similar M (ofnumbers)357: (of angles)
= equal (Thales, Aristotle) 151

d»i«o*upi$i, uniform (of lines 01 curves) 40,
161-1

«rra (Ywla), acute (angle) 181

(Jfiryiimu, acute-angled 187

ffnp B«* eVxfoi (or roirat) Q.E.D. (or F.) 57

ipfaywnot, right-angled: as used of quadn-
latcrats = rectangular 188-9

6pot, Spurpot, definition 143: original mean-
ing of Spot 143: m boundary, limit 181

rtytf, visual ray 166

rimj utTa\mfiBaw6iMtnu, ``taken together in
any manner'' 181

TapapoXI) Ty xuptvr, application of areas 36,
343-5; contrasted with bwtppoM) (exceed-
ing) and AXttir (faliing-short) 343: rapa-
fi*M contrasted with iriVrMii (construction)
343: application of terms to conics by
Apollonius 344-5

MpdJofot rim, 4, ``the Treasury of Para-
doxes'' 319

rofoK\irra, ``fall beside'' or ``awry'' 161

rapaw\ptaftn t complement, q. V.

ripas, extremity 165, 181 1 Wpai airf%hiin*
fflasuionius' definition of figure) 183

Ttpxpiibii (of angle), Ttpaxiueror (of Teet-
angle), contained 370: ri Sit iifxixbturov.

cwicethe rectangle contained 380: (of figure)

contained ot bounded i8j, 183,1841186, 187
rtpiipipaa, circumference 184
Trcpi<pfp/)t, circular 159
Ttpi$ef>ypawun, contained by a circumference

of a circle or by arcs of circles 181, 184
lrXirot, breadth 158-9
n\t ori{w (*pt>fi\Tiiia), ``(problem) in excess''

119
ittu, a mathematical instrument 370
ToXr\eupor t many-Bided figure 187
roptattoffat, to ``find'' or M furnish'' 115
rhpurpn, porism q.v.
Tpop\Ttpa. ¥ problem q.v.
TpvrryvbyxvM, leading', (of conversion) = oom-

filete J56-7: ipmrfotfatrnr (Smijnj'-n.) leading
theorem) contrasted with converse 157
rp6t, in geometry, various meanings 177
TpraiTLi, enunciation 149-30
Tporetvta, ``propound'' 118
rprof, prime, two senses of, 146
tttu, case 134
fart*, rational 137: farll tiitrpot rBt np>

tdicn (``rational diameter of 5'') 399
trytitlw, point 155-6
sriSuif, a mathematical instrument 371
rrtyni, point 156
uToixtTor , dement 1 14-6
rrptryy\<ir, re, round (circular), in Plato

159, 184: arpoyyuXbriit, roundness 183
TijprlpatTfui, conclusion (of a proposition)

119, 130
afoBerot, composite: (of lines or curves) 160:

(of surfaces) 170
aOveveit, convergence 181
avrio-rwBea, construct', special connotation

159,189: with eVrir 189: contrasted with

wafafiikkia (apply) 343
oxvrroypaptir, oxritucToypaia, represent-
ing (numbers) by figures of like shape 359
oXfowoioGoa or axfttw. roioCrra, forming a

figure (of a line or curve) 160-1
reraypJrtr (of a problem), ``ordered'' 118
TeraayttHOtUi, squaring, definitions of 149-

M, 4 10
TFFpdytaror, square: sometimes (but not in

Euclid) any four-angied figure 188
TrrpirXevpQT, quadrilateral 187
rofii, section, = point ef section 170, 171, 178
ToTuto* BtfLpTjfus, locus-theorem 319
Tbww. locus 319-31: room or space 13 n.:

place (where things may be found), thus

T6rotiyaXv6fierm%,io: ra/Mffocor 761*01319
ripros, instrument for drawing a circle 371
rptTrXevpor, three-sided figure 187
Tvxar truutior, a point at random 159
irrtpSaX, exceeding, with reference to method

of application of areas 36, 343-, 386-7
uirtS, in expressions for an angle ft iri BAr

yWa) 149, and a rectangle 370
foriMiTtu, ``is by hypothesis'' 303, 311
trTQTtti/ttr, subtend, with ace. or irn and ace.

14J. 183. 350
apurprri) ypowt, determinate line (carve),

``forming a figure'' 160

\chapter*{English Index}

al-'Abbis b. Said al-Jauharl 85

``Abthiniathus'' (or ``Anthisathus'') 103

Abu '1 'Abbis sd-Fadl b. Hltim, see an-
Nairtxi

Abu'AbdaluUi Mob. b, Muadh al-JayySnt 90

Abu 'All al-Basfl 88

Abu 'Alt a] -Hasan b. al- Hasan b.al-Haitham
88,80

Abu Da'ud Sulaiman b. 'Uqba 85, 90

Abu Ja'far al-Kbizin 77, 8 5

Abu J a fir Muli. b. Muh. h. al-Hasaii
Naslraddln at-TusI, see Naslraddln

AbQ Muh. b. AbdalbaqT al-Bagdadi al-Faradt
8/1., 90

Abu Muh. a] -Hasan b. 'Ubaidallih b. Sulai-
min b. Wahb 87

Abu N'asr Gars il-Ni'ma 90

Abu Nasr MansOr b. 'Alt b- 'Iraq 90

AbQ Mast Muh. b, Muh. b. Tarkhan b.
Uzlag al-Firabl 88

Aba Sahl Wljan b. Rustam al-Kuhl 88

Abu Said Sinan l>. Thibit b. Qurra 88

Abu 'Uthmin ad-Dimasbqi is, 77

Aba '1 Waft al-Burjinl 77, 85, 86

Aba Yusuf Va'tjftb b. Ishiq b- as-$abb*h al-
Kindl 86

Aba Yusuf Ya'qfib b. Muh. ar-Rail 86

Adjacent {l$i(fjt). meaning 1S1

Aeuaeas (or Aigeias) of Hierapolis 18, 311

Aganis 17-81 191

Ahmad b. al-Husain al-Ahwiil al-Kalib 89

Ahmad b. 'Umar a]-KarabisT 85

al-Ahwirl 89

Aigeias (?Aenaeas) of Hierapolis )8, 311

Alexander A ph rod is ten sis 7 n., 19

Algebra, geometrical, 371-*: classical method
was that of Eucl. 11. (cf.\ Apollonius) 373:
preferable to semi- algebraical method 377-
8: semi-algebraical method due to Heron
373, and favoured by Pappus 373: geome-
trical equivalents of algebraical operations
374: algebraical equivalents of propositions
in Book 11. 371-3

All b. Ahmad Abu ') Qasim al-Antlkl 66

Allman, G. J- 135 «., 318, 351

Alternate (angles) 308

Alternative proofs, interpolated, 58, 59

Amaldi 175, 179-80, 193, 101, 313, 318

Ambiguous case 306-7

Amphinomus 1131 118, 130 n.

Amyclas of Heraclea 1 17

Analysis (and synthesis) 18; alternative
proofs of xtll. i-j by, 137: definitions of,

interpolated, 138: described by Pappus
138-9: modem studies of Greek analysis
1 39: theoretical and problematical analysis
1 38: Treasury of analysis (rAw amXv4-
lifvm) 8, 10, 11, 138: method of analysis
and precautions necessary to 139-40:
analysis and synthesis of problems 140-1;
two parts of analysis (a) transformation,
(b) resolution , and two parts of synthesis,
(a) construction, (b) demonstration 141:
example from Pappus 141-1: analysis
should also reveal hoptaptn (conditions of
possibility) 141

Analytical method 36: supposed discovery
of, by Plato 134, 137

Anaximander 370

Anchor- ring 163

And run 116

Angle. Curvilineal and rectilineal, Euclid's
definition of, 1 76'' sq.: definition criticised
by Syrianus 176: Aristotle's notion of
angle as icMait 1 76: Apollonius' view of,
as amir action 176, 1 77: Plutarch and
Carpus on, 177: to which category does it
belong? quantum, Plutarch, Carpus, ``A-
ganis 177, Euclid 1781 quote, Aristotle
and £udemus 177-8: relation, Euclid 178:
Syrianus 1 compromise 1 78: treatise on the
Angle by Eudemus 34, 38, 177-8: classifi-
cation of angles (Geminus) 178-9: eurvt-
lineal and ``mixed'' angles 16, 176-9,
korn-iikt [KfoaroctSfy) 177, 178, 181, 165,
lune-tike (pi)roe<?t) i6 t 176-9, scraptr-Wu
(fuoTpMtSijsj 178: angle ofn segment 1531
angle of a semicircle 181, 153: definitions
of angle classified 1 79: recent Italian views
179-81: angle as cluster of straight lines
or rays iSo-i, defined by Veronese 180:
as part of a plane (``angular sector'') 179-
80: flat angle (Veronese etc.) 180-1, 169:
three kinds of angles, which is prior
(Aristotle)? 181-1: adjacent angles 181:
alternate 308: similar ( = equal) 178, 181,
151: vertical 178: exterior and interior
(to a figure) 163, 180: exterior when re-
entrant 163: interior and opposite 180:
construction by Apollonius of angle equal
to angle 196: angle in a semicircle, theorem
of, 317-19: trisect ion of angle, by conchoid
of Nicomedes »6$-6, by qundratrix of
Hippias 166, by spiral of Archimedes 167

a]- Antaki 86

Antiphon 7*.. 35

Anthtsathus'' (or ``Abthiniathus'') 203
Apatamba-Sulba- Sutra 352: evidence in, as
to early discovery of EdcL \prop{1}{47} and use
of gnomon 360-4: Burk's claim that
Indians had discovered the irrational 563—
4; approximation to *Ji and Thibaut's
explanation 361, 363-4: inaccurate values
of w in, 364
Apollodorus 4t Legist icus'' 37, 319, 351
Apollonius: disparaged by Pappus in com-
parison with Euclid 3: supposed by some
Arabians to be author of the Elements 5:
a ``carpenter** 5: on elementary geometry
44: on the tine 159: on the turtle 176:
general definition of diameter 325: tried to
prove axioms 43, 61, 313-3: his ``general
treatise'' 43: constructions by, for bisection
of straight line 268, for a perpendicular
170, for an angle equal 1.0 an angle 296:
on para I Id -axiom (?) 41-3: adaptation to
conics of theory of application of areas
344-5: geometrical algebra in , 373: Plant
Loci 14, 1591330. Planet* form 151; com-
parison of dodecahedron and icosahedron
6; on the eoc Alias $4, 41, 16* -. on unordered
irrationals 43 , 115: 138, 188,211, 313,246,

*49> *S9p 37*, 373

Application of areas 36, 343-5  contrasted
with exceeding and failing-short 343:
complete method equivalent to geometric
solution of mixed quadratic equation 344-5,
383-5, 386-8: adaptation to conics (Apol-
lonius) 344-5: application contrasted with
construction (Proclus) 343

``Aqaton'' SS

Arabian editors and commentators 75-90

Arabic numerals in scholia to Book X-,
nth c, , 71

Archibald, R. C 9*1., 10

Archimedes 20, 21, m6, 141: ``postulates''
in, j 2 Op 1 33: famous  ' lemma ``(assumption)
known as Postulate of Archimedes 234:
``PorismV'in, n a., 13: spiral of, 36, 367:
on straight line 166: on plane 171-2: 225,

U9* 370

Archytas to

Areskong, M. E. 113

Arethas, Bishop of Caesarea 48; owned
Bodleian \textsc{ms.}\  (B) 47-8: had famous Plato
MS, of Patmos (Cod. Clarkianus) written 48

ArgyTus, Isaak 74

Aristae us 138: on conics 3: Solid Loci 16,
329,: comparison of Ave {regular solid]
figures 6

Aristotelian Problems 166, 182, 187

Aristotle: on nature of elements 116: on
first principles 117 sqq.: on definitions 117,
1 19-20, [43-41 [46-50: on distinction be-
tween hypotheses and definitions 1 19, 120,
between hypotheses and postulates 118,
119, between hypotheses and axioms 120:
od axioms 119-11: axioms indemon-
strable 121: on definition by negation
156-7  On points 1 55-6, 165: on tines,
definitions of 158-9, classification of 159-

60: quotes Plato's definition of straight
line 166: on definitions of surface 170:
on the angle 1 76-8  on priority as between
right and acute angles 18 [-3: onjSgure
and definition of 183-3: definitions of
``squaring'' 149-50, 410: on parallels 190-
»p 308-9: on gn#mon M $$i t 355, 359: on
attributes k*t varr6t and -wpOrrav kw6\ov
319, 310, 325: on the objection 135: on
reduction 135: on redwtio ad aosurdutn
136; on the infinite 233-4 1 supposed pos-
tulate or axiom about divergent Lines taken
by Proclus from, 45, 107 1 gives pre-Eucli-
dean proof of l. 5 352-3: on theorem of
angle in a semicircle 149: on sum of angles
of triangle 31 9-21: on sum of exterior
angles of polygon 322: 38,45, n?, 150**-,
iBl, 184, 185, 187, r88, 195, 102, 303,

331, 232, 33T t 326, 259, 262-3, 283, 4II

Arithmetical calculations in scholia to Bk, x.

tun

al-Arjani, Ibn Rahawaihi 86

Ashkal at-ta'sis 5 n.

Asbraf Sbamsaddln as-Samarqandl, Muh. b,
5''*» 89

Astaroff, Ivan 1(3

Asymptotic (non-secant) lines 40, i6i t 203

Athelhard of Bath 78, 93-6

Athena eus 30, 351

Atbenaeus of Cyiicus 117

August, £. F. 103

Austin, W. 103, in

Autolycus, On the moving sphere 17

Avicenna 77, 89

Axioms, distinguished from postulates by
Amtotte 1 1 8-9, by Proclus (Geminus and
``others'') 40, 121-3: Proclus on diffi-
culties in distinctions 133-4: distinguished
from hypotheses, by Aristotle 120-1, by
Proclus l*f~4J indemonstrable 111: at-
tempt by Apollonius to prove 233-3:
= ``common {things) ``or ``common
opinions' 3 in Aristotle 120, 33i: common
to all sciences 119, 120: called ``common
notions'' in Euclid 121, 221: which are
genuine? 33 1 sqq*: Proclus recognises five

332, Heron three 222: interpolated axioms
234, 233: Pappus* additions to axioms
a 5, *33* 224, *3*: axioms of congruence,

(1) Euclid's Common Notion 4, 234-7,

(2) modern ``systems (Fasch, Veronese and
Hilbert) 228-31: ``axiom'' with Stoics =
every simple declaratory statement 41, 121

Bacon, Roger 94, 416

Balbus, de mensuris 91

Barbarin 219

Barlaam, arithmetical commentary on Eucl. II.

74
Barrow, 103, roj, 110, 111

Base, meaning 248-9
Basel, editio prineeps of Eucl. 100- 1
Basil ides of Tyre 5, 6
Baudhayana Sulba-Sutra 360
Uayfius (Ba'i'f, Lasare) [00

Becker, J. K. 174

Beet 176

Beha-ad-din 417

Beltrami, E. no

Benjamin of Lesbos 113

Bergh, p.~400-1

Bernard, Edward, 103

Besthom and Heiberg, edition of al-Hajjaj's

translation and an-Nairīzī's commentary

ai, V}n., 79(1.
Bhaskira 35 j, 41S
Billingsley, Sir Henry 109-10, 418
al-Blrunl 90

Bjiirnbo, Axel Anthon ifn., 03
Boccaccio $6
Bodleian us. (B) 47, 48
Boeeich 331, 371
Boethius 92, 95, 1S4
Bologna \textsc{ms.}\  (b) 49
Bolyai, I 119

Bolyai, W. 174-5, a'9> 318
Boliano 167

Boncompagni 93 «., 104 n.
Bono! a, K. ioj, 119
Borelli, Giovanni Alfonso 10$, 194
Boundary (tyer) 183, 183
Brkenhjelm, p.~R. Ill
Breitkopf, I oh. Gottlfeb Immanuel 97
Bretschneider 136(1., 137, 19s, 304, 344,

354. JjS
Briconnet. Francois 100
``Bride, Theorem of the,'' = M.v.t\. 1. 47,417-8
``Bride's Chair,'' name for \prop{1}{47}, 417-8
Briggs, Henry 101

Brit- Mm. palimpsest, 7th— 8th c, 30
Bryson 8 m.
Burk, A. 35a, 360-4
BUrkten 179
Buteo (Borrel), Johannes 104

Cabailas, Nicolaus and Theodorus 74

Caiani, Angelo 101

Camcrarius, Joachim tot

Cameier, J. G. 103, 193

Camorano, Rodrigo til

Campanus, Johannes, 3, 78, 94-96, 104,

too, no, 407
Candalla, Franciscus Flussatea {Francois de

Foil, Comte de Candale) 3, 104, no
Cantor, Moriti 171, 304, 318, 310, 333,

3SL 3S5, 357-8, 360, 101
Carduchi, L. 111
Carpus, on Astronomy 34, 43: 45, 117, ia8,

'77
Cast, technical term 134: cases interpolated

J8, 59

Casm 4 b., 1711.

Cassiodorus, Magnus Anrelius 91

Caraldi, Pietro Antonio loh

Catoplrka, attributed to Euclid, probably
Tbeon's 1 7: Catoptrics of Heron 1 1 , 153

``Cause'': consideration of, omitted by com-
mentators 19, 45: definition should state
cause (Aristotle) 149: cause— middle term
(Aristotle) 149: question whether geometry

should investigate cause (Ge minus), 45,
130*.

Censorinus 91

Centre, tirrpor 184-5

Ceria Aristetcliea 35

Chasles on Purisms of Euclid 10, 11, 14, 15

Chaucer: Dttttarium in 416-7, 418

Chinese, knowledge of triangle 3, 4, 5, 3£i ``
``Chou-pei™ 35J

Chou Kung 361

Chrysippus 330

Cicero 91, 351

Circle: definition of, 183-5: = round, vrtttrr
-fiXar (Plato) 184: = ntinww
(Aristotle) 1B4: a plant figure 183-4:
centre of, 184-5: pole of, 183: bisected by
diameter (T hales) 185, (Saccheri) 185-6:
intersections with straight line 137-81
171-4, with another circle 138-40, 341-3,

193-4

Circumference, irrpvpiptui 184

Cissoid i6t, 164, 176, 330

Clairuut 318

Claymundus, Joan. 101

Clavius (Christoph Klau ?) 103, 105, 194,

13»> 38 • 39'- 407
Cleonides, Introduction to Harmony 17
Cochliae or roth lion (cylindrical helix) r6l
Code* Leideniis 399, t: 11, 17 »., 79 n.
Coets, Hendrik 109
Commandinu* 4, 101, 103, 104-5, IO *> Ito >

in, 407: scholia included in translation

of Elements 73: edited (with Dee) De

divisionibus 8, 0, no
Commentators on Eucl, criticised by Proclus

a. ' 6 . « ,

Common Notions: s= axioms 01 , 1 10- 1,111-1:
which ire genuine ? iiisq,: meaning and
appropriation of term 11 1: called ``axioms ``
by Proclus sir

CompIemsnt,Trapa.*\HfiiojiAi meaning of, 341:
``about diameter'' 341: not necessarily
parallelograms 341: use for application of
areas J41-3

Composite, <rur#rn», {of lines) 160, (of sur-
faces) 170

Conchoids 160-1, 165-6, 330

Conclusion, trufiiripafftta: necessary part of a
proposition 119-30: particular conclusion
immediately made general 131: definition
merely stating conclusion 149

Congruence-Axioms or Postulates: Common
Notion 4 in Euclid 114-5: modern systems
of (Pasch, Veronese, Hubert), 118-31

Congruence theorems for trian gles, recapitula-
tion of, 305-6

Conics, of Euclid 3, 16: of Aristaeus 3, 16:
of Apolloniiis 3, (6: fundamental property
as proved by Apollonius equivalent tq
Cartesian equation 344-5: focus-directrix
property proved by Pappus 15

Constantinus Lascaris 3

Construct (ovrteravtm) contrasted with
describe on 348, with apply to 343: special
connotation 159, 189

Construction, taxaSKtvii, one of formal di-
visions of a proposition no: sometimes
unnecessary 730: turns nominal into teal
definition 146: mechanical 151, 387
Continuity, Principle of, 134 sq., 9411,171,194
Conversion, geometrical; distinct from logical
156: ``leading'' and partial varieties 156-7,

Copernicus 101

Cordonis, Mattheus 97

Cosmic figures ( = five regular solids) 413-4

Cratistus 133

Crelle. 011 the plant 171-4

Cteaibius 20, 11, 39 n,

Gunn, Samuel in

Curtie, Maximilian, editor of an-Nair3 u,

78. pa, 94, £ 97»-
Curves, classification of: w line
Cylindrical helix 161 , 161, 339* 330
Czech*, Jo. 113

Dasypodius (Rauchfuss), Conrad 73, 102

Date of Euclid S. 133, 141* 385, 391

Deahna 174

DechaJes, Claude Francois Milliet 106, 107*
108, no

Ded ek i nd 's Postulate, and applications 1 35 -40

Dee, John 109, no: discovered De divin-
onibus 8, 9

De/tnitiati, in sense of ``closer statement ``
(6ioptCfn6t), one of formal divisions of a
proposition 119' may be unnecessary 130

Definitions: Aristotle on, 117, no, no, 143:
a class of t Arsis (Aristotle) 110: distin-
guished from hypotheses 119, but confused
therewith by Proclos I3i-s: must be
assumed 117-9, but say nothing about
existence (except in the case of a few
primary things) 119, 143: terms for, Gpoi
and AptapM 143: «b/ and nominal defi-
nitions (real = nominal ptuj postulate or
proof), Mill anticipated by Aristotle, Sac-
cberl and Leibniz 143*5: Aristotle's re-
quirements in, [46 50, exceptions 148:
should state tame or middle term and be
genetie [49-50: Aristotle on unscientific
definitions [4x py) wporipiitv) 148-9: Euclid's
definitions agree generally with Aristotle's
doctrine 146: interpolated definitions 61,
61 1 definitions of technical terms in Arts*
totle and Heron, not in Euclid 150

De levi et pondcroso, tract 18

Demetrius Cydonius yi

Democritns 38: treatise on irrationals 413

De Morgan 346, 36b, 369, 384, 10 1, 3 98,
300, 309, 313, 3141 315p 39> 370

Desargues 193

Describe wt {Araypfair dr6) contrasted with
(instruct 34 8

De 2olt 31 S

Diagonal [Uay vriw) 185

``Diagonal'' numbers: see ``Side- ``and
``diagonal ``numbers

Diameter (3uv**rp«) T of circle or parallelo-
gram 185: as applied to figures generally

335: ``rational'' and ``irrational'' diameter
of 5 (Plato) 399, taken from Pythagoreans

309-400,413

Diels, \prop{2}{413}

Dimensions (cf.\ Stafrfatit) 157, 158: Aris-
totle's view of, 158-9

Dinostraths 117, 466

Diodes 164

Diodorus 303

Diogenes Laertius 37, 305, 317, 351

Dionysius, friend of Heron, 4t

Diophantus So

Diorismtts [6topiCfii) — [a) ``definition'' or
** specification,'' a formal division of a
proposition 119: [b) condition of possibility
118, determines how far solution possible
and in how many ways 130-1, 343: dio-
rismi said to have been discovered by
Leon Ef6l revealed by analysis 143: in-
troduced by Sit dJf 393: first instances in
Elements 134, 193

Dippc 108

Direction, as primary notion, discussed 179:
direction -theory of parallels 191-1

Distance, iidtrnjaa: = radius 1 99: in Aristotle
has usual general sense and m dim ension 199

Division (method of), Plato's 134

Divisions {o/Jigures) by Euclid 8, 9: trans-
lated by Muhammad al-Bugdadi 8: found
(by Woepcke) in Arabic 9, and (by Dee)
in Latin translation 8, 9: 110

Dodecahedron in sphere 411

Dodgson, C- L 194, 754, 161, 313

Dou, Jan Fieterszoon 108

Duhamel [39, 318

Duliarnon*, name for Eucl. J. 47, 4 1 6, 418

Egyptians, knowledge of 3* + 4''= 5 35a

Elbe, Thyra 1 1 3

Elefuga, name for End. 1. 5, 416-7

Elemental pre- Euclidean Elements, by Hip
pocratcs of Chios, Leon 1 [6, Theudius 1 17:
contributions to, by Eudoxus 1, 37, Theae-
tetus I, 37, Hermotimus of Colophon
117: Euclid's Elements, ultimate aims of 2 ,
113-6: commentators on £9-45, Proclui
to, 39-45 and passim. Heron 10-S4, an-
Niirizi 31-34, Porphyry 34, Pappus 34-
37, Simplicius 38, Aenacas (Aigeias) *8:
mss. of 46-51; Theon*s changes in text
5*~5: nieans of comparing Theonine with
ante-Theonine text 5J-53: interpolations
before Theon's time 5H-63: scholia 64-74:
external sources throwing light on text.
Heron, Taurus, Sextus Empiricus, Proclus,
lamblichus 61-3: Arabic translations (1)
byal-tfajjaj 75, 76, 79, Bo, 83-4, (3) by
Ishaq and Thabit b, Qurra 75-80/83-4,
(3) NasTraddln at-TusJ 77-80, 84: Hebrew
translation by Moses b. Tibbon or Jakob
b. Machir 76: Arabian versions compared
with Greek text 79-3, with one another
83, 84: translation by Boethius 93: old
translation of 10th c*, 91: translation by
Athelhard 93-6, Gherard of Cremona 93-4,

Campanus 94-6, 07-1 io etc., Zamberti
98-100, Command inn 5 104-5 * introduc-
tion into England, toth c, oj: translation
by Billi Lesley 109-10: Greek texts, ediiio
princcpt too- 1, Gregory's 101-3, Peyrard's
103, August's 103, Heibetg's*af«>B: trans-
lations and editions generally 97-1 13: on
the nature of tkmtnts (Proclus) 1 14-6,
(Menaechmus} 1 1 4. (Aristotle) 116: Proclus
on advantages of Euclid's Elements 115:
immediate recognition of, 1 16: first princi-
ples of, definitions, postulates, common
notions (axioms) 117-24: technical terms
In connexion with, 135-41: no definitions
of such technical terms 1 50; sections of
Book 1. 308

EHnitam 05

Engel and Slack el 319, 31 r

Enriques, F. 1 r 3, 1 j;, 1 75, r 93, 19 j, 30 1 , 31 3

Enunciation (wpbraats), one of formal di-
visions of a proposition 199-30

Epicureans, objection to 1. 10 41, 387;
Savile on, 387

Equality, in sense different from that of
congruence (-''equivalent,'' Legendre)
317-8: two senses of equal(i) ``divisibly-
equal'' (Hilbert) or ``equivalent by sum''
(Amaldi),(i) ``equal in content ``(Hilbert)
ot ``equivalent by difference'' (Amaldi)
338: modern definition of, 228

Eratosthenes I: contemporary with Archi-
medes i, a

Errard, Jean, de Bar-Ie-Duc 108

Erycinus 37, 390, 349;

Euclid: account of, in Proclus' summary 1;
date 1-1: allusions to in Archimedes 1:
(according to Proclus) a Platonist 3: taught
at Alexandria 3: Pappus on personality
of, 3: story of fin Stobaeus) 3 1 not ``of
Megaia'' 3. 4: supposed to have been
bom at Geia 4: Arabian traditions about,
4, S: ``of Tyre'' 4-6: ``of Tus'' 4, 5 «. 1
Arabian derivation of name (``key of
geometry ``) 6: Elements, ultimate aim of,
a, 115-6: other works, Conks 16, Pttu-
dtsria 7, Data 8, 133, 141, 385, 391, On
divisions (of figures) 8, 9, Porisms 10-15,
Surface-loci 15, 16, Phaenomena 16, 17,
Optics 1 7 , Elements of Music or Seeiio
Canonis 17: on ``three- and four-line
locus 3 ' 3: Arabian list of works 17, 18:
bibliography 91-113

Eudemus 29: On the Angle 34, 38, 177-8:
History of Geometry 34, 35-8, 578, 395.
304- 317. 3''. 387, 411

Eudoxus 1, 37, 74, u(i: discoverer of theory
of proportion as expounded generally in
Bits, v., vi. 137, 351: on the golden
section 137: founder of method of ex-
haustion 334: inventor of a certain curve,
the hippopede, horse* fetter 163: possibly
wrote Sphaerica 17

Euler, Leonhard 401

Eutocias j{, 3J, 39, 141, 161, [64, 159, 317,
3'9. 33o. 373

Exterior and interior (of angles) 163, 280
Extremity, ripea 181, 183

Falk, H. 113

aE-Faradi 8 »,, 90

Figure, as viewed by Plato 181, by Aristotle
182-3, by Euclid 183: according to Posi-
donius is confining boundary only 41, 183:
figures bounded by two lines classified 187:
angle-less {dywvtov) figure 187

Figures, printing of, 97

Ftkrist 4 »., 5*1., 17, »c, 34, 35, 17: list of
Euclid'a works in 17, 18

Finaeus, Orontius (Oronce Fine) 101, 104

Flauti, Vincenzo 107

Florence \textsc{ms.}\  Laurent, xxvtii. 3, (F) 47

Flussates, see Cauda] U

Forcadel, Pierre 108

Fourier (73-4

Franeisci tunica, ``Franciscan's cowl,'' name
for Eucl. 1- 4;, 41S

Frankland, W. B. 173, 199

Frischauf r 74

Garti 17 ft.

Gauss 171, 193, 194, soi, 119, 311

Geminus: name 38-9: tide of work (#**o-
iiXia) quoted from by Proclus 39: ele-
ments of astronomy 38: coram, on Posi-
donius 39: Proclus obligations to, 39—42:
on postulates and axioms 121-3 - on theo-
rems and problems 1 »8: two classifications
of lines (or curves) 160-1: on homoeo-
meric (uniform) lines 162: on ``mixed''
lines (curves) and surfaces 162: classifica-
tion of surfaces 170, of angles 178-9:
on parallels 191: on Postulate 4, 300:
on stages of proof of theorem of 1. 33,
317-M: »7-S, 37, 44, 4S, 74. ``33 «.
303, 365, 330

Geometrical algebra 372-4 1 Euclid's method
in Book It, evidently the classical method
373: preferable to semi-algebraical method

„ 377-8

Georgius Fa chyme res 417

Gherard of Cremona, translator of Elements
9 3-4 \ of an- N ai rizi 's com mentary 11 , 1 7 n * ,
94: of tract De dtvisionidus 9, ion,

Giordano, Vitalc 106, 176

Given, Se*5tvii«T p different senses* 131-3

Gnomon: Literally M that enabling (something)
to be known ``64, 370: successive senses of,
(1) upright marker of sundial 181,185,171-
1, introduced into Greece by Anjuimander
370, ti) carpenter's square for drawing
right angles 371, (3] figure placed round
square to make larger square 351, 371,
Indian use of gnomon in this sense 361,
{4) use extended by Euclid to parallelograms
37 1 * {5) hy Heron and Thean to any figure*
371-1: Euclid's method of denoting in
figure 383: arithmetical u e of, 358-60,
37'

``Gnomon- wist h ' (tar A yydipora.), old name
for perpendicular (ndfftTot) 50, 161, 171

Gotland, A. 133, 134

``Golden section ``= section in extreme and

mean ratio 1 37: connexion with theory of

irrationals 137
``Goose's foot'' [pes anseris), name for

Eud. in. 7, 99, 4 1 H
Gow, James 1 35 *r.
Gracilis, Stephanus 101-1
Grandi, Guido 107
GreenhiU, Sir George 415, 418
Gregory, David 101-3
Gregory of St Vincent 401, 404
Gromaiict 91 n. t 95
Grynaens 100-1

al-Haitham 88, 89

al-Hajjaj h. Yiisuf b. Matar, translator of the
Elements 17, 75, 76, 79, 80, 83, 84

Halifax, William ro6, no

Halliwell 05 n.

Hankel, H. 139, 14L 131. 134. 344. 3S4

Harmonica of Ptolemy, Comm. on, 17

Harmony, Introduction to, not by Euclid r 7

HirOn sr-Rashld 73

al- Hasan h, 'Uliaidallih b. Sulaiman b,
Wahb 87

Hauff, J. K. F. to8

''Heavy and Light,'' tract on, iS

Heiberg, J. L. passim

Helix, cylindrical 161, 1G1, 319, 330

Helmholfct 116, 117

Henrici and Treutlein 313, 4O4

Henrion, Denis roB

Herigtme, Pierre 108

Herlm, Christian 100

Hermotimus of Colophon r, 1 17

Herodotus 37 »., 370

``Heromidea'' i s 8

Heron of Alexandria, meehantevi, date of
10-1: Heron and Vitruvius so-i; com-
mentary on Euclid's Elements 10-4:
direct proof of \prop{1}{73}, 301: comparison of
areas of triangles in 1. 14, 334-5: addi-
tion to 1-47, 366-8: apparently originated
semi-algebraical method of proving theo-
rems of Book 11, 373, 378: 137 »,, rjg,
163, 168, 1 Jo, rjr-i, rj6, 183, 184, 185,
188, 1 89, 111, 113, 143, 113, 183, 38;,
109, 351. 369, 371, 403, 407, 408

Heron, Proclus instructor 19

``Herandes'' 156

Hieronymus of Rhodes 303

Hilbert 157, 193, lot, 118-31, *49. 313.
318

Hipparchus 4 n. , 10, 30 n., 74 n.

Hippasus 4IX

Hippias of Elis 41, 165-6, 413

Hippocrates of Chios 8 it., 19, 35, 38, 116,

35. '3*»-. 386-7> 4 '3
Hippopede (fn-ev W£»}), a certain carve used

by Eudoxus 161-3, r ?6
Hoffmann, Heinrich 107
Hoffmann, John Jos. Ign. ro8, 365
Holtimann, Wilhelm (Xylander) 107
Bsmoesmerie (uniform) lines 40, 161, 161

Hornlike (angle), masntiHn 177, 178, 181,

165
Horsley, Samuel 106
Hoilel, J- 119
Hudson, John 101
Huttsch, F. 17 »., 74, 319, 400
HuruLin I). labia al-'Ibidt 75
Hypotheses, in Plato 11s: in Aristotle 118-

10: confused by Proclus with definition*

111-1: geometer's hypotheses not raise

(Aristotle) 119
Hypothetical construction 109
Hypsicles 5: author of Book xiv, 3, 6

lamblichus 63, 83, 417

Ibn al-'Amid 86

Ibn al-Haitham 88, 89

Ibn al-Lubudl 90

Ibn Rahawaihi al-Arjanl 86

Ibn Slna (Avicenna) 7;, 89

``Iflaton'' 88

Incomposite (of tines) 160- l , (of surfaces) 170

Indivisible lines (Jtoku yoauiat), theory of,
rebutted 168

Infinite, Aristotle on the, 131-4 1 infinite
division not assumed, but proved, by geo-
meters 16S

Infinity, parallels meeting at, 191-3

Ingrami, G. 1 75, 193, ins, 101, 117-8

Interior and exterior (of angles) 163, 180:
interior and opposite angle 180

Interpolations in the Elements before Theon's
time 58-63: by Theon 46, 55-6: 1. 40
interpolated 338

Irrational: discovered with reference to 1
35l.4H.4H~3  claim of India to priority
of discovery 363-4: ``irrational diameter
of 5'' {Pythagoreans and Plato) 399-400,
413: approximation to Ji by means of
``side- and ``diagonal- numbers 399-
401 , to J 1 and «/3 in sexagesimal fractions
7+fl-: Indian approximation to -J 1 36 r,
363-4: unordered irrationals (Apollonius)
41, t is: irrational ratio (dpprrror Xayot) 137

Isaacus Monachus for Argyrus) 73-4, 407

Ishiq b. I.lunain b. Ishaq al-'lbadl, Abu
Ya qub, translation ol Elements by, 75-80,

83-4
Isma'il b. Bulbul 88
Isoperimetric (or isometric) figures: Pappus

and Zenodorus on, 16, 17, 333
isosceles {laoaitt\iri) 1871 of numbers (= even)

188: isosceles right-angled triangle 351

Jakob b. Machir 76

Jan, C. 17

al-Jauhari, al-' Abbas b. Said 85

al-Jayyanl 90

Joannes Pediasimus 71-3

Junge, G-, on attribution of theorem of \prop{1}{47}

and discovery of irrationals to Pythagoras

35h 41'. 4'3

Kastner, A, G. 78, 97, tot
al-Karublsi 85

Kitviyana iiulbaSutra 360

Keill, John lot,, 1 £0-1 1

Kepler 193

al-Khasin, AbQ Jafar 77, 85

Killing, W, 194, 319, a? £-6, 135, 341, 172

al-Kindi j jj„ 86

Flamroth, M. 75-84

Klau (?), Chrisloph = Clavius 105

Kluge), G. S. «»

Knesa, Jakob III

Knoche jia, 3j«., 73

Kroll, W. 399-400

al-Kuhl 88

Lambert, J. H. 111-3

Lordlier, Dionysius III, 146, J 50, 198,

. «•*

Laseaxis, Constant inns 3

Leading theorems (as distinct from converse)
157: leading variety of conversion 156-7

Let lit, John no

Leftvre, Jacques 100

Legendre, Adrien Marie ui, 169, 113-9

Leibnii 145, 169, 176, 194

Leiden its. 399, 1 of al-Hsjjij and tut-
Nairīzī 11

Lemma 114; meaning 133-4: lemmas inter-
polated 59-60, especially from Pappus 67

Leodamas of Thasos 36, 134

Leon 116

Leonardo of Pisa 9 n., 10

Lcucippus 413

Linderup, II. C, 113

Line: Platonic definition 158: objection of
Aristotle 158: ``magnitude extended one
wav'' {Aristotle, ``Heromides'') 158:
``divisible or continuous one way'' (Aris-
totle) 158-9: ``flux of point'' 159: Apol-
lonlus on, if.ii: classification of lines, Plato
and Aristotle 159-60, Heron 139-60,
Geminus, first classification 160-1, second
161: straight (tWita), curved (MWfAdt),
circular (repit), spiral -shaped MAixo-
w*f), beat (*ixafi)i/nj) h broken (f»ff\a-
fitinf), round {rrpayyGXoi) 159, composite
(ffOrfartt), incomposite [asrvrQeTo*}, ``form-
ing a figure'' (rxijftVorwovffa), determinate
(fjipiefiJrTf), indeterminate {iopurrtn) 160:
*' asymptotic'' or nan -secant [daOfnrrwTM},
secant (ffvMrrwror) 161: simple, ``mixed''
161— 1: homoeomtrii (uniform) 161-1:
P roc I us on lines without extremities 165:
loci on lints 319, 330

Linear, loci 330: problems 330

Liomardo da Vinci, proof of 1, 47 365-6

Lippert 88 ».

Lobachewsky, K. L 174-5, nj, 119

Locus-theorems (rwiira fftvpittsara) and ioci

Sritroi); locus defined by Proclus 319:
oci likened by Chrysippus to Platonic
idea* 330- 1: locus-theorems and loci ( 1 ) on
lines (a) plane loci (straight lines and
circles) (b) solid loci (conics), (1) on sur-
faces 319: corresponding distinction be-
tween plant and solid problems, to which

Pappus adds linear problems 330: further
distinction in Pappus between ( 1) t<t*xT.xe\
(1) Juf oroi (3) ivaoTpoquKoi roroi 330:
Proclus regards locus in I- 35, III- ii, 31
as an area which is locus of area (parallelo-
gram or triangle) 330

Logical conversion, distinct from geometrical
156

Logical deductions 156, 184-5, 3°°< logical
equivalents 309, 314-5

Loreni, J. F. 107-8

Loria, Gino 7 «., ion., 11 «., 11 n.

Luca Paeiuolo 98-9, 1 00, 418

Lucas, Edouard 41S

Lundgren, F. A, A. 113

Machir, Jakob b. 76

Magni, Domenico 106

Magnitude: common definition vicious 1 48

al-Mahant 85

al-Ma'mun, Caliph 75

Manitius, C. 38

Mansion, P, 419

al-Mansvir, Caliph 75

Manuscripts of Elements 46-51

11 Marriage, Figure of'' (Plutarch), name for
Pythagorean triangle (3, 4, 5), 417

Martianus Capella 91, 155

Martin, T. H. 10, 19 »., 30 n.

Mas'ud b. al-Qass al-Bagdidi 90

Maximus Planudes, scholia and lectures on
Elements 71

meguar= axis 93

Mehler, F. G. 404

Menaechmus: story of M . and Alexander 1:
on elements 114; 117, J 15, 133''-

Menelaus 11, 13: direct proof of 1.15 300

Menge, H. 16 n., 17

Middle term, or cause, in geometry, illus-
trated by \prop{3}{31} 149

Mil!, J. S. 144

``Mixed'' (lines) ifii-i: (surfaces) 161, 170:
different meanings of ``mixed'' 161

Moceoigo, Prince 97-8

Mollweide. C. B, 108

Mondore (Montaureus), Pierre 101

Moses b. Tibbon 76

Motion, in mathematics 116: motion -with-
out deformation considered by Helmholtx
necessary to geometry 116-7, but shown
by Veronese to be peiiiio principii 116-7

Mttller, J. H. T. 189

Mllller, J. W. 365

Muhammad (b. 'Abdalbiqi) al-Bagdadt,
translator of De dwiiianiius 8 *., 90, 1 10

M ub, b. Ahmad AbQ 'r* Raihan al-Birunl 90

Muh. b. Asbraf Shamsaddtn as-Samarqandl

Mull. b. Isa Abu ``Abdallah al-Mahani 8j

Munich \textsc{ms.}\  of enunciations (ft) 94-5
. ``' adli
Rumi in., 90

Musa b- Muh. b. Mahmud Qadlaade ar-

Afurie, Elements sf {Seetio Canonis\, by

Euclid 17
al'Musla'sim. Caliph 90

al-Mutawakkil, Caliph 75

an-Nairtz!, A hi '1 'AbbSsal-Fadl b. Hatim,
»i-i4, 85, r8+, 190, 191, 195, 313, iii,
158, 170, 185, 190, 30 3, 316, 364, 367,

3*9. 373. 4<>5. 408
Napoleon 103
Nasi rid din at-Tusl 4, s ``• 77> 8 4> 8 9>

108-10
Naslf b. Yumn (Yaman) al-Qass 76, 77, 87
Neide, J. G. C- 103
Nicomachm 91, 417
Nicomedes 43, 160— 1, 165-6
Nipsus, Marcos Junius 305
Nominal and real definitions: see Definitions
``Nuptial Number'' m Plato's Geometrical

Number in Republic 417

Objection (trrrwit), technical term, in
geometry 135, 157, 160, 165: in logic
(Aristotle) 13 j

0fon° 61, 151, t88

Otnopides of Chios 34, 36, 116, 171, 195,

37L 4«4

Ofterdinger, L. F. 9

Olympiodorus 39

Oppermann 151

Optics of Euclid 17

Oresme, N. 97

Orontius Finaeus (Oronce Fine) tot, 104

Ozanam, Jaques 107, 108

Paciuolo, Luca 98-91 100, 418

Pamphile 317, 319

Pappus; contrasts Euclid and Apollonius 3:
on Euclid's Poris/m ro-r4, Surface-loci
15. 16, Dates 8: on Treasury of Analysis
8, lo, 11, 138: commentary on Elements
14-7, partly preserved in scholia 66:
evidence of scholia as to Pappus' text
66-7 1 lemmas in Book x. interpolated
from, 67: on Analysis and Synthesis 1 38-9,
141-1: additional axioms by, ij, 193, 194,
131: on converse of Post. 4 35, sot:
proof of I- 3 by, 154: extension of 1. 47
366: semi-algebraical methods in 373,
378: on loci 319, 330: on conchoids 161,
366: on quadratrix 366: on isoperimctric
figures 16, 17, 333: on paradoxes of
Erycinos 17, 390: 30, 39, 133 n., 137,
'S'. ``5- 388. 391, 401

Papyrus, Herculanensis No. 106 1 50, 184;
Oxyrhynchus 50: Fayum 51, 337, 338:
Rhind 304, 3ji

Paradoxes, in geometry 188: of Eryrinus
27, 190, 319: an ancient ``Budget of
Paradoxes 319

Parallelogram { * parallelogram mic area),
first introduced 335: rectangular parallelo-
gram 370

Parallels: Aristotle on, 190, 191-3: defini
tions. l>y ``Aganis'' 191, by Geminus 191
Posidomus 190, Simplicius 190: as equi'
distanls 190-1, 194: direction-theory of,
191-1, 194: definitions classified 193-4

Veronese's definition and postulate 194:
Parallel Postulate, ite Postulate 3:
Legendie's attempt to establish theory of
113-9

Paris \textsc{mss.}\  of Elements, (p) 49: (q) 50

Pascb, M. 157, 118, 130

``Peacock's tail,'' name for \prop{3}{8} 99, 418

Pediasimus, Joannes 71-3

Feet, T. Eric 331

Peithon 103

Peletarius (Jacques Peletier) 103, 104, 149,
407

Fena 104

Perpendicular (codcro*): definition t8l:
``plane'' and ``solid'' 171: perpendicular
and obliques 391

Perseus 41, 161-3

Peseh, J. G. van, De Prodi fentibsts 13 sqq.,
39 n.

Petrus Montaureus {Pierre Mondore) 103

Peyrard and Vatican \textsc{ms.}\  190 (P) 46, 47,
103: 108

Pfleiderer, C. F, 168, 198

Phacnomena of Euclid 16, 17

Phi lip pus of Med ma t, 116

Phillips, George 111

Philo of Byzantium 10, 13: proof of \prop{1}{8}
163-4

Fhilolaus 34, 331, 371, 399, 411

Philoponus 45, iot -3

Pirekenstein, A. E. Burkh. von 107

Plane (or plane surface): Pinto's definition
of, 171: Produs' and Simplicius' inter-
pretation of Euclid's def. 171: possible
origin of Euclid's def. 171: Archimedes'
assumption 171, 171: other ancient defini-
tions of, in Froclus, Heron, Theon of
Smyrna, an-NairUI 171-3: ``Simson's''
definition and Gauss on 171-3: Crelle's
tract on, 173-41 other definitions by
Fourier 173, Deahni. 174, J. K- Becker
174, Leibniz 176, Beez 176: evolution of,
by Bolyai and Lobachewsky 174-51
Enriqucs and Amaldi, Ingrami, Veronese
and Hilbert on, 175

``Plane loci'' 319-30: Plata Lett of Apol-
lonius 14, 159, 330

``Plane problems'' 319

Ptanudes, Maximus 71

Plato: 1, 1, 3, 137, ijs-6, IJ9. 184, 187.
103, 331, 411, 417: supposed invention of
Analysis by, 134: def. of straight line 165-
6: def. of plane surface 171! generation of
cosmic figures by putting together triangles
136, 413: rule for rational right-angled
triangles 336, 337, 339, 360, 385: ``rational
diameter of 3 ``399, 413

``Platonic'' figures 3, 413-4

Play fair, John 103, 111: ``Playfair's ``
Avium 310: used to prove \prop{1}{19}, 311, and
Eucl. Post. 5, 313: comparison of Axiom
with Post. 3, 313-4

Pliny 333

Plutarch 19, 37. '77. 343. 3S>> 4' 7

Point: Pythagorean definition of, 155: inter-
pretation of Euclid's definition 155: Plato's
view of, and Aristotle's criticism 155-6:
attributes of, according to Aristotle 1 56:
terms for {ffriypij, oijfietor) 156: other
definitions by ``Herundes,'' Posidonius
156, Simp] [ci us 11}: negative character of
Euclid's def, 156: is it sufficient? 156:
motion of, produces line 1 57: an-Nairfil
on, r J7: modem explanations by abstrac-
tion 157

Folybius 331

Polygon: sum of interior angles (Proclus 1
proof) 313: sum of exterior angles 311

**Pens asinorum'' 415-6: ``Pontaux Ants''
ibid.

Porism: two senses [3: (1) = corollary 134,
178-9 1 interpolated Ponsms (corollaries)
60-1, 38 r: (1) as used in Porisms of Euclid,
distinguished from theorems and problems
10, r t: account of the Perisms given by
Pappus ro-13: modem restorations by
Simson and Chasles 14: views of Heiberg
ri, 14, and Zculben 15

Porphyry 17: commentary on Euclid 34:
Symmiita Ha, 34, 44: 136, 17;, 183, 187

Posidonius, the Stoic 30, 17, i8n., 30, 189,
197: book directed against the Epicurean
Zeno 34, 43: on parallels 40, 100: defini-
tion of figure 41, 183

Postulate, distinguished from axiom, by
Aristotle nB-9, by Proclus (Gemmus
and ``others'') 1*1-3: from hypothesis,
by Aristotle tio-l, by Proclus [31-41
postulates in Archimedes no, 113:
Euclid's view of, reconcileable with
Aristotle's mo-jo, 114: postulates do not
confine us to ruler and compasses 114:
Postulate* 1, 1, significance of, 195-0:
famous ``Postulate of Archimedes'' 134

Postulate 4: significance of, 100: proofs of,
resting on other postulates r *oo-i, 131*
converse true only when angles rectilineal
(Pappus) lot

Postulate 5: due to Euclid himself 101:
Proclus on, IM-3: attempts to prove,
Ptolemy SO4-6, Proclus 106-8, Nasiraddln
at-Tusl io8-ro, Wallis 11c— 1, Ssccheri
1 1 1 -1 , Lambert 1 1 1 -3: subst i tu tes for,
``Playfair's'' axiom (in Proclus) 110, others
by Proclus 107, 110, Posidonius and
Gemmus 110, Legendre 113, Hi, 110,
Wallis 110, Cm hot, Laplace, Loreni,
W, Bolyai, Gauss, Worpitiky, Clairaut,
Veronese, Ingrami 110 1 Post, j proved
from, and compared with, ``Playfair's''
axiom 313-4: 1. 30 is logical equivalent
of, 110

Foils, Robert 111, 146

Prime (of numbers): two senses of, 146

Principles, First n 7- (14

Problem, distinguished from theorem 114-8:
problems classified according to number of
solutions (a) one solution, ordered (Tmry-
iti'D.) {i) a definite number, intermediate
fjUta) f>) an infinite number of solutions,

unordered (iraim) 118: in widest sense
anything propounded (possible or not) but
generally a construction which is possible
118-9: another classification (1} problem
in excess (rXfordfor), asking too much 119,
(1) deficient problem (AXtWr wp6fi\ifua) t
giving too little 119

Proclus: details of career 19-30: remarks
on earlier commentators 19, 33, 45; com-
mentary on End. I, sources of, 19-45,
object and character of, 31-1: com-
mentary probably not continued, though
continuation intended 31-3: books
quoted by name in, 34: famous ``sum-
mary ``37-8: list of writers quoted 44:
his own contributions 44-5: character of
MS- used by, 61, 63: on the nature of
elements and things elementary 1 14-6: on
advantages of Euclid's Elements, and
their object 115-6: on first principles,
hypotheses, postulates, axioms 111-4: on
difficulties in three distinctions between
postulates and axioms 113: on theorems
and problems 124-9: attempt to prove
Postulate 5 106-8: on Eucl. \prop{1}{47}, 350,
411: on discovery of five regular solids
413: commentary on Plato's Republic,
allusion in, to ``side-'' and ``diagonal-''
numbers in connexion with Eucl. 11. 9, 10
399-400

Proof ' fdwAotifir), necessary part of pro-
position 110-30

Proposition t formal divisions of, 119-131

Protarchus 5

Psellui, Michael, scholia by, 70, 71

Pstudaria of Euclid 7: Psettaographtmataln.

Pseudoboethius 91

Ptolemy I.: I, 1: story of Euclid and
Ptolemy 1

Ptolemy II. Philadelphia 10

Ptolemy VII. (Euergetes II.), Physcon ic

Ptolemy, Claudius 11, 30 n.\ Harmonica of ,
and commentary on 17: an Parallel-Pos-
tulate 18*., 34, 43, 45: attempt to prove
it 104-6

Punch on ``Pens Asinoruin'' 416

Pythagoras 4 »., 36: supposed discoverer of
the irrational 351, 41 1, 411, of application
0/ areas 343-4, of theorem of \prop{1}{47} 343-4,
350-4, 411. 41s, of construction of five
regular solids 413-4; story of sacrifice 37,
3+3, 350: probable method of discovery of
\prop{1}{47} and proof of, 351-5: suggestions by
Bretschneider and Ilankel 3 J 4, by Zeuthen
355-6: rule for forming right-angled tri-
angles in rational numbers 331, 356-9, 383

Pythagoreans 10, 36, 155, 188, 179, 411-414:
term for surface (xpout) 169: angles of tri-
angle equal to two right angles, theorem
and proof 3 17-10: three polygons which in
contact fill space round point 318: method
of application of areas (including exceeding
and falling short) 343, 3B4, 403: gnomon
Pythagorean 351: ``rational ``and ``ir-
rational diameter of 5'' 399-400, 413

Qirllzade ar-Rumi j«., 90

Q.E.D. (or F.) 57

al-Qifli n., 94

Quadratic eq nation, geometrical solution of,
383-5, 386-8: solution assumed by Hippo-
crates 386-7

Quadrserix 365-6, 330

Quadrature [TeTpaywvtrfii), definitions of,

'♦? . .

Quadrilaterals, varieties of, 188-90

Quintilian 333

Qusta b- Luqa al-Ba'labakkl, translator of

``Books xiv, xv'' 76, 87, 88

Radius, no Greek word for, 190

Ramus, Petals (Pierre de la Ramee) 104

Ratdolt, Erhard 78, 97

Rational {fart): (of ratios) 137: ``rational
diameter of 5 ``399-400, 413: rational
right-angled triangles, see right-angled
triangles

Rauchfuss, tee Dasypodius

Rausenberger, O. 157, 175, 313

ar-RazI, Abu Yusuf Ya'qub b. Muh. 86

Rectangle: = rectangular parallelogram 370:
``rectangle contained by'' 370

Rectilineal angle: definitions classified 179—
81: rectilineal figure 187: ``rectilineal
segment'' 196

Rtductio ad absurdum 1 34: described by
Aristotle and Prochts r3<S: synonyms for,
in Aristotle 136: a variety of Analysis
140: by exhaustion 185, »93 I nominal
avoidance of 369

Reduction (4ritT«7i)), technical term, ex-
plained by Aristotle and Proctns 135:
first ``reduction'' of a difficult construction
due to Hippocrates 135

Regiomontanus (Johannes Miiller of Kdnigs-
berg) 93, 96, 100

Reyher, Samuel 107

Rhaeticus 101

Rhind Papyrus 304, 351

Rhomboid 63, 151, 189

Rhombus 6i t 151, meaning and derivation
189

Riccardi, p.~96, ill, 101

Riemann, B- 119, 173, 174, 180

Right angle: definition 181; drawing straight
line at right angles to another, Apollonius'
construction for, 370: construction when
drawn at extremity of second line (Heron)
370

Right-angled triangles, rational: rule for
finding, by Pythagoras 356-9, by Plato
3S*i 3S7i 3S9p 3»°. 3*5: discovery of rules
by means of gnomons 358-60: connexion
of rules with Eucl- II- 4, 8, 360: rational
right-angled triangles in Apastamba 361,

Roth 357-8

Rouchi and de Comberousse 313
Rndd, Capt. Thos. 110
Ruellius, Joan. (Jean Rue!) 100
Russell, Bertrand 117, 149

Saccheri, Gerolamo 106, 114-5, 167-8,
i8j-6, iQ4, t 107-8, ioc-i

Sa'ld b. Mas'Qd b. at-Qass 90

Sathapatha-Brahmana 362

Savile. Henry 105, 160, 145, 150, 161

Scalene (<riraAfjPOi or ffxaArfjr) 187-8: of
numbers ( = odd) 188: of cone (Apollonius)
1S8

Schessler, Chr. 107

Scheubel, Joan. 101 , 107

Schiaparelli, G. \prop{5}{163}

Schmidt. Max C. p.~304, 319

Scholia to Elements and MSs. of, 64-74:
historical information in, 64: evidence \x\ t
as to text 64-5 66-7: sometimes inter*
polated in text 67 i classes of, ``Schol-
Vat.'' 65-9, ``Schol. Vind.'' 69-70: miscel-
huieons 71-4: ``Schol. Vat.'' partly derived
from Pappus' commentary 66: many
scholia partly extracted from Proclus on
Bk. \prop{1}{66}, 69, 7a 1 numerical illustrations
in, in Greek and Arabic numerals 71:
scholia by Psellus 70-1, by Maximus
Planudes 73, Joannes PedJasimus 7.-3
scholia in Latin published by G. Valla,
Commandinus, Conrad Dasypodius 73
scholia on Eucl. \prop{2}{13} 407

Schooten, Franz van 108

Schopenhauer 117, 354

Schotten, H. 167, 174, 179, 191-3, 101

Schumacher 311

Schur, F. 318

Schweikart, F. K. 319

Scipio Vegius 99

Sectio Canenis attributed to Euclid 17

Section (rop): = point of section 1 70, 171,
383: ``tie section'' = ``golden section'' q.v.

Segment of circle, angle of, 353: segment
less than semicircle called iir 187

Semicircle 186 I centre of, 186: angle of,
18a, 153

Stqt 304

Serenus of Antinoeia 103

Serle, George no

Setting-out (fftfcjit), one of forma] divisions
of a proposition 119: may be omitted 130

Sexagesimal Tractions in scholia to Book x.

Sextus Empiricus 61 , 63, 184
Shamsaddln as-Samarqanol 5 rr. t 89
``Side-'' and ``diagonal-'' numbers, described
398-400: due to Pythagoreans 400; con-
nexion with Eucl. \prop{2}{9}, 10 396-400: use
for approximation to *ji 399
Sigboto 94

``Similar'' (=equal) angles 181, 151: ``simi-
lar'' numbers 357
Simon, Max 108, 155, 157-8, 167, 101,

SiropKcius: commentary on Euclid 371-8;

on tones of Hippocrates 39, %$<> 386-7;

on Eademufi 1 style 35 , 38: on parallel*

1 90-1*: «, 167, 17 J, 184, 185, ig;, 103,

«3* 114, 413
Siinson. Robert: on Euclid's Pviismi 14:

on M vitiations * in Elements due to Theon
46, 103, 104, 106, in, [48: definition
of plane 171-3: 185, rS6, 355, 150, 287,

*93- *9*- 3''> l* 8 * i* 3 8 7i 4<»3

Sind b. All Abu \Tuyib 86

Smith, D. E, 363/417

Smith and Bryant 404

* Solid loci ``3191 330: Solid Loci of Aris-
tarns 1 -ft, 319

``Solid problems'' 319, 330

Speusippug 115

.S/Aamfd, early treatise on, 17

Spiral, ``single-turn'' 131-3*1., 164- j: in
Pappus = eylindrical helix 165

Spiral of Archimedes 16, 167

Spire (tore) or JT/iWf surface 163, 170;
varieties of 163

Spine curves or sections, discovered by
Perseus 161, 161-4

Steenstra, Pybo 109

Steiner, Jakob 193

StcinmeU, MoriU toi

St einchn eider, M. 8«., 76 sqq.

Stephaiius Gracilis joi-i

Stephen Clericus 47

Stobaeus 3

Stoic ♦'axioms 11 41, 111: illustrations [B*ty-
fnn\ 319

StoU, 0. 35 8

Stone, E. 105

Straight line: pre- Euclidean (Platonic) de-
finition 165-6 h Archimedes* assumption
respecting, 166; Euclid's definition, inter-
preted by Proclus and SimpLicius 166-7:
language and construction of, 167, and
conjecture as to origin 168 J other defj'
ni lions 168-9, m Heron 1 68, by Leib-
niz 169 by Legend re (69: two straight
lines cannot enclose a space 10,5-6* can*
not have a common segment 196-9: one
or two cannot make a figure 169, 183;
division of straight line into any number
of equal parts (an-Nairīzī) 326

Stromer, Marten 113

Studemnnd, W. 92 n.

St Vincent, Gregory of, 401, 404

Subtend, meaning and const ruction 249 ,
183, 350

Suidas 370

Sulaiman b. 'Usma (or Oqba) 85, 90

Superposition: Euclid's dislike of method
of, 235, 349: apparently assumed by Aris-
totleas legitimate 216: used by Archimedes
335: objected to by Peletarius 349: no use
theoretically, but merely furnishes practical
test of equality 117: Bertrand Russell on,

«7j *49
Surface: Pythagorean term for, xpoid { = col-
our, or skin) 169: terms for, in Plato and
Aristotle 169; iridwtta in Euclid (not
i-KlTrtSor) 169: alternative definition of, in
Aristotle 170: produced by motion of
line 170: divisions or sections of solids
are surfaces 170, 171: classifications of
surfaces by Heron and Gemmu* 170: com-

posite, incomposite, simple, mixed 170:
spin) surfaces 163, 170: homoeomeric
(uniform) 170: spheroids 170: plane sur-
face, see plane: loci on surfaces 319, 330

Surface-loci of Euclid 15, 16, 330: Pappus'
lemmas on, 15, 16

Suter, H. 8 tt., 17 w., 18 71., 35 «,, 78 n.,

85-90
Suvoroff, Fr, 113
Swmden, J. II. van 169
Synthesis, see Analysis and Synthesis
Syrianus 30, 44, 176, 178

Tacquet, Andre' 103, iq;, in

Taittirlya-Samhita 363

Tannery, p.~7 «♦, 37-40, 44* *** J *3i ``»

113, 114, 125, 132, 305, 353, 412, 417
TWrikh al-HuktiMtl 4 n.
TartagUa, Niccolo 3, 103, 106
Taurinus, F, A, 219
Taurus 62, J 84

Taylor, H. M. 248, 377-8, 404
Taylor, Th- 259
Thabit b. Qurra, translator of Elements

42, 75-80, 82, 84, 87, 94: proof of I.

47 34-5
Thales 36, 37, 185, 252, 253, 278, 3i7t
318, 319: on distance of ship from snore

3=>+- 5

Tbeaetetus 1, 37

Theodorus Antiochita 71

Theodorus Cabas i las 72

Theodorus Metochita 3

Theodorus of Cyrene 4 n, 412-3

Theognis 371

Theon of Alexandria: edition of Elements
46: changes made by, 46; Simson on
``vitiations'' by, 46: principles for detect-
ing his alterations, by comparison of P,
ancient papyri and ``Theomne'' ms$, 51-
3: character of changes by, 54-8

Theon of Smyrna in, 357, 358, 371, 39 s

Theorem and problem, distinguished by
Speusippus 125, Amphinomus 125, 128,
Menaechmus 125, Zenodotus, Fosidonius
1 16. Euclid r 26, Carpus 137, 128:
views of Proclus 127-8, and of Geminus
128: ``general 11 and ``not -general'' (or
partial) theorems (Proclus) 325

Theudius of Magnesia II Jf

Tbibaut, B- F. a*'

Thibaut, C-: On Sulvasutras 360, 363-4

Thompson, Thomas Perronet 112

Thucydides 333

Tibbon, Moses b. 76

Tiraboschi 94 «.

Title I, K, 38, 39

Todhunter, I, J12 189, 2461 258, 377,
283, 293, 298, 307

Tonstall, Cuthbert 100

Tore 163

Transformation of areas 346-7, 410

Trapezium 1 Euclid's definition his own 189:
further division into trapezia and trape-
zoids (Posidonius, Heron) 189-90; a
theorem on area of parallel -trapezium

338-9

Trtasury tf Attatyat (AttLkxibiuoat Ttaror}
8, 10. 1 [, 138

Trendelenburg 1461*., 1 48, 149

Treullein, p.~358-6°

Triangle: seven species of, 188: ``four-
sided ``triangle, called alto ``barb- like''
{juioit4i) and (by Zenodorus) xoikoytlt-
vtor 17, 188: construction of isosceles and
scalene triangles 343

Trisection of an angle 165-7

at-Tusl, lee Nasiraddin

Unger, E. S. 108, 169

Vacca, Giovanni 113

Vachtchenko-Zakhartchenko it]

Vailati, G. 144 n., Mi''.

Valerius Maiimus 3

Valla, G., Dt expdendii et fugitndii rtbtu

73. 98
Van Swinden 169
Vatican \textsc{ms.}\  190 (P) 46, 47
Vaui, Cam de 10
Verona palimpsest 91
Veronese, G, 157, 168, 175, 180, 193-+,

195, 101, u(4-7, ``8, 149, 318
Vtrtxal (angles) 178
Viennese MS, (V) 48, 49
Vinci, Llonando da 365-6
Vitali, G. 13;
Vitruvius 351; Vitruvius and Heron 10,

11
Viviani. Vincenzo 107, 401

Vogt, Heinrich 360, 364, 411-414
Vooght, C. J. 108

Wachsmuth, C. 31 n., 73

Wallis, John 103: edited Coram, on Ptol-
emy's Harmenua 17: attempt to prove
Post. 5 iro-i

Weber (H.) and Wellstein (J.) 157

Weissenborn, H. 78 n., 91 »., 94 s., 95,

96. 97 *> +'8
Whiston, W. Ill
Williamson, James in, 193
Witt, H. A. 113
Woepcke, F., discovered De divistottibus in

Arabic and published translation 9: on

Pappus' commentary on Elcmmlt 15, 66,

77: 8s «., 86, 87

Xenocraies 168, 413
Ximenes, Leonardo 107
Xylan der 107

Yahya b. Khalid b. Barmak 75

Yahya b. Muh. b. 'Abdan b. Abdalwahid

(tbn a]-Lubudl) 90
Yrinus = Heron 11
Yuhanni b. Yusuf b. al-Haritb b. el-Bitrtq

al-Qaas 76, 87

Zamberti, Bartolomeo 98-100, 101, 104, 106
Zeno the Epicurean 34, 196, 197, 199, 141
Zenodorus j6, 17, 1 88, 333
Zenodotus 116

Zeuthen, H. G. 15, 1(3, 139, 141, 146ft.,
IS'. 35S-*. ite- 3*3- 3 8;, 39''. 399

