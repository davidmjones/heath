\documentclass[reqno]{heath}

\begin{document}

\frontmatter

\title{The Thirteen Books of Euclid's Elements}

\begin{titlepage}

\begin{center}

\begingroup

\openup\bigskipamount

{\LARGE \textbf{THE THIRTEEN BOOKS OF}}

\bigskip

{\Huge \textbf{EUCLID'S ELEMENTS}}

\medskip

TRANSLATED FROM THE TEXT OF HEIBERG

\bigskip

{\Large WITH INTRODUCTION AND COMMENTARY}

\endgroup

\bigskip
\bigskip
\bigskip

\begingroup

BY

\medskip

Sir {\large THOMAS L. HEATH},

{\small K.C.B., K.C.V.O., F.R.S.,}

\medskip

{\SMALL SC.D. CAMB., HON. D.SC. OXFORD}

{\SMALL HONORARY FELLOW (SOMETIME FELLOW) OF TRINITY COLLEGE CAMBRIDGE}

\endgroup

\vfill

\emph{SECOND EDITION}

\medskip

\emph{REVISED WITH ADDITIONS}

\vfill
\vfill

\end{center}

\clearpage

\thispagestyle{empty}

\null

\vfill

\emph{Library of Congress Catalog Card Number: 56-4336}

\end{titlepage}

\chapter*{Preface}

``There never has been, and till we see it we never
shall believe that there can be, a system of geometry
worthy of the name, which has any material departures (we do
not speak of \emph{corrections} or \emph{extensions} or \emph{developments}) from
the plan laid down by Euclid.'' De Morgan wrote thus in
October 1848 (\emph{Short supplementary remarks on the first six
Books of Euclid's Elements} in the \emph{Companion to the Almanac}
for 1849); and I do not think that, if he had been living
to-day, he would have seen reason to revise the opinion so
deliberately pronounced sixty years ago. It is true that in the
interval much valuable work has been done on the continent
in the investigation of the first principles, including the
formulation and classification of axioms or postulates which
are necessary to make good the deficiencies of Euclid's own
explicit postulates and axioms and to justify the further
assumptions which he tacitly makes in certain propositions,
content apparently to let their truth be inferred from observation of
the figures as drawn; but, once the first principles are
disposed of, the body of doctrine contained in the recent text-books of
elementary geometry does not, and from the nature
of the case cannot, show any substantial differences from that
set forth in the \emph{Elements}. In England it would seem that far
less of scientific value has been done; the efforts of a multitude
of writers have rather been directed towards producing
alternatives for Euclid which shall be more suitable, that is to say,
easier, for schoolboys. It is of course not surprising that, in
these days of short cuts, there should have arisen a movement
to get rid of Euclid and to substitute a ``royal road to
geometry''; the marvel is that a book which was not written
for schoolboys but for grown men (as all internal evidence
shows, and in particular the essentially theoretical character
of the work and its aloofness from anything of the nature of
``practical'' geometry) should have held its own as a
school-book for so long. And now that Euclid's proofs and
arrangement are no longer required from candidates at examinations
there has been a rush of competitors anxious to be first in the
field with a new text-book on the more ``practical'' lines which
now find so much favour. The natural desire of each teacher
who writes such a text-book is to give prominence to some
special nostrum which he has found successful with pupils.
One result is, too often, a loss of a due sense of proportion;
and, in any case, it is inevitable that there should be great
diversity of treatment. It was with reference to such a danger
that Lardner wrote in 1846: ``Euclid once superseded, every
teacher would esteem his own work the best, and every school
would have its own class book. All that rigour and exactitude
which have so long excited the admiration of men of science
would be at an end. These very words would lose all definite
meaning. Every school would have a different standard;
matter of assumption in one being matter of demonstration in
another; until, at length, \textsc{Geometry}, in the ancient sense of
the word, would be altogether frittered away or be only
considered as a particular application of Arithmetic and
Algebra.'' It is, perhaps, too early yet to prophesy what will
be the ultimate outcome of the new order of things; but it
would at least seem possible that history will repeat itself and
that, when chaos has come again in geometrical teaching,
there will be a return to Euclid more or less complete for the
purpose of standardising it once more.

But the case for a new edition of Euclid is independent of
any controversies as to how geometry shall be taught to
schoolboys. Euclid's work will live long after all the text-books
of the present day are superseded and forgotten. It is one
of the noblest monuments of antiquity; no mathematician
worthy of the name can afford not to know Euclid, the real
Euclid as distinct from any revised or rewritten versions
which will serve for schoolboys or engineers. And, to know
Euclid, it is necessary to know his language, and, so far as it
can be traced, the history of the ``elements'' which he
collected in his immortal work.

This brings me to the \emph{raison d'\^etre} of the present edition.
A new translation from the Greek was necessary for two
reasons. First, though some time has elapsed since the
appearance of Heiberg's definitive text and prolegomena,
published between 1883 and 1888, there has not been, so far
as I know, any attempt to make a faithful translation from it
into English even of the Books which are commonly read.
And, secondly, the other Books, \book{vii} to \book{x} and
\book{xiii}, were not
included by Simson and the editors who followed him, or
apparently in any English translation since Williamson's
(1781–8), so that they are now practically inaccessible to
English readers in any form.

In the matter of notes, the edition of the first six Books
in Greek and Latin with notes by Camerer and Hauber
(Berlin, 1824–5) is a perfect mine of information. It would
have been practically impossible to make the notes more
exhaustive at the time when they were written. But the
researches of the last thirty or forty years into the history of
mathematics (I need only mention such names as those of
Bretschneider, Hankel, Moritz Cantor, Hultsch, Paul Tannery,
Zeuthen, Loria, and Heiberg) have put the whole subject
upon a different plane. I have endeavoured in this edition
to take account of all the main results of these researches up
to the present date. Thus, so far as the geometrical Books
are concerned, my notes are intended to form a sort of
dictionary of the history of elementary geometry, arranged
according to subjects; while the notes on the arithmetical
Books \book{vii}—\book{ix} and on Book~\book{x} follow the same plan.

I desire to express here my thanks to my brother,
Dr R. S. Heath, Vice-Principal of Birmingham University,
for suggestions on the proof sheets and, in particular, for the
reference to the parallelism between Euclid's definition of
proportion and Dedekind's theory of irrationals, to Mr R. D.
Hicks for advice on a number of difficult points of translation,
to Professor A. A. Bevan for help in the transliteration of
Arabic names, and to the Curators and Librarian of the
Bodleian Library for permission to reproduce, as frontispiece,
a page from the famous Bodleian MS. of the \emph{Elements}.
Lastly, my best acknowledgments are due to the Syndics of
the Cambridge University Press for their ready acceptance
of the work, and for the zealous and efficient cooperation of
their staff which has much lightened the labour of seeing the
book through the Press.

\byline{T. L H.}{\emph{November}, 1908.}

\chapter*{Preface to the Second Edition}

I like to think that the exhaustion of the first edition of
this work furnishes a new proof (if such were needed)
that Euclid is far from being defunct or even dormant, and
that, so long as mathematics is studied, mathematicians will
find it necessary and worth while to come back again and
again, for one purpose or another, to the twenty-two-centuries-old
book which, notwithstanding its imperfections, remains the
greatest elementary textbook in mathematics that the world is
privileged to possess.

The present edition has been carefully revised throughout,
and a number of passages (sometimes whole pages) have been
rewritten, with a view to bringing it up to date. Some not
inconsiderable additions have also been made, especially in the
Excursuses to Volume~I, which will, I hope, find interested
readers.

Since the date of the first edition little has happened in the
domain of geometrical teaching which needs to be chronicled.
Two distinct movements however call for notice.

The first is a movement having for its object the mitigation
of the difficulties (affecting in different ways students, teachers
and examiners) which are found to arise from the multiplicity
of the different textbooks and varying systems now in use for
the teaching of elementary geometry. These difficulties have
evoked a widespread desire among teachers for the
establishment of an agreed sequence to be generally adopted in teaching
the subject. One proposal to this end has already been made:
but the chance of the acceptance of an agreed sequence has in
the meantime been prejudiced by a second movement which
has arisen in other quarters.

I refer to the movement in favour of reviving, in a modified
form, the proposal made by Wallis in 1663 to replace Euclid's
Parallel-Postulate by a Postulate of Similarity (as to which see
pp.~210–11 of Volume~I of this work). The form of Postulate
now suggested is an assumption that ``Given one triangle,
there can be constructed, on any arbitrary base, another triangle
equiangular with (or similar to) the given triangle.'' It may
perhaps be held that this assumption has the advantage of not
referring, in the statement of it, to the fact that a straight line
is of unlimited length; but, on the other hand, as is well known,
Saccheri showed (1733) that it involves more than is necessary
to enable Euclid's Postulate to be proved. In any case it
would seem certain that a scheme based upon the proposed
Postulate, if made scientifically sound, must be more difficult
than the procedure now generally followed. This being so,
and having regard to the facts (1)~that the difference between
the suggested Postulate and that of Euclid is in effect so slight
and (2)~that the historic interest of Euclid's Postulate is so
great, I am of opinion that the proposal is very much to be
deprecated.

\byline{T. L. H.}{\emph{December} 1925.}

\tableofcontents

\mainmatter

\part*{Introduction}

\chapter{Euclid and the Traditions about Him}

As in the case of the other great mathematicians of Greece, so in
Euclid's case, we have only the most meagre particulars of the life
and personality of the man.

Most of what we have is contained in the passage of Proclus' summary
relating to him, which is as follows\footnote{Proclus, ed. Friedlein,
  p.~68, 6--20.}:

``Not much younger than these (sc.\ Hermotimus of Colophon and
Philippus of Medma) is Euclid, who put together the Elements,
collecting many of Eudoxus' theorems, perfecting many of Theaetetus',
and also bringing to irrefragable demonstration the things which were
only somewhat loosely proved by his predecessors. This man
lived\footnote{The word \greek{γέγονε} must apparently mean
  ``flourished,'' as Heiberg understands it
  (\emph{Litterargeschichtliche Studien über Euclid}, 1882, p.~26),
  not ``was born,'' as Hankel took it; otherwise part of Proclus'
  argument would lose its cogency.} in the time of the first
Ptolemy. For Archimedes, who came immediately after the first
(Ptolemy)\footnote{So Heiberg understands \greek{ἐπιβαλὼν τῷ πρώτῳ}
  (sc. \greek{Πτολεμαίῳ}). Friedlein's text has \greek{καί} between
  \greek{ἐπιβαλὼν} and \greek{τῷ πρώτῳ} and it is right to remark
  that another reading is \greek{καί ἐν τῷ πρώτῳ} (without
  \greek{ἐπιβαλών}) which has been translated ``in his first
  \emph{book},'' by which is understood \emph{On the Sphere and
    Cylinder}~\textsc{i}., where (1)~in Prop.~2 are the words ``let
  $BC$ be made equal to $D$ \emph{by the second \emph{(proposition)}
    of the first} of Euclid's (books),'' and (2)~in Prop.~6 the words
  ``For these things are handed down in the Elements'' (without the
  name of Euclid).  Heiberg thinks the former passage is referred to,
  and that Proclus must therefore have had before him the words ``by
  the second of the first of Euclid'': a fair proof that they are
  genuine, though in themselves they would be somewhat suspicious.},
makes mention of Euclid: and, further, they say that Ptolemy once
asked him if there was in geometry any shorter way than that of the
elements, and he answered that there was no royal road to
geometry\footnote{The same story is told in Stobaeus,
  \emph{Ecl}.\ (\textsc{ii}. p.~228, 30, ed.\ Wachsmuth) about
  Alexander and Menaechmus. Alexander is represented as having asked
  Menaechmus to teach him geometry concisely, but he replied: ``O
  king, through the country there are royal roads and roads for common
  citizens, but in geometry there is one road for all.''}.  He is then
younger than the pupils of Plato but older than Eratosthenes and
Archimedes; for the latter were contemporary with one another, as
Eratosthenes somewhere says.''

This passage shows that even Proclus had no direct knowledge of
Euclid's birthplace or of the date of his birth or death. He proceeds
by inference. Since Archimedes lived just after the first Ptolemy, and
Archimedes mentions Euclid, while there is an anecdote about
\emph{some} Ptolemy and Euclid, \emph{therefore} Euclid lived in the
time of the first Ptolemy.

We may infer then from Proclus that Euclid was intermediate between
the first pupils of Plato and Archimedes. Now Plato died in 347/6,
Archimedes lived 287–212, Eratosthenes \emph{c}.~284-204~\bc. Thus
Euclid must have flourished \emph{c}.~300~\bc, which date agrees well with
the fact that Ptolemy reigned from 306 to 283~\bc.

It is most probable that Euclid received his mathematical training in
Athens from the pupils of Plato; for most of the geometers who could
have taught him were of that school, and it was in Athens that the
older writers of elements, and the other mathematicians on whose works
Euclid's \emph{Elements} depend, had lived and taught. He may himself
have been a Platonist, but this does not follow from the statements of
Proclus on the subject. Proclus says namely that he was of the school
of Plato and in close touch with that philosophy\footnote{Proclus,
  p. 68, 20, \greek{καὶ τῇ προαιπέσει δέ Πλατωνικός ἐστι καὶ τῇ
    φιλοσοφίᾳ ταύτῃ οἰκεῖος}.}. But this was only an attempt of a New
Platonist to connect Euclid with his philosophy, as is clear from the
next words in the same sentence, ``for which reason also he set before
himself, as the end of the whole Elements, the construction of the
so-called Platonic figures.'' It is evident that it was only an idea
of Proclus' own to infer that Euclid was a Platonist because his
\emph{Elements} end with the investigation of the five regular solids,
since a later passage shows him hard put to it to reconcile the view
that the construction of the five regular solids was the end and aim
of the \emph{Elements} with the obvious fact that they were intended
to supply a foundation for the study of geometry in general, ``to make
perfect the understanding of the learner in regard to the whole of
geometry\footnote{\emph{ibid}. p.~71, 8.}.'' To get out of the
difficulty he says\footnote{\emph{ibid}. p.~70, 19 sqq.} that, if one
should ask him what was the aim (\greek{σκοπός}) of the treatise, he
would reply by making a distinction between Euclid's intentions (1)~as
regards the subjects with which his investigations are concerned,
(2)~as regards the learner, and would say as regards (1)~that ``the
whole of the geometer's argument is concerned with the cosmic
figures.'' This latter statement is obviously incorrect It is true
that Euclid's \emph{Elements} end with the construction of the five
regular solids; but the planimetrical portion has no direct relation
to them, and the arithmetical no relation at all; the propositions
about them are merely the conclusion of the stereometrical division of
the work.

One thing is however certain, namely that Euclid taught, and founded a
school, at Alexandria. This is clear from the remark of Pappus about
Apollonius\footnote{Pappus, \r7, p.~678, 10–11,
  \greek{συσχολάσας τοῖς ὑπὸ Εὐκλείδου μαθηταῖς ἐν Άλεξανδρείς\?
    πλεῖστον χρόνον, ὅθεν ἔσχε καὶ τὴν τοιαύτην ἕξιν οὐκ ἀμαθῆ}.}:
``he spent a very long time with the pupils of Euclid at Alexandria,
and it was thus that he acquired such a scientific habit of thought,''

It is in the same passage that Pappus makes a remark which might, to
an unwary reader, seem to throw some light on the personality of
Euclid.  He is speaking about Apollonius' preface to the first book of
his \emph{Conics}, where he says that Euclid had not completely worked
out the synthesis of the ``three- and four-line locus,'' which in fact
was not possible without some theorems first discovered by himself.
Pappus says on this\footnote{Pappus, \r7.\ pp.~676, 25–678,
  6. Hultsch, it is true, brackets the whole passage pp.~676, 25–678,
  15, but apparently on the ground of the diction only,}: ``Now
Euclid—regarding Aristaeus as deserving credit for the discoveries he
had already made in conics, and without anticipating him or wishing to
construct anew the same system (such was his scrupulous fairness and
his exemplary kindliness towards all who could advance mathematical
science to however small an extent), being moreover in no wise
contentious and, though exact, yet no braggart like the other
[Apollonius]—wrote so much about the locus as was possible by means of
the conics of Aristaeus, without claiming completeness for his
demonstrations.'' It is however evident, when the passage is examined
in its context, that Pappus is not following any tradition in giving
this account of Euclid: he was offended by the terms of Apollonius'
reference to Euclid, which seemed to him unjust, and he drew a fancy
picture of Euclid in order to show Apollonius in a relatively
unfavourable light.

Another story is told of Euclid which one would like to believe true.
According to Stobaeus\footnote{Stobaeus, \emph{l.c.}}, ``some one who
had begun to read geometry with Euclid, when he had learnt the first
theorem, asked Euclid, 'But what shall I get by learning these
things?' Euclid called his slave and said 'Give him threepence, since
he must make gain out of what he learns.'\,''

In the middle ages most translators and editors spoke of Euclid as
Euclid \emph{of Megara}. This description arose out of a confusion
between our Euclid and the philosopher Euclid of Megara who lived
about 400~\bc. The first trace of this confusion appears in Valerius
Maximus (in the time of Tiberius) who says\footnote{\r8.~12, exṭ~1.}
that Plato, on being appealed to for a solution of the problem of
doubling the cubical altar, sent the inquirers to ``Euclid the
geometer.'' There is no doubt about the reading, although an early
commentator on Valerius Maximus wanted to correct ``Eucliden'' into
``\emph{Eudoxum},'' and this correction is clearly right. But, if
Valerius Maximus took Euclid the geometer for a contemporary of Plato,
it could only be through confusing him with Euclid of Megara. The
first specific reference to Euclid as Euclid of Megara belongs to the
14th century, occurring in the \greek{ὑπομνηματισμοί} of Theodorus
Metochita (d.~1332) who speaks of ``Euclid of Megara, the Socratic
philosopher, contemporary of Plato,'' as the author of treatises on
plane and solid geometry, data, optics etc.: and a Paris
\textsc{ms}.\ of the 14th century has ``Euclidis philosophi Socratici
liberelementorum,'' The misunderstanding was general in the period
from Campanus' translation (Venice 1482) to those of Tartaglia (Venice
1565) and Candalla (Paris 1566). But one Constantinus Lascaris
(d.~about 1493) had already made the proper distinction by saying of
our Euclid that ``he was different from him of Megara of whom Laertius
wrote, and who wrote dialogues''\footnote{Letter to Fernandus Acuna,
  printed in Maurolycus, \emph{Historia Siciliae}, fol.~21~r.\ (see
  Heiberg, \emph{Euklid-Studien}, pp.~22–3, 25).}; and to Commandinus
belongs the credit of being the first translator\footnote{Preface to
  translation (Pisauri, 1572).} to put the matter beyond doubt: ``Let
us then free a number of people from the error by which they have been
induced to believe that our Euclid is the same as the philosopher of
Megara'' etc.

Another idea, that Euclid was born at Gela in Sicily, is due to the
same confusion, being based on Diogenes Laertius'
description\footnote{Diog.\ L.\ \r2.~106, p.~58 ed.\ Cobet.} of
the philosopher Euclid as being ``of Megara, or, according to some, of
Gela, as Alexander says in the \greek{Διαδοχαί}.''

In view of the poverty of Greek tradition on the subject even as early
as the time of Proclus (410–485~\ad), we must necessarily take
\emph{cum grano} the apparently circumstantial accounts of Euclid
given by Arabian authors; and indeed the origin of their stories can
be explained as the result (1)~of the Arabian tendency to romance, and
(2)~of misunderstandings.

We read\footnote{Casiri, \emph{Bibliotheca Arabico-Hispana
    Escurialensis}, \r1.~p.~339, Casiri's source is al-Qifṭī
  (d.~1248), the author of the \emph{Ta'rīkh al-Ḥukamā}, a collection
  of biographies of philosophers, mathematicians, astronomers etc.}
that ``Euclid, son of Naucrates, grandson of Zenarchus\footnote{The
  \emph{Fihrist} says ``son of Naucrates, the son of Berenice~(?)''
  (see Suter's translation in \emph{Abhandlungren zur
    Gesch.\ d.\ Math}. Heft, 1892. p.~16).}, called the author of
geometry, a philosopher of somewhat ancient date, a Greek by
nationality domiciled at Damascus, born at Tyre, most learned in the
science of geometry, published a most excellent and most useful work
entitled the foundation or elements of geometry, a subject in which no
more general treatise existed before among the Greeks: nay, there was
no one even of later date who did not walk in his footsteps and
frankly profess his doctrine. Hence also Greek, Roman and Arabian
geometers not a few, who undertook the task of illustrating this work,
published commentaries, scholia, and notes upon it, and made an
abridgment of the work itself. For this reason the Greek philosophers
used to post up on the doors of their schools the well-known notice:
`Let no one come to our school, who has not first learned the elements
of Euclid.'\,” The details at the beginning of this extract cannot be
derived from Greek sources, for even Proclus did not know anything
about Euclid's father, while it was not the Greek habit to record the
names of grandfathers, as the Arabians commonly did. Damascus and Tyre
were no doubt brought in to gratify a desire which the Arabians always
showed to connect famous Greeks in some way or other with the
East. Thus Naṣīraddīn, the translator of the \emph{Elements}, who was
of Ṭūs in Khurāsān, actually makes Euclid out to have been
``Thusinus'' also\footnote{The same predilection made the Arabs
  describe Pythagoras as a pupil of the wise Salomo, Hipparchus as the
  exponent of Chaldaean philosophy or as the Chaldaean, Archimedes as
  an Egyptian etc. (Ḥājī Khalfa, \emph{Lexicon Bibliographicum}, and
  Casiri).}. The readiness of the Arabians to run away with an idea is
illustrated by the last words of the extract. Everyone knows the story
of Plato's inscription over the porch of the Academy: ``let no one
unversed in geometry enter my doors''; the Arab turned geometry into
\emph{Euclid's} geometry, and told the story of Greek philosophers in
general and ''\emph{their} Academies.''

Equally remarkable are the Arabian accounts of the relation of Euclid
and Apollonius\footnote{The authorities for these statements quoted by
  Casiri and Ḥājī Khalfa are al-Kindī's tract \emph{de instituto libri
    Euclidis} (al-Kindī died about 873) and a commentary by Qāḍīzāde
  ar-Rūmī (d.\ about 1440) on a book called \emph{Ashkāl at-ta' sīs}
  (fundamental propositions) by Ashraf Shanuaddīn as-Samarqandī
  (\emph{c}.~1176) consisting of elucidations of 35 propositions
  selected from the first books of Euclid.  Naṣīraddīn likewise says
  that Euclid cut out two of 15 books of elements then existing and
  published the rest under his own name.  According to Qāḍīzāde the
  king heard that there was a celebrated geometer named Euclid at
  \emph{Tyre}: Naṣīraddīn says that he sent for Euclid of Ṭūs.}.
According to them the \emph{Elements} were originally written, not by
Euclid, but by a man whose name was Apollonius, a carpenter, who wrote
the work in 15 books or sections\footnote{So says the \emph{Fihrist},
  Suter (\emph{op.\ cit.}  p.~49) thinks that the author of the
  \emph{Fihrist} did not suppose Apollonius \emph{of Perga} to be the
  writer of the \emph{Elements}, as later Arabian authorities did, but
  that he distinguished another Apollonius whom he calls ``a
  carpenter.''  Suter's argument is based on the fact that the
  \emph{Fihrist}'s article on Apollonius (of Perga) says nothing of
  the \emph{Elements}; and that it gives the three great
  mathematicians, Euclid, Archimedes and Apollonius, in the correct
  chronological order.}.  In the course of time some of the work was
lost and the rest became disarranged, so that one of the kings at
Alexandria who desired to study geometry and to master this treatise
in particular first questioned about it certain learned men who
visited him and then sent for Euclid who was at that time famous as a
geometer, and asked him to revise and complete the work and reduce it
to order. Euclid then rewrote it in 13 books which were thereafter
known by his name. (According to another version Euclid composed the
13 books out of commentaries which he had published on two books of
Apollonius on conics and out of introductory matter added to the
doctrine of the five regular solids.) To the thirteen books were added
two more books, the work of others (though some attribute these also
to Euclid) which contain several things not mentioned by Apollonius.
According to another version Hypsicles, a pupil of Euclid at
Alexandria, offered to the king and published Books \book{xiv} and
\book{xv}, it being also stated that Hypsicles had ``discovered'' the
books, by which it appears to be suggested that Hypsicles had edited
them from materials left by Euclid.

We observe here the correct statement that Books \book{xiv} and
\book{xv} were not written by Euclid, but along with it the incorrect
information that Hypsicles, the author of Book \book{xiv}, wrote Book
\book{xv} also.

The whole of the fable about Apollonius having preceded Euclid and
having written the \emph{Elements} appears to have been evolved out of
the preface to Book \book{xiv} by Hypsicles, and in this way; the Book
must in early times have been attributed to Euclid, and the inference
based upon this assumption was left uncorrected afterwards when it was
recognised that Hypsicles was the author. The preface is worth
quoting:

``Basilides of Tyre, O Protarchus, when he came to Alexandria and met
my father, spent the greater part of his sojourn with him on account
of their common interest in mathematics. And once, when examining the
treatise written by Apollonius about the comparison between the
dodecahedron and the icosahedron inscribed in the same sphere,
(showing) what ratio they have to one another, they thought that
Apollonius had not expounded this matter properly, and accordingly
they emended the exposition, as I was able to learn from my
father. And I myself, later, fell in with another book published by
Apollonius, containing a demonstration relating to the subject, and I
was greatly interested in the investigation of the problem. The book
published by Apollonius is accessible to all—for it has a large
circulation, having apparently been carefully written out later—but I
decided to send you the comments which seem to me to be necessary, for
you will through your proficiency in mathematics in general and in
geometry in particular form an expert judgment on what I am about to
say, and you will lend a kindly ear to my disquisition for the sake of
your friendship to my father and your goodwill to me.''

The idea that Apollonius preceded Euclid must evidently have been
derived from the passage just quoted. It explains other things
besides. Basilides must have been confused with \greek{Βασιλεύς}, and
we have a probable explanation of the ``Alexandrian king,'' and of the
``learned men who visited'' Alexandria. It is possible also that in
the ``Tyrian'' of Hypsicles' preface we have the origin of the notion
that Euclid was born in Tyre. These inferences argue, no doubt, very
defective knowledge of Greek: but we could expect no better from those
who took the \emph{Organon} of Aristotle to be ``instrumentum musicum
pneumaticum,'' and who explained the name of Euclid, which they
variously pronounced as \emph{Uclides} or \emph{Icludes}, to be
compounded of \emph{Ucli} a key, and \emph{Dis} a measure, or, as some
say, geometry, so that \emph{Uclides} is equivalent to the \emph{key
  of geometry}!

Lastly the alternative version, given in brackets above, which says
that Euclid made the \emph{Elements} out of commentaries which he
wrote on two books of Apollonius on conics and prolegomena added to
the doctrine of the five solids, seems to have arisen, through a like
confusion, out of a later passage\footnote{Heiberg's Euclid,
  vol.~v.~p.~6.} in Hypsicles' Book \book{xiv}: ``And this is
expounded by Aristaeus in the book entitled 'Comparison of the five
figures,' and by Apollonius in the second edition of his comparison of
the dodecahedron with the icosahedron.'' The ``doctrine of the five
solids'' in the Arabic must be the ``Comparison of the five figures''
in the passage of Hypsicles, for nowhere else have we any information
about a work bearing this title, nor can the Arabians have had. The
reference to the \emph{two books} of Apollonius on \emph{conics} will
then be the result of mixing up the fact that Apollonius wrote a book
on conics with the \emph{second edition} of the other work mentioned
by Hypsicles.  We do not find elsewhere in Arabian authors any mention
of a commentary by Euclid on Apollonius and Aristaeus: so that the
story in the passage quoted is really no more than a variation of the
fable that the \emph{Elements} were the work of Apollonius.

\chapter{Euclid's Other Works}

In giving a list of the Euclidean treatises other than the \emph{Elements},
I shall be brief: for fuller accounts of them, or speculations with
regard to them, reference should be made to the standard histories of
mathematics\footnote{See, for example, Loria, \emph{Le scienze esatte nell' antica
  Grecia}, 1914, pp.~245–268; T. L. Heath, \emph{History of Greek
  Mathematics}, 1921, \r1.~pp.~421–446. Cf. Heiberg,
  \emph{Litterargeschichtliche Studien über Euklid}, pp.~36–153;
  \emph{Euclidis opera omnia}, ed.\ Heiberg and Menge,
  Vols.\ \r6.–\r8.}.

I will take first the works which are mentioned by Greek authors,

\section{The \emph{Pseudaria}}

I mention this first because Proclus refers to it in the general
remarks in praise of the \emph{Elements} which he gives immediately after
the mention of Euclid in his summary. He says\footnote{Proclus, p.~70, 1–18.}; ``But, inasmuch
as many things, while appearing to rest on truth and to follow from
scientific principles, really tend to lead one astray from the principles
and deceive the more superficial minds, he has handed down methods
for the discriminative understanding of these things as well, by the
use of which methods we shall be able to give beginners in this study
practice in the discovery of paralogisms, and to avoid being misled.
This treatise, by which he puts this machinery in our hands, he
entitled (the book) of Pseudaria, enumerating in order their various
kinds, exercising our intelligence in each case by theorems of all
sorts, setting the true side by side with the false, and combining
the refutation of error with practical illustration. This book then is
by way of cathartic and exercise, while the Elements contain the
irrefragable and complete guide to the actual scientific investigation
of the subjects of geometry.''

The book is considered to be irreparably lost. We may conclude
however from the connexion of it with the \emph{Elements} and the reference
to its usefulness for beginners that it did not go outside the domain
of elementary geometry\footnote{Heiberg points out that Alexander Aphrodisiensis appears to allude
  to the work in his commentary on Aristotle's \emph{Sophistici
  Elenchi} (fol.~25~\emph{b}): ``Not only those (\greek{ἔλεγχοι})
  which do not start from the principles of the science under which
  the problem is classed\dots but also those which do start from the
  proper principles of the science but in some respect admit a
  paralogism, e.g.\ the \emph{Pseudographemata} of Euclid.'' Tannery
  (\emph{Bull. des sciences math. et astr.} 2\tsup{e} Série,
  \textsc{vi}., 1882, 1\tsup{ère} Partie, p.~147) conjectures that it
  may be from this treatise that the same commentator got his
  Information about the quadratures of the circle by Antiphon and
  Bryson, to say nothing of the lunules of Hippocrates. I think
  however that there is is objection to this theory so far as regards
  Bryson; for Alexander distinctly says that Bryson's quadrature did
  \emph{not} start from the proper principles of geometry, but from
  some principles more general.}.

\section{The \emph{Data}}

The \emph{Data} (\greek{δεδομένα}) are included by Pappus in the
\emph{Treasury of Analysis} (\greek{τόπος ἀναλυόμενος}), and he
describes their contents\footnote{Pappus, \r7.\ p.~638.}.  They are
still concerned with elementary geometry, though forming part of the
introduction to higher analysis. Their form is that of propositions
proving that, if certain things in a figure are given (in magnitude,
in species, etc.), something else is given. The subject-matter is much
the same as that of the planimetrical books of the \emph{Elements}, to
which the \emph{Data} are often supplementary. We shall see this later
when we come to compare the propositions in the \emph{Elements} which
give us the means of solving the general quadratic equation with the
corresponding propositions of the \emph{Data} which give the
solution. The \emph{Data} may in fact be regarded as elementary
exercises in analysis.

It is not necessary to go more closely into the contents, as we
have the full Greek text and the commentary by Marinus newly
edited by Menge and therefore easily accessible\footnote{Vol. \r6. in
  the Teubner edition of \emph{Euclidis opera omnia} by Heiberg and
  Menge. A translation of the \emph{Data} is also included in Simson's
  Euclid (though naturally his text left much to be desired).}.

\section{The book \emph{On divisions (of figures)}}

This work (\greek{περὶ διαιρέσεων βιβλίον}) is mentioned by
  Proclus\footnote{Proclus, p.~69, 4.}.
In one place he is speaking of the conception or definition (\greek{λόγος})
of figure, and of the divisibility of a figure into others differing from
it in kind; and he adds: ``For the circle is divisible into parts unlike
in definition or notion (\greek{ἀνόμοια τῷ λόγῳ}), and so is each of
  the
rectilineal figures; this is in fact the business of the writer of the
Elements in his Divisions, where he divides given figures, in one case
into like figures, and in another into unlike\footnote{\ibid\ 144,
  22–26.}.'' ``Like'' and ``unlike''
here mean, not ``similar'' and ``dissimilar'' in the technical sense, but
``like'' or ``unlike \emph{in definition} or \emph{notion}''
  (\greek{λόγῳ}): thus to divide a
triangle into triangles would be to divide it into ``like'' figures, to
divide a triangle into a triangle and a quadrilateral would be to
divide it into ``unlike'' figures.

The treatise is lost in Greek but has been discovered in the
Arabic.  First John Dee discovered a treatise \emph{De divisionibus}
  by one
Muhammad Bagdadinus\footnote{Stetnschneider places him in the 10th
  c. H. Suter (\emph{Bibliotheca Mathematica}, \r4\tsub{3}, 1903,
  pp.~24, 27) identifies him with Abū (Bekr) Muḥ. b. `Abdalbāqī
  al-Baġdādī, Qādī (Judge) of Māristān (\emph{circa} 1070–1141), to
  whom he also attributes the \emph{Liber judei \emph{(?judicis)}
  super decimum Euclidis} translated by Gherard of Cremona.} and
  handed over a copy of it (in Latin) in
1563 to Commandinus. who published it, in Dee's name and his own,
in 1570\footnote{\emph{De superficierum divisionibus liber Machometo
  Bagdadino adscriptus, nunc primum Ioannis Dee Londinensis et
  Federici Commandini Urbinatis opera in lucem editus}, Pisauri, 1570,
  afterwards included in Gregory's Euclid (Oxford, 1703).}.  Dee did
  not himself translate the tract from the Arabic; he
found it in Latin in a \textsc{ms.}\ which was then in his own
  possession but
was about 20 years afterwards stolen or destroyed in an attack by a
mob on his house at Mortlake\footnote{R.~C. Archibald, \emph{Euclid's
  Book on the Division of Figures with a restoration based on
  Woepcke's text and on the Practica geometriae of Leonardo Pisano},
  Cambridge, 1915, pp.~4–9.}. Dee, in his preface addressed to
Commandinus, says nothing of his having \emph{translated} the book, but
only remarks that the very illegible \textsc{ms.}\ had caused him much trouble
and (in a later passage) speaks of ``the actual, very ancient, copy
  from
which I \emph{wrote out}…'' (in ipso unde descripsi vetustissimo
  exemplari).
The Latin translation of this tract from the Arabic was probably made
by Gherard of Cremona (1114–1187), among the list of whose numerous
translations a ``liber divisionum'' occurs.  The Arabic original
  cannot
have been a direct translation from Euclid, and probably was not even
a direct adaptation of it; it contains mistakes and unmathematical
expressions, and moreover does not contain the propositions about
the division of a circle alluded to by Proclus. Hence it can scarcely
have contained more than a fragment of Euclid's work.

But Woepcke found in a \textsc{ms.}\ at Paris a treatise in Arabic on
  the
division of figures, which he translated and published in
  1851\footnote{\emph{Journal Asiatique}, 1851, p.~233~sqq.}. It is
expressly attributed to Euclid in the \textsc{ms.}\ and corresponds to
  the
description of it by Proclus. Generally speaking, the divisions are
divisions into figures of the same kind as the original figures,
  e.g.\ of
triangles into triangles; but there are also divisions into ``unlike ``
figures, e.g.\ that of a triangle by a straight line parallel to the base.
The missing propositions about the division of a circle are also here:
``to divide into two equal parts a given figure bounded by an arc
of a circle and two straight lines including a given angle'' and ``to
draw in a given circle two parallel straight lines cutting off a certain
part of the circle.'' Unfortunately the proofs are given of only four
propositions (including the two last mentioned) out of~36, because
the Arabic translator found them too easy and omitted them. To
illustrate the character of the problems dealt with I need only take
one more example: ``To cut off a certain fraction from a
  (parallel-)trapezium by a straight line which passes through a given
  point lying
inside or outside the trapezium but so that a straight line can be
drawn through it cutting both the parallel sides of the trapezium.''
The genuineness of the treatise edited by Woepcke is attested by the
facts that the four proofs which remain are elegant and depend on
propositions in the \emph{Elements}, and that there is a lemma with a
  true
Greek ring: ``to apply to a straight line a rectangle equal to the
rectangle contained by \emph{$AB$, $AC$ and deficient by a square}.''
  Moreover
the treatise is no fragment, but finishes with the words ``end of the
treatise,'' and is a well-ordered and compact whole. Hence we may
safely conclude that Woepcke's is not only Euclid's own work but
the whole of it. A restoration of the work, with proofs, was attempted
by Ofterdinger\footnote{L. F. Ofterdinger, \emph{Beiträge zur
  Wiederherstellung der Schrift des Euklides über die Theilung der
  Figuren}, Ulm, 1853.}, who however does not give Woepcke's
  props. 30, 31, 34, 35, 36.  We have now a satisfactory restoration,
  with ample notes and an introduction, by R.~C. Archibald, who used
  for the purpose Woepcke's text and a section of Leonardo of Pisa's
  \emph{Practica geometriae} (1220)\footnote{There is a remarkable
  similarity between the propositions of Woepcke's text and those of
  Leonardo, suggesting that Leonardo may have had before him a
  translation (perhaps by Gherard of Cremona) of the Arabic tract.}.

\section{The \emph{Porisms}}

It is not possible to give in this place any account of the
controversies about the contents and significance of the three lost books
of Porisms, or of the important attempts by Robert Simson and
Chasles to restore the work. These may be said to form a whole
literature, references to which will be found most abundantly given
by Heiberg and Loria, the former of whom has treated the subject
from the philological point of view, most exhaustively, while the
latter, founding himself generally on Heiberg, has added useful
details, from the mathematical side, relating to the attempted
restorations, etc.\footnote{Heiberg, \emph{Euklid-Studien}, pp.~56–79,
  and Loria, \emph{op.\ cit.}, pp.~253–265.} It must suffice here to
  give an extract from the only
original source of information about the nature and contents of the
\emph{Porisms}, namely Pappus\footnote{Pappus, ed. Hultsch,
  \r7.\ pp.~648–660. I put in square brackets the words bracketed by
  Hultsch.}. In his general preface about the books
composing the \emph{Treasury of Analysis} (\greek{τόπος ἀναλυόμενος})
  he says:

``After the Tangencies (of Apollonius) come, in three books, the
  Porisms of Euclid, [in the view of many] a collection most
  ingeniously devised for the analysis of the more weighty problems,
  [and] although nature presents an unlimited number of such
  porisms\footnote{I adopt Heiberg's reading of a comma here instead
    of a full stop.}, [they have added nothing to what was written
    originally by Euclid, except that some before my time have shown
    their want of taste by adding to a few (of the propositions)
    second proofs, each (proposition) admitting of a definite number
    of demonstrations, as we have shown, and Euclid having given one
    for each, namely that which is the most lucid. These porisms
    embody a theory subtle, natural, necessary, and of considerable
    generality, which is fascinating to those who can see and produce
    results].

``Now all the varieties of porisms belong, neither to theorems nor
problems, but to a species occupying a sort of intermediate position
[so that their enunciations can be formed like those of either theorems
or problems], the result being that, of the great number of geometers,
some regarded them as of the class of theorems, and others of pro-
blems, looking only to the form of the proposition. But that the
ancients knew better the difference between these three things is
clear from the definitions. For they said that a theorem is that
which is proposed with a view to the demonstration of the very
thing proposed, a problem that which is thrown out with a view to
the construction of the very thing proposed, and a porism that which
is proposed with a view to the producing of the very thing proposed.
[But this definition of the porism was changed by the more recent
writers who could not produce everything, but used these elements
and proved only the fact that that which is sought really exists, but
did not produce it\footnote{\sidefig{introI_1}
Heiberg points out that Props.~5–9 of Archimedes' treatise \emph{On
  Spirals} are porisms in this sense. To take Prop.~5 as an example,
  $DBF$ is a tangent to a circle with centre $K$. It is then possible,
  says Archimedes, to draw a straight line $KHF$, meeting the
  circumference in $H$ and the tangent in~$F$, such that
  \[ FH : HK < (\arc BH) : c \]
  where $c$ is the circumference of \emph{any} circle. To prove this
  he assumes the following construction. $E$ being any straight line
  greater than $c$, he says: let $KG$ be parallel to $DF$, ``and let
  the line $GH$ equal to $E$ be placed \emph{verging} to the point
  $B$.''  Archimedes must of course nave known how to effect this
  construction, which requires conics. But that it is \emph{possible}
  requires very little argument, for if we draw any straight line
  $BHG$ meeting the circle in $H$ and $KG$ in $G$, it is obvious that
  as $G$ moves away from~$C$, $HG$ becomes greater and greater and may
  be made as great as we please. The ``later writers ``would no doubt
  have contented themselves with this consideration without actually
  \emph{constructing} $HG$.} and were accordingly confuted by the
  definition
and the whole doctrine. They based their definition on an incidental
characteristic, thus: A porism is that which falls short of a
locus-theorem in respect of its hypothesis\footnote{As Heiberg says,
  this translation is made certain by a preceding passage of Pappus
  (p.~648, 1–3) where he compares two enunciations, the latter of
  which ``falls short of the former in \emph{hypothesis} but goes
  beyond it in \emph{requirement}.''  E.g.\ the first enunciation
  requiring us, given three circles, to draw a circle touching all
  three, the second may require us, given only two circles (one less
  datum), to draw a circle touching them and \emph{of a given size}
  (an extra requirement).}. Of this kind of porisms loci
are a species, and they abound in the Treasury of Analysis; but
this species has been collected, named and handed down separately
from the porisms, because it is more widely diffused than the other
species]. But it has further become characteristic of porisms that,
owing to their complication, the enunciations are put in a contracted
form, much being by usage left to be understood; so that many
geometers understand them only in a partial way and are ignorant of
the more essential features of their contents,

``[Now to comprehend a number of propositions in one enunciation
is by no means easy in these porisms, because Euclid himself has not
in fact given many of each species, but chosen, for examples, one or a
few out of a great multitude\footnote{I translate Heiberg's reading
  with a full stop here followed by \greek{πρὸς ἀρχῇ δὲ ὄμως}
  [\greek{πρὸς ἀρχὴν (δεδομένον)} Hultsch] \greek{τοῦ πρώτου
  βιβλίου}\dots.}. But at the beginning of the first book
he has given some propositions, to the number of ten, of one species,
namely that more fruitful species consisting of loci.]  Consequently,
finding that these admitted of being comprehended in one enunciation,
we have set it out thus:
\begin{quote}
If, in a system of four straight lines\footnote{The four straight
  lines are described in the text as (the sides) \greek{ὑπτίου ἤ
  παρυπτίου} i.e. sides of two' sorts of quadrilaterals which Simson
  tries to explain (see p.~120 of the \emph{Index Graecitatis} of
  Hultsch's edition of Pappus).} which cut each other
two and two, three points on one straight line be given while the
rest except one lie on different straight lines given in position,
the remaining point also will lie on a straight line given in
position\footnote{In other words (Chasles, p.~23; Loria, p.~256), if a
  triangle be so deformed that each of its sides turns about one of
  three points in a straight line, and two of its vertices lie on two
  straight lines given in position, the third vertex will also lie on
  a straight line.}.
\end{quote}

``This has only been enunciated of four straight lines, of which not
more than two pass through the same point, but it is not known (to
most people) that it is true of any assigned number of straight lines
if enunciated thus:
\begin{quote}
If any number of straight lines cut one another, not more
than two (passing) through the same point, and all the points
(of intersection situated) on one of them be given, and if each of
those which are on another (of them) lie on a straight line given
in position—
\end{quote}
or still more generally thus:
\begin{quote}
if any number of straight lines cut one another, not more than
two (passing) through the same point, and all the points (of
intersection situated) on one of them be given, while of the other
points of intersection in multitude equal to a triangular number
a number corresponding to the side of this triangular number lie
respectively on straight lines given in position, provided that of
these latter points no three are at the angular points of a triangle
(\emph{sc.}\ having for sides three of the given straight lines)—each
  of the
remaining points will lie on a straight line given in
  position\footnote{Loria (p.~256, \emph{n}.~3) gives the meaning of
  this as follows, pointing out that Simson was the discoverer of it:
  ``If a complete $n$-lateral be deformed so that its sides
  respectively turn about $n$ points on a straight line, and $n - 1$
  of its $n(n - 1)/2$ vertices move on as many straight lines, the
  other $(n-1_(n-2)/2$ of its vertices likewise move on as many
  straight lines; but it is necessary that it should be impossible to
  form with the $(n - 1)$ vertices any triangle having for sides the
  sides of the polygon.''}.
\end{quote}
``It is probable that the writer of the Elements was not unaware
of this but that he only set out the principle; and he seems, in the
case of all the porisms, to have laid down the principles and the
seed only [of many important things], the kinds of which should be
distinguished according to the differences, not of their hypotheses, but
of the results and the things sought [All the hypotheses are different
from one another because they are entirely special, but each of the
results and things sought, being one and the same, follow from many
different hypotheses.]

``We must then in the first book distinguish the following kinds of
things sought:
\begin{quote}

``At the beginning of the book\footnote{Reading, with Heiberg,
  \greek{τοῦ Βιβλίου} [\greek{τοῦ ζ’} Hultsch].} is this proposition:

I. `\emph{If from two given points straight tines be drawn meeting
on a straight line given in position, and one cut off from a straight
line given in position (a segment measured) to a given point on it,
the other will also cut off from another (straight line a segment)
liaving to the first a given ratio.}'

``Following on this (we have to prove)

II. that such and such a point lies on a straight line given
in position;

III. that the ratio of such and such a pair of straight lines
is given;''

etc. etc, (up to \r39.).
\end{quote}

``The three books of the porisms contain 38 lemmas; of the
theorems themselves there are 171.''

Pappus further gives lemmas to the \emph{Porisms} (pp.~866–918, ed.\
Hultsch).

With Pappus' account of Porisms must be compared the passages
of Proclus on the same subject.  Proclus distinguishes two senses in
which the word \greek{pórisma} is used. The first is that of
  \emph{corollary} where
something appears as an incidental result of a proposition, obtained
without trouble or special seeking, a sort of bonus which the
investigation has presented us with\footnote{Proclus, pp.~212, 14;
  301, 22.}.  The other sense is that of Euclid's
\emph{Porisms}\footnote{\ibid~pp.212, 12. ``The term porism is used of
  certain problems, like the \emph{Porisms} written by Euclid.''}, In
  this sense\footnote{\ibid~pp.~301, 25 sqq.} ``\emph{porism} is the
  name given to things which
are sought, but need some finding and are neither pure bringing into
existence nor simple theoretic argument. For (to prove) that the
angles at the base of isosceles triangles are equal is a matter of
theoretic argument, and it is with reference to things existing that
such knowledge is (obtained).  But to bisect an angle, to construct a
triangle, to cut off, or to place—all these things demand the making
of something; and to find the centre of a given circle, or to find the
greatest common measure of two given commensurable magnitudes,
or the like, is in some sort between theorems and problems.  For in
these cases there is no bringing into existence of the things sought,
but finding of them, nor is the procedure purely theoretic.  For it is
necessary to bring that which is sought into view and exhibit it to
the eye. Such are the porisms which Euclid wrote, and arranged in
three books of Porisms.''

Proclus' definition thus agrees well enough with the first, ``older,''
definition of Pappus. A porism occupies a place between a theorem
and a problem: it deals with something already \emph{existing}, as a theorem
does, but has to \emph{find} it (e.g.\ the centre of a circle), and,
  as a certain
operation is therefore necessary, it partakes to that extent of the
nature of a problem, which requires us to construct or produce
something not previously existing. Thus, besides \prop{3}{1} of the
  \emph{Elements}
and \prop{10}{3, 4} mentioned by Proclus, the following propositions
  are real porisms: \prop{3}{25}, \prop{6}{11–13}, \prop{7}{33, 34,
  36, 39}, \prop{8}{2, 4}, \prop{10}{10}, \prop{13}{18}. Similarly in
  Archimedes \emph{On the Sphere and Cylinder} \prop{1}{2–6}
might be called porisms.

The enunciation given by Pappus as comprehending ten of Euclid's
propositions may not reproduce the \emph{form} of Euclid's enunciations;
but, comparing the result to be proved, that certain points lie on
straight lines given in position, with the class indicated by
  \r2.\ above,
where the question is of such and such a point lying on a straight line
given in position, and with other classes, e.g.\ (\r5.) that such and such a
line is given in position, (\r6.) that such and such a line verges to
  a given
point, (\r23.) that there exists a given point such that straight
  lines
drawn from it to such and such (circles) will contain a triangle given
in species, we may conclude that a usual form of a porism was ``to
prove that it is possible to find a point with such and such a property''
or ``a straight line on which lie all the points satisfying given
conditions'' etc.

Simson defined a porism thus: ``Porisma est propositio in qua
proponitur demonstrare rem aliquant, vel plures datas esse, cui, vel
quibus, ut et cuilibet ex rebus innumeris, non quidem datis, sed quae
ad ea quae data sunt eandem habent relationem, convenire ostendendum
est affectionem quandam communem in propositione
descriptam\footnote{This was thus expressed by Chasles: ``Le porisme
  est une proposition dans laquelle on demande de démontrer qu'une
  chose ou plusieurs choses sont \emph{données}, qui, ainsi que l'une
  quelconque d'une infinité d'autres choses non données, mais dont
  chacune est avec des choses données dans une même relation, ont une
  certaine propriété commune, décrite dans la proposition.''}.''

From the above it is easy to understand Pappus' statement that
\emph{loci} constitute a large class of porisms. A \emph{locus} is
well defined by Simson thus: ``Locus est propositio in qua propositum
est datam esse demonstrare, vel invenire lineam aut superficiem cuius
quodlibet punctum, vel superficiem in qua quaelibet linea data lege
descripta, communem quandam habet proprietatem in propositione
descriptam.''  Heiberg cites an excellent instance of a \emph{locus}
  which is a \emph{porism}, namely the following proposition quoted by
  Eutocius\footnote{Commentary on Apollonius' \emph{Conics}
    (vol.~\r2.~p.~180, ed.\ Heiberg).} from the \emph{Plane Loci} of
  Apollonius:

``Given two points in a plane, and a ratio between unequal straight
  lines, it is possible to draw, in the plane, a circle such that the
  straight lines drawn from the given points to meet on the
  circumference of the circle have (to one another) a ratio the same
  as the given ratio,''

A difficult point, however, arises on the passage of Pappus, which
says that a porism is ``that which, in respect of its hypothesis,
falls short of a locus-theorem ``(\greek{τοπικοῦ θεωρήματος}). Heiberg
explains it by comparing the porism from Apollonius' \emph{Plane Loci}
just given with Pappus' enunciation of the same thing, to the effect
that, if from two given points two straight lines be drawn meeting in
a point, and these straight lines have to one another a given ratio,
the point will lie on either a straight line or a circumference of a
circle given in position.  Heiberg observes that in this latter
enunciation something is taken into the hypothesis which was not in
the hypothesis of the enunciation of the porism, viz.\ ``that the
ratio of the straight lines is the same.''  I confess this does not
seem to me satisfactory: for there is no real difference between the
enunciations, and the supposed difference in hypothesis is very like
playing with words. Chasles says: ``\emph{Ce qui constitue le porisme
  est ce qui manque à l'hypothèse d'un théorème local} (en d'autres
termes, le porisme est inférieur, par l'hypothèse, au théorème local;
e'est-á-dire que quand quelques parties d'une proposition locale n'ont
pas dans l'énoncé la détermination qui ieur est propre, cette
proposition cesse d'être regardée comme un théorème et devient un
porisme).'' But the subject still seems to require further
elucidation.

While there is so much that is obscure, it seems certain (1)~that the
\emph{Porisms} were distinctly part of higher geometry and not of
elementary geometry, (2)~that they contained propositions belonging to
the modern theory of transversals and to projective geometry. It
should be remembered too that it was in the course of his researches
on this subject that Chasles was led to the idea of \emph{anharmonic
  ratios}.

Lastly, allusion should be made to the theory of
Zeuthen\footnote{\emph{Die Lehre von den Kegelschnitten im Altertum},
  chapter~\r8.}  on the subject of the porisms. He observes that the
only porism of which Pappus gives the complete enunciation, ``If from
two given points straight lines be drawn meeting on a straight line
given in position, and one cut off from a straight line given in
position (a segment measured) towards a given point on it, the other
will also cut off from another (straight line a segment) bearing to
the first a given ratio,'' is also true if there be substituted for
the first given straight line a conic regarded as the ``locus with
respect to four lines,'' and that this extended porism can be used for
completing Apollonius' exposition of that locus. Zeuthen concludes
that the \emph{Porisms} were in part by-products of the theory of
conics and in part auxiliary means for the study of conics, and that
Euclid called them by the same name as that applied to corollaries
because they were corollaries with respect to conics. But there
appears to be no evidence to confirm this conjecture.

\section{The \emph{Surface-loci} (\greek{τόποι πρὸς ἐπιφανείᾳ})}

The two books on this subject are mentioned by Pappus as part of the
\emph{Treasury of Analysis}\footnote{Pappus, \r7.\ p.~636.}.  As the
other works in the list which were on plane subjects dealt only with
straight lines, circles, and conic sections, it is \emph{a priori}
likely that among the loci in this treatise (loci which are surfaces)
were included such loci as were cones, cylinders and spheres. Beyond
this all is conjecture based on two lemmas given by Pappus in
connexion with the treatise.

(1)~The first of these lemmas\footnote{\ibid~\r7.~p.~1004.} and the
figure attached to it are not satisfactory as they stand, but a
possible restoration is indicated by Tannery\footnote{\emph{Bulletin
    des sciences math.\ et astron.}, 2\tsup{e} Série, \r6.~149.}.  If
the latter is right, it suggests that one of the loci contained all
the points on the elliptical parallel sections of a cylinder and was
therefore an oblique circular cylinder.  Other assumptions with regard
to the conditions to which the lines in the figure may be subject
would suggest that other loci dealt with were cones regarded as
containing all points on particular elliptical parallel sections of
the cones\footnote{Further particulars will be found in \emph{The
    Works ef Archimedes}, pp.~lxii–lxiv, and in Zeuthen, \emph{Die
    Lehre von Kegelschnitten}, p.~425~sqq.}.

(2)~In the second lemma Pappus states and gives a complete proof of
the focus-and-directrix property of a conic, viz.\ that \emph{the
  locus of a point whose distance from a given point is in a given
  ratio to its distance from a fixed line is a conic section, which is
  an ellipse, a parabola or a hyperbola according as the given ratio
  is less than, equal to, or greater than unity}\footnote{Pappus,
  \r7.~pp.~1006–1014, and Hultsch's Appendix, pp.~1270–3.}.  Two
conjectures are possible as to the application of this theorem in
Euclid's \emph{Surface-loci}, (\emph{a})~It may have been used to
prove that the locus of a point whose distance from a given straight
line is in a given ratio to its distance from a given plane is a
certain cone.  (\emph{b})~It may have been used to prove that the
locus of a point whose distance from a given point is in a given ratio
to its distance from a given plane is the surface formed by the
revolution of a conic about its major or conjugate axis\footnote{For
  further details see \emph{The Works of Archimedes}, pp.~lxiv, lxv,
  and Zeuthen, \emph{l.~c.}}.  Thus Chasles may have been correct in
his conjecture that the \emph{Surface-loci} dealt with surfaces of
revolution of the second degree and sections of the
same\footnote{\emph{Aperçu historique}, pp.~273–44.}.

\section{The \emph{Conics}}

Pappus says of this lost work: ``The four books of Euclid's Conics
were completed by Apollonius, who added four more and gave us eight
books of Conics\footnote{Pappus, \r7.~p.~672.}.'' It is probable that
Euclid's work was lost even by Pappus' time, for he goes on to speak
of ``Aristaeus, who wrote the \emph{still extant} five books of Solid
Loci connected with the conics.''  Speaking of the relation of
Euclid's work to that of Aristaeus on conics regarded as loci, Pappus
says in a later passage (bracketed however by Hultsch) that Euclid,
regarding Aristaeus as deserving credit for the discoveries he had
already made in conics, did not (try to) anticipate him or construct
anew the same system. We may no doubt conclude that the book by
Aristaeus on solid loci preceded Euclid's on conics and was, at least
in point of originality, more important.  Though both treatises dealt
with the same subject-matter, the object and the point of view were
different; had they been the same, Euclid could scarcely have
refrained, as Pappus says he did, from attempting to improve upon the
earlier treatise.  No doubt Euclid wrote on the general theory of
conics as Apollonius did, but confined himself to those properties
which were necessary for the analysis of the \emph{Solid Loci} of
Aristaeus. The \emph{Conics} of Euclid were evidently superseded by
the treatise of Apollonius.

As regards the contents of Euclid's \emph{Conics}, the most important
source of our information is Archimedes, who frequently refers to
propositions in conics as well known and not needing proof, adding in
three cases that they are proved in the ``elements of conics'' or in
``the conics,'' which expressions must clearly refer to the works of
Aristaeus and Euclid\footnote{For details of these propositions see my
  \emph{Apollonius of Perga}, pp.~xxxv, xxxvi.}.

Euclid still used the old names for the conics (sections of a
right-angled, acute-angled, or obtuse-angled cone), but he was aware
that an ellipse could be obtained by cutting a cone in any manner by a
plane not parallel to the base (assuming the section to lie wholly
between the apex of the cone and its base) and also by cutting a
cylinder. This is expressly stated in a passage from the
\emph{Phaenomena} of Euclid about to be
mentioned\footnote{\emph{Phaenomena}, ed.\ Menge, p.~6: ``if a cone or
  a cylinder be cut by a plane not parallel to the base, the section
  is a section of an acute-angled cone, which is like a shield
  (\greek{θυρεός}).''}.

\section{The \emph{Phaenomena}}

This is an astronomical work and is still extant. A much interpolated
version appears in Gregory's Euclid.  An earlier and better recension
is however contained in the \textsc{ms.}\ Vindobonensis
philos.\ Gr.~103, though the end of the treatise, from the middle of
prop.~16 to the last (18), is missing. The book, now edited by
Menge\footnote{\emph{Euclidis opera omnia}, vol.~\r8., 1916,
  pp.~2–156.} , consists of propositions in \emph{spheric} geometry.
Euclid based it on Autolycus' work \greek{περὶ κινουμένης σφαίρας},
but also, evidently, on an earlier text-book of \emph{Sphaerica} of
exclusively mathematical content.  It has been conjectured that the
latter textbook may have been due to Eudoxus\footnote{Heiberg,
  \emph{Euklid-Studien}, p.~46; Hultsch, \emph{Autolycus}, p.~xii;
  A. A. Björnbo, \emph{Studien über Menelaos' Sphärk}
  (\emph{Abhandlungen zur Geschichte der mathematischen
    Wissenschaften}, \r14.~1902), p.~56~sqq.}.

\section{The \emph{Optics}}

This book needs no description, as it has been edited by Heiberg
recently\footnote{\emph{Euclidis opera omnia}, vol.~\r7. (1895).},
both in its genuine form and in the recension by Theon.  The
\emph{Catoptrica} published by Heiberg in the same volume is not
genuine, and Heiberg suspects that in its present form it may be
Theon's. It is not even certain that Euclid wrote \emph{Catoptrica} at
all, as Proclus may easily have had Theon's work before him and
inadvertently assigned it to Euclid\footnote{Heiberg, Euclid's
  \emph{Optics}, \emph{etc}. p.~l.}.

\section{}

Besides the above-mentioned works, Euclid is said to have written the
\emph{Elements of Music}\footnote{Proclus, p.~69, 3.} (\greek{αἱ κατὰ
  μουσικὴν στοιχειώσεις}). Two treatises are attributed to Euclid in
our \textsc{mss.}\ of the \emph{Musici}, the \greek{κατατομὴ κανόνος},
\emph{Sectio canonis} (the theory of the intervals), and the
\greek{εἰσαγωγὴ ἁπμονική} (introduction to harmony)\footnote{Both
  treatises edited by Jan in \emph{Musici Scriptores Graeci}, 1895,
  pp.~113–166, 167–207;, and by Menge in \emph{Euclidis opera omnia},
  vol.~\r8., 1916, pp.~157–183, 185–223.}. The first, resting on the
Pythagorean theory of music, is mathematical, and the style and
diction as well as the form of the propositions mostly agree with what
we find in the \emph{Elements}. Jan thought it genuine, especially as
almost the whole of the treatise (except the preface) is quoted
\emph{in extenso}, and Euclid is twice mentioned by name, in the
commentary on Ptolemy's \emph{Harmonica} published by Wallis and
attributed by him to Porphyry.  Tannery was of the opposite
opinion\footnote{\emph{Comptes rendus de l'Acad.\ des inscriptions et
    belles-lettres}, Paris, 1904, pp.~439—445.  Cf.\ \emph{Bibliotheca
    Mathematica}, \r6\tsub{3}, 1905–6, p.~225, note~1.}. The latest
editor, Menge, suggests that it may be a redaction by a less competent
hand from the genuine Euclidean \emph{Elements of Music}. The second
treatise is not Euclid's, but was written by Cleonides, a pupil of
Aristoxenus\footnote{Heiberg, \emph{Euklid-Studien}, pp.~52–55; Jan,
  \emph{Musici Scriptores Graeci}, pp.~169–174.}.

Lastly, it is worth while to give the Arabians' list of Euclid's
works.  I take this from Suter's translation of the list of
philosophers and mathematicians in the \emph{Fihrist}, the oldest
authority of the kind that we possess\footnote{H. Suter, \emph{Das
    Mathematiker-Verzeichnis im Fihrist} in \emph{Abhandlungen zur
    Geschichte der Mathematik}, \r6., 1892, pp.~1–87 (see especially
  p.~17).  Cf.\ Casiri, 1.~339, 340, and Gartz, \emph{De interpretibus
    et explanatoribus}, 1823, pp.~4,~5.}. ``To the writings of Euclid
belong further [in addition to the \emph{Elements}]: the book of
Phaenomena; the book of Given Magnitudes [\emph{Data}]; the book of
Tones, known under the name of Music, not genuine; the book of
Division, emended by Thābit; the book of Utilisations or Applications
[\emph{Porisms}], not genuine; the book of the Canon; the book of the
Heavy and Light; the book of Synthesis, not genuine; and the book of
Analysis, not genuine.''

It is to be observed that the Arabs already regarded the book of Tones
(by which must be meant the \greek{εἰσαγωγὴ ἁρμονική}) as spurious.
The book of Division is evidently the book on \emph{Divisions (of
  figures)}.  The next book is described by Casiri as ``liber de
utilitate suppositus.''  Suter gives reason for believing the
\emph{Porisms} to be meant\footnote{Suter, \emph{op.~cit.}\ pp.~49,
  50, Wenrich translated the word as ``utilia.'' Suter says that the
  nearest meaning of the Arabic word as of ``porism'' is \emph{use},
  \emph{gain} (Nutzen, Gewinn), while a further meaning is
  explanation, observation, addition: a gain arising out of what has
  preceded (cf.\ Proclus' definition of the porism in the sense of a
  corollary).}, but does not apparently offer any explanation of why
the work is supposed to be spurious. The book of the Canon is clearly
the \greek{κατατομὴ κανόνος}.  The book on ``the Heavy and Light'' is
apparently the tract \emph{De levi et ponderoso}, included in the
Basel Latin translation of 1537, and in Gregory's edition. The
fragment, however, cannot safely be attributed to Euclid, for (1)~we
have nowhere any mention of his having written on mechanics, (2)~it
contains the notion of specific gravity in a form so clear that it
could hardly be attributed to anyone earlier than
Archimedes\footnote{Heiberg, \emph{Euklid-Studien}, pp.~9,~10.}. Suter
thinks\footnote{Suter, \emph{op.~cit.}\ p.~50.} that the works on
Analysis and Synthesis (said to be spurious in the extract) may be
further developments of the \emph{Data} or \emph{Porisms}, or may be
the interpolated proofs of \emph{Eucl.}\ \prop{13}{1–5}, divided into
\emph{analysis} and \emph{synthesis}, as to which see the notes on
those propositions.

\chapter{Greek Commentators on the \emph{Elements} other
than Proclus}

That there was no lack of commentaries on the \emph{Elements} before
the time of Proctus is evident from the terms in which Proclus refers
to them; and he leaves as in equally little doubt as to the value
which, in his opinion, the generality of them possessed. Thus he says
in one place (at the end of his second prologue)\footnote{Proclus,
  p.~84,~8.}:

``Before making a beginning with the investigation of details, I warn
those who may read me not to expect from me the things which have been
dinned into our ears \emph{ad nauseam} (\greek{διατεθρύληται}) by
those who have preceded me, viz.\ lemmas, cases, and so forth. For I
am surfeited with these things and shall give little attention to
them.  But I shall direct my remarks principally to the points which
require deeper study and contribute to the sum of philosophy, therein
emulating the Pythagoreans who even had this common phrase for what I
mean `a figure and a platform, but not a figure and
sixpence\footnote{i.e.\ we reach a certain height, use the platform so
  attained as a base on which to build another stage, then use that as
  a base and so on.}.'\,”

In another place\footnote{Proclus, p.~200, 10.} he says: ``Let us now
turn to the elucidation of the things proved by the writer of the
Elements, selecting the more subtle of the comments made on them by
the ancient writers, while cutting down their interminable
diffuseness, giving the things which are more systematic and follow
scientific methods, attaching more importance to the working-out of
the real subject-matter than to the variety of cases and lemmas to
which we see recent writers devoting themselves for the most part.''

At the end of his commentary on Eucl.~\r1.\ Prockis
remarks\footnote{\ibid~p.~432, 15.} that the commentaries then in
vogue were full of all sorts of confusion, and contained no account of
\emph{causes}, no dialectical discrimination, and no philosophic
thought.

These passages and two others in which Proclus refers to ``the
commentators\footnote{\ibid~p.~389, 11; p.~328, 16.}'' suggest that
these commentators were numerous.  He does not however give many
names; and no doubt the only important commentaries were those of
Heron, Porphyry, and Pappus.

\section{Heron}

Proclus alludes to Heron twice as Heron
\emph{mechanicus}\footnote{Proclus, p.~305, 24; p.~346, 13.}, in
another place\footnote{\ibid~p.~41, 10.} he associates him with
Ctesibius, and in the three other passages\footnote{\ibid~p.~196, 16;
  p.~323, 7: p.~429, 13.} where Heron is mentioned there is no reason
to doubt that the same person is meant, namely Heron of
Alexandria. The date of Heron is still a vexed question. In the early
stages of the controversy much was made of the supposed relation of
Heron to Ctesibius. The best \textsc{ms.}\ of Heron's
\emph{Belopoeica} has the heading \greek{Ἥρωνος Κτησιβίου Βελοποιϊκά},
and an anonymous Byzantine writer of the tenth century, evidently
basing himself on this title, speaks of Ctesibius as Heron's
\greek{καθηγητής}, ``master'' or ``teacher.'' We know of two men of
the name of Ctesibius. One was a barber who lived in the time of
Ptolemy Euergetes~II, i.e.\ Ptolemy~VII, called Physcon (died
117~\bc)), and who is said to have made an improved
water-organ\footnote{Athenaeus, \emph{Deipno-Soph.} iv., c.~75, p.~174
  \emph{b}–\emph{c}.}.  The other was a mechanician mentioned by
Athenaeus as having made an elegant drinking-horn in the time of
Ptolemy Philadelphus (285–247~\bc)\footnote{\ibid~xi., c.~97, p.~497
  \emph{b}–\emph{c}.}.  Martin\footnote{Martin, \emph{Recherches sur
    la vie et les ouvrages d' Héron d' Alexandrie}, Paris, l854,
  p.~27.} took the Ctesibius in question to be the former and
accordingly placed Heron at the beginning of the first century~\bc,
say 126–50~\bc. But Philo of Byzantium\footnote{Philo,
  \emph{Mechan.\ Synt.}, p.~50, 38, ed.~Schöne.}, who repeatedly
mentions Ctesibius by name, says that the first mechanicians had the
advantage of being under kings who loved fame and supported the
arts. Hence our Ctesibius is more likely to have been the earlier
Ctesibius who was contemporary with Ptolemy~II Philadelphus.

But, whatever be the date of Ctesibius, we cannot safely conclude that
Heron was his immediate pupil.  The title ``Heron's (edition of)
Ctesibius's Belopoeica'' does not, in fact, justify any inferenee as
to the interval of time between the two works.

We now have better evidence for a \emph{terminus post quem}. The
\emph{Metrica} of Heron, besides quoting Archimedes and Apollonius,
twice refers to ``the books about straight lines (chords) in a
circle'' (\greek{ἐν τοῖς περὶ τῶν ἐν κύκλῳ εὐθειῶν}). Now we know of
no work giving a Table of Chords earlier than that of Hipparchus.  We
get, therefore, at once, 150~\bc. or thereabouts as the \emph{terminus
  post quem}.  But, again, Heron's \emph{Mechanica} quotes a
definition of ``centre of gravity'' as given by ``Posidonius, a
Stoic'': and, even if this Posidonius lived before Archimedes, as the
context seems to imply, it is certain that another work of Heron's,
the \emph{Definitions}, owes something to Posidonius of Apamea or
Rhodes, Cicero's teacher (135–51~\bc).  This brings Heron's date down
to the end of the first century~\bc, at least.

We have next to consider the relation, if any, between Heron and
Vitruvius. In his \emph{De Architectura} brought out apparently in
14~\bc, Vitruvius quotes twelve authorities on \emph{machinationes}
including Archytas (second), Archimedes (third), Ctesibius (fourth)
and Philo of Byzantium (sixth), but does not mention Heron.  Nor is it
possible to establish inter-dependence between Vitruvius and Heron;
the differences between them seem on the whole more numerous and
important than the resemblances (e.g.\ Vitruvius uses~3 as the value
of~$\pi$, while Heron always uses the Archimedean value~3). The
inference is that Heron can hardly have written earlier than the first
century~\ad

The most recent theory of Heron's date makes him later than Claudius
Ptolemy the astronomer (100–178~\ad). The arguments are mainly
these. (1)~Ptolemy claims as a discovery of his own a method of
measuring the distance between two places (as an arc of a great circle
on the earth's surface) in the case where the places are neither on
the same meridian nor on the same parallel circle.  Heron, in his
\emph{Dioptra}, speaks of this method as of a thing generally known to
experts. (2)~The dioptra described in Heron's work is a fine and
accurate instrument, much better than anything Ptolemy had at his
disposal. (3)~Ptolemy, in his work \greek{Περὶ ῥοπῶν}, asserted that
water with water round it has no weight and that the diver, however
deep he dives, does not feel the weight of the water above him. Heron,
strangely enough, accepts as true what Ptolemy says of the diver, but
is dissatisfied with the explanation given by ``some,'' namely that it
is because water is uniformly heavy—this seems to be equivalent to
Ptolemy's dictum that water in water has no weight—and he essays a
different explanation based on Archimedes. (4)~It is suggested that
the Dionysius to whom Heron dedicated his \emph{Definitions} is a
certain Dionysius who was \emph{praefectus urbi} in 301~\ad.

On the other hand Heron was earlier than Pappus, who was writing under
Diocletian (284–305~\ad), for Pappus alludes to and draws upon the
works of Heron. The net result, then, of the most recent research is
to place Heron in the third century~\ad\ and perhaps little earlier
than Pappus. Heiberg\footnote{\emph{Heronis Alexandrini opera},
  vol. \r5. (Teubner, 1914), p.~ix.} accepts this conclusion, which
may therefore, perhaps, be said to hold the field for the
present\footnote{Fuller details of the various arguments will be found
  in my \emph{History of Greek Mathematics}, 1921, vol.~\r2.,
  pp.~298–306.}.

That Heron wrote a systematic commentary on the \emph{Elements} might
be inferred from Proclus, but it is rendered quite certain by
references to the commentary in Arabian writers, and particularly in
an-Nairīzī's commentary on the first ten Books of the \emph{Elements}.
The \emph{Fihrist} says, under Euclid, that ``Heron wrote a commentary
on this book [the \emph{Elements}], endeavouring to solve its
difficulties\footnote{\emph{Das Matkematiker-Verzeichniss im Fihrist}
  (tr.\ Suter), p.~16.}''; and under Heron, ``He wrote: the book of
explanation of the obscurities in Euclid\footnote{\ibid~p.~22.}…''
An-Nairīzī's commentary quotes Heron by name very frequently, and
often in such a way as to leave no doubt that the author had Heron's
work actually before him. Thus the extracts are given in the first
person, introduced by ``Heron says'' (``Dixit Yrinus'' or ``Heron'');
and in other places we are told that Heron ``says nothing,'' or ``is
not found to have said anything,'' on such and such a proposition. The
commentary of an-Nairīzī is in part edited by Besthorn and Heiberg
from a Leiden \textsc{ms.}\ of the translation of the \emph{Elements}
by al-Ḥajjāj with the commentary attached\footnote{\emph{Codex
    Leidensis} 399, 1. \emph{Euclidis Elementa ex interpretatione
    al-Hadschdschadschii cum commentariis al-Narizii}. Five parts
  carrying the work to the end of Book~\r6.\ were issued in
  1893. 1897, 1900, 1905 and 1910 respectively.}.  But this
\textsc{ms.}\ only contains six Books, and several pages in the first
Book, which contain the comments of Simplicius on the first twenty-two
definitions of the first Book, are missing. Fortunately the commentary
of an-Nairīzī has been discovered in a more complete form, in a Latin
translation by Gherardus Cremonensis of the twelfth century, which
contains the missing comments by Simplicius and an-Nairīzī's comments
on the first ten Books. This valuable work has recently been edited by
Curtze\footnote{\emph{Anaritii in decem libros priores elementorum
    Euclidis commentarii ex interpretatione Gherardi
    Cremonensis…edidit} Maximilianus Curtze (Teubner, Leipzig,
  1899).}.

Thus from the three sources, Proclus, and the two versions of
an-Nairīzī, which supplement one another, we are able to form a very
good idea of the character of Heron's commentary. In some cases
observations given by Proclus without the name of their author are
seen from an-Nairīzī to be Heron's; in a few cases notes attributed by
Proclus to Heron are found in an-Nairīzī without Heron's name; and,
curiously enough, one alternative proof (of \prop{1}{25}) given as
Heron's by Proclus is introduced by the Arab with the remark that he
has not been able to discover who is the author.

Speaking generally, the comments of Heron do not seem to have
contained much that can be called important. We find

(1)~A few general notes, e.g. that Heron would not admit more than
three axioms.

(2)~Distinctions of a number of particular cases of Euclid's
propositions according as the figure is drawn in one way or in
another.

Of this class are the different cases of \prop{1}{35, 36}, \prop{3}{7,
  8} (where the chords to be compared are drawn on different sides of
the diameter instead of on the same side), \prop{3}{12} (which is not
Euclid's, but Heron's own, adding the case of external contact to that
of internal contact in \prop{3}{11}), \prop{6}{19} (where the triangle
in which an additional line is drawn is taken to be the \emph{smaller}
of the two), \prop{7}{19} (where he gives the particular case of three
numbers in continued proportion, instead of four proportionals).

(3)~Alternative proofs. Of these there should be mentioned
(\emph{a})~the proofs of \prop{2}{1–10} ``without a figure,'' being
simply the algebraic forms of proof, easy but uninstructive, which are
so popular nowadays, the proof of \prop{3}{25} (placed after
\prop{3}{30} and starting from the arc instead of the chord),
\prop{3}{10} (proved by \prop{3}{9}), \prop{3}{13} (a proof preceded
by a lemma to the effect that a straight line cannot meet a circle in
more than two points). Another class of alternative proof is
(\emph{b})~that which is intended to meet a particular objection
(\greek{ἔνστασις}) which had been or might be raised to Euclid's
construction. Thus in certain cases he avoids \emph{producing} a
particular straight line, where Euclid produces it, in order to meet
the objection of any one who should deny our right to assume that
there is \emph{any space available}\footnote{Cf.\ Proclus, 275, 7
  \greek{εἰ δὲ λέγοι τις τόπον μὴ εἰδέναι}…, 289, 18 \greek{λέγει οὖν
    τις ὅτι οὐκ ἔστι τόπος}….}.  Of this class are Heron's proofs of
\prop{1}{11}, \prop{1}{20}, and his note on \prop{1}{16}. Similarly on
\prop{1}{48} he supposes the right-angled triangle which is
constructed to be constructed on the \emph{same} side of the common
side as the given triangle is. A third class (\emph{c})~is that which
avoids \emph{reductio ad absurdum}. Thus, instead of indirect proofs,
Heron gives direct proofs of \prop{1}{19} (for which he requires, and
gives, a preliminary lemma), and of \prop{1}{25}.

(4)~Heron supplies certain \emph{converses} of Euclid's propositions,
e.g.\ converses of
\prop{2}{12, 13},
\prop{8}{27}.

(5)~A few additions to, and extensions of, Euclid's propositions are
also found. Some are unimportant, e.g.\ the construction of isosceles
and scalene triangles in a note on~\prop{1}{1}, the construction of
\emph{two} tangents in \prop{3}{17}, the remark that \prop{7}{3} about
finding the greatest common measure of three numbers can be applied to
as many numbers as we please (as Euclid tacitly assumes in
\prop{7}{31}). The most important extension is that of \prop{3}{20} to
the case where the angle at the circumference is greater than a right
angle, and the direct deduction from this extension of the result of
\prop{3}{22}. Interesting also are the notes on \prop{1}{37} (on
\prop{1}{24} in Proclus), where Heron proves that two triangles with
two sides of one equal to two sides of the other and with the included
angles supplementary are equal, and compares the areas where the sum
of the two included angles (one being supposed greater than the other)
is less or greater than two right angles, and on \prop{1}{47}, where
there is a proof (depending on preliminary lemmas) of the fact that,
in the figure of the proposition, the straight lines $AL$, $BK$, $CF$
meet in a point. After \prop{4}{16} there is a proof that, in a
regular polygon with an even number of sides, the bisector of one
angle also bisects its opposite, and an enunciation of the
corresponding proposition for a regular polygon with an odd number of
sides.

Van Pesch\footnote{\emph{De Procli fontibus} Lugduni-Batavorum, 1900,}
gives reason for attributing to Heron certain other notes found in
Proclus, viz.\ that they are designed to meet the same sort of points
as Heron had in view in other notes undoubtedly written by him. These
are (\emph{a})~alternative proofs of \prop{1}{5}, \prop{1}{17},
and~\prop{1}{32}, which avoid the producing of certain straight lines,
(\emph{b})~an alternative proof of \prop{1}{9} avoiding the
construction of the equilateral triangle on the side of $BC$ opposite
to~$A$; (\emph{c})~partial converses of \prop{1}{35–38}, starting from
  the equality of the areas and the fact of the parallelograms or
  triangles being in the same parallels, and proving that the bases
  are the same or equal, may also be Heron's.  Van Pesch further
  supposes that it was in Heron's commentary that the proof by
  Menelaus of \prop{1}{25} and the proof by Philo of \prop{1}{8} were
  given.

The last reference to Heron made by an-Nairīzī occurs in the note on
\prop{8}{27}, so that the commentary of the former must at least have
reached that point.

\section{Porphyry}

The Porphyry here mentioned is of course the Neo-Platonist who lived
about 232–304~\ad\ Whether he really wrote a systematic commentary on
the \emph{Elements} is uncertain. The passages in Proclus which seem
to make this probable are two in which he mentions him (1)~as having
demonstrated the necessity of the words ``not on the same side'' in
the enunciation of \prop{1}{14}\footnote{Proclus, pp.~297, 1–298,
  10.}, and (2)~as having pointed out the necessity of understanding
correctly the enunciation of \prop{1}{26}, since, if the particular
injunctions as to the sides of the triangles to be taken as equal are
not regarded, the student may easily fall into
error\footnote{\ibid~p.~351, 13, 14 and the pages preceding.}.  These
passages, showing that Porphyry carefully analysed Euclid's
\emph{enunciations} in these cases, certainly suggest that his remarks
were part of a systematic commentary.  Further, the list of
mathematicians in the \emph{Fihrist} gives Porphyry as having written
``a book on the Elements.'' It is true that Wenrich takes this book to
have been a work by Porphyry mentioned by Suidas and Proclus
(\emph{Theolog.\ Platan.}), \greek{περὶ ἀρχῶν}
libri~\r2.\footnote{\emph{Fihrist} (tr.\ Suter), p.~9, 10 and p.~45
  (note~5).}

There is nothing of importance in the notes attributed to Porphyry
by Proclus.

(1)~Three alternative proofs of \prop{1}{20}, which avoid
\emph{producing} a side of the triangle, are assigned to Heron and
Porphyry without saying which belonged to which.  If the first of the
three was Heron's, I agree with van Pesch that it is more probable
that the two others were both Porphyry's than that the second was
Heron's and only the third Porphyry's. For they are similar in
character, and the third uses a result obtained in the
second\footnote{Van Pesch, \emph{De Procli fontibus}, pp.~129, 130.
  Heiberg assigned them as above in his \emph{Euklid-Studien}
  (p.~160), but seems to have changed his view later.  (See
  Besthorn-Heiberg, \emph{Codex Leidensis}, p.~93, note~2.)}.

(2)~Porphyry gave an alternative proof of \prop{1}{18} to meet a
childish objection which is supposed to require the part of $AC$ equal
to $AB$ to be cut off from $CA$ and not from~$AC$.

Proclus gives a precisely similar alternative proof of \prop{1}{6} to
meet a similar supposed objection; and it may well be that, though
Proclus mentions no name, this proof was also Porphyry's, as van Pesch
suggests\footnote{Van Pesch, \emph{op.~cit.}\ pp.~130–1.}.

Two other references to Porphyry found in Proclus cannot have anything
to do with commentaries on the \emph{Elements}. In the first a work
called the \greek{Συμμικτά} is quoted, while in the second a
philosophical question is raised.

\section{Pappus}

The references to Pappus in Proclus are not numerous; but we have
other evidence that he wrote a commentary on the \emph{Elements}.
Thus a scholiast on the definitions of the \emph{Data} uses the phrase
``as Pappus says at the beginning of his (commentary) on the 10th
(book) of Euclid\footnote{Euclid's \emph{Data},
  ed.\ Menge. p.~262.}.'' Again in the \emph{Fihrist} we are told that
Pappus wrote a commentary to the tenth book of Euclid in two
parts\footnote{\emph{Fihrist} (tr.\ Sutter), p.~22.}.  Fragments of
this still survive in a \textsc{ms.}\ described by
Woepcke\footnote{\emph{Mémoires présentés à ĺacadémie des sciences},
  1856, \r14. pp.~658–719.}, Paris. No.~952. 2 (supplément arabe de la
Bibliothèque impériale), which contains a translation by Abū `Uthmān
(beginning of 10th century) of a Greek commentary on Book~\r10. It is
in two books, and there can now be no doubt that the author of the
Greek commentary was Pappus\footnote{Woepcke read the name of the
  author, in the title of the first book, as \emph{B.los} (the dot
  representing a missing vowel). He quotes also from other
  \textsc{mss.}\ (e.g.\ of the \emph{Ta'rīkh al-Ḥukamā} and of the
  \emph{Fihrist}) where he reads the name of the commentator as
  \emph{B.lis}, \emph{B.n.s} or \emph{B.l.s}.  Woepcke takes this
  author to be Valens, and thinks it possible that he may be the same
  as the astrologer Vettius Valens.  This Heiberg
  (\emph{Euklid-Studien}, pp.~169, 170) proves to be impossible,
  because, while one of the \textsc{mss.}\ quoted by Woepcke says that
  ``\emph{B.n.s}, le \emph{Roûmi} (late-Greek) was later than Claudius
  Ptolemy and the \emph{Fihrist} says ``\emph{B.l.s}, le \emph{Roûmi}
  wrote a commentary on Ptolemy's \emph{Planisphaerium}, Vettius
  Valens seems to have Lived under Hadrian, and must therefore have
  been an elder contemporary of Ptolemy. But Suter shows
  (\emph{Fihrist}, p.~22 and p.~54, note~92) that \emph{Banos} is only
  distinguished from \emph{Babos} by the position of a certain dot,
  and \emph{Balos} may also easily have arisen from an original
  \emph{Babos} (there is no P in Arabic), so that Pappus must be the
  person meant.  This is further confirmed by the fact that the
  \emph{Fihrist} gives this author and Valens as the subjects of two
  separate paragraphs, attributing to the latter astrological works
  only.}.  Again Eutocius, in his note on Archimedes, \emph{On the
  Sphere and Cylinder} \r1.~13, says that Pappus explained in his
commentary on the \emph{Elements} how to inscribe in a circle a
polygon similar to a polygon inscribed in another circle; and this
would presumably come in his commentary on Book~\r12., just as the
problem is solved in the second scholium on Eucl.~\prop{12}{1}. Thus
Pappus' commentary on the \emph{Elements} must have been pretty
complete, an additional confirmation of this supposition being
forthcoming in the reference of Marinus (a pupil and follower of
Proclus) in his preface to the \emph{Data} to ``the commentaries of
Pappus on the book\footnote{Heiberg, \emph{Euklid-Studien}, p.~173;
  Euclid's \emph{Data}, ed.\ Menge, pp.~256, lii.}.''

The actual references to Pappus in Proclus are as follows:

(1)~On the Postulate~(4) that all right angles are equal, Pappus is
quoted as saying that the converse, viz.\ that all angles equal to a
right angle are right, is not true\footnote{Proclus, pp.~189, 190.},
since the angle included between the arcs of two semicircles which are
equal, and have their diameters at right angles and terminating at one
point, is equal to a right angle, but is not a right angle.

(2) On the axioms Pappus is quoted as saying that, in addition to
Euclid's axioms, others are on record as well
(\greek{συναναγράφεσθαι}) about unequals added to equals and equals
added to unequals\footnote{\ibid~p.~197, 6–10.}; these, says Proclus,
follow from the Euclidean axioms, while others given by Pappus are
involved by the definitions, namely those which assert that ``all
parts of the plane and of the straight line coincide with one
another,'' that ``a point divides a straight line, a line a surface,
and a surface a solid,'' and that ``the infinite is (obtained) in
magnitudes both by addition and diminution\footnote{\ibid~p.~198,
  3–15.}.''

(3)~Pappus gave a pretty proof of \prop{1}{5}. This proof has, I
think, been wrongly understood; on this point see my note on the
proposition.

(4)~On \prop{1}{47} Proclus says\footnote{Proclus, p. 429, 9–15.}:
``As the proof of the writer of the Elements is manifest, I think that
it is not necessary to add anything further, but that what has been
said is sufficient, since indeed those who have added more, like Heron
and Pappus, were obliged to make use of what is proved in the sixth
book, without attaining any important result.'' We shall see what
Heron's addition consisted of; what Pappus may have added we do not
know, unless it was something on the lines of his extension of
\prop{1}{47} found in the \emph{Synagoge} (\r4.~p.~176, ed.\ Hultsch).

We may fairly conclude, with van Pesch\footnote{Van Pesch, \emph{De
    Procli fontibus}, p.~134 sqq.}, that Pappus is drawn upon in
various other passages of Proclus where he quotes no authority, but
where the subject-matter reminds us of other notes expressly assigned
to Pappus or of what we otherwise know to have been favourite
questions with him. Thus:

1.~We are reminded of the curvilineal angle which is equal to but not
a right angle by the note on \prop{1}{32} to the effect that the
converse (that a figure with its interior angles together equal to two
right angles is a triangle) is not true unless we confine ourselves to
rectilineal figures. This statement is supported by reference to a
figure formed by four semicircles whose diameters form a square, and
one of which is turned inwards while the others are turned outwards.
The figure forms two angles ``equal to'' right angles in the sense
described by Pappus on Post.\ref{post:4}, while the other curvilineal
angles are not considered to be angles at all, and are left out in
summing the internal angles. Similarly the allusions in the notes on
\prop{1}{4, 23} to curvilineal angles of which certain moon-shaped
angles (\greek{μηνοειδεῖς}) are shown to be ``equal to'' rectilineal
angles savour of Pappus.

2.~On \prop{1}{9} Proclus says\footnote{Proclus, p.~272, 10.} that
``Others, starting from the Archimedean spirals, divided any given
rectilineal angle in any given ratio.''  We cannot but compare this
with Pappus \r4.~p.~286, where the spiral is so used; hence this note,
including remarks immediately preceding about the conchoid and the
quadratrix, which were used for the same purpose, may very well be due
to Pappus.

3.~The subject of isoperimetric figures was a favourite one with
Pappus, who wrote a recension of Zenodorus' treatise on the
subject\footnote{Pappus, v.\ pp.~304–350; for Zenodorus' own treatise
  see Hultsch's Appendix, pp.~1189–1211.}.  Now on \prop{1}{35}
Proclus speaks\footnote{Proclus, pp.~396–8.} about the paradox of
parallelograms having equal area (between the same parallels) though
the two sides between the parallels may be of any length, adding that
of parallelograms with equal perimeter the rectangle is greatest if
the base be given, and the square greatest if the base be not given
etc. He returns to the subject on \prop{1}{37} about
triangles\footnote{\ibid~pp.~403–4.}.
Compare\footnote{\ibid~pp.~236–7.} also his note on \prop{1}{4}. These
notes may have been taken from Pappus.

4.~Again, on \prop{1}{21}, Proclus remarks on the paradox that
straight lines may be drawn from the base to a point within a triangle
which are (1)~together greater than the two sides, and (2)~include a
less angle, provided that the straight lines may be drawn from points
in the base other than its extremities. The subject of straight lines
satisfying condition (1)~was treated at length, with reference to a
variety of cases, by Pappus\footnote{Pappus, \r3. pp.~104–130.}, after
a collection of ``paradoxes'' by Erycinus, of whom nothing more is
known. Proclus gives Pappus' first case, and adds a rather useless
proof of the possibility of drawing straight lines satisfying
condition (2)~\emph{alone}, adding that ``the proposition stated has
been proved by me without using the parallels of the
commentators\footnote{Proclus, p.~328, 15.}.''  By ``the
commentators'' Pappus is doubtless meant.

5.~Lastly, the ``four-sided triangle,'' called by Zenodorus the
``hollow-angled,''\footnote{Proclus, p.~165, 24; cf.\ pp.~328, 329.}
is mentioned in the notes on \r1.~Def.~24–29 and \prop{1}{21}.  As
Pappus wrote on Zenodorus' work in which the term
occurred\footnote{See Pappus, ed.\ Hultsch, pp.~1154, 1206.}, Pappus
may be responsible for these notes.

\section{Simplicius}.

According to the \emph{Fihrist}\footnote{\emph{Fihrist} (tr.\ Suter),
  p.~21.}, Simplicius the Greek wrote ``a commentary to the beginning
of Euclid's book, which forms an introduction to geometry.'' And in
fact this commentary on the definitions, postulates and axioms
(including the postulate known as the Parallel-Axiom) is preserved in
the Arabic commentary of an-Nairīzī\footnote{An-Nairīzī,
  ed.\ Besthorn-Heiberg, pp.~9–41, 119–133, ed.\ Curtis, pp.~1–37,
  65–73. The \emph{Codex Leidensis}, from which Besthorn and Heiberg's
  edition is taken, has unfortunately lost some leaves, so that there
  is a gap from Def.~1 to Def.~13 (parallels). The loss is, however,
  made good by Curtze's edition of the translation by Gherard of
  Cremona.}. On two subjects this commentary of Simplicius quotes a
certain ``Aganis,'' the first subject being the definition of an
angle, and the second the definition of parallels and the
parallel-postulate. Simplicius gives word for word, in a long passage
placed by an-Nairīzī after \prop{1}{29}, an attempt by ``Aganis'' to
prove the parallel-postulate. It starts from a definition of parallels
which agrees with Geminus' view of them as given by
Proclus\footnote{Proclus, p.~177, 21.}, and is closely connected with
the definition given by Posidonius\footnote{\ibid~p.~176, 7.}. Hence
it has been assumed that ``Aganis'' is none other than Geminus, and
the historical importance of the commentary of Simplicius has been
judged accordingly. But it has been recently shown by Tannery that the
identification of ``Aganis'' with Geminus is practically
impossible\footnote{\emph{Bibliotheca Mathematica}, \r2\tsub{3}, 1900,
  pp.~9–11.}.  In the translation of Besthorn-Heiberg Aganis is called
by Simplicius in one place ``philosophus Aganis,'' in another
``magister noster Aganis,'' in Gherard's version he is ``socius
Aganis'' and ``socius noster Aganis.'' These expressions seem to leave
no doubt that Aganis was a contemporary and friend, if not master, of
Simplicius; and it is impossible to suppose that Simplicius
(fl.\ about 500~\ad) could have used them of a man who lived four and
a half centuries before his time. A phrase in Simplicius'
word-for-word quotation from Aganis leads to the same conclusion. He
speaks of people who objected ``even in ancient times'' (iam
antiquitus) to the use by geometers of this postulate. This would not
have been an appropriate phrase had Geminus been the writer. I do not
think that this difficulty can he got over by Suter's
suggestion\footnote{\emph{Zeitschrift für Math.\ u.\ Physik}, \r44.,
  hist.-litt.\ Abth.\ p.~61.} that the passages in question may have
been taken out of \emph{Heron's} commentary, and that an-Nairīzī may
have forgotten to name the author; it seems clear that Simplicius is
the person who described ``Aganis.''  Hence we are driven to suppose
that Aganis was not Geminus, but some unknown contemporary of
Simplicius\footnote{The above argument seems to me quite
  insuperable. The other arguments of Tannery do not, however, carry
  conviction to my mind. I do not follow the reasoning based on
  Aganis' definition of an angle. It appears to me a pure assumption
  that Geminus would have seen that Posidonius' definition of
  parallels was not admissible. Nor does it seem to me to count for
  much that Proclus, while telling us that Geminus held that the
  postulate ought to be proved and warned the unwary against hastily
  concluding that two straight lines approaching one another must
  necessarily meet (cf.\ a curve and its asymptote), gives no hint
  that Geminus did try to prove the postulate. It may well be that
  Proclus omitted Geminus' ``proof'' (if he wrote one) because he
  preferred Ptolemy's attempt which he gives
  (pp.~365–7).}. Considerable interest will however continue to attach
to the comments of Simplicius so fortunately preserved.

Proclus tells us that one Aegaeas (? Aenaeas) of Hierapolis wrote an
epitome of the \emph{Elements}\footnote{Proclus, p.~361, 21.}; but we
know nothing more of him or of it.

\chapter{Proclus and His Sources\protect\footnote{My task in this chapter is made easy by the appearance, in the nick
  of time, of the dissertation \emph{De Procli fontibus} by J. G. van
  Pesch (Lugduni-Batavorum, Apud L. van Nifterik,
  \textsc{mdcccc}). The chapters dealing directly with the subject
  show a thorough acquaintance on the part of the author with all the
  literature hearing on it; he covers the whole field and he exercises
  a sound and sober judgment in forming his conclusions.  The same
  cannot always be said of his only predecessor in the same inquiry,
  Tannery (in \emph{La Géométrie grecque}, 1887), who often robs his
  speculations of much of their value through his proneness to run
  away with an idea; he does so in this case, basing most of his
  conclusions on an arbitrary and unwarranted assumption as to the
  significance of the words \greek{οἰ περί τινα} (e.g.\ \greek{Ἤρωνα},
  \greek{Ποσειδώνιον} etc) as used in Proclus.}}

It is well known that the commentary of Proclus on Eucl.\ Book \r1.
is one of the two main sources of information as to the history of
Greek geometry which we possess, the other being the \emph{Collection}
of Pappus. They are the more precious because the original works of
the forerunners of Euclid, Archimedes and Apollonius are lost, having
probably been discarded and forgotten almost immediately after the
appearance of the masterpieces of that great trio.

Proclus himself lived 410–485~\ad, so that there had already passed a
sufficient amount of time for the tradition relating to the
pre-Euclidean geometers to become obscure and defective. In this
connexion a passage is quoted from Simplicius\footnote{Simplicius on
  Aristotle's \emph{Physics}, ed. Diels, pp.~54–69.} who, in his
account of the quadrature of certain lunes by Hippocrates of Chios,
while mentioning two authorities for his statements, Alexander
Aphrodisiensis (about 220~\ad) and Eudemus, says in one
place\footnote{\ibid~p.~68, 32.}, ``As regards Hippocrates of Chios we
must pay more attention to Eudemus, \emph{since he was nearer the
  times}, being a pupil of Aristotle.''

The importance therefore of a critical examination of Proclus'
commentary with a view to determining from what original sources he
drew need not be further emphasised.

Proclus received his early training in Alexandria, where Olympiodorus
was his instructor in the works of Aristotle, and mathematics was
taught him by one Heron\footnote{Cf.\ Martin, \emph{Recherches sur la
    vie et les ouvrages d'Héron d'Alexandrie}, pp.~240–2.} (of course
a different Heron from the ``\emph{mechanicus} Hero'' of whom we have
already spoken). He afterwards went to Athens where he was imbued by
Plutarch, and by Syrianus, with the Neo-Platonic philosophy, to which
he then devoted heart and soul, becoming one of its most prominent
exponents. He speaks everywhere with the highest respect of his
masters, and was in turn regarded with extravagant veneration by his
contemporaries, as we learn from Marinus his pupil and biographer. On
the death of Syrianus he was put at the head of the Neo-Platonic
school. He was a man of untiring industry, as is shown by the number
of books which he wrote, including a large number of commentaries,
mostly on the dialogues of Plato. He was an acute dialectician, and
pre-eminent among his contemporaries in the range of his
learning\footnote{Zeller calls him ``Der Gelehrte, dem kein Feld
  damaligen Wissens verschlossen ist.''}; he was a competent
mathematician; he was even a poet. At the same time he was a believer
in all sorts of myths and mysteries and a devout worshipper of
divinities both Greek and Oriental.

Though he was a competent mathematician, he was evidently much more a
philosopher than a mathematician\footnote{Van Pesch observes that in
  his commentaries on the \emph{Timaeus} (pp.~671–2) he speaks as no
  real mathematician could have spoken. In the passage referred to the
  question is whether the sun occupies a middle place among the
  planets. Proclus rejects the view of Hipparchus and Ptolemy because
  ``\greek{ὁ θεουργός}'' (sc.\ the Chaldean, says Zeller) thinks
  otherwise, ``whom it is not lawful to disbelieve.''  Martin says
  rather neatly, ``Pour Proclus, les Éléments d'Euclide ont ĺheureuse
  chance de n'être contredits ni par les Oracles chaldaïques, ni par
  les spéculations des pythagoriciens anciens et nouveaux……''}.  This
is shown even in his commentary on Eucl.\ \r1., where, not only in the
Prologues (especially the first), but also in the notes themselves, he
seizes any opportunity for a philosophical digression. He says himself
that he attaches most importance to ``the things which require deeper
study and contribute to the sum of philosophy\footnote{Proclus, p.~84,
  13.}''; alternative proofs, cases, and the like (though he gives
many) have no attraction for him; and, in particular, he attaches no
value to the addition of Heron to \prop{1}{47}\footnote{\ibid~p.~429,
  12.}, which is of considerable mathematical interest.  Though he
esteemed mathematics highly, it was only as a handmaid to philosophy.
He quotes Plato's opinion to the effect that ``mathematics, as making
use of hypotheses, falls short of the non-hypothetical and perfect
science\footnote{\ibid~p.~31, 20. }''…“Let us then not say that Plato
excludes mathematics from the sciences, but that he declares it to be
secondary to the one supreme science\footnote{\ibid~p.~32, 2.}.''  And
again, while ``mathematical science must be considered desirable in
itself, though not with reference to the needs of daily life,'' ``if
it is necessary to refer the benefit arising from it to something
else, we must connect that benefit with intellectual knowledge
(\greek{νοερὰν γνῶσιν}), to which it leads the way and is a
propaedeutic, clearing the eye of the soul and taking away the
impediments which the senses place in the way of the knowledge of
universals (\greek{τῶν ὅλων})\footnote{\ibid~p.~27, 27 to 28, 7;
  cf.\ also p.~21, 25, pp.~46, 47.}.''

We know that in the Neo-Platonic school the younger pupils learnt
mathematics; and it is clear that Proclus taught this subject, and
that this was the origin of the commentary. Many passages show him as
a master speaking to scholars. Thus ``we have illustrated and made
plain all these things in the case of the first problem, but it is
necessary that \emph{my hearers} should make the same inquiry as
regards the others as well\footnote{Proclus, p.~210, 18.},'' and ``I
do not indicate these things as a merely incidental matter but as
preparing us beforehand for the doctrine of the
Timaeus\footnote{\ibid~p. 384, 2.}.''  Further, the pupils whom he was
addressing were \emph{beginners} in mathematics; for in one place he
says that he omits ``for the present'' to speak of the discoveries of
those who employed the curves of Nicomedes and Hippias for trisecting
an angle, and of those who used the Archimedean spiral for dividing an
angle in any given ratio, because these things would be too difficult
for beginners (\greek{δυσθεωρήτους τοῖς
  εἰσαγομένοις})\footnote{\ibid~p.~272, 12.}. Again, if his pupils had
not been beginners, it would not have been necessary for Proclus to
explain what is meant by saying that sides subtend certain
angles\footnote{\ibid~p.~238, 12.}, the difference between
\emph{adjacent} and \emph{vertical} angles\footnote{\ibid~p.~298, 14.}
etc., or to exhort them, as he often does, to work out other
particular cases for themselves, for practice (\greek{γυμνασίας
  ἕνεκα})\footnote{Cf.\ p.~224, 15 (on \r1.2).}.

The commentary seems then to have been founded on Proclus' lectures to
beginners in mathematics. But there are signs that it was revised and
re-edited for a larger public; thus he gives notice in one
place\footnote{\ibid~p. 84, 9.} ``to those who shall come upon'' his
work (\greek{τοῖς ἐντευξομένοις}). There are also passages which could
not have heen understood by the beginners to whom he lectured,
e.g.\ passages about the cylindrical helix\footnote{\ibid~p.~105.},
conchoids and cissoids\footnote{\ibid~p.~113.}. These passages may
have been added in the revised edition, or, as van Pesch conjectures,
the explanations given in the lectures may have been much fuller and
more comprehensible to beginners, and they may haw; been shortened on
revision.

In his comments on the propositions of Euclid, Proclus generally
proceeds in this way: first he gives explanations regarding Euclid's
proofs, secondly he gives a few different cases, mainly for the sake
of practice, and thirdly he addresses himself to refuting objections
raised by cavillers to particular propositions. The latter class of
note he deems necessary because of ``sophistical cavils'' and the
attitude of the people who rejoiced in finding paralogisms and in
causing annoyance to scientific men\footnote{\ibid~p.~375, 8.}. His
commentary does not seem to have been written for the purpose of
correcting or improving Euclid. For there are very few passages of
mathematical content in which Proclus can be supposed to be
propounding anything of his own; nearly all are taken from the works
of others, mostly earlier commentators, so that, for the purpose of
improving on or correcting Euclid, there was no need for his
commentary at all. Indeed only in one place does he definitely bring
forward anything of his own to get over a difficulty which he finds in
Euclid\footnote{\ibid~pp.~368–373.}; this is where he tries to prove
the parallel-postulate, after first giving Ptolemy's attempt and then
pointing out objections to it. On the other hand, there are a number
of passages in which he extols Euclid; thrice\footnote{Proclus,
  p.~280, 9; p.~282, 10; pp.~335, 336.} also he supports Euclid
against Apollonius where the latter had given proofs which he
considered better than Euclid's (\prop{1}{10, 11, and 23}).

Allusion must be made to the debated question whether Proclus
continued his commentaries beyond Book~\r1, His intention to do so is
clear from the following passages. Just after the words above quoted
about the trisection etc.\ of an angle by means of certain curves he
says, ``For we may perhaps more appropriately examine these things on
the third book, where the writer of the Elements bisects a given
circumference\footnote{\ibid~p.~272, 14.}.'' Again, after saying that
of all parallelograms which have the same perimeter the square is the
greatest ``and the rhomboid least of all,'' he adds: ``But this we
will prove in another place; for it is more appropriate to the
(discussion of the) hypotheses of the second
book\footnote{\ibid~p.~398, 18.}.''  Lastly, when alluding (on
\prop{1}{45}) to the squaring of the circle, and to Archimedes'
proposition that any circle is equal to the right-angled triangle in
which the perpendicular is equal to the radius of the circle and the
base to its perimeter, he adds, ``But of this
elsewhere\footnote{\ibid~p.~423, 6.}''; this may imply, an intention
to treat of the subject on Eucl.~\r12., though Heiberg doubts
it\footnote{Heiberg, \emph{Euklid-Studien}, p.~165, note.}.  But it is
clear that, at the time when the commentary on Book~\r1.\ was written,
Proclus had not yet begun to write on the other Books and was
uncertain whether he would be able to do so: for at the end he
says\footnote{Proclus, p.~432, 9,}, ``For my part, if I should be able
to discuss the other books\footnote{The words in the Greek are:
  \greek{εἰ μὲν δυνηθείημεν καὶ τοῖς λοιποῖς τὸν αὐτὸν τρόπον
    ἐξελθεῖν}.  For \greek{ἐξελθεῖν} Heiberg would read
  \greek{ἐπεξελθεῖν}.} in the same manner, I should give thanks to the
gods; but, if other cares should draw me away, I beg those who are
attracted by this subject to complete the exposition of the other
books as well, following the same method, and addressing themselves
throughout to the deeper and better defined questions involved''
(\greek{τὸ πραγματειῶδες πανταχοῦ καὶ εὐδιαίρετον μεταδιώκοντας}).

There is in fact no satisfactory evidence that Proclus did actually
write any more commentaries than that on Book~\r1.\footnote{True, a
  Vatican \textsc{ms.}\ has a collection of scholia on
  Books~\r1.\ (extracts from the extant commentary of Proclus), \r2.,
  \r5., \r6., \r10.\ headed \greek{Εἰς τὰ Εὐκλείδου στοιχεῖα
    προλαμβανόμενα ἐκ τῶν Πρόκλου στροπάδην καὶ κατ’ ἐπιτομήν}.
  Heiberg holds that this title itself suggests that the authorship of
  Proclus was limited to the scholia on Book~\r1.; for
  \greek{προλαμβανόμενα ἐκ τῶν Πρόκλου} suits extracts from Proclus'
  \emph{prologues}, but hardly scholia to later Books.  Again, a
  certain scholium (Heiberg in \emph{Hermes}. \r38., 1903, p.~341,
  No.~17) purports to quote words from the end of ``a scholium of
  Proclus'' on \prop{10}{9}. The words quoted are from the scholium
  \r10.\ No.~62, one of the Scholia Vaticana. But none of the other,
  older, sources connect Proclus' name with \r10.\ No.~62; it is
  probable therefore that a Byzantine, who had in his \textsc{ms.}\ of
  Euclid the collection of Schol.\ Vat.\ and knew that those on
  Book~\r1.\ came from Proclus, himself attached Proclus' name to the
  others.} The contrary view receives support from two facts pointed
out by Heiberg, viz.\ (1)~that the scholiast's copy of Proclus was not
so much better than our \textsc{mss.}\ as to suggest that the
scholiast had further commentaries of Proclus which have vanished for
us\footnote{While one class of scholia (Schol.\ Vat.)\ have some
  better readings than our \textsc{mss.}\ of Proclus have, and partly
  fill up the gaps at \prop{1}{36, 37} and \prop{1}{41–43}, the other
  class (Schol.\ Vind.)\ derive from an inferior Proclus
  \textsc{ms.}\ which also had the same lacunae.1}; (2)~that there is
no trace in the scholia of the notes which Proclus promised in the
passages quoted above.

Coming now to the question of the sources of Proclus, we may say that
everything goes to show that his commentary is a compilation, though a
compilation ``in the better sense'' of the term\footnote{Knoche,
  \emph{Untersuchungenüber des Proklus Diadochus Commentar zu Euklid's
    Elementen} (1862), p.~11.}. He does not even give us to understand
that we shall find in it much of his own; ``let us,'' he says, ``now
turn to the exposition of the theorems proved by Euclid, selecting the
more subtle of the comments made on them by the ancient writers, and
cutting down their interminable diffuseness…\footnote{Proclus, p.~200,
  10—13.}'': not a word about anything of his own. At the same time,
he seems to imply that he will not necessarily on each occasion quote
the source of each extract from an earlier commentary; and, in fact,
while he quotes the name of his authority in many places, especially
where the subject is important, in many others, where it is equally
certain that he is not giving anything of his own, he mentions no
authority. Thus he quotes Heron by name six times; but we now know,
from the commentary of an-Nairīzī, that a number of other passages,
where he mentions no name, are taken from Heron, and among them
the. not unimportant addition of an alternative proof to
\prop{1}{19}. Hence we can by no means conclude that, where no authority
is mentioned, Proclus is giving notes of his own. The presumption is
generally the other way; and it is often possible to arrive at a
conclusion, either that a particular note is not Proclus' own, or that
it is definitely attributable to someone else, by applying the
ordinary principles of criticism. Thus, where the note shows an
unmistakable affinity to another which Proclus definitely attributes
to some commentator by name, especially when both contain some
peculiar and distinctive idea, we cannot have much doubt in assigning
both to the same commentator\footnote{Instances of the application of
  this criterion will be found in the discussion of Proclus'
  indebtedness to the commentaries of Heron, Porphyry and
  Pappus.}. Again, van Pesch finds a criterion in the form of a note,
where the explanation is so condensed as to be only just intelligible;
the note is that in which a converse of \prop{1}{32} is
proved\footnote{Van Pesch attributes this converse and proof to
  Pappus, arguing from the fact that the proof is followed by a
  passage which, on comparison with Pappus' note on the postulate that
  all right angles are equal, he feels justified in assigning to
  Pappus, I doubt if the evidence is sufficient.} the proposition
namely that a rectilineal figure which has all its interior angles
together equal to two right angles is a triangle.

It is not safe to attribute a passage to Proclus himself because he
uses the first person in such expressions as ``I say'' or ``I will
prove''—for he was in the habit of putting into his own words the
substance of notes borrowed from others—nor because, in speaking of an
objection raised to a particular proposition, he uses such expressions
as ``perhaps someone may object'' (\greek{ἴσως δ’ ἄν τινες
  ἐνσταῖεν}…): for sometimes other words in the same passage, indicate
that the objection had actually been taken by someone\footnote{Van
  Pesch illustrates this by an objection refuted in the note on
  \prop{1}{9}, p.~273, 11~sqq.  After using the above expression to
  introduce the objection, Proclus uses further on (p.~273, 25) the
  term ``they say'' (\greek{φασίν}).}.  Speaking generally, we shall
not be justified in concluding that Proclus is stating something new
of his own unless he indicates this himself in express terms.

As regards the form of Proclus' references to others by name, van
Pesch notes that he very seldom mentions the particular \emph{work}
from which he is borrowing. If we leave out of account the references
to Plato's dialogues, there are only the following references to
books: the \emph{Bacchae} of Philolaus\footnote{Proclus, p.~22, 15.},
the \emph{Symmitka} of Porphyry\footnote{\ibid~p.~56, 25.}, Archimedes
\emph{On the Sphere and Cylinder}\footnote{\ibid~p.~71, 18.},
Apollonius \emph{On the cochlias}\footnote{\ibid~p.~105, 5.}, a book
by Eudemus on \emph{The Angle}\footnote{\ibid~p.~125, 8.}, a whole
book of Posidonius directed against Zeno of the Epicurean
sect\footnote{\ibid~p.~200, 2.}, Carpus'
\emph{Astronomy}\footnote{\ibid~p.~241, 19.}, Eudemus' \emph{History
  of Geometry}\footnote{\ibid~p. 352, 15.}, and a tract by Ptolemy on
the parallel-postulate\footnote{\ibid~p.~362, 15.}.

Again, Proclus does not always indicate that he is quoting something
at second-hand. He often does so, e.g.\ he quotes Heron as the
authority for a statement about Philippus, Eudemus as attributing a
certain theorem to Oenopides etc.; but he says on \prop{1}{12} that
``Oenopides first investigated this problem, thinking it useful for
astronomy'' when he cannot have had Oenopides' work before him.

It has been said above that Proclus was in the habit of stating in his
own words the substance of the things which he borrowed. We are
prepared for this when we find him stating that he will select the
best things from ancient commentaries and ``cut short their
interminable diffuseness,'' that he will ``briefly describe''
(\greek{συντομως ἱστορῆσαι}) the other proofs of \prop{1}{20} given by
Heron and Porphyry and also the proofs of \prop{1}{25} by Menelaus and
Heron. But the best evidence is of course to be found in the passages
where he quotes works still extant, e.g.\ those of Plato, Aristotle
and Plotinus. Examination of these passages shows great divergences
from the original; even where he purports to quote textually, using
the expressions ``Plato says,'' or ``Plotinus says,'' he by no means
quotes word for word\footnote{See the passages referred to by van
  Pesch (p.~70). The most glaring case is a passage (p.~21, 10) where
  he quotes Plotinus, using the expression ``Plotinus says……''
  Comparison with Plotinus, \emph{Ennead}.\ 1.~3, 3, shows that
  \emph{very few} words are those of Plotinus himself; the rest
  represent Plotinus' views in Proclus' own language.}. In fact, he
seems to have had a positive distaste for quoting textually from other
works. He cannot conquer this even when quoting from Euclid; he says
in his note on \prop{1}{22}, ``we will follow the words of the
geometer'' but fails, nevertheless, to reproduce the text of Euclid
unchanged\footnote{Proclus, p.~330, 19 sqq.}.

We now come to the sources themselves from which Proclus drew in
writing his commentary. Three have already been disposed of,
viz.\ Heron, Porphyry and Pappus, who had all written commentaries on
the \emph{Elements}\footnote{See pp.~20 to 27 above.}. We go on to

\textbf{Eudemus}, the pupil of Aristotle, who, among other works,
wrote a history of arithmetic, a history of astronomy, and a history
of geometry.  The importance of the last mentioned work is attested by
the frequent use made of it by ancient writers. That there was no
other history of geometry written after the time of Eudemus seems to
be proved by the remark of Proclus in the course of his famous
summary: ``Those who compiled histories bring the development of this
science up to this point. \emph{Not much younger than these is
  Euclid}\footnote{Proclus, p.~68, 4–7.}….'' The loss of Eudemus'
history is one of the gravest which fate has inflicted upon us, for it
cannot be doubted that Eudemus had before him a number of the actual
works of earlier geometers, which, as before observed, seem to have
vanished completely when they were superseded by the treatises of
Euclid, Archimedes and Apoilonius. As it is, we have to be thankful
for the fragments from Eudemus which such writers as Proclus have
preserved to us.

I agree with van Pesch\footnote{\emph{De Procli fontibus}. pp.~73–75.}
that there is no sufficient reason for doubting that the work of
Eudemus was accessible to Proclus at first hand. For the later writers
Simplicius and Eutocius refer to it in terms such as leave no room for
doubt that they had it before them.  I have already quoted a passage
from Simplicius' account of the lunes of Hippocrates to the effect
that Eudemus must be considered the best authority since he lived
nearer the times\footnote{See above, p.~29.}. In the same place
Simplicius says\footnote{Simplicius, \emph{loc.\ cit.}, ed.\ Diels,
  p.~60, 17.}, ``I will set out what Eudemus says word for word
(\greek{κατὰ λέξιν λεγόμενα}), adding only a little explanation in the
shape of reference to Euclid's Elements \emph{owing to the
  memorandum-like style of Eudemus} (\greek{διὰ τὸν ὑπομνηματικὸν
  τρόπον τοῦ Εὐδήμον}) who sets out his explanations in the
abbreviated form usual with ancient writers.  Now in the second book
of the history of geometry he writes as follows\footnote{5\?}.'' It is
not possible to suppose that Simplicius would have written in this way
about the style of Eudemus if he had merely been copying certain
passages second-hand out of some other author and had not the original
work itself to refer to. In like manner, Eutocius speaks of the
paralogisms handed down in connexion with the attempts of Hippocrates
and Antiphon to square the circle\footnote{Archimedes, ed.\ Heiberg,
  vol.~\r3.\ p.~228.}, ``with which I imagine that those are
accurately acquainted who have examined (\greek{ἐπεσκεμμένους}) the
geometrical history of Eudemus and know the Ceria Aristotelica.''  How
could the contemporaries of Eutocius have \emph{examined} the work of
Eudemus unless it was still extant in his time?

The passages in which Proclus quotes Eudemus by name as his authority
are as follows:

(1)~On \prop{1}{26} he says that Eudemus in his history of geometry
referred this theorem to Thales, inasmuch as it was necessary to
Thales' method of ascertaining the distance of ships from the
shore\footnote{Proclus, p.~352, 14–18.}.

(2)~Eudemus attributed to Thales the discovery of
Eucl.\ \prop{1}{15}\footnote{\ibid~p.~299, 3.}, and

(3)~to Oenopides the problem of \prop{1}{23}\footnote{\ibid~p.~333, 5.}.

(4)~Eudemus referred the discovery of the theorem in \prop{1}{32} to
the Pythagoreans, and gave their proof of it, which Proclus
reproduces\footnote{\ibid~p.~379, 1–16.}.

(5)~On \prop{1}{44} Proclus tells us\footnote{\ibid~p.~419, 15–18.} that Eudemus says
that ``these things are ancient, being discoveries of the Pythagorean
muse, the application (\greek{παραβοή}) of areas, their exceeding
(\emph{ὑπερβολή}) and their falling short (\greek{ἔλλειψις}).'' The
next words about the appropriation of these terms (parabola, hyperbola
and ellipse) by later writers (i.e.\ Apollonius) to denote the conic
sections are of course not due to Eudemus.

Coming now to notes where Eudemus is not named by Proclus, we may
fairly conjecture, with van Pesch, that Eudemus was really the
authority for the statements (1)~that Thales first proved that a
circle is bisected by its diameter\footnote{\ibid~p.~157, 10, 11.}
(though the proof by \emph{reductio ad absurdum} which follows in
Proclus cannot be attributed to Thales\footnote{Cantor
  (\emph{Gesch.\ d.\ Math.} \r1\tsub{3}, p.~221) points out the
  connexion between the \emph{reductio ad absurdum} and the analytical
  method said to have been discovered by Plato, Proclus gives the
  proof by \emph{reductio ad absurdum} to meet an imaginary critic who
  desires a mathematical proof; possibly Thales may have been
  satisfied with the argument in the same sentence which mentions
  Thales, ``the cause of the bisection being the unswerving course of
  the straight line through the centre.''}), (2)~that ``Plato made
over to Leodamas the analytical method, by means of which \emph{it is
  recorded} (\greek{ἱστόρηται}) that the latter too made many
discoveries in geometry\footnote{Proclus, p.~211, 19–23.},'' (3)~that
the theorem of \prop{1}{5} was due to Thales, and that for equal
angles he used the more archaic expression ``similar''
angles\footnote{\ibid~p.~250, 20.}, (4)~that Oenopides first
investigated the problem of \prop{1}{12}, and that he called the
perpendicular the \emph{gnomonic} line (\greek{κατὰ
  γνώμονα})\footnote{\ibid~p.~283, 7–10.}, (5)~that the theorem that
only three sorts of polygons can fill up the space round a point,
viz.\ the equilateral triangle, the square and the regular hexagon,
was Pythagorean\footnote{\ibid~pp.~304, 11–305, 3.}. Eudemus may also
be the authority for Proclus' description of the two methods, referred
to Plato and Pythagoras respectively, of forming right-angled
triangles in whole numbers\footnote{\ibid~pp.~428, 7–429, 9. }.

We cannot attribute to Eudemus the beginning of the note on
\prop{1}{47} where Proclus says that ``if we listen to those who like
to recount ancient history, we may find some of them referring this
theorem to Pythagoras and saying that he sacrificed an ox in honour of
his discovery\footnote{\ibid~p.~426, 6–9.}.'' As such a sacrifice was
contrary to the Pythagorean tenets, and Eudemus could not have been
unaware of this, the story cannot rest on his authority. Moreover
Proclus speaks as though he were not certain of the correctness of the
tradition; indeed, so far as the story of the sacrifice is concerned,
the same thing is told of Thales in connexion with his discovery that
the angle in a semi-circle is a right angle\footnote{Diogenes
  Laertius, \r1. 24, p.~6, ed.\ Cobet.}, and Plutarch is not certain
whether the ox was sacrificed on the discovery of \prop{1}{47} or of
the problem about application of areas\footnote{Plutarch, \emph{non
    posse suaviter vivi secundum Epicurum}, \r2; \emph{Symp.} \r8,
  2.}. Plutarch's doubt suggests that he knew of no evidence for the
story beyond the vague allusion in the distich of Apollodorus
``Logisticus'' (the ``calculator'') cited by Diogenes Laertius also\footnote{Diog.\ Laert.\ \r8. 12, p.~207, ed.\ Cobet:
\begin{verse}
\greek{Ἡνίκα Πυθαγόρης τὸ περικλεές εὔρετο βράμμα,}\\
\greek{κεῖν’ ἐφ’ ὅτῳ κλεινὴν ἤγαγε Βουθυσίην.}
\end{verse}
See on this subject Tannery, \emph{La Géométrie grecque}, p.~105.};
and Proclus may have had in mind this couplet with the passages of
Plutarch.

We come now to the question of the famous historical summary given by
Proclus\footnote{Proclus, pp.~64–70.}. No one appears to maintain that
Eudemus is the author of even the early part of this summary in the
form in which Proclus gives it. It is, as is well known, divided into
two distinct parts, between which comes the remark, ``Those who
compiled histories\footnote{The plural is well explained by Tannery,
  \emph{La Géométrie grecque}, pp.~73, 74. No doubt the author of the
  summary tried to supplement Eudemus by means of any other histories
  which threw light on the subject. Thus e.g.\ the allusion (p.~64,
  21) to the Nile recalls Herodotus. Cf.\ the expression in Proclus,
  p. 64, 19, \greek{παρὰ τῶν πολλῶν ἱστόρηται}.} bring the development
of this science up to this point.  Not much younger than these is
Euclid, who put together the Elements, collecting many of the theorems
of Eudoxus, perfecting many others by Theaetetus, and bringing to
irrefragable demonstration the things which had only been somewhat
loosely proved by his predecessors.'' Since Euclid was later than
Eudemus, it is impossible that Eudemus can have written this. Yet the
style of the summary after this point does not show any such change
from that of the former portion as to suggest different
authorship. The author of the earlier portion recurs frequently to the
question of the origin of the elements of geometry in a way in which
no one would be likely to do who was not later than Euclid; and it
must be the same hand which in the second portion connects Euclid's
Elements with the work of Eudoxus and Theaetetus\footnote{Tannery,
  \emph{La Géométrie grecque}, p.~75.}.

If then the summary is the work of one author, and that author not
Eudemus, who is it likely to have been? Tannery answers that it is
Geminus\footnote{\ibid~pp.~66–75.}; but I think, with van Pesch, that
he has failed to show why it should be Geminus rather than
another. And certainly the extracts which we have from Geminus' work
suggest that the sort of topics which it dealt with was quite
different; they seem rather to have been general questions of the
content of mathematics, and even Tannery admits that historical
details could only have come incidentally into the
work\footnote{\ibid~p.~19.}.

Could the author have been Proclus himself? Circumstances which seem
to suggest this possibility are (1)~that, as already stated, the
question of the origin of the Elements is kept prominent, (2)~that
there is no mention of Democritus, whom Eudemus would not be likely to
have ignored, while a follower of Plato would be likely enough to do
him the injustice, following the example of Plato who was an opponent
of Democritus, never once mentions him, and is said to have wished to
burn all his writings\footnote{Diog.\ Laertius, \r9. 40, p.~237,
  ed.\ Cobet.}, and (3)~the allusion at the beginning to the
``inspired Aristotle'' (\greek{ὁ δαιμονιος
  Ἀριστοτέλης})\footnote{Proclus, p. 64, 8.}, though this may easily
have been inserted by Proclus in a quotation made by him from someone
else. On the other hand there are considerations which suggest that
Proclus himself was not the writer.  (1)~The style of the whole
passage is not such as to point to him as the author. (2)~If he wrote
it, it is hardly conceivable that he would have passed over in silence
the discovery of the analytical method, the invention of Plato to
which he attached so much importance\footnote{Proclus, p. 211, 19
  sqq.; the passage is quoted above, p.~36\?.}.

There is nothing improbable in the conjecture that Proclus quoted
the summary from a compendium of Eudemus' history made by some
later writer: but as yet the question has not been definitely settled.
All that is certain is that the early part of the summary must have
been made up from scattered notices found in the great work of
Eudemus.

Proclus refers to another work of Eudemus besides the history, viz.\ a
book on \emph{The Angle} (\greek{βιβλίον περὶ γωνίας})\footnote{\ibid,
  p.~125, 8.}. Tannery assumes that this must have been part of the
history, and uses this assumption to confirm his idea that the history
was arranged according to \emph{subjects}, not according to
chronological order\footnote{Tannery, \emph{La Géométrie gecque},
  p.~26.}. The phraseology of Proclus however unmistakably suggests a
separate work; and that the history was \emph{chronologically}
arranged seems to be clearly indicated by the remark of Simplicius
that Eudemus ``also counted Hippocrates among the more ancient
writers'' (\greek{ἐν τοῖς παλαιοτέροις})\footnote{Simplicius,
  ed.\ Diels, p.~69, 23.}.

The passage of Simplicius about the lunes of Hippocrates throws
considerable light on the style of Eudemus' history. Eudemus wrote in
a memorandum-like or summary manner (\greek{τὸν ὑπομνηματικὸν τρόπον
  τοῦ Εὐδήμου})\footnote{\ibid~p.~60, 29,} when reproducing what he
found in the ancient writers; sometimes it is clear that he left out
altogether proofs or constructions of things by no means
easy\footnote{Cf.\ Simplicius, p.~63, 19 sqq.; p.~64. 25 sqq.; also
  Usener's note ``de supplendis Hippocratis quas omisit Eudemus
  constructionibus'' added to Diels' preface, pp. xxiii—xxvi.}.

\subsection*{Geminus}

The discussions about the date and birthplace of Geminus form a whole
literature, as to which I must refer the reader to Manitius and
Tittel\footnote{Manitius, \emph{Gemini elementa astronomiae} (Teubner,
  1898), pp. 237- — 151; Tittel, art.\ ``Geminos'' in Pauly-Wissowa's
  \emph{Real-Encyclopädie der classischen Altertumswissenschaft},
  vol.~\r7. 1910.}, Though the name looks like a Latin name (Gemǐnus),
Manitius concluded that, since it appears as \greek{Γεμῖνος} in all
Greek \textsc{mss.}\ and as \greek{Γεμεῖνος} in some inscriptions, it
is Greek and possibly formed from \greek{γεμ} as \greek{Ἐργῖνος} is
from \greek{ἐργ} and \greek{Ἀλεξῖνος} from \greek{ἀλεξ} (cf.\ also
\greek{Ἰκτῖνος}, \greek{Κρατῖνος}). Tittel is equally positive that it
is Gemǐnus and suggests that \greek{Γεμῖνος} is due to a false analogy
with \greek{Ἀλεξῖνος} etc.\ and \greek{Γεμεῖνος} wrongly formed on the
model of \greek{Ἀντωνεῖνος}, \greek{Ἀγριππεῖνα}.  Geminus, a Stoic
philosopher, born probably in the island of Rhodes, was the author of
a comprehensive work on the classification of mathematics, and also
wrote, about 73–67~\bc, a not !ess comprehensive commentary on the
meteorological textbook of his teacher Posidonius of Rhodes.

It is the former work in which we are specially interested here.
Though Proclus made great use of it, he does not mention its title,
unless we may suppose that, in the passage (p.~177, 24) where, after
quoting from Geminus a classification of lines which never meet, he
says, ``these remarks I have selected from the \greek{φιλοκαλία} of
Geminus,'' \greek{φιλοκαλία} is a title or an alternative
title. Pappus however quotes a work of Geminus ``on the classification
of the mathematics'' (\greek{ἐν τῷ περὶ τῆς τῶν μαθημάτων
  τάξεως})\footnote{Pappus, ed,\ Hultsch, p.~1026, 9.}, while Eutocius
quotes from ``the sixth book of the doctrine of the mathematics''
(\greek{ἐν τῷ ἕκτῳ τῆς τῶν μαθημάτων θεωρίας})\footnote{Apollonius,
  ed.\ Heiberg, vol.~\r2. p.~170.}.  Tannery\footnote{Tannery,
  \emph{La Géométrie gecque}, pp.~18, 19.} pointed out that the former
title corresponds well enough to the long extract\footnote{Proclus,
  pp.~38, 1–42, 8.} which Proclus gives in his first prologue, and
also to the fragments contained in the \emph{Anonymi variae
  collectiones} published by Hultsch at the end of his edition of
Heron\footnote{Heron, ed.\ Hultsch, pp.~246, 16–249, 12.}; but it does
not suit most of the other passages borrowed by Proclus, The correct
title was therefore probably that given by Eutocius, \emph{The
  Doctrine}, or \emph{Theory}, \emph{of the Mathematics}; and Pappus
probably refers to one particular portion of the work, say the first
Book. If the sixth Book treated of conics, as we may conclude from
Eutocius, there must have been more Books to follow, because Proclus
has preserved us details about higher curves, which must have come
later. If again Geminus finished his work and wrote with the same
fulness about the other branches of mathematics as he did about
geometry, there must have been a considerable number of Books
altogether. At all events it seems to have been designed to give a
complete view of the whole science of mathematics, and in fact to be a
sort of encyclopaedia of the subject.

I shall now indicate first the certain, and secondly the probable,
obligations of Proclus to Geminus, in which task I have only to follow
van Pesch, who has embodied the results of Tittel's similar inquiry
also\footnote{Van Pesch, \emph{De Procli fontibus}, pp.~97–113.  The
  dissertation of Tittel is entitled \emph{De Gemini Stoici studiis
    mathematicis} (1895).}.  I shall only omit the passages as regards
which a case for attributing them to Geminus does not seem to me to
have been made out.

First come the following passages which must be attributed to
Geminus, because Proclus mentions his name:

(1)~(In the first prologue of Proclus\footnote{Proclus, pp.~38, 1–42,
  8, except the allusion in p.~41, 8–10, to Ctesibius and Heron and
  their pneumatic devices (\greek{θαυματοποιϊκή}), as regards which
  Proclus' authority may be Pappus (\r8.~p.~1024, 14–27) who uses very
  similar expressions.  Heron, even if not later than Geminus, could
  hardly have been included in a historical work by him.  Perhaps
  Geminus may have referred to Ctesibius only, and Proclus may have
  inserted ``and Heron'' himself.}) on the division of mathematical
sciences into arithmetic, geometry, mechanics, astronomy, optics,
geodesy, canonic (science of musical harmony), and logistic
(apparently arithmetical problems);

(2)~(in the note on the definition of a straight line) on the
classification of lines (including curves) as simple (straight or
circular) and mixed, composite and incomposite, uniform
(\greek{ὁμοιομερεῖς}) and non-uniform (\greek{ἀνομοιομερεῖς}), lines
``about solids'' and lines produced by cutting solids, including conic
and spiric sections\footnote{Proclus, pp.~103, 21–107, 10; pp.~111,
  1–113, 3.};

(3)~(in the note on the definition of a plane surface) on similar
distinctions extended to surfaces and solids\footnote{\ibid~pp.~117,
  14–120, 12, where perhaps in the passage pp.~117, 22–118, 23, where
  perhaps in the passage pp.~117, 22–118, 23 we may have Geminus' own
  words.};

(4)~(in the note on the definition of parallels) on lines which
\emph{do not meet} (\greek{ἀσύμπτωτοι}) but which are not on that
account parallel, e.g.\ a curve and its asymptote, showing that the
property of \emph{not meeting} does not make lines parallel—a
favourite observation of Geminus—and, incidentally, on \emph{bounded}
lines or those which \emph{enclose a figure} and those which do
not\footnote{\ibid~pp.~176, 18–177, 25; perhaps also p.~175.  The note
  ends with the words ``These things too we have selecled from
  Geminus' \greek{φιλοκαλία} for the elucidation of the matters in
  question.''  Tannery (p.~27) takes these words coming at the end of
  the commentary on the definitions as referring to the whole of the
  portion of the commentary dealing with the definitions.  Van Pesch
  properly regards them as only applying to the note on
  \emph{parallels}. This seems to me clear from the use of the word
  \emph{too} (\greek{τοσαῦτα καί}).};

(5)~(in the same note) the definition of parallels given by
Posidonius\footnote{Proclus, p.~176, 5–17.};

(6)~on the distinction between postulates and axioms, the futility of
trying to prove axioms, as Apollonius tried to prove Axiom~1, and the
equal incorrectness of assuming what really requires proof, ``as
Euclid did in the fourth postulate [equality of right angles] and in
the fifth postulate [the
  parallel-postulate]\footnote{\ibid~pp.~178–182, 4; pp.~183, 14–184,
  10; cf.\ p.~188, 3–11.}'';

(7)~on Postulates 1, 2, 3, which Geminus makes depend on the idea of a
straight line being described by the motion of a
point\footnote{\ibid~p.~185, 6–25.};

(8)~(in the note on Postulate~5) on the inadmissibility in geometry of
an argument which is merely plausible, and the danger in this
particular case owing to the existence of lines which do converge ad
infinitum and yet never meet\footnote{\ibid~p.~192, 5–29.};

(9)~(in the note on \prop{1}{1}) on the subject-matter of geometry,
theorems, problems and \greek{διορισμοί} (conditions of possibility)
for problems\footnote{\ibid~pp.~200, 21–202, 25.};

(10)~(in the note on \prop{1}{5}) on a generalisation of \prop{1}{5}
by Geminus through the substitution for the rectilineal base of ``one
uniform line (curve),'' by means of which he proved that the only
``uniform lines'' (alike in all their parts) are a straight line, a
circle, and a cylindrical helix\footnote{Proclus, p.~251, 2–11.};

(11)~(in the note on \prop{1}{10}) on the question whether a line is
made up of indivisible parts (\greek{ἀμερῆ}), as affecting the problem
of bisecting a given straight line\footnote{\ibid~pp.~277, 25–279,
  11.};

(12) (in the note on \prop{1}{35}) on \emph{topical}, or
\emph{locus}-theorems\footnote{\ibid~pp.~394, 11–395, 2 and p.~395,
  13–21}, where the illustration of the equal parallelograms described
between a hyperbola and its asymptotes may also be due to
Geminus\footnote{\ibid~p.~395, 8–12.}.

Other passages which may fairly be attributed to Gem in us, though his
name is not mentioned, are the following:

(1)~in the prologue, where there is the same allusion as in the
passage~(8) above to a remark of Aristotle that it is equally absurd
to expect scientific proofs from a rhetorician and to accept mere
plausibilities from a geometer\footnote{\ibid~pp.~33. 21–34, 1.};

(2)~a passage in the prologue about the subject-matter, methods, and
bases of geometry, the latter including axioms and
postulates\footnote{\ibid~pp.~57, 9–58, 3.};

(3)~another on the definition and nature of
\emph{elements}\footnote{\ibid~pp.~72, 3–75, 4. };

(4)~a remark on the Stoic use of the term axiom for every simple
statement (\greek{ἀπόφανσις ἁπλῆ})\footnote{\ibid~p.~77, 3–6.};

(5)~another discussion on theorems and problems\footnote{\ibid~pp.~77,
  7–78, 13, and 79, 3–81, 4.}, in the middle of which however there
are some sentences by Proclus himself\footnote{\ibid~pp.~78, 13–79,
  2.}.

(6)~another passage, in connexion with Def.~3, on lines including or
not including a figure (with which cf.\ part of the passage~(4)
above)\footnote{\ibid~pp.~102, 22–103, 18.};

(7)~a classification of different sorts of angles according as they
are contained by simple or mixed lines (or
curves)\footnote{\ibid~pp.~126, 7–127, 16.};

(8)~a similar classification of figures\footnote{\ibid~pp.~159,
  12–160, 9.}, and of plane figures\footnote{\ibid~pp.~162, 27–164,
  6.};

(9)~Posidonius' definition of a \emph{figure}\footnote{\ibid~p.~143, 5–11. };

(10)~a classification of triangles into seven
kinds\footnote{\ibid~p.~168, 4–12.};

(11)~a note distinguishing lines (or curves) producible indefinitely
or not so producible, whether forming a figure or not forming a figure
(like the ``single-turn spiral'')\footnote{\ibid~p.~187, 19–27.};

(12)~passages distinguishing different sorts of
problems\footnote{\ibid~pp.~220, 7–222, 14; also p.~330, 6–9.},
different sorts of theorems\footnote{\ibid~pp.~244, 14–246, 12.}, and
two sorts of converses (complete and partial)\footnote{\ibid~pp.~252,
  5–254, 20.};

(13)~the definition of the term ``porism ``as used in the title of
Euclid's \emph{Porisms}, as distinct from the other meaning of
``corollary''\footnote{\ibid~pp.~301, 21–302, 13.};

(14)~a note on the Epicurean objection to \prop{1}{20} as being
obvious even to an ass\footnote{\ibid~pp.~322, 4–323. 3.};

(15)~a passage on the properties of parallels, with allusions to
Apollonius' \emph{Conics}, and the curves invented by Nicomedes,
Hippias and Perseus\footnote{Proclus, pp.~355, 20–356, 16. };

(16)~a passage on the parallel-postulate regarded as the converse of
\prop{1}{17}\footnote{\ibid~p.~364, 9–11; pp.~364, 20–365, 4.}.

Of the authors to whom Proclus was indebted in a less degree the most
important is \textbf{Apollonius of Perga}. Two passages allude to his
\emph{Conics}\footnote{\ibid~p.~71, 19; p.~356, 8, 6.}, one to a work
on irrationals\footnote{\ibid~p.~74, 23, 24.} and two to a treatise
\emph{On the cochlias} (apparently the cylindrical helix) by
Apollonius\footnote{\ibid~pp.~105, 5, 6, 14, 15. }. But more important
for our purpose are six references to Apollonius in connexion with
elementary geometry,

(1)~He appears as the author of an attempt to explain the idea of a
line (possessing length but no breadth) by reference to daily
experience, e.g.\ when we tell someone to measure, merely, the length
of a road or of a wall\footnote{\ibid~p.~100, 5–19.}; and doubtless
the similar passage showing how we may in like manner get a notion of
a surface (without depth) is his also\footnote{\ibid~p.~114, 20–25}.

(2)~He gave a new general definition of an
angle\footnote{\ibid~p.~123, 15–19 (cf.\ p.~124, 17. p.~125, 17).}.

(3)~He tried to prove certain axioms\footnote{\ibid~p.~183, 13, 14.},
and Proclus gives his attempt to prove Axiom~1, word for
word\footnote{\ibid~pp.~194, 25–195, 5.}.

Proclus further quotes:

(4)~Apollonius' solution of the problem in Eucl.\ \prop{1}{10},
avoiding Euclid's use of \prop{1}{9}\footnote{\ibid~pp.279, 16–280,
  4.}.

(5)~his solution of the problem in \prop{1}{11}, differing only
slightly from Euclid's\footnote{\ibid~p.~282, 8–19.}, and

(6)~his solution of the problem in
\prop{1}{23}\footnote{\ibid~pp.~335, 16–336, 5.}.

Heiberg\footnote{\emph{Philologus}, vol. \r43. p.~489.} conjectures
that Apollonius departed from Euclid's method in these propositions
because he objected to solving problems of a more general, by means of
problems of a more particular, character. Proclus however considers
all three solutions inferior to Euclid's; and his remarks on
Apollonius' handling of these elementary matters generally suggest
that he was nettled by criticisms of Euclid in the work containing the
things which he quotes from Apollonius, just as we conclude that
Pappus was offended by the remarks of Apollonius about Euclid's
incomplete treatment of the ``three- and four-line locus\footnote{See
  above, pp.~2, 3.}.'' If this was the case, Proclus can hardly have
got his information about these things at second-hand; and there seems
to be no reason to doubt that he had the actual work of Apollonius
before him. This work may have been the treatise mentioned by Marinus
in the words ``Apollonius in his general treatise'' (\greek{Ἀπολλώνιος
  ἐν τῇ καθόλου πραγματείᾳ})\footnote{\emph{Marinus in Euclidis Data},
  ed.\ Menge, p.~234, 16.}.  If the notice in the
\emph{Fihrist}\footnote{\emph{Fihrist}, tr.\ Suter, p.~19.}  stating,
on the authority of Thābit b.~Qurra, that Apollonius wrote a tract on
the parallel-postulate be correct, it may have been included in the
same work. We may conclude generally that, in it, Apollonius tried to
remodel the beginnings of geometry, reducing the number of axioms,
appealing, in his definitions of lines, surfaces etc., more to
experience than to abstract reason, and substituting for certain
proofs others of a more general character.

The probabilities are that, in quoting from the tract of Ptolemy in
which he tried to prove the paralel-postulate, Proclus had the actual
work before him. For, after an allusion to it as ``a certain
book\footnote{Proclus, p.~191, 23.}'' he gives two long
extracts\footnote{\ibid~pp.~362, 14–363, 19; pp.~365, 7–367, 27.}, and
at the beginning of the second indicates the title of the tract, ``in
the (book) about the meeting of straight lines produced from (angles)
less than two right angles,'' as he has very rarely done in other
cases.

Certain things from \textbf{Posidonius} are evidently quoted at
second-hand, the authority being Geminus (e.g.\ the definitions of
\emph{figure} and \emph{parallels}); but besides these we have
quotations from a separate work which he wrote to controvert Zeno of
Sidon, an Epicurean who had sought to destroy the whole of
geometry\footnote{\ibid~p.~200, 1–3.}. We are told that Zeno had
argued that, even if we admit the fundamental principles
(\greek{ἀρχαί}) of geometry, the deductions from them cannot be proved
without the admission of something else as well, which has not been
included in the said principles\footnote{\ibid~pp.~199, 11—200,
  1.}. On \prop{1}{1} Proclus gives at some length the arguments of
Zeno and the reply of Posidonius as regards this
proposition\footnote{\ibid~pp.~214, 18–215, 13; pp.~216, 10–218, 11.}.
In this case Zeno's ``something else'' which he considers to be
assumed is the fact that two straight lines cannot have a common
segment, and then, as regards the ``proof'' of it by means of the
bisection of a circle by its diameter, he objects that it has been
assumed that two \emph{circumferences} (arcs) of circles cannot have a
common part. Lastly, he makes up, for the purpose of attacking it,
another supposed ``proof'' of the fact that two straight lines cannot
have a common part. Proclus appears, more than once, to be quoting the
actual words of Zeno and Posidonius; in particular, two expressions
used by Posidonius about ``the acrid Epicurean'' (\greek{τὸν δριμὺν
  Ἐπικούρειον})\footnote{\ibid~p.~216, 21.} and his
``misrepresentations'' (\greek{Ποσειδώνιός φησι τὸν Ζήνωνα
  συκοφαντεῖν})\footnote{\ibid~p.218, 1.}. It is not necessary to
suppose that Proclus had the original work of Zeno before him, because
Zeno's arguments may easily have been got from Posidonius' reply; but
he would appear to have quoted direct from the latter at all events.

The work of Carpus \emph{mechanicus} (a treatise on astronomy) quoted
from by Proclus\footnote{\ibid~pp.~241, 19–243, 11.} must have been
accessible to him at first-hand, because a portion of the extract from
it about the relation of theorems and problems\footnote{\ibid~pp.~242,
  22—243, 11.} is reproduced word for word. Moreover, if he were not
using the book itself, Proclus would hardly be in a position to
question whether the introduction of the subject of theorems and
problems was opportune in the place where it was found (\greek{εἰ μὲν
  κατὰ καιρὸν ἤ μή, παρείσθω πρὸς τὸ παρόν})\footnote{Proclus, p.~241,
  21, 22.}.

It is of course evident that Proclus had before him the original works
of Plato, Aristotle, Archimedes and Plotinus, as well as the
\greek{Συμμικτά} of Porphyry and the works of his master Syrianus
(\greek{ὁ ἡμέτερος καθηγεμόν})\footnote{\ibid~p.~123, 19,}, from whom
he quotes in his note on the definition of an angle. Tannery also
points out that he must have had before him a group of works
representing the Pythagorean tradition on its mystic, as distinct from
its mathematical, side, from Philolaus downwards, and comprising the
more or less apocryphal \greek{ἱερός λόγος} of Pythagoras, the Oracles
(\greek{λόγια}), and Orphic verses\footnote{Tannery, \emph{La
    Géométrie grecque}, pp.~25, 26.}.

Besides quotations from writers whom we can identify with more or less
certainty, there are many other passages which are doubtless quoted
from other commentators whose names we do not know. A list of such
passages is given by van Pesch\footnote{Van Pesch, \emph{De Procli
    fontibus}, p.~139.}, and there is no need to cite them here.

Van Pesch also gives at the end of his work\footnote{\ibid~p.~155.} a
convenient list of the books which, as the result of his
investigation, he deems to have been accessible to and directly used
by Proclus. The list is worth giving here, on the same ground of
convenience. It is as follows:

Eudemus: \emph{history of geometry}.

Geminus: \emph{the theory of the mathematical sciences}.

Heron: \emph{commentary on the Elements of Euclid}.

Porphyry: \emph{commentary on the Elements of Euclid}.

Pappus: \emph{commentary on the Elements of Euclid}.

Apollonius of Perga: a work relating to elementary geometry.

Ptolemy: \emph{on the parallel-postulate}.

Posidonius: a book controverting Zeno of Sidon.

Carpus: \emph{astronomy}.

Syrianus: a discussion on the \emph{angle}.

Pythagorean philosophical tradition.

Plato's works.

Aristotle's works.

Archimedes' works,

Plotinus: \emph{Enneades}.

Lastly we come to the question what passages, if any, in the
commentary of Proclus represent his own contributions to the subject.
As we have seen, the \emph{onus probandi} must be held to rest upon
him who shall maintain that a particular note is original on the part
of Proclus, Hence it is not enough that it should be impossible to
point to another writer as the probable source of a note; we must have
a positive reason for attributing it to Proclus, The criterion must
therefore be found either (1)~in the general terms in which Proclus
points out the deficiencies in previous commentaries and indicates the
respects in which his own will differ from them, or (2)~in specific
expressions used by him in introducing particular notes which may
indicate that he is giving his own views. Besides indicating that he
paid more attention than his predecessors to questions requiring
deeper study (\greek{τὸ πραγματειῶδες}) and ``pursued clear
distinctions'' (\greek{τὸ εὐδιαίρετον
  μεταδιώκοντας})\footnote{Proclus, p.~84, 13, p.~432, 14, 15.}—by
which he appears to imply that his predecessors had confused the
different departments of their commentaries, viz.\ lemmas, cases, and
objections (\greek{ἐνστάσεις})\footnote{cf.\ \ibid~p.~289, 11–15;
  p.~432, 15–17.}—Proclus complains that the earlier commentators had
failed to indicate the ultimate grounds or \emph{causes} of
propositions\footnote{\ibid~p.~432, 17.}. Although it is from Geminus
that he borrowed a passage maintaining that it is one of the proper
functions of geometry to inquire into causes (\greek{τὴν αἰτίαν καὶ τὸ
  διὰ τί})\footnote{\ibid~p.~202, 9–25.}, yet it is not likely that
Geminus dealt with Euclid's propositions one by one; and consequently,
when we find Proclus, on \prop{1}{8, 16, 17, 18, 32, and
  47}\footnote{See Proclus, p.~270, 5–14 (\prop{1}{8}); pp.~309,
  3–310, 8 (\prop{1}{16}); pp.~310, 19–311, 23 (\prop{1}{17}), p.~316,
  14–318, 2 (\prop{1}{18}); p.~384, 13–21 (\prop{1}{32}); pp.~426,
  22–427, 8 (\prop{1}{47}).}, endeavouring to explain \emph{causes},
we have good reason to suppose that the explanations are his own.

Again, his remarks on certain things which he quotes from Pappus
can scarcely be due to anyone else, since Pappus is the latest of the
commentators whose works he appears to have used. Under this
head, come

(1)~his objections to certain new axioms introduced by
Pappus\footnote{\ibid~p.~198, 5–15.},

(2)~his conjecture as to how Pappus came to think of his alternative
proof of \prop{1}{5}\footnote{\ibid~p.~250, 12–19.},

(3)~an addition to Pappus' remarks about the curvilineal angle which
is equal to a right angle without being one\footnote{\ibid~.p~190,
  9–23.}.

The defence of Geminus against Carpus, who combated his view of
theorems and problems, is also probably due to
Proclus\footnote{\ibid~p.~243, 12–29..}, as well as an observation on
\prop{1}{38} to the effect that \prop{1}{35–38} are really
comprehended in \prop{6}{1} as particular
cases\footnote{\ibid~pp.~405, 6–406, 9.}.

Lastly, we can have no hesitation in attributing to Proclus himself
(1)~the criticism of Ptolemy's attempt to prove the
parallel-postulate\footnote{\ibid~p.~368, 1–23.}, and (2)~the other
attempted proof given in the same note\footnote{\ibid~pp.~371, 11–373,
  2.} (on \prop{1}{29}) and assuming as an axiom that ``if from one
point two straight lines forming an angle be produced \emph{ad
  infinitum} the distance between them when so produced \emph{ad
  infinitum} exceeds any finite magnitude (i.e.\ length),'' an
assumption which purports to be the equivalent of a statement in
Aristotle\footnote{Aristotle, \emph{de caelo}, \prop{1}{5}
  (271~b~28–30).}.  It is introduced by words in which the writer
appears to claim originality for his proof: ``To him who desires to
see this proved (\greek{κατασκευαζόμενον}) \emph{let it be said by us}
(\greek{λεγέσθω παρ’ ἤμῶν})'' etc.\footnote{Proclus, p..~371, 10.}
Moreover, Philoponus, in a note on Aristotle's
\emph{Anal.\ post.}\ \r1.~10, says that ``the geometer (Euclid)
assumes this as an axiom, but it wants a great deal of proof, insomuch
that both Ptolemy and Proclus wrote a whole book upon
it\footnote{Berlin Aristotle, vol.~\r4. p.~214~a~9–12.}.''

\chapter{The Text\protect\footnote{The material for the whole of this chapter is taken from Heiberg's
  edition of the \emph{Elements}, introduction to vol.~\r5., and from
  the same scholar's \emph{Litterargeschichtliche Studien über
    Euklid}, p.~174 sqqf.\ and \emph{Paralipomena zu Euklid} in
  \emph{Hermes}, \r38., 2903}.}

It is well known that the title of Simson's edition of Euclid (first
brought out in Latin and English in 1756) claims that, in it, ``the
errors by which Theon, or others, have long ago vitiated these books
are corrected, and some of Euclid's demonstrations are restored''; and
readers of Simson's notes are familiar with the phrases used, where
anything in the text does not seem to him satisfactory, to the effect
that the demonstration has been spoiled, or things have been
interpolated or omitted, by Theon ``or some other unskilful editor.''
Now most of the \textsc{mss.}\ of the Greek text prove by their titles
that they proceed from the recension of the \emph{Elements} by Theon;
they purport to be either ``from the edition of Theon'' (\greek{ἐκ τῆς
  Θέωνος}) or ``from the lectures of Theon'' (\greek{ἀπὸ συνουσιῶν τοῦ
  Θέωνος}). This was Theon of Alexandria (4th c.~\ad) who also wrote a
commentary on Ptolemy, in which there occurs a passage of the greatest
importance in this connexion\footnote{\r1. p.~201 ed.\ Halma = p.~50
  ed.~Basel.}: ``But that sectors in equal circles are to one another
as the angles on which they stand \emph{has been proved by me in my
  edition of the Elements at the end of the sixth book}.'' Thus Theon
himself says that he edited the \emph{Elements} and also that the
second part of \prop{6}{33}, found in nearly all the \textsc{mss.}, is
his addition.

This passage is the key to the whole question of Theon's changes in
the text of Euclid; for, when Peyrard found in the Vatican the
\textsc{ms.}~190 which contained neither the words from the titles of
the other \textsc{mss.}\ quoted above nor the interpolated second part
of \prop{6}{33}, he was justified in concluding, as he did, that in
the Vatican \textsc{ms.}\ we have an edition more ancient than
Theon's. It is also clear that the copyist of~P, or rather of its
archetype, had before him the two recensions and systematically gave
the preference to the earlier one; for at \prop{13}{6} in~P the first
hand has added a note in the margin: ``This theorem is not given in
most copies of the \emph{new edition}, but is found in those of the
old.''  Thus we are more fortunate than Simson, since our judgment of
Theon's recension can be formed on the basis, not of mere conjecture,
but of the documentary evidence afforded by a comparison of the
Vatican \textsc{ms.}\ just mentioned with what we may conveniently
call, after Heiberg, the Theonine \textsc{mss.}

The \textsc{mss.}\ used for Heiberg's edition of the \emph{Elements}
are the following:

(1)~P = Vatican \textsc{ms.}\ numbered 190, 4to, in two volumes
(doubtless one originally); 10th~c.

This is the \textsc{ms.}\ which Peyrard was able to use; it was sent
from Rome to Paris for his use and bears the stamp of the Paris
Imperial Library on the last page. It is well and carefully written.
There are corrections some of which are by the original hand, but
generally in paler ink, others, still pretty old, by several different
hands, or by one hand with different ink in different places (P m.~2),
and others again by the latest hand (P m.~rec.). It contains, first,
the \emph{Elements} \r1.–\r13.\ with scholia, then Marinus' commentary
on the Data (without the name of the author), followed by the
\emph{Data} itself and scholia, then the \emph{Elements} \r14.,
\r15.\ (so called), and lastly three books and a part of a fourth of a
commentary by Theon \greek{εἰς τοὺς προχείρους κανόνας Πτολεμαίου}.

The other \textsc{mss.}\ are ``Theonine.''

(2)~F = \textsc{ms.}\ \r38,~3, in the Laurentian Library at Florence,
4to; 10th~c.

This \textsc{ms.}\ is written in a beautiful and scholarly hand and
contains the \emph{Elements} \r1.–\r15., the \emph{Optics} and the
\emph{Phaenomena}, but is not well preserved.  Not only is the
original writing renewed in many places, where it had become faint, by
a later hand of the 16th~c., but the same hand has filled certain
smaller lacunae by gumming on to torn pages new pieces of parchment,
and has replaced bodily certain portions of the \textsc{ms.}, which
had doubtless become illegible, by fresh leaves. The larger gaps so
made good extend from Eucl.\ \prop{7}{12} to \prop{9}{15}, and from
\prop{12}{3} to the end; so that, besides the conclusion of the
\emph{Elements}, the \emph{Optics} and \emph{Phaenomena} are also in
the later hand, and we cannot even tell what in addition to the
\emph{Elements} \r1.–\r13.\ the original
\textsc{ms.}\ contained. Heiberg denotes the later hand by \greek{φ}
and observes that, while in restoring words which had become faint and
filling up minor lacunae the writer used no other \textsc{ms.}, yet in
the two larger restorations he used the Laurentian
\textsc{ms.}\ \r28,~6, belonging to the 13th–14th~c. The latter
\textsc{ms.}\ (which Heiberg denotes by~f) was copied from the
Viennese \textsc{ms.}~(V) to be described below.

(3)~B = Bodleian \textsc{ms.}, D'Orville \r10. 1 inf.~2, 30, 4to;
\ad~888.  This \textsc{ms.}\ contains the \emph{Elements}
\r1.–\r15.\ with many scholia. Leaves 15–118 contain \prop{1}{14}
(from about the middle of the proposition) to the end of Book~\r6.,
and leaves 123–387 (wrongly numbered~397) Books \r7.–\r15, in one and
the same elegant hand (9th~c.). The leaves preceding leaf~15 seem to
have been lost at some time, leaves 6 to~14 (containing Elem.\ \r1. to
the place in \prop{1}{14} above referred to) being carelessly written
by a later hand on thick and common parchment (13th~c.). On leaves 2
to~4 and~122 are certain notes in the hand of Arethas, who also wrote
a two-line epigram on leaf~5, the greater part of the scholia in
uncial letters, a few notes and corrections, and two sentences on the
last leaf, the first of which states that the \textsc{ms.}\ was
written by one Stephen \emph{clericus} in the year of the world 6397
(= 888~\ad), while the second records Arethas' own acquisition of it.
Arethas lived from, say, 865 to 939~\ad.  He was Archbishop of
Caesarea and wrote a commentary on the Apocalypse. The portions of his
library which survive are of the greatest interest to palaeography on
account of his exact notes of dates, names of copyists, prices of
parchment etc.  It is to him also that we owe the famous Plato
\textsc{ms.}\ from Patmos (Cod.\ Clarkianus) which was written for him
in November~895\footnote{See Pauly-Wissowa, \emph{Real-Encyclopädie
    der class.\ Altertumswissenschaft}, vol.~\r2, 1896,}.

(4) V - Viennese \textsc{ms.}\ Philos.\ Gr. No.~103; probably 12th~c.

This \textsc{ms.}\ contains 292 leaves, Eucl.\ \emph{Elements}
\r1.–\r15.\ occupying leaves 1 to~254, after which come the
\emph{Optics} (to leaf~71), the \emph{Phaenomena} (mutilated at the
end) from leaf~272 to leaf~282, and lastly scholia, on leaves 283
to~292, also imperfect at the end. The different material used for
different parts and the varieties of handwriting make it necessary for
Heiberg to discuss this \textsc{ms.}\ at some length\footnote{Heiberg,
  vol.~\r5. pp.~xxix—xxxiii.}. The handwriting on leaves 1 to~183
(Book~\r1.\ to the middle of \prop{10}{105}) and on leaves 203 to~234
(from \prop{11}{31}, towards the end of the proposition, to
\prop{13}{7}, a few lines down) is the same; between leaves 184
and~202 there are two varieties of handwriting, that of leaves 184
to~189 and that of leaves 200 (verso) to~202 being the same. Leaf~235
begins in the same handwriting, changes first gradually into that of
leaves 184 to~189 and then (verso) into a third more rapid cursive
writing which is the same as that of the greater part of the scholia,
and also as that of leaves 243 and~282, although, as these leaves are
of different material, the look of the writing and of the ink seems
altered.  There are corrections both by the first and a second hand,
and scholia by many hands.  On the whole, in spite of the apparent
diversity of handwriting in the \textsc{ms.}, it is probable that the
whole of it was written at about the same time, and it may (allowing
for changes of material, ink etc) even have been written by the same
man.  It is at least certain that, when the Laurentian
\textsc{ms.}\ \r28,~6 was copied from it, the whole \textsc{ms.}\ was
in the condition in which it is now, except as regards the later
scholia and leaves 283 to~292 which are not in the Laurentian
\textsc{ms.}, that \textsc{ms.}\ coming to an end where the
\emph{Phaenomena} breaks off abruptly in~V\@.  Hence Heiberg
attributes the whole \textsc{ms.}\ to the 12th~c.

But it was apparently in two volumes originally, the first consisting
of leaves 1 to~183; and it is certain that it was not all copied at
the same time or from one and the same original. For leaves 184 to~202
were evidently copied from two \textsc{mss.}\ different both from one
another and from that from which the rest was copied. Leaves 184 to
the middle of leaf~189 (recto) must have been copied from a
\textsc{ms.}\ similar to~P, as is proved by similarity of readings,
though not from~P itself. The rest, up to leaf~202, were copied from
the Bologna \textsc{ms.}~(b) to be mentioned below. It seems clear
that the content of leaves 184 to~202 was supplied from other
\textsc{mss.}\ because there was a lacuna in the original from which
the rest of~V was copied.

Heiberg sums up his conclusions thus. The copyist of~V first copied
leaves 1 to~183 from an original in which two \emph{quaterntones} were
missing (covering from the middle of Eucl.\ \prop{10}{105} to near the
end of \prop{11}{31}). Noticing the lacuna he put aside one
\emph{quaternio} of the parchment used up to that point. Then he
copied onwards from the end of the lacuna in the original to the end
of the \emph{Phaenomena}.  After this he looked about him for another
\textsc{ms.}\ from which to fill up the lacuna; finding one, he copied
from it as far as the middle of leaf 189 (recto). Then, noticing that
the \textsc{ms.}\ from which he was copying was of a different class,
he had recourse to yet another \textsc{ms.}\ from which he copied up
to leaf~202, At the same time, finding that the lacuna was longer than
he had reckoned for, he had to use twelve more leaves of a different
parchment in addition to the quatemio which he had put aside. The
whole \textsc{ms.}\ at first formed two volumes (the first containing
leaves 1 to~183 and the second leaves 184 to~282); then, after the
last leaf had perished, the two volumes were made into one to which
two more \emph{quaterniones} were also added. A few leaves of the
latter of these two have since perished,

(5)~b = \textsc{ms.}\ numbered 18–19 in the Communal Library at
Bologna, in two volumes, 4to; 11th~c.

This \textsc{ms.}\ has scholia in the margin written both by the first
hand and by two or three later hands; some are written by the latest
hand, Theodorus Cabasilas (a descendant apparently of Nicolaus
Cabasilas, 14th~c.) who owned the \textsc{ms.}\ at one time. It
contains (\emph{a}) in 14 \emph{quaterniones} the definitions and the
enunciations (without proofs) of the \emph{Elements} \r1.–\r13, and of
the \emph{Data}, (\emph{b})~in the remainder of the volumes the
\emph{Proem to Geometry} (published among the \emph{Variae
  Collectiones} in Hultsch's edition of Heron, pp.~252, 24 to~274, 14)
followed by the \emph{Elements} \r1.–\r13.\ (part of \prop{13}{18} to
the end being missing), and then by part of the \emph{Data} (from the
last three words of the enunciation of Prop.~38 to the end of the
penultimate clause in Prop.~87, ed.\ Menge). From \prop{11}{36}
inclusive to the end of \r12.\ this \textsc{ms.}\ appears to represent
an entirely different recension. Heiberg is compelled to give this
portion of~b separately in an appendix. He conjectures that it is due
to a Byzantine mathematician who thought Euclid's proofs too long and
tiresome and consequently contented himself with indicating the course
followed\footnote{\emph{Zeitschrift für Math.\ u.\ Physik}, \r29.,
  hist.-litt.\ Abtheilung, p.~13.}.  At the same time this Byzantine
must have had an excellent \textsc{ms.}\ before him, probably of the
ante-Theonine variety of which the Vatican \textsc{ms.}~190 (P) is the
sole representative.

(6)~p = Paris \textsc{ms.}\ 2466, 4to; 12th~c.

This manuscript is written in two hands, the finer hand occupying
leaves 1 to~53 (recto), and a more careless hand leaves 53 (verso)
to~64, which are of the same parchment as the earlier leaves, and
leaves 65 to~239, which are of a thinner and rougher parchment showing
traces of writing of the 8th–9th~c.\ (a Greek version of the Old
Testament). The \textsc{ms.}\ contains the \emph{Elements}
\r1.–\r13.\ and some scholia after Books \r11., \r12.\ and \r13.

(7)~q = Paris \textsc{ms.}\ 2344, folio; 12th c.

It is written by one hand but includes scholia by many hands.  On
leaves 1 to~16 (recto) are scholia with the same title as that found
by Wachsmuth in a Vatican \textsc{ms.}\ and relied upon by him to
prove that Proclus continued his commentaries beyond
Book~\r1.\footnote{\greek{[εἰς τ]ὰ τοῦ Εὐκλείδου στοιχεῖα
    προλαμβανόμενα ἐκ το›ν Πρόκλου σποράδην καὶ κατ’ έπιτομήν}.
  cf.\ p.~32, note 8, above.} Leaves 17 to~357 contain the
\emph{Elements} \r1.–\r13.\ (except that there is a lacuna from the
middle of \prop{8}{25} to the \greek{ἔκθεσισ} of \prop{9}{14}); before
Books \r7.\ and \r10.\ there are some leaves filled with scholia only,
and leaves 358 to~366 contain nothing but scholia.

(8)~Heiberg also used a palimpsest in the British Museum
(Add.\ 17211). Five pages are of the 7th–8th~c.\ and are contained
(leaves 49–53) in the second volume of the Syrian
\textsc{ms.}\ Brit.\ Mus.\ 687 of the 9th~c.; half of leaf~50 has
perished. The leaves contain various fragments from
Book~\r10.\ enumerated by Heiberg, Vol.~\r3., p.~v, and nearly the
whole of \prop{13}{14}.

Since his edition of the \emph{Elements} was published, Heiberg has
collected further material bearing on the history of the
text\footnote{Heiberg, \emph{Paralipomena zu Euklid} in \emph{Hermes},
  \r38., 1903, pp.~46–74, 161–201, 321—356.}.  Besides giving the
results of further or new examination of \textsc{mss.}, he has
collected the fresh evidence contained in an-Nairīzī's commentary, and
particularly in the quotations from Heron's commentary given in it
(often word for word), which enable us in several cases to trace
differences between our text and the text as Heron had it, and to
identify some interpolations which actually found their way into the
text from Heron's commentary itself; and lastly he has dealt with some
valuable fragments of ancient papyri which have recently come to
light, and which are especially important in that the evidence drawn
from them necessitates some modification in the views expressed in the
preface to Vol.~\r5.\ as to the nature of the changes made in Theon's
recension, and in the principles laid down for differentiating between
Theon's recension and the original text, on the basis of a comparison
between P and the Theonine \textsc{mss.}\ alone.

The fragments of ancient papyri referred to are the following.

1.~\emph{Papyrus Herculanensis} No.~1061\footnote{Described by Heiberg
  in \emph{Oversigt over det kngl.\ danske Videnskabernes Selskabs
    Forhandlinger}, 1900, p.~161.}.

This fragment quotes Def.~15 of Book~\r1.\ in Greek, and omits the
words \greek{ἢ καλεῖται περιφέρεια}, ``which is called the
circumference,'' found in all our \textsc{mss.}, and the further
addition \greek{πρὸς τὴν τοῦ κύκλου περιφέρειαν} also found in
practically all the \textsc{mss.}  Thus Heiberg's assumption that both
expressions are interpolations is now confirmed by this oldest of all
sources.

2.~\emph{The Oxyrhynchus Papyri}~\r1.\ p.~58, No.~\r29.\ of the 3rd or
4th~c.

This fragment contains the enunciation of Eucl.\ \prop{2}{5} (with
figure, apparently without letters, immediately following, and not, as
usual in our \textsc{mss.}, at the end of the proof) and before it the
part of a word \greek{περιεχομε} belonging to \prop{2}{4} (with room
for -\greek{νῳ ὀρθογωνίῳ ὅπερ ἔδει δεῖξαι} and a stroke to mark the
end), showing that the fragment \emph{had not} the Porism which
appears in all the Theonine \textsc{mss.}\ and (in a later hand) in~P,
and thereby confirming Heiberg's assumption that the Porism was due to
Theon.

3.~A fragment in \emph{Fayum towns and their papyri}, p.~96,
No.~\r9.\ of 2nd or 3rd~c.

This contains \prop{1}{39} and \prop{1}{41} following one another and
almost complete, showing that \prop{1}{40} was wanting, whereas it is
found in all the \textsc{mss.}\ and is recognised by Proclus.
Moreover the text of the beginning of \prop{1}{39} is better than
ours, since it has no double \greek{διορισμός} but omits the first
(``I say that they are also in the same parallels ``) and has
``\emph{and}'' instead of ``\emph{for} let $AD$ be joined ``in the
next sentence.  It is clear that \prop{1}{40} was interpolated by
someone who thought there ought to be a proposition following
\prop{1}{39} and related to it as \prop{1}{38} is related to
\prop{1}{37} and \prop{1}{36} to \prop{1}{35}, although Euclid nowhere
uses \prop{1}{40}, and therefore was not likely to include it.  The
same interpolator failed to realise that the words ``let $AD$ be
joined'' were part of the \greek{ἔκθεσις} or \emph{setting-out}, and
took them for the \greek{κατασκευή} or ``construction'' which
generally follows the \greek{διορισμός} or ``particular statement ``
of the conclusion to be proved, and consequently thought it necessary
to insert a \greek{διορισμός} \emph{before} the words.

The conclusions drawn by Heiberg from a consideration of particular
readings in this papyrus along with those of our \textsc{mss.}\ will
be referred to below.

We now come to the principles which Heiberg followed, when preparing
his edition, in differentiating the original text from the Theonine
recension by means of a comparison of the readings of~P and of the
Theonine \textsc{mss.}  The rules which he gives are subject to a
certain number of exceptions (mostly in cases where one
\textsc{ms.}\ or the other shows readings due to copyists' errors),
but in general they may be relied upon to give conclusive results.

The possible alternatives which the comparison of~P with the Theonine
\textsc{mss.}\ may give in particular passages are as follows:

I.~There may be \emph{agreement} in three different degrees.

(1)~P and \emph{all} the Theonine \textsc{mss.}\ may agree.

In this case the reading common to all, even if it is corrupt or
interpolated, is more ancient than Theon, i.e.\ than the 4th~c.

(2) P may agree with \emph{some} (only) of the Theonine \textsc{mss.}

In this case Heiberg considered that the latter give the true reading
of Theon's recension, and the other Theonine \textsc{mss.}\ have
departed from it

(3) P and \emph{one} only of the Theonine \textsc{mss.}\ may agree.

In this case too Heiberg assumed that the \emph{one} Theonine
\textsc{ms.}\ which agrees with~P gives the true Theonine reading, and
that this rule even supplies a sort of measure of the quality and
faithfulness of the Theonine \textsc{mss.}  Now none of them agrees
alone with~P in preserving the true reading so often as~F. Hence F
must be held to have preserved Theon's recension more faithfully than
the other Theonine \textsc{mss.}; and it would follow that in those
portions where F fails us P must carry rather more weight even though
it may differ from the Theortine \textsc{mss.}\ BVpq. (Heiberg gives
many examples in proof of this, as of his main rules generally, for
which reference must be made to his \emph{Prolegomena} in Vol.~\r5.)
The specially close relation of F and P is also illustrated by
passages in which they have the same errors; the explanation of these
common errors (where not due to accident) is found by Heiberg in the
supposition that they existed, but were not noticed by Theon, in the
original copy in which he made his changes.

Although however F is by far the best of the Theonine \textsc{mss.},
there are a considerable number of passages where one of the others
(B, V, p or~q) \emph{alone} with~P gives the genuine reading of
Theon's recension.

As the result of the discovery of the papyrus fragment containing
\prop{1}{39, 41}, the principles above enunciated under (2) and~(3)
are found by Heiberg to require some qualification. For there is in
some cases a remarkable agreement between the papyrus and the Theonine
\textsc{mss.}\ (some or all) as against~P\@. This shows that Theon
took more trouble to follow older \textsc{mss.}, and made fewer
arbitrary changes of his own, than has hitherto been supposed. Next,
when the papyrus agrees with some of the Theonine
\textsc{mss.}\ against~P, it must now be held that these
\textsc{mss.}\ (and not, as formerly supposed, those which agree
with~P) give the true reading of Theon. If it were otherwise, the
agreement between the papyrus and the Theonine \textsc{mss.}\ would be
accidental: but it happens too often for this. It is clear also that
there must have been contamination between the two recensions;
otherwise, whence could the Theonine \textsc{mss.}\ which agree with~P
and not with the papyrus have got their readings? The influence of the
P class on the Theonine~F is especially marked.

II.~There may be \emph{disagreement} between P and all the Theonine
\textsc{mss.}

The following possibilities arise,

(1)~The Theonine \textsc{mss.}\ differ also among themselves.

In this case Heiberg considered that P nearly always has the true
reading, and the Theonine \textsc{mss.}\ have suffered interpolation
in different ways after Theon's time.

(2)~The Theonine \textsc{mss.}\ all combine against~P.

In this case the explanation was assumed by Heiberg to be one or other
of the following.
\begin{enumerate}
% TBD: lowercase greek labels

\item The common reading is due to an error which cannot be
imputed to Theon (though it may have escaped him when putting
together the archetype of his edition); such error may either have
arisen accidentally in all alike, or (more frequently) may be
referred to a common archetype of all the \textsc{mss.}

\item There may be an accidental error in~P; e.g.\ something has
  dropped out of~P in a good many places, generally through
  \greek{ὁμοιοτέλευτον}.

\item There may be words interpolated in~P.

\item Lastly, \emph{we may have in the Theonine \textsc{mss.}\ a
  change made by Theon himself}.
\end{enumerate}
(The discovery of the ancient papyrus showing readings agreeing with
some, or with all, of the Theonine \textsc{mss.}\ against~P now makes
tt necessary to be very cautious in applying these criteria.)

It is of course the last class~(\greek{δ}) of changes which we have to
investigate in order to get a proper idea of Theon's recension.

Heiberg first observes, as regards these, that we shall find that
Theon, in editing the \emph{Elements}, altered hardly anything without
some reason, often inadequate according to our ideas, but still some
reason which seemed to him sufficient.  Hence, in cases of very slight
differences where both the Theonine \textsc{mss.}\ and~P have readings
good and probable in themselves, Heiberg is not prepared to put the
differences down to Theon. In those passages where we cannot see the
least reason why Theon, if he had the reading of~P before him, should
have altered it, Heiberg would not at once assume the superiority of~P
unless there was such a consistency in the differences as would
indicate that they were due not to accident but to design. In the
absence of such indications, he thinks that the ordinary principles of
criticism should be followed and that proper weight should be attached
to the antiquity of the sources. And it cannot be denied that the
sources of the Theonine version are the more ancient. For not only is
the British Museum palimpsest~(L), which is intimately connected with
the rest of our \textsc{mss.}, at least two centuries older than~P,
but the other Theonine \textsc{mss.}\ are so nearly allied that they
must be held to have had a common archetype intermediate between them
and the actual edition of Theon; and, since they themselves are as old
as, or older than~P, their archetype must have been much
older. Heiberg gives (pp.~xlvi, xlvii) a list of passages where, for
this reason, he has followed the Theonine \textsc{mss.}\ in preference
to~P.

It has been mentioned above that the copyist of~P or rather of its
archetype wished to give an ancient recension. Therefore (apart from
clerical errors and interpolations) the first hand in~P may be relied
upon as giving a genuine reading even where a correction by the first
hand has been made at the same time. But in many places the first hand
has made corrections afterwards; on these occasions he must have used
new sources, e.g.\ when inserting the scholia to the first Book which
P alone has, and in a number of passages he has made additions from
Theonine \textsc{mss.}

We cannot make out any ``family tree'' for the different Theonine
\textsc{mss.}  Although they all proceeded from a common archetype
later than the edition of Theon itself, they cannot have been copied
one from the other; for, if they had been, how could it have come
about that in one place or other each of them agrees \emph{alone}
with~P in preserving the genuine reading?  Moreover the great variety
in their agreements and disagreements indicates that they have all
diverged to about the same extent from their archetype. As we have
seen that P contains corrections from the Theonine family, so they
show corrections from~P or other \textsc{mss.}\ of the same
family. Thus V has part of the lacuna in the \textsc{ms.}\ from which
it was copied filled up from a \textsc{ms.}\ similar to~P, and has
corrections apparently derived from the same; the copyist, however, in
correcting~V, also used another \textsc{ms.}\ to which he alludes in
the additions to \prop{9}{19} and 30 (and also on \prop{10}{23} Por.):
``in the book of the Ephesian (this) is not found,'' Who this Ephesian
of the 12th~c.\ was, we do not know.

We now come to the alterations made by Theon in his edition of the
\emph{Elements}.  I shall indicate \emph{classes} into which these
alterations may be divided but without details (except in cases where
they affect the mathematical content as distinct from form or language
pure and simple)\footnote{Exhaustive details under all the different
  heads are given by Heiberg (Vol.~\r5. pp.~lii–lxxv).}.

I.~\emph{Alterations made by Theon where he found, or thought he
  found, mistakes in the original.}

1. Real blots in the original which Theon saw and tried to remove.

(\emph{a}) Euclid has a porism (corollary) to \prop{6}{19}, the
enunciation of which speaks of similar and similarly described
\emph{figures} though the proposition itself refers only to triangles,
and therefore the porism should have come after \prop{6}{20}. Theon
substitutes \emph{triangle} for \emph{figure} and proves the more
general porism after~\prop{6}{20}.

(\emph{b}) In \prop{9}{19} there is a statement which is obviously
incorrect.  Theon saw this and altered the proof by reducing four
alternatives to two, with the result that it fails to correspond to
the enunciation even with Theon's substitution of ``if'' for ``when''
in the enunciation.

(\emph{c}) Theon omits a porism to \prop{9}{11}, although it is necessary for
the proof of the succeeding proposition, apparently because, owing to
an error in the text (\greek{κατὰ τὸν} corrected by Heiberg into
\greek{ἐπὶ τὸ}), he could not get out of it the right sense.

(\emph{d}) I should also put into this category a case which Heiberg
classifies among those in which Theon merely fancied that he found
mistakes, viz.\ the porism to \prop{5}{7} stating that, if four
magnitudes are proportional, they are proportional inversely. Theon
puts this after \prop{5}{4} with a proof, which however has no
necessary connexion with \prop{5}{4} but is obvious from the
definition of proportion,

(\emph{e}) I should also put under this head \prop{11}{1}, where
Euclid's argument to prove that two straight lines cannot have a
common segment is altered.

2.~Passages which seemed to Theon to contain blots, and which he
therefore set himself to correct, though more careful consideration
would have shown that Euclid's words are right or at least may be
excused and offer no difficulty to an intelligent reader. Under this
head come:

(\emph{a}) an alteration in \prop{3}{24}.

(\emph{b}) a perfectly unnecessary alteration, in \prop{6}{14}, of
``equiangular parallelograms'' into ``parallelograms having one angle
equal to one angle,'' where Theon followed the false analogy of
\prop{6}{15}.

(\emph{c}) an omission of words in \prop{5}{26}, owing to his having
been misled by a wrong figure.

(\emph{d}) an alteration of the order of \r11.\ Deff.~27,~28.

(\emph{e}) the substitution of ``parallelepipedal solid'' for ``cube''
in \prop{11}{38}, because Theon observed, correctly enough, that it
was true of the parallelepipedal solid in general as well as of the
cube, but failed to give weight to the fact that Euclid must have
given the particular case of the cube for the simple reason that that
was all he wanted for use in \prop{13}{17}.

(\emph{f}) the substitution of the letter \greek{Φ} for \greek{Ω} ($V$
for~$Z$ in my figure) because he saw that the perpendicular from
\greek{Κ} to \greek{ΒΦ} would fall on~\greek{Φ} itself, so that
\greek{Φ}, \greek{Ω} coincide. But, if the substitution is made, it
should be proved that \greek{Φ}, \greek{Ω} coincide. Euclid can hardly
have failed to notice the fact, but it may be that he deliberately
ignored it as unnecessary for his purpose, because he did not want to
lengthen his proposition by giving the proof.

II.~\emph{Emendations intended to improve the form or diction of Euclid.}

Some of these emendations of Theon affect passages of appreciable
length. Heiberg notes about ten such passages; the longest is in
Eucl.\ \prop{12}{4} where a whole page of Heiberg's text is affected
and Theon's version is put in the Appendix. The kind of alteration may
be illustrated by that in \prop{9}{15} where Euclid uses successively
the propositions \prop{7}{24, 25}, quoting the enunciation of the
former but not of the latter; Theon does exactly the reverse. In a few
of the cases here quoted by Heiberg, Theon shortened the original
somewhat.

But, as a rule, the emendations affect only a few words in each
sentence. Sometimes they are considerable enough to alter the
conformation of the sentence, sometimes they are trifling alterations
``more magistellorum ineptorum'' and unworthy of Theon.  Generally
speaking, they were prompted by a desire to change anything which was
out of the common in expression or in form, in order to reduce the
language to one and the same standard or norm. Thus Theon changed the
order of words, substituted one word for another where the latter was
used in a sense unusual with Euclid (e.g.\ \greek{ἐπειδήπερ},
``since,'' for \greek{ὅτι} in the sense of ``because''), or one
expression for another in like circumstances (e.g.\ where, finding
``that which was enjoined would be done'' in a \emph{theorem},
\prop{7}{31}, and deeming the phrase more appropriate to a
\emph{problem}, he substituted for it ``that which is sought would be
manifest''; probably also and for similar reasons he made certain
variations between the two expressions usual at the end of
propositions \greek{ὅπερ ἔδει δεῖξαι}, and \greek{ὅπερ ἔδει ποιῆσαι}
\emph{quod erat demonstrandum} and \emph{quod erat
  faciendum}). Sometimes his alterations show carelessness in the use
of technical terms, as when he uses \greek{ἅπτεσθαι} (to \emph{meet})
for \greek{ἐφάπτεσθαι} (to \emph{touch}) although the ancients
carefully distinguished the two words. The desire of keeping to a
standard phraseology also led Theon to omit or add words in a number
of cases, and also, sometimes, to change the lettering of figures.

But Theon seems, in editing the \emph{Elements}, to have bestowed the
most attention upon

III.~\emph{Additions designed to supplement or explain Euclid.}

First, he did not hesitate to interpolate whole propositions where he
thought there was room or use for them. We have already mentioned the
addition to \prop{6}{33} of the second part relating to
\emph{sectors}, for which Theon himself takes credit in his commentary
on Ptolemy.  Again, he interpolated the proposition commonly known as
\prop{7}{22} (\emph{ex aequo in proportione perturbata} for numbers,
corresponding to \prop{5}{23}), and perhaps also \prop{7}{20}, a
particular case of \prop{7}{19} as \prop{6}{17} is of \prop{6}{16}.
He added a second case to \prop{6}{27}, a porism to \prop{2}{4}, a
second porism to \prop{3}{16}, and a lemma after \prop{10}{12};
perhaps also the porism to \prop{5}{19} and the first porism to
\prop{6}{20}. He also inserted alternative proofs here and there,
e.g.\ in \prop{2}{4} (where the alternative differs little from the
original) and in \prop{7}{31}; perhaps also in \prop{10}{1, 6, and 9}.

Secondly, he sometimes repeats an argument where Euclid had said ``For
the same reason,'' adds specific references to points, straight lines
etc.\ in the figures in order to exclude the possibility of mistake
arising from Euclid's reference to them in general terms, or inserts
words to make the meaning of Euclid more plain,
e.g.\ \emph{componendo} and \emph{alternately}, where Euclid had left
them out. Sometimes he thought to increase by his additions the
mathematical precision of Euclid's language in enunciations or
elsewhere, sometimes to make smoother and clearer things which Euclid
had expressed with unusual brevity and harshness or carelessness, in
reliance on the intelligence of his readers.

Thirdly, he supplied intermediate steps where Euclid's argument seemed
too rapid and not easy enough to follow. The form of these additions
varies; they are sometimes placed as a definite intermediate step with
``therefore'' or ``so that,'' sometimes they are additions to the
statement of premisses, sometimes phrases introduced by ``since,''
``for'' and the like, after the inference.

Lastly, there is a very large class of additions of a word, or one or
two words, for the sake of clearness or consistency. Heiberg gives a
number of examples of the addition of such nouns as ``triangle,''
``square,'' ``rectangle,'' ``magnitude,'' ``number,'' ``point,''
``side,'' ``circle,'' ``straight line,'' ``area'' and the like, of
adjectives such as ``remaining,'' ``right,'' ``whole,''
``proportional,'' and of other parts of speech, even down to words
like ``is'' (\greek{ἐστί}) which is added 600 times, \greek{δή},
\greek{ἄρα}, \greek{μέν}, \greek{γάρ}, \greek{καί} and the like.

IV.~\emph{Omissions by Theon}

Heiberg remarks that, Theon's object having been, as above shown, to
amplify and explain Euclid, we should not naturally have expected to
find him doing much in the contrary process of compression, and it is
only owing to the recurrence of a certain sort of omissions so
frequently (especially in the first Books) as to exclude the
hypothesis of their being all due to chance that we are bound to
credit him, with alterations making for greater brevity. We have seen,
it is true, that he made omissions as well as additions for the
purpose of reducing the language to a certain standard form. But there
are also a good number of cases where in the enunciation of
propositions, and in the \emph{exposition} (the re-statement of them
with reference to the figure), he has left out words because,
apparently, he regarded Euclid's language as being \emph{too} careful
and precise.  Again, he is apparently responsible for the frequent
omission of the words \greek{ὅπερ ἔδει δεῖξαι} (or \greek{ποιῆσαι}),
\textsc{q.e.d.}\ (or \textsc{f.}), at the end of propositions. This is
often the case at the end of porisms, where, in omitting the words,
Theon seems to have deliberately departed from Euclid's practice. The
\textsc{ms.}~P seems to show clearly that, where Euclid put a porism
at the end of a proposition, he omitted the \textsc{q.e.d.}\ at the
end of the proposition but inserted it at the end of the porism, as if
he regarded the latter as being actually a part of the proposition
itself. As in the Theonine \textsc{mss.}\ the \textsc{q.e.d.}\ is
generally omitted, the omission would seem to have been due to Theon.
Sometimes in these cases the \textsc{q.e.d.}\ is interpolated at the
end of the proposition.

Heiberg summed up the discussion of Theon's edition by the remark that
Theon evidently took no pains to discover and restore from
\textsc{mss.}\ the actual words which Euclid had written, but aimed
much more at removing difficulties that might be felt by learners in
studying the book. His edition is therefore not to be compared with
the editions of the Alexandrine grammarians, but rather with the work
done by Eutocius in editing Apollonius and with an interpolated
recension of some of the works of Archimedes by a certain Byzantine,
Theon occupying a position midway between these two editors, being
superior to the latter in mathematical knowledge but behind Eutocius
in industry (these views now require to be somewhat modified, as above
stated). But however little Theon's object may be approved by those of
us who would rather know the \emph{ipsissima verba} of Euclid, there
is no doubt that his work was approved by his pupils at Alexandria for
whom it was written; and his edition was almost exclusively used by
later Greeks, with the result that the more ancient text is only
preserved to us in one \textsc{ms.}

As the result of the above investigation, we may feel satisfied that,
where P and the Theonine \textsc{mss.}\ agree, they give us (except in
a few accidental instances) Euclid as he was read by the Greeks of the
4th~c. But even at that time the text had been passed from hand to
hand through more than six centuries, so that it is certain that it
had already suffered changes, due partly to the fault of copyists and
partly to the interpolations of mathematicians.  Some errors of
copyists escaped Theon and were corrected in some \textsc{mss.}\ by
later hands. Others appear in all our \textsc{mss.}\ and, as they
cannot have arisen accidentally in all, we must put them down to a
common source more ancient than Theon. A somewhat serious instance is
to be found in \prop{3}{8}; and the use of \greek{ἁπτέσθω} for
\greek{ἐφαπτέσθω} in the sense of ``touch'' may also be mentioned, the
proper distinction between the words having been ignored as it was by
Theon also.  But there are a number of imperfections in the
ante-Theonine text which it would be unsafe to put down to the errors
of copyists, those namely where the good \textsc{mss.}\ agree and it
is not possible to see any motive that a copyist could have had for
altering a correct reading.  In these cases it is possible that the
imperfections are due to a certain degree of carelessness on the part
of Euclid himself; for it is not possible ``Euclidem ab omni naevo
vindicare,'' to use the words of Saccheri\footnote{\emph{Euclides ab
    omni naevo vindicatus}, Mediolani, 1733.}, and consequently Simson
is not right in attributing to Theon and other editors all the things
in Euclid to which mathematical objection can be taken. Thus, when
Euclid speaks of ``the ratio compounded of the sides'' for ``the ratio
compounded of the \emph{ratios of the} sides,'' there is no reason for
doubting that Euclid himself is responsible for the more slip-shod
expression.  Again, in the Books \r11.–\r13.\ relating to solid
geometry there are blots neither few nor altogether unimportant which
can only be attributed to Euclid himself\footnote{Cf.\ especially the
  assumption, without proof or definition, of the criterion for
  \emph{equal} solid angles, and the incomplete proof of
  \prop{12}{17}.}; and there is the less reason for hesitation in so
attributing them because solid geometry was then being treated in a
thoroughly systematic manner for the first time.  Sometimes the
conclusion (\greek{συμπέρασμα}) of a proposition does not correspond
exactly to the enunciation, often it is cut short with the words
\greek{καὶ τὰ ἑξῆς} ``and the rest'' (especially from
Book~\r10.\ onwards), and very often in Books \r8., \r9.\ it is
omitted.  Where all the \textsc{mss.}\ agree, there is no ground for
hesitating to attribute the abbreviation or omission to Euclid;
though, of course, where one or more \textsc{mss.} have the longer
form, it must be retained because this is one of the cases where a
copyist has a temptation to abbreviate.

Where the true reading is preserved in one of the Theonine
\textsc{mss.}\ alone, Heiberg attributes the wrong reading to a
mistake which arose before Theon's time, and the right reading of the
single \textsc{ms.}\ to a successful correction.

We now come to the most important question of the \emph{Interpolations
  introduced before Theon's time}.

I. Alternative proofs or additional cases.

It is not in itself probable that Euclid would have given two proofs
of the same proposition; and the doubt as to the genuineness of the
alternatives is increased when we consider the character of some of
them and the way in which they are introduced.  First of all, we have
those of \prop{6}{20} and \prop{12}{17} introduced by ``we shall prove
this otherwise \emph{more readily} (\greek{προχειρότερον})'' or that
of \prop{10}{90} ``it is possible to prove \emph{more shortly}
(\greek{συντομώτερον}).'' Now it is impossible to suppose that Euclid
would have given one proof as that definitely accepted by him and then
added another with the express comment that the latter has certain
advantages over the former. Had he considered the two proofs and come
to this conclusion, he would have inserted the latter in the received
text instead of the former. These alternative proofs must therefore
have been interpolated. The same argument applies to alternatives
introduced with the words ``or even this'' (\greek{ἢ καὶ οὕτως}), ``or
even otherwise'' (\greek{ἣ καὶ ἄλλως}). Under this head come the
alternatives for the last portions of \prop{3}{7, 8}; and Heiberg also
compares the alternatives for parts of \prop{3}{31} (that the angle in
a semicircle is a right angle) and \prop{13}{18}, and the alternative
proof of the lemma after \prop{10}{32}. The alternatives to
\prop{10}{105 and 106}, again, are condemned by the place in which
they occur, namely after an alternative proof to \prop{10}{115}.  The
above alternatives being all admitted to be spurious, suspicion must
necessarily attach to the few others which are in themselves
unobjectionable, Heiberg instances the alternative proofs to
\prop{3}{9}, \prop{3}{10}, \prop{6}{30}, \prop{6}{31} and
\prop{11}{22}, observing that it is quite comprehensible that any of
these might have occurred to a teacher or editor and seemed to him,
rightly or wrongly, to be better than the corresponding proofs in
Euclid.  Curiously enough, Simson adopted the alternatives to
\prop{3}{9, 10} in preference to the genuine proofs. Since Heiberg's
preface was written, his suspicion has been amply confirmed as regards
\prop{3}{10} by the commentary of an-Nairīzī (ed.\ Curtze) which shows
not only that this alternative is Heron's, but also that the
substantive proposition \prop{3}{12} in Euclid is also Heron's, having
been given by him to supplement \prop{3}{11} which must originally
have been enunciated of circles ``touching one another'' simply,
i.e.\ so as to include the case of external as well as internal
contact, though the proof covered the case of internal contact
only. ``Euclid, in the 11th proposition,'' says Heron, ``supposed two
circles touching one another internally and wrote the proposition on
this case, proving what it was required to prove in it.  \emph{But I
  will show how it is to be proved if the contact be
  external}\footnote{An-Nairīzī, ed.\ Curtze, p.~121.}.'' This
additional proposition of Heron's is by way of adding another case,
which brings us to that class of interpolation. It was the practice of
Euclid and the ancients to give only one case (generally the most
difficult one) and to leave the others to be investigated by the
reader for himself.  One interpolation of a second case (\prop{6}{27})
is due, as we have seen, to Theon.  The two extra cases of
\prop{11}{23} were manifestly interpolated before Theon's time, for
the preliminary distinction of three cases, ``(the centre) will either
be within the triangle $LMN$, or on one of the sides, or
outside. First let it be within,'' is a spurious addition (B and V
only). Similarly an unnecessary case is interpolated in \prop{3}{11}.

II. Lemmas,

Heiberg has unhesitatingly placed in his Appendix to \r3.\ certain
lemmas interpolated either by Theon (on \prop{10}{13}) or later
writers (on \prop{10}{27, 29, 31, 32, 33, 34}, where V only has the
lemmas).  But we are here concerned with the lemmas found in all the
\textsc{mss.}, which however are, for different reasons, necessarily
suspected. We will deal with the Book~\r10.\ lemmas last.

(1) There is an \emph{a priori} ground of objection to those lemmas
which come \emph{after} the propositions to which they relate and
prove properties used in those propositions; for, if genuine, they
would be a sign of faulty arrangement such as would not be likely in a
systematic work so carefully ordered as the \emph{Elements}. The lemma
to \prop{6}{22} is one of this class, and there is the further
objection to it that in \prop{6}{28} Euclid makes an assumption which
would equally require a lemma though none is found. The lemma after
\prop{12}{4} is open to the further objections that certain altitudes
are used but are not drawn in the figure (which is not in the manner
of Euclid), and that a peculiar expression ``parallelepipedal solids
\emph{described on} (\greek{ἀναγραφόμενα ἀπό}) \emph{prisms}'' betrays
a hand other than Euclid's. There is an objection on the score of
language to the lemma after \prop{13}{2}, The lemmas on \prop{11}{23},
\prop{13}{13}, \prop{13}{18}, besides coming after the propositions to
which they relate, are not very necessary in themselves and, as
regards the lemma to \prop{13}{13}, it is to be noticed that the
writer of a gloss in the proposition could not have had it, and the
words ``as will be proved afterwards'' in the text are rightly
suspected owing to differences between the \textsc{ms.}\ readings. The
lemma to \prop{12}{2} also, to which Simson raised objection, comes
\emph{after} the proposition; but, if it is rejected, the words ``as
was proved before'' used in \prop{12}{5 and 18}, and referring to this
lemma, must be struck out.

(2)~Reasons of substance are fatal to the lemma before \prop{10}{60},
which is really assumed in \prop{10}{44} and therefore should have
appeared there if anywhere, and to the lemma on \prop{10}{20}, which
tries to prove what is already stated in \r10.~Def.~4.

We now come to the remaining lemmas in Book~\r10., eleven in number,
which come \emph{before} the propositions to which they relate and
remove difficulties in the way of their demonstration. That before
\prop{10}{42} introduces a set of propositions with the words ``that
the said irrational straight lines are uniquely divided…we will prove
after premising the following lemma,'' and it is not possible to
suppose that these words are due to an interpolator; nor are there any
objections to the lemmas before \prop{10}{14, 17, 22, 33, 54}, except
perhaps that they are rather easy. The lemma before \prop{10}{10} and
\prop{10}{10} itself should probably be removed from the
\emph{Elements}; for \prop{10}{10} really uses the following
proposition \prop{10}{11}, which is moreover numbered~10 by the first
hand in~P, and the words in \prop{10}{10} referring to the lemma ``for
we learnt (how to do this)'' betray the interpolator. Heiberg gives
reason also for rejecting the lemmas before \prop{10}{19 and 24} with
the words ``in any of the aforesaid ways'' (omitted in the Theonine
\textsc{mss.})\ in the enunciations of \prop{10}{19, 24} and in the
\emph{exposition} of \prop{10}{20}. Lastly, the lemmas before
\prop{10}{29} may be genuine, though there is an addition to the
second of them which is spurious.

Heiberg includes under this heading of interpolated lemmas two which
purport to be substantive propositions, \prop{11}{38} and
\prop{13}{6}. These must be rejected as spurious for reasons which
will be found in detail in my notes on \prop{11}{37} and \prop{13}{6}
respectively.  The latter proposition is only quoted once (in
\prop{13}{17}); probably the words quoting it (with \greek{γραμμή}
instead of \greek{εὐθεῖα}) are themselves interpolated, and Euclid
thought the fact stated a sufficiently obvious inference from
\prop{13}{1}.

III. Porisms (or corollaries).

Most of the porisms in the text are both genuine and necessary; but
some are shown by differences in the \textsc{mss.}\ not to be so,
e.g.\ those to \prop{1}{15} (though Proclus has it), \prop{3}{31} and
\prop{6}{20} (Por.~2).  Sometimes parts of porisms are interpolated.
Such are the last few lines in the porisms to \prop{4}{5},
\prop{6}{8}; the latter addition is proved later by means of
\prop{6}{4, 8}, so that the writer of these proofs could not have had
the addition to \prop{6}{8}~Por.\ before him.  Lastly, interpolators
have added a sort of proof to some porisms, as though they were not
quite obvious enough; but to add a demonstration is inconsistent with
the idea of a porism, which, according to Proclus, is a by-product of
a proposition appearing without our seeking it

IV.~Scholia.

Several interpolated scholia betray themselves by their wording,
e.g.\ those given by Heiberg in the Appendix to Book~\r10.\ and
containing the words \greek{καλεῖ}, \greek{ἐκάλεσε} (``he calls'' or
``called''); these scholia were apparently written as marginal notes
before Theon's time, and, being adopted as such by Theon, found their
way into the text in~P and some of the Theonine \textsc{mss.}  The
same thing no doubt accounts for the interpolated analyses and
syntheses to \prop{13}{1–5}, as to which see my note on \prop{13}{1}.

V.~Interpolations in Book~\r10.

First comes the proposition ``\emph{Let it be proposed to us} to show
that in square figures the diameter is incommensurable in length with
the side,'' which, with a scholium after it, ends the tenth Book. The
form of the enunciation is suspicious enough and the proposition, the
proof of which is indicated by Aristotle and perhaps was Pythagorean,
is perfectly unnecessary when \prop{10}{9} has preceded. The scholium
ends with remarks about commensurable and incommensurable solids,
which are of course out of place before the Books on solids. The
scholiast on Book~\r10.\ alludes to this particular scholium as being
due to ``Theon and some others.'' But it is doubtless much more
ancient, and may, as Heiberg conjectures have been the beginning of
Apollonius' more advanced treatise on incommensurables. Not only is
everything in Book~\r10.\ after \prop{10}{115} interpolated, but
Heiberg doubts the genuineness even of \prop{10}{112–115}, on the
ground that \prop{10}{111} rounds off the theory of incommensurables as
we want it in the Books on solid geometry, while \prop{10}{112–115}
are not really connected with what precedes, nor wanted for the later
Books, but seem to form the starting-point of a new and more elaborate
theory of irrationals,

VI.~Other minor interpolations are found of the same character as
those above attributed to Theon. First there are two places
(\prop{11}{35} and \prop{11}{26}) where, after ``similarly we shall
prove'' and ``for the same reason,'' an actual proof is nevertheless
given.  Clearly the proofs are interpolated; and there are other
similar interpolations. There are also interpolations of intermediate
steps in proofs, unnecessary explanations and so on, as to which I
need not enter into details.

Lastly, following Heiberg's order, I come to

VII.~Interpolated definitions, axioms etc.

Apart from \r6.~Def.~5 (which may have been interpolated by Theon
although it is found written in the margin of~P by the first hand),
the definition of a segment of a circle in Book~\r1.\ is interpolated,
as is clear from the fact that it occurs in a more appropriate place
in Book~\r3.\ and Proclus omits it.  \book{6}~{Def.~2} (reciprocal
figures) is rightly condemned by Simson—perhaps it was taken from
Heron—and Heiberg would reject \book{7}{Def.~10}, as to which see my
note on that definition. Lastly the double definition of a solid angle
(\book{11}{Def.~11}) constitutes a difficulty. The use of the word
\greek{ἐπιφάνεια} suggests that the first definition may have been
older than Euclid, and he may have quoted it from older
\emph{elements}, especially as his own definition which follows only
includes solid angles contained by \emph{planes}, whereas the other
includes other sorts (cf.\ the words \greek{γραμμῶν},
\greek{γραμμαῖς}) which are also distinguished by Heron (Def.~22). If
the first definition had come last, it could have been rejected
without hesitation: but it is not so easy to reject the first part up
to and including ``otherwise'' (\greek{ἄλλως}).  No difficulty need be
felt about the definitions of ``oblong,'' ``rhombus,'' and
``rhomboid,'' which are not actually used in the \emph{Elements}; they
were no doubt taken from earlier \emph{elements} and given for the
sake of completeness.

As regards the axioms or, as they are called in the text, \emph{common
  notions} (\greek{κοιναὶ ἔννοιαι}), it is to be observed that Proclus
says\footnote{Proclus, pp.~194, 10 sqq.} that Apollonius tried to
prove ``the axioms,'' and he gives Apollonius' attempt to prove
Axiom~1.  This shows at all events that Apollonius had \emph{some} of
the axioms now appearing in the text. But how could Apollonius have
taken a controversial line against Euclid on the subject of axioms if
these axioms had not been Euclid's to his knowledge? And, if they had
been interpolated between Euclid's time and his own, how could
Apollonius, living so comparatively short a time after Euclid, have
been ignorant of the fact? Therefore \emph{some} of the axioms are
Euclid's (whether he called them \emph{common notions}, or
\emph{axioms}, as is perhaps more likely since Proclus calls them
axioms): and we need not hesitate to accept as genuine the first three
discussed by Proclus, viz.\ (1)~things equal to the same equal to one
another, (2)~if equals be added to equals, wholes equal, (3)~if equals
be subtracted from equals, remainders equal. The other two mentioned
by Proclus (whole greater than part, and congruent figures equal) are
more doubtful, since they are omitted by Heron, Martianus Capella, and
others. The axiom that ``two lines cannot enclose a space'' is however
clearly an interpolation due to the fact that \prop{1}{4} appeared to
require it. The others about equals added to unequals, doubles of the
same thing, and halves of the same thing are also interpolated; they
are connected with other interpolations, and Proclus clearly used some
source which did not contain them.

Euclid evidently limited his formal axioms to those, which seemed to
him most essential and of the widest application; for he not
unfrequently assumes other things as axiomatic, e.g.\ in \prop{7}{28}
that, if a number measures two numbers, it measures their difference.

The differences of reading appearing in Proclus suggest the question
of the comparative purity of the sources used by Proclus, Heron and
others, and of our text. The omission of the definition of a segment
in Book~\r1.\ and of the old gloss ``which is called the
circumference'' in \book{1}{Def.~15} (also omitted by Heron, Taurus,
Sextus Empiricus and others) indicates that Proclus had better sources
than we have; and Heiberg gives other cases where Proclus omits words
which are in all our \textsc{mss.}\ and where Proclus' reading should
perhaps be preferred. But, except in these instances (where Proclus
may have drawn from some ancient source such as one of the older
commentaries), Proclus' \textsc{ms.}\ does not seem to have been among
the best.  Often it agrees with our worst \textsc{mss.}, sometimes it
agrees with~F where F alone has a certain reading in the text, so that
(e.g.\ in \prop{1}{15} Por.)\ the common reading of Proclus and F must
be rejected, thrice only does it agree with P alone, sometimes it
agrees with P and some Theonine \textsc{mss.}, and once it agrees with
the Theonine \textsc{mss.}\ against P and other sources.

Of the other external sources, those which are older than Theon
generally agree with our best \textsc{mss.}, e.g.\ Heron, allowing for
the difference in the plan of his definitions and the somewhat free
adaptation to his purpose of the Euclidean definitions in Books \r10.,
\r11.

Heiberg concludes that the \emph{Elements} were most spoiled by
interpolations about the 3rd~c., for Sextus Empiricus had a correct
text, while Iamblichus had an interpolated one; but doubtless the
purer text continued for a long time in circulation, as we conclude
from the fact that our \textsc{mss.}\ are free from interpolations
already found in Iamblichus' \textsc{ms.}

\chapter{The Scholia}

Heiberg has collected scholia, to the number of about 1500, in
Vol.~\r5. of his edition of Euclid, and has also discussed and
classified them in a separate short treatise, in which he added a few
others\footnote{Heiberg, \emph{Om Scholierne til Euklids Elementer},
  Kjøbenhavn, 1888. The tract is written in Danish, but, fortunately
  for those who do not read Danish easily, the author has appended
  (pp.~70–78) a résumé in French.}.

These scholia cannot be regarded as doing much to facilitate the
reading of the \emph{Elements}f . As a rule, they contain only such
observations as any intelligent reader could make for himself. Among
the few exceptions are \book{11}Nos.~\prop*{11}{33}, \prop*{11}{35}
(where \prop{11}{22, 23} are extended to solid angles formed by any
number of plane angles), \hyperref[prop:XII_85]{\book{12}{No.~85}}
(where an assumption tacitly made by Euclid in \prop{12}{17} is
proved), \book{9}{Nos.~\prop*{9}{28}, \prop*{9}{29}} (where the
  scholiast has pointed out the error in the text of \prop{9}{19}).

Nor are they very rich in historical information; they cannot be
compared in this respect with Proclus' commentary on Book~\r1.\ or
with those of Eutocius on Archimedes and Apollonius.  But even under
this head they contain some things of interest,
e.g.\ \hyperref[prop:II_11]{\book{2}{No.~11}} explaining that the
gnomon was invented by geometers for the sake of brevity, and that its
name was suggested by an incidental characteristic, namely that ``from
it the whole is known (\greek{γνωρίζεται}), either of the whole area
or of the remainder, when it (the \greek{γνώμων}) is either placed
round or taken away''; \hyperref[prop:2_13]{\book{2}{No.~13}}, also on the gnomon;
\hyperref[prop:4_2]{\book{4}{No.~2}} stating that Book~\r4.\ was the discovery of the
Pythagoreans; \hyperref[prop:5_1]{\book{5}{No.~1}} attributing the content of Book~\r5.\ to
Eudoxus; \hyperref[prop:10_1]{\book{10}{No.~1}} with its allusion to the discovery of
incommensurability by the Pythagoreans and to Apollonius' work on
irrationals; \hyperref[prop:10_62]{\book{10}{No.~62}} definitely attributing \prop{10}{9} to
Theaetetus; \hyperref[prop:13_1]{\book{13}{No.~1}} about the ``Platonic'' figures, which
attributes the cube, the pyramid, and the dodecahedron to the
Pythagoreans, and the octahedron and icosahedron to Theaetetus.

Sometimes the scholia are useful in connexion with the settlement of
the text, (1)~directly, e.g.\ \hyperref[prop:3_16]{\book{3}{No.~16}} on the interpolation of
the word ``within'' (\greek{ἐντός}) in the enunciation of \prop{3}{6},
and \hyperref[prop:10_1]{\book{10}{No.~1}} alluding to the discussion by ``Theon and some
others'' of irrational ``surfaces'' and ``solids,'' as well as
``lines,'' from which we may cnlcude that the scholium at the end of
Book~\r10.\ is not genuine; (2)~indirectly in that they sometimes
throw light on the connexion of certain \textsc{mss.}

Lastly, they have their historical importance as enabling us to judge
of the state of mathematical science at the times when they were
written.

Before passing to the classification of the scholia, Heiberg remarks
that we must separate from them a number of additions in the nature of
scholia which are found in the text of our \textsc{mss.}\ but which
can, in one way or another, be proved to be spurious. As they are
found both in~P and in the Theonine \textsc{mss.}, they must have been
in the \textsc{mss.}\ anterior to Theon (4th~c.). But they are, in
great part, only found in the margin of~P and the Theonine
\textsc{mss.}; in~V they are half in the text and half in the
margin. This can hardly be explained except on the supposition that
these additions were originally (in the \textsc{mss.}\ before Theon)
in the margin, and that Theon kept them there in his edition, but that
they afterwards found their way gradually into the text of~P as well
as of the Theonine \textsc{mss.}, or were omitted altogether, while
particular \textsc{mss.}\ have in certain places preserved the old
arrangement.  Of such spurious additions Heiberg enumerates the
following: the axiom about equals subtracted from unequals, the last
lines of the porism to \prop{6}{8}, second porisms to \prop{5}{19} and
to \prop{6}{20}, the porism to \prop{3}{31}, \book{6}{Def.~5}, various
additions in Book~\r10., the analyses and syntheses of \prop{13}{1–5},
an the proposition \prop{13}{6}.

The two first classes of scholia distinguished by Heiberg are denoted
by the convenient abbreviations ``Schol.\ Vat.'' and ``Schol.\ Vind.''

I. \textbf{Schol.\ Vat.}

It is first necessary to set out the letters by which Heiberg denotes
certain collections of scholia.

P = Scholia in P written by the first hand.

B = Scholia in B by a hand of the same date as the
\textsc{ms.}\ itself, generally that of Arethas.

F = Scholia in F by the first hand.

Vat = Scholia of the Vatican \textsc{ms.}\ 204 of the 10th~c., which
has these scholia on leaves 198–205 (the end is missing) as an
independent collection. It does not contain the text of the
\emph{Elements}.

V\tsup{c} = Scholia found on leaves 283—292 of~V and written in the
same hand as that part of the \textsc{ms.}\ itself which begins at
leaf~235.

Vat.~192 = a Vatican \textsc{ms.}\ of the 14th~c.\ which contains,
after (1)~the \emph{Elements} \r1.–\r13.\ (without scholia), (2)~the
\emph{Data} with scholia, (3)~Marinus on the \emph{Data}, the
Schol.\ Vat.\ as an independent collection and in their entirety,
beginning with \hyperref[prop:1_88]{\book{1}{No.~88}} and ending with \hyperref[prop:13_44]{\book{13}{No.~44}}.

The Schol.\ Vat., the most ancient and important collection of
scholia, comprise those which are found in PBF Vat.\ and, from
\prop{7}{12} to \prop{9}{15}, in PB Vat.\ only, since in that portion
of the \emph{Elements} F was restored by a later hand without scholia;
they also include \hyperref[prop:1_88]{\book{1}{No.~88}} which only
happens to be erased in~F, and \book{9}{Nos.~\prop*{9}{28},
  \prop*{9}{29}} which may be left out because F, here has a different
text.  In F and Vat.\ the collection ends with Book~\r10.; but it must
also include Schol.\ FB of Books \r11.—\r13., since these are found
along with Schol.\ Vat.\ to Books \r1.–\r10.\ in several
\textsc{mss.}\ (of which Vat.~192 is one) as a separate
collection. The Schol.\ Vat.\ to Books~\r10.–\r13.\ are also found in
the collection V\tsup{c} (where, curiously enough,
\book{13}{Nos.~\prop*{13}{43}, \prop*{13}{44}} are at the
beginning). The Schol.\ Vat.\ accordingly include Schol.\ PBV\tsup{c}
Vat.~192, and doubtless also those which are found in two of these
sources. The total number of scholia classified by Heiberg as
Schol.\ Vat.\ is~138.

As regards the contents of Schol.\ Vat.\ Heiberg has the following
observations. The thirteen scholia to Book~\r1.\ are extracts made
from Proclus by a writer thoroughly conversant with the subject, and
cleverly recast (with some additions). Their author does not seem to
have had the two lacunae which our text of Proclus has (at the end of
the note on \prop{1}{36} and the beginning of the next note, and at
the beginning of the note on \prop{1}{43}), for the scholia
\book{1}{Nos.~\prop*{1}{125} and \prop*{1}{137}} seem to fill the gaps
appropriately, at least in part. In some passages he had better
readings than our \textsc{mss.}\ have. The rest of Schol.\ Vat.\ (on
Books \r2.–\r13.)\ are essentially of the same character as those on
Book~\r1., containing prolegomena, remarks on the object of the
propositions, critical remarks on the text, converses, lemmas; they
are, in general, exact and true to tradition. The reason of the
resemblance between them and Proclus appears to be due to the fact
that they have their origin in the commentary of Pappus, of which we
know that Proclus also made use. In support of the view that Pappus is
the source, Heiberg places some of the Schol.\ Vat.\ to
Book~\r10.\ side by side with passages from the commentary of Pappus
in the Arabic translation discovered by Woepcke\footnote{\emph{Om
    Scholierne til Euklids Elementer}, pp.~11, 12:
  cf.\ \emph{Euklid-Studien}, pp.~170, 171; Woepcke, \emph{Mémoires
    présent.\ à l'Acad.\ des Sciences}, 1856, \r14. p.~658~sqq.}; he
also refers to the striking confirmation afforded by the fact that
\hyperref[prop:12_2]{\book{12}{No.~2}} contains the solution of the
problem of inscribing in a given circle a polygon similar to a polygon
inscribed in another circle, which problem Eutocius
says\footnote{Archimedes, ed.\ Heiberg, \r3. p.~28, 19–22.} that
Pappus gave in his commentary on the \emph{Elements}.

But, on the other hand, Schol.\ Vat.\ contain some things which cannot
have come from Pappus, e.g.\ the allusion in \hyperref[prop:10_1]{\book{10}{No.~1}} to Theon
and irrational surfaces and solids, Theon being later than Pappus;
\hyperref[prop:3_10]{\book{3}{No.~10}} about porisms is more like Proclus' treatment of the
subject than Pappus', though one expression recalls that of Pappus
about \emph{forming} (\greek{σχηματίζεσθαι}) the enunciations of
porisms like those of either theorems or problems.

The Schol.\ Vat.\ give us important indications as regards the text of
the EUmtntt as Pappus had it. In particular, they show that he could
not have had in his text certain of the lemmas in Book~\r10.\ For
example, three of these are identical with what we find in
Schol.\ Vat.\ (the lemma to \prop{10}{17} = Schol.\ \r10.\ No.~106,
and the lemmas to \prop{10}{54, 60} come in Schol.\ \r10.\ No.~328);
and it is not possible to suppose that these lemmas, if they were
already in the text, would also be given as scholia. Of these three
lemmas, that before \prop{10}{60} has already been condemned for other
reasons; the other two, unobjectionable in themselves, must be
rejected on the ground now stated. There were four others against
which Heiberg found nothing to urge when writing his prolegomena to
Vol.~\r5., viz.\ the lemmas before \prop{10}{42}, \prop{10}{14},
\prop{10}{2}Z and \prop{10}{33}. Of these, the lemma to \prop{10}{22}
is not reconcilable with Schol.~\r10.\ No.~161, which takes up the
assumption in the text of Eucl.\ \prop{10}{22} as if no lemma had gone
before.  The lemma to \prop{10}{42}, which, on account of the words
introducing it (see p.~\pageref{60} above), Heiberg at first hesitated
to regard as an interpolation, is identical with Schol.~\r10.\ No~
270.  It is true that in Schol.~\r10.\ No.~269 we find the words
``this lemma has been proved before (\greek{ἐν τοῖς ἔμπροσθεν}), but
it shall also be proved now for convenience' sake (\greek{τοῦ ἑτοίμου
  ἕνεκα})'' and it is possible to suppose that ``before'' may mean in
Euclid's text \emph{before} \prop{10}{42}; but a proof in that place
would surely have been as ``convenient'' as could be desired, and it
is therefore more probable that the proof had been given by
\emph{Pappus} in some earlier place. (It may be added that the lemma
to \prop{10}{14}, which is identical with the lemma to \prop{11}{23},
condemned on other grounds, is for that reason open to suspicion.)

Heiberg's conclusion is that \emph{all} the lemmas are spurious, and
that most or all of them have found their way into the text from
Pappus' commentary, though at a time anterior to Theon's edition,
since they are found in all our \textsc{mss.}  This enables us to fix
a date for these interpolations, namely the first half of the 4th~c.

Of course Pappus had not in his text the interpolations which, from
the fact of their appearing only in some of our \textsc{mss.}, are
seen to be later than those above-mentioned. Such are the lemmas which
are found in the text of~\r5 only after \prop{10}{29} and
\prop{10}{31} respectively and are given in Heiberg's Appendix to
Book~\r10.\ (numbered 10 and~11).  On the other hand it appears from
Woepcke's tract\footnote{Woepcke, \emph{op.\ cit.}\ p.~702.} that
Pappus already had \prop{10}{115} in his text: though it does not
follow from this that the proposition is genuine but only that
interpolations began very early.

Theon interpolated a proposition (or lemma) between \prop{10}{12} and
\prop{10}{13} (No.~5 in Heiberg's Appendix). Schol.\ Vat.\ has the
same thing (\hyperref[prop:10_125]{\book{10}{No.~125}}). The writer of the scholia therefore
did not find this lemma in the text.  Schol.\ Vat.~\r9. Nos.~28, 29
show that neither did he find in his text the alterations which Theon
made in Eucl.\ \prop{9}{19}; the scholia in fact only agree with the
text of~P, not with Theon's.  This suggests that Schol.\ Vat.\ were
written for use with a \textsc{ms.}\ of the ante-Theonine recension
such as~F is. This probability is further confirmed by a certain
independence which P~shows in several places when compared with the
Theonine \textsc{mss.}  Not only has P better readings in some
passages, but more substantial divergences; and, in particular, the
absence in~P of three notes of a historical character which are added,
wholly or partly from Proclus, in the Theonine \textsc{mss.}\ attests
an independent and more primitive point of view in~P.

In view of the distinctive character of~P, it is possible that some of
the scholia found in it in the first hand, but not in the other
sources of Schol.\ Vat., also belong to that collection; and several
circumstances confirm this. Schol.~\r13.\ No.~45, found in P only,
which relates to a passage in Eucl.\ \prop{13}{13}, shows that certain
words in the text, though older than Theon, are interpolated; and, as
the scholium is itself older than Theon, is headed ``third lemma,''
and follows a ``second lemma'' relating to a passage in the text
immediately preceding, which ``second lemma'' belongs to
Schol.\ Vat.\ and is taken from Pappus, the ``third'' in all
probability came from Pappus also. The same is true of
Schol.\ \hyperref[prop:12_72]{\book{12}{No.~72}} and \hyperref[prop:13_69]{\book{13}{No.~69}}, which are
respectively identical with the propositions vulgo \prop{11}{38}
(Heiberg, App.\ to Book \hyperref[prop:11_3]{\book{11}{No.~3}}) and \prop{13}{6}; for both
of these interpolations are older than Theon. Moreover most of the
scholia which P in the first hand alone has are of the same character
as Schol,\ Vat. Thus \hyperref[prop:7_7]{\book{7}{No.~7}} and \hyperref[prop:13_1]{\book{13}{No.~1}} introducing
Books~\r7.\ and \r13.\ respectively are of the same historical
character as several of Schol.\ Vat.; that \hyperref[prop:7_7]{\book{7}{No.~7}} appears in
the \emph{text} of~P at the beginning of Book~\r7.\ constitutes no
difficulty.  There are a number of \emph{converses}, remarks on the
relation of propositions to one another, explanations such as
\hyperref[prop:12_89]{\book{12}{No.~89}} in which it is remarked that \greek{Φ}, \greek{Ω},
in Euclid's figure to \prop{12}{17} ($Z$, $V$ in my figure) are really
the same point but that this makes no difference in the proof. Two
other Schol.~P on \prop{12}{17} are connected by their headings with
\hyperref[prop:12_72]{\book{12}{No.~72}} mentioned above, \hyperref[prop:11_10]{\book{11}{No.~10}} (P) is only
another form of \hyperref[prop:11_11]{\book{11}{No.~11}} (B); and B often, alone with~P, has
preserved Schol.\ Vat.  On the whole Heiberg considers some 40 scholia
found in~P alone to belong to Schol.\ Vat.

The history of Schol.\ Vat.\ appears to have been, in its main
outlines, the following. They were put together after 500~\ad, since
they contain extracts from Proclus, to which we ought not to assign a
date too near to that of Proclus' work itself; and they must at least
be earlier than the latter half of the 9th~c., in which B was written.
As there must evidently have been several intermediate links between
the archetype and~B, we must assign them rather to the first half of
the period between the two dates, and it is not improbable that they
were a new product of the great development of mathematical studies at
the end of the 6th~c.\ (Isidorus of Miletus). The author extracted
what he found of interest in the commentary of Proclus on
Book~\r1.\ and in that of Pappus on the rest of the work, and put
these extracts in the margin of a \textsc{ms.}\ of the class of~P.\@
AS there are no scholia to \prop{1}{1–22}, the first leaves of the
archetype or of one of the earliest copies must have been lost at an
early date, and it was from that mutilated copy that partly P and
partly a \textsc{ms.}\ of the Theonine class were taken, the scholia
being put in the margin in both. Then the collection spread through
the Theonine \textsc{mss.}, gradually losing some scholia which could
not be read or understood, or which were accidentally or deliberately
omitted.  Next it was extracted from one of these \textsc{mss.}\ and
made into a separate work which has been preserved, in part, in its
entirety (Vat.~192 etc.)\ and, in part, divided into sections, so that
ihe scholia to Books~\r10.–\r13.\ were detached (V\tsup{c}). It had
the same fate in the \textsc{mss.}\ which kept the original
arrangement (in the margin), and in consequence there are some
\textsc{mss.}\ where the scholia to the stereometric Books are
missing, those Books having come to be less read in the period of
decadence. It is from one of these \textsc{mss.}\ that the collection
was extracted as a separate work such as we find it in
Vat.\ (10th~c.).

II. The second great division of the scholia is \textbf{Schol.\ Vind.}

This title is taken from the Viennese \textsc{ms.}\ (V), and the
letters used by Heiberg to indicate the sources here in question are
as follows.

V\tsup{a} = scholia in V written by the same hand that copied the
\textsc{ms.}\ itself from fol.~235 onward.

q = scholia of the Paris \textsc{ms.}\ 2344 (q) written by the first
hand.

l = scholia of the Florence \textsc{ms.}\ Laurent \r38,~2 written in
the 13th–14th~c., mostly in the first hand, but partly in two later
hands.

V\tsup{b} = scholia in~V written by the same hand as the first part
(leaves 1–183) of the \textsc{ms.}\ itself; V\tsup{b} wrote his
scholia after~V\tsup{a}.

q\tsup{1} = scholia of the Paris \textsc{ms.}\ (q) found here and
there in another hand of early date.

Schol.\ Vind.\ include scholia found in V\tsup{a}q. l~is nearly
related to~q; and in fact the three \textsc{mss.}\ which, so far as
Euclid's text is concerned, show no direct interdependence, are. as
regards their scholia, derived from one original. Heiberg proves this
by reference to the readings of the three in two passages (found in
Schol.~\hyperref[prop:1_109]{\book{1}{No.~109}} and \hyperref[prop:10_39]{\book{10}{No.~39}} respectively). The
common source must have contained, besides the scholia found in the
three \textsc{mss.}\ V\tsup{a}ql, those also which are contained in
two of them, for it is more unlikely that two of the three should
contain common interpolations than that a particular scholium should
drop out of one of them.  Besides V\tsup{a} and~q, the scholia
V\tsup{b} and q\tsup{1} must equally be referred to Schol.\ Vind.,
since the greater part of their scholia are found in~l. There is a
lacuna in~q from Eucl.\ \prop{8}{25} to \prop{9}{14}, so that for this
portion of the Elements Schol.\ Vind.\ are represented by~\r6\ only,
Heiberg gives about 450 numbers in all as belonging to this
collection.

Schol.\ Vind.\ did not all come from one source; this is shown by
differences of substance, e.g.\ between \book{10}{Nos.~\prop*{10}{36} and \prop*{10}{39}}, and
by differences of time of writing: e.g.\ \hyperref[prop:6_52]{\book{6}{No.~52}} refers at
the beginning to No.~55 with the words ``as the scholium has it'' and
is therefore later than that scholium; \hyperref[prop:10_247]{\book{10}{No.~247}} is also
later than \hyperref[prop:10_246]{\book{10}{No.~246}}.

The scholia to Book~\r1.\ are here also extracts from Proclus, but
more copious and more verbatim than in Schol.\ Vat.  The author has
not always understood Proclus; and he had a text as bad as that of our
\textsc{mss.}, with the same lacunae. The scholia to the other Books
are partly drawn (1)~from Schol.\ Vat., the
\textsc{mss.}\ representing Schol.\ Vind., and Schol.\ Vat.\ in these
cases showing nearly all possible combinations; but there is no
certain trace in Schol.\ Vind.\ of the scholia peculiar to P\@. The
author used a copy of Schol.\ Vat.\ in the form in which they were
attached to the Theonine text; thus Schol.\ Vind.\ correspond to BF
Vat., where these diverge from~P, and especially closely to~B. Besides
Schol.\ Vat., the editors of Schol.\ Vind.\ used (2)~other old
collections of scholia of which we find traces in B and~F;
Schol.\ Vind.\ have also some scholia common with~b.  The scholia
which Schol.\ Vind.\ have in common with BF come from two different
sources, and were apparently afterwards introduced into the other
\textsc{mss.}; one result of this is that several scholia are
reproduced twice.

But, besides the scholia derived from these sources,
Schol.\ Vind.\ contain a large number of others of late date,
characterised by incorrect language or by triviality of content (there
are many examples in numbers, citations of propositions used, absurd
\greek{ἀπορίαι}, and the like). Unlike Schol.\ Vat., these scholia
often quote words from Euclid as a heading (in one case a heading is
inserted in Schol.\ Vind.\ where a scholium without the heading is
quoted from Schol.\ Vat., see~\hyperref[prop:5_14]{\book{5}{No.~14}}). The explanations
given often presuppose very little knowledge on the part of the reader
and frequently contain obscurities and gross errors.

Schol.\ Vind.\ were collected for use with a \textsc{ms.}\ of the
Theonine class; this follows from the fact that they contain a note on
the proposition \emph{vulgo} \prop{7}{22} interpolated by Theon (given
in Heiberg's App.~to Vol.~\r2.\ p.~430), Since the scholium to
\prop{12}{39} given in V and~p in the text after the title of
Book~\r8.\ quotes the proposition as \prop{7}{39}, it follows that
this scholium must have been written before the interpolation of the
two propositions \emph{vulgo} \prop{7}{20, 22}; Schol.\ Vind.\ contain
(\hyperref[prop:VII_80]{\book{7}{No.~80}}) the first sentence of it,
but without the heading referring to
\prop{7}{39}. Schol.\ \hyperref[prop:7_97]{\book{7}{No.~97}} quotes
\prop{7}{33} as \prop{7}{34}, so that the proposition \emph{vulgo}
\prop{7}{22} may have stood in the scholiast's text but not the later
interpolation \emph{vulgo} \prop{7}{20} (later because only found in~B
in the margin by the first hand). Of course the scholiast had also the
interpolations earlier than Theon.

For the date of the collection we have a lower limit in the date
(12th~c.) of \textsc{mss.}\ in which the scholia appear. That it was
not much earlier than the 12th~c.\ is indicated (1)~by the poverty of
its contents, (2)~by the quality of the \textsc{ms.}\ of Proclus which
was used in the compilation of it (the Munich \textsc{ms.}\ used by
Friedlein with which the scholiast's excerpts are essentially in
agreement belongs to the 11th–12th~c.), (3)~by the fact that
Schol.\ Vind.\ appear only in \textsc{mss.}\ of the 12th~c.\ and no
trace of them is found in our \textsc{mss.}\ belonging to the
9th–10th~c.\ in which Schol.\ Vat.\ are found. The collection may
therefore probably be assigned to the 11th~c. Perhaps it may be in
part due to Psellus who lived towards the end of that century: for in
a Florence \textsc{ms.}\ (Magliabecch. \r11,, 53 of the
15th~c.)\ containing a mathematical compendium intended for use in the
reading of Aristotle the scholia \book{1}{Nos.~\prop*{1}{40} and
  \prop*{1}{49}} appear with the name of Psellus attached.

Schol.\ Vind.\ are not found without the admixture of foreign elements
in any of our three sources. In~l there are only very few such in the
first hand. In~q there are several new scholia in the first hand, for
the most part due to the copyist himself. The collection of scholia on
Book~\r10.\ in~q (Heiberg's q\tsup{c}) is also in the first hand; it
is not original, and it may perhaps be due to Psellus (Maglb.\ has
some definitions of Book~\r10.\ with a heading ``scholia of…Michael
Psellus on the definitions of Euclid's 10th \emph{Element}'' and
Schol.\ \hyperref[prop:10_9]{\book{10}{No.~9}}), whose name must have been attached to it in
the common source of Maglb.\ and~q; to a great extent it consists of
extracts from Schol.\ Vind.\ taken from the same source as~\r6.  The
scholia q\tsup{1} (in an ancient hand in~q), confined to Book~\r2.,
partly belong to Schol.\ Vind.\ and partly correspond to b\tsup{1}
(Bologna \textsc{ms.}), q\tsup{a} and~q\tsup{b} are in one hand
(Theodorus Antiochita), the nearest to the first hand of~q; they are
doubtless due to an early possessor of the \textsc{ms.}\ of whom we
know nothing more.

V\tsup{a} has, besides Schol.\ Vind., a number of scholia which also
appear in other \textsc{mss.}, one in BFb, some others in~P, and some
in~v (Codex Vat.\ 1038, 13th~c.); these scholia were taken from a
source in which many abbreviations were used, as they were often
misunderstood by V\tsup{a}.  Other scholia in V\tsup{a} which are not
found in the older sources—some appearing in V\tsup{a} alone—are also
not original, as is proved by mistakes or corruptions which they
contain; some others may be due to the copyist himself.

V\tsup{b} seldom has scholia common with the other older sources; for
the most part they either appear in V\tsup{b} alone or only in the
later sources as v or~F\tsup{2} (later scholia in~F), some being
original, others not.  In Book~\r10.\ V\tsup{b} has three series of
numerical examples, (1)~with Greek numerals, (2)~alternatives added
later, also mostly with Greek numerals, (3)~with Arabic numerals. The
last class were probably the work of the copyist himself. These
examples (cf.\ p.~\pageref{74} below) show the facility with which the
Byzantines made calculations at the date of the
\textsc{ms.}\ (12th~c.). They prove also that the use of the Arabic
numerals (in the East-Arabian form) was thoroughly established in the
12th~c.; they were actually known to the Byzantines a century earlier,
since they appear, in the first hand, in an Escurial \textsc{ms.}\ of
the 11th~c.

Of collections in other hands in~V distinguished by Heiberg (see
preface to Vol.~\r5.), V\tsup{1} has very few scholia which are found
in other sources, the greater part being original; V\tsup{2}, V\tsup{3}
are the work of the copyist himself; V\tsup{4} are so in part only,
and contain several scholia from Schol.\ Vat.\ and other sources.
V\tsup{3} and V\tsup{4} are later than 13th–14th~c., since they are
not found in~f (cod.\ Laurent~\r38, 6) which was copied from~V and
contains, besides V\tsup{a} V\tsup{b}, the greater part of V\tsup{1}
and \hyperref[prop:6_20]{\book{6}{No.~20}} of V\tsup{2} (in the text).

In~P there are, besides P\tsup{3} (a quite late hand, probably one of
the old Scriptores Graeci at the Vatican), two late hands (P\tsup{2}),
one of which has some new and independent scholia, while the other has
added the greater part of Schol.\ Vind., partly in the margin and
partly on pieces of leaves stitched on.

Our sources for Schol.\ Vat.\ also contain other elements. In~P there
were introduced a certain number of extracts from Proclus, to
supplement Schol.\ Vat.\ to Book~\r1.; they are all written with a
different ink from that used for the oldest part of the \textsc{ms.},
and the text is inferior. There are additions in the other sources of
Schol.\ Vat.\ (F and~B) which point to a common source for~FB and
which are nearly all found in other \textsc{mss.}, and, in particular,
in Schol.\ Vind., which also used the same source; that they are not
assignable to Schol.\ Vat.\ results only from their not being found in
Vat. Of other additions in~F, some are peculiar to~F and some common
to it and~b; but they are not original. F\tsup{2} (scholia in a later
hand in~F) contains three original scholia; the rest come from
V\@. B~contains, besides scholia common to it and~F, b or other
sources, several scholia which seem to have been put together by
Arethas, who wrote at least a part of them with his own hand.

Heiberg has satisfed himself, by a closer study of~b, that the scholia
which he denotes by~b, \greek{β} and b\tsup{1} are by one hand; they
are mostly to be found in other sources as well, though some are
original.  By the same hand (Theodoras Cabasilas, 15th~c.) are also
the scholia denoted by b\tsup{2}, B\tsup{2}, b\tsup{3} and B\tsup{3}.
These scholia come in great part from Schol.\ Vind., and in making
these extracts Theodoras probably used one of our sources,~l, mistakes
in which often correspond to those of Theodoras. To one scholium is
attached the name of Demetrius (who must be Demetrius Cydonius, a
friend of Nicolaus Cabasilas, 14th~c.); but it could not have been
written by him, since it appears in B and Schol.\ Vind.  Nor are all
the scholia which bear the name of Theodoras due to Theodoras himself,
though some are so.

As B\tsup{3} (a late hand in~B) contains several of the original
scholia of~b\tsup{2}, B\tsup{3} must have used b itself as his source,
and, as all the scholia in B\tsup{3} are in~b, the latter is also the
source of the scholia in~B\tsup{3} which are found in other
\textsc{mss.}  B and~b were therefore, in the 15th~c., in the hands of
the same person; this explains, too, the fact that b in a late hand
has some scholia which can only come from~B\@. We arrive then at the
conclusion that Theodoras Cabasilas, in the 15th~c., owned both the
\textsc{mss.}\ B and~b, and that he transferred to B scholia which he
had before written in~b, either independently or after other sources,
and inversely transferred some scholia from B to~b. Further, B\tsup{2}
are earlier than Theodoras Cabasilas, who certainly himself wrote
B\tsup{3} as well as b\tsup{2} and b\tsup{3}.

An author's name is also attached to the scholia \scholia{6}{No. 6}
and \scholia{10}{No.~223}, which are attributed to Maximus Planudes
(end of 13th~c.)\ along with scholia on \prop{1}{31}, \prop{10}{14}
and \prop{10}{18} found in~l in a quite late hand and published on
pp.~46, 47 of Heiberg's dissertation. These seem to have been taken
from lectures of Planudes on the \emph{Elements} by a pupil who used~l
as his copy.

There are also in~l two other Byzantine scholia, written by a late
hand, and bearing the names Ioannes and Pediasimus respectively; these
must in like manner have been written by a pupil after lectures of
Ioannes Pediasimus (first half of 14th~c.), and this pupil must also
have used~l.

Before these scholia were edited by Heiberg, very few of them had been
published in the original Greek. The Basel \emph{editio princeps} has
a few (\scholia{5}{No.~1}, \scholia{6}{Nos.\ 3, 4} and some in
Book~\r10.)\ which are taken, some from the Paris
\textsc{ms.}\ (Paris.\ Gr.~2343) used by Grynaeus, others probably
from the Venice \textsc{ms.}\ (Marc.~301) also used by him; one
published by Heiberg, not in his edition of Euclid but in his paper on
the scholia, may also be from Venet.~301, but appears also in
Paris.\ Gr.~2342. The scholia in the Basel edition passed into the
Oxford edition in the text, and were also given by August in the
Appendix to his Vol.~\r2.

Several specimens of the two series of scholia (Vat.\ and Vind.)\ were
published by C.~Wachsmuth (\emph{Rhein.\ Mus.}\ \r18.\ p.~132
sqq.)\ and by Knoche (\emph{Untersuchungen über die neu aufgefundenen
  Scholien des Proklus}, Herford, 1865).

The scholia published in Latin were much more numerous.  G.~Valla
(\emph{De expetendis et fugiendis rebus}, 1501) reproduced apparently
some 200 of the scholia included in Heiberg's edition.  Several of
these he obtained from two Modena \textsc{mss.}\ which at one time
were in his possession (Mutin.\ \r3~B, 4 and \r2~E, 9, both of the
15th~c.); but he must have used another source as well, containing
extracts from other series of scholia, notably Schol.\ Vind.\ with
which he has some 87 scholia in common. He has also several that are
new.

Commandinus included in his translation under the title ``Scholia
antiqua ``the greater part of the Schol.\ Vat.\ which he certainly
obtained from a \textsc{ms.}\ of the class of Vat.~192; on the whole
he adhered closely to the Greek text. Besides these scholia
Commandinus has the scholia and lemmas which he found in the Basel
\emph{editio princeps}, and also three other scholia not belonging to
Schol.\ Vat., as well as one new scholium (to \prop{12}{13}) not
included in Heiberg's edition, which are distinguished by different
type and were doubtless taken from the Greek \textsc{ms.}\ used by him
along with the Basel edition.

In Conrad Dasypodius' \emph{Lexicon matkematicum} published in 1573
there is (on fol.~42–44) ``Graecum scholion in definitiones Euclidis
libri quinti elementorum appendicis loco propter pagellas vacantes
annexum.'' This contains four scholia, and part of two others,
published in Heiberg's edition, with some variations of readings, and
with some new matter added (for which see pp.~64–6 of Heiberg's
pamphlet). The source of these scholia is revealed to us by another
work of Dasypodius, \emph{Isaaci Monachi Scholia in Euclidis
  elementorum geometriae sex priores libros per C.~Dasypodium in
  latinum sermonem translata et in lucem edita} (1579). This work
contains, besides excerpts from Proclus on Book~\r1.\ (in part closely
related to Schol.\ Vind.), some 30 scholia included in Heiberg's
edition, several new scholia, and the above-mentioned scholia to the
definitions of Book~\r5.\ published in Greek in 1573. After the
scholia follow ``Isaaci Monachi prolegomena in Euclidis Elementorum
geometriae libros'' (two definitions of geometry) and ``Varia
miscellanea ad geometriae cognitionem necessaria ab Isaaco Monacho
collecta'' (mostly the same as pp.~252, 24–272, 27 in the \emph{Variae
  Collectiones} included in Hultsch's Heron); lastly, a note of
Dasypodius to the reader says that these scholia were taken ``ex
clarissimi viri Joannis Sambuci antiquo codice manu propria Isaaci
Monachi scripto.'' Isaak Monachus is doubtless Isaak Argyrus, 14th~c.;
and Dasypodius used a \textsc{ms.}\ in which, besides the passage in
Hultsch's \emph{Variae Collectiones}, there were a number of scholia
marked in the margin with the name of Isaak (cf.\ those in~b under the
name of Theodorus Cabasilas). Whether the new scholia are original
cannot be decided until they are published in Greek; but it is not
improbable that they are at all events independent arrangements of
older scholia. All but five of the others, and all but one of the
Greek scholia to Book~\r5., are taken from Schol.\ Vat.; three of the
excepted ones are from Schol.\ Vind., and the other three seem to come
from~F (where some words of them are illegible, but can be supplied by
means of Mut.\ \r3~B, 4, which has chese three scholia and generally
shows a certain likeness to Isaak's scholia).

Dasypodius also published in 1564 the arithmetical commentary of
Barlaam the monk (14th c.) on Eucl.\ Book~\r2., which finds a place in
Appendix~\r4.\ to the Scholia in Heiberg's edition.

Hultsch has some remarks on the origin of the
scholia\footnote{Art.\ ``Eukleides'' in Pauly-Wissowa's
  \emph{Real-Encyclopädie}.}.  He observes that the scholia to
Book~\r1.\ contain a considerable portion of Geminus' commentary on
the definitions and are specially valuable because they contain
extracts from Geminus only, whereas Proclus, though drawing mainly
upon him, quotes from others as well. On the postulates and axioms the
scholia give more than is found in Proclus.  Hultsch conjectures that
the scholium on Book \r5., No.~3, attributing the discovery of the
theorems to Eudoxus but their arrangement to Euclid, represents the
tradition going back to Geminus, and that the scholium
\scholia{13}{No.~1}, has the same origin.

A word should be added about the numerical illustrations of Euclid's
propositions in the scholia to Book~\r10. x They contain a large
number of calculations with sexagesimal fractions\footnote{Hultsch has
  written upon these in \emph{Bibliotheca Mathematica}, \r5\tsub{3},
  1904, pp.~225–233.}; the fractions go as far as
\emph{fourth-sixtieths} ($1/60^4$).  Numbers expressed in these
fractions are handled with skill and include some results of
surprising accuracy\footnote{Thus $\sqrt{(27)}$ is given (allowing for
  a slight correction by means of the context) as $5$ $11'$ $46''$
  $10'''$, which gives for $\sqrt{3}$ the value $1$ $43'$ $55''$
  $23'''$, being the same value as that given by Hipparchus in his
  Table of Chords, and correct to the seventh decimal place. Similarly
  $\sqrt{8}$ is given as $2$ $49'$ $42''$ $20'''$ $10''''$, which is
  equivalent to $\sqrt{2} = 1.41421335$.  Hultsch gives instances of
  the various operations, addition, subtraction, etc., carried out in
  these fractions, and shows how the extraction of the square root was
  effected. Cf.\ T.~L. Heath, \emph{History of Greek Mathematics},
  \r1., pp.~59–63.}.

\chapter{Euclid in Arabia}

We are told by Ḥājī Khalfa\footnote{\emph{Lexicon bibliogr.\ et
    encyclop.}\ ed.\ Flügel, \v3. pp.~91, 92.} that the Caliph
al-Mansūr (754–775) sent a mission to the Byzantine Emperor as the
result of which he obtained from him a copy of Euclid among other
Greek books, and again that the Caliph al-Ma'mūn (813–833) obtained
manuscripts of Euclid, among others, from the Byzantines. The version
of the \emph{Elements} by al-Ḥajjāj b.~Yusuf b.~Matar is, if not the
very first, at least one of the first books translated from the Greek
into Arabic\footnote{Klamroth, \emph{Zeitschrift der Deutschen
    Morgenländischen Gesellschaft}, \r35. p.~303.}.  According to the
\emph{Fihrist}\footnote{\emph{Fihrist} (tr.\ Suter), p.~16.} it was
translated by al-Ḥajjāj twice; the first translation was known as
``Hārūni'' (``for Hārūn''), the second bore the name ``Ma'mūni''
(``for al-Ma'mūn'') and was the more trustworthy.  Six Books of the
second of these versions survive in a Leiden \textsc{ms.}\ (Codex
Leidensis 399,~1) now in part published by Besthorn and
Heiberg\footnote{\emph{Codex Leidensis} 399, 1. \emph{Euclidis
    Elementa ex interpretatione al-Hadschdschadschii cum commentariis
    al-Narizii}, Hauniae, part \r1. i.\ 1893, part \r1. ii.\ 1897,
  part \r2. i\ 1900, part \r2. ii.\ 1905, part \r3. i.\ 1910.}.  In
the preface to this \textsc{ms.}\ it is stated that, in the reign of
Hārūn ar-Rashīd (786–809), al-Ḥajjāj was commanded by Yaḥyā b.~Khālid
b.~Barmak to translate the book into Arabic. Then, when al-Ma'mūn
became Caliph, as he was devoted to learning, al-Ḥajjāj saw that he
would secure the favour of al-Ma'mūn ``if he illustrated and expounded
this book and reduced it to smaller dimensions. He accordingly left
out the superfluities, filled up the gaps, corrected or removed the
errors, until he had gone through the book and reduced it, when
corrected and explained, to smaller dimensions, as in this copy, but
without altering the substance, for the use of men endowed with
ability and devoted to learning, the earlier edition being left in the
hands of readers.''

The \emph{Fihrist} goes on to say that the work was next translated by
Isḥāq b.~Ḥunain, and that this translation was improved by Thābit
b.~Qurra. This Abū Ya`qūb Isḥāq b.~Ḥunain b.~Isḥāq al-`Ibādī (d.~910)
was the son of the most famous of Arabic translators, Hunain b. Isḥāq
al-`Ibādī (809–873), a Christian and physician to the Caliph
al-Mutawakkil (847–861). There seems to be no doubt that Isḥāq, who
must have known Greek as well as his father, made his translation
direct from the Greek. The revision must apparently have been the
subject of an arrangement between Isḥāq and Thābit, as the latter died
in 901 or nine years before Isḥāq, Thābit undoubtedly consulted Greek
\textsc{mss.}  for the purposes of his revision. This is expressly
stated in a marginal note to a Hebrew version of the \emph{Elements},
made from Isḥāq's, attributed to one of two scholars belonging to the
same family, viz.\ either to Moses b.~Tibbon (about 1244–1274) or to
Jakob b.~Machir (who died soon after 1306)\footnote{Steinschneider,
  \emph{Zeitschrift für Math.\ u.\ Physik}, \r31.,
  hist.-litt.\ Abtheilung, pp.~85, 86, 99.}.  Moreover Thābit
observes, on the proposition which he gives as \prop{9}{31}, that he
had not found this proposition and the one before it in the Greek but
only in the Arabic; from which statement Klamroth draws two
conclusions, (1)~that the Arabs had already begun to interest
themselves in the authenticity of the text and (2)~that Thābit did not
alter the numbers of the propositions in Isḥāq's
translation\footnote{Klamroth, p.~279.}. The \emph{Fihrist} also says
that Yuḥannā al-Qass (i.e.\ ``the Priest'') had seen in the Greek copy
in his possession the proposition in Book~\r1.\ which Thābit took
credit for, and that this was confirmed by Naẓīf, the physician, to
whom Yuhannā had shown it.  This proposition may have been wanting in
Isḥāq, and Thābit may have added it, but without claiming it as his
own discovery\footnote{Steinschneider, p.~88.}. As a fact,
\prop{1}{45} is missing in the translation by al-Ḥajjāj.

The original version of Isḥāq \emph{without} the improvements by
Thābit has probably not survived any more than the first of the two
versions by al-Ḥajjāj; the divergences between the \textsc{mss.}\ are
apparently due to the voluntary or involuntary changes of copyists,
the former class varying according to the degree of mathematical
knowledge possessed by the copyists and the extent to which they were
influenced by considerations of practical utility for teaching
purposes\footnote{Klamroth, p.~306.}. Two \textsc{mss.}\ of the
Isḥāq-Thābit version exist in the Bodleian Library (No.~279 belonging
to the year 1238, and No.~280 written in 1260–1)\footnote{These
  \textsc{mss.} are described by Nicoll and Pusey, \emph{Catatogus
    cod.\ mss.\ orient,\ bibl.\ Bodleianae}, pt.~\r2. 1835
  (pp.~257—262).}; Books \r1.–\r13.\ are in the Isḥāq-Thābit version,
the non-Euclidean Books \r14., \r15.\ in the translation of Qusṭā
b.~Lūqā at-Ba`labakki (d.~about 912). The first of these
\textsc{mss.}\ (No.~279) is that (O) used by Klamroth for the purpose
of his paper on the Arabian Euclid. The other \textsc{ms.}\ used by
Klamroth is (K) Kjøbenhavn~\v81, undated but probably of the 13th~c,
containing Books \r5.–\r15., Books \r5.–\r10.\ being in the
Isḥāq-Thābit version, Books \r11.–\r13.\ purporting to be in
al-Ḥajjāj's translation, and Books \r14, \r15.\ in the version of
Qusṭā b.~Lūqā.  In not a few propositions K and O show not the
slightest difference, and, even where the proofs show considerable
differences, they are generally such that, by a careful comparison, it
is possible to reconstruct the common archetype, so that it is fairly
clear that we have in these cases, not two recensions of one
translation, but arbitrarily altered and shortened copies of one and
the same recension\footnote{Klamroth, pp.~306–8.}. The Bodleian
\textsc{ms.}\ No.~280 contains a preface, translated by Nicoll, which
cannot be by Thābit himself because it mentions Avicenna (980–1037)
and other later authors. The \textsc{ms.}\ was written at Marāġa in
the year 1260–1 and has in the margin readings and emendations from
the edition of Naṣīraddīn aṭ-Ṭūsī (shortly to be mentioned) who was
living at Marāġa at the time, is it possible that aṭ-Ṭūsī himself is
the author of the preface\footnote{Steinschneider, p.~98. Heiberg has
  quoted the whole of this preface in the \emph{\ZMP}, \r29.,
  hist.-litt.\ Abth.\ p.~16.}? Be this as it may, the preface is
interesting because it throws light on the liberties which the
Arabians allowed themselves to take with the text. After the
observation that the book (in spite of the labours of many editors) is
not free from errors, obscurities, redundancies, omissions etc., and
is without certain definitions necessary for the proofs, it goes on to
say that the man has not yet been found who could make it perfect, and
next proceeds to explain (1)~that Avicenna ``cut out postulates and
many Definitions'' and attempted to clear up difficult and obscure
passages, (2)~that Abū 'l Wafā al-Būzjānī (939–99?) ``introduced
unnecessary additions and left out many things of great importance and
entirely necessary,'' inasmuch as he was too long in various places in
Book~\r6.\ and too short in Book~\r10.\ where he left out entirely the
proofs of the \emph{apotomae}, while he made an unsuccessful attempt
to emend \prop{12}{14}, (3)~that Abū Ja`far al-Khāzin (d.~between 961
and 971) arranged the postulates excellently but ``disturbed the
number and order of the propositions, reduced several propositions to
one'' etc.  Next the preface describes the editor's own
claims\footnote{This seems to include a rearrangement of the contents
  of Books \r14., \r15. added to the \emph{Elements}.} and then ends
with the sentences, ``But we have kept to the order of the books and
propositions in the work itself (i.e.\ Euclid's) except in the twelfth
and thirteenth books. For we have dealt in Book~\r13.\ with the
(solid) bodies and in Book~\r12.\ with the surfaces by themselves.''

After Thābit the \emph{Fihrist} mentions Abū `Uthmān ad-Dimashqī as
having translated some Books of the \emph{Elements} including
Book~\r10.  (It is Abū `Uthmān's translation of Pappus' commentary on
Book~\r10.\ which Woepcke discovered at Paris.) The \emph{Fihrist}
adds also that ``Naẓīf the physician told me that he had seen the
tenth Book of Euclid in Greek, that it had 40 propositions more than
the version in common circulation which had 109 propositions, and that
he had determined to translate it into Arabic.''

But the third form of the Arabian Euclid actually accessible to us is
the edition of Abū Ja`far Muḥ.\ b.~Muḥ. b.~al-Ḥasan Naṣīraddīn aṭ-Ṭūsī
(whom we shall call aṭ-Ṭūsī for short), born at Ṭūs (in Khurāsān) in
1201 (d.~1274). This edition appeared in two forms, a larger and a
smaller. The larger is said to survive in Florence only (Pal.~272 and
313, the latter \textsc{ms.}\ containing only six Books); this was
published at Rome in 1594, and, remarkably enough, some copies of this
edition are to be found with 12 and some with 13 Books, some with a
Latin title and some without\footnote{Suter, \emph{Die Mathematiker
    und Astronomen der Araber}, p.~151. The Latin title is
  \emph{Euclidis elementorum geometricorum libri tredecim. Ex
    traditione doctissimi Nasiridini Tusini nunc primum arabice
    impressi}. Romae in typographia Medicea \textsc{mdxciv}. Cum
  licentia superiorum.}.  But the book was printed in Arabic, so that
Kästner remarks that he will say as much about it as can be said about
a book which one cannot read\footnote{Kästner, \emph{Geschichte der
    Mathematik}, \r1. p.~367.}.  The shorter form, which however, in
most \textsc{mss.}, is in 15 Books, survives at Berlin, Munich,
Oxford, British Museum (974, 1334\footnote{Suter has a note that this
  \textsc{ms.} is very old, having been copied from the original in
  the author's lifetime.}, 1335), Paris (2465, 2466), India Office,
and Constantinople; it was printed at Constantinople in 1801, and the
first six Books at Calcutta in 1824\footnote{Suter, p.~151.}.

Aṭ-Ṭūsī's work is however not a \emph{translation} of Euclid's text,
but a re-written Euclid based on the older Arabic translations. In
this respect it seems to be like the Latin version of the
\emph{Elements} by Campanus (Campano), which was first published by
Erhard Ratdolt at Venice in 1482 (the first printed edition of
Euclid\footnote{Described by Kästner, \emph{Geschichte der
    Mathematik}, \r1. pp.~289–299, and by Weissenbom, \emph{Die
    Ubersetzungen des Euklid durch Campano und Zamberti}, Halle
  a.\ S., 1882, pp.~1–7. See also \emph{infra}, Chapter~\r8,
  p.~97.}). Campanus (13th~c.)\ was a mathematician, and it is likely
enough that he allowed himself the same liberty as aṭ-Ṭūsī in
reproducing Euclid.  Whatever may be the relation between Campanus'
version and that of Athelhard of Bath (about 1120), and whether, as
Curtze thinks\footnote{Sonderabdruck des \emph{Jahresberichtes über
    die Fortschritte der klassischen Alterhumswissenschaft vom
    Okt}. 1879–1882, Berlin, 1884.}, they both used one and the same
Latin version of 10th–11th~c., or whether Campanus used Athelhard's
version in the same way as aṭ-Ṭūsī used those of his
predecessors\footnote{Klamroth, p.~271.}, it is certain that both
versions came from an Arabian source, as is evident from the
occurrence of Arabic words in them\footnote{Curtze, \emph{op.~cit.}
  p.~20; Heiberg, \emph{Euklid-Studien}, p.~178.}. Campanus' version
is not of much service for the purpose of forming a judgment on the
relative authenticity of the Greek and Arabian tradition; but it
sometimes preserves traces of the purer source, as when it omits
Theon's addition to \prop{6}{33}\footnote{Heiberg's Euclid,
  vol.~\r5. p.~ci.}. A curious circumstance is that, while Campanus'
version agrees with aṭ-Ṭūsī's in the number of the propositions in all
the genuine Euclidean Books except \r5.\ and \r9., it agrees with
Athelhard's in having 34 propositions in Book~\r5.\ (as against 25 in
other versions), which confirms the view that the two are not
independent, and also leads, as Klamroth says, to this dilemma: either
the additions to Book~\r5.\ are Athelhard's own, or he used an Arabian
Euclid which is not known to us\footnote{Klamroth, pp.~273—4.}.
Heiberg also notes that Campanus' Books \r14., \r15.\ show a certain
agreement with the preface to the Thābit-Isḥāq version, in which the
author claims to have (1)~given a method of inscribing spheres in the
five regular solids, (2)~carried further the solution of the problem
how to inscribe any one of the solids in any other and (3)~noted the
cases where this could not be done\footnote{Heiberg, \emph{\ZMP},
  \r29,, hist.-litt.\ Abtheilung, p.~21.}.

With a view to arriving at what may be called a common measure of the
Arabian tradition, it is necessary to compare, in the first place, the
numbers of propositions in the various Books. Ḥājī Khalfa says that
al-Ḥajjāj's translation contained 468 propositions, and Thābit's 478;
this is stated on the authority of aṭ-Ṭūsī, whose own edition
contained 468\footnote{Klamroth, p.~274; Steinschneider, \emph{\ZMP},
  \r31., hist.-litt.\ Abth.\ p.~98.}.  The fact that Thābit's version
had 478 propositions is confirmed by an index in the Bodleian
\textsc{ms.}\ 279 (called O by Klamroth). A register at the beginning
of the Codex Leidensis 399,~1 which gives Isḥāq's numbers (although
the translation is that of al-Ḥajjāj) apparently makes the total 479
propositions (the number in Book~\r14.\ being apparently 11, instead
of the 10 of~O\footnote{Besthorn-Heiberg read ``11?'' as the number,
  Klamroth had read it as 21 (p.~273).}). I subjoin a table of
relative numbers taken from Klamroth, to which I have added the
corresponding numbers in August's and Heiberg's editions of the Greek
text

\begin{table}[H]
\def\={\relax\rule{12pt}{1pt}}
\begin{tabular}{crrrrrr}
& \multicolumn{3}{c}{The Arabian Euclid}
& \multicolumn{3}{c}{The Greek Euclid}\\
Books & Isḥāq & aṭ-Ṭūsī & Campanus & Gregory & August & Heiberg\\
\r1   & 48    & 48      & 48       & 48      & 48     & 48 \\
\r2   & 14    & 14      & 14       & 14      & 14     & 14 \\
\r3   & 36    & 36      & 36       & 37      & 37     & 37 \\
\r4   & 16    & 16      & 16       & 16      & 16     & 16 \\
\r5   & 25    & 25      & 34       & 25      & 25     & 25 \\
\r6   & 33    & 32      & 32       & 33      & 33     & 33 \\
\r7   & 39    & 39      & 39       & 41      & 41     & 39 \\
\r8   & 27    & 25      & 25       & 27      & 27     & 27 \\
\r9   & 38    & 36      & 39       & 36      & 36     & 36 \\
\r10  & 109   & 107     & 107      & 117     & 116    & 115 \\
\r11  & 41    & 41      & 41       & 40      & 40     & 39 \\
\r12  & 15    & 15      & 15       & 18      & 18     & 18 \\
\r13  & 21    & 18      & 18       & 18      & 18     & 18 \\
      & \= & \= & \=   & \=  & \= & \= \\
      & 462   & 452     & 464      & 470     & 469    & 465 \\\relax
[\r14 & 10    & 10      & 18       &  7      &        &  ? \\
\r15  &  6    &  6      & 14       & 10      &        &    \\
      & \= & \= & \=   & \=  & \= & \= \\
      & 478   & 468     & 495      & 487     &        &  ?]
\end{tabular}
\end{table}

The numbers in the case of Heiberg include all propositions which he
has printed in the text; they include therefore \prop{13}{6} and
\prop{3}{12} now to be regarded as spurious, and \prop{10}{112–115}
which he brackets as doubtful. He does not number the propositions in
Books \r14., \r15., but I conclude that the numbers in~P reach at
least 9 in \r14., and 9 in~\r15.

The \emph{Fihrist} confirms the number 109 for Book~\r10., from which
Klamroth concludes that Isḥāq's version was considered as by far the
most authoritative.

In the text of~O, Book~\r4.\ consists of 17 propositions and
Book~\r14.\ of 12, differing in this respect from its own table of
contents; \prop{4}{15, 16} in~O are really two proofs of the same
proposition.

In al-Ḥajjāj's version Book~\r1.\ consists of 47 propositions only,
\prop{1}{45} being omitted. It has also one proposition fewer in
Book~\r3., the Heronic proposition \prop{3}{12} being no doubt
omitted.

In speaking of particular propositions, I shall use Heiberg's
numbering, except where otherwise stated.

The difference of 10 propositions between Thābit-Isḥāq and aṭ-Ṭūsī is
accounted for thus:

(1)~The three propositions \prop{6}{12} and \prop{10}{28, 29} which
both Isḥāq and the Greek text have are omitted in aṭ-Ṭūsī,

(2)~Isḥāq divides each of the propositions \prop{13}{1–3} into two,
making six instead of three in aṭ-Ṭūsī and in the Greek.

(3)~Isḥāq has four propositions (numbered by him \prop{8}{24, 25},
\prop{9}{30, 31}) which are neither in the Greek Euclid nor in
aṭ-Ṭūsī.  Apart from the above differences al-Ḥajjāj (so far as we
know), Isḥāq and aṭ-Ṭūsī agree, but their Euclid shows many
differences from our Greek text. These differences we will classify as
follows\footnote{See Klamroth, pp.~275–6, 280, 282—4, 314–15, 326;
  Heibeig, vol.~\r5. pp.~xcvi, xcvii.}.

\section{Prepositions}

The Arabian Euclid omits \prop{7}{20, 22} of Gregory's and August's
editions (Heiberg, App.~to Vol.~\r2. pp.~428–32); \prop{8}{16, 17};
\prop{10}{7, 8, 13, 16, 24, 112, 113, 114}, besides a lemma
\emph{vulgo} \prop{10}{13}, the proposition \prop{10}{117} of
Gregory's edition, and the scholium at the end of the Book (see for
these Heiberg's Appendix to Vol.~\r3.\ pp.~382, 408–416);
\prop{11}{38} in Gregory and August (Heiberg, App.~to
Vol.~\r4.\ p.~354); \prop{12}{6, 13, 14}; (also all but the first
third of Book~\r15.).

The Arabian Euclid makes \prop{3}{11, 12} into one proposition, and
divides some propositions (\prop{10}{31, 32}; \prop{11}{31, 34};
\prop{13}{1–3}) into two each.

The order is also changed in the Arabic to the following extent.
\prop{5}{12, 13} are interchanged and the order in Books \r6., \r7.,
\r9.–\r13.~is

\r6. 1–8, 13, 11, 12, 9, 10, 14–17, 19, 20, 18, 21, 22, 24, 26, 23,
25, 27–30, 32, 31, 33.

\r7. 1–20, 22, 21, 23–28, 31, 32, 29, 30, 33–39.

\r9. 1–13, 20, 14–19, 21–25, 27, 26, 28–36, with two new propositions
coming before prop.~30.

\r10. 1–6, 9–12. 15, 14, 17–23, 26–28, 25, 29–30, 31, 32, 33—111, 115.

\r11. 1–30, 31, 32, 34, 33, 35–39.

\r12. 1–5, 7, 9, 8, 10, 12, 11, 15, 16–18.

\r13. 1–3, 5, 4, 6, 7, 12, 9, 10, 8, 11, 13, 15, 14. 16–18.

\section{Definitions}

The Arabic omits the following definitions: \book{4}{Deff.~3–7},
\book{7}{Def.~9} (or~10), \book{11}{Deff.~5–7, 15, 17. 23, 25–28}; but
it has the spurious definitions \book{6}{Deff.~2, 5}, and those of
\emph{proportion} and \emph{ordered proportion} in
Book~\r5.\ (Deff.~8, 19 August), and wrongly interchanges
\book{5}{Deff.~11, 12} and also \book{6}{Deff.~3, 4}.

The order of the definitions is also different in Book~\r7.\ where,
after Def.~11, the order is 12, 14, 13, 15, 16, 19, 20, 17, 18, 21,
22, 23, and in Book~\r11.\ where the order is 1, 2, 3, 4, 8, 10, 9, 13,
14, 16, 12, 21, 22, 18, 19, 20, 11, 24.

\section{Lemmas and porisms}

All are omitted in the Arabic except the porisms to \prop{6}{8},
\prop{8}{2}, \prop{10}{3}; but there are slight additions here and
there, not found in the Greek, e.g.\ in \prop{8}{14, 15} (in~K).

\section{Alternative proofs}

These are all omitted in the Arabic, except that in \prop{10}{105,
  106} they are substituted for the genuine proofs; but one or two
alternative proofs are peculiar to the Arabic (\prop{6}{32} and
\prop{8}{4, 6}).

The analyses and syntheses to \prop{13}{1–5} are also omitted in the
Arabic.

Klamroth is inclined, on a consideration of all these differences, to
give preference to the Arabian tradition over the Greek (1)~``on
historical grounds,'' subject to the proviso that no Greek
\textsc{ms.}\ as ancient as the 8th~c.\ is found to contradict his
conclusions, which are based generally (2)~on the improbability that
the Arabs would have omitted so much if they had found it in their
Greek \textsc{mss.}, it being clear from the \emph{Fihrist} that the
Arabs had already shown an anxiety for a pure text, and that the old
translators were subjected in this matter to the check of public
criticism. Against the ``historical grounds,'' Heiberg is able to
bring a considerable amount of evidence\footnote{Heiberg in
  \emph{\ZMP}, \r29., hist.-litt.\ Abth.\ p.~3~sqq.}.  First of all
there is the British Museum palimpsest~(L) of the 7th or the beginning
of the 8th~c. This has fragments of propositions in Book~\r10.\ which
are omitted in the Arabic; the numbering of one proposition, which
agrees with the numbering in other Greek \textsc{ms.}, is not
comprehensible on the assumption that eight preceding propositions
were omitted in it, as they are in the Arabic; and lastly, the
readings in~L are tolerably like those of our \textsc{mss.}, and
surprisingly tike those of~B. It is also to be noted that, although P
dates from the 10th~c.\ only, it contains, according to all
appearance, an ante-Theonine recension.

Moreover there is positive evidence against certain omissions by the
Arabians. Aṭ-Ṭūsī omits \prop{6}{12}, but it is scarcely possible
that, if Eutocius had not had it, he would have quoted \prop{6}{23} by
that number\footnote{Apollonius, ed.\ Heiberg, vol.~\r2. p.~218,
  3—5.}.  This quotation of \prop{6}{23} by Eutocius also tells
against Isḥāq who has the proposition as \prop{6}{25}.  Again,
Simplicius quotes \prop{6}{10} by that number, whereas it is
\prop{6}{13} in Isḥāq; and Pappus quotes, by number, \prop{13}{2}
(Isḥāq 3, 4), \prop{13}{3, 4} (Isḥāq~8), \prop{13}{16} (Isḥāq~19).  On
the other hand the contraction of \prop{3}{11, 12} into one
proposition in the Arabic tells in favour of the Arabic.

Further, the omission of certain porisms in the Arabic cannot be
supported; for Pappus quotes the porism to
\prop{13}{17}\footnote{Pappus, \r5.~p.~436, 5.}, Proclus those to
\prop{2}{4}, \prop{3}{1}, \prop{7}{2}\footnote{Proclus, pp.~303–4.}
and Simplicius that to \prop{4}{15}.

Lastly, some propositions omitted in the Arabic are required in later
propositions. Thus \prop{10}{13} is used in \prop{10}{18, 22, 23, 26}
etc.; \prop{10}{17} is wanted in \prop{10}{18, 26, 36}; \prop{12}{6,
  13} are required for \prop{12}{11} and \prop{13}{15} respectively.

It must also be remembered that some of the things which were properly
omitted by the Arabians are omitted or marked as doubtful in Greek
\textsc{mss.}\ also, especially in~P, and others are rightly suspected
for other reasons (e.g.\ a number of alternative proofs, lemmas, and
porisms, as well as the analyses and syntheses of \prop{12}{1–5}. On
  the other hand, the Arabic has certain interpolations peculiar to
  our inferior \textsc{mss.}\ (cf.\ the definition \book{6}{Def.~2}
  and those of \emph{proportion} and \emph{ordered proportion}),

Heiberg comes to the general conclusion that, not only is the Arabic
tradition not to be preferred offhand to that of the Greek
\textsc{mss.}, but it must be regarded as inferior in authority. It is
a question how far the differences shown in the Arabic are due to the
use of Greek \textsc{mss.}\ differing from those which have been most
used as the basis of our text, and how far to the arbitrary changes
made by the Arabians themselves. Changes of order and arbitrary
omissions could not surprise us, in view of the preface above quoted
from the Oxford \textsc{ms.}\ of Thābit-Isḥāq, with its allusion to
the many important and necessary things left out by Abū 'l Wafā and to
the author's own rearrangement of Books \r12., \r13.  But there is
evidence of differences due to the use by the Arabs of other Greek
\textsc{mss.}  Heiberg\footnote{\emph{\ZMP}, \r29.,
  hist.-litt.\ Abth.\ p.~6~sqq.} is able to show considerable
resemblances between the Arabic text and the Bologna \textsc{ms.}~b in
that part of the \textsc{ms.}\ where it diverges so remarkably from
our other \textsc{mss.}\ (see the short description of it above,
p.~49); in illustration he gives a comparison of the proofs of
\prop{12}{7} in~b and in the Arabic respectively, and points to the
omission in both of the proposition given in Gregory's edition as
\prop{11}{38}, and to a remarkable agreement between them as regards
the order of the propositions of Book~\r12.  As above stated, the
remarkable divergence of~b only affects Books~\r11.\ (at end) and
\r12.; and Book~\r13.\ in~b shows none of the transpositions and other
peculiarities of the Arabic. There are many differences between b and
the Arabic, especially in the definitions of Book~\r11, as well as in
Book~\r13.  It is therefore a question whether the Arabians made
arbitrary changes, or the Arabic form is the more ancient, and b has
been altered through contact with other \textsc{mss.}  Heiberg points
out that the Arabians must be alone responsible for their definition
of a prism, which only covers a prism with a triangular base. This
could not have been Euclid's own, for the word \emph{prism} already
has the wider meaning in Archimedes, and Euclid himself speaks of
prisms with parallelograms and polygons as bases (\prop{11}{39};
\prop{12}{10}). Moreover, a Greek would not have been likely to leave
out the definitions of the ``Platonic ``regular solids.

Heiberg considers that the Arabian translator had before him a
\textsc{ms.}\ which was related to~b, but diverged still further from
the rest of our \textsc{mss.}  He does not think that there is
evidence of the existence of a redaction of Books \r1.–\r10.\ similar
to that of Books \r11., \r12.\ in~b; for Klamroth observes that it is
the Books on solid geometry (\r11.–\r13) which are more remarkable
than the others for omissions and shorter proofs, and it is a
noteworthy coincidence that it is just in these Books that we have a
divergent text in~b.

An advantage in the Arabic version is the omission of
\book{7}{Def.~10}, although, as Iamblichus had it, it may have been
deliberately omitted by the Arabic translator. Another advantage is
the omission of the analyses and syntheses of \prop{13}{1–5}; but
again these may have been omitted purposely, as were evidently a
number of porisms which are really necessary.

One or two remarks may be added about the Arabic versions as compared
with one another. Al-Ḥajjāj's object seems to have been less to give a
faithful reflection of the original than to write a useful and
convenient mathematical text-book. One characteristic of it is the
careful references to earlier propositions when their results are
used. Such specific quotations of earlier propositions are rare in
Euclid; but in al-Ḥajjāj we find not only such phrases as ``by
prop.\ so and so,'' ``which was proved'' or ``which we showed how to
do in prop.\ so and so,'' but also still longer phrases. Sometimes he
\emph{repeats} a construction, as in \prop{1}{44} where, instead of
constructing ``the parallelogram $BEFG$ equal to the triangle~$C$ in
the angle~$EBG$ which is equal to the angle~$D$)'' and placing it in a
certain position, he produces $AB$ to~$G$, making $BG$ equal to half
$DE$ (the base of the triangle $CDE$ in his figure), and on $GB$ so
constructs the parallelogram $BHKG$ by \prop{1}{42} that it is equal
to the triangle $CDE$, and its angle $GBH$ is equal to the given
angle.

Secondly, al-Ḥajjāj, in the arithmetical books, in the theory of
proportion, in the applications of the Pythagorean \prop{1}{47}, and
generally where possible, illustrates the proofs by numerical
examples. It is true, observes Klamroth, that these examples are not
apparently separated from the commentary of an-Nairīzī, and might not
therefore have been due to al-Ḥajjāj himself; but the marginal notes
to the Hebrew translation in Munich \textsc{ms.}~36 show that these
additions were in the copy of al-Ḥajjāj used by the translator, for
they expressly give these proofs in numbers as variants taken from
al-Ḥajjāj\footnote{Klamroth, p.~310; Steinschneider, pp.~85–6.}.

These characteristics, together with al-Ḥajjāj's freer formulation of
the propositions and expansion of the proofs, constitute an
intelligible reason why Isḥāq should have undertaken a fresh
translation from the Greek.  Klamroth calls Isḥāq's version a model of
a good translation of a mathematical text; the introductory and
transitional phrases are stereotyped and few in number, the technical
terms are simply and consistently rendered, and the less formal
expressions connect themselves as closely with the Greek as is
consistent with intelligibility and the character of the Arabic
language. Only in isolated cases does the formulation of definitions
and enunciations differ to any considerable extent from the
original. In general, his object seems to have been to get rid of
difficulties and unevennesses in the Greek text by neat devices, while
at the same time giving a faithful reproduction of
it\footnote{Klamroth, p.~290, illustrates Isḥāq's method by his way of
  distinguishing \greek{ἐφαρμόζειν} (to be congruent with) and
  \greek{e)farmózesqai} (to be applied to), the confusion of which by
  translators was animadverted on by Savile.  Isḥāq avoided the
  confusion by using two entirely different words.}.

There are curious points of contact between the versions of al-Ḥajjāj
and Thābit-Isḥāq. For example, the definitions and enunciations of
propositions are often word for word the same.  Presumably this is
owing to the fact that Isḥāq found these definitions and enunciations
already established in the schools in his time, where they would no
doubt be learnt by heart, and refrained from translating them afresh,
merely adopting the older version with some changes\footnote{Klamroth,
  pp.~310–1.}.  Secondly, there is remarkable agreement between the
Arabic versions as regards the figures, which show considerable
variations from the figures of the Greek text, especially as regards
the letters; this is also probably to be explained in the same way,
all the later translators having most likely borrowed al-Ḥajjāj's
adaptation of the Greek figures\footnote{\ibid~p.~287.}.  Lastly, it
is remarkable that the version of Books \r11.–\r13.\ in the Kjøbenhavn
\textsc{ms.}~(K), purporting to be by al-Ḥajjāj, is almost exactly the
same as the Thābit-Isḥāq version of the same Books in~O\@.  Klamroth
conjectures that Isḥāq may not have translated the Books on solid
geometry at all, and that Thābit took them from al-Ḥajjāj, only making
some changes in order to fit them to the translation of
Isḥāq\footnote{\ibid~pp.~304—5.}.

From the facts (1)~that aṭ-Ṭūsī's edition had the same number of
propositions (468) as al-Ḥajjāj's version, while Thābit-Isḥāq's had
478, and (2)~that aṭ-Ṭūsī has the same careful references to earlier
propositions, Klamroth concludes that aṭ-Ṭūsī deliberately preferred
al-Ḥajjāj's version to that of Isḥāq\footnote{\ibid~p.~274.}.
Heiberg, however, points out (1)~that aṭ-Ṭūsī left out \prop{6}{12}
which, if we may judge by Klamroth's silence, al-Ḥajjāj had, and
(2)~al-Ḥajjāj's version had one proposition less in Books~\r1.\ and
\r3.\ than aṭ-Ṭūsī has.  Besides, in a passage quoted by Ḥājī
Khalfa\footnote{Hājī Khalfa, \r1. p.~383.} from aṭ-Ṭūsī, the latter
says that ``he separated the things which, in the approved editions,
were taken from the archetype from the things which had been added
thereto,'' indicating that he had compiled his edition from
\emph{both} the earlier translations\footnote{Heiberg, \emph{\ZMP},
  \r29., hist.-litt.\ Abth.\ pp.~2,~3.}.

There were a large number of Arabian commentaries on, or reproductions
of, the \emph{Elements} or portions thereof, which will be found fully
noticed by Steinschneide\footnote{Steinschneider, \emph{\ZMP}, \r31.,
  hist.-litt.\ Abth.\ pp.~86~sqq.}.  I shall mention here the
commentators etc.\ referred to in the \emph{Fihrist}, with a few
others.

1.~Abū 'l `Abbās al-Faḍl b.~Ḥātim \textbf{an-Nairīzī} (born at Nairīz,
died about 922) has already been mentioned\footnote{Steinschneider,
  p.~86, \emph{Fihrist} (tr.\ Suter), pp.~16, 67; Suter, \emph{Die
    Mathematiker und Astronomen der Araber} (1900), p.~45.}. His
commentary survives, as regards Books \r1.–\r6., in the Codex
Leidensis 399,~1, now edited, as to four Books, by Besthorn and
Heiberg, and as regards Books \r1.–\r10.\ in the Latin translation
made by Gherard of Cremona in the 12th~c.\ and now published by Curtze
from a Cracow \textsc{ms.}\footnote{\emph{Supplementum ad Euclidis
    opera omnia}, ed.\ Heiberg and Menge, Leipzig, 1899.} Its
importance lies mainly in the quotations from Heron and Simplicius.

2.~Ahmad b.~(Umar \textbf{al-Karābīsī} (date uncertain, probably
9th–10th~c.), ``who was among the most distinguished geometers and
arithmeticians\footnote{\emph{Fihrist}, pp.~16, 38; Steinschneider,
  p.~87; Suter, p.~65.}.''

3.~Al-`Abbās b.~Sa`īd \textbf{al-Jauharī} (fl.~830) was one of the
astronomical observers under al-Ma'mūn, but devoted himself mostly to
geometry. He wrote a commentary to the whole of the \emph{Elements},
from the beginning to the end; also the ``Book of the propositions
which he added to the first book of Euclid\footnote{\emph{Fihrist},
  pp.~16, 25; Steinschneider, p.~88. Suter, p.~12.}.''

4.~Muh.\ b.~`Īsā Abū `Abdallāh \textbf{al-Māhānī} (d.\ between 874 and
884) wrote, according to the \emph{Fihrist}, (1)~a commentary on
Eucl.\ Book~\r5., (2)~``On proportion,'' (3)~``On the 26 propositions
of the first Book of Euclid which are proved without \emph{reductio ad
  absurdum}\footnote{\emph{Fihrist}, pp.~16, 25, 58.}.''  The work
``On proportion'' survives and is probably identical with, or part of,
the commentary on Book~\r7.\footnote{Suter, p.~26, note, quotes the
  Paris \textsc{ms.} 2467, 16° containing the work ``on proportion''
  as the authority for this conjecture.} He also wrote, what is not
mentioned by the \emph{Fihrist}, a commentary on Eucl.\ Book~\r10., a
fragment of which survives in a Paris \textsc{ms.}\footnote{\text{ms.}
  2457, 39° (cf.\ Woepcke in \emph{Mém.\ prés.\ à l'acad.\ des
    sciences}, \r14., 1856, p.~669).}

5.~Abū Ja`far \textbf{al-Khāzin} (i.e.\ ``the treasurer'' or
``librarian''), one of the first mathematicians and astronomers of his
time, was born in Khurāsān and died between the years 961 and~971. The
\emph{Fihrist} speaks of him as having written a commentary on the
whole of the \emph{Elements}\footnote{\emph{Fihrist}, p.~17.}, but
only the commentary on the beginning of Book~\r10.\ survives (in
Leiden, Berlin and Paris); therefore either the notes on the rest of
the Books have perished, or the \emph{Fihrist} is in
error\footnote{Suter, p.~58, note~b.}.  The latter would seem more
probable, for, at the end of his commentary, al-Khāzin remarks that
the rest had already been commented on by Sulaiman b.~`Uṣma (Leiden
\textsc{ms.})\footnote{Steinschneider, p.~89.} or `Oqba (Surer), to be
mentioned below. Al-Khāzin's method is criticised unfavourably in the
preface to the Oxford \textsc{ms.}\ quoted by Nicoll (see
p.~\pageref{77} above).

6. Abū 'l \textbf{Wafā al-Būzjānī} (940–997), one of the greatest
Arabian mathematicians, wrote a commentary on the \emph{Elements}, but
did not complete it\footnote{\emph{Fihrist}, p.~17.}.  His method is
also unfavourably regarded in the same preface to the Oxford
\textsc{ms.}~280.  According to Ḥājī Khalfa, he also wrote a book on
geometrical constructions, in thirteen chapters.  Apparently a book
answering to this description was compiled by a gifted pupil from
lectures by Abū 'l Wafā, and a Paris \textsc{ms.}\ (Anc.\ fonds 169)
contains a Persian translation of this work, not that of Abū 'l Wafā
himself.  An analysis of the work was given by
Woepcke\footnote{Woepcke, \emph{Journal Asiatique},
  Sér.\ v.\ T.\ v.\ pp.~218–256 and 309–359.}, and some particulars
will be found in Cantor\footnote{\emph{Gesch.\ d.\ Math.}
  vol.~\r1\tsub{3}, pp.~743–6.}. Abū 'l Wafā also wrote a commentary
on Diophantus, as well as a separate ``book of proofs to the
propositions which Diophantus used in his book and to what he (Abū 'l
Wafā) employed in his commentary\footnote{\emph{Fihrist}, p.~39;
  Suter, p.~71.}.''

7. \emph{Ibn Rāhawaihi al-Arjānī} also commented on
Eucl.\ Book~\r10.\footnote{\emph{Fihrist}, p.~17; Suter, p.~17.}

8. `Alī b.~Ahmad Abū 'l-Qāsim \textbf{al-Anṭākī} (d.~987) wrote a
commentary on the whole book\footnote{\emph{Fihrist}, p.~17.}; part of
it seems to survive (from the 5th Book onwards) at Oxford
(Catal.\ \textsc{mss.}\ orient.\ \prop{2}{281})\footnote{Suter, p.~64.}.

9. \textbf{Sind b.\ `Alī} Abū 'ṭ-Ṭaiyib was a Jew who went over to
Islam in the time of al-Ma'mūn, and was received among his
astronomical observers, whose head he became\footnote{\emph{Fihrist},
  p.~17, 29; Suter, pp.~13, 14.} (about 830); he died after 864. He
wrote a commentary on the whole of the \emph{Elements}; ``Abū `Alī saw
nine books of it, and a part of the tenth\footnote{\emph{Fihrist},
  p.~17.}.''  His book ``On the Apotomae and the Medials,'' mentioned
by the \emph{Fihrist}, may be the same as, or part of, his commentary
on Book~\r10.

10. Abū Yūsuf Ya`qūb b.~Muḥ. \textbf{ar-Rāzī} ``wrote a commentary on
Book~\r10., and that an excellent one, at the instance of Ibn
al-`Amīd\footnote{\emph{Fihrist}, p.~17; Suter, p.~66.}.''

11. The \emph{Fihrist} next mentions \textbf{al-Kindī} (Abū Yūsuf
Ya`qūb b.~Isḥāq b.~as-Ṣabbāḥ al-Kindī, d.~about 873), as the author
(1)~of a work ``on the objects of Euclid's book,'' in which occurs the
statement that the \emph{Elements} were originally written by
Apollonius, the carpenter (see above, p.~\pageref{5} and note), (2)~of
a book ``on the improvement of Euclid's work,'' and (3)~of another
``on the improvement of the 14th and 15th Books of Euclid.'' ``He was
the most distinguished man of his time, and stood alone in the
knowledge of the old sciences collectively; he was called `the
philosopher of the Arabians'; his writings treat of the most different
branches of knowledge, as logic, philosophy, geometry, calculation,
arithmetic, music, astronomy and others\footnote{\emph{Fihrist},
  p.~17, 10–15.}.'' Among the other geometrical works of al-Kindī
mentioned by the \emph{Fihrist}\footnote{The mere catalogue of
  al-Kindī's works on the various branches of science takes up four
  octavo pages (11–15) of Suter's translation of the \emph{Fihrist}.}
are treatises on the closer investigation of the results of Archimedes
concerning the measure of the diameter of a circle in terms of its
circumference, on the construction of the figure of the two mean
proportionals, on the approximate determination of the chords of the
circle, on the approximate determination of the chord (side) of the
nonagon, on the division of triangles and quadrilaterals and
constructions for that purpose, on the manner of construction of a
circle which is equal to the surface of a given cylinder, on the
division of the circle, in three chapters etc.

12.~The physician \emph{Naẓīf b.~Yumn} (or Yaman) al-Qass (``the
priest'') is mentioned by the \emph{Fihrist} as having seen a Greek
copy of Eucl.\ Book~\r10.\ which had 40 more propositions than that
which was in general circulation (containing 109), and having
determined to translate it into Arabic\footnote{\emph{Fihrist},
  pp.~16, 17.}.  Fragments of such a translation exist at Paris,
Nos.\ 18 and~34. of the \textsc{ms.}\ 2457 (952, 2 Suppl.\ Arab.\ in
Woepcke's tract); No.~18 contains ``additions to some propositions of
the 10th Book, existing in the Greek language\footnote{Woepcke,
  \emph{Mém.\ prés.\ à l'acad.\ des science}, \r14. pp.~666, 668.}.''
Naẓīf must have died about 990\footnote{Suter, p.~68.},

13.~\textbf{Yūḥannā} b.~Yūsuf b.~al-Ḥārith b.~al-Biṭrīq
\textbf{al-Qass} (d.\ about 980) lectured on the \emph{Elements} and
other geometrical books, made translations from the Greek, and wrote a
tract on the ``proof'' of the case of two straight lines both meeting
a third and making with it, on one side, two angles together less than
two right angles\footnote{\emph{Fihrist}, p.~38; Suter. p.~60.}.
Nothing of his appears to survive, except that a tract ``on rational
and irrational magnitudes,'' No.~48 in the Paris \textsc{ms.}\ just
mentioned, is attributed to him.

14.~Abū Muḥ.\ \textbf{al-Ḥasan b.}\ 'Ubaidallāh b.~Sulaimān
\textbf{b~Wahb} (d.~901) was a geometer of distinction, who wrote
works under the two distinct titles ``A commentary on the difficult
parts of the work of Euclid ``and ``The Book on
Proportion\footnote{\emph{Fihrist}, p.~26, and Suter's note,
  p.~60.}.'' Suter thinks that another reading is possible in the case
of the second title, and that it may refer to the Euclidean work ``on
the divisions (of figures)\footnote{Suter, p.~211, note~23,}.''

15.~\textbf{Qusṭā b.~Lūqā} al-Ba(labakkī (d.~about 912), a physician,
philosopher, astronomer, mathematician and translator, wrote ``on the
difficult passages of Euclid's book'' and ``on the solution of
arithmetical problems from the third book of
Euclid\footnote{\emph{Fihrist}, p.~43,}''; also an ``introduction to
geometry,'' in the form of question and
answer\footnote{\emph{Fihrist}, p.~43; Suter, p.~41.}.

16. \textbf{Thābit b.~Qurra} (826–901), besides translating some parts
of Archimedes and Books \r5.–\r7.\ of the \emph{Conics} of Apollonius,
and revising Isḥāq's translation of Euclid's \emph{Elements}, also
revised the translation of the \emph{Data} by the same Isḥāq and the
book \emph{On divisions of figures} translated by an anonymous
writer. We are told also that he wrote the following works: (1)~On the
Premisses (Axioms, Postulates etc.)\ of Euclid, (2)~On the
Propositions of Euclid, (3)~On the propositions and questions which
arise when two straight lines are cut by a third (or on the ``proof''
of Euclid's famous postulate).  The last tract is extant in the
\textsc{ms.}\ discovered by Woepcke (Paris 2457> 32\tsup{0}). He is
also credited with ``an excellent work'' in the shape of an
``Introduction to the Book of Euclid,'' a treatise on Geometry
dedicated to Ismā̀il b.~Bulbul, a Compendium of Geometry, and a large
number of other works for the titles of which reference may be made to
Suter, who also gives particulars as to which are
extant\footnote{Suter, pp.~34–8.}.

17. Abū Sa`īd \textbf{Sinān} b.~Thābit b.~Qurra, the son of the
translator of Euclid, followed in his father's footsteps as geometer,
astronomer and physician. He wrote an ``improvement of the book of……on
the Elements of Geometry, in which he made various additions to the
original.'' It is natural to conjecture that \emph{Euclid} is the name
missing in this description (by Ibn abī Uṣaibi`a); Casiri has the name
Aqāṭon\footnote{\emph{Fihrist} (ed.\ Suter), p.~59, note 132; Suter,
  p.~52, note~b.}.  The latest editor of the \emph{Ta'rīkh al-Ḥukamā},
however, makes the name to be Iflāton (= Plato), and he refers to the
statement by the \emph{Fihrist} and Ibn al-Qiftī attributing to Plato
a work on the Elements of Geometry translated by Qusṭā. It is just
possible, therefore, that at the time of Qusṭā the Arabs were
acquainted with a book on the Elements of Geometry translated from the
Greek, which they attributed to Plato\footnote{See Suter in
  \emph{Bibliotheca Mathematica} \r4\tsub{3}, 1903–4, pp.~296–7,
  review of Julius Lippert's \emph{Ibn al-Qiftī.  Ta'rīch al-ḥukama=},
  Leipzig, 1903.}. Sinān died in~943.

18.~Abū Sahl Wījan (or Waijan) b.~Rustam \textbf{al-Kūhī} (fl.~988),
  born at Kūh in Tabaristān, a distinguished geometer and astronomer,
  wrote, according to the \emph{Fihrist}, a ``Book of the Elements''
  after that of Euclid\footnote{\emph{Fihrist}, p.~40.}; the 1st and
  2nd Books survive at Cairo, and a part of the 3rd Book at Berlin
  (5922)\footnote{Suter, p.~75,}.  He wrote also a number of other
  geometrical works: Additions to the 2nd Book of Archimedes on the
  Sphere and Cylinder (extant at Paris, at Leiden, and in the India
  Office), On the finding of the side of a heptagon in a circle (India
  Office and Cairo), On two mean proportionals (India Office), which
  last may be only a part of the Additions to Archimedes' On the
  Sphere and Cylinder, etc.

19.~Abū Naṣr Muḥ.\ b.~Muḥ.\ b.~Ṭarkhān b.~Uzlaġ \textbf{al-Fārābī}
(870–950) wrote a commentary on the difficulties of the introductory
matter to Books~\r1.\ and \r5.\footnote{Suter, p.~55.} This appears to
survive in the Hebrew translation which is, with probability,
attributed to Moses b.~Tibbon\footnote{Steinschneider, p.~92.}.

20.~Abū `Alī al-Ḥasan b.~al-Ḥasan \textbf{b.~al-Haitham} (about
965–1039), known by the name Ibn al-Haitham or Abū `Alī al-Baṣrī, was
a man of great powers and knowledge, and no one of his time approached
him in the field of mathematical science. He wrote several works on
Euclid the titles of which, as translated by Woepcke from Usaibi`a,
are as follows\footnote{Steinschneider, pp.~92–3.}:

1.~Commentary and abridgment of the \emph{Elements}.

2.~Collection of the Elements of Geometry and Arithmetic, drawn from
the treatises of Euclid and Apollonius.

3.~Collection of the Elements of the Calculus deduced from the
principles laid down by Euclid in his \emph{Elements}.

4.~Treatise on ``measure'' after the manner of Euclid's
\emph{Elements},

5.~Memoir on the solution of the difficulties in Book~\r1.

6.~Memoir for the solution of a doubt about Euclid, relative to
Book~\r5.

7.~Memoir on the solution of a doubt about the stereometric portion.

8.~Memoir on the solution of a doubt about Book~\r12.

9.~Memoir on the division of the two magnitudes mentioned in
\prop{10}{1} (the theorem of exhaustion).

10.~Commentary on the definitions in the work of Euclid (where
Steinschneider thinks that some more general expression should be
substituted for ``definitions'').

The last-named work (which Suter calls a commentary on the
\emph{Postulates} of Euclid) survives in an Oxford
\textsc{ms.}\ (Catal.\ \textsc{mss.}\ orient.\ \r1.~908) and in
Algiers (1446, 1\tsup{o}).

A Leiden \textsc{ms.}\ (966) contains his Commentary ``on the
difficult places ``up to Book~\r5.  We do not know whether in this
commentary, which the author intended to form, with the commentary on
the Musādarāt, a sort of complete commentary, he had collected the
separate memoirs on certain doubts and difficult passages mentioned in
the above list.

A commentary on Book~\r5.\ and following Books found in a Bodleian
\textsc{ms.}\ (Catal.\ \r2.\ p.~262) with the title ``Commentary on
Euclid and solution of his difficulties'' is attributed to b.~Haitham;
this might be a continuation of the Leiden \textsc{ms.}

The memoir on \prop{10}{1} appears to survive at St Petersburg,
\textsc{ms.}\ de l'lnstitut des langues orient. 192, 5° (Rosen,
Catal.\ p.~125).

21. \textbf{Ibn Sīnā}, known as \textbf{Avicenna} (980–1037), wrote a
Compendium of Euclid, preserved in a Leiden \textsc{ms.}\ No.~1445,
and forming the geometrical portion of an encyclopaedic work embracing
Logic, Mathematics, Physics and Metaphysics\footnote{Steinschneider,
  p.~92; Suter, p.~89.}.

22. Ahmad b.~al-Ḥusain \textbf{al-Ahwāzī} al-Kātib wrote a commentary
on Book~\r10., a fragment of which (some 10 pages) is to be found at
Leiden (970), Berlin (5923) and Paris (2467, 18°)\footnote{Suter,
  p.~57.}.

25.~Naṣīraddīn \textbf{aṭ-Ṭūsī} (1201–1274) who, as we have seen,
brought out a Euclid in two forms, wrote:

1.~A treatise on the postulates of Euclid (Paris, 2467, 5\tsup{o}).

2.~A treatise on the 5th postulate, perhaps only a part of the
foregoing (Berlin, 5942, Paris, 2467, 6°).

3. Principles of Geometry taken from Euclid, perhaps identical with
No.~1 above (Florence, Pal.~298).

4. 105 problems out of the \emph{Elements} (Cairo). He also edited the
\emph{Data} (Berlin, Florence, Oxford etc.)\footnote{Suter,
  pp.~150–1.},

24. Muḥ. b.~Ashraf Shamsaddīn \textbf{as-Samarqandī} (fl.~1276) wrote
``Fundamental Propositions, being elucidations of 35 selected
propositions of the first Books of Euclid,'' which are extant at Gotha
(1496 and 1497), Oxford (Catal.\ \r1.~967, 2\tsup{o} and
Brit.~Mus.\footnote{Suter, p.~157.}.

25.~Mūsā b.~Muḥ.\ b.~Maḥmūd, known as \textbf{Qāḍīzāde} ar-Rūmī
(i.e.\ the son of the judge from Asia Minor), who died between 1436
and 1446, wrote a commentary on the ``Fundamental Propositions'' just
mentioned, of which many \textsc{mss.}\ are extant\footnote{\ibid~~p.~175.}.  It
contained biographical statements about Euclid alluded to above
(p.~\pageref{5}. note),

26.~Abū Dā'ūd Sulaimān \textbf{b.~`Uqba}, a contemporary of al-Khāzin
(see above, No.~5), wrote a commentary on the second half of
Book~\r10., which is, at least partly, extant at Leiden (974) under
the title ``On the binomials and apotomae found in the 10th Book of
Euclid\footnote{\ibid~p.~56.}.''

27. The Codex Leidensis 399,~1 containing al-Ḥajjāj's translation of
Books \r1.–\r6.\ is said to contain glosses to it by Sa`īd b.~Mas`ūd
b.~al-Qass, apparently identical with Abū Naṣr Ġars al-Na`ma, son of
the physician Mas`ūd b.~al-Qass al-Baġdādī, who lived in the time of
the last Caliph al-Musta`ṣim (d.~1258)\footnote{\ibid~pp.~153–4,
  227.}.

28.~\textbf{Abū Muḥammad b.~Abdalbāqī} al-Baġdādī al-Faraḍī (d.~1141,
at the age of over 70 years) is stated in the \emph{Ta'rīkh al-Ḥukamā}
to have written an excellent commentary on Book~\r10.\ of the
\emph{Elements}, in which he gave numerical examples of the
propositions\footnote{Gartz, p.~14; Steinschneider, pp.~94—5.}. This
is published in Curtze's edition of an-Nairīzī where it occupies pages
252–386\footnote{Suter in \emph{Bibliotheca Mathematica}, \r4\tsub{3},
  1903, pp.~25, 295; Suter has also an article an its contents,
  \emph{Bibliotheca Mathematica}, \r7\tsub{3}, 1906–7, pp.~234–251.}.

29.~Yaḥyā b.~Muḥ.\ b.~`Abdān b.~`Abdalwāḥid, known by the name of Ibn
al-Lubūdī (1210–1268), wrote a Compendium of Euclid, and a short
presentation of the postulates\footnote{Steinschneider, p.~94; Suter,
  p.~146.}.

30.~Abū `Abdallāh Muḥ.\ b.~Mu`ādh al-Jayyānī wrote a commentary on
Eucl.\ Book~\r5.\ which survives at Algiers (1446, 3°)\footnote{Suter,
  \emph{Nachträge und Berichtigungen}, in \emph{Abhandlungen zur
    Gesch.\ der math.\ Wissenschaften}, \r14., 1902, p.~170,}.

31.~Abū Naṣr Mansūr b.~`Alī b.~'Irāq wrote, at the instance of
Muḥ.\ b.~Ahmad Abū 'r-Raiḥān al-Bīrūnī (973–1048), a tract ``on a
doubtful (difficult) passage in Eucl.\ Book~\r13.''\ (Berlin,
5925)\footnote{Suter, p.~81, and \emph{Nachträge}, p.~172.},

\chapter{Principal Translations and Editions of the Elements}

Cicero is the first Latin author to mention Euclid\footnote{\emph{De
    oratore} \r3. 132.}; but it is not likely that in Cicero's time
Euclid had been translated into Latin or was studied to any
considerable extent by the Romans; for, as Cicero says in another
place\footnote{\emph{Tusc.}\ \r1.~5.}, while geometry was held in high
honour among the Greeks, so that nothing was more brilliant than their
mathematicians, the Romans limited its scope by having regard only to
its utility for measurements and calculations.  How very little
theoretical geometry satisfied the Roman agrimensores is evidenced by
the work of Balbus \emph{de mensuris}\footnote{\emph{Gromatici
    veteres}, \r1. 97 sq.\ (ed.\ F. Blume, K. Lachmann and A. Rudorff.
  Berlin, 1848, 1852).}, where some of the definitions of
Eucl.\ Book~\r1.\ are given. Again, the extracts from the
\emph{Elements} found in the fragment attributed to Censorinus
(fl.~238~\ad)\footnote{Censorinus, ed.\ Hultsch, pp.~60–3.} are
confined to the definitions, postulates, and common notions. But by
degrees the \emph{Elements} passed even among the Romans into the
curriculum of a liberal education; for Martianus Capella speaks of the
effect of the enunciation of the proposition ``how to construct an
equilateral triangle on a given straight line'' among a company of
philosophers, who, recognising the first proposition of the
\emph{Elements}, straightway break out into encomiums on
Euclid\footnote{Martianus Capella, \r6. 724.}. But the \emph{Elements}
were then (\emph{c}.~470~\ad)\ doubtless read in Greek; for what
Martianus Capella gives\footnote{\ibid~\r6, 708~sq.} was drawn from a
Greek source, as is shown by the occurrence of Greek words and by the
wrong translation of \book{1}{def.~1} (``punctum vero est cuius pars
\emph{nihil} est''). Martianus may, it is true, have quoted, not from
Euclid himself, but from Heron or some other ancient source.

But it is clear from a certain palimpsest at Verona that some scholar
had already attempted to translate the \emph{Elements} into Latin.
This palimpsest\footnote{Cf.\ Cantor, \r1\tsub{3}, p.~565.} has part
of the ``Moral reflections on the Book of Job ``by Pope Gregory the
Great written in a hand of the 9th~c.\ above certain fragments which
in the opinion of the best judges date from the 4th~c.\ Among these
are fragments of Vergil and of Livy, as well as a geometrical fragment
which purports to be taken from the 14th and 15th Books of Euclid, As
a matter of fact it is from Books~\r12.\ and \r13.\ and is of the
nature of a free rendering, or rather a new arrangement, of Euclid
with the propositions in different order\footnote{The fragment was
  deciphered by W. Studemund, who communicated his results to
  Cantor.}.  The \textsc{ms.}\ was evidently the translator's own
copy, because some words are struck out and replaced by synonyms. We
do not know whether the translator completed the translation of the
whole, or in what relation his version stood to our other sources.

Magnus Aurelius Cassiodorus (b.~about 475~\ad)\ in the geometrical
part of his encyclopaedia \emph{De artibus ac disciplines liberalium
  literarum} says that geometry was represented among the Greeks by
Euclid, Apollonius, Archimedes, and others, ``of whom Euclid was given
us translated into the Latin language by the same great man
Boethius''; also in his collection of letters\footnote{Cassiodorus,
  \emph{Variae}, \r1. 45, p.~40, 12 ed. Mommsen.} is a letter from
Theodoric to Boethius containing the words, ``for in your
translations…Nicomachus the arithmetician, and Euclid the geometer,
are heard in the Ausonian tongue.'' The so-called Geometry of Boethius
which has come down to us by no means constitutes a translation of
Euclid. The \textsc{mss.}\ variously give five, four, three or two
Books, but they represent only two distinct compilations, one normally
in five Books and the other in two. Even the latter, which was edited
by Friedlein, is not genuine\footnote{See especially Weissenborn in
  \emph{Abhandlungen zur Gesch.\ d.\ Math.}\ \r2. p.~185 sq.; Heiberg
  in \emph{Philologus}, \r43. p.~507 sq.; Cantor, \r1\tsub{3}, p.~580
  sq.}, but appears to have been put together in the 11th~c.\ from
various sources. It begins with the definitions of Eucl.~\r1., and in
these are traces of perfectly correct readings which are not found
even in the \textsc{mss.}\ of the 10th~c.\ but which can be traced in
Proclus and other ancient sources; then come the Postulates (five
only), the Axioms (three only), and after these some definitions of
Eucl.\ \r2., \r3., \r4.  Next come the enunciations of Eucl.~\r1., of
ten propositions of Book~\r2., and of some from Books \r3., \r4., but
always without proofs; there follows an extraordinary passage which
indicates that the author will now give something of his own in
elucidation of Euclid, though what follows is a literal translation of
the proofs of Eucl.\ \prop{1}{1–3}. This latter passage, although it
affords a strong argument against the genuineness of this part of the
work, shows that the Pseudoboethius had a Latin translation of Euclid
from which he extracted the three propositions.

Curtze has reproduced, in the preface to his edition of the
translation by Gherard of Cremona of an-Nairīzī's Arabic commentary on
Euclid, some interesting fragments of a translation of Euclid taken
from a Munich \textsc{ms.}\ of the 10th~c. They are on two leaves used
for the cover of the \textsc{ms.}\ (Bibliothecae Regiae Universitatis
Monacensis 2° 757) and consist of portions of Eucl.\ \prop{1}{37, 38}
and \prop{2}{8}, translated literally word for word from the Greek
text. The translator seems to have been an Italian (cf.\ the words
``capitolo nono'' used for the ninth prop, of Book~\r2.)\ who knew
very little Greek and had moreover little mathematical knowledge. For
example, he translates the capital letters denoting points in figures
as if they were numerals: thus \greek{τὰ ΑΒΓ, ΔΕΖ} is translated ``que
primo secundo et tertio quarto quinto et septimo,'' T becomes
``tricentissimo'' and so on. The Greek \textsc{ms.}\ which he used was
evidently written in uncials, for \greek{ΔΕΖΘ} becomes in one place
``quod autem septimo nono,'' showing that he mistook \greek{ΔΕ} for
the particle \greek{δέ}, and \greek{καὶ ὁ ΣΤΥ} is rendered ``sicut
tricentissimo et quadringentissimo,'' showing that the letters must
have been written \greek{ΚΑΙΟCΤU}.

The date of the Englishman Athelhard (Æthelhard) is approximately
fixed by some remarks in his work \emph{Perdifficiles Quaestiones
  Naturales} which, on the ground of the personal allusions they
contain, must be assigned to the first thirty years of the
12th~c.\footnote{Cantor, \emph{Gesch.\ d.\ Math.}\ \r1\tsub{3},
  p.~906.} He wrote a number of philosophical works. Little is known
about his life. He is said to have studied at Tours and Laon, and to
have lectured at the latter school. He travelled to Spain, Greece,
Asia Minor and Egypt, and acquired a knowledge of Arabic, which
enabled him to translate from the Arabic into Latin, among other
works, the \emph{Elements} of Euclid. The date of this translation
must be put at about 1120.  \textsc{mss.}\ purporting to contain
Athelhard's version are extant in the British Museum (Harleian
No.~5404 and others), Oxford (Trin.\ Coll.~47 and Ball.\ Coll.~257 of
12th~c.), Nürnberg (Johannes Regiomontanus' copy) and Erfurt.

Among the very numerous works of Gherard of Cremona (1114–1187) are
mentioned translations of ``15 Books of Euclid ``and of the
\emph{Data}\footnote{Boncompagni, \emph{Della vita e delle opere di
    Gherardo Cremonese}, Rome, 1851, p.~5.}. Till recently this
translation of the \emph{Elements} was supposed to be lost; but Axel
Anthon Bjørnbo has succeeded (1904) in discovering a translation from
the Arabic which is different from the two others known to us (those
by Athelhard and Campanus respectively), and which he, on grounds
apparently convincing, holds to be Gherard's.  Already in 1901 Bjørnbo
had found Books \r10.–\r15.\ of this translation in a \textsc{ms.}\ at
Rome (Codex Reginensis lat.\ 1268 of 14th~c.)\footnote{Described in an
  appendix to \emph{Studien über Menelaos' Sphärik}
  (\emph{Abhandlungen zur Geschichte der mathematischen
    Wissenschaften}, \r14., 1902).}; but three years later he had
traced three \textsc{mss.}\ containing the whole of the same
translation at Paris (Cod.\ Paris.\ 7216, 15th~c.), Boulogne-sur-Mer
(Cod.\ Bononiens.\ 196, 14th~c.), and Bruges (Cod.\ Brugens.\ 521,
14th~c.), and another at Oxford (Cod.\ Digby 174, end of
12th~c.)\ containing a fragment, \prop{11}{2} to~\r14. The occurrence
of Greek words in this translation such as \emph{rombus},
\emph{romboides} (where Athelhard keeps the Arabic terms),
\emph{ambligonius}, \emph{orthogonius}, \emph{gnomo}, \emph{pyramis}
etc., show that the translation is independent of Athelhard's. Gherard
appears to have had before him an old translation of Euclid from the
Greek which Athelhard also often followed, especially in his
terminology, using it however in a very different manner. Again, there
are some Arabic terms, e.g.\ \emph{meguar} for \emph{axis of
  rotation}, which Athelhard did not use, but which is found in almost
all the translations that are with certainty attributed to Gherard of
Cremona; there occurs also the expression ``superficies equidistantium
laterum et rcctorum angulorum,'' found also in Gherard's translation
of an-Nairīzī, where Athelhard says ``parallelogrammum rectangulum.''
The translation is much clearer than Athelhard's: it is neither
abbreviated nor ``edited'' as Athelhard's appears to have been; it is
a word-for-word translation of an Arabic \textsc{ms.}\ containing a
revised and critical edition of Thābit's version. It contains several
notes quoted from Thābit himself (\emph{Thebit dixit}), e.g.\ about
alternative proofs etc.\ which Thābit found ``in another Greek
\textsc{ms.},'' and is therefore a further testimony to Thābit's
critical treatment of the text after Greek \textsc{mss.}  The new
editor also added critical remarks of his own, e.g.\ on other proofs
which he found in other Arabic versions, but not in the Greek: whence
it is clear that he compared the Thābit version before him with other
versions as carefully as Thābit collated the Greek \textsc{mss.}
Lastly, the new editor speaks of ``Thebit qui transtulit hunc librum
in arabicam linguam'' and of ``translatio Thebit,'' which may tend to
confirm the statement of al-Qiftī who credited Thābit with an
independent translation, and not (as the \emph{Fihrist} does) with a
mere improvement of the version of Isḥāq b,~Ḥunain.

% TBD
4 See \emph{Bibliotheca Mathematica}, \r7\tsub{3}, 1905–6, pp.~242–8.

Gherard's translation of the Arabic commentary of an-Nairīzī on the
first ten Books of the Elements was discovered by Maximilian Curtze in
a \textsc{ms.}\ at Cracow and published as a supplementary volume to
Heiberg and Menge's Euclid\footnote{\emph{Anaritii in decem libros
    priores Elementorum Euclidis Commentarii ex interpretatione
    Gherardi Cremonensis in codice Cracoviensi 569 servata} edidit
  Maximilianus Curtze, Leipzig (Teubner), 1899.}: it will often be
referred to in this work.

Next in chronological order comes Johannes Campanus (Campano) of
Novara. He is mentioned by Roger Bacon (1214–1294) as a prominent
mathematician of his time\footnote{Cantor, \r2\tsub{1}, p.~88.}, and
this indication of his date is confirmed by the fact that he was
chaplain to Pope Urban~IV, who was Pope from 1261 to
1281\footnote{Tiraboschi, \emph{Storia della letteratura italiana},
  IV, \r4. 145–160.}.  His most important achievement was his edition
of the \emph{Elements} including the two Books \r14.\ and~\r15\ which
are not Euclid's. The sources of Athelhard's and Campanus'
translations, and the relation between them, have been the subject of
much discussion, which does not seem to have led as yet to any
definite conclusion. Cantor (\r2\tsub{1}, p.~91) gives
references\footnote{H. Weissenborn in \emph{\ZMP}, \r25., Supplement,
  pp.~143–166, and in his monograph. \emph{Die Übersetzungen des
    Euklid durch Campano und Zamberti} (1882); Max.\ Curtze in
  \emph{Philologische Rundschau} (1881), \r1. pp.~943–950, and in
  \emph{Jahresbericht über die Fortschritte der classischen
    Alterthumswissenschaft}, \r40. (1884, \r3.)\ pp.~19–22; Heiberg in
  \emph{\ZMP}, \r35., hist.-litt.\ Abth., pp.~48–58 and pp.~81–6.} and
some particulars. It appears that there is a \textsc{ms.}\ at Munich
(Cod.\ lat.\ Mon.\ 13021) written by Sigboto in the 12th~c.\ at
Pru:fning near Regensburg, and denoted by Curtze by the letter~R,
which contains the enunciations of part of Euclid. The Munich
\textsc{mss.}\ of Athelhard and Campanus' translations have many
enunciations textually identical with those in~R, so that the source
of all three must, for these enunciations, have been the same; in
others Athelhard and Campanus diverge completely from~R, which in
these places follows the Greek text and is therefore genuine and
authoritative. In the 32nd definition occurs the word ``elinuam,'' the
Arabic term for ``rhombus,'' and throughout the translation are a
number of Arabic figures. But R~was not translated from the Arabic, as
is shown by (among other things) its close resemblance to the
translation from Euclid given on pp.~377 sqq.\ of the \emph{Gromatici
  Veteres} and to the so-called geometry of Boethius. The explanation
of the Arabic figures and the word ``elinuam ``in Def.~32 appears to
be that R~was a late copy of an earlier original with corruptions
introduced in many places; thus in Def.~32 a part of the text was
completely lost and was supplied by some intelligent copyist who
inserted the word ``elinuam,'' which was known to him, and also the
Arabic figures. Thus Athelhard certainly was not the first to
translate Euclid into Latin; there must have been in existence before
the 11th~c.\ a Latin translation which was the common source of~R, the
passage in the \emph{Gromatici}, and ``Boethius.'' As in the two
latter there occur the proofs as well as the enunciations of
\prop{1}{1–3}, it is possible that this translation originally
contained the proofs also.  Athelhard must have had before him this
translation of the enunciations, as well as the Arabic source from
which he obtained his proofs. That some sort of translation, or at
least fragments of one, were available before Athelhard's time even in
England is indicated by some old English verses\footnote{Quoted by
  Halliwell in \emph{Rara Mathematica} (p.~56 note) from \textsc{ms.}
  Bib.\ Reg.\ Mus.\ Brit.\ 17 A.~\r1. f.~2\tsup{b}–3.}:
\begin{verse}
``The clerk Euclide on this wyse hit fonde\\
Thys craft of gemetry yn Egypte londe\\
Yn Egypte he tawghte hyt ful wyde,\\
In dyvers londe on every syde.\\
Mony erys after warde y understonde\\
Yer that the craft com ynto thys londe.\\
Thys craft com into England, as y yow say,\\
Yn tyme of good kyng Adelstone's day,''
\end{verse}
which would put the introduction of Euclid into England as far back
as 924–940~\ad.

We now come to the relation between Athelhard and Campanus.  That
their translations were not independent, as Weissenborn would have us
believe, is clear from the fact that in all \textsc{mss.}\ and
editions, apart from orthographical differences and such small
differences as are bound to arise when \textsc{mss.}\ are copied by
persons with some knowledge of the subject-matter, the definitions,
postulates, axioms, and the 364 enunciations are word for word
identical in Athelhard and Campanus; and this is the case not only
where both have the same text as~R but where they diverge from
it. Hence it would seem that Campanus used Athelhard's translation and
only developed the proofs by means of another redaction of the Arabian
Euclid. It is true that the difference between the proofs of the
propositions in the two translations is considerable; Athelhard's are
short and compressed, Campanus' clearer and more complete, following
the Greek text more closely, though still at some distance. Further,
the arrangement in the two is different; in Athelhard the proofs
regularly precede the enunciations, Campanus follows the usual
order. It is a question how far the differences in the proofs, and
certain additions in each, are due to the two translators themselves
or go back to Arabic originals. The latter supposition seems to Curtze
and Cantor the more probable one. Curtze's general view of the
relation of Campanus to Athelhard is to the effect that Athelhard's
translation was gradually altered, from the form in which it appears
in the two Erfurt \textsc{mss.}\ described by Weissenborn, by
successive copyists and commentators \emph{who had Arabic originals
  before them}, until it took the form which Campanus gave it and in
which it was published.  In support of this view Curtze refers to
Regiomontanus' copy of the Athelhard-Campanus translation.  In
Regiomontanus' own preface the title is given, and this attributes the
translation to Athelhard; but, while this copy agrees almost exactly
with Athelhard in Book~\r1., yet, in places where Campanus is more
lengthy, it has similar additions, and in the later Books, especially
from Book \r3.\ onwards, agrees absolutely with Campanus;
Regiomontanus, too, himself implies that, though the translation was
Athelhard's, Campanus had revised it; for he has notes in the margin
such as the following, ``Campani est hec,'' ``dubito an demonstret hie
Campanus'' etc.

We come now to the printed editions of the whole or of portions of the
\emph{Elements}. This is not the place for a complete bibliography,
such as Riccardi has attempted in his valuable memoir issued in five
parts between 1887 and 1893, which makes a large book in
itself\footnote{\emph{Saggio di una Bibliografia Euclidea}, memoria
  del Prof.\ Pietro Riccardi (Bologna, 1887, 1888, 1890, 1893).}.  I
shall confine myself to saying something of the most noteworthy
translations and editions. It will be convenient to give first the
Latin translations which preceded the publication of the \emph{editio
  princeps} of the Greek text in 1533, next the most important
editions of the Greek text itself, and after them the most important
translations arranged according to date of first appearance and
languages, first the Latin translations after 1533, then the Italian,
German, French and English translations in order.

It may be added here that the first allusion, in the West, to the
Greek text as still extant is found in Boccaccio's commentary on the
\emph{Divina Commedia} of Dante\footnote{\r1. p.~404.}.  Next Johannes
Regiomontanus, who intended to publish the \emph{Elements} after the
version of Campanus, but with the latter's mistakes corrected, saw in
Italy (doubtless when staying with his friend Bessarion) some Greek
\textsc{mss.}\ and noticed how far they differed from the Latin
version (see a letter of his written in the year 1471 to Christian
Roder of Hamburg)\footnote{Published in C.~T. de Murr's
  \emph{Memorabilia Bibliothecarum Norimbergensium},
  Part~\r1. p.~190~sqq.}.

\section{Latin translations prior to 1533}

1482. In this year appeared the first printed edition of Euclid, which
was also the first printed mathematical book of any importance. This
was printed at Venice by Erhard Ratdolt and contained Campanus'
translation\footnote{Curtze (An-Nairīzī, p.~xiii) reproduces the
  heading of the first page of the text as follows (there is no
  title-page)\?[old enlish]: Preclarissimū opus elemento\? Euclidis
  megarēsis \=vna cū cōmentis Campani pspicacissimi in artē geometriā
  incipit felicit', after which the definitions begin at once.  Other
  copies have the shorter heading: Preclarissimus liber elementorum
  Euclidis perspicacissimi: in artem Geometrie incipit quam
  foelicissime.  At the end stands the following: \?[XXX] Opus
  elementorū euclidis megarensis in geometriā artē Jn id quoqz\?[eth]
  Campani pspicacissimi Cōmentationes finiūt.  Erhardus ratdolt
  Augustensis impressor solertissimus . venetijs impressit . Anno
  salutis . M.cccc.lxxxij . Octauis . Caleñ . Juñ . Lector . Vale.},
Ratdolt belonged to a family of artists at Augsburg, where he was born
about 1443. Having learnt the trade of printing at home, he went in
1475 to Venice, and founded there a famous printing house which he
managed for 11 years, after which he returned to Augsburg and
continued to print important books until 1516. He is said to have died
in 1528. Kästner\footnote{Kästner, \emph{Geschichte der Mathematik},
  \r1. p.~289 sqq. See also Weissenborn, \emph{Die Übersetzungen des
    Euklid durch CAmpano und Zamberti}, pp.~1–7.} gives a short
description of this first edition of Euclid and quotes the dedication
to Prince Mocenigo of Venice which occupies the page opposite to the
first page of text. The book has a margin of $2\frac{1}{2}$ inches,
and in this margin are placed the figures of the propositions.
Ratdolt says in his dedication that at that time, although books by
ancient and modern authors were printed every day in Venice, little or
nothing mathematical had appeared: a fact which he puts down to the
difficulty involved by the figures, which no one had up to that time
succeeded in printing. He adds that after much labour he had
discovered a method by which figures could be produced as easily as
letters\footnote{``Mea industria non sine maximo labore effect vt qua
  facilitate litterarum elementa imprimuntur ea etiam geometrice
  figure conficerentur.''}. Experts are in doubt as to the nature of
Ratdolt's discovery.  Was it a method of making figures up out of
separate parts of figures, straight or curved lines, put together as
letters are put together to make words? In a life of Joh.\ Gottlob
Immanuel Breitkopf, a contemporary of Kästner's own, this member of
the great house of Breitkopf is credited with this particular
discovery.  Experts in that same house expressed the opinion that
Ratdolt's figures were woodcuts, while the letters denoting points in
the figures were like the other letters in the text; yet it was with
carved wooden blocks that printing began. If Ratdolt was the first to
print geometrical figures, it was not long before an emulator arose;
for in the very same year Mattheus Cordonis of Windischgratz employed
woodcut mathematical figures in printing Oresme's \emph{De
  latitudintbus}\footnote{Curtze in \emph{\ZMP}, \r20.,
  hist.-litt.\ Abth.\ p.~58.}.  How eagerly the opportunity of
spreading geometrical knowledge was seized upon is proved by the
number of editions which followed in the next few years. Even the year
1482 saw two forms of the book, though they only differ in the first
sheet.  Another edition came out in i486 (\emph{Ulmae, apud
  Io.~Regerum}) and another in 1491 \emph{Vincentiae per Leonardum de
  Basilea et Gulielmum de Papia}, but without the dedication to
  Mocenigo who had died in the meantime (1485). If Campanus added
  anything of his own, his additions are at all events not
  distinguished by any difference of type or otherwise; the
  enunciations are in large type, and the rest is printed continuously
  in smaller type.  There are no superscriptions to particular
  passages such as \emph{Euclides ex Campano}, \emph{Campanus},
  \emph{Campani additio}, or \emph{Campani annotatio}, which are found
  for the first time in the Paris edition of 1516 giving both
  Campanus' version and that of Zamberti (presently to be men tioned).

1501. G.~Valla included in his encyclopaedic work \emph{De expetendis
  et fugiendis rebus} published in this year at Venice (\emph{in
  aedibus Aldi Romani}) a number of propositions with proofs and
scholia translated from a Greek \textsc{ms.}\ which was once in his
possession (cod.\ Mutin.\ \r3~B,~4 of the 15th~c.).

1505. In this year Bartolomeo Zamberti (Zambertus) brought out at
Venice the first translation, from the Greek text, of the whole of the
\emph{Elements}. From the title\footnote{The title begins thus:
  ``Euclidis megaresis philosophi platonicj mathematicarum
  disciplinarum Janitoris: Habent in hoc volumine quicunque ad
  mathematicam substantiam aspirant: elementorum libros xiij cum
  expositione Theonis insignis mathematici.  quibus multa quae deerant
  ex lectione gracea sumpta addita sunt nec non plurima peruersa et
  praepostere: voluta in Campani interpretatione: ordinata digesta et
  castigata sunt etc.''  For a description of the book see
  Weissenborn, p.~12 sqq.}, as well as from his prefaces to the
\emph{Catoptrica} and \emph{Data}, with their allusions to previous
translators ``who take some things out of authors, omit some, and
change some,'' or ``to that most barbarous translator'' who filled a
volume purporting to be Euclid's ``with extraordinary scarecrows,
nightmares and phantasies,'' one object of Zamberti's translation is
clear. His animus against Campanus appears also in a number of notes,
e.g.\ when he condemns the terms ``helmuain'' and ``helmuariphe ``used
by Campanus as barbarous, unLatin etc., and when he is roused to wrath
by Campanus' unfortunate mistranslation of \book{5}{Def.~5}.  He does
not seem to have had the penetration to see that Campanus was
translating from an Arabic, and not from a Greek, text.  Zamberti
tells us that he spent seven years over his translation of the
thirteen Books of the \emph{Elements}.  As he seems to have been born
in 1473, and the \emph{Elements} were printed as early as 1500, though
the complete work (including the \emph{Phaenomena}, \emph{Optica},
\emph{Catoptrica}, \emph{Data} etc.)\ has the date 1505 at the end, he
must have translated Euclid before the age of 30. Heiberg has not been
able to identify the \textsc{ms.}\ of the \emph{Elements} which
Zamberti used; but it is clear that it belonged to the worse class of
\textsc{mss.}, since it contains most of the interpolations of the
Theonine variety. Zamberti, as his title shows, attributed the
\emph{proofs} to Theon.

1509. As a counterblast to Zamberti, Luca Paciuolo brought out an
edition of Euclid, apparently at the expense of Ratdolt, at Venice
(\emph{per Paganinum de Paganinis}), in which he set himself to
vindicate Campanus. The title-page of this now very rare
edition\footnote{See Weissenborn, p.~30 sqq.} begins thus: ``The works
of Euclid of Megara, a most acute philosopher and without question the
chief of all mathematicians, translated by Campanus their \emph{most
  faithful interpreter}.'' It proceeds to say that the translation had
been, through the fault of copyists, so spoiled and deformed that it
could scarcely be recognised as Euclid.  Luca Faciuolo accordingly has
polished and emended it with the most critical judgment, has corrected
129 figures wrongly drawn and added others, besides supplying short
explanations of difficult passages. It is added that Scipio Vegius of
Milan, distinguished for his knowledge ``\emph{of both languages}''
(i.e.\ of course Latin and Greek), as well as in medicine and the more
sublime studies, had helped to make the edition more perfect. Though
Zamberti is not once mentioned, this latter remark must have reference
to Zamberti's statement that his translation was from the Greek text;
and no doubt Zamberti is aimed at in the wish of Paciuolo's ``that
others too would seek to acquire knowledge instead of merely showing
off, or that they would not try to make a market of the things of
which they are ignorant, as it were (selling) smoke\footnote{``Atque
  utinam et alii cognoscere vellent non ostentare aut ea quae nesciunt
  veluti fumum venditare non conarentur.''}.''  Weissenborn observes
that, while there are many trivialities in Paciuolo's notes, they
contain some useful and practical hints and explanations of terms,
besides some new proofs which of course are not difficult if one takes
the liberty, as Paciuolo does, of diverging from Euclid's order and
assuming for the proof of a proposition results not arrived at till
later. Two not inapt terms are used in this edition to describe the
figures of \prop{3}{7, 8}, the former of which is called the
\emph{goose's foot} (\emph{pes anseris}), the second the
\emph{peacock's tail} (\emph{cauda pavonis}) Paciuolo as the
\emph{castigator} of Campanus' translation, as he calls himself,
failed to correct the mistranslation of
\book{5}{Def.~5}\footnote{Campanus' translation in Ratdolt's edition
  is. as follows: ``Quantitates quae dicuntur continuam habere
  proportionalitatem, sunt, quarum equè multiplicia aut equa sunt aut
  equè sibi sine interruptione addunt aut minuunut'' (!), to which
  Campanus adds the note: ``Continuè proportionalia sunt quorum omnia
  multiplicia equalia sunt continuè proportionalie.  Sed noluit ipsam
  diffinitionem proponere sub hac forma, quia tunc diffiniret idem per
  idem, aperte (?a parte) tamen rei est istud cum sua diffinitione
  convertibile.''}.  Before the fifth Book he inserted a discourse
which he gave at Venice on the 15th August, 1508, in S. Bartholomew's
Church, before a select audience of 500, as an introduction to his
elucidation of that Book.

1516. The first of the editions giving Campanus' and Zamberti's
translations in conjunction was brought out at Paris (\emph{in
  officina Henrici Stephani e regione scholae Decretorum}). The idea
that only the enunciations were Euclid's, and that Campanus was the
author of the proofs in his translation, while Theon was the author of
the proofs in the Greek text, reappears in the title of this edition;
and the enunciations of the added Books \r14., \r15.\ are also
attributed to Euclid, Hypsicles being credited with the
proofs\footnote{``Euclidis Megarensis Geometricorum Elementorum Libri
  \r15\@.  Campani Galli transalpini in eosdem commentariorum
  libri~\r15\@.  Theonis Alexandrini Bartholomaeo Zamberto Veneto
  interprete, in tredecim priores, commentationum libri~\r13\@.
  Hypsiclis Alexandrini in duos posteriores, eodem Bartholomaeo
  Zamberto Veneto interprete, commentariorum libri~\r2.''  On the last
  page (261) is a similar statement of content, but wish the
  difference that the expression ``ex Campani…deinde Theonis…et
  Hypsiclis…\emph{traditionibus}.''  For description see Weissenborn,
  p.~56~sqq.}.  The date is not on the title-page nor at the end, but
the letter of dedication to François Brieonnet by Jacques Lefevre is
dated the day after the Epiphany, 1516. The figures are in the
margin. The arrangement of the propositions is as follows: first the
enunciation with the heading \emph{Euclides ex Campano}, then the
proof with the note \emph{Campanus}, and after that, as \emph{Campani
  additio}, any passage found in the edition of Campanus' translation
but not in the Greek text; then follows the text of the enunciation
translated from the Greek with the heading \emph{Euclides ex
  Zamberto}, and lastly the proof headed \emph{Theo ex
  Zamberto}. There are separate figures for the two proofs. This
edition was reissued with few changes in 1537 and 1546 at Basel
(\emph{apud Iohannem Hervagium}), but with the addition of the
\emph{Phaenomena}, \emph{Optica}, \emph{Catoptrica} etc.  For the
edition of 1537 the Paris edition of 1516 was collated with ``a Greek
copy'' (as the preface says) by Christian Herlin, professor of
mathematical studies at Strassburg, who however seems to have done no
more than correct one or two passages by the help of the Basel
\emph{editio princeps} (1533), and add the Greek word in cases where
Zamberti's translation of it seemed unsuitable or inaccurate.

We now come to

\section{Editions of the Greek text}

1533 is the date of the \emph{editio princeps}, the title-page of
which reads as follows:
\begin{center}\openup\jot
{\large ΕΥΚΛΕΙΔΟΥ ΣΤΟΙΧΕΙΩΝ ΒΙΒΛ$\smallblacktriangleright$\?\ ΙΕ̄$\smallblacktriangleright$\?}

ΕΚ ΤΩΝ ΘΕΩΝΟΣ ΣΥΝΟΥΣΙΩΝ.

Εἰς τοῦ αὐτοῦ τὸ πρῶτον, ἐξηγημάτων Πρόκλου Βιβλ. δ̄.

Adiecta praefatiuncula in qua de disciplinis

Mathematicis nonnihil.

BASILEAE APVD I0AN. HERVAGIVM ANNO

M.D.XXXIII. MENSE SEPTEMBRI.
\end{center}

The editor was Simon Grynaeus the elder (d.~1541), who, after working
at Vienna and Ofen, Heidelberg and Tübingen, taught last of all at
Basel, where theology was his main subject. His ``praefatiuncula'' is
addressed to an Englishman, Cuthbert Tonstall (1474–1559), who, having
studied first at Oxford, then at Cambridge, where he became Doctor of
Laws, and afterwards at Padua, where in addition he learnt
mathematics—mostly from the works of Regiomontanus and Paciuolo—wrote
a book on arithmetic\footnote{\emph{De arte supputandi libri
    quatuor.}}as ``a farewell to the sciences,'' and then, entering
politics, became Bishop of London and member of the Privy Council, and
afterwards (1530) Bishop of Durham.  Grynaeus tells us that he used
two \textsc{mss.}\ of the text of the \emph{Elements}, entrusted to
friends of his, one at Venice by ``Lazarus Bayfius'' (Lazare de Baïf,
then the ambassador of the King of France at Venice), the other at
Paris by ``Ioann.\ Rvellius ``(Jean Ruel, a French doctor and a Greek
scholar), while the commentaries of Proclus were put at the disposal
of Grynaeus himself by ``Ioann.\ Claymundus'' at Oxford.  Heiberg has
been able to identify the two \textsc{mss.}\ used for the text; they
are (1)~cod.\ Venetus Marcianus 301 and (2)~cod.\ Paris.\ gr.~2343 of
the 16th~c., containing Books \r1.–\r14., with some scholia which are
embodied in the text. When Grynaeus notes in the margin the readings
from ``the other copy,'' this ``other copy'' is as a rule the Paris
\textsc{ms.}, though sometimes the reading of the Paris
\textsc{ms.}\ is taken into the text and the ``other copy'' of the
margin is the Venice \textsc{ms.}  Besides these two
\textsc{mss.}\ Grynaeus consulted Zamberti, as is shown by a number of
marginal notes referring to ``Zampertus'' or to ``latinum exemplar''
in certain propositions of Books \r9.–\r11.  When it is considered
that the two \textsc{mss.}\ used by Grynaeus are among the worst, it
is obvious how entirely unauthoritative is the text of the
\emph{editio princeps}.  Yet it remained the source and foundation of
later editions of the Greek text for a long period, the editions which
followed being designed, not for the purpose of giving, from other
\textsc{mss.}, a text more nearly representing what Euclid himself
wrote, but of supplying a handy compendium to students at a moderate
price.

1536. Orontius Finaeus (Oronce Fine) published at Paris (\emph{apud
  Simonem Colinaeum}) ``demonstrations on the first six books of
Euclid's elements of geometry,'' ``in which the Greek text of Euclid
himself is inserted in its proper places, with the Latin translation
of Barth.  Zamberti of Venice,'' which seems to imply that only the
enunciations were given in Greek. The preface, from which Kästner
quotes\footnote{Kästner, \r1. p.~260.}, says that the University of
Paris at that time required, from all who aspired to the laurels of
philosophy, a most solemn oath that they had attended lectures on the
said first six Books. Other editions of Fine's work followed in 1544
and 1551.

1545. The \emph{enunciations} of the fifteen Books were published in
Greek, with an Italian translation by Angelo Caiani, at Rome
(\emph{apud Antonium Bladum Asulanum}). The translator claims to have
corrected the books and ``purged them of six hundred things which did
not seem to savour of the almost divine genius and the perspicuity of
Euclid\footnote{Heiberg, vol.~\r5. p.~cvii.}.''

1549. Joachim Camerarius published the enunciations of the first six
Books in Greek and Latin (Leipzig). The book, with preface, purports
to be brought out by Rhaeticus (1514–1576), a pupil of
Copernicus. Another edition with proofs of the propositions of the
first three Books was published by Moritz Steinmetz in 1577 (Leipzig);
a note by the printer attributes the preface to Camerarius himself.

1550. Ioan.\ Scheubel published at Basel (also \emph{per
  Ioan.\ Hervagium}) the first six Books in Greek and Latin ``together
with true and appropriate proofs of the propositions, without the use
of letters'' (i.e.\ letters denoting points in the figures), the
various straight lines and angles being described in
words\footnote{Kästner, \r1. p.~359.}.

1557 (also 1558). Stephanus Gracilis published another edition
(repeated 1573, 1578, 1598) of the enunciations (alone) of Books
\r1.–\r15.\ in Greek and Latin at Paris (\emph{apud Gulielmum
  Cavellat}).  He remarks in the preface that for want of time he had
changed scarcely anything in Books \r1.–\r6., but in the remaining
Books he had emended what seemed obscure or inelegant in the Latin
translation, while he had adopted in its entirety the translation of
Book~\r10.\ by Pierre Mondoré (Petrus Montaureus), published
separately at Paris in 1551.  Gracilis also added a few ``scholia.''

1564. In this year Conrad Dasypodius (Rauchfuss), the inventor and
maker of the clock in Strassburg cathedral, similar to the present
one, which did duty from 1571 to 1789, edited (Strassburg,
Chr.\ Mylius) (1)~Book~\r1.\ of the \emph{Elements} in Greek and Latin
with scholia, (2)~Book\r2.\ in Greek and Latin with Barlaam's
arithmetical version of Book~\r2., and (3)~the \emph{enunciations} of
the remaining Books \r3.–\r13.  Book~\r1.\ was reissued with
``vocabula quaedam geometrica'' of Heron, the enunciations of all the
Books of the \emph{Elements}, and the other works of Euclid, all in
Greek and Latin.  In the preface to~(1) he says that it had been for
twenty-six years the rule of his school that all who were promoted
from the classes to public lectures should learn the first Book, and
that he brought it out, because there were then no longer any copies
to be had, and in order to prevent a good and fruitful regulation of
his school from falling through. In the preface to the edition of 1571
he says that the first Book was generally taught in all gymnasia and
that it was prescribed in his school for the first class.  In the
preface to~(3) he tells us that he published the enunciations of Books
\r3.–\r13.\ in order not to leave his work unfinished, but that, as it
would be irksome to carry about the whole work of Euclid in extenso,
he thought it would be more convenient to students of geometry to
learn the \emph{Elements} if they were compressed into a smaller book.

1620. Henry Briggs (of Briggs' logarithms) published the first six
Books in Greek with a Latin translation after Commandinus, ``corrected
in many places'' (London, G.~Jones).

1703 is the date of the Oxford edition by David Gregory which, until
the issue of Heiberg and Menge's edition, was still the only edition
of the complete works of Euclid\footnote{\greek{ΕΥΚΛΕΙΔΟΥ ΤΑ
    ΣΩΖΟΜΕΝΑ}.  Euclidis quae supersunt omnia.  Ex recensione Davidis
  Gregorii M.D. Astronomia Professoris Saviliani et R.S.S.  Oxoniae, e
  Theatro Sheldoniano, An.\ Dom.\ \r1703.}.  In the Latin translation
attached to the Greek text Gregory says that he followed Commandinus
in the main, but corrected numberless passages in it by means of the
books in the Bodleian Library which belonged to Edward Bernard
(1638–1696), formerly Savilian Professor of Astronomy, who had
conceived the plan of publishing the complete works of the ancient
mathematicians in fourteen volumes, of which the first was to contain
Euclid's \emph{Elements} \r1.–\r15.  As regards the Greek text,
Gregory tells us that he consulted, as far as was necessary, not a few
\textsc{mss.}\ of the better sort, bequeathed by the great Savile to
the University, as well as the corrections made by Savile in his own
hand in the margin of the Basel edition. He had the help of John
Hudson, Bodley's Librarian, who punctuated the Basel text before it
went to the printer, compared the Latin version with the Greek
throughout, especially in the \emph{Elements} and \emph{Data}, and,
where \emph{they differed} or \emph{where he suspected the Greek
  text}, consulted the Greek \textsc{mss.}\ and put their readings in
the margin if they agreed with the Latin and, if they did not agree,
affixed an asterisk in order that Gregory might judge which reading
was geometrically preferable. Hence it is clear that no Greek
\textsc{ms.}, but the Basel edition, was the foundation of Gregory's
text, and that Greek \textsc{mss.}\ were only referred to in the
special passages to which Hudson called attention.

1814–1818. A most important step towards a good Greek text was taken
by F.~Peyrard, who published at Paris, between these years, in three
volumes, the \emph{Elements} and \emph{Data} in Greek, Latin and
French\footnote{\emph{Euclidis quae supersunt.}  \emph{Les \?OEuvres
    d'Euclide, en Grec, en Latin et en Français d'après un manuscrit
    très-ancien, qui était resté inconnu jusqu'à nos jours.}  Par
  F. Peyrarḍ Ouvrage approuvé par l'Institut de France (Paris, chez
  M. Patris).}.  At the time (1808) when Napoleon was having valuable
\textsc{mss.}\ selected from Italian libraries and sent to Paris,
Peyrard managed to get two ancient Vatican \textsc{mss.}\ (190 and
1038) sent to Paris for his use (Vat.~204 was also at Paris at the
time, but all three were restored to their owners in 1814). Peyrard
noticed the excellence of Cod.\ Vat.~190, adopted many of its
readings, and gave in an appendix a conspectus of these readings and
those of Gregory's edition; he also noted here and there readings from
Vat.~1038 and various Paris \textsc{mss.}  He therefore pointed the
way towards a better text, but committed the error of correcting the
Basel text instead of rejecting it altogether and starting afresh.

1824–1825. A most valuable edition of Books \r1.–\r6.\ is that of
J.~G. Camerer (and C.~F, Hauber) in two volumes published at
Berlin\footnote{\emph{Euclidis elementorum libri sex priores graece et
    latine commentario e scriptis veterum ac recentiorum
    mathematicorum et Pfleidereri maxime illustrati} (Berolini,
  sumptibus G. Reimeri).  Tom.~\r1. 1824; tom.~\r2. 1825.}.  The Greek
text is based on Peyrard, although the Basel and Oxford editions were
also used. There is a Latin translation and a collection of notes far
more complete than any other I have seen and well nigh
inexhaustible. There is no editor or commentator of any mark who is
not quoted from; to show the variety of important authorities drawn
upon by Camerer, I need only mention the following names: Proclus,
Pappus, Tartaglia, Commandinus, Clavius, Peletier, Barrow, Borelli,
Wallis, Tacquet, Austin, Simson, Playfair.  No words of praise would
be too warm for this veritable encyclopaedia of information.

1825. J.~G.~C. Neide edited, from Peyrard, the text of Books
\r1.–\r6., \r11.\ and \r12. (\emph{Halis Saxoniae}).

1826-9. The last edition of the Greek text before Heiberg's is that of
E.~F. August, who followed the Vatican \textsc{ms.}\ more closely than
Peyrard did, and consulted at all events the Viennese
\textsc{ms.}\ Gr.~103 (Heiberg's~V). August's edition (Berlin, 1826–9)
contains Books \r1.–\r13.

\section{Latin versions or commentaries after 1533}

1545. Petrus Ramus (Pierre de la Ramée, 1515–1572) is credited with a
translation of Euclid which appeared in 1545 and again in 1549 at
Paris\footnote{Described by Boncompagni, \emph{Bullettino},
  \r2. p.~389.}.  Ramus, who was more rhetorician and logician than
geometer, also published in his \emph{Scholae mathematicae} (1559,
Frankfurt; 1569, Basel) what amounts to a series of lectures on
Euclid's \emph{Elements}, in which he criticises Euclid's arrangement
of his propositions, the definitions, postulates and axioms, all from
the point of view of logic.

1557. Demonstrations to the geometrical Elements of Euclid, six Books,
by Peletarius (Jacques Peletier). The second edition (1610) contained
the same with the addition of the ``Greek text of Euclid''; but only
the \emph{enunciations} of the propositions, as well as the
definitions etc., are given in Greek (with a Latin translation), the
rest is in Latin only.  He has some acute observations, for instance
about the ``angle'' of contact.

1559. Johannes Buteo, or Borrel (1492–1573), published in an appendix
to his book \emph{De quadratura circuli} some notes ``on the errors of
Campanus, Zambertus, Orontius, Peletarius, Pena, interpreters of
Euclid.'' Buteo in these notes proved, by reasoned argument based on
original authorities, that Euclid himself and not Theon was the author
of the proofs of the propositions.

1566. Franciscus Flussates Candalla (François de Foix, Comte de
Candale, 1 502–1594) ``restored'' the fifteen Books, following, as he
says, the terminology of Zamberti's translation from the Greek, but
drawing, for his proofs, on both Campanus and Theon (i.e.\ Zamberti)
except where mistakes in them made emendation necessary. Other
editions followed in 1578, 1602, 1695 (in Dutch).

1572. The most important Latin translation is that of Commandinus
(1509–1575) of Urbino, since it was the foundation of most
translations which followed it up to the time of Peyrard, including
that of Simson and therefore of those editions, numerous in England,
which give Euclid ``chiefly after the text of Simson.''  Simson's
first (Latin) edition (1756) has ``ex versione Latina Federici
Commandini'' on the title-page.  Commandinus not only followed the
original Greek more closely than his predecessors but added to his
translation some ancient scholia as well as good notes of his own. The
title of his work is
\begin{quote}
\emph{Euclidis elementorum libri~\r15, una cum scholiis antiquis.  A
  Federico commandino Urbinate nuper in latinum conversi,
  commentariisque quibusdam illustrati} (Pisauri, apud Camillum
Francischinum).
\end{quote}

He remarks in his preface that Orontius Finaeus had only edited six
Books without reference to any Greek \textsc{ms.}, that Peletarius had
followed Campanus' version from the Arabic rather than the Greek text,
and that Candalla had diverged too far from Euclid, having rejected as
inelegant the proofs given in the Greek text and substituted faulty
proofs of his own. Commandinus appears to have used, in addition to
the Basel \emph{editio princeps}, some Greek \textsc{ms.}, so far not
identified; he also extracted his ``scholia antiqua'' from a
\textsc{ms.}\ of the class of Vat.~192 containing the scholia
distinguished by Heiberg as ``Schol.\ Vat.'' New editions of
Commandinus' translation followed in 1575 (in Italian), 1619, 1749 (in
English, by Keill and Stone), 1756 (Books~\r1.–\r6., \r11., \r12.\ in
Latin and English, by Simson), 1763 (Keill).  Besides these there were
many editions of parts of the whole work, e.g.\ the first six Books.

1574. The first edition of the Latin version by
Clavius\footnote{\emph{Euclidis elementorum librĭ\r15.  Accessit
    \r16. de solidorum regularium comparatione.  Omnes perspicuis
    demonstrationibus, accuratisque scholiis illustrati.  Auctore
    Christophoro Clavio} (Romae, apud Vincentium Accoltum), 2~vols.}
(Christoph Klau [?], born at Bamberg 1537, died 1612) appeared in
1574, and new editions of it in 1589, 1591, 1603, 1607, 1612. It is
not a translation, as Clavius himself states in the preface, but it
contains a vast amount of notes collected from previous commentators
and editors, as well as some good criticisms and elucidations of his
own. Among other things, Clavius finally disposed of the error by
which Euclid had been identified with Euclid of Megara. He speaks of
the differences between Campanus who followed the Arabic tradition and
the ``commentaries of Theon,'' by which he appears to mean the
Euclidean proofs as handed down by Theon; he complains of predecessors
who have either only given the first six Books, or have rejected the
ancient proofs and substituted worse proofs of their own, but makes an
exception as regards Commandinus, ``a geometer not of the common sort,
who has lately restored Euclid, in a Latin translation, to his
original brilliancy.'' Clavius, as already stated, did not give a
translation of the Elements but rewrote the proofs, compressing them
or adding to them, where he thought that he could make them
clearer. Altogether his book is a most useful work.

1621. Henry Savile's lectures (\emph{Praelectiones resdecim in
  principium Elementorum Euclidis Oxoniae habitae} \textsc{mdc.xx}.,
Oxonii 1621), though they do not extend beyond \prop{1}{8}, are
valuable because they grapple with the difficulties connected with the
preliminary matter, the definitions etc., and the tacit assumptions
contained in the first propositions,

1654 André Tacquet's \emph{Elementa geometriae planae et solidae}
containing apparently the eight geometrical Books arranged for general
use in schools. It came out in a large number of editions up to the
end of the eighteenth century.

1655, Barrow's \emph{Euclidis Eiementorum Libri~\r15\ breviter
  demonstrati} is a book of the same kind.  In the preface (to the
edition of 1659) he says that he would not have written it but for the
fact that Tacquet gave only eight Books of Euclid.  He compressed the
work into a very small compass (less than 400 small pages, in the
edition of 1659, for the whole of the fifteen Books and the
\emph{Data}) by abbreviating the proofs and using a large quantity of
symbols (which, he says, are generally Oughtred's). There were several
editions up to 1732 (those of 1660 and 1732 and one or two others are
in English).

1658. Giovanni Alfonso Borelli (1608–1679) published \emph{Euclides
  restitutus}, on apparently similar lines, which went through three
more editions (one in Italian, 1663).

1660. Claude François Milliet Dechales' eight geometrical Books of
Euclid's \emph{Elements} made easy. Dechales' versions of the
\emph{Elements} had great vogue, appearing in French, Italian and
English as well as Latin.  Riccardi enumerates over twenty editions.

1733. Saccheri's \emph{Euclides ab omni naevo vindicatus sive conatus
  geometricus quo stabiliuntur prima ipsa geometriae principia} is
important for his elaborate attempt to grove the parallel-postulate,
forming an important stage in the history of the development of
non-Euclidean geometry.

1756. Simson's first edition, in Latin and in English. The Latin
title is
\begin{quote}
\emph{Euclidis elementorum libri priores sex, item undecimus et
  duodecimus, ex versione latina Federici Commandini; sublatis iis
  quibus olim libri hi a Theone, aliisve, vitiali sunt, et quibusdam
  Euclidis demonstrationibus restitutis. A Roberto Simson M.D\@.}
Glasguae, in aedibus Academicis excudebant Robertus et Andreas Foulis,
Academiae typographi.
\end{quote}

1802. \emph{Euclidis elementorum libri priores \r12\ ex Commandini et
  Gregorii versianibus latinis. In usum juventutis Academicae…}by
Samuel Horsley, Bishop of Rochester. (Oxford, Clarendon Press.)

\section{Italian versions or commentaries}

1543. Tartaglia's version, a second edition of which was published in
1565\footnote{The title-page of the edition of 1565; is as follows:
  \emph{Euclide Megarense philosopho, solo introduttore delle scientie
    mathematice, diligentemente rassettato, et alla intregrità
    ridotto, per il degno professore di tal scientie Nicolo Tartalea
    Brisciano.  secundo le due tradottioni.  con una ampla espositione
    dello istesso tradottore di nuouo aggiunta.  talmente chiara, che
    ogni mediocre ingegno, senza la notitia, ouer suffragio di alcun
    altra scientia con facilità serà capace a poterlo intendere.} In
  Venetia, Appresso Curtio Troiano, 1565.}, and a third in 1585. It
does not appear that he used any Greek text, for in the edition of
1565 he mentions as available only ``the first translation by
Campano,'' ``the second made by Bartolomeo Zamberto Veneto who is
still alive,'' ``the editions of Paris or Germany in which they have
included both the aforesaid translations,'' and ``our own translation
into the vulgar (tongue).''

1575. Commandinus' translation turned into Italian and revised by him.

1613. The first six Books ``reduced to practice'' by Pietro Antonio
Cataldi, re-issued in 1620, and followed by Books \r7.–\r9.\ (1621)
and Book~\r10.\ (1625).

1663. Borelli's Latin translation turned into Italian by Domenico
Magni.

1680. \emph{Euclide restituto} by Vitale Giordano.

1690 Vincenzo Viviani's \emph{Elementi piani e solidi di Euclide}
(Book~\r5.\ in 1674).

1731. \emph{Elementi geometrici piani e solidi di Euclide} by Guido
Grandi. No translation, but an abbreviated version, of which new
editions followed one another up to 1806.

1749. Italian translation of Dechales with Ozanam's corrections and
additions, re-issued 1785, 1797.

1752. Leonardo Ximenes (the first six Books). Fifth edition,
1819.

1818. Vincenzo Flauti's \emph{Corso di geometriea elementare e
  sublime} (4~vols.)\ contains (Vol.~\r1.)\ the first six Books, with
additions and a dissertation on Postulate~5, and
(Vol.~\r2.)\ Books~\r11., \r12.  Flauti also published the first six
Books in 1827 and the \emph{Elements of geometry of Euclid} in 1843
and 1854.

\section{German}

1558. The arithmetical Books \r7.–\r9. by Scheubel\footnote{\emph{Das
    sibend acht und neunt buch des hochberümbten Mathematici Euclidis
    Megarensis…durch Magistrum Johann Scheybl, der löblichen
    universitet zu Tübingen, des Euclidis und Arithmetic Ordinarien,
    auss dem latein ins teutsch gebracht….}}  (cf.\ the edition of the
first six Books, with enunciations in Greek and Latin, mentioned
above, under date 1550).

1562. The version of the first six Books by Wilhelm Holtzmann
(Xylander)\footnote{\emph{Die sechs erste Bücher Euclidis vom anfang
    oder grund der Geometrj…Auss Griechischer sprach in die Teütsch
    gebracht aigentlich erklärt…Demassen vormals in Teütscher sprach
    nie gesehen worden…Durch Wilhelm Holtzman genant Xylander von
    Augspurg.}  Getruckht zu Basel.}.  This work has its interest as
the first edition in German, but otherwise it is not of importance.
Xylander tells us that it was written for practical people such as
artists, goldsmiths, builders etc., and that, as the simple amateur is
of course content to know facts, without knowing how to prove them, he
has often left out the proofs altogether. He has indeed taken the
greatest possible liberties with Euclid, and has not grappled with any
of the theoretical difficulties, such as that of the theory of
parallels.

1651. Heinrich Hoffmann's \emph{Teutscher Euclides} (2nd edition
1653), not a translation.

1694. Ant.\ Ernst Burkh. v.~Pirckenstein's \emph{Teutsch Redender
  Euclides} (eight geometrical Books), ``for generals, engineers
etc.''\ ``proved in a new and quite easy manner.'' Other editions
1699, 1744.

1697. Samuel Reyher's \emph{In teutscher Sprache vorgestellter
  Euclides} (six Books), ``made easy, with symbols algebraical or
derived from the newest art of solution.''

1714. \emph{Euclidis \r15\ Bücher teutsch}, ``treated in a special and
brief manner, yet completely,'' by Chr.\ Schessler (another edition in
1729).

1773. The first six Books translated from the Greek for the use of
schools by J.~F. Lorenz. The first attempt to reproduce Euclid in
German word for word.

1781. Books \r11., \r12.\ by Lorenz (supplementary to the preceding).
Also \emph{Euklid's Elemente fünfzehn Bücher} translated from
the Greek by Lorenz (second edition 1798; editions of 1809, 1818,
1824 by Mollweide, of 1840 by Dippe). The edition of 1824, and
I presume those before it, are shortened by the use of symbols and
the compression of the enunciation and ``setting-out'' into one.

1807. Books \r1.–\r6., \r11., \r12.\ ``newly translated from the
Greek,'' by J.~K.~F. Hauff.

1828. The same Books by Joh.\ Jos.\ Ign.\ Hoffmann ``as guide to
instruction in elementary geometry,'' followed in 1832 by observations
on the text by the same editor.

1833. \emph{Die Geometrie des Euklid und das Wesen derselben} by
E.~S. Unger; also 1838, 1851.

1901. Max Simon, \emph{Euclid und die seeks planimetrischen Bücher}.

\section{French}

1564–1566. Nine Books translated by Pierre Forcadel, a pupil and
friend of P.~de la Ramée.

1604. The first nine Books translated and annotated by Jean Errard de
Bar-le-Duc; second edition, 1605.

1615. Denis Henrion's translation of the 15 Books (seven editions up
to 1676),

1639. The first six Books ``demonstrated by symbols, by a method very
brief and intelligible,'' by Pierre Hengone, mentioned by Barrow as
the only editor who, before him, had used symbols for the exposition
of Euclid.

1672. Eight Books ``rendus plus faciles'' by Claude Franc'ois Milliet
Dechales, who also brought out \emph{Les élémens d'Euclide expliqués
  d'une manière nouvelle et très facile}, which appeared in many
editions, 1672, 1677, 1683 etc.\ (from 1709 onwards revised by
Ozanam), and was translated into Italian (1749 etc.)\ and English (by
William Halifax, 1685).

1804. In this year, and therefore before his edition of the Greek
text, F.~Peyrard published the \emph{Elements} literally translated
into French. A second edition appeared in 1809 with the addition of
the fifth Book, As this second edition contains Books \r1.–\r6. \r11.,
\r12.\ and \prop{10}{1}, it would appear that the first edition
contained Books \r1.–\r4., \r6., \r11., \r12.\ Peyrard used for this
translation the Oxford Greek text and Simson.

\section{Dutch}

1606. Jan Pieterszoon Dou (six Books). There were many later editions.
Kästner, in mentioning one of 1702, says that Dou explains in his
preface that he used Xylander's translation, but, having afterwards
obtained the French translation of the six Books by Errard de
Bar-le-Duc (see above), the proofs in which sometimes pleased him more
than those of the German edition, he made his Dutch version by the
help of both.

1617. Frans van Schooten, ``The Propositions of the Books of Euclid's
Elements''; the fifteen Books in this version ``enlarged'' by Jakob
van Leest in 1662.

1695. C.~J. Vooght, fifteen Books complete, with Candalla's ``16th.''

1702. Kendfik Coets, six Books (also in Latin, 1692); several editions
up to 1752.  Apparently not a translation: but an edition for school
use.

1763. Pybo Steenstra, Books \r1.–\r6., \r11., \r12., likewise an
abbreviated version, several times reissued until 1825.

\section{English}

1570 saw the first and the most important translation, that of Sir
Henry Billingsley. The title-page is as follows:
\begin{quote}
\begin{center}
{\large THE ELEMENTS}

OF GEOMETRIE

of the most auncient Philosopher

EVCLIDE

of Megara
\end{center}

\emph{Faithfully (now first) translated into the English toung, by
\emph{H. Billingsley}, Citizen of London. Whereunto are annexed
certaine Scholies, Annotations, and Inuentions, of the best
Mathematiciens, both of time past, and in this our age.}

\emph{With a very fruitfull Preface by \emph{M.~I. Dee}, specifying the
chiefe Mathematicall Sciẽces, what they are, and whereunto commodious:
where, also, are disclosed certaine new Secrets Mathematicall and
Mechanicall, vntill these our daies, greatly missed.}
\begin{center}
Imprinted at London by John Daye.
\end{center}
\end{quote}

The Preface by the translator, after a sentence observing that without
the diligent study of Euclides Elementes it is impossible to attain
unto the perfect knowledge of Geometry, proceeds thus. ``Wherefore
considering the want and lacke of such good authors hitherto in our
Englishe tounge, lamenting also the negligence, and lacke of zeale to
their countrey in those of our nation, to whom God hath geuen both
knowledge and also abilitie to translate into our tounge, and to
publishe abroad such good authors and bookes (the chiefe instrumentes
of all learninges): seing moreouer that many good wittes both of
gentlemen and of others of all degrees, much desirous and studious of
these artes, and seeking for them as much as they can, sparing no
paines, and yet frustrate of their intent, by no meanes attaining to
that which they seeke: I haue for their sakes, with some charge and
great trauaile, faithfully translated into our vulgare toũge, and set
abroad in Print, this booke of Euclide. Whereunto I haue added easie
and plaine declarations and examples by figures, of the definitions.
In which booke also ye shall in due place finde manifolde additions,
Scholies, Annotations, and Inuentions; which I haue gathered out of
many of the most famous and chiefe Mathematicies, both of old time,
and in our age: as by diligent reading it in course, ye shall well
perceaue….''

It is truly a monumental work, consisting of 464 leaves, and therefore
928 pages, of folio size, excluding the lengthy preface by Dee.  The
notes certainly include all the most important that had ever been
written, from those of the Greek commentators, Proclus and the others
whom he quotes, down to those of Dee himself on the last books.
Besides the fifteen Books, Billingsley included the ``sixteenth''
added by Candalla, The print and appearance of the book are worthy of
its contents; and, in order that it may be understood how no pains
were spared to represent everything in the clearest and most perfect
form, I need only mention that the figures of the propositions in
Book~\r11.\ are nearly all duplicated, one being the figure of Euclid,
the other an arrangement of pieces of paper (triangular, rectangular
etc.)\ pasted at the edges on to the page of the book so that the
pieces can be turned up and made to show the real form of the solid
figures represented.

Billingsley was admitted Lady Margaret Scholar of St John's College,
Cambridge, in 1551, and he is also said to have studied at Oxford, but
he did not take a degree at either University. He was afterwards
apprenticed to a London haberdasher and rapidly became a wealthy
merchant. Sheriff of London in 1584, he was elected Lord Mayor on 31st
December, 1596, on the death, during his year of office, of Sir Thomas
Skinner.  From 1589 he was one of the Queen's four ``customers,'' or
farmers of customs, of the port of London.  In 1591 he founded three
scholarships at St John's College for poor students, and gave to the
College for their maintenance two messuages and tenements in Tower
Street and in Mark Lane, Allhallows, Barking.  He died in 1606.

1651. \emph{Elements of Geometry. The first \r6\ Boocks: In a
  compendious form contracted and demonstrated} by Captain Thomas
Rudd, with the mathematicall preface of John Dee (London).

1660. The first English edition of Barrow's Euclid (published in Latin
in 1655), appeared in London.  It contained ``the whole fifteen books
compendiously demonstrated''; several editions followed, in 1705,
1722, 1732, 1751.

1661. \emph{Euclid's Elements of Geometry, with a supplement of divers
  Propositions and Corollaries.  To which is added a Treatise of
  regular Solids by Campane and Ftussat; likewise Euclid's Data and
  Marinus his Preface.  Also a Treatise of the Divisions of
  Superficies, ascribed to Machomet Bagdedine, but published by
  Commandine at the request of J.~Dee of London.}  Published by care
and industry of John Leeke and Geo.\ Serle, students in the
Math.\ (London). According to Potts this was a second edition of
Billingsley's translation.

1685. William Halifax's version of Dechales' ``Elements of Euclid
explained in a new but most easy method'' (London and Oxford).

1705. \emph{The English Euclide; being the first six Elements of
  Geometry, translated out of the Greek, with annotations and usefull
  supplements by \emph{Edmund Scarburgh}} (Oxford). A noteworthy and
useful edition.

1708. Books \r1.–\r6., \r11., \r12., translated from Commandinus'
Latin version by Dr John Keill, Savilian Professor of Astronomy at
Oxford.

Keill complains in his preface of the omissions by such editors as
Tacquet and Dechales of many necessary propositions
(e.g.\ \prop{6}{27–29}), and of their substitution of proofs of their
own for Euclid's. He praises Barrow's version on the whole, though
objecting to the ``algebraical `` form of proof adopted in Book~\r2.,
and to the excessive use of notes and symbols, which (he considers)
make the proofs \emph{too} short and thereby obscure: his edition was
therefore intended to hit a proper mean between Barrow's excessive
brevity and Clavius' prolixity.

Keill's translation was revised by Samuel Cunn and several times
reissued. 1749 saw the eighth edition, 1772 the eleventh, and 1782 the
twelfth.

1714. W.~Whiston's English version (abridged) of \emph{The Elements of
  Euclid with select theorems out of Archimedes by the learned
  Andr.\ Tacquet.}

1756. Simson's first English edition appeared in the same year as his
Latin version under the title:
\begin{quote}
\emph{The Elements of Euclid, viz.\ the first six Books together with
  the eleventh and twelfth. In this Edition the Errors by which Theon
  or others have long ago vitiated these Books are corrected and some
  of Euclid's Demonstrations are restored.}  By Robert Simson
(Glasgow).
\end{quote}

As above stated, the Latin edition, by its title, purports to be ``ex
versione latina Federici Commandini,'' but to the Latin edition, as
well as to the English editions, are appended
\begin{quote}
\emph{Notes Critical and Geometrical; containing an Account of those
  things in which this Edition differs from the Greek text; and the
  Reasons of the Alterations which have been made. As also
  Observations on some of the Propositions.}
\end{quote}

Simson says in the Preface to some editions (e.g.\ the tenth, of 1799)
that ``the translation is much amended by the friendly assistance of a
learned gentleman.''

Simson's version and his notes are so well known as not to need any
further description. The book went through some thirty successive
editions. The first five appear to have been dated 1756, 1762, 1767,
1772 and 1775 respectively; the tenth 1799, the thirteenth 1806, the
twenty-third 1830, the twenty-fourth 1834, the twenty-sixth 1844.  The
\emph{Data} ``in like manner corrected'' was added for the first time
in the edition of 1762 (the first octavo edition).

1781, 1788. In these years respectively appeared the two volumes
containing the complete translation of the whole thirteen Books by
James Williamson, the last English translation which reproduced Euclid
word for word. The title is
\begin{quote}
\emph{The Elements of Euclid, with Dissertations intended to assist
  and encourage a critical examination of these Elements, as the most
  effectual means of establishing a juster taste upon mathematical
  subjects than that which at present prevails}. By James Williamson.
\end{quote}

In the first volume (Oxford, 1781) he is described as ``M.A. Fellow of
Hertford College,'' and in the second (London, printed by
T.~Spilsbury, 1788) as ``B.D.'' simply. Books \r5., \r6.\ with the
Conclusion in the first volume are paged separately from the rest

1781. \emph{An examination of the first six Books of Euclid's
  Elements}, by William Austin (London).

1795. John Playfair's first edition, containing ``the first six Books
of Euclid with two Books on the Geometry of Solids.'' The book reached
a fifth edition in 1819, an eighth in 1831, a ninth in 1836, and a
tenth in 1846.

1826. Riccardi notes under this date \emph{Euclid's Elements of
  Geometry containing the whole twelve Books translated into English,
  from the edition of Peyrard}, by George Phillips. The editor, who
was President of Queens' College, Cambridge, 1857–1892, was born in
1804 and matriculated at Queens' in 1826, so that he must have
published the book as an undergraduate.

1828. A very valuable edition of the first six Books is that of
Dionysius Lardner, with commentary and geometrical exercises, to which
he added, in place of Books \r11., \r12., a Treatise on Solid Geometry
mostly based on Legendre.  Lardner compresses the propositions by
combining the enunciation and the setting-out, and he gives a vast
number of riders and additional propositions in smaller print.  The
book had reached a ninth edition by 1846, and an eleventh by 1855.
Among other things, Lardner gives an Appendix ``on the theory of
parallel lines,'' in which he gives a short history of the attempts to
get over the difficulty of the parallel' postulate, down to that of
Legendre.

1833. T.~Perronet Thompson's \emph{Geometry without axioms, or the
  first Book of Euclid's Elements with alterations and notes; and an
  intercalary book in which the straight line and plane are derived
  from properties of the sphere, with an appendix containing notices
  of methods proposed for getting over the difficulty in the twelfth
  axiom of Euclid}.

Thompson (1783–1869) was 7th wrangler 1802, midshipman 1803, Fellow of
Queens' College, Cambridge, 1804, and afterwards general and
politician.  The book went through several editions, but, having been
well translated into French by Van Tenac, is said to have received
more recognition in France than at home.

1845. Robert Potts' first edition (and one of the best) entitled:
\begin{quote}
\emph{Euclid's Elements of Geometry chiefly from the text of Dr Simson
  with explanatory notes…to which is prefixed an introduction
  containing a brief outline of the History of Geometry.  Designed for
  the use of the higher forms in Public Schools and students in the
  Universities} (Cambridge University Press, and London, John
W. Parker), to which was added (1847) \emph{An Appendix to the larger
  edition of Euclid's Elements of Geometry, containing additional
  notes on the Elements, a short tract on transversals, and hints for
  the solution of the problems etc.}
\end{quote}

1862. Todhunter's edition.

The later English editions I will not attempt to enumerate; their name
is legion and their object mostly that of adapting Euclid for school
use, with all possible gradations of departure from his text and
order.

\section{Spanish}

1576. The first six Books translated into Spanish by Rodrigo Çamorano.

1637. The first six Books translated, with notes, by L.~Carduchi.

1689. Books \r1.\r6., \r11., \r12., translated and explained by Jacob
Knesa.

\section{Russian}

1739. Ivan Astaroff (translation from Latin).

1789. Pr.~Suvoroff and Yos.\ Nikitin (translation from Greek).

1880. Vachtchenko-Zakhartchenko.

(1817. A translation into Polish by Jo.\ Czecha.)

\section{Swedish}

1744. Mårten Strömer, the first six Books; second edition 1748.  The
third edition (1753) contained Books \r11.–\r12.\ as well; new
editions continued to appear till 1884,

1836. H.~Falk, the first six Books.

1844, 1845, 1859. P.~R. Bråkenhjelm, Books \r1.–\r6., \r11., \r12.

1850. F.~A.~A. Lundgren.

1850. H.~A. Witt and M.~E. Areskong, Books \r1.–\r6., \r11., \r12.

\section{Danish}

1745. Ernest Gottlieb Ziegenbalg.

1803. H.~C. Linderup, Books \r1.–\r6.

\section{Modern Greek}

1820. Benjamin of Lesbos.

I should add a reference to certain editions which have appeared
in recent years.

A Danish translation (\emph{Euklid's Elementer} oversat af Thyra Eibe)
was completed in 1912; Books \r1.–\r2.\ were published (with an
Introduction by Zeuthen) in 1897, Books \r3.–\r4.\ in 1900, Books
\r5.–\r6.\ in 1904, Books \r7.–\r13.\ in 1912.

The Italians, whose great services to elementary geometry are more
than once emphasised in this work, have lately shown a noteworthy
disposition to make the \emph{ipsissima verba} of Euclid once more the
object of study.  Giovanni Vacca has edited the text of
Book~\r1.\ (\emph{Il prima libro degli Elementi}. Testo greco,
versione italiana, introduzione e note, Firenze 1916.)  Federigo
Enriques has begun the publication of a complete Italian translation
(\emph{Gli Elementi d' Enclide e la critica antica e moderna}); Books
\r1.–\r4.\ appeared in 1925 (Alberto Stock, Roma).

An edition of Book~\r1.\ by the present writer was published in 1918
(\emph{Euclid in Greek, Book~\r1., with Introduction and Notes},
Camb.\ Univ.\ Press).

\chapter{}

\section{On the Nature of \emph{Elements}}

It would not be easy to find a more lucid explanation of the terms
\emph{element} and \emph{elementary}, and of the distinction between
them, than is found in Proclus\footnote{Proclus, \emph{Comm.\ on
    Eucl.}\ \r1., ed.\ Friedlein, pp.~72~sqq.}, who is doubtless, here
as so often, quoting from Geminus. There are, says Proclus, in the
whole of geometry certain leading theorems, bearing to those which
follow the relation of a principle, all-pervading, and furnishing
proofs of many properties.  Such theorems are called by the name of
\emph{elements}; and their function may be compared to that of the
letters of the alphabet in relation to language, letters being indeed
called by the same name in Greek (\greek{στοιχεῖα}).

The term \emph{elementary}, on the other hand, has a wider
application: it is applicable to things ``which extend to greater
multiplicity, and, though possessing simplicity and elegance, have no
longer the same dignity as the \emph{elements}, because their
investigation is not of general use in the whole of the science,
e.g.\ the proposition that in triangles the perpendiculars from the
angles to the transverse sides meet in a point.''

``Again, the term \emph{element} is used in two senses, as Menaechmus
says.  For that which is the means of obtaining is an element of that
which is obtained, as the first proposition in Euclid is of the
second, and the fourth of the fifth. In this sense many things may
even be said to be elements of each other, for they are obtained from
one another. Thus from the fact that the exterior angles of
rectilineal figures are (together) equal to four right angles we
deduce the number of right angles equal to the internal angles (taken
together)\footnote{\greek{τὸ πλῆθος τῶν ἐντὸς ὀρθαῖς ἴσων.} If the
  text is right, we must apparently take it as ``the number of the
  angles equal to right angles that there are inside,'' i.e.\ that are
  made up by the internal angles.}, and \emph{vice versa}.  Such an
element is like a \emph{lemma}.  But the term \emph{element} is
otherwise used of that into which, being more simple, the composite is
divided; and in this sense we can no longer say that everything is an
element of everything, but only that things which are more of the
nature of principles are elements of those which stand to them in the
relation of results, as postulates are elements of theorems. It is
according to this signification of the term \emph{element} that the
elements found in Euclid were compiled, being partly those of plane
geometry, and partly those of stereometry. In like manner many writers
have drawn up elementary treatises in arithmetic and astronomy.

``Now it is difficult, in each science, both to select and arrange in
due order the elements from which all the rest proceeds, and into
which all the rest is resolved. And of those who have made the attempt
some were able to put together more and some less; some used shorter
proofs, some extended their investigation to an indefinite length;
some avoided the method of \emph{reductio ad absurdum}, some avoided
\emph{proportion}; some contrived preliminary steps directed against
those who reject the principles; and, in a word, many different
methods have been invented by various writers of elements.

``It is essential that such a treatise should be rid of everything
superfluous (for this is an obstacle to the acquisition of knowledge);
it should select everything that embraces the subject and brings it to
a point (for this is of supreme service to science); it must have
great regard at once to clearness and conciseness (for their opposites
trouble our understanding); it must aim at the embracing of theorems
in general terms (for the piecemeal division of instruction into the
more partial makes knowledge difficult to grasp). In all these ways
Euclid's system of elements will be found to be superior to the rest;
for its utility avails towards the investigation of the primordial
figures\footnote{\greek{τῶν ἀρχικῶν σχημάτων}, by which Proclus
  probably means the regular polyhedra (Tannery, p.~143\emph{n}.).},
its clearness and organic perfection are secured by the progression
from the more simple to the more complex and by the foundation of the
investigation upon common notions, while generality of demonstration
is secured by the progression through the theorems which are primary
and of the nature of principles to the things sought.  As for the
things which seem to be wanting, they are partly to be discovered by
the same methods, like the construction of the scalene and isosceles
(triangle), partly alien to the character of a selection of elements
as introducing hopeless and boundless complexity, like the subject of
\emph{unordered irrationals} which Apoilonius worked out at
length\footnote{We have no more than the most obscure indications of
  the character of this work in an Arabic \textsc{ms.}\ analysed by
  Woepcke, \emph{Essai d'une restitution de travaux perdus
    d'Apollonius sur les quantités irrationelles d'après des
    indications tirées d'un manuscrit arabe} in \emph{Mémoires
    présentés à l'académie des sciences}, \r14 658–720, Paris, 1856.
  Cf.\ Cantor, \emph{Gesch.\ d.\ Math.}\ \r1\tsub{3}, pp.~348–9:
  details are also given in my notes to book~\book{10}.}, and partly
developed from things handed down (in the elements) as causes, like
the many species of angles and of lines.  These things then have been
omitted in Euclid, though they have received full discussion in other
works; but the knowledge of them is derived from the simple
(elements).''

Proclus, speaking apparently on his own behalf, in another place
distinguishes two objects aimed at in Euclid's \emph{Elements}. The
first has reference to the \emph{matter} of the investigation, and
here, like a good Platonist, he takes the whole subject of geometry to
be concerned with the ``cosmic figures,'' the five regular solids,
which in Book~\r13.\ are constructed, inscribed in a sphere and
compared with one another.  The second object is relative to the
learner; and, from this standpoint, the elements may be described as
``a means of perfecting the learner's understanding with reference to
the whole of geometry. For, starting from these (elements), we shall
be able to acquire knowledge of the other parts of this science as
well, while without them it is impossible for us to get a grasp of so
complex a subject, and knowledge of the rest is unattainable. As it
is, the theorems which are most of the nature of principles, most
simple, and most akin to the first hypotheses are here collected, in
their appropriate order; and the proofs of all other propositions use
these theorems as thoroughly well known, and start from them. Thus
Archimedes in the books on the sphere and cylinder, Apollonius, and
all other geometers, clearly use the theorems proved in this very
treatise as constituting admitted principles\footnote{Proclus, pp.~70,
  19–71, 21.}.''

Aristotle too speaks of \emph{elements} of geometry in the same sense.
Thus: ``in geometry it is well to be thoroughly versed in the
elements\footnote{\emph{Topics} \r8.~14, 163~b~23.}''; ``in general
the first of the elements are, given the definitions, e.g.\ of a
straight line and of a circle, most easy to prove, although of course
there are not many data that can be used to establish each of them
because there are not many middle terms\footnote{\emph{Topics} \r8.~3,
  158~b~35.}``; ``among geometrical propositions we call those
`elements' the proofs of which are contained in the proofs of all or
most of such propositions\footnote{\emph{Metaph.}\ 998~a~25.}``; ``(as
in the case of bodies), so in like manner we speak of the elements of
geometrical propositions and, generally, of demonstrations; for the
demonstrations which come first and are contained in a variety of
other demonstrations are called elements of those demonstrations…the
term element is applied by analogy to that which, being one and small,
is useful for many
purposes\footnote{\emph{Metaph.}\ 1014~a~35–b~5.}.''

\section{\emph{Elements} Anterior to Euclid's}

The early part of the famous summary of Proclus was no doubt drawn, at
least indirectly, from the history of geometry by Eudemus; this is
generally inferred from the remark, made just after the mention of
PhiHppus of Medma, a disciple of Plato, that ``those who have written
histories bring the development of this science up to this point.'' We
have therefore the best authority for the list of writers of elements
given in the summary. Hippocrates of Chios (fl.~in second half of
5th~c.)\ is the first; then Leon, who also discovered \emph{diorismi},
put together a more careful collection, the propositions proved in it
being more numerous as well as more serviceable\footnote{Proclus,
  p.~66, 20 \greek{ὥστε τὸν Λέοντα καὶ τὰ στοιχεῖα συνθεῖναι τῷ τε
    πλήθει καὶ τῇ χρείᾳ τῶν δεικνυμένων ἐπιμελέστερον}.}.  Leon was a
little older than Eudoxus (about 408–355~\bc)\ and a little younger
than Plato (428/7–347/6~\bc), but did not belong to the latter's
school. The geometrical text-book of the Academy was written by
Theudius of Magnesia, who, with Amyclas of Heraclea, Menaechmus the
pupil of Eudoxus, Menaechmus' brother Dinostratus and Athenaeus of
Cyzicus consorted together in the Academy and carried on their
investigations in common.  Theudius ``put together the elements
admirably, making many partial (or limited) propositions more
general\footnote{Proclus, p.~67, 14 \greek{καὶ γὰρ τὰ στοιχεῖα καλῶς
    συνέταξεν καὶ πολλὰ τῶν μερικῶν [\greek{ὁρικῶν} (?) Friedlein]
    καθολικω/τερα ἐποίησεν}.}.'' Eudemus mentions no text-book after
that of Theudius, only adding that Hermotimus of Colophon ``discovered
many of the elements\footnote{Proclus, p.~67, 22 \greek{τῶν στοιχείων
    πολλὰ ἀνεῦρε}.}.'' Theudius then must be taken to be the immediate
precursor of Euclid, and no doubt Euclid made full use of Theudius as
well as of the discoveries of Hermotimus and all other available
material.  Naturally it is not in Euclid's \emph{Elements} that we can
find much light upon the state of the subject when he took it up; but
we have another source of information in Aristotle.  Fortunately for
the historian of mathematics, Aristotle was fond of mathematical
illustrations; he refers to a considerable number of geometrical
propositions, definitions etc., in a way which shows that his pupils
must have had at hand some text-book where they could find the things
he mentions; and this text-book must have been that of Theudius.
Heiberg has made a most valuable collection of mathematical extracts
from Aristotle\footnote{\emph{Mathematisches zu Aristoteles} in
  \emph{Abhandlungen zur Gesch.\ d.\ math.\ Wissenschaften},
  \r18. Heft (1904), pp.~1—49.}, from which much is to be gathered as
to the changes which Euclid made in the methods of his predecessors;
and these passages, as well as others not included in Heiberg's
selection, will often be referred to in the sequel.

\section{First Principles: Definitions, Postulates,
and Axioms}

On no part of the subject does Aristotle give more valuable
information than on that of the first principles as, doubtless,
generally accepted at the time when he wrote.  One long passage in the
\emph{Posterior Analytics} is particularly full and lucid, and is
worth quoting \emph{in extenso}.  After laying it down that every
demonstrative science starts from necessary
principles\footnote{\emph{Anal.\ post.}\ \r1.~6, 74~b~5.}, he
proceeds\footnote{\ibid~\r1.~10, 76~a~31–77~a~4.}:

``By first principles in each genus I mean those the truth of which it
is not possible to prove.  What is \emph{denoted} by the first (terms)
and those derived from them is assumed; but, as regards their
\emph{existence}, this must be assumed for the principles but proved
for the rest. Thus what a unit is, what the straight (line) is, or
what a triangle is (must be assumed); and the existence of the unit
and of magnitude must also be assumed, but the rest must be proved.
Now of the premisses used in demonstrative sciences some are peculiar
to each science and others common (to all), the latter being common by
analogy, for of course they are actually useful in so far as they are
applied to the subject-matter included under the particular
science. Instances of first principles peculiar to a science are the
assumptions that a line is of such and such a character, and similarly
for the straight (line); whereas it is a common principle, for
instance, that, if equals be subtracted from equals, the remainders
are equal.  But it is enough that each of the common principles is
true so far as regards the particular genus (subject-matter); for (in
geometry) the effect will be the same even if the common principle be
assumed to be true, not of everything, but only of magnitudes, and, in
arithmetic, of numbers.

``Now the things peculiar to the science, the existence of which must
be assumed, are the things with reference to which the science
investigates the essential attributes, e.g.\ arithmetic with reference
to units, and geometry with reference to points and lines.  With these
things it is assumed that they exist and that they are of such and
such a nature.  But, with regard to their essential properties, what
is assumed is only the meaning of each term employed: thus arithmetic
assumes the answer to the question what is (meant by) `odd' or `even,'
`a square' or `a cube,' and geometry to the question what is (meant
by) `the irrational' or `deflection' or (the so-called) `verging' (to
a point); but that there are such things is proved by means of the
common principles and of what has already been demonstrated.
Similarly with astronomy.  For every demonstrative science has to do
with three things, (1)~the things which are assumed to exist, namely
the genus (subject-matter) in each case, the essential properties of
which the science investigates, (2)~the common axioms so-called, which
are the primary source of demonstration, and (3)~the properties with
regard to which all that is assumed is the meaning of the respective
terms used.  There is, however, no reason why some sciences should not
omit to speak of one or other of these things.  Thus there need not be
any supposition as to the existence of the genus, if it is manifest
that it exists (for it is not equally clear that number exists and
that cold and hot exist); and, with regard to the properties, there
need be no assumption as to the meaning of terms if it is clear: just
as in the common (axioms) there is no assumption as to what is the
meaning of subtracting equals from equals, because it is well known.
But none the less is it true that there are three things naturally
distinct, the subject-matter of the proof, the things proved, and the
(axioms) from which (the proof starts).

``Now that which is \emph{per se} necessarily true, and must
necessarily be thought so, is not a hypothesis nor yet a postulate.
For demonstration has not to do with reasoning from outside but with
the reason dwelling in the soul, just as is the case with the
syllogism.  It is always possible to raise objection to reasoning from
outside, but to contradict the reason within us is not always
possible.  Now anything that the teacher assumes, though it is matter
of proof, without proving it himself, is a hypothesis if the thing
assumed is believed by the learner, and it is moreover a hypothesis,
not absolutely, but relatively to the particular pupil; but, if the
same thing is assumed when the learner either has no opinion on the
subject or is of a contrary opinion, it is a postulate. This is the
difference between a hypothesis and a postulate; for a postulate is
that which is rather contrary than otherwise to the opinion of the
learner, or whatever is assumed and used without being proved,
although matter for demonstration.  Now definitions are not
hypotheses, for they do not assert the existence or non-existence of
anything, while hypotheses are among propositions.  Definitions only
require to be understood: a definition is therefore not a hypothesis,
unless indeed it be asserted that any audible speech is a hypothesis.
A hypothesis is that from the truth of which, if assumed, a conclusion
can be established.  Nor are the geometer's hypotheses false, as some
have said: I mean those who say that `you should not make use of what
is false, and yet the geometer falsely calls the line which he has
drawn a foot long when it is not, or straight when it is not
straight.'  The geometer bases no conclusion on the particular line
which he has drawn being that which he has described, but (he refers
to) what is \emph{illustrated} by the figures.  Further, the postulate
and every hypothesis are either universal or particular statements;
definitions are neither'' (because the subject is of equal extent with
what is predicated of it).

Every demonstrative science, says Aristotle, must start from
indemonstrable principles: otherwise, the steps of demonstration would
be endless.  Of these indemonstrable principles some are
(\emph{a})~common to all sciences, others are (\emph{b})~particular,
or peculiar to the particular science; (\emph{a})~the common
principles are the \emph{axioms}, most commonly illustrated by the
axiom that, if equals be subtracted from equals, the remainders are
equal. Coming now to (\emph{b})~the principles peculiar to the
particular science which must be assumed, we have first the
\emph{genus} or subject-matter, the \emph{existence} of which must be
assumed, viz.\ magnitude in the case of geometry, the unit in the case
of arithmetic.  Under this we must assume \emph{definitions} of
manifestations or attributes of the genus, e.g.\ straight lines,
triangles, deflection etc.  The definition in itself says nothing as
to the existence of the thing defined: it only requires to be
understood.  But in geometry, in addition to the \emph{genus} and the
\emph{definitions}, we have to assume the \emph{existence} of a few
\emph{primary} things which are defined, viz.\ points and lines only:
the existence of everything else, e.g.\ the various figures made up of
these, as triangles, squares, tangents, and their properties,
e.g.\ incommensurability etc., has to be proved (as it is proved by
construction arid demonstration).  In arithmetic we assume the
\emph{existence} of the \emph{unit}: but, as regards the rest, only
the \emph{definitions}, e.g.\ those of odd, even, square, cube, are
assumed, and \emph{existence} has to be \emph{proved}.  We have then
clearly distinguished, among the indemonstrable principles,
\emph{axioms} and \emph{definitions}.  A \emph{postulate} is also
distinguished from a \emph{hypothesis}, the latter being made with the
assent of the learner, the former without such assent or even in
opposition to his opinion (though, strangely enough, immediately after
saying this, Aristotle gives a wider meaning to ``postulate'' which
would cover ``hypothesis'' as well, namely whatever is assumed, though
it is matter for proof, and used without being proved).  Heiberg
remarks that there is no trace in Aristotle of Euclid's Postulates,
and that ``postulate'' in Aristotle has a different meaning.  He seems
to base this on the alternative description of postulate,
indistinguishable from a hypothesis; but, if we take the other
description in which it is distinguished from a hypothesis as being an
assumption of something which is a proper subject of demonstration
without the assent or against the opinion of the learner, it seems to
fit Euclid's Postulates fairly well, not only the first three
(postulating three constructions), but eminently also the other two,
that all right angles are equal, and that two straight lines meeting a
third and making the internal angles on the same side of it less than
two right angles will meet on that side.  Aristotle's description also
seems to me to suit the ``postulates'' with which Archimedes begins
his book \emph{On the equilibrium of planes}, namely that equal
weights balance at equal distances, and that equal weights at unequal
distances do not balance but that the weight at the longer distance
will prevail.

Aristotle's distinction also between \emph{hypothesis} and
\emph{definition}, and between \emph{hypothesis} and \emph{axiom}, is
clear from the following passage: ``Among immediate syllogistic
principles, I call that a \emph{thesis} which it is neither possible
to prove nor essential for any one to hold who is to learn anything;
but that which it is necessary for any one to hold who is to learn
anything whatever is an \emph{axiom}: for there are some principles of
this kind, and that is the most usual name by which we speak of them.
But, of theses, one kind is that which assumes one or other side of a
predication, as, for instance, that something exists or does not
exist, and this is a \emph{hypothesis}; the other, which makes no such
assumption, is a definition.  For a definition is a thesis: thus the
arithmetician posits (\greek{τίθεται}) that a unit is that which is
indivisible in respect of quantity; but this is not a hypothesis,
since what is meant by a unit and the fact that a unit exists are
different things\footnote{\emph{Anal.\ post.}\ \r1. 2, 72~a~14–24.}.''

Aristotle uses as an alternative term for axioms ``common (things),''
\greek{τὰ κοινά}, or ``common opinions'' (\greek{κοιναὶ δόξαι}) as in
the following passages. ``That, when equals are taken from equals, the
remainders are equal is (a) common (principle) in the case of all
quantities, but mathematics takes a separate department
(\greek{ἀπολαβοῦσα}) and directs its investigation to some portion of
its proper subject-matter, as e.g.\ lines or angles, numbers, or any
of the other quantities\footnote{\emph{Metaph.}\ 1061~b~19–24.}.''
The common (principles), e.g.\ that one of two contradictories must be
true, that equals taken from equals etc., and the
like\footnote{\emph{Anal.\ post.}\ \r`.~11, 77~a~30.}….''  With regard
to the principles of demonstration, it is questionable whether they
belong to one science or to several.  By principles of demonstration I
mean the common opinions from which all demonstration proceeds,
e.g.\ that one of two contradictories must be true, and that it is
impossible for the same thing to be and not
be\footnote{\emph{Metaph.}\ 996~b~26–30.}.''  Similarly ``every
demonstrative (science) investigates, with regard to some
subject-matter, the essential attributes, starting from the common
opinions\footnote{\emph{Metaph.}\ 997~a~20–22.}.''  We have then here,
as Heiberg says, a sufficient explanation of Euclid's term for axioms,
viz.\ \emph{common notions} (\greek{κοιναὶ ἔννοιαι}), and there is no
reason to suppose it to be a substitution for the original term due to
the Stoics: cf.\ Proclus' remark that, according to Aristotle and the
geometers, axiom and common notion are the same
thing\footnote{Proclus, p.~194, 8.}.

Aristotle discusses the \emph{indemonstrable} character of the axioms
in the \emph{Metaphysics}.  Since ``all the demonstrative sciences use
the axioms\footnote{\emph{Metaph.}\ 997~a~10.},'' the question arises,
to what science does their discussion
belong\footnote{\ibid~996~b~26,}?  The answer is that, like that of
Being (\greek{οὐσία}), it is the province of the (first)
philosopher\footnote{\ibid~1005~a~21–b~11.}.  It is impossible that
there should be demonstration of everything, as there would be an
infinite series of demonstrations: if the axioms were the subject of a
demonstrative science, there would have to be here too, as in other
demonstrative sciences, a \emph{subject-genus}, its \emph{attributes}
and corresponding \emph{axioms}\footnote{\ibid~997~a~5–8.}; thus there
would be axioms behind axioms, and so on continually.  The axiom is
the most firmly established of all
principles\footnote{\ibid~1005~b~11–17.}.  It is ignorance alone that
could lead anyone to try to prove the
axioms\footnote{\ibid~1006~a~5.}; the supposed proof would be a
\emph{petitio principii}\footnote{\ibid~1006~a~17.}.  If it is
admitted that not everything can be proved, no one can point to any
principle more truly indemonstrable\footnote{\ibid~1006~a~10}.  If any
one thought he could prove them, he could at once be refuted; if he
did not attempt to say anything, it would be ridiculous to argue with
him: he would be no better than a
vegetable\footnote{\ibid~1006~a~11–15.}.  The first condition of the
possibility of any argument whatever is that words should signify
something both to the speaker and to the hearer: without this there
can be no reasoning with any one.  And, if any one admits that words
can mean anything to both hearer and speaker, he admits that something
can be true without demon- stration. And so
on\footnote{\ibid~1006~a~18~sqq.}.

It was necessary to give some sketch of Aristotle's view of the first
principles, if only in connexion with Proclus' account, which is as
follows.  As in the case of other sciences, so ``the compiler of
elements in geometry must give separately the principles of the
science, and after that the conclusions from those principles, not
giving any account of the principles but only of their consequences.
No science proves its own principles, or even discourses about them:
they are treated as self-evident….  Thus the first essential was to
distinguish the principles from their consequences.  Euclid carries
out this plan practically in every book and, as a preliminary to the
whole enquiry, sets out the common principles of this science.  Then
he divides the common principles themselves into \emph{hypotheses},
\emph{postulates}, and \emph{axioms}.  For all these are different
from one another: an axiom, a postulate and a hypothesis are not the
same thing, as the inspired Aristotle somewhere says.  But, whenever
that which is assumed and ranked as a principle is both known to the
learner and convincing in itself, such a thing is an axiom, e.g.\ the
statement that things which are equal to the same thing are also equal
to one another. When, on the other hand, the pupil has not the notion
of what is told him which carries conviction in itself, but
nevertheless lays it down and assents to its being assumed, such an
assumption is a \emph{hypothesis}.  Thus we do not preconceive by
virtue of a common notion, and without being taught, that the circle
is such and such a figure, but, when we are told so, we assent without
demonstration.  When again what is asserted is both unknown and
assumed even without the assent of the learner, then, he says, we call
this a \emph{postulate}, e.g.\ that all right angles are equal.  This
view of a postulate is clearly implied by those who have made a
special and systematic attempt to show, with regard to one of the
postulates, that it cannot be assented to by any one straight off.
According then to the teaching of Aristotle, an axiom, a postulate and
a hypothesis are thus distinguished\footnote{Proclus, pp.~75, 10–77,
  2.}.''

We observe, first, that Proclus in this passage confuses
\emph{hypotheses} and \emph{definitions}, although Aristotle had made
the distinction quite plain.  The confusion may be due to his having
in his mind a passage of Plato from which he evidently got the phrase
about ``not giving an account of'' the principles.  The passage
is\footnote{\emph{Republic}, \r6.~510~c. Cf.\ Aristotle,
  \emph{Nic.\ Eth.}\ 1151~a~17.}: ``I think you know that those who
treat of geometries and calculations (arithmetic) and such things take
for granted (\greek{ὑποθέμενοι}) odd and even, figures, angles of
three kinds, and other things akin to these in each subject, implying
that they know these things, and, though using them as hypotheses, do
not even condescend to give any account of them either to themselves
or to others, but begin from these things and then go through
everything else in order, arriving ultimately, by recognised methods,
at the conclusion which they started in search of.'' But the
hypothesis is here the assumption, e.g.\ `that \emph{there may he such
  a thing} as length without breadth, henceforward called a
line\footnote{H.~Jackson, \emph{Journal of Philology},
  vol.~\r10. p.~144.},' and so on, without any attempt to show that
there is such a thing; it is mentioned in connexion with the
distinction between Plato's `superior' and `inferior' intellectual
method, the former of which uses successive hypotheses as
stepping-stones by which it mounts upwards to the idea of Good.

We pass now to Proclus' account of the difference between
\emph{postulates} and \emph{axioms}.  He begins with the view of
Geminus, according to which ``they differ from one another in the same
way as theorems are also distinguished from problems. For, as in
theorems we propose to see and determine what follows on the
premisses, while in problems we are told to find and do something, in
like manner in the \emph{axioms} such things are assumed as are
manifest of themselves and easily apprehended by our untaught notions,
while in the \emph{postulates} we assume such things as are easy to
find and effect (our understanding suffering no strain in their
assumption), and we require no complication of
machinery\footnote{Proclus, pp.~178, 12—179, 8.  In illustration
  Proclus contrasts the drawing of a straight Line or a circle with
  the drawing of a ``single-turn spiral ``or of an equilateral
  triangle, the spiral requiring more complex machinery and even the
  equilateral triangle needing a certain method. ``For the geometrical
  intelligence will say that by conceiving a straight line fixed at
  one end but, as regards the other end, moving round the fixed end,
  and a point moving along the straight line from the fixed end, I
  have described the single-turn spiral; for the end of the straight
  line describing a circle, and the point moving on the straight line
  simultaneously, when they arrive and meet at the same point,
  complete such a spiral.  And again, if I draw equal circles, join
  their common point to the centres of the circles and draw a straight
  line from one of the centres to the other, I shall have the
  equilateral triangle.  These things then are far from being
  completed by means of a single act or of a moment's thought''
  (p.~180. 8–21).}.''…'' Both must have the characteristic of being
simple and readily grasped, I mean both the postulate and the axiom;
but the postulate bids us contrive and find some subject-matter
(\greek{ὕλη}) to exhibit a property simple and easily grasped, while
the axiom bids us assert some essential attribute which is
self-evident to the learner, just as is the fact that fire is hot, or
any of the most obvious things\footnote{Proclus, p.~181, 4–11.}.''

Again, says Proclus, ``some claim that all these things are alike
postulates, in the same way as some maintain that all things that are
sought are problems.  For Archimedes begins his first book on
\emph{Inequilibrium}\footnote{It is necessary to coin a word to render
  \greek{ἀνισορροπιῶν}, which is moreover in the plural.  The title of
  the treatise as we have it is \emph{Equilibria of planes or centres
    of gravity of planes} in Book~\r1\ and \emph{Equilibria of planes}
  in Book~\r2.} with the remark `I postulate that equal weights at
equal distances are in equilibrium,' though one would rather call this
an axiom.  Others call them all axioms in the same way as some regard
as theorems everything that requires demonstration\footnote{Proclus,
  p.~181, 16–23.}.''

``Others again will say that postulates are peculiar to geometrical
subject-matter, while axioms are common to all investigation which is
concerned with quantity and magnitude.  Thus it is the geometer who
knows that all right angles are equal and how to produce in a straight
line any limited straight line, whereas it is a common notion that
things which are equal to the same thing are also equal to one
another, and it is employed by the arithmetician and any scientific
person who adapts the general statement to his own
subject\footnote{\ibid~p.~182, 6–14.}.''

The third view of the distinction between a postulate and an axiom is
that of Aristotle above described\footnote{Pp.~118, 119.}.

The difficulties in the way of reconciling Euclid's classification of
postulates and axioms with any one of the three alternative views are
next dwelt upon.  If we accept the first view according to which an
axiom has reference to something known, and a postulate to something
done, then the 4th postulate (that all right angles are equal) not a
postulate; neither is the 5th which states that, if a straight line
falling on two straight lines makes the interior angles on the same
side less than two right angles, the straight lines, if produced
indefinitely, will meet on that side on which are the angles less than
two right angles.  On the second view, the assumption that two
straight lines cannot enclose a space, ``which even now,'' says
Proclus, ``some add as an axiom,'' and which is peculiar to the
subject-matter of geometry, like the fact that all right angles are
equal, is not an axiom.  According to the third (Aristotelian) view,
``everything which is confirmed (\greek{πιστοῦται}) by a sort of
demonstration will be a postulate, and what is incapable of proof will
be an axiom\footnote{Proclus, pp.~i82, 21–183, 13.}.''  This last
statement of Proclus is loose, as regards the axiom, because it omits
Aristotle's requirement that the axiom should be a self-evident truth,
and one that must be admitted by any one who is to learn anything at
all, and, as regards the postulate, because Aristotle calls a
postulate something assumed without proof though it is ``matter of
demonstration'' (\greek{ἀποδεικτὸν ὄν}), but says nothing of a
\emph{quasi}-demonstration of the postulates.  On the whole I think it
is from Aristotle that we get the best idea of what Euclid understood
by a postulate and an axiom or common notion.  Thus Aristotle's
account of an axiom as a principle common to all sciences, which is
self-evident, though incapable of proof, agrees sufficiently with the
contents of Euclid's \emph{common notions} as reduced to five in the
most recent text (not omitting the fourth, that ``things which
coincide are equal to one another'').  As regards the
\emph{postulates}, it must be borne in mind that Aristotle says
elsewhere\footnote{\emph{Anal.\ post.}\ \r1.~25, 86~a~33–35.} that,
``other things being equal, that proof is the better which proceeds
from the fewer postulates or hypotheses or propositions.'' If then we
say that a geometer must lay down as principles, first certain axioms
or common notions, and then an \emph{irreducible minimum} of
postulates in the Aristotelian sense concerned only with the
subject-matter of geometry, we are not far from describing what Euclid
in fact does.  As regards the postulates we may imagine him saying:
Besides the common notions there are a few other things which I must
assume without proof, but which differ from the common notions in that
they are not self-evident.  The learner may or may not be disposed to
agree to them; but he must accept them at the outset on the superior
authority of his teacher, and must be left to convince himself of
their truth in the course of the investigation which follows.  In the
first place certain simple constructions, the drawing and producing of
a straight line, and the drawing of a circle, must be assumed to be
possible, and with the constructions the existence of such things as
straight lines and circles; and besides this we must lay down some
postulate to form the basis of the theory of parallels.''  It is true
that the admission of the 4th postulate that all right angles are
equal still presents a difficulty to which we shall have to recur.

There is of course no foundation for the idea, which has found its way
into many text-books, that ``the object of the postulates is to
declare that the only instruments the use of which is permitted in
geometry are the \emph{rule} and
\emph{compass}\footnote{Cf.\ Lardner's Euclid: also Todhunter.}.''

\section{Theorems and Problems}

``Again the deductions from the first principles,'' says Proclus,
``are divided into \textbf{problems} and \textbf{theorems}, the former
embracing the generation, division, subtraction or addition of
figures, and generally the changes which are brought about in them,
the latter exhibiting the essential attributes of
each\footnote{Proclus, p.~77, 7–12.}.''

``Now, of the ancients, some, like Speusippus and Amphinomus, thought
proper to call them all theorems, regarding the name of theorems as
more appropriate than that of problems to theoretic sciences,
especially as these deal with eternal objects.  For there is no
becoming in things eternal, so that neither could the problem have any
place with them, since it promises the generation and making of what
has not before existed, e.g.\ the construction of an equilateral
triangle, or the describing of a square on a given straight line, or
the placing of a straight line at a given point.  Hence they say it is
better to assert that all (propositions) are of the same kind, and
that we regard the generation that takes place in them as referring
not to actual \emph{making} but to \emph{knowledge}, when we treat
things existing eternally as if they were subject to becoming: in
other words, we may say that everything is treated by way of theorem
and not by way of problem\footnote{\ibid~pp.~77, 15–78, 8.}
(\greek{πάντα θεωρηματικῶς ἀλλ’ οὐ προβληματικῶς λαμβάνεσθαι}).

``Others on the contrary, like the mathematicians of the school of
Menaechmus, thought it right to call them all problems, describing
their purpose as twofold, namely in some cases to furnish
(\greek{πορίσασθαι}) the thing sought, in others to take a determinate
object and see either what it is, or of what nature, or what is its
property, or in what relations it stands to something else.

``In reality both assertions are correct.  Speusippus is right because
the problems of geometry are not like those of mechanics, the latter
being matters of sense and exhibiting becoming and change of every
sort.  The school of Menaechmus are right also because the discoveries
even of theorems do not arise without an issuing-forth into matter, by
which I mean intelligible matter.  Thus forms going out into matter
and giving it shape may fairly be said to be like processes of
becoming. For we say that the motion of our thought and the
throwing-out of the forms in it is what produces the figures in the
imagination and the conditions subsisting in them.  It is in the
imagination that constructions, divisions, placings, applications,
additions and subtractions (take place), but everything in the mind is
fixed and immune from becoming and from every sort of
change\footnote{\ibid~pp.~78, 8–79, 2.}.''

``Now those who distinguish the theorem from the problem say that
every problem implies the possibility, not only of that which is
predicated of its subject-matter, but also of its opposite, whereas
every theorem implies the possibility of the thing predicated but not
of its opposite as well.  By the subject-matter I mean the genus which
is the subject of inquiry, for example, a triangle or a square or a
circle, and by the property predicated the essential attribute, as
equality, section, position, and the like.  When then any one
enunciates thus, \emph{To inscribe an equilateral triangle in a
  circle}, he states a \emph{problem}; for it is also possible to
inscribe in it a triangle which is not equilateral.  Again, if we take
the enunciation \emph{On a given limited straight line to construct an
  equilateral triangle}, this is a \emph{problem}; for it is possible
also to construct one which is not equilateral.  But, when any one
enunciates that \emph{In isosceles triangles the angles at the base
  are equal}, we must say that he enunciates a \emph{theorem}; for it
is not also possible that the angles at the base of isosceles
triangles should be unequal.  It follows that, if any one were to use
the form of a problem and say \emph{In a semicircle to describe a
  right angle}, he would be set down as no geometer. For every angle
in a semicircle is right\footnote{Proclus, pp.~79, 11–80, 5.}.''

''Zenodotus, who belonged to the succession of Oenopides, but was a
disciple of Andron, distinguished the theorem from the problem by the
fact that the theorem inquires what is the property predicated of the
subject-matter in it, but the problem what is the cause of what effect
(\greek{τίνος ὄντος τί ἐστιν}).  Hence too Posidonius defined the one
(the problem) as a proposition in which it is inquired whether a thing
exists or not (\greek{εἰ ἔστιν ἦ μή}), the other (the
theorem\footnote{In the text we have \greek{τὸ δὲ πρόβλημα} answering
  to \greek{τὸ μὲν} without substantive: \greek{πρόβλημα} was
  obviously inserted in error,}) as a proposition in which it is
inquired what (a thing) is or of what nature (\greek{τί ἐστιν ἦ ποῖόν
  τι}); and he said that the theoretic proposition must be put in a
declaratory form, e.g., \emph{Any triangle has two sides (together)
  greater than the remaining side} and \emph{In any isosceles triangle
  the angles at the base are equal}, but that we should state the
problematic proposition as if inquiring whether it is possible to
construct an equilateral triangle upon such and such a straight line.
For there is a difference between inquiring absolutely and
indeterminately (\greek{ἁπλῶς τε καὶ ἀορίστως}) whether there exists a
straight line from such and such a point at right angles to such and
such a straight line and investigating which is the straight line at
right angles\footnote{Proclus, pp.~80, 15–81, 4.}.''

``That there is a certain difference between the problem and the
theorem is clear from what has been said; and that the Elements of
Euclid contain partly problems and partly theorems will be made
manifest by the individual propositions, where Euclid himself adds at
the end of what is proved in them, in some cases, 'that which it was
required to do,' and in others, `that which it was required to prove,'
the latter expression being regarded as characteristic of theorems, in
spite of the fact that, as we have said, demonstration is found in
problems also. In problems, however, even the demonstration is for the
purpose of (confirming) the construction: for we bring in the
demonstration in order to show that what was enjoined has been done;
whereas in theorems the demonstration is worthy of study for its own
sake as being capable of putting before us the nature of the thing
sought.  And you will find that Euclid sometimes interweaves theorems
with problems and employs them in turn, as in the first book, while at
other times he makes one or other preponderate.  For the fourth book
consists wholly of problems, and the fifth of
theorems\footnote{Proclus, p.~81, 5–22.}.''

Again, in his note on Eucl.\ \prop{1}{4}, Proclus says that Carpus,
the writer on mechanics, raised the question of theorems and problems
in his treatise on astronomy.  Carpus, we are told, ``says that the
class of problems is in order prior to theorems.  For the subjects,
the properties of which are sought, are discovered by means of
problems.  Moreover in a problem the enunciation is simple and
requires no skilled intelligence; it orders you plainly to do such and
such a thing, \emph{to construct an equilateral triangle}, or,
\emph{given two straight lines, to cut off from the greater (a
  straight line) equal to the lesser}, and what is there obscure or
elaborate in these things?  But the enunciation of a theorem is a
matter of labour and requires much exactness and scientific judgment
in order that it may not turn out to exceed or fail short of the
truth; an example is found even in this proposition (\prop{1}{4}), the
first of the theorems.  Again, in the case of problems, one general
way has been discovered, that of \emph{analysis}, by following which
we can always hope to succeed; it is this method by which the more
obscure problems are investigated.  But, in the case of theorems, the
method of setting about them is hard to get hold of since `up to our
time,' says Carpus, `no one has been able to hand down a general
method for their discovery.  Hence, by reason of their easiness, the
class of problems would naturally be more simple.'  After these
distinctions, he proceeds: `Hence it is that in the Elements too
problems precede theorems, and the Elements begin from them; the first
theorem is fourth in order, not because the fifth\footnote{\greek{τὸ
    πέμπτον} This should apparently be the fourth because in the next
  words it is implied that none of the first three propositions ate
  required in proving it.} is proved from the problems, but because,
even if it needs tor its demonstration none of the propositions which
precede it, it was necessary that they should be first because they
are problems, while it is a theorem.  In fact, in this theorem he uses
the common notions exclusively, and in some sort takes the same
triangle placed in different positions; the coincidence and the
equality proved thereby depend entirely upon sensible and distinct
apprehension.  Nevertheless, though the demonstration of the first
theorem is of this character, the problems properly preceded it,
because in general problems are allotted the order of
precedence\footnote{Proclus, pp.~241, 19–243, 11.}.''

Proclus himself explains the position of Prop.~4 after Props.~1–3 as
due to the fact that a theorem about the essential properties of
triangles ought not to be introduced before we know that such a thing
as a triangle can be constructed, nor a theorem about the equality of
sides or straight lines until we have shown, by constructing them,
that there can be two straight lines which are equal to one
another\footnote{\ibid~pp.~233, 21–234, 6.}.  It is plausible enough
to argue in this way that Props.\ 2 and~3 at all events should precede
Prop.~4.  And Prop.~1 is used in Prop.~2, and must therefore precede
it.  But Prop.~1 showing how to construct an \emph{equilateral}
triangle on a given base is not important, in relation to Prop.~4, as
dealing with the ``production of triangles'' in general: for it is of
no use to say, as Proclus does, that the construction of the
equilateral triangle is ``common to the three species (of
triangles)\footnote{Proclus, p.~234, 21},'' as we are not in a
position to know this at such an early stage.  The existence of
triangles in general was doubtless assumed as following from the
existence of straight lines and points in one plane and from the
possibility of drawing a straight line from one point to another.

Proclus does not however seem to reject definitely the view of Carpus,
for he goes on\footnote{\ibid~p.~243, 12–25.}: ``And perhaps problems
are in order before theorems, and especially for those who need to
ascend from the arts which are concerned with things of sense to
theoretical investigation.  But in dignity theorems are prior to
problems…. It is then foolish to blame Geminus for saying that the
theorem is more perfect than the problem. For Carpus himself gave the
priority to problems in respect of \emph{order}, and Geminus to
theorems in point of more perfect \emph{dignity},'' so that there was
no real inconsistency between the two.

Problems were classified according to the number of their possible
solutions.  Amphinomus said that those which had a unique solution
(\greek{μοναχῶς}) were called ``ordered'' (the word has dropped out in
Proclus, but it must be \greek{τεταγμένα}, in contrast to the third
kind, \greek{ἄτακτα}); those which had a definite number of solutions
``intermediate'' (\greek{μέσα}); and those with an infinite variety of
solutions ``unordered'' (\greek{ἄτακτα})\footnote{\ibid~p.~220,
  7—12.}'.  Proclus gives as an example of the last the problem
\emph{To divide a given straight line into three parts in continued
  proportion}\footnote{\ibid~pp.~220, 16–221, 6.}.  This is the same
thing as solving the equations $x + y +z = a$, $xz = y^2$.  Proclus'
remarks upon the problem show that it was solved, like all quadratic
equations, by the method of ``application of areas.''  The straight
line $a$ was first divided into any two parts, $(x + z)$ and $y$,
subject to the sole limitation that $(x + z)$ must not be less than
$2y$, which limitation is the \greek{διορισμός}, or condition of
possibility.  Then an area was applied to $(x + z)$, or $(a — y)$,
``falling short by a square figure'' (\greek{ἐλλεῖπον εἴδει
  τετραγώνῳ}) and equal to the square on~$y$.  This determines $x$
and~$z$ separately in terms of~$a$ and~$y$.  For, if $z$ be the side
of the square by which the area (i.e.\ rectangle) ``falls short,'' we
have $\{ (a —y) — z \} z = y^2$, whence $2z = (a — y) \pm \sqrt{\{ (a
  - y)^2 - 4y^2 \}}$.  And $y$ may be chosen arbitrarily, provided
that it is not greater than $a/3$.  Hence there are an infinite number
of solutions.  If $y = a/3$, then, as Proclus remarks, the three parts
are equal.

Other distinctions between different kinds of problems are added by
Proclus.  The word ``problem,'' he says, is used in several senses.
In its widest sense it may mean anything ``propounded
``(\greek{προτεινόμενον}), whether for the purpose of instruction
(\greek{μαθήσεως}) or construction (\greek{ποιήσεως}).  (In this
sense, therefore, it would include a theorem.)  But its special sense
in mathematics is that of something ``propounded with a view to a
theoretic construction\footnote{Proclus, p.~221, 7–11.}.''

Again you may apply the term (in this restricted sense) even to
something which is \emph{impossible}, although it is more
appropriately used of what is \emph{possible} and neither asks too
much nor contains too little in the shape of data.  According as a
problem has one or other of these defects respectively, it is called
(1)~a problem \emph{in excess} (\greek{πλεονάζον}) or (2)~a
\emph{deficient} problem (\greek{ἐλλιπὲς πρόβλημα}).  The problem
\emph{in excess} (1)~is of two kinds, (\emph{a}) a problem in which
the properties of the figure to be found are either
\emph{inconsistent} (\greek{ἀσύμβατα}) or non-existent
(\greek{ἀνύπαρκτα}), in which case the problem is called impossible,
or (\emph{b})~a problem in which the enunciation is merely redundant:
an example of this would be a problem requiring us to construct an
equilateral triangle with its vertical angle equal to two-thirds of a
right angle; such a problem is possible and is called ``more than a
problem'' (\greek{μεῖζον ἦ πρόβλημα}).  The \emph{deficient}
problem~(2) is similarly called ``less than a problem''
(\greek{ἔλασσον ἢ πρόβλημα}), its characteristic being that something
has to be added to the enunciation in order to convert it from
indeterminateness (\greek{ἀοριστία}) to order (\greek{τάξις}) and
scientific determinateness (\greek{ὅρος ἐπιστημονικός}): such would be
a problem bidding you ``to construct an isosceles triangle,'' for the
varieties of isosceles triangles are unlimited.  Such ``problems'' are
not problems in the proper sense (\greek{κυρίως λεγόμενα προβλήματα}),
but only equivocally\footnote{\ibid~pp.~221, 13—222, 14.}.

\section{The Formal Divisions of a Proposition}

``Every problem,'' says Proclus\footnote{\ibid~pp.~203, 1–204, 13;
  204, 23–205, 8.}, ``and every theorem which is complete with all its
parts perfect purports to contain in itself all of the following
elements: \textbf{enunciation} (\greek{πρότασις}),
\textbf{setting-out} (\greek{ἔκθεσις}), \textbf{definition} or
\textbf{specification} (\greek{διορισμός}), \textbf{construction} or
\textbf{machinery} (\greek{κατασκευή}), \textbf{proof}
(\greek{ἀπόδειξις}), \textbf{conclusion} (\greek{συμπέρασμα}).  Now of
these the \emph{enunciation} states what is given and what is that
which is sought, the perfect \emph{enunciation} consisting of both
these parts.  The \emph{setting-out} marks off what is given, by
itself, and adapts it beforehand for use in the investigation.  The
\emph{definition} or \emph{specification} states separately and makes
clear what the particular thing is which is sought.  The
\emph{construction} or \emph{machinery} adds what is wanting to the
datum for the purpose of finding what is sought.  The \emph{proof}
draws the required inference by reasoning scientifically from
acknowledged facts.  The \emph{conclusion} reverts again to the
\emph{enunciation}, confirming what has been demonstrated.  These are
all the parts of problems and theorems, but the most essential and
those which are found in all are \emph{enunciation}, \emph{proof},
\emph{conclusion}.  For it is equally necessary to know beforehand
what is sought, to prove this by means of the intermediate steps, and
to state the proved fact as a conclusion; it is impossible to dispense
with any of these three things.  The remaining parts are often brought
in, but are often left out as serving no purpose.  Thus there is
neither \emph{setting-out} nor \emph{definition} in the problem of
constructing an isosceles triangle having each of the angles at the
base double of the remaining angle, and in most theorems there is no
\emph{construction} because the \emph{setting-out} suffices without
any addition for proving the required property from the data.  When
then do we say that the \emph{setting-out} is wanting?  The answer is,
when there is nothing \emph{given} in the \emph{enunciation}; for,
though the enunciation is in general divided into what is given and
what is sought, this is not always the case, but sometimes it states
only what is sought, i.e.\ what must be known or found, as in the case
of the problem just mentioned.  That problem does not, in fact, state
beforehand with what datum we are to construct the isosceles triangle
having each of the equal angles double of the remaining angle, but
(simply) that we are to find such a triangle….  When, then, the
enunciation contains both (what is given and what is sought), in that
case we find both \emph{definition} and \emph{setting-out}, but,
whenever the datum is wanting, they too are wanting.  For not only is
the \emph{setting-out} concerned with the datum, but so is the
\emph{definition} also, as, in the absence of the datum, the
\emph{definition} will be identical with the enunciation.  In fact,
what could you say in defining the object of the aforesaid problem
except that it is required to find an isosceles triangle of the kind
referred to?  But that is what the \emph{enunciation} stated.  If then
the \emph{enunciation} does not include, on the one hand, what is
given and, on the other, what is sought, there is no
\emph{setting-out} in virtue of there being no datum, and the
\emph{definition} is left out in order to avoid a mere repetition of
the \emph{enunciation}.''

The constituent parts of an Euclidean proposition will be readily
identified by means of the above description.  As regards the
\emph{definition} or \emph{specification} (\greek{διορισμός}) it is to
be observed that we have here only one of its uses.  Here it means a
closer definition or description of the object aimed at, by means of
the concrete lines or figures set out in the \greek{ἔκθεσις}; instead
of the general terms used in the enunciation; and its purpose is to
rivet the attention better, as Proclus indicates in a later passage
(\greek{πρόπον τινὰ προσεχείας ἐστὶν αἴτιος ὁ
  διορισμός})\footnote{Proclus, p.~2o8, 21.}.

The other technical use of the word to signify the limitations to
which the possible solutions of a problem are subject is also
described by Proclus, who speaks of \greek{διορισμοί} determining
``whether what is sought is impossible or possible, and how far it is
practicable and in how many ways\footnote{\ibid~p.~202, 3.}''; and the
\greek{διορισμός} in this sense appears in Euclid as well as in
Archimedes and Apollonius. Thus we have in Eucl.\ \prop{1}{22} the
enunciation ``From three straight lines which are equal to three given
straight lines to construct a triangle,'' followed immediately by the
limiting condition (\greek{διορισμός}). ``Thus two of the straight
lines taken together in any manner must be greater than the remaining
one.''  Similarly in \prop{6}{28} the \emph{enunciation} ``To a given
straight line to apply a parallelogram equal to a given rectilineal
figure and falling short by a parallelogrammic figure similar to a
given one'' is at once followed by the necessary condition of
possibility: ``Thus the given rectilineal figure must not be greater
than that described on half the line and similar to the defect.''

Tannery supposed that, in giving the other description of the
\greek{διορισμός} as quoted above, Proclus, or rather his guide, was
using the term incorrectly. The \greek{διορισμός} in the better known
sense of the determination of limits or conditions of possibility was,
we are told, invented by Leon.  Pappus uses the word in this sense
only.  The other use of the term might, Tannery thought, be due to a
confusion occasioned by the use of the same words (\greek{δεῖ δή}) in
introducing the parts of a proposition corresponding to the two
meanings of the word \greek{διορισμός}\footnote{\emph{La Géométrie
    grecque}, p.~149 note. Where \greek{δεῖ δὴ} introduces the closer
  description of the pioblem we may translate, ``it is then requited''
  or ``thus it is required'' (to construct etc.); when it introduces
  the condition of possibility we may translate ``thus it is necessary
  etc.''  Heiberg originally wrote \greek{δεῖ δὲ} in the latter sense
  in \prop{1}{21} on the authority of Proclus and Eutocius, and
  against that of the \textsc{mss.}  Later, on the occasion of
  \prop{11}{13}, he observed that he should have followed the
  \textsc{mss.}\ and written \greek{δεῖ δὴ} which he found to be,
  after all, the right reading in Eutocius (Apollonius, ed.\ Heiberg,
  11. p.~178). \greek{δεῖ δὴ} is also the expression used by
  Diophantus for introducing conditions of possibility.}.  On the
other hand it is to be observed that Eutocius distinguishes clearly
between the two uses and implies that the difference was well
known\footnote{See the passage of Eutocius referred to in last note.}.
The \greek{διορισμός} in the sense of condition of possibility follows
immediately on the enunciation, is even part of it; the
\greek{διορισμός} in the other sense of course comes immediately after
the \emph{setting-out}.

Proclus has a useful observation respecting the \emph{conclusion} of a
proposition\footnote{Proclus, p.~207, 4–25.}.  ``The conclusion they
are accustomed to make double in a certain way: I mean, by proving it
in the given case and then drawing a general inference, passing, that
is, from the partial conclusion to the general.  For, inasmuch as they
do not make use of the individuality of the subjects taken, but only
draw an angle or a straight line with a view to placing the datum
before our eyes, they consider that this same fact which is
established in the case of the particular figure constitutes a
conclusion true of every other figure of the same kind.  They pass
accordingly to the general in order that we may not conceive the
conclusion to be partial.  And they are justified in so passing, since
they use for the demonstration the particular things set out, not
\emph{quâ} particulars, but \emph{quâ} typical of the rest.  For it is
not in virtue of such and such a size attaching to the angle which is
set out that I effect the bisection of it, but in virtue of its being
rectilineal and nothing more.  Such and such size is peculiar to the
angle set out, but its quality of being rectilineal is common to all
rectilineal angles. Suppose, for example, that the given angle is a
right angle. If then I had employed in the proof the fact of its being
right, I should not have been able to pass to every species of
rectilineal angle; but, if I make no use of its being right, and only
consider it as rectilineal, the argument will equally apply to
rectilineal angles in general.''

\section{Other Technical Terms}

\subsection{Things said to be given}

Proclus attaches to his description of the formal divisions of a
proposition an explanation of the different senses in which the word
\emph{given} or \emph{datum} (\greek{δεδομένον}) is used in
geometry. ``Everything that is given is given in one or other of the
following ways, \emph{in position}, \emph{in ratio}, \emph{in
  magnitude}, or \emph{in species}.  The point is given in position
only, but a line and the rest may be given in all the
senses\footnote{Proclus, p.~205, 13–15.}.''

The illustrations which Proclus gives of the four senses in which a
thing may be \emph{given} are not altogether happy, and, as regards
things which are given \emph{in position}, \emph{in magnitude}, and
\emph{in species}, it is best, I think, to follow the definitions
given by Euclid himself in his book of \emph{Data}.  Euclid does not
mention the fourth class, things given \emph{in ratio}, nor apparently
do any of the great geometers.

(1)~\emph{Given in position} really needs no definition; and, when
Euclid says (\emph{Data}, Def.~4) that ``Points, lines and angles are
said to be \emph{given in position} which always occupy the same
place,'' we are not really the wiser.

(2)~\emph{Given in magnitude} is defined thus (\emph{Data}, Def.~1):
``Areas, lines and angles are called \emph{given in magnitude} to
which we can find equals.''  Proclus' illustration is in this case the
following: when, he says, two unequal straight lines are given from
the greater of which we have to cut off a straight line equal to the
lesser, the straight lines are obviously \emph{given in magnitude},
``for greater and less, and finite and infinite are predications
peculiar to magnitude.'' But he does not explain that part of the
implication of the term is that a thing is given in magnitude
\emph{only}, and that, for example, its position is not given and is a
matter of indifference

(3)~\emph{Given in species}.  Euclid's definition (\emph{Data},
Def.~3) is: ``Rectilineal figures are said to be \emph{given in
  species} in which the angles are severally given and the ratios of
the sides to one another are given.'' And this is the recognised use
of the term (cf.\ Pappus, \emph{passim}).  Proclus uses the term in a
much wider sense for which I am not aware of any authority. Thus, he
says, when we speak of (bisecting) a given rectilineal angle, the
angle is given in species by the word \emph{rectilineal}, which
prevents our attempting, by the same method, to bisect a curvilineal
angle!  On Eucl.\ \prop{1}{9}, to which he here refers, he says that
an angle is given in species when e.g.\ we say that it is right or
acute or obtuse or rectilineal or ``mixed,'' but that the actual angle
in the proposition is given in species only.  As a matter of fact, we
should say that the actual angle in the figure of the proposition is
given \emph{in magnitude} and not \emph{in species}, part of the
implication of \emph{given in species} being that the actual magnitude
of the thing \emph{given in species} is indifferent; an angle cannot
be \emph{given in species} in this sense at all.  The confusion in
Proclus' mind is shown when, after saying that a right angle is given
\emph{in species}, he describes a third of a right angle as given
\emph{in magnitude}.

No better example of what is meant by \emph{given in species}, in its
proper sense, as limited to rectilineal figures, can be quoted than
the given parallelogram in Eucl.\ \prop{6}{28}, to which the required
parallelogram has to be made similar; the former parallelogram is in
fact \emph{given in species}, though its actual size, or scale, is
indifferent

(4)~\emph{Given in ratio} presumably means something which is given by
means of its ratio to some other given thing. This we gather from
Proclus' remark (in his note on \prop{1}{9}) that an angle may be
given in ratio ``as when we say that it is double and treble of such
and such an angle or, generally, greater and less.'' The term,
however, appears to have no authority and to serve no purpose.
Proclus may have derived it from such expressions as ``in a given
ratio'' which are common enough.

\subsection{Lemma}

``The term \emph{lemma}'' says Proclus\footnote{Proclus, pp.~211,
  1–212, 4.}, ``is often used of any proposition which is assumed for
the construction of something else: thus it is a common remark that a
proof has been made out of such and such lemmas.  But the special
meaning of \emph{lemma} in geometry is a proposition requiring
confirmation.  For when, in either construction or demonstration, we
assume anything which has not been proved but requires argument, then,
because we regard what has been assumed as doubtful in itself and
therefore worthy of investigation, we call it a
\emph{lemma}\footnote{It would appear, says Tannery (p.~151~\emph{n}),
  that Geminus understood a lemma as being simply
  \greek{λαμβανόμενον}, something assumed (cf.\ the passage of
  Proclus, p.~73, 4, relating to Menaechmus' view of \emph{elements}):
  hence we cannot consider ourselves authorised in attributing to
  Geminus the more technical definition of the term here given by
  Proclus, according to which it is only used of propositions not
  proved beforehand. This view of a lemma must be considered as
  relatively modern. It seems to have had its origin in an
  imperfection of method. In the course of a demonstration it was
  necessary to assume a proposition which required proof, but the
  proof of which would, if inserted in the particular place, break the
  thread of the demonstration: hence it was necessary either to prove
  it beforehand as a preliminary proposition or to postpone it to be
  proved afterwards (\greek{ὡς ἐξῆς δειχθήσεται}).  when, after the
  time of Geminus, the progress of original discovery in geometry was
  arrested, geometers occupied themselves with the study and
  elucidation of the works of the great mathematicians who had
  preceded them. This involved the investigation of propositions
  explicitly quoted or tacitly assumed in the great classical
  treatises; and naturally it was found that several such remained to
  be demonstrated, either because Lhe authors had omitted them as
  being easy enough to be left to the reader himself to prove, or
  because books in which they were proved had been lost in the
  meantime. Hence arose a class of complementary or auxiliary
  propositions which were called lemmas. Thus Pappus gives in his
  Book~\r7\ a collection of lemmas in elucidation of the treatises of
  Euclid and Apollonius included in the so-called ``Treasury of
  Analysis (\greek{τόπος ἀναλυόμενος}).  When Proclus goes on to
  distinguish three methods of discovering lemmas, \emph{analysis},
  \emph{division}, and \emph{reductio ad absurdum}, he seems to imply
  that the principal business of contemporary geometers was the
  investigation of these auxiliary propositions.}, differing as it
does from the postulate and the axiom in being matter of
demonstration, whereas they are immediately taken for granted, without
demonstration, for the purpose of confirming other things.  Now in the
discovery of lemmas the best aid is a mental aptitude for it.  For we
may see many who are quick at solutions and yet do not work by method;
thus Cratistus in our time was able to obtain the required result from
first principles, and those the fewest possible, but it was his
natural gift which helped him to the discovery.  Nevertheless certain
methods have been handed down. The finest is the method which by means
of \emph{analysis} carries the thing sought up to an acknowledged
principle, a method which Plato, as they say, communicated to
Leodamas\footnote{This passage and another from Diogenes Laertius
  (\r3.~24, p.~74 ed.\ Cobet) to the effect that ``He [Plato]
  explained (\greek{εἰσηγήσατο}) to Leodamas of Thasos the method of
  inquiry by analysis ``have been commonly understood as ascribing to
  Plato the \emph{invention} of the method of analysis; but Tannery
  points out forcibly (pp.~112, 113) bow difficult it is to explain in
  what Plato's discovery could have consisted if \emph{analysis} be
  taken in the sense attributed to it in Pappus, where we can see no
  more than a series of successive \emph{reductions} of a problem
  until it is finally reduced to a known problem. On the other hand,
  Proclus' words about carrying up the thing sought to ``an
  acknowledged principle'' suggest that what he had in mind was the
  process described at the end of Book~\r6\ of the \emph{Republic} by
  which the dialectician (unlike the mathematician) uses hypotheses as
  stepping-stones up to a principle which is not hypothetical, and
  then is able to descend step by step verifying every one of the
  hypotheses by which he ascended. This description does not of course
  refer to mathematical analysis, but it may have given rise to the
  idea that analysis was Plato's discovery, since \emph{analysis} and
  \emph{synthesis} following each other are related in the same way as
  the upward and the downward progression in the dialectician's
  intellectual method. And it may be that Plato's achievement was to
  observe the importance, from the point of view of logical rigour, of
  the confirmatory synthesis following analysis, and to regularise in
  this way and elevate into a completely irrefragable method the
  partial and uncertain analysis upon which the works of his
  predecessors depended.}, and by which the latter, too, is said to
have discovered many things in geometry. The second is the method of
\emph{division}\footnote{Here again the successive bipartitions of
  genera into species such as we find in the \emph{Sophist} and
  \emph{Republic} have very little to say to geometry, and the very
  fact that they are here mentioned side by side with analysis
  suggests that Proclus confused the latter with the philosophical
  method of \emph{Rep.}~\r6.}, which divides into its parts the genus
proposed for consideration and gives a starting-point for the
demonstration by means of the elimination of the other elements in the
construction of what is proposed, which method also Plato extolled as
being of assistance to all sciences. The third is that by means of the
\emph{reductio ad absurdum}, which does not show what is sought
directly but refutes its opposite and discovers the truth
incidentally.''

\subsubsection{Case}

``The \emph{case}\footnote{Tannery rightly remarks (p.~152) that the
  subdivision of a theorem or problem into several cases is foreign to
  the really classic form; the ancients preferred, where necessary, to
  multiply enunciations. As, however, some omissions necessarily
  occurred, the writers of lemmas naturally added separate
  \emph{cases}, which in some instances found their way into the text.
  A good example is Euclid \prop{1}{7}, the second case of which, as
  it appears in our text-books, was interpolated. On the commentary of
  Proclus on this proposition Th.~Taylor rightly remarks that ``Euclid
  everywhere avoids a multitude of cases.''} (\greek{πτῶσις}),''
Proclus proceeds\footnote{Proclus, p.~212, 5–11.}, ``announces
different ways of construction and alteration of positions due to the
transposition of points or lines or planes or solids.  And, in
general, all its varieties are seen in the figure, and this is why it
is called case, being a transposition in the construction.''

\subsection{Porism}

``The term \emph{porism} is used also of certain problems such as the
Porisms written by Euclid.  But it is specially used when from what
has been demonstrated some other theorem is revealed at the same time
without our propounding it, which theorem has on this very account
been called a \emph{porism} (corollary) as being a sort of incidental
gain arising from the scientific demonstration\footnote{Tannery notes
  however that, so far from distinguishing his corollaries from the
  conclusions of his propositions, Euclid inserts them before the
  closing words ``(being) what it was required to do'' or ``to
  prove.'' In fact the porism-corollary is with Euclid rather a
  modified form of the regular conclusion than a separate
  proposition.}.'' Cf.\ the note on \prop{1}{15}.

\subsection{Objection}

``The \emph{objection} (\greek{ἔνστασις}) obstructs the whole course
of the argument by appearing as an obstacle (or crying `halt,'
\greek{ἀπαντο›σα}) either to the construction or to the
demonstration. There is this difference between the \emph{objection}
and the \emph{case}, that, whereas he who propounds the case has to
prove the proposition to be true of it, he who makes the objection
does not need to prove anything: on the contrary it is necessary to
destroy the objection and to show that its author is saying what is
false\footnote{Proclus, p.~212, 18–23.}.''

That is, in general the \emph{objection} endeavours to make it appear
that the demonstration is not true in every case; and it is then
necessary to prove, in refutation of the objection, either that the
supposed case is impossible, or that the demonstration \emph{is} true
even for that case.  A good instance is afforded by
Eucl.\ \prop{1}{7}, The text-books give a second case which is not in
the original text of Euclid.  Proclus remarks on the proposition as
given by Euclid that the objection may conceivably be raised that what
Euclid declares to be impossible may after all be possible in the
event of one pair of straight lines falling completely within the
other pair.  Proclus then refutes the objection by proving the
impossibility in that case also.  His proof then came to be given in
the text-books as part of Euclid's proposition.

The \emph{objection} is one of the technical terms in Aristotle's
logic and its nature is explained in the \emph{Prior
  Analytics}\footnote{\emph{Anal.\ prior.}\ \r2.~26, 69~a~37.}. ``An
\emph{objection} is a proposition contrary to a proposition….
Objections are of two sorts, general or partial….  For when it is
maintained that an attribute belongs to every (member of a class), we
object either that it belongs to none (of the class) or that there is
some one (member of the class) to which it does not belong.''

\subsection{Reduction}

This is again an Aristotelian term, explained in the \emph{Prior
  Analytics}\footnote{\ibid~\r2.~25, 69~a~20.}.  It is well described
by Proclus in the following passage:

``Reduction (\greek{ἀπαγωγή}) is a transition from one problem or
theorem to another, the solution or proof of which makes that which is
propounded manifest also.  For example, after the doubling of the cube
had been investigated, they transformed the investigation into another
upon which it follows, namely the finding of the two means; and from
that time forward they inquired how between two given straight lines
two mean proportionals could be discovered.  And they say that the
first to effect the reduction of difficult constructions was
Hippocrates of Chios, who also squared a lune and discovered many
other things in geometry, being second to none in ingenuity as regards
constructions\footnote{Proclus, pp.~212, 24–213, \r2. This passage has
  frequently been taken as crediting Hippocrates with the discovery of
  the method of geometrical reduction: cf.\ Taylor (Translation of
  Proclus, \r2.~ p.~26), Allman (p.~41~\emph{n.}, 59), Gow (pp.~169,
  170), As Tannery remarks (p.~110), if the particular reduction of
  the duplication problem to that of the two means is the first noted
  in history, it is difficult to suppose that it was really the first;
  for Hippocrates must have found instances of it in the Pythagorean
  geometry. Bretschneider, I think, comes nearer the truth when he
  boldly (p.~99) translates: ``This reduction \emph{of the aforesaid
    construction} is said to have been first given by Hippocrates.''
  The words are \greek{πρῶτον δέ φασι τῶν ἀποπουμένων διαγραμμάτων τὴν
    ἀπαγωγὴν ποιήσασθαι}, which must, literally, be translated as in
  the text above; but, when Proclus speaks vaguely of ``difficult
  constructions,'' he probably means to say simply that ``this first
  recorded instance of a reduction of a difficult construction is
  attributed to Hippocrates.''}.''

\subsection{Reductio ad absurdum}

This is variously called by Aristotle ``\emph{reductio ad absurdum}''
(\greek{ἡ εἰς τὸ ἀδύνατον ἀπαγωγή})\footnote{Aristotle,
  \greek{Anal.\ prior.}\ \r1.~7, 29~b~5; \r1.~44, 50~a~30.}, ``proof
\emph{per impossible}'' (\greek{ἡ διὰ τοῦ ἀδυνάτου
  δειξις})\footnote{\ibid~\r1. 21, 39~b~32; \r1.~29, 45~a~35.},
``proof leading to the impossible''
(\greek{ἀπόδειξις})\footnote{\emph{Anal.\ post.}\ \r1.~24, 85~a~16
  etc.}.  It is part of ``proof (starting) from a
hypothesis\footnote{\emph{Anal.\ prior.}\ \r1.~23,
  40~b~25.}''`(\greek{ἡ εἰς τὸ ἀδύνατον ἄγουσα ἀπόδειξις}). ``All
(syllogisms) which reach the conclusion \emph{per impossibile} reason
out a conclusion which is false, and they prove the original
contention (by the method starting) from a hypothesis, when something
impossible results from assuming the contradictory of the original
contention, as, for example, when it is proved that the diagonal (of a
square) is incommensurable because, if it be assumed commensurable, it
will follow that odd (numbers) are equal to even
(numbers)\footnote{\emph{Anal.\ prior.}\ \r1.~23, 41~a~24.}.'' Or
again, ``proof (leading) to the impossible differs from the direct
(\greek{δεικτικῆσ}) in that it assumes what it desires to destroy
[namely the hypothesis of the falsity of the conclusion] and then
reduces it to something admittedly false, whereas the direct proof
starts from premisses admittedly true\footnote{\ibid~\r2.~14,
  62~b~29.}.''

Proclus has the following description of the \emph{reductio ad
  absurdum}.  ``Proofs by \emph{reductio ad absurdum} in every case
reach a conclusion manifestly impossible, a conclusion the
contradictory of which is admitted.  In some cases the conclusions are
found to conflict with the common notions, or the postulates, or the
hypotheses (from which we started); in others they contradict
propositions previously established\footnote{Proclus, p.~254,
  22–27.}''…''Every \emph{reductio ad absurdum} assumes what conflicts
with the desired result, then, using that as a basis, proceeds until
it arrives at an admitted absurdity, and, by thus destroying the
hypothesis, establishes the result originally desired.  For it is
necessary to understand generally that all mathematical arguments
either proceed from the first principles or lead back to them, as
Porphyry somewhere says.  And those which proceed from the first
principles are again of two kinds, for they start either from common
notions and the clearness of the self-evident alone, or from results
previously proved; while those which lead back to the principles are
either by way of assuming the principles or by way of destroying
them. Those which assume the principles are called analyses, and the
opposite of these are \emph{syntheses}—for it is possible to start
from the said principles and to proceed in the regular order to the
desired conclusion, and this process is \emph{synthesis}—while the
arguments which would destroy the principles are called
\emph{reductiones ad absurdum}.  For it is the function of this method
to upset something admitted as clear\footnote{Proclus, p.~255,
  8–26.}.''

\subsection{Analysis and Synthesis}

It will be seen from the note on Eucl.\ \prop{13}{1} that the
\textsc{mss.}\ of the \emph{Elements} contain definitions of
\emph{Analysis} and \emph{Synthesis} followed by alternative proofs of
\prop{13}{1–5} after that method. The definitions and alternative
proofs are interpolated, but they have great historical interest
because of the possibility that they represent an ancient method of
dealing with these propositions, anterior to Euclid. The propositions
give properties of a line cut ``in extreme and mean ratio,'' and they
are preliminary to the construction and comparison of the five regular
solids.  Now Pappus, in the section of his \emph{Collection} dealing
with the latter subject\footnote{Pappus, \r5.~p.~410 sqq.}, says that
he will give the comparisons between the five figures, the pyramid,
cube, octahedron, dodecahedron and icosahedron, which have equal
surfaces, ``not by means of the so-called analytical inquiry, by which
some of the ancients worked out the proofs, but by the synthetical
method\footnote{\ibid~pp.~410, 27–412, 2.}….''  The conjecture of
Bretschneider that the matter interpolated in Eucl.\ \r13.\ is a
survival of investigations due to Eudoxus has at first sight much to
commend it\footnote{Bretschneider, p.~168. See however Heiberg's
  recent suggestion (\emph{Paralipomena zu Euklid} in \emph{Hermes},
  \r38., 1903) that the author was Heron. The suggestion is based on a
  comparison with the remarks on analysis and synthesis quoted from
  Heron by an-Nairīzī (ed.\ Curtze, p.~89) at the beginning of his
  commentary on Eucl.\ Book~\r2.  On the whole, this suggestion
  commends itself to me more than that of Bretschneider.}.  In the
first place, we are told by Proclus that Eudoxus ``greatly added to
the number of the theorems which Plato originated regarding \emph{the
  section}, and employed in them the method of
analysis\footnote{Proclus, p.~67, 6.}.''  It is obvious that
``\emph{the section}'' was some particular section which by the time
of Plato had assumed great importance; and the one section of which
this can safely be said is that which was called the ``golden
section,'' namely, the division of a straight line in extreme and mean
ratio which appears in Eucl.\ \prop{2}{11} and is therefore most
probably Pythagorean.  Secondly, as Cantor points out\footnote{Cantor,
  \emph{Gesch.\ d.\ Math.}\ \r1\tsub{3}, p.~241.}, Eudoxus was the
founder of the theory of proportions in the form in which we find it
in Euclid \r5., \r6., and it was no doubt through meeting, in the
course of his investigations, with proportions not expressible by
whole numbers that he came to realise the necessity for a new theory
of proportions which should be applicable to incommensurable as well
as commensurable magnitudes. The ``golden section'' would furnish such
a case.  And it is even mentioned by Proclus in this connexion. He is
explaining\footnote{Proclus, p, 60, 7–9.} that it is only in
arithmetic that all quantities bear ``rational'' ratios (\greek{ῥητὸς
  λόγος}) to one another, while in geometry there are ``irrational''
ones (\greek{ἄρρητος}) as well. ``Theorems about sections like those
in Euclid's second Book are common to both [arithmetic and geometry]
\emph{except that in which the straight line is cut in extreme and
  mean ratio}\footnote{\ibid~p.~60, 16–19.}.''

The definitions of \emph{Analysis} and \emph{Synthesis} interpolated
in Eucl.~\r13.\ are as follows (I adopt the reading of B and V, the
only intelligible one, for the second).

``\textbf{Analysis} is an assumption of that which is sought as if it
were admitted $<$~and the passage~$>$ through its consequences to
something admitted (to be) true.

``\textbf{Synthesis} is an assumption of that which is admitted
$<$~and the passage~$>$ through its consequences to the finishing or
attainment of what is sought.''

The language is by no means clear and has, at the best, to be filled
out.

Pappus has a fuller account\footnote{Pappus, \r7.~pp.~634–6.}:

``The so-called \greek{ἀναλυόμενος} (`Treasury of Analysis') is, to
put it shortly, a special body of doctrine provided for the use of
those who, after finishing the ordinary Elements, are desirous of
acquiring the power of solving problems which may be set them
involving (the construction of) lines, and it is useful for this
alone.  It is the work of three men, Euclid the author of the
Elements, Apollonius of Perga, and Aristaeus the elder, and proceeds
by way of analysis and synthesis.

``\textbf{Analysis} then takes that which is sought as if it were
admitted and passes from it through its successive consequences to
something which is admitted as the result of synthesis: for in
analysis we assume that which is sought as if it were (already) done
(\greek{γεγονός}), and we inquire what it is from which this results,
and again what is the antecedent cause of the latter, and so on, until
by so retracing our steps we come upon something already known or
belonging to the class of first principles, and such a method we call
analysis as being solution backwards (\greek{ἀνάπαλιν λύσιν}).

``But in \textbf{synthesis}, reversing the process, we take as already
done that which was last arrived at in the analysis and, by arranging
in their natural order as consequences what were before antecedents,
and successively connecting them one with another, we arrive finally
at the construction of what was sought; and this we call synthesis.

``Now analysis is of two kinds, the one directed to searching for the
truth and called \emph{theoretical}, the other directed to finding
what we are told to find and called \emph{problematical}, (1)~In the
\emph{theoretical} kind we assume what is sought as if it were
existent and true, after which we pass through its successive
consequences, as if they too were true and established by virtue of
our hypothesis, to something admitted: then (\emph{a}), if that
something admitted is true, that which is sought will also be true and
the proof will correspond in the reverse order to the analysis, but
(\emph{b}), if we come upon something admittedly false, that which is
sought will also be false. (2)~In the \emph{problematical} kind we
assume that which is propounded as if it were known, after which we
pass through its successive consequences, taking them as true, up to
something admitted: if then (\emph{a}) what is admitted is possible
and obtainable, that is, what mathematicians call given, what was
originally proposed will also be possible, and the proof will again
correspond in reverse order to the analysis, but if (\emph{b})~we come
upon something admittedly impossible, the problem will also be
impossible.''

The ancient Analysis has been made the subject of careful studies by
several writers during the last half-century, the most complete being
those of Hankel, Duhamel and Zeuthen; others by Ofterdinger and Cantor
should also be mentioned\footnote{Hankel, \emph{Zur Geschichte der Mathematik in Alterthum und
  Mittelalter}, 1874, pp.~137–150;
   Duhmamel, \emph{Des méthodes dans les sciences de raisonnement},
  Part~\r1, 3~ed., Paris, 1885, pp.~39–68;
   Zeuthen, \emph{Geschichte der Mathematik im Altertum und
  Mittelalter}, 1896, pp.~92–104;
   Ofterdinger, \emph{Beiträge zur Geschichte der griechischen
  Mathematik}, Ulm, 1860;
   Cantor, \emph{Geschichte der Mathematik}, \r1\tsub{3}, pp.~220–2.}.

The method is as follows.  It is required, let us say, to prove that a
certain proposition~A is true.  We assume as a hypothesis that A~is
true and, starting from this we find that, if A~is true, a certain
other proposition~B is true; if B~is true, then~C; and so on until we
arrive at a proposition~K which is \emph{admittedly} true.  The object
of the method is to enable us to infer, in the reverse order, that,
since K~is true, the proposition~A originally assumed is true.  Now
Aristotle had already made it clear that false hypotheses might lead
to a conclusion which is true. There is therefore a possibility of
error unless a certain precaution is taken. While, for example, B~may
be a necessary consequence of~A, it may happen that A~is not a
necessary consequence of~B, Thus, in order that the reverse inference
from the truth of~K that A~is true may be logically justified, it is
necessary that each step in the chain of inferences should be
unconditionally convertible.  As a matter of fact, a very large number
of theorems in elementary geometry are unconditionally convertible, so
that in practice the difficulty in securing that the successive steps
shall be convertible is not so great as might be supposed.  But care
is always necessary.  For example, as Hankel says\footnote{Hankel,
  p.~139.}, a proposition may not be unconditionally convertible in
the form in which it is generally quoted.  Thus the proposition ``The
vertices of all triangles having a common base and constant vertical
angle lie on a circle'' cannot be converted into the proposition that
``All triangles with common base and vertices lying on a circle have a
constant vertical angle''; for this is only true if the further
conditions are satisfied (1)~that the circle passes through the
extremities of the common base and (2)~that only that part of the
circle is taken as the locus of the vertices which lies on one side of
the base. If these conditions are added, the proposition is
unconditionally convertible. Or again, as Zeuthen
remarks\footnote{Zeuthen, p.~103.}, K~may be obtained by a series of
inferences in which A or some other proposition in the series is only
\emph{apparently} used; this would be the case e.g.\ when the method
of modern algebra is being employed and the expressions on each side
of the sign of equality have been inadvertently multiplied by some
composite magnitude which is in reality equal to zero.

Although the above extract from Pappus does not make it clear that
each step in the chain of argument must be convertible in the case
taken, he almost implies this in the second part of the definition jf
Analysis where, instead of speaking of the consequences B,~C…
successively following from~A, he suddenly changes the expression and
says that we inquire \emph{what it is} (B) \emph{from which} A
\emph{follows} (A~being thus the consequence of~B, instead of the
reverse), and then what (viz.~C) is the antecedent cause of~B; and in
practice the Greeks secured what was wanted by always insisting on the
analysis being confirmed by subsequent synthesis, that is, they
laboriously worked backwards the whole way from~K to~A, reversing the
order of the analysis, which process would undoubtedly bring to light
any flaw which had crept into the argument through the accidental
neglect of the necessary precautions.

\emph{Reductio ad absurdum a variety of analysis}

In the process of analysis starting from the hypothesis that a
proposition~A is true and passing through B,~C… as successive
consequences we may arrive at a proposition~K which, instead of being
admittedly true, is either admittedly false or the contradictory of
the original hypothesis~A or of some one or more of the propositions
B,~C… intermediate between A and~K\@. Now correct inference from a
true proposition cannot lead to a false proposition; and in this case
therefore we may at once conclude, without any inquiry whether the
various steps in the argument are convertible or not, that the
hypothesis~A is false, for, if it were true, all the consequences
correctly inferred from it would be true and no incompatibility could
arise.  This method of proving that a given hypothesis is \emph{false}
furnishes an indirect method of proving that a given hypothesis~A is
\emph{true}, since we have only to take the \emph{contradictory} of~A
and to prove that it is false.  This is the method of \emph{reductio
  ad absurdum}, which is therefore a variety of analysis.  The
contradictory of~A, or not-A, will generally include more than one
case and, in order to prove its falsity, each of the cases must be
separately disposed of: e.g., if it is desired to prove that a certain
part of a figure is \emph{equal} to some other part, we take
separately the hypotheses (1)~that it is \emph{greater}, (2)~that it
is \emph{less}, and prove that each of these hypotheses leads to a
conclusion either admittedly false or contradictory to the hypothesis
itself or to some one of its consequences.

\textbf{Analysis as applied to problems}

It is in relation to problems that the ancient analysis has the
greatest significance, because it was the one general method which the
Greeks used for solving all ``the more abstruse problems'' (\greek{τὰ
  ἀσαφέστερα τῶν προβλημάτων})\footnote{Proclus, p.~242, 16, 17.}.

We have, let us suppose, to construct a figure satisfying a certain
set of conditions.  If we are to proceed at all methodically and not
by mere guesswork, it is first necessary to ``analyse'' those
conditions.  To enable this to be done we must get them clearly in our
minds, which is only possible by assuming all the conditions to be
actually fulfilled, in other words, by supposing the problem
solved. Then we have to transform those conditions, by all the means
which practice in such cases has taught us to employ, into other
conditions which are necessarily fulfilled if the original conditions
are, and to continue this transformation until we at length arrive at
conditions which we are in a position to satisfy\footnote{Zeuthen,
  p.~93.}.  In other words, we must arrive at some relation which
enables us to \emph{construct} a particular part of the figure which,
it is true, has been hypothetically assumed and even drawn, but which
nevertheless really requires to be \emph{found} in order that the
problem may be solved.  From that moment the particular part of the
figure becomes one of the \emph{data}, and a fresh relation has to be
found which enables a fresh part of the figure to be determined by
means of the original data and the new one together.  When this is
done, the second new part of the figure also belongs to the data; and
we proceed in this way until all the parts of the required figure are
found\footnote{Hankel, p.~141.}.  The first part of the analysis down
to the point of discovery of a relation which enables us to say that a
certain new part of the figure not belonging to the original data is
\emph{given}, Hankel calls the \emph{transformation}; the second part,
in which it is proved that all the remaining parts of the figure are
``given,'' he calls the \emph{resolution}. Then follows the
\emph{synthesis}, which also consists of two parts, (1)~the
\emph{construction}, in the order in which it has to be actually
carried out, and in general following the course of the second part of
the analysis, the \emph{resolution}; (2)~the \emph{demonstration} that
the figure obtained does satisfy all the given conditions, which
follows the steps of the first part of the analysis, the
\emph{transformation}, but in the reverse order.  The second part of
the analysis, the \emph{resolution}, would be much facilitated and
shortened by the existence of a systematic collection of \emph{Data}
such as Euclid's book bearing that title, consisting of propositions
proving that, if in a figure certain parts or relations are
\emph{given}, other parts or relations are also \emph{given}.  As
regards the first part of the analysis, the \emph{transformation}, the
usual rule applies that every step in the chain must be
unconditionally convertible; and any failure to observe this condition
will be brought to light by the subsequent synthesis.  The second
part, the \emph{resolution}, can be directly turned into the
\emph{construction} since that only is \emph{given} which can be
constructed by the means provided in the \emph{Elements}.

It would be difficult to find a better illustration of the above than
the example chosen by Hankel from Pappus\footnote{Pappus,
  \r7.~pp.~830–2.}.

\emph{Given a circle $ABC$ and two points $D$, $E$ external to it, to
  draw straight lines $DB$, $KB$ from $D$, $E$ to a point~$B$ on the
  circle such that, if $DB$, $KB$ produced meet the circle again in
  $C$, $A$, $AC$ shall be parallel to $DE$.}

\textbf{Analysis.}

Suppose the problem solved and the tangent at $A$ drawn, meeting $ED$
produced in~$F$.

(Part I. \emph{Transformation}.)

Then, since $AC$ is parallel to $DE$, the angle at~$C$ is equal to the
angle~$CDE$.

But, since $FA$ is a tangent, the angle at~$C$ is equal to the
angle~$FAE$.

Therefore the angle $FAE$ is equal to the angle $CDE$, whence $A$,
$B$, $D$, are concyclic.

Therefore the rectangle $AE$, $EB$ is equal to the rectangle $FE$,
$ED$.

\sidefig{introI_2}

(Part II, \emph{Resolution}.)

But the rectangle $AE$, $EB$ is given, because it is equal to the
square on the tangent from~$E$.

Therefore the rectangle $FE$, $ED$ is given;\0
and, since $ED$ is given, $FE$ is given (in length). [\emph{Data}, 57.]

But $FE$ is given in position also, so
that $F$ is also given. [\emph{Data}, 27.]

Now $FA$ is the tangent from a given point $F$ to a circle~$ABC$ given
in position; therefore $FA$ is given in position and
magnitude. [\emph{Data}, 90.]

And $F$ is given; therefore $A$ is given.

But $E$ is also given; therefore the straight line $AE$ is given in
position, [\emph{Data}, 26.]

And the circle $ABC$ is given in position; therefore the point~$B$ is
also given. [\emph{Data}, 25.]

But the points $D$, $E$ are also given;
therefore the straight lines $DB$, $BE$ are also given in position.

\textbf{Synthesis.}

(Part I. \emph{Construction})

Suppose the circle $ABC$ and the points $D$, $E$ given.

Take a rectangle contained by $ED$ and by a certain straight line $EF$
equal to the square on the tangent to the circle from~$E$.

From $F$ draw $FA$ touching the circle in~$A$; join $ABE$ and then
$DB$, producing $DB$ to meet the circle at~$C$. Join~$AC$.

I say then that $AC$ is parallel to~$DE$.

(Part II. \emph{Demonstration}.)

Since, by hypothesis, the rectangle $FE$, $ED$ is equal to the square
on the tangent from~$E$, which again is equal to the rectangle $AE$,
$EB$, the rectangle $AE$, $EB$ is equal to the rectangle $FE$, $ED$,

Therefore $A$, $B$, $D$, $F$ are concyclic,
whence the angle $FAE$ is equal to the angle~$BDE$.

But the angle $FAE$ is equal to the angle $ACB$ in the alternate
segment;\0
therefore the angle $ACB$ is equal to the angle $BDE$.

Therefore $AC$ is parallel to $DE$.

In cases where a \greek{διορισμός} is necessary, i.e.\ where a
solution is only possible under certain conditions, the analysis will
enable those conditions to be ascertained. Sometimes the
\greek{διορισμός} is stated and proved at the end of the analysis,
e.g.\ in Archimedes, \emph{On the Sphere and Cylinder}, \prop{2}{7};
sometimes it is stated in that place and the proof postponed till
after the end of the synthesis, e.g.\ in the solution of the problem
subsidiary to \emph{On the Sphere and Cylinder}, \prop{2}{4},
preserved in Eutocius' commentary on that proposition. The analysis
should also enable us to determine the number of solutions of which
the problem is susceptible.

\section{The Definitions}

\textbf{General. ``Real'' and ``Nominal'' Definitions.}

It is necessary, says Aristotle whenever any one treats of any whole
subject, to divide the genus into its primary constituents, those
which are indivisible in species respectively: e.g.\ number must be
divided into triad and dyad; then an attempt must be made in this way
to obtain definitions, e.g.\ of a straight line, of a circle, and of a
right angle\footnote{\emph{Anal. post.}\ \r2.\ 13, 96~b~15.}.

The word for definition is \greek{ὅρος}. The original meaning of this
word seems to have been ``boundary,'' ``landmark.'' Then we have it in
Plato and Aristotle in the sense of standard or determining principle
(``id quo alicuius rei natura constituitur vel definitur,''
\emph{Index Aristotelicus})\footnote{cf.\ \emph{De anima}, \r1.~1,
  404~a~9, where ``breathing'' is spoken of as the \greek{ὅρος} of
  ``life,'' and the many passages in the \emph{Politics} where the
  word is used to denote that which gives its \emph{special character}
  to the several forms of government (virtue being the \greek{ὄρος} of
  aristocracy, wealth of oligarchy, liberty of democracy, 1294~a~10),
  Plato, \emph{Republic}, \r8,~551~\textsc{c}.}; and closely connected
with this is the sense of definition. Aristotle uses both \greek{ὅρος}
and \greek{ὁρισμός} for definition, the former occurring more
frequently in the \emph{Topics}, the latter in the \emph{Metaphysics}.

Let us now first be clear as to what a definition does \emph{not} do.
There is nothing in connexion with definitions which Aristotle takes
more pains to emphasise than that a definition asserts nothing as to
the \emph{existence} or \emph{non-existence} of the thing defined.  It
is an answer to the question what a thing is (\greek{τί ἐστι}), and
does not say that it is (\greek{ὅτι ἔστι}).  The \emph{existence} of
the various things defined has to be \emph{proved}, except in the case
of a few primary things in each science, the existence of which is
indemonstrable and must be assumed among the first principles of each
science; e.g.\ points and lines in geometry must be assumed to exist,
but the existence of everything else must be proved.  This is stated
clearly in the long passage quoted above under First
Principles\footnote{\emph{Anal.\ post.}\ \r1.~10, 76~a~31 sqq.}.  It
is reasserted in such passages as the following. ``The (answer to the
question) \emph{what is a man} and \emph{the fact that a man exists}
are different things\footnote{\ibid~\r2.~7, 92~b~10.}.''  ``It is
clear that, even according to the view of definitions now current,
those who define things do not prove that they
exist\footnote{\ibid~92~b~19.}.''  ``We say that it is by
\emph{demonstration} that we must show that everything exists, except
essence (\greek{εἰ μὴ οὐσία εἴη}).  But the \emph{existence} of a
thing is never essence; for the \emph{existent} is not a genus.
Therefore there must be demonstration that a thing exists.  Thus,
\emph{what is meant by triangle} the geometer assumes, but that it
exists he has to prove\footnote{\ibid~92~b~12 sqq.}.''  ``Anterior
knowledge of two sorts is necessary: for it is necessary to
presuppose, with regard to some things, that they \emph{exist}; in
other cases it is necessary to understand \emph{what} the thing
described is, and in other cases it is necessary to do both. Thus,
with the fact that one of two contradictories must be true, we must
know that it exists (is true); of the triangle we must know that it
means such and such a thing; of the unit we must know both what it
means and that it exists\footnote{\emph{Anal.\ post.}\ \r1.~1, 71~a~11
  sqq.}.''  What is here so much insisted on is the very fact which
Mill pointed out in his discussion of earlier views of Definitions,
where he says that the so-called \emph{real} definitions or
definitions of \emph{things} do not constitute a different kind of
definition from \emph{nominal} definitions, or definitions of
\emph{names}; the former is simply the latter \emph{plus} something
else, namely a covert assertion that the thing defined exists, ``This
covert assertion is not a definition but a postulate. The definition
is a mere identical proposition which gives information only about the
use of language, and from which no conclusion affecting matters of
fact can possibly be drawn. The accompanying postulate, on the other
hand, affirms a fact which may lead to consequences of every degree of
importance.  It affirms the actual or possible existence of Things
possessing the combination of attributes set forth in the definition:
and this, if true, may be foundation sufficient on which to build a
whole fabric of scientific truth\footnote{Mill's \emph{System of
    Logic}, Bk.~\r1.~ch.~viii.}.''  This statement really adds nothing
to Aristotle's doctrine\footnote{It is true that it was in opposition
  to ``the ideas of most of the \emph{Aristotelian logicians}''
  (rather than of Aristotle himself) that Mill laid such stress on his
  point of view.  Cf.\ his observation: ``We have already made, and
  shall often have to repeat, the remark, that the philosophers who
  overthrew Realism by no means got rid of the consequences of
  Realism, but retained long afterwards, in their own philosophy,
  numerous propositions which could only have a rational meaning as
  part of a Realistic system. It had been handed down from Aristotle,
  and probably from earlier times, as an obvious truth, that the
  science of geometry is deduced from definitions. This, so long as a
  definition was considered to be a proposition `unfolding the nature
  of the thing,' did well enough.  But Hobbes followed and rejected
  utterly the notion that a definition declares the nature of the
  thing, or does anything but state the meaning of a name; yet he
  continued to affirm as broadly as any of his predecessors that the
  \greek{ἀρχαί}, \emph{principia}, or original premisses of
  mathematics, and even of all science, are definitions; producing the
  singular paradox that systems of scientific truth, nay, all truths
  whatever at which we arrive by reasoning, are deduced from the
  arbitrary conventions of mankind concerning the signification of
  words.'' Aristotle was guilty of no such paradox; on the contrary,
  he exposed it as plainly as did Mill.}; it has even the slight
disadvantage, due to the use of the word ``postulate'' to describe
``the covert assertion'' in all cases, of not definitely pointing out
that there are cases where existence has to be \emph{proved} as
distinct from those where it must be \emph{assumed}.  It is true that
the existence of a definiend may have to be taken for granted
provisionally until the time comes for proving it; but, so far as
regards any case where existence must be proved sooner or later, the
provisional assumption would be for Aristotle, not a \emph{postulate},
but a \emph{hypothesis}.  In modern times, too, Mill's account of the
true distinction between \emph{real} and \emph{nominal} definitions
had been fully anticipated by Saccheri\footnote{This has been fully
  brought out in two papers by G.~Vailati, \emph{La teoria
    Aristotelica della definizione} (\emph{Rivista di Filosofia e
    scienze affini}. 1903), and \emph{Di un' opera dimenticata del
    P.~Saccheri} (``Logica Demonstrativa,'' 1697) (in \emph{Rivista
    Filosofica}, 1903).}, the editor of \emph{Euclides ab omni naevo
  vindicatas} (1733), famous in the history of non-Euclidean geometry.
In his \emph{Logica Demonstrativa} (to which he also refers in his
Euclid) Saccheri lays down the clear distinction between what he calls
\emph{definitiones quid nominis} or \emph{nominales}, and
\emph{definitiones quid rei} or \emph{reales}, namely that the former
are only intended to explain the meaning that is to be attached to a
given term, whereas the latter, besides declaring the meaning of a
word, affirm at the same time the existence of the thing defined or,
in geometry, the possibility of constructing it.  The \emph{definitio
  quid nominis} becomes a \emph{definitio quid rei} ``by means of a
\emph{postulate}, or when we come to the question whether the thing
\emph{exists} and it is answered affirmatively\footnote{``Definitio
  \emph{quid nominis} nata est evadere definitio \emph{quid rei} per
  \emph{postulatum} vel dum venitur ad quaestionem \emph{an est} et
  respondetur affirmative.''}.''  \emph{Definitiones quid nominis} are
in themselves quite arbitrary, and neither require nor are capable of
proof; they are merely provisional and are only intended to be turned
as quickly as possible into \emph{definitiones quid rei}, either
(1)~by means of a postulate in which it is asserted or conceded that
what is defined exists or can be constructed, e.g.\ in the case of
\emph{straight lines} and \emph{circles}, to which Euclid's first
three postulates refer, or (2)~by means of a demonstration reducing
the construction of the figure defined to the successive carrying-out
of a certain number of those elementary constructions, the possibility
of which is \emph{postulated}. Thus \emph{definitiones quid rei} are
in general obtained as the result of a series of
demonstrations. Saccheri gives as an instance the construction of a
square in Euclid \prop{1}{46}. Suppose that it is objected that Euclid
had no right to define a square, as he does at the beginning of the
Book, when it was not certain that such a figure exists in nature; the
objection, he says, could only have force if, before proving and
making the construction, Euclid had assumed the aforesaid figure as
given.  That Euclid is not guilty of this error is clear from the fact
that he never presupposes the existence of the square as defined until
after \prop{1}{46}.

Confusion between the \emph{nominal} and the \emph{real} definition as
thus described, i.e.\ the use of the former in demonstration before it
has been turned into the latter by the necessary proof that the thing
defined exists, is according to Saccheri one of the most fruitful
sources of illusory demonstration, and the danger is greater in
proportion to the ``complexity'' of the definition, i.e.\ the number
and variety of the attributes belonging to the thing defined.  For the
greater is the possibility that there may be among the attributes some
that are \emph{incompatible}, i.e.\ the simultaneous presence of which
in a given figure can be proved, by means of \emph{other} postulates
etc.\ forming part of the basis of the science, to be impossible.

The same thought is expressed by Leibniz also. ``If,'' he says, ``we
give any definition, and it is not clear from it that the idea, which
we ascribe to the thing, is possible, we cannot rely upon the
demonstrations which we have derived from that definition, because, if
that idea by chance involves a contradiction, it is possible that even
contradictories may be true of it at one and the same time, and thus
our demonstrations will be useless.  Whence it is clear that
definitions are not arbitrary.  And this is a secret which is hardly
sufficiently known\footnote{\emph{Opuscules et fragments inédits de
    Leibniz}, Paris, Alcan, 1903, p.~431.  Quoted by Vailati.}.''
Leibniz' favourite illustration was the ``regular polyhedron with ten
faces,'' the impossibility of which is not obvious at first sight.

It need hardly be added that, speaking generally, Euclid's
definitions, and his use of them, agree with the doctrine of Aristotle
that the definitions themselves say nothing as to the existence of the
things defined, but that the existence of each of them must be proved
or (in the case of the ``principles'') \emph{assumed}.  In geometry,
says Aristotle, the existence of points and lines only must be
assumed, the existence of the rest being proved.  Accordingly Euclid's
first three postulates declare the possibility of constructing
straight lines and circles (the only ``lines'' except straight lines
used in the \emph{Elements}).  Other things are defined and afterwards
constructed and proved to exist: e.g.\ in Book \book{1}{Def.~20}, it
is explained what is meant by an equilateral triangle; then
(\prop{1}{1}) it is proposed to construct it, and, when constructed,
it is proved to agree with the definition.  When a square is defined
(\book{1}{Def.~22}), the question whether such a thing really exists
is left open until, in \prop{1}{46}, it is proposed to construct it
and, when constructed, it is proved to satisfy the
definition\footnote{Trendelenburg, \emph{Elementa Logices
    Aristoteleae}, \S50,}.  Similarly with the right angle
(\book{1}{Def.~10}, and \prop{1}{11}) and parallels
(\book{1}{Def.~23}, and \prop{1}{27–29}).  The greatest care is taken
to exclude mere presumption and imagination.  The transition from the
subjective definition of names to the objective definition of things
is made, in geometry, by means of \emph{constructions} (the first
principles of which are postulated), as in other sciences it is made
by means of experience\footnote{Trendelenburg, \emph{Erläuterungen zu
    den Elementen der aristotelischen Logik}, 3~ed.\ p.~107.  On
  construction as proof of existence in ancient geometry
  cf.\ H.~G. Zeuthen, \emph{Die geometrische Construction als
    ``Existenzbeweis'' in der antiken Geometrie} (in
  \emph{Mathematische Annalen}, 4.~Band).}.

\textbf{Aristotle's requirements in a definition.}

We now come to the positive characteristics by which, according to
Aristotle, scientific definitions must be marked.

\emph{First}, the different attributes in a definition, when taken
separately, cover more than the notion defined, but the combination of
them does not Aristotle illustrates this by the ``triad,'' into which
enter the several notions of number, odd and prime, and the last ``in
both its two senses (\emph{a})~of not being measured by any (other)
number (\greek{ὡς μὴ μετρεῖσθαι ἀριθμῷ}) and (\emph{b}) of not being
obtainable by adding numbers together'' (\greek{ὁς μὴ συγκεῖσθαι ἐξ
  ἀριθμω‹ν}), a unit not being a number.  Of these attributes some are
present in all other odd numbers as well, while the last [primeness in
  the second sense] belongs also to the dyad, but in nothing but the
triad are they all present\footnote{\emph{Anal.\ post.}\ \r2.~13,
  96~a~33–b~1.}.''  The fact can be equally well illustrated from
geometry.  Thus, e.g.\ into the definition of a square
(Eucl.\ \book{1}{Def.~22}) there enter the several notions of figure,
four-sided, equilateral, and right-angled, each of which covers more
than the notion into which all enter as
attributes\footnote{Trendelenburg, \emph{Erläuterungen}, p.~108.}.

\emph{Secondly}, a definition must be expressed in terms of things
which are prior to, and better known than, the things
defined\footnote{\emph{Topics} \r6.~4, 141~a~26 sqq.}. This is clear,
since the object of a definition is to give us knowledge of the thing
defined, and it is by means of things prior and better known that we
acquire fresh knowledge, as in the course of demonstrations.  But the
terms ``prior'' and ``better known'' xare, as usual susceptible of two
meanings; they may mean (1)~\emph{absolutely} or \emph{logically}
prior and better known, or (2)~better known \emph{relatively to us}.
In the absolute sense, or from the standpoint of reason, a point is
better known than a line, a line than a plane, and a plane than a
solid, as also a unit is better known than number (for the unit is
prior to, and the first principle of, any number).  Similarly, in the
absolute sense, a letter is prior to a syllable.  But the case is
sometimes different relatively to us; for example, a solid is more
easily realised by the senses than a plane, a plane than a line, and a
line than a point.  Hence, while it is more scientific to begin with
the \emph{absolutely} prior, it may, perhaps, be permissible, in case
the learner is not capable of following the scientific order, to
explain things by means of what is more intelligible \emph{to him}.
``Among the definitions framed on this principle are those of the
point, the line and the plane; all these explain what is prior by
means of what is posterior, for the point is described as the
extremity of a line, the line of a plane, the plane of a solid.''
But, if it is asserted that such definitions by means of things which
are more intelligible relatively only to a particular individual are
really definitions, it will follow that there may be many definitions
of the same thing, one for each individual for whom a thing is being
defined, and even different definitions for one and the same
individual at different times, since at first sensible objects are
more intelligible, while to a better trained mind they become less so.
It follows therefore that a thing should be defined by means of the
absolutely prior and not the relatively prior, in order that there may
be one sole and immutable definition.  This is further enforced by
reference to the requirement that a good definition must state the
\emph{genus} and the \emph{differentiae}, for these are among the
things which are, in the absolute sense, better known than, and prior
to, the species (\greek{τῶν ἁπλο›ς γνωριμωτέρων καὶ προτέρων τοῦ
  εἴδους ἐστίν}).  For to destroy the genus and the differentia is to
destroy the species, so that the former are \emph{prior} to the
species; they are also \emph{better known}, for, when the species is
known, the genus and the differentia must necessarily be known also,
e.g.\ he who knows ``man'' must also know ``animal'' and
``land-animal,'' but it does not follow, when the genus and
differentia are known, that the species is known too, and hence the
species is less known than they are\footnote{\emph{Topics} \r6.~4,
  141~b~25–34,}.  It may be frankly admitted that the scientific
definition will require superior mental powers for its apprehension;
and the extent of its use must be a matter of discretion. So far
Aristotle; and we have here the best possible explanation why Euclid
supplemented his definition of a point by the statement in
\book{1}{Def.~3} that \emph{the extremities of a line are points} and
his definition of a surface by \book{1}{Def.~6} to the effect that
\emph{the extremities of a surface are lines}. The supplementary
explanations do in fact enable us to arrive at a better understanding
of the formal definitions of a point and a line respectively, as is
well explained by Simson in his note on Def.~1. Simson says, namely,
that we must consider a solid, that is, a magnitude which has length,
breadth and thickness, in order to understand aright the definitions
of a point, a line and a surface. Consider, for instance, the boundary
common to two solids which are contiguous or the boundary which
divides one solid into two contiguous parts; this boundary is a
surface.  We can prove that it has no thickness by taking away either
solid, when it remains the boundary of the other; for, if it had
thickness, the thickness must either be a part of one solid or of the
other, in which case to take away one or other solid would take away
the thickness and therefore the boundary itself: which is impossible.
Therefore the boundary or the surface has no thickness.  In exactly
the same way, regarding a line as the boundary of two contiguous
surfaces, we prove that the line has no breadth; and, lastly,
regarding a point as the common boundary or extremity of two lines, we
prove that a point has no length, breadth or thickness.

\textbf{Aristotle on unscientific definitions.}

Aristotle distinguishes three kinds of definition which are
unscientific because founded on what is \emph{not} prior (\greek{μὴ ἐκ
  προτέρων}).  The \emph{first} is a definition of a thing by means of
its opposite, e,g. of ``good'' by means of ``bad''; this is wrong
because opposites are naturally evolved together, and the knowledge of
opposites is not uncommonly regarded as one and the same, so that one
of the two opposites cannot be better known than the other.  It is
true that, in some cases of opposites, it would appear that no other
sort of definition is possible: e.g.\ it would seem impossible to
define double apart from the half and, generally, this would be the
case with things which in their very nature (\greek{καθ’ αὑτά}) are
\emph{relative} terms (\greek{πρός τι λέγεται}), since one cannot be
known without the other, so that in the notion of either the other
must be comprised as well\footnote{\emph{Topics} \r6.~4,
  142~a~22–31.}.  The \emph{second} kind of definition which is based
on what is not prior is that in which there is a complete circle
through the unconscious use in the definition itself of the notion to
be defined though not of the name\footnote{\ibid~142~a~34–b~6.}.
Trendelenburg illustrates this by two current definitions, (1)~that of
magnitude as that which can be increased or diminished, which is bad
because the positive and negative comparatives ``more'' and ``less''
presuppose the notion of the positive ``great,'' (2)~the famous
Euclidean definition of a straight line as that which ``lies evenly
with the points on itself'' (\greek{ἐξ ἴσου τοῖς ἐφ’ ἑαυτῆς σημείοις
  κεῖται}), where ``lies evenly'' can only be understood with the aid
of the very notion of a straight line which is to be
defined\footnote{Trendelenburg, \emph{Erläuterungen}, p.~115.}. The
\emph{third} kind of vicious definition from that which is not prior
is the definition of one of two coordinate species by means of its
coordinate (\greek{ἀντιδιῃρημένον}), e.g.\ a definition of ``odd'' as
that which exceeds the even by a unit (the second alternative in
Eucl.\ \book{7}{Def.~7}); for ``odd'' and ``even'' are coordinates,
being \emph{differentiae} of number\footnote{\emph{Topics} \r6.~4,
  142~b 7–10.}.  This third kind is similar to the first. Thus, says
Trendelenburg, it would be wrong to define a \emph{square} as ``a
\emph{rectangle} with equal sides.''

\textbf{Aristotle's third requirement.}

A third general observation of Aristotle which is specially relevant
to geometrical definitions is that ``to know \emph{what} a thing is
(\greek{τί ἐστιν}) is the same as knowing \emph{why} it is (\greek{διὰ
  τί ἐστιν})\footnote{\emph{Anal.\ post.}\ \r2.~2, 90~a~31.}.''
``\emph{What} is an eclipse?  A deprivation of light from the moon
through the interposition of the earth.  \emph{Why} does an eclipse
take place?  Or \emph{why} is the moon eclipsed?  Because the light
fails through the earth obstructing it.  \emph{What} is harmony?  A
ratio of numbers in high or low pitch.  \emph{Why} does the
high-pitched harmonise with the low-pitched?  Because the high and the
low have a numerical ratio to one
another\footnote{\emph{Anal.\ post.}\ \r2.~2, 90~a~15–21.}.''  ``We
seek the \emph{cause} (\greek{τὸ διότι}) when we are already in
possession of the \emph{fact} (\greek{τὸ ὅτι}).  Sometimes they both
become evident at the same time, but at all events the cause cannot
possibly be known [as a cause] before the fact is
known\footnote{\ibid~\r2.~8, 93~a~17.}.''  ``It is impossible to know
what a thing is if we do not know \emph{that} it
is\footnote{\ibid~93~a~10.}.''  Trendelenburg paraphrases: ``The
definition of the notion does not fulfil its purpose until it is made
\emph{genetic}.  It is the producing cause which first reveals the
essence of the thing….  The nominal definitions of geometry have only
a provisional significance and are superseded as soon as they are made
genetic by means of construction.'' e.g.\ the genetic definition of a
parallelogram is evolved from Eucl.\ \prop{1}{31} (giving the
construction for parallels) and \prop{1}{33} about the lines joining
corresponding ends of two straight lines parallel and equal in length.
Where existence is proved by construction, the cause and the fact
appear \emph{together}\footnote{Trendelenburg, \emph{Erläuterungen},
  p.~110.}.

Again, ``it is not enough that the defining statement should set forth
the fact, as most definitions do; it should also contain and present
the cause; whereas in practice what is stated in the definition is
usually no more than a conclusion (\greek{συμπέρασμα}).  For example,
what is quadrature?  The construction of an equilateral right-angled
figure equal to an oblong.  But such a definition expresses merely the
conclusion.  Whereas, if you say that quadrature is the discovery of a
mean proportional, then you state the reason\footnote{\emph{De anima}
  \r2.~2, 413~a~13–20.}.'' This is better understood if we compare the
statement elsewhere that ``the cause is the middle term, and this is
what is sought in all cases\footnote{\emph{Anal.\ post.}\ \r2.~2,
  90~a~6.},'' and the illustration of this by the case of the
proposition that the angle in a semi-circle is a right angle.  Here
the middle term which it is sought to establish by means of the figure
is that the angle in the semi-circle is equal to \emph{the half of two
  right angles}.  We have then the syllogism: Whatever is half of two
right angles is a right angle; the angle in a semi-circle is the half
of two right angles; therefore (\emph{conclusion}) the angle in a
semi-circle is a right angle\footnote{\ibid~\r2.~11, 94~a~28.}.  As
with the demonstration, so it should be with the definition, A
definition which is to show the \emph{genesis} of the thing defined
should contain the middle term or cause; otherwise it is a mere
statement of a conclusion. Consider, for instance, the definition of
``quadrature'' as ``making a square equal in area to a rectangle with
unequal sides,'' This gives no hint as to whether a solution of the
problem is possible or how it is solved: but, if you add that to find
the mean proportional between two given straight lines gives another
straight line such that the square on it is equal to the rectangle
contained by the first two straight lines, you supply the necessary
middle term or cause\footnote{Other passages in Aristotle may be
  quoted to the like effect: e.g.\ \emph{Anal.\ post.}\ \r1.~2, 71~b~9
  ``We consider that we know a particular thing in the absolute sense,
  as distinct from the sophistical and incidental sense, when we
  consider that we know the cause on account of which the thing is, in
  the sense of knowing that it is the cause of that thing and that it
  cannot be otherwise,'' \ibid~\r1.~13, 79~a~2 ``For here to know the
  \emph{fact} is the function of those who are concerned with sensible
  things, to know the \emph{cause} is the function of the
  mathematician; it is he who possesses the proofs of the causes, and
  often he does not know the fact.'' In view of such passages it is
  difficult to see how Proclus came to write (p.~202, 11) that
  Aristotle was the originator (\greek{Ἀριστοτέλους κατάρξαντος}) of
  the idea of Amphinomus and others that geometry does not investigate
  the cause and the \emph{why} (\greek{τὸ διὰ τί}. To this Geminus
  replied that the investigation of the cause does, on the contrary,
  appear in geometry. ``For how can it be maintained that it is not
  the business of the geometer to inquire for what reason, on the one
  hand, an infinite number of equilateral polygons are inscribed in a
  circle, but, on the other hand, it is not possible to inscribe in a
  sphere an infinite number of polyhedral figures, equilateral,
  equiangular, and made up of similar plane figures ?  Whose business
  is it to ask this question and find the answer to it if it is not
  that of the geometer? Now when geometers reason \emph{per
    impossibile} they are content to discover the property, but when
  they argue by direct proof, if such proof be only partial
  (\greek{ἐπὶ μέρους}), this does not suffice for showing the cause;
  if however it is general and applies to all like cases, the why
  (\greek{τὸ διὰ τί}) is at once and concurrently made evident.''}.

\textbf{Technical term a not defined by Euclid.}

It will be observed that what is here defined, ``quadrature'' or
``squaring'' (\greek{τετραγωνισμός}), is not a geometrical figure, or
an attribute of such a figure or a part of a figure, but a technical
term used to describe a certain problem.  Euclid does not define such
things; but the fact that Aristotle alludes to this particular
definition as well as to definitions of \emph{deflection}
(\greek{κεκλάσθαι}) and of verging (\greek{νεύειν}) seems to show that
earlier text-books included among definitions explanations of a number
of technical terms, and that Euclid deliberately omitted these
explanations from his \emph{Elements} as surplusage.  Later the
tendency was again in the opposite direction, as we see from the much
expanded Definitions of Heron, which, for example, actually include a
definition of a \emph{deflected line} (\greek{κεκλασμένη
  γραμμή})\footnote{Heron, Def.~12
  (vol.~\r4.\ Heib.\ pp.~22-24).}. Euclid uses the passive of
\greek{κλᾶν} occasionally\footnote{e.g.\ in \r2.~20 and in \emph{Data}
  89,}, but evidently considered it unnecessary to explain such terms,
which had come to bear a recognised meaning.

The mention too by Aristotle of a definition of \emph{verging}
(\greek{νεύειν}) suggests that the problems indicated by this term
were not excluded from elementary text-books before Euclid. The type
of problem (\greek{νεῦσις}) was that of placing a straight line across
two lines, e.g.\ two straight lines, or a straight line and a circle,
so that it shall \emph{verge} to a given point (i.e.\ pass through it
if produced) and at the same time the intercept on it made by the two
given lines shall be of given length.  In general, the use of conics
is required for the theoretical solution of these problems, or a
mechanical contrivance for their practical solution\footnote{Cf.\ the
  chapter on \greek{νεύσεις} in \emph{The Works of Archimedes},
  pp.~c–cxxii.}.  Zeuthen, following Oppermann, gives reasons for
supposing, not only that mechanical constructions were
\emph{practically} used by the older Greek geometers for solving these
problems, but that they were \emph{theoretically} recognised as a
permissible means of solution when the solution could not be effected
by means of the straight line and circle, and that it was only in
later times that it was considered necessary to use conics in
\emph{every} case where that was possible\footnote{Zeuthen \emph{Die
    Lehre von den Kegelschnitten im Altertum}, ch.~12, p.~262.}.
Heiberg\footnote{Heiberg, \emph{Mathematisches zu Aristoteles},
  p.~16.} suggests that the allusion of Aristotle to \greek{νεύσεις}
perhaps confirms this supposition, as Aristotle nowhere shows the
slightest acquaintance with conics.  I doubt whether this is a safe
inference, since the problems of this type included in the elementary
text-books might easily have been limited to those which could be
solved by ``plane'' methods (i.e.\ by means of the straight line and
circle).  We know, e.g., from Pappus that Apollonius wrote two Books
on \emph{plane} \greek{νεύσεις}\footnote{Pappus \r7.~pp.~670—2.}.  But
one thing is certain, namely that Euclid deliberately excluded this
class of problem, doubtless as not being essential in a book of
Elements.

\textbf{Definitions not afterwards used.}

Lastly, Euclid has definitions of some terms which he never afterwards
uses, e.g.\ oblong (\greek{ἑτερόμηκες}), rhombus, rhomboid. The
``oblong'' occurs in Aristotle; and it is certain that all these
definitions are survivals from earlier books of Elements.

\part{Book I}

\section*{Definitions}

\begin{enumerate}

\item\label{def:I_1} A \textbf{point} is that which has no part.

\item\label{def:I_2} A \textbf{line} is breadthless length,

\item\label{def:I_3} The extremities of a line are points.

\item\label{def:I_4} A \textbf{straight line} is a line which lies
  evenly with the points on itself.

\item\label{def:I_5} A \textbf{surface} is that which has length and
  breadth only.

\item\label{def:I_6} The extremities of a surface are lines.

\item\label{def:I_7} A \textbf{plane surface} is a surface which lies
  evenly with the straight lines on itself.

\item\label{def:I_8} A \textbf{plane angle} is the inclination to one
  another of two lines in a plane which meet one another and do not
  lie in a straight line.

\item\label{def:I_9} And when the lines containing the angle are
  straight, the angle is called \textbf{rectilineal}.

\item\label{def:I_10} When a straight line set up on a straight line
  makes the adjacent angles equal to one another, each of the equal
  angles is \textbf{right}, and the straight line standing on the
  other is called a \textbf{perpendicular} to that on which it stands.

\item\label{def:I_11} An \textbf{obtuse angle} is an angle greater
  than a right angle.

\item\label{def:I_12} An \textbf{acute angle} is an angle less than a
  right angle.

\item\label{def:I_13} A \textbf{boundary} is that which is an
  extremity of anything.

\item\label{def:I_14} A \textbf{figure} is that which is contained by
  any boundary or boundaries.

\item\label{def:I_15} A \textbf{circle} is a plane figure contained by
  one line such that all the straight lines falling upon it from one
  point among those lying within the figure are equal to one another;

\item\label{def:I_16} And the point is called the \textbf{centre} of
  the circle.

\item\label{def:I_17} A \textbf{diameter} of the circle is any
  straight line drawn through the centre and terminated in both
  directions by the circumference of the circle, and such a straight
  line also bisects the circle.

\item\label{def:I_18} A \textbf{semicircle} is the figure contained by
  the diameter and the circumference cut off by it.  And the centre of
  the semicircle is the same as that of the circle.

\item\label{def:I_19} \textbf{Rectilineal figures} are those which are
  contained by straight lines, trilateral figures being those
  contained by three, quadrilateral those contained by four, and
  multilateral those contained by more than four straight lines.

\item\label{def:I_20} Of trilateral figures, an \textbf{equilateral
  triangle} is that which has its three sides equal, an
  \textbf{isosceles triangle} that which has two of its sides alone
  equal, and a \textbf{scalene triangle} that which has its three
  sides unequal.

\item\label{def:I_21} Further, of trilateral figures, a
  \textbf{right-angled triangle} is that which has a right angle, an
  \textbf{obtuse-angled triangle} that which has an obtuse angle, and
  an \textbf{acute-angled triangle} that which has its three angles
  acute.

\item\label{def:I_22} Of quadrilateral figures, a \textbf{square} is
  that which is both equilateral and right-angled; an \textbf{oblong}
  that which is right-angled but not equilateral; a \textbf{rhombus}
  that which is equilateral but not right-angled; and a
  \textbf{rhomboid} that which has its opposite sides and angles equal
  to one another but is neither equilateral nor right-angled. And let
  quadrilaterals other than these be called \textbf{trapezia}.

\item\label{def:I_23} \textbf{Parallel} straight lines are straight
  lines which, being in the same plane and being produced indefinitely
  in both directions, do not meet one another in either direction.

\end{enumerate}

\section*{Postulates}

Let the following be postulated:

\begin{enumerate}

\item\label{post:1} To draw a straight line from any point to any point.

\item\label{post:2} To produce a finite straight line continuously in a straight
  line.

\item\label{post:3} To describe a circle with any centre and distance.

\item\label{post:4} That all right angles are equal to one another.

\item\label{post:5} That, if a straight line falling on two straight
  lines make the interior angles on the same side less than two right
  angles, the two straight lines, if produced indefinitely, meet on
  that side on which are the angles less than the two right angles.

\end{enumerate}

\section*{Common Notions}

\begin{enumerate}

\item\label{cn:1} Things which are equal to the same thing are also
  equal to one another.

\item\label{cn:2} If equals be added to equals, the wholes are equal.

\item\label{cn:3} If equals be subtracted from equals, the remainders
  are equal.

\item\label{cn:4}\relax [7] Things which coincide with one another are
  equal to one another.

\item\label{cn:5}\relax [8] The whole is greater than the part

\end{enumerate}

\section*{Definition 1}

\greek{Σημεῖόν ἐστιν, οὖ μέρος οὐέν.}

\emph{A \emph{point} is that which has no part.}

An exactly parallel use of \greek{μέρος} (\greek{ἐστί}) in the
singular is found in Aristotle, \emph{Metaph.}~1035 b 32 \greek{μέρος
  μὲν οὔν ἐστὶ καὶ τοῦ εἴδους},
literally ``There is a \emph{part} even of the form'' ; Bonitz
translates as if the plural were used, ``Theile giebt es,'' and the
meaning is simply ``even the form is \emph{divisible} (into parts).''
Accordingly it would be quite justifiable to translate in this case
``A point is that which is \emph{indivisible into parts}.''

Martianus Capella (5th c.~\ad)\ alone or almost alone translated
differently, ``Punctum est cuius pars \emph{nihil} est,'' ``a point is
that a part of which is \emph{nothing}.''  Notwithstanding that Max
Simon (\emph{Euclid und die sechs planimetrischen Bücker}, 1901) has
adopted this translation (on grounds which I shall presently mention),
I cannot think that it gives any sense. If a part of a point is
\emph{nothing}, Euclid might as well have said that a point is
\emph{itself} ``nothing,'' which of course he does not do.

\subsubsection*{Pre-Euclidean definitions}

It would appear that this was not the definition given in earlier
text-books; for Aristotle (\emph{Topics} \prop{6}{4}, 141~b~20), in
speaking of ``\emph{the} definitions'' of point, line, and surface,
says that they \emph{all} define the prior by means of the posterior,
a point as an extremity of a line, a line of a surface, and a surface
of a solid.

The first definition of a point of which we hear is that given by the
Pythagoreans (cf.\ Proclus, p.~95, 21), who defined it as a ``monad
having position'' or ``with position added'' (\greek{μονὰς προσλαβοῦσα
  θέσιν}).  It is frequently used by Aristotle, either in this exact
form (cf.\ \emph{De anima} \prop{1}{4}, 409~a~6) or its equivalent:
e.g.\ in \emph{Metaph.}\ 1016~b~24 he says that that which is
indivisible every way in respect of magnitude and \emph{quâ} magnitude
but has not position is a \emph{monad}, while that which is similarly
indivisible and has position is a \emph{point}.

Plato appears to have objected to this definition. Aristotle says
(\emph{Metaph.}\ 992~a~20) that he objected ``to this genus [that of
  points] as being a geometrical fiction (\greek{γεωμετρικὸν δόγμα}),
and called a point the beginning of a line (\greek{ἀρχὴ γραμμῆς}),
while again he frequently spoke of `indivisible lines.'\,'' To which
Aristotle replies that even ``indivisible lines'' must have
extremities, so that the same argument which proves the existence of
lines can be used to prove that points exist It would appear therefore
that, when Aristotle objects to the definition of a point as the
extremity of a line (\greek{πέρας γραμμῆς}) as unscientific
(\emph{Topics} \prop{6}{4}, 141~b~21), he is aiming at Plato.  Heiberg
conjectures (\emph{Mathematisthes zu Aristoteles}, p.~8) that it was
due to Plato's influence that the word for ``point'' generally used by
Aristotle (\greek{στιγμή}) was replaced by \greek{σημεῖον} (the
regular term used by Euclid, Archimedes and later writers), the latter
term (=\emph{nota}, a conventional mark) probably being considered
more suitable than \greek{στιγμή} (a \emph{puncture}) which might
appear to claim greater \emph{reality} for a point.

Aristotle's conception of a point as that which is indivisible and has
position is further illustrated by such observations as that a point
is not a \emph{body} (\emph{De caelo} \r2.~13, 296~a~17) and has no
\emph{weight} (\ibid~\r3.~1, 299~a~30); again, we can make no
distinction between a point and the \emph{place} (\greek{τόπος}) where
it is (\emph{Physics} \r4.~1, 209~a~11).  He finds the usual
difficulty in accounting for the transition from the indivisible, or
infinitely small, to the finite or divisible magnitude. A point being
\emph{indivisible}, no accumulation of points, however far it may be
carried, can give us anything divisible, whereas of course a line is a
divisible magnitude.  Hence he holds that points cannot make up
anything continuous like a line, point cannot be continuous with point
(\greek{οὐ γάρ ἐστιν ἐχόμενον σημεῖον σημείου ἣ στιγμὴ στιγμῆς},
\emph{De gen.\ et corr.}\ \prop{1}{2}, 317~a~10), and a line is not
\emph{made up} of points (\greek{οὐ σύγκειται ἐκ στιγμο›ν},
\emph{Physics} \r4.~8, 215~b~19).  A point, he says, is like the
\emph{now} in time: \emph{now} is indivisible and is not a \emph{part}
of time, it is only the beginning or end, or a division, of time, and
similarly a point may be an extremity, beginning or division of a
line, but is not part of it or of magnitude (cf.\ \emph{De caelo}
\r3.~1, 300~a~14, \emph{Physics} \r4.~11, 220~a~1–21, \r6.~1, 231~b~6
sqq.). It is only by \emph{motion} that a point can generate a line
(\emph{De anima} \prop{1}{4}, 409~a~4) and thus be the origin of
magnitude.

\subsubsection*{Other ancient definitions}

According to an-Nairīzī (ed.\ Curtze, p.~3) one ``Herundes'' (not so
far identified) defined a point as ``the indivisible beginning of all
magnitudes,'' and Posidonius as ``an extremity which has no dimension,
or an extremity of a line.''

\subsubsection*{Criticisms by commentators}

Euclid's definition itself is of course practically the same as that
which Aristotle's frequent allusions show to have been then current,
except that it omits to say that the point must have position. Is it
then sufficient, seeing that there are other things which are without
parts or indivisible, e.g.\ the \emph{now} in time, and the
\emph{unit} in number?  Proclus answers (p.~93, 18) that the point is
the only thing \emph{in the subject-matter of geometry} that is
indivisible.  Relatively therefore to the particular science the
definition is sufficient.  Secondly, the definition has been over and
over again criticised because it is purely negative.  Proclus' answer
to this is (p.~94, 10) that negative descriptions are appropriate to
first principles, and he quotes Parmenides as having described his
first and last cause by means of negations merely.  Aristotle too
admits that it may sometimes be necessary for one framing a definition
to use negations, e.g.\ in defining privative terms such as ``blind'';
and he seems to accept as proper the negative element in the
definition of a point, since he says (\emph{De anima} \r3.~6,
430~b~20) that ``the point and every division [e.g.\ in a length or in
  a period of time], and that which is indivisible in this sense, ts
exhibited as privation (\greek{δηλοῦται ὡς στέρησις}).''

Simplicius (quoted by an-Nairīzī) says that ``a point is the beginning
of magnitudes and that from which they grow; it is also the only thing
which, having position, is not divisible.'' He, like Aristotle, adds
that it is by its \emph{motion} that a point can generate a magnitude:
the particular magnitude can only be ``of one dimension,'' viz.\ a
line, since the point does not ``spread itself'' (dimittat).
Simplicius further observes that Euclid defined a point negatively
because it was arrived at by detaching surface from body, line from
surface, and finally point from line. ``Since then body has three
dimensions it follows that a point [arrived at after successively
  eliminating all three dimensions] has \emph{none of the dimensions},
and has no part.''  This of course reappears in modern treatises
(cf.\ Rausenberger, \emph{Elementar-geometrie des Punktes, der Geraden
  und der Ebene}, 1887, p.~7).

An-Nairīzī adds an interesting observation. ``If any one seeks to know
the essence of a point, a thing more simple than a line, let him, in
the sensible world, think of the centre of the universe and the
\emph{poles}.''  But there is nothing new under the sun: the same idea
is mentioned, in an Aristotelian treatise, in controverting those who
imagine that the poles have some influence in the motion of the
sphere, ``when the poles have no magnitude but are extremities and
points'' (\emph{De motu animalium} 3, 699~a~21).

\subsubsection*{Modern views}

In the new geometry represented by the excellent treatises which start
from new systems of postulates or axioms, the result of the profound
study of the fundamental principles of geometry during recent years (I
need only mention the names of Pasch, Veronese, Enriques and Hilbert),
points come before lines, but the vain effort to define them \emph{a
  priori} is not made; instead of this, the nearest material things in
nature are mentioned as illustrations, with the remark that it is from
them that we can get the abstract idea.  Cf.~the full statement as
regards the notion of a point in Weber and Wellstein,
\emph{Encyclopädie der eiementaren Mathematik}, \r2., 1905,
p.~9. ``This notion is evolved from the notion of tbe real or supposed
\emph{material} point by the process of limits, i.e.\ by an act of the
mind which sets a term to a series of presentations in itself
unlimited.  Suppose a grain of sand or a mote in a sunbeam, which
continually becomes smaller and smaller.  In this way vanishes more
and more the possibility of determining still smaller atoms in the
grain of sand, and there is evolved, so we say, with growing
certainty, the presentation of the point as a definite position in
space which is one and is incapable of further division.  But this
view is untenable; we have, it is true, some idea how the grain of
sand gets smaller and smaller, but only so long as it remains just
visible; after that we are completely in the dark, and we cannot see
or imagine the further diminution.  That this procedure comes to an
end is unthinkable; that nevertheless there exists a term beyond which
it cannot go, we must believe or postulate without ever reaching it…It
is a pure act of \emph{will}, not of the understanding.'' Max Simon
observes similarly (\emph{Euclid}, p.~25) ``The notion 'point' belongs
to the limit-notions (Grenzbegriffe), the necessary conclusions of
continued, and in themselves unlimited, senes of presentations.'' He
adds, ``The point is the limit of localisation; if this is more and
more energetically continued, it leads to the limit-notion 'point,'
better 'position,' which at the same time involves a change of notion.
Content of space vanishes, relative \emph{position} remains. 'Point'
then, according to our interpretation of Euclid, is the extremest
limit of that which we can still think of (not observe) as a
\emph{spatial} presentation, and if we go further than that, not only
does extension cease but even relative \emph{place}, and in this sense
the 'part' is \emph{nothing}.''  I confess I think that even the
meaning which Simon intends to convey is better expressed by ``it has
\emph{no} part'' than by ``the part is nothing,'' since to take a
``part'' of a thing in Euclid's sense of the result of a simple
division, corresponding to an arithmetical fraction, would not be to
change the \emph{notion} from that of the thing divided to an entirely
different one.

\section*{Definition 2}

\greek{Γραμμὴ δὲ μῆκος ἀπλατές.}

\emph{A \emph{line} is breadthless length.}

This definition may safely be attributed to the Platonic School, if
not to Plato himself.  Aristotle (\emph{Topics} \r6.~6, 143~b~11)
speaks of it as open to objection because it ``divides the genus by
negation,'' length being necessarily either breadthless or possessed
of breadth; it would seem however that the objection was only taken in
order to score a point against the Platonists, since he says
(\ibid~143~b~29) that the argument is ``of service \emph{only} against
those who assert that the genus [sc.~length] is one numerically, that
is, those who assume \emph{ideas},'' e.g.\ the idea of length
(\greek{αὐτὸ μῆκος}) which they regard as a genus: for if the genus,
being one and self-existent, could be divided into two species, one of
which asserts what the other denies, it would be self-contradictory
(Waitz),

Proclus (pp.~96, 21–97, 3) observes that, whereas the definition of a
point is merely negative, the line introduces the first ``dimension,''
and so its definition is to this extent positive, while it has also a
negative element which denies to it the other ``dimensions''
(\greek{διαστάσεις}).  The negation of both breadth and depth is
involved in the single expression ``breadthless'' (\greek{ἀπλατες}),
since everything that is without breadth is also destitute of depth,
though the converse is of course not true.

\subsubsection*{Alternative definitions}

The alternative definition alluded to by Proclus, \greek{μέγεθος ἐφ’
  ἓν διαστατόν} ``magnitude in one dimension'' or, better perhaps,
``magnitude extended one way'' (since \greek{διάστασις} as used with
reference to line, surface and solid scarcely corresponds to our use
of ``dimension'' when we speak of ``one,'' ``two,'' or ``three
dimensions''), is attributed by an-Nairīzī to ``Heromides,'' who must
presumably be the same as ``Herundes,'' to whom he attributes a
certain definition of a point.  It appears however in substance in
Aristotle, though Aristotle does not use the adjective
\greek{διαστατόν}, nor does he apparently use \greek{διάστασις} except
of \emph{body} as having \emph{three} ``dimensions'' or ``having
dimension (or extension) \emph{all} ways (\greek{πάντῃ}),'' the
``dimensions'' being in his view (1)~up and down, (2)~before and
behind, and (3)~right and left, and ``up'' being the principle or
beginning of \emph{length}, ``right'' of breadth, and ``before'' of
depth (\emph{De caelo} \r2.~2, 284~b~24). A line is, according to
Aristotle, a magnitude ``\emph{divisible} in one way only''
(\greek{μονοχῃ‹ διαιρετόν}), in contrast to a magnitude divisible in
two ways (\greek{διχῃ‹ διαιρετόν}), or a surface, and a magnitude
divisible ``in all or in three ways'' (\greek{πάντῃ καὶ τριχῆ
  διαιρετόν}), or a body (\emph{Metaph.}\ 1016~b~25–27); or it is a
magnitude ``\emph{continuous} one way (or in one direction),'' as
compared with magnitudes continuous \emph{two} ways or \emph{three}
ways, which curiously enough he describes as ``breadth ``and ``depth''
respectively (\greek{μέγεθος δὲ τὸ μὲν ἐφ’ ἓν συνεχὲς μῆκος, τὸ δ’ ἐπὶ
  δύο πλάτος, τὸ δ’ ἐπὶ τρία βάθοσ}, \emph{Metaph.}\ 1020~a~11),
though he immediately adds that ``length'' means a line, ``breadth'' a
surface, and ``depth'' a body.

Proclus gives another alternative definition as ``\emph{flux of a
  point}'' (\greek{ῥύσις συμείου}), i.e.\ the path of a point when
  moved.  This idea is also alluded to in Aristode (\emph{De anima}
  \prop{1}{4}, 409~a~4 above quoted): ``they say that a line by its
  motion produces a surface, and a point by its motion a line.''
  ``This definition,'' says Proclus (p.~97, 8–13), ``is a perfect one
  as showing the essence of the line: he who called it the flux of a
  point seems to define it from its genetic cause, and it is not every
  line that he sets before us, but only the immaterial line, for it is
  this that is produced by the point, which, though itself
  indivisible, is the cause of the existence of things divisible.''

Proclus (p.~100, 5–19) adds the useful remark, which, he says, was
current in the school of Apollonius, that we have the notion of a line
when we ask for the length of a road or a wall measured merely as
length; for in that case we mean something irrespective of breadth,
viz.\ distance in one ``dimension.''  Further we can obtain sensible
perception of a line if we look at the division between the light and
the dark when a shadow is thrown on the earth or the moon; for clearly
the division is without breadth, but has length.

\subsubsection*{Species of ``lines.''}

After defining the ``line'' Euclid only mentions \emph{one} species of
line, the straight line, although of course another species appears in
the definition of a circle later. He doubtless omitted all
\emph{classification} of lines as unnecessary for his purpose,
whereas, for example, Heron follows up his definition of a line by a
division of lines into (1)~those which are ``straight'' and (2)~those
which are not, and a further division of the latter into
(\emph{a})~``circular circumferences,'' (\emph{b})~``spiral-shaped''
(\greek{ἑλικοειδεῖς}) lines and (\emph{c}) ``curved''
(\greek{καμπύλαι}) lines generally, and then explains the four terms.
Aristotle tells us (\emph{Metaph.}\ 986~a~25) that the Pythagoreans
distinguished straight (\greek{εὐθύ}) and curved (\greek{καμπύλον}),
and this distinction appears in Plato (cf.\ \emph{Republic}
\r10.~602~c) and in Aristotle (cf.\ ``to a line belong the attributes
straight or curved,'' \emph{Anal.\ post.}\ \prop{1}{4}, 73~b~19; ``as
in mathematics it is useful to know what is meant by the terms
straight and curved,'' \emph{De anima} \prop{1}{1}, 402~b~19).  But
from the class of ``curved'' lines Plato and Aristotle separate off
the \greek{περιφερής} or ``circular'' as a distinct species often
similarly contrasted with straight.  Aristotle seems to recognise
broken lines forming an angle as one line: thus ``a line, if it be
bent (\greek{κεκαμμένη}), but yet continuous, is called one''
  (\emph{Metaph.}\ 1016~a~2); ``the straight line is more one than the
  bent line'' (\ibid~1016~a~12).  Cf.\ Heron, Def.~12, ``A broken line
  (\greek{κεκλασμένη γραμμή}) so-called is a line which, when
  produced, does not meet \emph{itself}.''

When Proclus says that both Plato and Aristotle divided lines into
those which are ``straight,'' ``circular'' (\greek{περιφερής}) or ``a
mixture of the two,'' adding, as regards Plato, that he included in
the last of these classes ``those which are called helicoidal among
plane (curves) and (curves) formed about solids, and such species of
curved lines as arise from sections of solids ``(p.~104, 1–5), he
appears to be not quite exact. The reference as regards Plato seems to
be to \emph{Parmenides} 145~\textsc{b}: ``At that rate it would seem
that the one must have shape, either straight or round
(\greek{στρογγύλου}) or some combination of the two''; but this
scarcely amounts to a formal classification of lines.  As regards
Aristotle, Proclus seems to have in mind the passage (\emph{De caelo}
\r1.~2, 268~b~17) where it is stated that ``all \emph{motion} in
space, which we call translation (\greek{φορά}), is (in) a straight
line, a circle, or a combination of the two; for the first two are the
only simple (\emph{motions}).''

For completeness it is desirable to add the substance of Proclus'
account of the classification of lines, for which he quotes Geminus as
his authority.

\subsubsection*{Geminus' first classification of lines}

This begins (p.~111, 1–9) with a division of lines into
\emph{composite} (\greek{σύνθετος}) and incomposite
(\greek{ἀσύνθετος}).  The only illustration given of the
\emph{composite} class is the ``broken line which forms an angle''
(\greek{ἡ κεκλασμένη καὶ γωνίαν ποιοῦσα}); the subdivision of the
\emph{incomposite} class then follows (in the text as it stands the
word ``composite'' is clearly an error for ``incomposite'').  The
subdivisions of the incomposite class are repeated in a later passage
(pp.~176, 27–177, 23) with some additional details.  The following
diagram reproduces the effect of both versions as far as possible (all
the illustrations mentioned by Proclus being shown in brackets).

lines

composite
(broken line forming an angle)

incomposite

forming a figure
\greek{σχηματοποιοῦσαι}
or determinate
\greek{ὡρισμέναι}
(circle, ellipse, cissoid)

not forming a figure
or
indeterminate
\greek{ἀόριστοι}
and
extending without limit
\greek{ἐπ’ ἄπειρον ἑκβαλλόμεναι}
(straight line, parabola, hyperbola, conchoid)

The additional details in the second version, which cannot easily be
shown in the diagram, are as follows:

(1)~Of the lines which extend without limit, some do not \emph{form a
  figure} at all (viz.\ the straight line, the parabola and the
hyperbola); but some first ``come together and form a figure''
(i.e.\ have a loop), ``and, for the rest, extend without limit
``(p.~177, 8).

\infig{conchoid}

As the only other curve, besides the parabola and the hyperbola, which
has been mentioned as proceeding to infinity is the \emph{conchoid}
(of Nicomedes), we can hardly avoid the conclusion of
Tannery\footnote{\emph{Notes pour l'histoire des lignes et surfaces
    courbes dans l'antiquité} in \emph{Bulletin des sciences
    mathém.\ et astronom.}\ 2 sér.\ \r8.\ (1884), pp.~108–9
  (\emph{Mémoires scientifiques}, \r2.~p.~23).} that the curve which
has a loop and then proceeds to infinity is a variety of the
\emph{conchoid} itself.  As is well known, the ordinary conchoid
(which was used both for doubling the cube and for trisecting the
angle) is obtained in this way.  Suppose any number of rays passing
through a fixed point (the \emph{pole}) and intersecting a fixed
straight line; and suppose that points are taken on the rays, beyond
the fixed straight line, such that the portions of the rays
intercepted between the fixed straight line and the point are equal to
a constant \emph{distance} (\greek{διάστημα}), the locus of the points
is a conchoid which has the fixed straight line for asymptote. If the
``distance''~$a$ is measured from the intersection of the ray with the
given straight line, not in the direction away from the pole, but
towards the pole, we obtain three other curves according as $a$ is
less than, equal to, or greater than~$b$, the distance of the pole
from the fixed straight line, which is an asymptote in each case. The
case in which $a > b$ gives a curve which forms a loop and then
proceeds to infinity in the way Proclus describes.  Now we know both
from Eutocius (\emph{Comm.\ on Archimedes}, ed.\ Heiberg, \r3.\ p.~98)
and Proclus (p.~272, 3–7) that Nicomedes wrote on conchoid\emph{s} (in
the plural), and Pappus (\r4.\ p.~244, 18) says that besides the
``first'' (used as above stated) there were ``the second, the third
and the fourth which are useful for other theorems.''

(2)~Proclus next observes (p.~177, 9) that, of the lines which extend
without limit, some are ``asymptotic'' (\greek{ἀσύμπτωτοι}), namely
``those which never meet, however they are produced,'' and some are
``\emph{symptotic}'' namely ``those which will meet sometime''; and,
of the ``asymptotic'' class, some are in one plane, and others
not. Lastly, of the ``asymptotic'' lines in one plane, some preserve
always the same distance from one another, while others continually
``lessen the distance, like the hyperbola with reference to the
straight line, and the conchoid with reference to the straight line.''

\subsubsection*{Geminus' second classification}

This (from Proclus, pp.~111, 9–20 and 112, 16–18) can be shown in a
diagram thus:

Incomposite lines
\greek{ἀσύνθετοι γραμμαί}

simple, \greek{ἁπλῆ}

mixed, \greek{μικτή}

making a figure
\greek{σχῆμα ποιῦσα}
(e.g.\ circle)

indeterminate
\greek{ἀόριστος}
(straight line)

lines in planes

lines on solids
\greek{αἱ ἐν τοῖς στερεοῖς}

line meeting itself
\greek{ἡ ἐν αὐτῇ συμπίπτουσα}
(e.g.\ cissoid)

extending without limit
\greek{ἡ ἐπ’ ἄπειρον ἐκβαλλομένη}

lines formed by \emph{sections}
\greek{αἱ κατὰ τὰς τομάς}
(e.g.\ conic sections, spirit curves)

lines \emph{round} solids
\greek{αἱ περὶ τὰ στερεά}
(e.g.\ \emph{helix} about a sphere or about a cone)

\emph{homoeomeric}
(cylindrical helix)

\emph{not homoeomeric}
(all othera)

\subsubsection*{Notes on classes of ``lines'' and on particular curves}

We will now add the most interesting notes found in Proclus with
reference to the above classifications or the particular curves
mentioned.

\textbf{1.~Homoeomeric lines.}

By this term (\greek{ὁμοιομερεῖς}) are meant lines which are alike in
all parts, so that in any one such curve any part can be made to
coincide with any other part.  Proclus observes that these lines are
only three in number, two being ``simple'' and in a plane (the
straight line and the circle), and the third ``mixed,'' (subsisting)
``about a solid,'' namely the cylindrical helix.  The latter curve was
also called the \emph{cochlias} or \emph{cochlion}, and its
\emph{homoeomeric} property was proved by Apollonius in his work
\greek{περὶ τοῦ κοχλίου} (Proclus, p.~105,~5). The fact that there are
only three homoeomeric lines was proved by Geminus, ``who proved, as a
preliminary proposition, that, if from a point (\greek{ἀπό του
  συμείου}, but on p.~251,~4 \greek{ἀφ’ ἑνὸς συμείου}) two straight
  lines be drawn to a homoeomeric line making equal angles with it,
  the straight lines are equal'' (pp.~112, 1–113, 3, cf.~p.~251,
  2–19).

\textbf{2. Mixed lines.}

It might be supposed, says Proclus (p.~105,~11), that the cylindrical
helix, being \emph{homoeomerie}, like the straight line and the
circle, must like them be \emph{simple}.  He replies that it is not
simple, but \emph{mixed}, because it is generated by \emph{two unlike}
motions.  Two \emph{like} motions, said Geminus, e.g.\ two motions at
the same speed in the directions of two adjoining sides of a square,
produce a \emph{simple} line, namely a straight line (the diagonal);
and again, if a straight line moves with its extremities upon the two
sides of a right angle respectively, this same motion gives a
\emph{simple} curve (a circle) for the locus of the middle point of
the straight line, and a \emph{mixed} curve (an ellipse) for the locus
of any other point on it (p.~106, 3–15).

Geminus also explained that the term ``mixed,'' as applied to curves,
and as applied to surfaces, respectively, is used in different senses.
As applied to curves, ``mixing'' neither means simple ``putting
together'' (\greek{σύνθεσις}) nor ``blending'' (\greek{κρᾶσις}).  Thus
the helix (or spiral) is a ``mixed'' line, but (1)~it is not ``mixed''
in the sense of ``putting together,'' as it would be if, say, part of
it were straight and part circular, and (2)~it is not mixed in the
sense of ``blending,'' because, if it is cut in any way, it does not
present the appearance of any simple lines (of which it might be
supposed to be compounded, as it were). The ``mixing'' in the case of
lines is rather that in which the constituents are destroyed so far as
their own character is concerned, and are replaced, as it were, by a
\emph{chemical} combination (\greek{ἔστιν ἐν αὐτῇ συνεφθαρμένα τὰ ἄκρα
  καὶ συγκεχυμένα}).  On the other hand ``mixed'' surfaces are mixed
in the sense of a sort of ``blending'' (\greek{katá tina krâsin}).
For take a cone generated by a straight line passing through a fixed
point and passing always through the circumference of a circle: if you
cut this by a plane parallel to that of the circle, you obtain a
circular section, and if you cut it by a plane through the vertex, you
obtain a triangle, the ``mixed'' surface of the cone being thus cut
into \emph{simple} lines (pp.~117, 22–118,~23).

\textbf{3. Spiric curves.}

These curves, classed with conics as being sections of solids, were
discovered by Perseus, according to an epigram of Perseus' own quoted
by Proclus (p.~112,~1), which says that Perseus found ``three lines
upon (or, perhaps, in addition to) five sections'' (\greek{τρεῖς
  γραμμὰς ἐπὶ πέντε τομαῖς}).  Proclus throws some light on these in
the following passages:

``Of the spiric sections, one is interlaced, resembling the
horse-fetter (\greek{ἵππου πέδη}); another is widened out in the
middle and contracts on each side (of the middle), a third is
elongated and is narrower in the middle, broadening out on each side
of it'' (p.~112,~4–8).

``This is the case with the \emph{spiric surface}; for it is conceived
as generated by the revolution of a circle remaining at right angles
[to a plane] and turning about a point which is not its centre [in
  other words, generated by the revolution of a circle about a
  straight line in its plane not passing through the centre]. Hence
the \emph{spire} takes three forms, for the centre [of rotation] is
either on the circumference, or within it, or without it. And if the
centre of rotation is on the circumference, we have the
\emph{continuous} spire (\greek{συνεχής}) if within, the
\emph{interlaced} (\greek{ἐμπεπλεγμένη}), and if without, the open
(\greek{διεχής}). And the spiric sections are three according to these
three differences'' (p.~119, 8–177).

``When the \emph{hippopede}, which is one of the spiric curves, forms
an angle with itself, this angle also is contained by mixed lines''
(p.~127, 1–3).

``Perseus showed for spirics what was their property
(\greek{σύμπτωμα})'' (p.~356,~12)

Thus the spiric surface was what we call a \emph{tore}, or (when open)
an \emph{anchor-ring}. Heron (Def.~97) says it was called
alternatively \emph{spire} (\greek{σπεῖρα}) or \emph{ring}
(\greek{κρίκος}); he calls the variety in which ``the circle cuts
itself,'' not ``interlaced,'' but ``cross itself''
(\greek{ἐπαλάττουσα}).

Tannery\footnote{\emph{Notes pour l'histoire des lignes et surfaces
    courbes dans l'antiquité} in \emph{Bulletin des sciences
    mathém.\ et astronom.}\ \r8.\ (1884), pp.~25–27 (\emph{Mémoires
    scientifiques}, \r2.~p.~24–28).} has discussed these passages, as
also did Schiaparelli\footnote{\emph{Die homocentrischen Sphären des
    Eudoxus, des Kallippus und des Aristoteles} (\emph{Abhandlungen
    zur Gesch.\ der Math.}\ \r1.~Heft, 1877, pp.~149–152).}.  It is
clear that Proclus' remark that the difference in the three curves
which he mentions corresponds to the difference between the three
surfaces is a slip, due perhaps to too hurried transcribing from
Geminus; all three arise from plane sections of the \emph{open}
anchor-ring.  If $r$ is the radius of the revolving circle, $a$ the
distance of its centre from the axis of rotation, $d$ the distance of
the plane section (supposed to be parallel to the axis) from the axis,
the three curves described in the first extract correspond to the
following cases:

(1)~$d=a-r$. In this case the curve is the \emph{hippopede}, of which
the lemniscate of Bernoulli is a particular case, namely that in which
$a = 2r$.

The name \emph{hippopede} was doubtless adopted for this one of
Perseus' curves on the ground of its resemblance to the
\emph{hippopede} of Eudoxus, which seems to have been the curve of
intersection of a sphere with a cylinder touching it internally.

(2)~$a + r > d > a$. Here the curve is an oval,

(3)~$a > d > a - r$. The curve is now narrowest in the middle.

Tannery explains the ``three lines upon (in addition to) five
sections'' thus.  He points out that with the \emph{open tore} there
are two other sections corresponding to

(4)~$d = a$: transition from (2) to~(3).

(5) $a -r > d > o$, in which case the section consists of two
symmetrical ovals.

He then shows that the sections of the \emph{closed} or
\emph{continuous tore}, corresponding to $a = r$, give curves
corresponding to (2), (3) and~(4) only.  Instead of (1) and~(5) we
have only a section consisting of two equal circles touching one
another.

On the other hand, the \emph{third} spire (the \emph{interlaced}
variety) gives three \emph{new} forms, which make a group of three in
addition to the first group of \emph{five} sections.

The difficulty which I see in this interpretation is the fact that,
just after ``three lines on five sections'' are mentioned, Proclus
describes three curves which were evidently the most important; but
these three belong to three of the five sections of the open tore, and
are not separate from them,

\textbf{4. The cissoid.}

This curve is assumed to be the same as that by means of which,
according to Eutocius (\emph{Comm.\ on Archimedes}, \r3.\ p.~66 sqq.),
Diocles in his book \greek{περὶ πυρίων} (\emph{On burning}-glasses)
solved the problem of doubling the cube.  It is the locus of points
which he found by the following construction.  Let $AC$, $BD$ be
diameters at right angles in a circle with centre~$O$.

Let $E$, $F$ be points on the quadrants $BC$, $BA$ respectively such
that the arcs $BE$, $Bf$ are equal.

\sidefig{cissoid}

Draw $EG$, $FH$ perpendicular to $CA$.

Join $AE$, and let $P$ be its intersection with $FH$.

The cissoid is the locus of all the points $P$ corresponding to
different positions of~$E$ on the quadrant $BC$ and of~$F$ at an equal
distance from $B$ along the arc~$BA$.

$A$ is the point on the curve corresponding to the position $C$ for
the point~$E$, and $B$ the point on the curve corresponding to the
position of~$E$ in which it coincides with~$B$.

It is easy to see that the curve extends in the direction $AB$
beyond~$B$, and that $CK$ drawn perpendicular to $CA$ is an asymptote.
It may be regarded also as having a branch $AD$ symmetrical with~$AB$,
and, beyond~$D$, approaching $KC$ produced as asymptote.

If $OA$, $OD$ are coordinate axes, the equation of the curve is obviously
\[
    y^2(a + x) = (a - x)^3,
\]
where $a$ is the radius of the circle.

There is a cusp at $A$, and it agrees with this that Proclus should
say (p.~126, 24) that ``cissoidal lines converging to one point like
the leaves of ivy—for this is the origin of their name—form an
angle.'' He makes the slight correction (p.~128, 5) that it is not two
\emph{parts} of a curve, but \emph{one} curve, which in this case
makes an angle.

But what is surprising is that Proclus seems to have no idea of the
curve passing outside the circle and having an asymptote, for he
several times speaks of it as a closed curve (forming a figure and
including an area): cf.\ p.~152,~7, ``the plane (area) cut off by the
cissoidal line has one bounding (line), but it has not in it a centre
such that all (straight lines drawn to the curve) from it are equal.''
It would appear as if Proclus regarded the cissoid as formed by the
\emph{four} symmetrical cissoidal arcs shown in the figure.

Even more peculiar is Proclus' view of the

\textbf{5, ``Single-turn Spiral.''}

This is really the spiral of Archimedes traced by a point starting
from the fixed extremity of a straight line and moving uniformly along
it, while simultaneously the straight line itself moves uniformly in a
plane about its fixed extremity.  In Archimedes the spiral has of
course any number of turns, the straight line making the same number
of complete revolutions.  Yet Proclus, while giving the same account
of the generation of the spiral (p.~180, 8–12), regards the
\emph{single-turn spiral} as actually \emph{stopping short} at the
point reached after one complete revolution of the straight line: ``it
is necessary to know that extending without limit is not a property of
all lines; for it neither belongs to the circle nor to the cissoid,
nor in general to lines which form figures; nor even to those which do
not form figures. For even the single-turn spiral does not extend
without limit—\emph{for it is constructed between two points}—nor does
any of the other lines so generated do so'' (p.~187, 19–25).  It is
curious that Pappus (\r8.~p.~1110~sqq.)\ uses the same term
\greek{μονόστροφος} to denote one turn, not of the spiral, but of the
\emph{cylindrical helix}.

\section*{Definition 3}

\greek{Γραμμῆς δὲ πέρατα σημεῖα.}

\emph{The extremities of a line art points.}

It being unscientific, as Aristotle said, to define a point as the
``extremity of a line'' (\greek{πέρας γραμμῆς}), thereby explaining
the prior by the posterior, Euclid defined a point differently; then,
as it was necessary to connect a point with a line, he introduced this
explanation after the definitions of both had been given.  This
compromise is no doubt his own idea; the same thing occurs with
reference to a surface and a line as its extremity in Def.~6, and with
reference to a solid and a surface as its extremity in
\book{11}{Def.~2}.

We miss a statement of the facts, equally requiring to be known, that
a ``division'' (\greek{διαίρεσις}) of a line, no less than its
``beginning'' or ``end,'' is a point (this is brought out by
Aristotle: cf.\ \emph{Metaph.}\ 1060~b~15), and that the
\emph{intersection} of two lines is also a point.  If these additional
explanations had been given, Proclus would have been spared the
difficulty which he finds in the fact that some of the lines used in
Euclid (namely infinite straight lines on the one hand, and circles on
the other) have no ``extremities.''  So also the ellipse, which
Proclus calls by the old name \greek{θυρεός} (``shield'').  In the
case of the circle and ellipse we can, he observes (p.~105,~7), take a
portion bounded by points, and the definition applies to that portion.
His rather far-fetched distinction between two aspects of a circle or
ellipse as a \emph{line} and as a \emph{closed figure} (thus, while
you are \emph{describing} a circle, you have two extremities at any
moment, but they disappear when it is finished) is an unnecessarily
elaborate attempt to establish the literal universality of the
``definition,'' which is really no more than an explanation that, if a
line has extremities, those extremities are points.

\section*{Definition 4}

\greek{Εὐθεῖα γραμμή ἐστιν, ἥτις ἐξ ἴσου τοῖς ἐφ’ ἑαυτῆς συμείοις κεῖται.}

\emph{A \emph{straight line} is a line which lies evenly with the
  points on itself.}

The only definition of a straight line authenticated as pre-Euclidean
is that of Plato, who defined it as ``\emph{that of which the middle
  covers the ends}'' (relatively, that is, to an eye placed at either
end and looking along the straight line).  It appears in the
\emph{Parmenides} 137~\textsc{e}: ``straight is whatever has its
middle in front of (i.e.\ so placed as to obstruct the view of) both
its ends'' (\greek{εὐθύ γε οὗ ἂ τὸ μέσον ἀμφοῖν τοῖν ἐσχάτοιν
  ἐπίπροσθεν ᾗ}).  Aristotle quotes it in equivalent terms
(\emph{Topics} \r6~11, 148~b~27), \greek{οὖ τὸ μέσον ἐπιπροσθεῖ τοῖς
  πέρασιν}; and, as he does not mention the name of its author, but
states it in combination with the definition of a line as the
extremity of a surface, we may assume that he used it as being well
known.  Proclus also quotes the definition as Plato's in almost
identical terms, \greek{ἧ τὰ μέσα τοῖς ἄκροις ἐπιπροσθεῖ} (p.~109,
21).  This definition is ingenious, but implicitly appeals to the
sense of sight and involves the postulate that the line of sight is
straight. (Cf.\ the Aristotelian \emph{Problems} 31, 20, 959~a~39,
where the question is why we can better observe straightness in a row,
say, of letters with one eye than with two.)  As regards the
straightness of ``visual rays,'' \greek{ὄψεις}, cf.\ Euclid's own
\emph{Optics}, Deff.~1, 2, assumed as \emph{hypotheses}, in which he
first speaks of the ``straight lines ``drawn from the eye, avoiding
the word \greek{ὄψεις}, and then says that the figure contained by the
\emph{visual rays} (\greek{ὄψεις}) is a cone with its vertex in the
eye.

As Aristotle mentions no definition of a straight line resembling
Euclid's, but gives only Plato's definition and the other explaining
it as the ``extremity of a surface,'' the latter being evidently the
current definition in contemporary textbooks, we may safely infer that
Euclid's definition was a new departure of his own.

\subsubsection{Proclus on Euclid's definition}

Coming now to the interpretation of Euclid's definition, \greek{εὐθεῖα
  γραμμή ἐστιν, ἥτις ἐξ ἴσου τοῖς ἐφ’ ἑαυτῆς σημείοις κεῖται}, we find
any number of slightly different versions, but none that can be
described as quite satisfactory; some authorities, e.g.\ Savile, have
confessed that they could make nothing of it.  It is natural to appeal
to Proclus first; and we find that he does in fact give an
interpretation which at first sight seems plausible.  He says (p.~109,
8~sq.)\ that Euclid ``shows by means of this that the straight line
alone [of all lines] occupies a distance (\greek{κατέχειν δίστημα})
equal to that between the points on it.  For, as far as one of the
points is distant from another, so great is the length
(\greek{μέγεθος}) of the straight line of which they are the
extremities; and this is the meaning of lying \greek{ἐξ ἴσου} to (or
with) the points on it ``[\greek{ἐξ ἴσου} being thus, apparently,
  interpreted as ``at'' (or ``over'') ``an equal distance'']. ``But if
you take two points on the circumference (of a circle) or any other
line, the distance cut off between them along the line is greater than
the interval separating them. And this is the case with every line
except the straight line.  Hence the ordinary remark, based on a
common notion, that those who journey in a straight line only travel
the necessary distance, while those who do not go straight travel more
than the necessary distance.''  (Cf.\ Aristotle, \emph{De caelo}
\prop{1}{4}, 271~a~13, ``we always call the distance of anything the
straight line'' drawn to it.) Thus Proclus would interpret somewhat in
this way: ``a straight line is that which represents extension equal
with (the distances separating) the points on it.'' This explanation
seems to be an attempt to graft on to Euclid's definition the
\emph{assumption} (it is a \greek{λαμβανόμενον}, not a definition) of
Archimedes (\emph{On the sphere and cylinder}~\r1.\ \emph{ad
  init.})\ that ``of all the lines which have the same extremities the
straight line is least.''  For this purpose \greek{ἐξ ἴσου} has
apparently to be taken as meaning ``at an equal distance,'' and again
``lying at an equal distance'' as equivalent to ``extending over (or
representing) an equal distance.''  This is difficult enough in
itself, but is seen to be an impossible interpretation when applied to
the similar definition of a plane by Euclid (Def.~7) as a surface
``which lies evenly with the straight lines on itself.'' In that
connexion Proclus tries to make the same words \greek{ἐξ ἴσου κεῖται}
mean ``extends over an equal \emph{area} with.'' He says namely
(p.~117,~2) that, ``if two straight lines are set out'' on the plane,
the plane surface ``occupies a space equal to that between the
straight lines.''  But two straight lines do not determine by
themselves any space at all; it would be necessary to have a
\emph{closed} figure with its boundaries in the plane before we could
arrive at the equivalent of the other assumption of Archimedes that
``of surfaces which have the same extremities, if those extremities
are in a plane, the plane is the least [in area].''  This seems to be
an impossible sense for \greek{ἐξ ἴσου} even on the assumption that it
means ``at an equal distance'' in the present definition.  The
necessity therefore of interpreting \greek{ἐξ ἴσου} similarly in both
definitions makes it impossible to regard it as referring to
\emph{distance} or \emph{length} at all.  It should be added that
Simplicius gave the same explanations as Proclus (an-Nairīzī, p.~5).

\subsubsection{The language and construction of the definition}

Let us now consider the actual wording and grammar of the phrase
\greek{ἥτις ἐξ ἴσου τοῖς ἐφ’ ἑαυτῆς σημείοις κεῖται}.  As regards the
expression if \greek{ἐξ ἴσου} we note that Plato and Aristotle (whose
use of it seems typical) commonly have it in the sense of ``on a
footing of equality'': cf.\ \greek{οἱ ἐξ ἴσου} in Plato's \emph{Laws}
777~\textsc{d}, 919~\textsc{d}; Aristotle, \emph{Politics} 1259~b~5
\greek{ἐξ ἴσου ει‹›ναι Βούλεται τὴν φύσιν} ``tend to be on an equality
in nature,'' \emph{Eth.\ Nic.}\ \r8.~12, 1161~a~8 \greek{ἐνταῦθα
  πάντες ἐξ ἴσου}, ``there all are on a footing of equality.''
Slightly different are the uses in Aristotle,
\emph{Eth.\ Nic.}\ \r10.~8, 1178~a~25 \greek{τῶν μὲν γὰρ ἀναγκαίων
  χρεία καὶ ἐξἴσου ἔστω}, ``both need the necessaries of life to the
same extent, Set us say''; \emph{Topics} \r9,~15, 174~a~32 \greek{ἐξ
  ἴσου ποιοῦντα τὴν ἐρώτησιν}, ``asking the question indifferently''
(i.e.\ without showing any expectation of one answer being given
rather than another). The natural meaning would therefore appear to be
``evenly placed'' (or balanced), ``in equal measure,''
``indifferently'' or ``without bias'' one way or the other. Next, is
the dative \greek{τοῖς ἐφ’ ἑαυτῆς σημείοις} constructed with \greek{ἐξ
  ἴσου} or with \greek{κεῖται}?  In the first case the phrase must
mean ``that which lies \emph{evenly with} (or in respect to) the
points on it,'' in the second apparently ``that which, in (or by) the
points on it, lies (or is placed) evenly (or uniformly).'' Max Simon
takes the first construction to give the sense ``die Gerade liegt in
gleicher Weise wie ihre Punkte. '' If the last words mean ``in the
same way as (or in like manner as) its points,'' I cannot see that
they tell us anything, although Simon attaches to the words the notion
of \emph{distance} (Abstand) like Proclus.  The second construction he
takes as giving ``die Gerade liegt für (durch) ihre Punkte
gleichmässig,'' ``the straight line lies symmetrically for (or
through) its points''; or, if \greek{κεῖται} is taken as the passive
of \greek{τίθημι}, ``die Gerade ist durch ihre Punkte gleichmässig
gegeben worden,'' ``the straight line is symmetrically determined by
its points.''  He adds that the idea is here \emph{direction}, and
that both \emph{direction} and \emph{distance} (as between two
different given points simply) would be to Euclid, as later to Bolzano
(\emph{Betrachtungen über einige Gegenstände der Elementargeometrie},
1804, quoted by Schotten, \emph{Inhalt und Methode des planimetrischen
  Unterricts}, \r2.~p.~16), primary irreducible notions.

While the language is thus seen to be hopelessly obscure, we can
safely say that the sort of idea which Euclid wished to express was
that of a line which presents the same shape at and relatively to all
points on it, without any irregular or unsymmetrical feature
distinguishing one part or side of it from another. Any such
irregularity could, as Saccheri points out (Engel and Stäckel,
\emph{Die Theorie der Parallellinien von Euklid bis Gauss}, 1895,
p.~109), be at once made perceptible by keeping the ends fixed and
turning the line about them right round; if any two positions were
distinguishable, e.g.\ one being to the left or right relatively to
another, ``it would not lie in a uniform manner between its points.''

\subsubsection{A conjecture as to its origin and meaning}

The question arises, what was the origin of Euclid's definition, or,
how was it suggested to him?  It seems to me that the basis of it was
really Plato's definition of a straight line as ``that line the middle
of which covers the ends.''  Euclid was a Platonist, and what more
natural than that he should have adopted Plato's definition in
substance, while regarding it as essential to change the form of words
in order to make it independent of any implied appeal to vision,
which, as a physical fact, could not properly find a place in a purely
geometrical definition?  I believe therefore that Euclid's definition
is simply an attempt (albeit unsuccessful, from the nature of the
case) to express, in terms to which a geometer could not object as not
being part of geometrical subject-matter, the same thing as the
Platonic definition.

The truth is that Euclid was attempting the impossible.  As Pfieiderer
says (Scholia to Euclid), ``It seems as though the notion of a
\emph{straight line}, owing to its simplicity, cannot be explained by
any regular definition which does not introduce words already
containing in themselves, by implication, the notion to be defined
(such e.g.\ are direction, equality, uniformity or evenness of
position, unswerving course), and as though it were impossible, if a
person does not already know what the term \emph{straight} here means,
to teach it to him unless by putting before him in some way a picture
or a drawing of it. '' This is accordingly done in such books as
Veronese's \emph{Elementi di geometria} (Part~\r1., 1904, p.~10): ``A
stretched string, e.g.\ a plummet, a ray of light entering by a small
hole into a dark room, are \emph{rectilineal} objects.  The image of
them gives us the abstract idea of the limited line which is called a
\emph{rectilineal segment}.''

\subsubsection{Other definitions}

We will conclude this note with some other famous definitions of a
straight line.  The following are given by Proclus (p.~110, 18–23).

1.~\emph{A line stretched to the utmost}, \greek{ἐπ’ ἄκρον τετραμένη
  γραμμή}. This appears in Heron also, with the words ``towards the
ends ``(\greek{ἐπὶ τὰ πέρατα}) added.  (Heron, Def.~4).

2.~\emph{Part of it cannot be in the assumed plane while part is in
  one higher up} (\greek{ἐν μετεωροτέρῳ}). This is a proposition in
Euclid (\prop{11}{1}).

3.`\emph{All its parts fit on all (other parts) alike}, \greek{πάντα
  αὐτῆς τὰ μέρη πᾶσιν ὁμοίως ἐφαρμόζει}.  Heron has this too (Def.~4),
but instead of ``alike'' he says \greek{παντοίως}, ``in all ways,''
which is better as indicating that the applied part may be applied one
way or the \emph{reverse} way, with the same result.

4.~\emph{That line which, when its ends remain fixed, itself remains
  fixed}, \greek{ἡ τῶν περάτων μενόντων καὶ αὐτὴ μένουσα}.  Heron's
addition to this, ``\emph{when it is, as it were, turned round in the
  same plane}'' (\greek{οἷον ἐν τῷ αὐτῷ ἐπιπέδῳ στρεφομένη}), and his
next variation, ``and about the same ends having always the same
position,'' show that the definition of a straight line as ``that
which does not change its position when it is turned about its
extremities (or any two points in it) as poles ``was no original
discovery of Leibniz, or Saccheri, or Krafft, or Gauss, but goes back
at least to the beginning of the Christian era.  Gauss' form of this
definition was: ``The line in which lie all points that, during the
revolution of a body (a part of space) about two fixed paints,
maintain their position unchanged is called a straight line.''
Schotten (\r1.~p.~315) maintains that the notion of a straight line
and its property of being determined by two points are unconsciously
assumed in this definition, which is therefore a logical ``circle.''

5.~\emph{That line which with one other of the same species cannot
  complete a figure}, \greek{ἡ μετὰ τῆς ὁμοειδοῦς μιᾶς σχῆμα μὴ
  ἀποτελοῦσα}. This is an obvious \greek{ὕστερον-πρότερον}, since it
assumes the notion of a \emph{figure}.

Lastly Leibniz' definition should be mentioned: \emph{A straight line
  is one which divides a plane into two halves identical in all but
  position}.  Apart from the fact that this definition introduces the
plane, it does not seem to have any advantages over the definition
last but one referred to.

Legendre uses the Archimedean property of a straight line as \emph{the
  shortest distance between two points}.  Van Swinden observes
(\emph{Elemente der Geometrie}, 1834, p.~4), that to take this as the
definition involves \emph{assuming} the proposition that any two sides
of a triangle are greater than the third and proving that straight
lines which have two points in common coincide tnroughout their length
(cf.\ Legendre, \emph{Éléments de Géométrie} \prop{1}{3}, 8).

The above definitions all illustrate the observation of Unger
(\emph{Die Geometrie des Euklid}, 1833): ``\emph{Straight} is a simple
notion, and hence all definitions of it must fail…. But if the proper
idea of a straight line has once been grasped, it will be recognised
in all the various definitions usually given of it; all the
definitions must therefore be regarded as \emph{explanations}, and
among them that one is the best from which further inferences can
immediately be drawn as to the essence of the straight line.''

\section*{Definition 5}

\greek{Ἐπιφάνεια δέ ἐστιν, ὃ μῆκος καὶ πλάτος μόνον ἔχει.}

A \emph{surface} it that which has length and breadth only.

The word \greek{ἐπιφάνεια} was used by Euclid and later writers to
denote \emph{surface} in general, while they appropriated the word
\greek{ἐπίπεδον} for \emph{plane} surface, thus making
\greek{ἐπίπεδον} a \emph{species} of the \emph{genus}
\greek{ἐπιφάνεια}.  A solitary use of \greek{ἐπιφάνεια} by Euclid when
a plane is meant (\book{11}{Def.~11}) is probably due to the fact that
the particular definition came from an earlier textbook. Proclus
(p.~116, 17) remarks that the older philosophers, including Plato and
Aristotle, used the words \greek{ἐπιφάνεια} and \greek{ἐπίπεδον}
indifferently for any kind of surface.  Aristotle does indeed use both
words for a surface, with perhaps a tendency to use \greek{ἐπιφάνεια}
more than \greek{ἐπίπεδον} for a surface not plane.
Cf.\ \emph{Categories} 6, 5~a~1~sq., where both words are used in one
sentence: ``You can find a common boundary at which the parts fit
together, a point in the case of a line, and a line in the case of a
surface (\greek{ἐπιφάνεια}); for the parts of the surface
(\greek{ἐπιπέδου}) do fit together at some common boundary.  Similarly
also in the ease of a body you can find a common boundary, a line or a
surface (\greek{ἐπιφάνεια}), at which the parts of the body fit
together.''  Plato however does not use \greek{ἐπιφάνεια} at all in
the sense of surface, but only \greek{ἐπίπεδον} for both
\emph{surface} and \emph{plane surface}.  There is reason therefore
for doubting the correctness of the notice in Diogenes Laertius,
\r3.~24, that Plato ``was the first philosopher to name, among
extremities, the \emph{plane} surface'' (\greek{ἐπίπεδος ἐπιφάνεια}).

\greek{ἐπιφάνεια} of course means literally the feature of a body
which is \emph{apparent} to the eye (\greek{ἐπιφανής}), namely the
surface.

Aristotle tells us (\emph{De sensu} 3, 439~a~31) that the Pythagoreans
called a surface \greek{χροιά}, which seems to have meant \emph{skin}
as well as \emph{colour}.  Aristotle explains the term with reference
to colour (\greek{χρο›μα}) as a thing inseparable from the extremity
(\greek{πέρας}) of a body.

\subsubsection*{Alternative definitions}

The definitions of a surface correspond to those of a line.  As in
Aristotle a line is a magnitude ``(extended) one way, or in one
'dimension'\,'' `(\greek{ἐφ’ ἕν}), ``continuous one way'' (\greek{ἐφ’
  ἓν συμεχές}), or ``divisible in one way'' (\greek{μοναχῇ
  διαιρετόν}), so a surface is a magnitude extended or continuous
\emph{two ways} (\greek{ἐπὶ δύο}), or divisible \emph{in two ways}
(\greek{διχῇ}).  As in Euclid a surface has ``length and breadth''
only, so in Aristotle ``breadth'' is characteristic of the surface and
is once used as synonymous with it (\emph{Metaph.}\ 1020~a~12), and
again ``lengths are made up of long and short, \emph{surfaces of broad
  and narrow}, and solids (\greek{ὅγκοι}) of deep and shallow''
(\emph{Metaph.}\ 1085~a~10).

Aristotle mentions the common remark that \emph{a line by its motion
  produces a surface} (\emph{De anima} \r1.~4, 409~a~4).  He also
gives the \emph{a posteriori} description of a surface as the
``extremity of a solid'' (\emph{Topics} \r6.~4, 141~b~21), and as
``the section (\greek{τουμή}) or division (\greek{διαίρεσις}) of a
body'' (\emph{Metaph.}\ 1060~b~14).

Proclus remarks (p.~114, 20) that we get a notion of a surface when we
measure areas and mark their boundaries in the sense of length and
breadth; and we further get a sort of perception of it by looking at
shadows, since these have no depth (for they do not penetrate the
earth) but only have length and breadth.

\subsubsection*{Classification of surfaces}

Heron gives (Def.~74, p.~50, ed.\ Heiberg) two alternative divisions
of surfaces into two classes, corresponding to Geminus' alternative
divisions of lines, viz.\ into (1)~\emph{incomposite} and
\emph{composite} and (2)~\emph{simple} and \emph{mixed}.

(1)~\emph{Incomposite} surfaces are ``those which, when produced, fall
into (or coalesce with) themselves'' (\greek{ὅσαι ἐκβαλλόμεναι αὐταὶ
  καθ’ ἑαυτῶν πίπτουσιν}), i.e.\ are of continuous curvature,
e.g.\ the sphere.

\emph{Composite} surfaces are ``those which, when produced, cut one
another.''  Of composite surfaces, again, some are (\emph{a})~made up
of non-homogeneous (elements) (\greek{ἐξ ἀνομοιογενῶν}) such as cones,
cylinders and hemispheres, others (\emph{b})~made up of homogeneous
(elements), namely the rectilineal (or polyhedral) surfaces.

(2)~Under the alternative division, \emph{simple} surfaces are the
plane and the spherical surfaces, but no others; the \emph{mixed}
class includes all other surfaces whatever and is therefore infinite
in variety.

Heron specially mentions as belonging to the mixed class
(\emph{a})~the surface of cones, cylinders and the like, which are a
mixture of plane and circular (\greek{μικταὶ ἐξ ἐπιπέδου καὶ
  περιφερείας}) and (\emph{b})~\emph{spiric} surfaces, which are ``a
mixture of two circumferences'' (by which he must mean a mixture of
two circular elements, namely the generating circle and its circular
motion about an axis in the same plane).

Proclus adds the remark that, curiously enough, \emph{mixed} surfaces
may arise by the revolution either of \emph{simple} curves, e.g.\ in
the case of the \emph{spire}, or of mixed curves, e.g.\ the
``right-angled conoid'' from a parabola, ``another conoid'' from the
hyperbola, the ``oblong'' (\greek{ἐπίμηκες}, in Archimedes
\greek{παραμᾶκες}) and ``flat ``(\greek{ἐπιπλατύ}) spheroids from an
ellipse according as it revolves about the major or minor axis
respectively (pp.~119, 6–120, 2). The homoeomeric surfaces, namely
those any part of which will coincide with any other part, are
\emph{two} only (the plane and the spherical surface), not three as in
the case of lines (p.~120,~7).

\section*{Definition 6}

\greek{Ἐπιφανείας δὲ πέρατα γραμμαί.}

\emph{The extremities of a surface are lines.}

It being unscientific, as Aristotle says, to define a line as the
extremity of a surface, Euclid avoids the error of defining the prior
by means of the posterior in this way, and gives a different
definition not open to this objection.  Then, by way of compromise,
and in order to show the connexion between a line and a surface, he
adds the equivalent of the definition of a line previously current as
an explanation.

As in the corresponding Def.~3 above, he omits to add what is made
clear by Aristotle (\emph{Metaph.}\ 1060~b~15) that a ``division''
(\greek{διαίρεσις}) or ``section'' (\greek{τομή}) of a solid or body
is also a surface, or that the common boundary at which two parts of a
solid fit together (\emph{Categories} 6, 5~a~2) may be a surface.

Proclus discusses how the fact stated in Def.~6 can be said to be true
of surfaces like that of the sphere ``which is bounded
(\greek{πεπέρασται}), it is true, but not by lines.'' His explanation
(p.~116, 8–14) is that, ``if we take the surface (of a sphere), so far
as it is extended two ways (\greek{διχῇ}), we shall find that it is
bounded by lines as to length and breadth; and if we consider the
spherical surface as possessing a form of its own and invested with a
fresh quality, we must regard it as having fitted end on to beginning
and made the two ends (or extremities) one, being thus one potentially
only, and not in actuality.''

\section*{Definition 7}

\greek{Ἐπίπεδος ἐπιφάνειά ἐστιν ἐξ ἴσου ταῖς ἐφ’ ἑαυτῆς εὐθείαις
  κεῖται.}

\emph{A \emph{plane surface} is a surface which lies evenly with the
  straight lines on itself.}

The Greek follows exactly the definition of a straight line
\emph{mutatis mutandis}, i.e.\ with \greek{παῖς…εὐθείαις} for
\greek{τοῖς…σημείοις}.  Proclus remarks that, in general, all the
definitions of a straight line can be adapted to the plane surface by
merely changing the \emph{genus}.  Thus, for instance, a plane surface
is ``a \emph{surface} the middle of which covers the ends'' (this
being the adaptation of Plato's definition of a straight line).
Whether Plato actually gave this as the definition of a plane surface
or not, I believe that Euclid's definition of a plane surface as
\emph{lying evenly with the straight lines on itself} was intended
simply to express the same idea without any implied appeal to vision
(just as in the corresponding case of the definition of a straight
line).

As already noted under Def.~4, Proclus tries to read into Euclid's
definition the Archimedean assumption that ``of surfaces which have
the same extremities, if those extremities are in a plane, the plane
is the least.''  But, as I have stated, his interpretation of the
words seems impossible, although it is adopted by Simplicius also (see
an-Nairīzī).

\subsubsection*{Ancient alternatives}

The other ancient definitions recorded are as follows.

1.~\emph{The surface which is stretched to the utmost} (\greek{ἐπ’
  ἄκρον τεταμένη}): a definition which Proclus describes as equivalent
to Euclid's definition (on Proclus' own view of that definition).
Cf.\ Heron, Def.~9, ``(a surface) which is right (and) stretched out''
(\greek{ὀρθὴ οὖσα ἀποτεταμένη}), words which he adds to Euclid's
definition.

2.\emph{The least surface among all those which have the same
  extremities}.  Proclus is here (p.~117,~9) obviously quoting the
Archimedean \emph{assumption}.

3.~\emph{A surface all the parts of which have the property of fitting
  on (each ether)} (Heron, Def.~9).

4.~\emph{A surface such that a straight line fits on all parts of it}
(Proclus, p.~117,~8), or such that the straight line fits on it all
ways, i.e.\ however placed (Proclus, p.~117,~20).

With this should be compared:

5.~``\emph{(A plane surface is) such that, if a straight line pass
  through two points on it, the line coincides wholly with it at every
  spot, all ways},'' i.e.\ however placed (one way or the reverse, no
matter how), \greek{ἡς ἐπειδὰν δύο σημείων ἅψηται εὐθεῖα, καὶ ὅλη αὐτῇ
  κατὰ πάντα τόπον παντοίως ἐφαρμόζεται} (Heron, Def.~9).  This
appears, with the words \greek{κατὰ πάντα τόπον παντοίως} omitted, in
Theon of Smyrna (p.~112,~5, ed.\ Hiller), so that it goes back at
least as far as the 1st~c.~\ad.  It is of course the same as the
definition commonly attributed to Robert Simson, and very widely
adopted as a substitute for Euclid's.

This same definition appears also in an-Nairīzī (ed.\ Curtze, p.~10)
who, after quoting Simplicius' explanation (on the same lines as
Proclus') of the meaning of Euclid's definition, goes on to say that
``others defined the plane surface as that in which it is possible to
draw a straight tine from any point to any other.''

\subsubsection*{Difficuties in ordinary definitions}

Gauss observed in a letter to Bessel that the definition of a plane
surface as \emph{a surface such that, if any him points in it be
  taken, the straight line joining them lies wholly in the surface}
(which, for short, we will call ``Simson's'' definition) contains more
than is necessary, in that a plane can be obtained by simply
projecting a straight line lying in it from a point outside the line
but also lying on the plane; in fact the definition includes a
theorem, or postulate, as well.  The same is true of Euclid's
definition of a plane as the surface which ``lies evenly with
(\emph{all}) the straight lines on itself,'' because it is sufficient
for a definition of a plane if the surface ``lies evenly'' with those
lines only which pass through a fixed point on it and each of the
several points of a straight line also lying in it but not passing
through the point.  But from Euclid's point of view it is immaterial
whether a definition contains more than the necessary minimum
\emph{provided} that the \emph{existence} of a thing possessing all
the attributes contained in the definition is afterwards proved.  This
however is not done in regard to the plane.  No proposition about the
nature of a plane as such appears before Book~\r11., although its
existence is presupposed in all the geometrical Books \r1.–\r4.\ and
\r6.; nor in Book~\r11.\ is there any attempt to prove, e.g.\ by
construction, the existence of a surface conforming to the definition.
The explanation may be that the existence of the plane as defined was
deliberately assumed from the beginning like that of points and lines,
the existence of which, according to Aristotle, must be assumed as
principles unproved, while the existence of everything else must be
proved; and it may well be that Aristotle would have included plane
surfaces with points and lines in this statement had it not been that
he generally took his illustrations from \emph{plane} geometry
(excluding solid).

But, whatever definition of a plane is taken, the evolution of its
essential properties is extraordinarily difficult.  Crelle, who wrote
an elaborate article \emph{Zur Theorie der Ebene} (read in the
Academie der Wissenschaften in 1834) of which account must be taken in
any full history of the subject, observes that, since the plane is the
field, as it were, of almost all the rest of geometry, while a proper
conception of it is necessary to enable Eucl.\ \prop{1}{1} to be
understood, it might have been expected that the theory of the plane
would have been the subject of at least the same amount of attention
as, say, that of parallels.  This however was far from being the case,
perhaps because the subject of parallels (which, for the rest,
presuppose the notion of a plane) is \emph{much easier} than that of
the plane.  The nature of the difficulties as regards the plane have
also been pointed out recently by Mr Frankland (\emph{The First Book
  of Euclid's Elements}, Cambridge, 1905); it would appear that,
whatever definition is taken, whether the simplest (as containing the
minimum necessary to determine a plane) or the more complex,
e.g.\ Simson's, some postulate has to be assumed in addition before
the fundamental properties, or the truth of the other definitions, can
be established.  Crelle notes the same thing as regards Simson's
definition, containing \emph{more} than is necessary.
\sidefig{defI_7a}Suppose a plane
in which lies the triangle $ABC$.  Let $AD$ join the vertex~$A$ to any
point~$D$ on~$EC$, and $BE$ the vertex~$B$ to any point~$E$ on~$CA$.
Then, according to the definition, $AD$ lies wholly in the plane of
the triangle; so does $BE$.  But, if both $AD$ and~$BE$ are to lie
wholly in the one plane, $AD$, $BE$ must intersect, say at~$F$: if
they did not, there would be two planes in question, not one. But the
fact that the lines intersect and that, say, $AD$ does not pass above
or below~$BE$, is by no means self-evident.

Mr~Frankland points out the similar difficulty as regards the simpler
\sidefig{defI_7b}
definition of a plane as the surface generated by a straight line
passing always through a fixed point and always intersecting a fixed
straight line.  Let $OPP'$, $OQQ'$ drawn from~$O$ intersect the
straight line $X$ at~$P$, $Q$ respectively.  Let $E$ be any third
point on~$X$: then it needs to be proved that $OR$ intersects $P'Q'$
in some point, say~$R'$.  Without some postulate, however, it is not
easy to see how to prove this, or even to prove that $P'Q'$
intersects~$X$.

\subsubsection*{Crelle's essay. Definitions by Fourier, Deahna, Becker,}

Crelle takes as the standard of a good definition that it shall be,
not only as simple as possible, but also the best adapted for
deducing, with the aid of the simplest possible principles, further
properties belonging to the thing defined.  He was much attracted by a
very lucid definition, due, he says, to Fourier, according to which
\emph{a plane is formed by the aggregate of all the straight lines
  which, passing through one point on a straight line in space, are
  perpendicular to that straight line}. (This is really no more than
an adaptation from Euclid's proposition \prop{11}{5}, to the effect
that, if one of four concurrent straight lines be at right angles to
each of the other three, those three are in one plane, which
proposition is also used in Aristotle, \emph{Meteorologica} \r3.~3,
373~a~13.)  But Crelle confesses that he had not been able to deduce
the necessary properties from this and had had to substitute the
definition, already mentioned, of a plane as \emph{the surface
  containing, throughout their whole length, all the straight lines
  passing through a fixed point and also intersecting a straight line
  in space}; and he only claims to have proved, after a long series of
propositions, that the ``Fourier''- or ``perpendicular''-surface and
the \emph{plane} of the other definition just given are identical,
after which the properties of the ``Fourier''-surface can be used
along with those of the plane. The advantage of the Fourier definition
is that it leads easily, by means of the two propositions that
triangles are equal in all respects (1)~when two sides and the
included angle are respectively equal and (2)~when all three sides are
respectively equal, to the property expressed in Simson's definition.
But Crelle uses to establish these two congruence-theorems a number of
propositions about \emph{equal angles}, \emph{supplementary} angles,
\emph{right} angles, \emph{greater} and \emph{less} angles; and it is
difficult to question the soundness of Schotten's criticism that these
notions in themselves really presuppose that of a plane.  The
difficulty due to Fourier's use of the word ``perpendicular,'' if that
were all, could no doubt be got over.  Thus Deahna in a dissertation
(Marburg, 1837) constructed a plane as follows.  Presupposing the
notions of a straight line and a sphere, he observes that, if a sphere
revolve about a diameter, all the points of its surface which move
describe closed curves (circles).  Each of these circles, during the
revolution, moves along itself, and one of them divides the surface of
the sphere into two congruent parts.  The aggregate then of the lines
joining the centre to the points of this circle forms the
\emph{plane}.  Again, J.~K. Becker (\emph{Die Elemente der Geometrie},
1877) pointed out that the revolution of a right angle about one side
of it produces a conical surface which differs from all other conical
surfaces generated by the revolution of other angles in the fact that
\emph{the particular cone coincides with the cone vertically opposite
  to it}: this characteristic might therefore be taken in order to get
rid of the use of the \emph{right angle}.

\subsubsection*{W. Bolyai and Lobachewsky}

Very similar to Deahna's equivalent for Fourier's definition is the
device of W.~Bolyai and Lobachewsky (described by Frischauf,
\emph{Elemente der absoluten Geometrie}, 1876). They worked upon a
fundamental idea first suggested, apparently, by Leibniz.  Briefly
stated, their way of evolving a \emph{plane} and a \emph{straight
  line} was as follows.  Conceive an infinite number of pairs of
concentric spheres described about two fixed points in space, $O$,
$0'$, as centres, and with equal radii, gradually increasing: these
pairs of equal spherical surfaces intersect respectively in
homogeneous curves (circles), and the ``Inbegriff'' or aggregate of
these curves of intersection forms a \emph{plane}.  If $A$ be a point
on one of these circles ($k$ say), suppose points $M$, $M'$ to start
simultaneously from~$A$ and to move in opposite directions at the same
speed till they meet at~$B$, say; $B$ then is ``opposite'' to~$A$, and
$A$, $B$ divide the circumference into two equal halves.  If the
points $A$, $B$ be held fast and the whole system be turned about them
until $O$ takes the place of $O'$, and $O'$ of~$O$, the circle $k$
will occupy the same position as before (though turned a different
way).  Two opposite points, $P$, $Q$ say, of each of the other circles
will remain stationary during the motion as well as $A$, $B$: the
``Inbegriff'' or aggregate of all such points which remain stationary
forms a straight line.  It is next observed that the plane as defined
can be generated by the revolution of the straight line about $OO'$,
and this suggests the following construction for a plane.  Let a
circle as one of the curves of intersection of the pairs of spherical
surfaces be divided as before into two equal halves $A$, $B$.  Let the
arc $ADB$ 
\sidefig{defI_7c}be similarly bisected at~$D$, and let $C$ be the middle
point of~$AB$.  This determines a straight line $CD$ which is then
\emph{defined} as ``perpendicular'' to~$AB$.  The revolution of~$CD$
about~$AB$ generates a plane.  The property stated in Simson's
definition is then proved by means of the congruence-theorems proved
in Eucl.\ \prop{1}{8} and \prop{1}{4}.  The first is taken as proved,
practically by considerations of symmetry and homogeneity.  If two
spherical surfaces, not necessarily equal, with centres~$O$, $O'$
intersect, $A$ and its ``opposite'' point $B$ are taken as before on
the curve of intersection (a circle) and, relatively to $00'$, the
point $A$ is taken to be convertible with~$B$ or any other point on
the homogeneous curve.  The second (that of Eucl.\ \prop{1}{4}) is
established by simple application.  Rausenberger objects to these
proofs on the grounds that the first \emph{assumes} that the two
spherical surfaces intersect in one single curve, not in several, and
that the second compares \emph{angles}: a comparison which, be says,
is possible only in a \emph{plane}, so that a plane is really
presupposed.  Perhaps as regards the particular comparison of angles
Rausenberger is hypercritical; but it is difficult to regard the
supposed proof of the theorem of Eucl.\ \prop{1}{8} as sufficiently
rigorous (quite apart from the use of the uniform \emph{motion} of
points for the purpose of bisecting lines).

\sidefig{defI_7d}

Simson's property is proved from the two congruence-theorems thus.
Suppose that $AB$ is ``perpendicular'' (as defined by Bolyai) to two
generators $CM$, $CN$ of a plane, or suppose $CM$, $CN$ respectively
to make with $AB$ two angles congruent with one another.  It is enough
to prove that, if $P$ be any point on the straight line $MN$, then
$CP$, just as much as $CM$, $CN$ respectively, makes with $AB$ two
angles congruent with one another and is therefore a generator.  We
prove successively the congruence of the following pairs of triangles:
\[
\begin{array}{ll}
ACM, &BCM\\
ACN, &BCN\\
AMN, &BMN\\
AMP, &BMP\\
ACP, &BCP,
\end{array}
\]
whence the angles $ACP$, $BCP$ are congruent.

\subsubsection*{Other views}

Enriques and Amaldi (\emph{Elementi di geometria}, Bologna, 1905),
Veronese (in his \emph{Elementi}) and Hilbert all assume as a
\emph{postulate} the property stated in Simson's definition. But
G.~Ingrami (\emph{Elementi di geometria}, Bologna, 1904) proves it tn
the course of a remarkable series of closely argued proposition based
upon a much less comprehensive postulate.  He evolves the theory of
the plane from that of a triangle, beginning with a triangle as a mere
\emph{three-side} (trilatero), i.e.\ a \emph{frame}, as it were.  His
postulate relates to the \emph{three-side} and is to the effect that
each ``(rectilineal) segment'' joining a vertex to a point of the
opposite side meets every segment similarly joining each of the other
two vertices to the points of the sides opposite to them respectively,
and, conversely, if a point be taken on a segment joining a vertex to
a point of the opposite side, and if a straight line be drawn from
another vertex to the point on the segment so taken, it will if
produced meet the opposite side.  A \emph{triangle} is then defined as
the figure formed by the aggregate of all the segments joining the
respective vertices of a \emph{three-side} to points on the opposite
sides.  After a series of propositions, Ingrami evolves a plane as
\emph{the figure formed by the ``half straight-lines'' which project
  from an internal point of the triangle the points of the perimeter},
and then, after two more theorems, proves that a plane is determined
by any three of its points which are not in a straight line, and that
\emph{a straight line which has two points in a plane has all its
  points in it}.

The argument by which Bolyai and Lobachewsky evolved the plane is of
course equivalent to the definition of a plane as \emph{the locus of
  all points equidistant from two fixed points in space}.

Leibniz in a letter to Giordano defined a plane as \emph{that surface
  which divides space into turn congruent parts}.  Adverting to
Giordano's criticism that you could conceive of surfaces and lines
which divided space or a plane into two congruent parts without being
plane or straight respectively, Beez (\emph{Über Euklidische und
  Nicht-Euklidische Geometrie}, 1888) pointed out that what was wanted
to complete the definition was the further condition that the two
congruent spaces could be slid along each other without the surfaces
ceasing to coincide, and claimed priority for his completion of the
definition in this way.  But the idea of \emph{all the parts} of a
plane fitting exactly on \emph{all other parts} is ancient, appearing,
as we have seen, in Heron, Def.~9.

\section*{Definitions 8, 9}

8. \greek{Ἐπίπεδος δὲ γωνία ἐστὶν ἡ ἐν ἐπιπέδῳ δύο γραμμῶν ἁπτομένων
  ἀλλήλων καὶ μὴ εὐθείας κειμένων πρὸς ἀλλήλας τῶν γραμμο›ν κλίσις.}

9. \greek{Ὅταν δὲ αἱ περιέχουσαι τὴν γωνίαν γραμμαὶ εὖθεῖαι ὦσιν,
  εὐθύγραμμος καλεῖται ἡ γωνία.}

8. \emph{A \emph{plane angle} is the inclination to one another of two
  lines in a plane which meet one another and do not lie in a straight
  line.}

9. \emph{And when the lines containing the angle are straight, the
  angle is called \emph{rectilineal}.}

The phrase ``not in a \emph{straight line}'' is strange, seeing that
the definition purports to apply to angles formed by \emph{curves} as
well as straight lines.  We should rather have expected
\emph{continuous} (\greek{συνεχής}) with one another; and Heron takes
this to be the meaning, since he at once adds an explanation as to
what is meant by lines not being \emph{continuous} (\greek{οὐ
  συνεχεῖς}).  It looks as though Euclid really intended to define a
\emph{rectilineal} angle, but on second thoughts, as a concession to
the then common recognition of curvilineal angles, altered ``straight
lines'' into ``lines'' and separated the definition into two.

I think all our evidence suggests that Euclid's definition of an angle
as \emph{inclination} (\greek{κλίσις}) was a new departure.  The word
does not occur in Aristotle; and we should gather from him that the
idea generally associated with an angle in his time was rather
\emph{deflection} or \emph{breaking} of lines (\greek{κλάσις}):
cf.\ his common use of \greek{κεκλάσθαι} and other parts of the verb
\greek{κλᾶν}, and also his reference to \emph{one bent line} forming
an angle (\greek{γὴν κεκαμμένην καὶ ἔχουσαν γωνίαν},
\emph{Metaph.}\ 1016~a~13)

Proclus has a long and elaborate note on this definition, much of
which (pp.~121, 12–126, 6) is apparently taken direct from a work by
his master Syrianus (\greek{ὁ ἡμετερος καθηγεμών}).  Two criticisms
contained in the note need occasion no difficulty.  One of these asks
how, if an angle be an inclination, one inclination can produce two
angles.  The other (p.~128,~2) is to the effect that the definition
seems to exclude an angle formed by one and the same curve with
itself, e.g.\ the complete \emph{cissoid} [at what we call the
  ``cusp''] or the curve known as the \emph{hippopede} (horse-fetter)
[shaped like a lemniscate].  But such an ``angle'' as this belongs to
higher geometry, which Euclid may well be excused for leaving out of
account in any case.

\subsubsection*{Other ancient definitions: Apollonius, Plutarch, Carpus}

Proclus' note records other definitions of great interest.  Apollonius
defined an angle as \emph{a contracting of a surface or a solid at one
  point under a broken line or surface} (\greek{ἢ στερεοῦ πρὸς ἑνὶ
  σημείῳ ὑπὸ κεκλασμένῃ γραμμῇ ἢ ἐπιφανείᾳ}), where again an angle is
supposed to be formed by \emph{one} broken line or surface.  Still
  more interesting, perhaps, is the definition by ``those who say that
  \emph{the first distance under the point} (\greek{τὸ πρῶτον διάστημα
    ὑπὸ τὸ σημεῖον}) \emph{is the angle}.  Among these is Plutarch,
  who insists that Apollonius meant the same thing; for, he says,
  there must be \emph{some} first distance under the breaking (or
  deflection) of the including lines or surfaces, though, the distance
  under the point being continuous, it is impossible to obtain the
  actual \emph{first}, since every distance is divisible without
  limit'' (\greek{ἐπ’ ἄπειρον}).  There is some vagueness in the use
  of the word ``distance'' (\greek{διάστημα}); thus it was objected
  that ``if we anyhow separate off the \emph{first} ``(\emph{distance}
  being apparently the word understood) ``and draw a straight line
  \emph{through it}, we get a \emph{triangle} and not one angle.''  In
  spite of the objection, I cannot but see in the idea of Plutarch and
  the others the germ of a valuable conception in infinitesimals, an
  attempt (though partial and imperfect) to get at the \emph{rate of
    divergence} between the lines at their point of meeting as a
  measure of the angle between them.

A third view of an angle was that of Carpus of Antioch, who said
``that the angle was a \emph{quantity} (\greek{ποσόν}), namely a
\emph{distance} (\greek{διάστημα}) between the lines or surfaces
containing it.  This means that it would be a distance (or divergence)
in one \emph{sense} (\greek{ἐφ’ ἒν διεστώς}), although the angle is
not on that account a straight line.  For it is not everything
\emph{extended in one sense} (\greek{τὸ ἐφ’ ἒν διαστατόν}) that is a
line.''  This very phrase ``extended one way'' being held to define a
\emph{line}, it is natural that Carpus' idea should have been
described as the greatest possible paradox (\greek{πάντων
  παραδοξότατον}).  The difficulty seems to have been caused by the
want of a different technical term to express a new idea; for Carpus
seems undoubtedly to have been anticipating the more modern idea of an
angle as representing \emph{divergence} rather than distance, and to
have meant by \greek{ἐφ’ ἓν} \emph{in one sense (rotationally)} as
distinct from \emph{one way} or \emph{in one dimension} (linearly).

\subsubsection*{To what category does an angle belong?}

There was much debate among philosophers as to the particular category
(according to the Aristotelian scheme) in which an angle should be
placed; is it, namely, a \emph{quantum} (\greek{ποσόν}), \emph{quale}
(\greek{ποιόν}) or \emph{relation} (\greek{πρός τι})?

1.~Those who put it in the category of \emph{quantity} argued from the
fact that a plane angle is divided by a line and a solid angle by a
surface.  Since, then, it is a surface which is divided by a line, and
a solid which is divided by a surface, they felt obliged to conclude
that an angle is a surface or a solid, and therefore a magnitude.  But
homogeneous finite magnitudes, e.g.\ plane angles, must bear a ratio
to one another, or one must be capable of being multiplied until it
exceeds the other.  This is, however, not the case with a rectilineal
angle and the \emph{horn-like} angle (\greek{κερατοειδής}), by which
latter is meant the ``angle'' between a circle and a tangent to it,
since (Eucl.\ \prop{3}{16}) the latter ``angle'' is less than any
rectilineal angle whatever.  The objection, it will be observed,
assumes that the two sorts of angle \emph{are} homogeneous.  Plutarch
and Carpus are classed among those who, in one way or other, placed an
angle among \emph{magnitudes}; and, as above noted, Plutarch claimed
Apollonius as a supporter of his view, although the word
\emph{contraction} (of a surface or solid) used by the latter does not
in itself suggest magnitude much more than Euclid's inclination.  It
was this last consideration which doubtless led ``Aganis,'' the
``friend'' (socius) apparently of Simplicius, to substitute for
Apollonius' wording ``\emph{a quantify which has dimensions and the
extremities of which arrive at one point}'' (an-Nairīzī, p.~13).

2.~Eudemus the Peripatetic, who wrote a whole work on the angle,
maintained that it belonged to the category of \emph{quality}.
Aristotle had given as his fourth variety of \emph{quality} ``figure
and the shape subsisting in each thing, and, besides these,
straightness, curvature, and the like'' (\emph{Categories} 8,
10~a~11).  He says that each individual thing is spoken of as
\emph{quale} in respect of its form, and he instances a triangle and a
square, using them again later on (\ibid~11~a~5) to show that it is
not all qualities which are susceptible of more and less; again, in
\emph{Physics} \r1.~5, 188~a~25 angle, straight, circular are called
kinds of figure.  Aristotle would no doubt have regarded deflection
(\greek{κεκλάσθαι}) as belonging to the same category with
straightness and curvature (\greek{καυπυλότησ}).  At all events,
Eudemus took up an angle as having its origin in the \emph{breaking}
or \emph{deflection} (\greek{κλάσις}) of lines: deflection, he argued,
was quality if straightness was, and that which has its origin in
quality is itself quality.  Objectors to this view argued thus.  If an
angle be a quality (\greek{ποιότης}) like heat or cold, how can it be
bisected, say?  It can in fact be divided; and, if things of which
divisibility is an essential attribute are varieties of \emph{quantum}
and not qualities, an angle cannot be a quality.  Further, the
\emph{more} and the \emph{less} are the appropriate attributes of
quality, not the equal and the unequal; if therefore an angle were a
quality, we should have to say of angles, not that one is greater and
another smaller, but that one is more an angle and another less an
angle, and that two angles are not unequal but \emph{dissimilar}
(\greek{ἀνόμοιοι}).  As a matter of fact, we are told by Simplicius,
538,~21, on Arist.\ \emph{De caelo} that those who brought the angle
under the category of \emph{quale} did call equal angles
\emph{similar} angles; and Aristotle himself speaks of \emph{similar}
angles in this sense in \emph{De caelo} 296~b~20, 311~b~34.

3.~Euclid and all who called an angle an inclination are held by
Syrianus to have classed it as a \emph{relation} (\greek{πρός τι}).
Yet Euclid certainty regarded angles as magnitudes; this is clear both
from the earliest propositions dealing specifically with angles,
e.g.\ \prop{1}{9, 13}, and also (though in another way) from his
describing an angle in the very next definition and always as
contained (\greek{περιεχομένη}) by the two lines forming it (Simon,
\emph{Euclid}, p.~28).

Proclus (i.e.\ in this case Syrianus) adds that the truth lies between
these three views.  The angle partakes in fact of all those
categories: it needs the \emph{quantity} involved in magnitude,
thereby becoming susceptible of equality, inequality and the like; it
needs the \emph{quality} given it by its \emph{form}, and lastly the
\emph{relation} subsisting between the lines or planes bounding it.

\subsubsection*{Ancient classification of ``angles.''}

An elaborate classification of angles given by Proclus (pp.~126,
7–127, 16) may safely be attributed to Geminus.  In order to show it
by a diagram it

Angles

on surfaces

in solids
(ἐν στερεοῖς)

on \emph{simple} surfaces

on \emph{mixed} surfaces
(e.g.\ cones, cylinders)

on \emph{planes}

on spherical surfaces

made by \emph{simple} lines

made by ``\emph{mixed}'' lines
e.g.\ the angle made by a curve, such as the \emph{cissoid} and
\emph{hippopede}, with itself)

by \emph{one} of each
(e.g.\ the angle formed by an ellipse and its axis or by an ellipse
and a circle)

line-line

line-circumf.

circumf.-circumf.

line-convex
(e.g.\ angle of a semicircle)

line-concave
e.g., \emph{horn-like}
(\greek{κερατοειδής})

convex-convex
(\greek{ἀμφίκυρτοι})

concave-concave
(\greek{ἀμφίκοιλοι})
or ``scraper-like''
(\greek{ξυστροειδεῖς})

mixed, or convex-concave
(e.g.\ those of \emph{lunes})

will be necessary to make a convention about terms.  Angles are to be
understood under each class, ``line-circumference'' means an angle
contained by a straight line and an arc of a circle, ``line-convex''
an angle contained by a straight line and a circular arc with
convexity \emph{outwards}, and so on in every case.

\subsubsection*{Definitions of angle classified}

As for the point, straight line, and plane, so for the \emph{angle},
Schotten gives a valuable summary, classification and criticism of the
different modern views up to date (\emph{Inhalt und Methode des
  planimetrischen Unterrichts}, \r2., 1893, pp.~94–183); and for later
developments represented by Veronese reference may be made to the
third article (by Amaldi) in \emph{Questioni riguardanti le
  matematiche elementari}, \r`.\ (Bologna, 1912).

With one or two exceptions, says Schotten, the definitions of an angle
may be classed in three groups representing generally the following
views:

1.~\emph{The angle is the difference of direction between two straight
  lines.}  (With this group may be compared Euclid's definition of an
angle as an inclination.)

2.~\emph{The angle is the quantity or amount (or the measure) of the
  rotation necessary to bring one of its sides from its own position
  to that of the other side without its moving out of the plane
  containing both.}

3.~\emph{The angle is the portion of a plane included between two
  straight tines in the plane which meet in a point (or two rays
  issuing from the point).}

It is remarkable however that nearly all of the text-books which give
definitions different from those in group~2 add to them something
pointing to a connexion between an angle and rotation: a striking
indication that the essential nature of an angle is closely connected
with rotation, and that a good definition must take account of that
connexion.

The definitions in the first group must be admitted to be tautologous,
or \emph{circular}, inasmuch as they really presuppose some conception
of an angle.  \emph{Direction} (as \emph{between two given points})
may no doubt be regarded as a primary notion; and it may be defined as
``the immediate relation of two points which the ray enables us to
realise'' (Schotten).  But ``a direction is no intensive magnitude,
and therefore two directions cannot have any quantitative difference''
(Bürklen).  Nor is direction susceptible of differences such as those
between qualities, e.g.\ colours.  Direction is a \emph{singular}
entity: there cannot be different \emph{sorts} or \emph{degrees} of
direction.  If we speak of ``a \emph{different} direction,'' we use
the word equivocally; what we mean is simply ``another `` direction.
The fact is that these definitions of an angle as a difference of
direction unconsciously appeal to something outside the notion of
direction altogether, to some conception equivalent to that of the
angle itself.

\subsubsection*{Recent Italian views}

The second group of definitions are (says Amaldi) based on the idea of
the rotation of a straight line or ray in a plane about a point: an
idea which, logically formulated, may lead to a convenient method of
introducing the angle.  But it must be made independent of
\emph{metric} conceptions, or of the conception of \emph{congruence},
so as to bring out \emph{first} the notion of an angle, and
\emph{afterwards} the notion of \emph{equal} angles.

The third group of definitions satisfy the condition of not including
metric conceptions; but they do not entirely correspond to our
intuitive conception of an angle, to which we attribute the character
of an entity in \emph{one} dimension (as Veronese says) with respect
to the \emph{ray} as element, or an entity in \emph{two} dimensions
with reference to \emph{points} as elements, which may be called an
\emph{angular sector}.  The defect is however easily remedied by
considering the angle as ``the aggregate of the rays issuing from the
vertex and comprised in the angular sector,''

Proceeding to consider the principal methods of arriving at the
logical formulation of the first superficial properties of the
\emph{plane} from which a definition of the angle may emerge, Amaldi
distinguishes two points of view (1)~the \emph{genetic}, (2)~the
\emph{actual}.

(1)~From the first point of view we consider the \emph{cluster of
  straight lines} or \emph{rays} (the aggregate of all the straight
lines in a plane passing through a point, or of all the rays with
their extremities in that point) as generated by the movement of a
straight line or ray in the plane, about a point.  This leads to the
\emph{postulation} of a \emph{closed order}, or \emph{circular
  disposition}, of the straight lines or rays in a cluster.  Next
comes the connexion subsisting between the disposition of any two
clusters whatever in one, plane, and so on.

(2)~Starting from the point of view of the \emph{actual}, we lay the
foundation of the definition of an angle in \emph{the division of the
  plane into two parts} (half-planes) \emph{by the straight line}.
Next, two straight lines ($a$, $b$) in the plane, intersecting at a
point~$O$, divide the plane into four \emph{regions} which are called
\emph{angular sectors} (convex); and finally the \emph{angle} ($ab$)
or ($ba$) may be defined as \emph{the aggregate of the rays issuing
  from~$O$ and belonging to the angular sector which has $a$ and~$b$
  for sides}.

Veronese's procedure (in his \emph{Elementi}) is as follows.  He
begins with the first properties of the plane introduced by the
following definition.

The figure given by all the straight lines joining the points of a
\sidefig{defI_8a}
straight line~$r$ to a point~$P$ outside it and by the parallel to~$r$
through~$P$ is called a \emph{cluster of straight lines}, a
\emph{cluster of rays}, or a \emph{plane}, according as we consider
the \emph{element} of the figure itself to be the \emph{straight
  line}, the \emph{ray} terminated at~$P$, or a \emph{point}.

[It will be observed that this method of producing a plane involves
  using the \emph{parallel} to~$r$.  This presents no difficulty to
  Veronese because he has previously defined parallels, without
  reference to the plane, by means of \emph{reflex} or \emph{opposite}
  figures, with respect to a point~$O$: ``two straight lines are
  called \emph{parallel}, if one of them contains two points opposite
  to (or the reflex of) two points of the other with respect to the
  middle point of a common transversal (of the two lines).'' He proves
  by means of a postulate that the parallel $r'$ does belong to the
  plane~$Pr$, Ingrami avoids the use of the parallel by defining a
  \emph{plane} as ``the figure formed by the half straight lines which
  project from an internal point of a triangle (i.e.\ a point on a
  line joining any vertex of a \emph{three-side} to a point of the
  opposite side) the points of its perimeter,'' and then defining a
  \emph{cluster} of rays as ``the aggregate of the half straight lines
  in a plane starting from a given point of the plane and passing
  through the points of the perimeter of a triangle containing the
  point'']

Veronese goes on to the definition of an angle. ``\emph{We call an
  angle a part of a cluster of rays, bounded by two rays (as the
  segment is a part of a straight line bounded hy two points).}

``\emph{An angle of the cluster, the bounding rays, of which are
  opposite, is called a \emph{flat angle}.}''

Then, after a postulate corresponding to postulates which he lays down
for a \emph{rectilineal segment} and for a \emph{straight line},
Veronese proves that \emph{all flat angles} are equal to one another.

\infig{defI_8b}

Hence he concludes that ``the cluster of rays is a homogeneous linear
system in which the element is the \emph{ray} instead of the
\emph{point}.  The cluster being a homogeneous linear system, all the
propositions deduced from [Veronese's] Post.~\r1\ for the straight
line apply to it, e.g.\ that relative to the sum and difference of the
segments: it is only necessary to substitute the ray for the point,
and the angle for the segment.''

\section*{Definitions 10, 11, 12}

10.~\greek{Ὅταν δὲ εὐθεῖα ἐπ’ εὐθεῖαν σταθεῖσα τὰς ἐφεξῆς γωνίας ἴσας
  ἀλλήλαις ποιῆ, ὀρθὴ ἑκατέρα τῶν ἴσων γωνιῶν ἐστί, καὶ ἡ ἐφεστηκυῖα
  εὐθεῖα κάθετος καλεῖται, ἐφ’ ἥν ἐφέστηκεν.}

11.~\greek{Ἀμβλεῖα γωνία ἐστὶν ἡ μείζων ὁρθῆς.}

12.~\greek{Ὀξεῖα δὲ ἡ ἐλάσσων ὀρθῆς.}

10.~\emph{When a straight line set up on a straight line makes the
  adjacent angles equal is one another, each of the equal angles is
  \emph{right}, and the straight line standing on the other is called
  a \emph{perpendicular} to that on which it stands.}

11.~\emph{An \emph{obtuse angle} is an angle greater than a right
  angle.}

12.~\emph{An \emph{acute angle} is an angle less than a right angle.}

\greek{ἐφεξῆς} is the regular term for \emph{adjacent} angles, meaning
literally ``(next) in order.''  I do not find the term used in
Aristotle of \emph{angles}, but he explains its meaning in such
passages as \emph{Physics} \r6.~1x, 131~b~8: ``those things are (next)
in order which have nothing of the same kind (\greek{συγγενές})
between them.''

\greek{κάθετος}, \emph{perpendicular}, means literally \emph{let
  fall}: the full expression is \emph{perpendicular straight line}, as
we see from the enunciation of Eucl.\ \prop{1}{11}, and the notion is
that of a straight line let fall \emph{upon the surface of the earth},
a \emph{plumb-line}.  Proclus (p.~283,~9) tells us that in ancient
times the perpendicular was called \emph{gnomon-wise} (\greek{κατὰ
  γνώμονα}), because the gnomon (an upright stick) was set up at right
angles to the horizon.

The three kinds of angles are among the things which according to the
Platonic Socrates (\emph{Republic} \r6.~510~c) the geometer assumes
and argues from, declining to give any account of them because they
are obvious.  Aristotle discusses the \emph{priority} of the right
angle in comparison with the acute (\emph{Metaph.}\ 1084~b~7;): in one
way the right angle is prior, i.e.\ \emph{in being defined}
(\greek{ὅτι ὥρισται}) and by its \emph{notion} (\greek{τῷ λόγῳ}), in
another way the acute is prior, i.e.\ as being a \emph{part}, and
because the right angle is divided into acute angles; the acute angle
is prior as \emph{matter}, the right angle in respect of \emph{form};
cf.\ also \emph{Metaph.}\ 1035~b~6, ``the notion of the right angle is
not divided into that of an acute angle, but the reverse; for, when
defining an acute angle, you make use of the right angle.''  Proclus
(p.~133, 15) observes that it is by the \emph{perpendicular} that we
measure the heights of figures, and that it is by reference to the
right angle that we distinguish the other rectilineal angles, which
are otherwise undistinguished the one from the other.

The Aristotelian \emph{Problems} (16, 4, 913~b~36) contain an
expression perhaps worth quoting.  The question discussed is why
things which fall on the ground and rebound make ``similar'' angles
with the surface on both sides of the point of impact; and it is
observed that ``the right angle is the \emph{limit} (\greek{ὅρος}) of
the opposite angles,'' where however ``opposite'' seems to mean, not
``supplementary'' (or acute and obtuse), but the equal angles made
with the surface on opposite sides of the perpendicular.

Proclus, after his manner, remarks that the statement that an angle
less than a right angle is acute is not true without qualification,
for (1)~the \emph{horn-like} angle (between the circumference of a
circle and a tangent) is less than a right angle, since it is less
than an \emph{acute} angle, but is not an acute angle, while (2)~the
``angle of a semicircle'' (between the arc and a diameter) is also
less than a right angle, but is not an acute angle.

The \emph{existence} of the right angle is of course proved in
\prop{1}{11}.

\section*{Definition 13}

\greek{Ὅρος ἐστίν, ὅ τινός ἐστι πέρας.}

\emph{A \emph{boundary} is that which is an extremity of anything.}

Aristotle also uses the words \greek{ὅρος} and \greek{πέρας} as
synonymous.  Cf.\ \emph{De gen.\ animal}, \r2.~6, 745~a~6, 9, where in
the expression ``limit of magnitude'' first one and then the other
word is used.

Proclus (p.~136, 8) remarks that the word boundary is appropriate to
the origin of geometry, which began from the measurement of areas of
ground and involved the marking of boundaries.

\section*{Definition 14}

\greek{Σχῆμά ἐστι τὸ ὑπό τινος ἥ τινων ὅρων περιεχόμενοι.}

\emph{A \emph{figure} is that which is contained by any boundary or
  boundaries.}

Plato in the \emph{Meno} observes that roundness
(\greek{στρογγυλότης}) or the \emph{round} is a ``figure,'' and that
\emph{the straight} and many other things are so too; he then inquires
what there is common to all of them, in virtue of which we apply the
term ``figure'' to them. His answer is (76~a): ``with reference to
every figure I say that \emph{that in which the solid terminates
  (\greek{τοῦτο, εἰς ὓ τὸ στερεὸν περαίνει}) is a figure}, or, to put
it briefly, \emph{a figure is an extremity of a solid}.''  The first
observation is similar to Aristotle's in the \emph{Physics} \r1.~5,
188~a~25, where \emph{angle}, \emph{straight}, and \emph{circular} are
mentioned as genera of figure.  In the \emph{Categories} 8, 10~a~11,
``figure'' is placed with straightness and curvedness in the category
of quality.  Here however ``figure'' appears to mean \emph{shape}
(\greek{μορφή}) rather than ``figure'' in our sense.  Coming nearer to
''figure'' in our sense, Aristotle admits that figure is ``\emph{a
  sort} of magnitude'' (\emph{De anima} \r3.~1, 425~a~18), and he
distinguishes \emph{plane figures} of two kinds, in language not
unlike Euclid's, as \emph{contained} by straight and circular lines
respectively: ``every plane figure is either rectilineal or formed by
circular lines (\greek{περιφερόγραμμον}), and the rectilineal figure
is contained by several lines, the circular by one line'' (\emph{De
  caelo} \r2.~4, 286~b~13).  He is careful to explain that a plane is
not a figure, nor a figure a plane, but that a plane figure
constitutes one notion and is a \emph{species} of the \emph{genus}
figure (\emph{Anal.\ post.} \r2.~3, 90~b~37).  Aristotle does not
attempt to define figure in general, in fact he says it would be
useless: ``From this it is clear that there is one definition of soul
in the same way as there is one definition of \emph{figure}; for in
the one case there is no figure except the triangle, quadrilateral,
and so on, nor is there any soul other than those above mentioned.  A
definition might be constructed which should apply to all figures but
not specially to any particular figure, and similarly with the species
of soul referred to.  [But such a general definition would serve no
  purpose.]  Hence it is absurd here as elsewhere to seek a general
definition which will not be properly a definition of anything in
existence and will not be applicable to the particular irreducible
species before us, to the neglect of the definition which is so
applicable'' (\emph{De anima} \r2.~3, 414~b~20–28).

Comparing Euclid's definition with the above, we observe that by
introducing boundary (\greek{ὅρος}) he at once excludes the
\emph{straight} which Aristotle classed as figure; he doubtless
excluded \emph{angle} also, as we may judge by (1)~Heron's statement
that ``neither one nor two straight lines can complete a figure,''
(2)~the alternative definition of a straight line as ``that which
cannot with another line of the same species form a figure,''
(3)~Geminus' distinction between the line which \emph{forms a figure}
(\greek{σχηματοποιοῦσα}) and the line which \emph{extends
  indefinitely} (\greek{ἐπ’ ἄπειρον ἐκβαλλομένη}), which latter term
includes a hyperbola and a parabola.  Instead of calling figure an
\emph{extremity} as Plato did in the expression ``extremity (or limit)
of a solid,'' Euclid describes a figure as \emph{that which has} a
boundary or boundaries.  And lastly, in spite of Aristotle's
objection, he does attempt a general definition to cover all kinds of
figure, solid and plane.  It appears certain therefore that Euclid's
definition is entirely his own.

Another view of a figure, recalling that of Plato in \emph{Meno}
76~\textsc{a}, is attributed by Proclus (p.~143, 8) to Posidonius.
The latter regarded the \emph{figure} as the \emph{confining
  extremity} or \emph{limit} (\greek{πέρας συγκλεῖον}), ``separating
the notion of figure from \emph{quantity} (or magnitude) and making it
the cause of \emph{definition}, \emph{limitation}, and
\emph{inclusion} (\greek{ποῦ ὡρίσθαι καὶ πεπεράσθαι καὶ τῆς
  περιοχῆς})… Posidonius thus seems to have in view only the boundary
placed round from outside, Euclid the whole content, so that Euclid
will speak of the circle as a figure in respect of its whole plane
(surface) and of its inclusion (from) without, whereas Posidonius
(makes it a figure) in respect of its circumference… Posidonius wished
to explain the notion of figure as itself \emph{limiting} and
\emph{confining} magnitude.''

Proclus observes that a logical and refining critic might object to
Euclid's definition as defining the genus from the species, since that
which is enclosed by one boundary and that which is enclosed by
several are both species of figure.  The best answer to this seems to
be supplied by the passage of Aristotle's \emph{De anima} quoted
above.

\section*{Definitions 15, 16}

15.~\greek{Κύκλος ἐστὶ σχῆμα ἐπιπεδον ὑπὸ μιᾶς γραμμῆς περιεχόμενον [ἣ
    καλεῖται περιφέρεια], πρὸς ἣν ἀφ’ ἑνὸς συμείου το›ν ἐντὸς τοῦ
  σχήματος κειμένων πᾶσαι αἱ προσπίπτουσαι εὐθεῖαι [πρὸς τὴν τοῦ
    κύκλου περιφέρειαν] ἴσαι ἀλλήλαις εἰσιν.}

16.~\greek{Κέντρον δὲ τοῦ κύκλου τὸ σημεῖον καλεῖται.}

15.~\emph{A \emph{circle} is a plane figure contained by one line such
  that all the straight lines falling upon it from one point among
  those lying within the figure are equal to one another;}

16.~\emph{And the point is called the \emph{centre} of the circle.}

The words \greek{ἣ καλεῖται περιφέρεια}, ``which is called the
circumference,'' and \greek{πρὸς τὴν τοῦ κύκλου περιφέρειαν}, ``to the
circumference of the circle,'' are bracketed by Heiberg because,
although the \emph{mss.}\ have them, they are omitted in other ancient
sources, viz.\ Proclus, Taurus, Sextus Empiricus and Boethius, and
Heron also omits the second gloss.  The recently discovered papyrus
Herculanensis No.~1061 also quotes the definition without the words in
question, confirming Heiberg's rejection of them (see Heiberg in
\emph{Hermes} \r38., 1903, p.~47).  The words were doubtless added in
view of the occurrence of the word ``circumference'' in Deff.~17, 18
immediately following, without any explanation.  But no explanation
was needed.  Though the word \greek{περιφέρεια} does not occur in
Plato, Aristotle uses it several times (1)~in the general sense of
contour without any special mathematical signification,
(2)~mathematically, with reference to the rainbow and the
circumference, as well as an arc, of a circle.  Hence Euclid was
perfectly justified in employing the word in Deff.~17, 18 and
elsewhere, but leaving it undefined as being a word universally
understood and not involving in itself any mathematical conception. It
may be added that an-Nairīzī had not the bracketed words in his text;
for he comments on and tries to explain Euclid's omission to define
the circumference.

The definition itself contained nothing new in substance.  Plato
(\emph{Parmenides} 137~\textsc{e}) says: ``\emph{Round} is, I take it,
that the extremes of which are every way equally distant from the
middle'' (\greek{στρογγύλον γέ τού ἐστι τοῦτο, οὖ ἂν τὰ ἔσχατα πανταχῇ
  ἀπὸ τοῦ μέσου ἴσον ἀπέχῃ}). In Aristotle we find the following
expressions: ``the circular (\greek{περιφερόγραμμον}) plane figure
bounded by one line'' (\emph{De caelo} \r2.~4, 286~b~13—16); ``the
plane equal (i.e.\ extending equally all ways) from the middle''
(\greek{ἐπίπεδον τὸ ἐκ τοῦ μέσου ἴσον}), meaning a circle
(\emph{Rhetoric} \r3.~6, 1407~b~27); he also contrasts with the circle
``any other figure which has not the lines from the middle equal, as
for example an egg-shaped figure'' (\emph{De caelo} \r2.~4, 287~a~19).
The word ``centre'' (\greek{κέντρον}) was also regularly used:
cf.\ Produs' quotation from the ``oracles'' (\greek{λόγια}), ``the
centre from which all (lines extending) as far as the rim are equal.''

The definition as it stands has no \emph{genetic} character.  It says
nothing as to the existence or non-existence of the thing defined or
as to the method of constructing it.  It simply explains what is meant
by the word ``circle,'' and is a provisional definition which cannot
be used until the existence of circles is proved or assumed.
Generally, in such a case, existence is proved by actual construction;
but here the possibility of constructing the circle as defined, and
consequently its existence, are \emph{postulated} (Postulate~3).  A
genetic definition might state that a circle is the figure described
when a straight line, always remaining in one plane, moves about one
extremity as a fixed point until it returns to its first position (so
Heron, Def.~27).

Simplicius indeed, who points out that the distance between the feet
of a pair of compasses is a straight line from the centre to the
circumference, will have it that Euclid intended by this definition to
show how to construct a circle by the revolution of a straight line
about one end as centre; and an-Nairīzī points to this as the
explanation (1)~of Euclid's definition of a circle as a \emph{plane
  figure}, meaning the whole surface bounded by the circumference, and
not the circumference itself, and (2)~of his omission to mention the
``circumference,'' since with this construction the circumference is
not drawn separately as a \emph{line}.  But it is not necessary to
suppose that Euclid himself did more than follow the traditional view;
for the same conception of the circle as a \emph{plane figure}
appears, as we have seen, in Aristotle.  While, however, Euclid is
generally careful to say the ``\emph{circumference} of a circle ``when
he means the circumference, or an arc, only, there are cases where
``circle'' means ``circumference of a circle,'' e.g.\ in \prop{3}{10}:
``A circle does not cut a circle in more points than two.''

Heron, Proclus and Simplicius are all careful to point out that the
centre is not the only point which is equidistant from all points of
the circumference.  The centre is the only point \emph{in the plane of
  the circle} (``lying within the figure,'' as Euclid says) of which
this is true; any point not in the same plane which is equidistant
from all points of the circumference is a \emph{pole}.  If you set up
a ``gnomon'' (an upright stick) at the centre of a circle (i.e.\ a
line through the centre perpendicular to the plane of the circle), its
upper extremity is a pole (Proclus, p.~153, 3); the perpendicular is
the locus of all such poles.

\section*{Definition 17}

\greek{Διάμετρος δὲ τοῦ κύκλου ἐστὶν εὐθεῖά τις διὰ τοῦ κέντρου ἠγμένη
  καὶ περατουμένη ἐφ’ ἑκάτερα τὰ μέρη ὑπὸ τῆς τοῦ κύκλου περιφερείας,
  ἥτις καὶ δίχα τέμνει τὸν κ’υκλον.}

\emph{A \emph{diameter} of the circle is any straight line drawn
  through the centre and terminated in both directions by the
  circumference of the circle, and such a straight line also bisects
  the circle.}

The last words, literally ``which (straight line) also bisects the
circle,'' are omitted by Simson and the editors who followed him.  But
they are necessary even though they do not ``belong to the
definition'' but only express a property of the diameter as defined.
For, without this explanation, Euclid would not have been justified in
describing as a \emph{semi}-circle a portion of a circle bounded by a
diameter and the circumference cut off by it.

Simplicius observes that the \emph{diameter} is so called because it
passes \emph{through} the whole surface of a circle as if
\emph{measuring} it, and also because it divides the circle into two
equal parts.  He might however have added that, in general, it is a
line passing through a figure where it is widest, as well as dividing
it equally: thus in Aristotle \greek{τὰ κατὰ διάμετρον κείμενα},
``things diametrically situated'' in space, are at their maximum
distance apart.  \emph{Diameter} was the regular word in Euclid and
elsewhere for the diameter of a \emph{square}, and also of a
parallelogram; \emph{diagonal} (\greek{διαγώνιος}) was a later term,
defined by Heron (Def.~67\?) as the straight line drawn from an angle
to an angle.

Proclus (p.~157, 10) says that Thales was the first to prove that a
circle is bisected by its diameter; but we are not told how he proved
it.  Proclus gives as the \emph{reason} of the property ``the
undeviating course of the straight line through the centre'' (a simple
appeal to symmetry), but adds that, if it is desired to prove it
mathematically, it is only necessary to imagine the diameter drawn and
one part of the circle applied to the other; it is then clear that
they must coincide, for, if they did not, and one fell inside or
outside the other, the straight lines from the centre to the
circumference would not all be equal: which is absurd.

Saccheri's proof is worth quoting.  It depends on three ``Lemmas''
immediately preceding, (1)~that two straight lines cannot enclose a
space, (2)~that two straight lines cannot have one and the same
segment common, (3)~that, if two straight lines meet at a point, they
do not touch, but cut one another, at it.

``Let $MDHNKM$ be a circle, $A$ its centre, $MN$ a diameter.  Suppose
the portion $MNKM$ of the circle turned about the fixed points $M$,
$N$, so that it ultimately comes near to or coincides with the
remaining portion $MNHDM$.

\sidefig{defI_17}

``Then (i)~the whole diameter $MAN$, with all
its points, clearly remains in the same position,
since otherwise two straight lines would enclose a
space (contrary to the first Lemma).

``(ii)~Clearly no point $K$ of the circumference $NKM$ falls within or
outside the surface enclosed by the diameter $MAN$ and the other part,
$NHDM$, of the circumference, since otherwise, contrary to the nature
of the circle, a radius as $AK$ would be less or greater than another
radius as~$AH$.

``(iii)~Any radius $MA$ can clearly be rectilineally produced only
along a single other radius~$AN$, since otherwise (contrary to the
second Lemma) two lines assumed straight, e.g.\ $MAN$, $MAH$, would
have one and the same common segment.

``(iv)~All diameters of the circle obviously cut one another in the centre
(Lemma~3 preceding), and they bisect one another there, by the general
properties of the circle.

``From all this it is manifest that the diameter $MAN$ divides its
circle and the circumference of it just exactly into two equal parts,
and the same may be generally asserted for every diameter whatsoever
of the same circle; which was to be proved.''

Simson observes that the property is easily deduced from \prop{3}{31
  and 24}; for it follows from \prop{3}{31} that the two parts of the
circle are ``similar segments'' of a circle (segments containing equal
angles, \book{3}{Def.~11}), and from \prop{3}{24} that they are equal
to one another.

\section*{Definition 18}

\greek{Ἡμικύκλιον δέ ἐστι τὸ περιεχόμενον σχῆμα ὑπό τε τῆς διαμέτρου
  καὶ τῆς ἀπολαμβανομένης ὑπ’ αὐτῆς περιφερείας.  κέντρον δὲ τοῦ
  ἡμικυκλίου τὸ αὐτό, ὃ καὶ τοῦ κύκλου ἐστίν.}

\emph{A \emph{semicircle} if the figure contained by the diameter and
  the circumference cut off by it. And the centre of the semicircle is
  the same as that of the circle.}

The last words, ``And the centre of the semicircle is the same as that
of the circle,'' are added from Proclus to the definition as it
appears in the \textsc{mss.}  Scarburgh remarks that a semicircle has
no centre, properly speaking, and thinks that the words are not
Euclid's, but only a note by Proclus.  I am however inclined to think
that they are genuine, if only because of the very futility of an
observation added by Proclus.  He explains, namely, that the
semicircle is the only plane figure that has its centre on its
perimeter~(!), ``so that you may conclude that the centre has three
positions, since it may be within the figure, as in the case of a
circle, or on the perimeter, as with the semicircle, or outside, as
with some conic lines (the single-branch hyperbola presumably)''!

Proclus and Simplicius point out that, in the order adopted by Euclid
for these definitions of figures, the first figure taken is that
bounded by \emph{one} line (the circle), then follows that bounded by
\emph{two} lines (the semicircle), then the triangle, bounded by
\emph{three} lines, and so on.  Proclus, as usual, distinguishes
different kinds of figures bounded by two lines (pp.~159, 14–160,
9). Thus they may be formed

(1)~by circumference and circumference, e.g.\ (\emph{a})~those forming
angles, as a \emph{lune} (\greek{τὸ μηνοειδές}) and the figure
included by two arcs with convexities outward, and (\emph{b})~the
\emph{angle-less} (\greek{ἀγώνιον}), as the figure included between
two concentric circles (the \emph{coronal});

(2)~by circumference and straight line, e.g.\ the semicircle or
segments of circles (\greek{ἁψῖδες} is a name given to those less than
a semicircle);

(3)~by ``mixed'' line and ``mixed'' line, e.g.\ two ellipses cutting
one another;

(4)~by ``mixed'' line and circumference, e.g.\ intersecting ellipse
and circle;

(5)~by ``mixed'' line and straight line, e.g.\ half an ellipse.

Following Def.~18 in the \textsc{mss.}\ is a definition of a
\emph{segment of a circle} which was obviously interpolated from
\book{3}{Def.~6}.  Proclus, Martianus Capella and Boethius do not give
it in this place, and it is therefore properly omitted.

\section*{Definitions 19, 20, 21}

19.~\greek{Σχήματα εὐθύγραμμά ἐστι τὰ ὑπὸ εὐθειῶν περιεχόμενα,
  τρίπλευρα μὲν τὰ ὑπὸ τριῶν, τετράπλευρα δὲ τὰ ὑπὸ τεσσάρων,
  πολύπλευρα δὲ τὰ ὑπὸ πλειόνων ἣ τεσσάρων εὐθειῶν περιεχόμενα.}

20.~\greek{Τῶν δὲ πριπλεύρων σχημάτων ἰσόπλευρον μὲν τρίγωνόν ἐστι τὸ
  τὰς τρεῖς ἴσας ἔχον πλευράς, ἰσοσκελὲς δὲ τὸ τὰς δύο μόνας ἴσας ἔχον
  πλευράς, σκαληνὸν δὲ τὸ τὰς τρεῖς ἀνίσους ἐχον πλευράς.}

21.~\greek{Ἔτι δὲ τῶν τριπλεύρων σχημάτων ὀρθογώνιον μὲν τρίγωνόν ἐστι
  τὸ ἔχον ὀρθὴν γωνίαν, ἀμβλυγώνιον δὲ τὸ ἔχον ἀμβλεῖαν γωνίαν,
  ὀξυγώνιον δὲ τὸ τὰς τρεῖς ὀξείας ἔχον γωνίας.}

19.~\emph{\emph{Rectilineal figures} are those which are contained by
  straight lines, \emph{trilateral} figures being those contained by
  three, \emph{quadrilateral} those contained by four, and
  \emph{multilateral} those contained by more than four straight
  lines.}

20.~\emph{Of trilateral figures, an \emph{equilateral triangle} is
  that which has its three sides equal, an \emph{isosceles triangle}
  that which has two of its sides alone equal, and a \emph{scalene
    triangle} that which has its three sides unequal.}

21.~\emph{Further, of trilateral figures, a \emph{right-angled
    triangle} it that which has a right angle, an \emph{obtuse-angled}
  triangle that which has an obtuse angle, and an \emph{acute-angled}
  triangle that which has its three angles acute.}

\section*{19}

The latter part of this definition, distinguishing \emph{three-sided},
\emph{four-sided} and \emph{many-sided} figures, is probably due to
Euclid himself, since the words \greek{πρίπλευρον},
\greek{τετράπλευρον} and \greek{πολύπλευρον} do not appear in Plato or
Aristotle (only in one passage of the \emph{Mechanics} and of the
\emph{Problems} respectively does even \greek{τετράπλευρον},
\emph{quadrilateral}, occur).  By his use of \greek{τετρύγωνον}
quadrilateral, Euclid seems practically to have put an end to any
ambiguity in the use by mathematicians of the word
\greek{τετρύγωονον}, literally ``four-angled (figure),'' and to have
got it restricted to the \emph{square}. cf.\ note on
Def.~\ref{def:I_22},

\section*{20}

\emph{Isosceles} (\greek{ἰσοσκελής}, with equal legs) is used by Plato
as well as Aristotle.  \emph{Scalene} (\greek{σκαληνός}, with the
variant \greek{σκαληνής}) is used by Aristotle of a triangle with no
two sides equal: cf.\ also Tim.\ Locr.\ 98~\textsc{b}.  Plato,
\emph{Euthyphro} 12~\textsc{d}, applies the term ``scalene'' to an
\emph{odd} number in contrast to ``isosceles'' used of an even number,
Proclus (p.~168, 24) seems to connect it with \greek{σκάζω}, to
\emph{limp}; others make it akin to \greek{σκολιός}, \emph{crooked},
\emph{aslant}.  Apollonius uses the same word ``scalene'' of an
\emph{oblique} circular cone.

Triangles are classified, first with reference to their sides, and
then with reference to their angles.  Proclus points out that seven
distinct species of triangles emerge: (1)~the \emph{equilateral}
triangle, (2)~three species of \emph{isosceles} triangles, the
right-angled, the obtuse-angled and the acute-angled, (3)~the same
three varieties of \emph{scalene} triangles.

Proclus gives an odd reason for the dual classification according to
sides and angles, namely that Euclid was mindful of the fact that it
is not every \emph{triangle} that is \emph{trilateral} also.  He
explains this statement by reference (p.~165, 22) to a figure which
some called \emph{barb-like} (\greek{ἀκιδοειδής}) while Zenodorus
called it \emph{hollow-angled} (\greek{κοιλογώνιος}).  Proclus
mentions it again in his note on \prop{1}{22} (p.~328, 21~sqq.)\ as
one of the paradoxes of geometry, observing that it is seen in the
figure of that proposition.  This ``triangle'' is merely a
\emph{quadrilateral} \sidefig{defI_21} with a re-entrant angle; and
the idea that it has only three angles is due to the non-recognition
of the fourth angle (which is greater than two right angles) as being
an angle at all. Since Proclus speaks of the \emph{four-sided
  triangle} as ``one of the paradoxes in geometry,'' it is perhaps not
safe to assume that the misconception underlying the expression
existed in the mind of Proclus alone; but there does not seem to be
any evidence that Zenodorus called the figure in question a triangle
(cf.\ Pappus, ed.\ Hultsch, pp.~1154, 1206).

\section*{Definition 22}

\greek{Τῶν δὲ τετραπλεύρων σχημάτων τετράγωνον μέν ἐστιν, ὃ ἰσόπλευρόν
  τέ ἐστι καὶ ὀρθογώνιον, ἑτερόμηκες δέ, ὃ ὀρθογώνιον μέν, οὐκ
  ἰσόπλευρον δέ, ῥόμβος δέ, ὃ ἰσόπλευρον μέν, οὐκ ὀρθογώνιον δέ,
  ῥουβοειδὲς δὲ τὸ τὰς ἀπεναντίον πλευράς τε καὶ γωνίας ἴσας ἀλλήλαις
  ἔχον, ὃ οὔτε ἰσόπλευρόν ἐστιν οὔτε ὀρθογώνιον τὰ δὲ παρὰ ταῦτα
  τετράπλευρα τραπέζια καλείσθω.}

\emph{Of quadrilateral figures, a \emph{square} is that which is both
  equilateral and right-angled; an \emph{oblong} that which is
  right-angled but not equilateral; a \emph{rhombus} that which is
  equilateral but not right-angled; and a \emph{rhomboid} that which
  has its opposite sides and angles equal te one another but is
  neither equilateral nor right-angled. And let quadrilaterals ether
  than these be called \emph{trapezia}.}

\greek{τετράγωνον} was already a \emph{square} with the Pythagoreans
(cf.\ Aristotle, \emph{Metaph.} 986~a~26), and it is so most commonly
in Aristotle; but in \emph{De anima} \r2.~3, 414~b~31 it seems to be a
quadrilateral, and in \emph{Metaph.}\ 1054~b~2, ``equal and
equiangular \greek{τετράγωνα},'' it cannot be anything else but
quadrilateral if ``equiangular'' is to have any sense.  Though, by
introducing \greek{τετράπλευρον} for any quadrilateral, Euclid enabled
ambiguity to be avoided, there seem to be traces of the older vague
use of \greek{τετράγωνον} in much later writers.  Thus Heron
(Def.~100) speaks of a cube as ``contained by six equilateral and
\emph{equiangular} \greek{τετράγωνα}'' and Proclus (p.~166, 10) adds
to his remark about the ``four-sided triangle'' that ``you might have
\greek{τετράγωνα} with more than the four sides,'' where
\greek{τετράγωνα} can hardly mean squares.

\greek{ἑτερόμηκες}, \emph{oblong} (with sides of \emph{different
  length}), is also a Pythagorean term.

The word \emph{right-angled} (\greek{ὀρθογώνιον}) as here applied to
quadrilaterals must mean \emph{rectangular} (i.e., practically, having
all its angles right angles); for, although it is tempting to take the
word in the same sense for a square as for a triangle (i.e.\ ``having
\emph{one} right angle''), this will not do in the case of the oblong,
which, unless it were stated that \emph{three} of its angles are right
angles, would not be sufficiently defined.

If it be objected, as it was by Todhunter for example, that the
definition of a square assumes more than is necessary, since it is
sufficient that, being equilateral, it should have one right angle,
the answer is that, as in other cases, the superfluity does not matter
from Euclid's point of view; on the contrary, the more of the
essential attributes of a thing that could be included in its
definition the better, provided that the existence of the thing
defined and its possession of all those attributes is proved before
the definition is actually used; and Euclid does this in the case of
the square by construction in \prop{1}{46}, making no use of the
definition before that proposition.

The word \emph{rhombus} (\greek{ῥόμβος}) is apparently derived from
\greek{ῥέμβω}, to \emph{turn round and} round, and meant among other
things a \emph{spinning-top}.  Archimedes uses the term \emph{solid
  rhombus} to denote a solid figure made up of two right cones with a
common circular base and vertices turned in opposite directions.  We
can of course easily imagine this solid generated by \emph{spinning};
and, if the cones were equal, the section through the common axis
would be a \emph{plane} rhombus, which would also be the
\emph{apparent} form of the spinning solid to the eye.  The difficulty
in the way of supposing the plane figure to have been named after the
solid figure is that in Archimedes the cones forming the solid are not
necessarily equal.  It is however possible that the solid to which the
name was originally given was made up of two equal cones, that the
plane rhombus then received its name from that solid, and that
Archimedes, in taking up the old name again, extended its
signification (cf.\ J.~H.~T. Müller, \emph{Beiträge zur Terminologie
  der griechischen Mathematiker}, i860, p.~20).  Proclus, while he
speaks of a rhombus as being like a shaken, i.e.\ deformed, square,
and of a rhomboid as an oblong that has been moved, tries to explain
the rhombus by reference to the appearance of a \emph{spinning} square
(\greek{τετράγωνον ῥομβούμενον}).

It is true that the definition of a rhomboid says more than is
necessary in describing it as having its opposite sides \emph{and
  angles} equal to one another.  The answer to the objection is the
same as the answer to the similar objection to the definition of a
square.

Euclid makes no use in the \emph{Elements} of the \emph{oblong}, the
\emph{rhombus} and the \emph{rhomboid}.  The explanation of his
inclusion of definitions of these figures is no doubt that they were
taken from earlier text-books.  From the words ``\emph{let}
quadrilaterals other than these \emph{be called} trapezia'' we may
perhaps infer that \emph{trapezium} was a new name or a new
application of an old name.

As Euclid has not yet defined parallel lines and does not anywhere
define a \emph{parallelogram}, he is not in a position to make the
more elaborate classification of quadrilaterals attributed by Proclus
to Posidonius and appearing also in Heron's Definitions.  It may be
shown by the following diagram, distinguishing seven species of
quadrilaterals.

Quadrilaterals

parallelograms

non-parallelograms

rectangular

non-rectangular

two sides parallel
(\emph{trapezium})

no sides parallel
(\emph{trapezoid})

\emph{square}

\emph{oblong}

\emph{rhombus}

\emph{rhomboid}

\emph{isosceles trapezium}

\emph{scalene trapezium}

It will be observed that, while Euclid in the above definition classes
as \emph{trapezia} all quadrilaterals other than squares, oblongs,
rhombi, and rhomboids, the word is in this classification restricted
to quadrilaterals having two sides (only) parallel, and
\emph{trapezoid} is used to denote the rest Euclid appears to have
used \emph{trapezium} in the restricted sense of a quadrilateral with
two sides parallel in his book, \greek{περὶ διαιρέσεων} (on divisions
of figures).  Archimedes uses it in the same sense, but in one place
describes it more precisely as a trapezium with its two sides
parallel.

\section*{Definition 23}

\greek{Παράλληλοί εισιν εὐθεῖαι, αἵτινες ἐν τῷ αὐτῷ ἐπιπέδῳ οὖσαι καὶ
  ἐκβαλλόμεναι εἰς ἄπειρον ἐφ’ ἑκάτερα τὰ μέρη ἐπὶ μηδέτερα
  συμπίπτουσιν ἀλλήλαις.}

\emph{\emph{Parallel} straight lines are straight lines which, being
  in the same plane and being produced indefinitely in both
  directions, do not meet one another in either direction.}

\greek{Παράλληλος} (alongside one another) written in one word does
not appear in Plato; but with Aristotle it was already a familiar
term.

\greek{εἰς ἄπειρον} cannot be translated ``to infinity'' because these
words might seem to suggest a \emph{region} or \emph{place} infinitely
distant, whereas \greek{εἰς ἄπειρον}, which seems to be used
indifferently with \greek{ἐπ’ ἄπειρον}, is adverbial, meaning
``without limit,'' i.e.\ ``indefinitely. '' Thus the expression is
used of a magnitude being ``infinitely divisible,'' or of a series of
terms extending without limit

\emph{In both directions}, \greek{ἐφ’ ἑκάτερα τὰ μέρη}, literally
``towards both the parts'' where ``parts'' must be used in the sense
of ``regions'' (cf.~Thuc.~\r2.~96).

It is clear that with Aristotle the general notion of parallels was
that of straight lines \emph{which do not meet}, as in Euclid: thus
Aristotle discusses the question whether to think that parallels do
meet should be called a geometrical or an ungeometrical error
(\emph{Anal.\ post.}\ \r1.~12, 77~b~22), and (more interesting still
in relation to Euclid) he observes that there is nothing surprising in
different hypotheses leading to the same error, as one might conclude
that parallels meet by starting from the assumption, either
(\emph{a})~that the interior (angle) is greater than the exterior, or
(\emph{b})~that the angles of a triangle make up more than two right
angles (\emph{Anal.\ prior.}\ \r2.~17, 66~a~11).

Another definition is attributed by Proclus to Posidonius, who said
that ``\emph{parallel lines are those which, (being) in one plane,
  neither converge nor diverge, but have all the perpendiculars equal
  which are drawn from the points of one line to the other}, while
such (straight lines) as make the perpendiculars less and less
continually do converge to one another; for the perpendicular is
enough to define (\greek{ὁρίζειν δύναται}) the heights of areas and
the distances between lines.  For this reason, when the perpendiculars
are equal, the distances between the straight lines are equal, but
when they become greater and less, the interval is lessened, and the
straight lines converge to one another in the direction in which the
less perpendiculars are'' (Proclus, p.~176, 6–17).

Posidonius' definition, with the explanation as to distances between
straight lines, their convergence and divergence, amounts to the
definition quoted by Simplicius (an-Nairīzī, p.~25, ed.\ Curtze) which
described straight lines as parallel \emph{if, when they are produced
  indefinitely both ways, the distance between them, or the
  perpendicular drawn from either of them to the other, is always
  equal and not different}.  To the objection that it should be
\emph{proved} that the distance between two parallel lines is the
perpendicular to them Simplicius replies that the definition will do
equally well if all mention of the \emph{perpendicular} be omitted and
it be merely stated that the \emph{distance} remains equal, although
``for \emph{proving} the matter in question it is necessary to say
that one straight line is perpendicular to both'' (an-Nairīzī,
ed.\ Besthorn-Heiberg, p.~9).  He then quotes the definition of ``the
philosopher Aganis'': ``\emph{Parallel straight lines are straight
  lines, situated in the same plane, the distance between which, if
  they are produced indefinitely in both directions at the same time,
  is everywhere the same}.''  (This definition forms the basis of the
attempt of ``Aganis'' to prove the Postulate of Parallels.)  On the
definition Simplicius remarks that the words ``situated in the same
plane'' are perhaps unnecessary, since, if the distance between the
lines is everywhere the same, and one does not incline at all towards
the other, they must for that reason be in the same plane.  He adds
that the ``distance'' referred to in the definition is the shortest
line which joins things disjoined. Thus, between point and point, the
distance is the straight line joining them; between a point and a
straight line or between a point and a plane it is the perpendicular
drawn from the point to the line or plane; ``as regards the distance
between two lines, that distance is, if the lines are parallel, one
and the same, equal to itself at all places on the lines, it is the
\emph{shortest} distance and, at all places on the lines,
perpendicular to both'' (\ibid~p, 10).

The same idea occurs in a quotation by Proclus (p.~177, 11) from
Geminus.  As part of a classification of lines which do not meet he
observes: ``Of lines which do not meet, some are in one plane with one
another, others not.  Of those which meet and are in one plane,
\emph{some are always the same distance from one another}, others
lessen the distance continually, as the hyperbola (approaches) the
straight line, and the conchoid the straight line (i.e.\ the asymptote
in each case).  For these, while the distance is being continually
lessened, are continually (in the position of) not meeting, though
they converge to one another; they never converge entirely, and this
is the most paradoxical theorem in geometry, since it shows that the
convergence of some lines is non-convergent.  But of lines which are
always an equal distance apart, those which are straight and never
make the (distance) between them smaller, and which are in one plane,
are parallel.''

Thus the \emph{equidistance}-theory of parallels (to which we shall
return) is very fully represented in antiquity.  I seem also to see
traces in Greek writers of a conception equivalent to the vicious
\emph{direction}-theory which has been adopted in so many modern
text-books.  Aristotle has an interesting, though obscure, allusion in
\emph{Anal.\ prior.}\ \r2.~16, 65~a~4 to a \emph{petitio principii}
committed by ``those who think that they draw parallels'' (or
``establish the theory of parallels,'' which is a possible translation
of \greek{τὰς παραλλήλους γράφειν}): ``for they unconsciously assume
such things as it is not possible to demonstrate if parallels do not
exist.''  It is clear from this that there was a vicious circle in the
then current theory of parallels; something which depended for its
truth on the properties of parallels was assumed in the actual proof
of those properties, e.g.\ that the three angles of a triangle make up
two right angles.  This is not the case in Euclid, and the passage
makes it clear that it was Euclid himself who got rid of the
\emph{petitio principii} in earlier text-books by formulating and
premising before \prop{1}{29} the famous Postulate~\ref{post:5},
which must ever be regarded as among the most epoch-making
achievements in the domain of geometry.  But one of the commentators
on Aristotle, Philoponus, has a note on the above passage purporting
to give the specific character of the \emph{petitio principii} alluded
to; and it is here that a \emph{direction}-theory of parallels may be
hinted at, whether Philoponus is or is not right in supposing that
this was what Aristotle had in mind.  Philoponus says: ``The same
thing is done by those who draw parallels, namely begging the original
question; for they will have it that it is possible to draw parallel
straight lines from the meridian circle, and they assume a point, so
to say, falling on the plane of that circle and thus they draw the
straight lines.  And what was sought is thereby assumed; for he who
does not admit the genesis of the parallels will not admit the point
referred to either.''  What is meant is, I think, somewhat as follows.
Given a straight line and a point through which a parallel to it is to
be drawn, we are to suppose the given straight line placed in the
plane of the meridian.  Then we are told to draw through the given
point another straight line in the plane of the meridian (strictly
speaking it should be drawn in a plane parallel to the plane of the
meridian, but the idea is that, compared with the size of the meridian
circle, the distance between the point and the straight line is
negligible); and this, as I read Philoponus, is supposed to be
equivalent to assuming a very distant point in the meridian plane and
joining the given point to it.  But obviously no ruler would stretch
to such a point, and the objector would say that we cannot really
direct a straight line to the assumed distant point except by drawing
it, without more ado, \emph{parallel} to the given straight line.  And
herein is the \emph{petitio principii}.  I am confirmed in seeing in
Philoponus an allusion to a \emph{direction}-theory by a remark of
Schotten on a similar reference to the meridian plane supposed to be
used by advocates of that theory.  Schotten is arguing that direction
is not in itself a conception such that you can predicate one
direction of two different lines. ``If any one should reply that
nevertheless many lines can be conceived \emph{which all have the
  direction from north to south},'' he replies that this represents
only a nominal, not a real, identity of direction.

Coming now to modern times, we may classify under three groups
practically all the different definitions that have been given of
parallels (Schotten, \emph{op.~cit.}\ \r2. p.~188 sqq.).

(1)~\emph{Parallel straight lines have no point common}, under which
general conception the following varieties of statement may be
included:

(\emph{a})~\emph{they do not cut one another},

(\emph{b})~\emph{they meet at infinity}, or

(\emph{c})~\emph{they have a common point at infinity}.

(2)~\emph{Parallel straight lines have the same, or like, direction or
  directions}, under which class of definitions must be included all
those which introduce transversals and say that the parallels
\emph{make equal angles with a transversal}.

(3)~\emph{Parallel straight lines have the distance between them
  constant}; with which group we may connect the attempt to explain a
parallel as \emph{the geometrical focus of all points which are
  equidistant from a straight line}.

But the three points of view have a good deal in common; some of them
lead easily to the others.  Thus the idea of the lines having no point
common led to the notion of their having a common point at infinity,
through the influence of modern geometry seeking to embrace different
cases under one conception; and then again the idea of the lines
having a common point at infinity might suggest their having the same
direction. The ``non-secant'' idea would also naturally lead to that
of equidistance~(3), since our observation shows that it is things
which come nearer to one another that tend to meet, and hence, if
lines are not to meet, the obvious thing is to see that they shall not
come nearer, i.e.\ shall remain the same distance apart.

We will now take the three groups in order.

(1)~The first observation of Schotten is that the varieties of this
group which regard parallels as (\emph{a})~meeting at infinity or
(\emph{b})~having a common point at infinity (first mentioned
apparently by Kepler, 1604, as a ``façon de parler'' and then used by
Desargues, 1639) are at least unsuitable definitions for elementary
text-books.  How do we know that the lines cut or meet at infinity?
We are not entitled to assume either that they do or that they do not,
because ``infinity'' is outside our field of observation and we cannot
verify either.  As Gauss says (letter to Schumacher), ``Finite man
cannot claim to be able to regard the infinite as something to be
grasped by means of ordinary methods of observation.'' Steiner, in
speaking of the rays passing through a point and successive points of
a straight line, observes that as the point of intersection gets
further away the ray moves continually in one and the same direction
(``nach einer und derselben Richtung hin'' ); only in one position,
that in which it is parallel to the straight line, ``there is \emph{no
  real cutting}'' between the ray and the straight line; what we have
to say is that the ray is ``\emph{directed towards the infinitely
  distant point on the straight line}.''  It is true that higher
geometry has to assume that the lines do meet at infinity: whether
such lines exist in nature or not does not matter (just as we deal
with ``straight lines'' although there is no such thing as a straight
line).  But if two lines do not cut at any finite distance, may not
the same thing be true at infinity also?  Are lines conceivable which
would not cut even at infinity but always remain at the same distance
from one another even there?  Take the case of a line of railway.
Must the two rails meet at infinity so that a train could not stand on
them there (whether we could \emph{see} it or not makes no
difference)?  It seems best therefore to leave to higher geometry the
conception of infinitely distant points on a line and of two straight
lines meeting at infinity, like \emph{imaginary} points of
intersection, and, for the purposes of elementary geometry, to rely on
the plain distinction between ``parallel'' and ``cutting'' which
average human intelligence can readily grasp.  This is the method
adopted by Euclid in his definition, which of course belongs to the
group (1)~of definitions regarding parallels as non-secant.

It is significant, I think, that such authorities as Ingrami
(\emph{Elementi di geometria}, 1904) and Enriques and Amaldi
(\emph{Elmenti di geometria}, 1905), after all the discussion of
principles that has taken place of late years, give definitions of
parallels equivalent to Euclid's: ``those straight lines in a plane
which have not any point in common are called parallels.''  Hilbert
adopts the same point of view, Veronese, it is true, takes a different
line.  In his great work \emph{Fondamenti di geometria}, 1891, he had
taken a ray to be parallel to another when a point at infinity on the
second is situated on the first; but he appears to have come to the
conclusion that this definition was unsuitable for his
\emph{Elementi}.  He avoids however giving the Euclidean definition of
parallels as ``straight lines in a plane which, though produced
indefinitely, never meet,'' because ``no one has ever seen two
straight lines of this sort,'' and because the postulate generally
used in connexion with this definition is not evident in the way that,
in the field of our experience, it is evident that only one straight
line can pass through two points.  Hence he gives a different
definition, for which he claims the advantage that it is independent
of the plane.  It is based on a definition of figures ``opposite to
one another with respect to a point'' (or \emph{reflex}
figures). ``Two figures are opposite to one another with respect to a
point~$O$, e.g.\ the figures $ABC$…and $A'B'C'$…, if to every point of
the one there corresponds one sole point of the other, and if the
segments $OA$, $OB$, $OC$, … joining the points of one figure to~$O$
are respectively equal and opposite to the segments $OA'$, $OB'$,
$OC'$, … joining to~$O$ the corresponding points of the second'':
then, a transversal of two straight lines being any segment having as
its extremities one point of one line and one point of the other,
``\emph{two straight lines are called parallel if one of them contains
  two points opposite to two points of the other with respect to the
  middle point of a common transversal}'' It is true, as Veronese
says, that the parallels so defined and the parallels of Euclid are in
substance the same; but it can hardly be said that the definition
gives as good an idea of the essential nature of parallels as does
Euclid's.  Veronese has to \emph{prove}, of course, that his parallels
have no point in common, and his ``Postulate of Parallels'' can hardly
be called more evident than Euclid's: ``If two straight lines are
parallel, they are figures opposite to one another with respect to the
middle points of all their transversal segments.''

(2)~The \emph{direction}-theory.

The fallacy of this theory has nowhere been more completely exposed
than by C.~L. Dodgson (\emph{Euclid and his modern Rivals}, 1879).
According to Killing (\emph{Einführung in die Grundlagen der
  Geometrie}, \r1.~p.~5) it would appear to have originated with no
less a person than Leibniz.  In the text-books which employ this
method the notion of \emph{direction} appears to be regarded as a
primary, not a derivative notion, since no definition is given.  But
we ought at least to know how the same direction or like directions
can be recognised when two different straight lines are in question.
But no answer to this question is forthcoming.  The fact is that the
whole idea as applied to non-coincident straight lines is derived from
knowledge of the properties of \emph{parallels}; it is a case of
explaining a thing by itself.  The idea of parallels being in the same
direction perhaps arose from the conception of an angle as a
\emph{difference} of direction (the hollowness of which has already
been exposed); sameness of direction for parallels follows from the
same ``difference of direction'' which both exhibit relatively to a
third line.  But this is not enough.  As Gauss said (\emph{Werke},
\r4.\ p.~365), ``If it [identity of direction] is recognised by the
equality of the angles formed with one third straight line, we do not
yet know without an antecedent proof whether this same equality will
also be found in the angles formed with a fourth straight line ``(and
any number of other transversals); and in order to make this theory of
parallels valid, so far from getting rid of axioms such as Euclid's,
you would have to assume as an axiom what is much less axiomatic,
namely that ``straight lines which make equal corresponding angles
with a certain transversal do so with \emph{any} transversal
``(Dodgson, p.~101).

(3)~In modern times the conception of parallels as \emph{equidistant}
straight lines was practically adopted by Clavius (the editor of
Euclid, born at Bamberg, 1537) and (according to Saccheri) by Borelli
(\emph{Euclides restitutus}, 1658) although they do not seem to have
\emph{defined} parallels in this way.  Saccheri points out that,
before such a definition can be used, it has to be \emph{proved} that
``the geometrical locus of points equidistant from a straight line is
a straight line.''  To do him justice, Clavius saw this and tried to
prove it: he makes out that the locus is a straight line according to
the definition of Euclid, because ``it lies evenly with respect to all
the points on it''; but there is a confusion here, because such
``evenness'' as the locus has is with respect to the straight line
from which its points are equidistant, and there is nothing to show
that it possesses this property with respect to itself.  In fact the
theorem cannot be proved without a postulate.

\section*{Postulate 1.}

\greek{Ἠιτήσθω ἀπὸ παντὸς συμείου ἐπὶ πᾶν σημεῖον εὐθεῖαν γραμμὴν
  ἀγαγεῖν.}

\emph{Let the following be postulated: to draw a straight line from
  any point to any point.}

\emph{From any point to any point.}  In general statements of this
kind the Greeks did not say, as we do, ``\emph{any} point,''
``\emph{any} triangle'' etc., but ``\emph{every} point,''
``\emph{every} triangle'' and the like.  Thus the words are here
literally ``from every point to every point.''  Similarly the first
words of Postulate~\ref{post:3} are ``with \emph{every} centre and
distance,'' and the enunciation, e.g., of \prop{1}{18} is ``In
\emph{every} triangle the greater side subtends the greater angle.''

It will be remembered that, according to Aristotle, the geometer must
in general assume \emph{what} a thing is, or its definition, but must
prove \emph{that} it is, i.e.\ the \emph{existence} of the thing
corresponding to the definition: only in the case of the two most
primary things, points and lines, does he assume, without proof, both
the definition and the existence of the thing defined.  Euclid has
indeed no separate assumption affirming the existence of points such
as we find nowadays in text-books like those of Veronese, Ingrami,
Enriques, ``there exist distinct points'' or ``there exist an infinite
number of points.''  But, as regards the only lines dealt with in the
\emph{Elements}, straight lines and circles, existence is asserted in
Postulates \ref{post:1} and~\ref{post:3} respectively.
Postulate~\ref{post:1} however does much more than (1)~postulate the
existence of straight lines.  It is (2)~an answer to a possible
objector who should say that you cannot, with the imperfect
instruments at your disposal, draw a mathematical straight line at
all, and consequently (in the words of Aristotle,
\emph{Anal.\ post.}\ \r1.~10, 76~b~41) that the geometer uses false
hypotheses, since he calls a line a foot long when it is not or
straight when it is not straight.  It would seem (if Gherard's
translation is right) that an-Nairīzī saw that one purpose of the
Postulate was to refute this criticism: ``the utility of the first
three postulates is (to ensure) that the weakness of our equipment
shall not prevent (scientific) demonstration'' (ed.\ Curtze, p.~30).
The fact is, as Aristotle says, that the geometer's demonstration is
not concerned with the particular imperfect straight line which he has
drawn, but with the ideal straight line of which it is the imperfect
representation.  Simplicius too indicates that the object of the
Postulate is rather to enable the drawing of a mathematical straight
line to be imagined than to assert that it can actually be realised in
practice: ``he would be a rash person who, taking things as they
actually are, should postulate the drawing of a straight line from
Aries to Libra.''

There is still something more that must be inferred from the Postulate
combined with the definition of a straight line, namely (3)~that the
straight line joining two points is \emph{unique}: in other words
that, \emph{if two straight lines} (``rectilineal segments,'' as
Veronese would call them) \emph{have the same extremities, they must
  coincide throughout their length}.  The omission of Euclid to state
this in so many words, though he assumes it in \prop{1}{4}, is no
doubt answerable for the interpolation in the text of the equivalent
assumption that two straight lines cannot enclose a space, which has
constantly appeared in \textsc{mss.}\ and editions of Euclid, either
among Axioms or Postulates.  That Postulate~\ref{post:1} included it,
by conscious implication, is even clear from Proclus' words in his
note on \prop{1}{4} (p.~239, 16): ``therefore two straight lines do
not enclose a space, and it was with knowledge of this fact that the
writer of the Elements said in the first of his Postulates, \emph{to
  draw a straight line from any point to any point}, implying that it
is \emph{one} straight line which would always join the two points,
not \emph{two}.''

Proclus attempts in the same note (p.~239) to \emph{prove} that two
straight lines cannot enclose a space, using as his basis the
definition of the diameter of a circle and the theorem, stated in it,
that any diameter divides the circle into two equal parts.

\sidefig{post_1a}

Suppose, he says, $ACB$, $ADB$ to be two straight lines enclosing a
space.  Produce them (beyond~$B$) indefinitely.  With centre $B$ and
distance $AB$ describe a circle, cutting the lines so produced in $F$,
$E$ respectively.

Then, since $ACBF$, $ADBE$ are both diameters
cutting off semi-circles, the arcs $AE$, $AEF$ are equal:
which is impossible. Therefore etc.

It will be observed, however, that the straight lines produced are
assumed to meet the circle given in two \emph{different} points $E$,
$F$, whereas, for anything we know, $E$, $F$ might coincide and the
straight lines have \emph{three} common points.  The proof is
therefore delusive.

\sidefig{post_1b}

Saccheri gives a different proof. From Euclid's definition of a
straight line as that which lies evenly with its points he infers
that, when such a line is turned about its two extremities, which
remain fixed, all the points on it must remain throughout in the same
position, and cannot take up different positions as the revolution
proceeds. ``In this view of the straight line the truth of the
assertion that two straight lines do not enclose a space is obviously
involved.  In fact, if two lines are given which enclose a space, and
of which the two points $A$ and~$X$ are the common extremities, it is
easily shown that neither, or else only one, of the two lines is
straight.''

It is however better to assume as a \emph{postulate} the fact,
inseparably connected with the idea of a straight line, that
\emph{there exists only one straight line containing two given
  points}, or, \emph{if two straight lines have two points in common,
  they coincide throughout}.

\section*{Postulate 2}

\greek{Καὶ πεπερασμένην εὐθεῖαν κατὰ τὸ συνεχὲς ἐπ’ εὐθείας ἐκβαλεῖν.}

\emph{To produce a finite straight line continuously in a straight line.}

I translate \greek{πεπερασμένην} by \emph{finite}, because that is the
received equivalent, and because any alternative word such as
\emph{limited}, \emph{terminated}, if applied to a straight line,
would equally fail to express what modern Italian geometers aptly call
a \emph{rectilineal segment}, that is, a straight line having
\emph{two} extremities.

Just as Post.~\ref{post:1} asserting the possibility of drawing a
straight line from any one point to another must be held to declare at
the same time that the straight line so drawn is unique, so
Post.~\ref{post:2} maintaining the possibility of producing a finite
straight line (a ``rectilineal segment'') continuously in a straight
line must also be held to assert that the straight line can only be
produced \emph{in one way} at either end, or that the produced part in
either direction is \emph{unique}; in other words, that \emph{two
  straight lines cannot have a common segment}.  This latter
assumption is not expressly appealed to by Euclid until \prop{11}{1}.
But it is needed at the very beginning of Book~\r1.  Proclus
(p.~214, 18) says that Zeno of Sidon, an Epicurean, maintained that
the very first proposition \prop{1}{1} requires it to be admitted that
``two straight lines cannot have the same segments''; otherwise $AC$,
$BC$ might meet before they arrive at~$C$ and have the rest of their
length common, in which case the actual triangle formed by them and
$AB$ would not be equilateral.  The assumption that two straight lines
cannot have a common segment is certainly necessary in \prop{1}{4},
where one side of one triangle is placed on that side of the other
triangle which is equal to it, and it is inferred that the two
coincide throughout their length: this would by no means follow if two
straight lines could have a common segment. Proclus (p.~215, 24),
while observing that Post.~\ref{post:2} clearly indicates that the
produced portion must be \emph{one}, attempts to prove it, but
\sidefig{post_2a}
unsuccessfully.  Both he and Simplicius practically use the same
argument.  Suppose, says Proclus, that the straight lines $AC$, $AD$
have $AB$ as a common segment.  With centre~$B$ and radius~$BA$
describe a circle (Post.~\ref{post:3}) meeting $AC$, $AD$ in $C$, $D$.
Then, since $ABC$ is a straight line through the centre, $AEC$ is a
semi-circle.  Similarly, $ABD$ being a straight line through the
centre, $AED$ is a semi-circle.  Therefore $AEC$ is equal to $AED$:
which is impossible.

Proclus observes that Zeno would object to this proof as really
depending on the assumption that ``two circumferences (of circles)
cannot have one portion common''; for this, he would say, is assumed
in the common proof by superposition of the fact that a circle is
bisected by a diameter, since that proof takes it for granted that, if
one part of the circumference cut off by the diameter, when applied to
the other, does not coincide with it, it must necessarily fall either
\emph{entirely} outside or \emph{entirely} inside it, whereas there is
nothing to prevent their coinciding, not altogether, but in part only;
and, until you really prove the bisection of a circle by its diameter,
the above proof is not valid.  Posidonius is represented as having
derided Zeno for not seeing that the proof of the bisection of a
circle by its diameter goes on just as well if the circumferences fail
to coincide \emph{in part} only.  But the true objection to the proof
above given is that the proof of the bisection of a circle by any
diameter \emph{itself} assumes that two straight lines cannot have a
common segment; for, if we wish to draw the diameter of a circle which
has its extremity at a given point of the circumference we have to
join the latter point to the centre (Post.~\ref{post:1}) and then to
\emph{produce} the straight line so drawn till it meets the circle
again (Post.~\ref{post:2}), and it is necessary for the proof that the
produced part shall be \emph{unique}.

Saccheri adopted the proper order when he gave, first the proposition
that two straight lines cannot have a common segment, and after that
the proposition that any diameter of a circle bisects the circle and
its circumference.

Saccheri's proof of the former is very interesting as showing the
thoroughness of his method, if not at the end entirely convincing.  It
is in five stages which I shall indicate shortly, giving the full
argument of the first only.

\sidefig{post_2b}

Suppose, if possible, that $AX$ is a common segment of both the
straight lines $AXB$, $AXC$, in one plane, produced beyond~$X$.  Then
describe about $X$ as centre, with radius $XB$ or $XC$, the arc $BMC$,
and draw through~$X$ to any point on it the straight line~$XM$.

(i)~I maintain that, with the assumption made, \emph{the line $AXM$ is
  also a straight line which is drawn from the point~$A$ to the
  point~$X$ and produced beyond~$X$}.

For, if this line were not straight, we could draw another straight
line $AM$ which for its part would be straight.  This straight line
will either (\emph{a})~cut one of the two straight lines $XB$, $XC$ in
a certain point~$K$ or (\emph{b}) enclose one of them, for instance
$XB$, in the area bounded by $AX$, $XM$ and $APLM$.

But the first alternative (\emph{a}) obviously contradicts the
foregoing lemma [that two straight lines cannot enclose a space],
since in that case the two lines $AXK$, $ATK$, which by hypothesis are
straight, would enclose a space.

The second possibility (\emph{b}) is at once seen to involve a similar
absurdity.  For the straight line $XB$ must, when produced beyond~$B$,
ultimately meet $APLM$ in a point~$L$.  Consequently the two lines
$AXBL$, $APL$, which by hypothesis are straight, would again enclose a
space.  If however we were to assume that the straight line $XB$
produced beyond~$B$ will ultimately meet either the straight line $XM$
or the straight line $XA$ in another point, we should in the same way
arrive at a contradiction.

From this it obviously follows that, on the assumption made, the line
$AXM$ is itself the straight line which was drawn from the point~$A$
to the point~$M$; and that is what was maintained.

The remaining stages are in substance these.

(ii)~\emph{If the straight line $AXB$, regarded as rigid, revolves
  about $AX$ as axis, it cannot assume two more positions in the same
  plane, so that, for example, in one position $XB$ should coincide
  with $XC$, and in the other-with $XM$}

[This is proved by considerations of symmetry.  $AXB$ cannot be
  altogether ``similar or equal to'' $AXC$, if viewed from the same
  side (left or right) of both: otherwise they would coincide, which
  by hypothesis they do not.  But there is nothing to prevent $AXB$
  viewed from one side (say the left) being ``similar or equal to
  ``$AXC$ viewed from the other side (i.e.\ the right), so that $AXB$
  \emph{can}, without any change, be brought into the position~$AXC$.

$AXB$ cannot however take the position of the other straight line
  $AXM$ as well.  If they were like on one side, they would coincide;
  if they were like on opposite sides, $AXM$, $AXC$ would be like on
  the same side and therefore coincide.]

(iii)~The other positions of $AXB$ during the revolution must be above
or below the original plane.

(iv)~It is next maintained that \emph{there \emph{is} a point~$D$ on
  the arc~$BC$ such that, if $XD$ is drawn, $AXD$ is not only a
  straight line but is such that viewed from the left side it is
  exactly ``similar or equal'' to what it is when viewed from the
  right side}.

[\emph{First}, it is proved that points $M$, $F$ can be found on the
  arc, corresponding in the same way as $B$, $C$ do, but nearer
  together, and of course $AXM$, $AXF$ are both straight lines.

\emph{Secondly}, similar corresponding points can be found still
nearer together, and so on continually, until either (\emph{a})~we
come to one point~$D$ such that $AXD$ is exactly like itself when the
right and left sides are compared, or (\emph{b})~there are two
ultimate points of this sort $M$, $F$, so that both $AXM$, $AXF$ have
this property.

\emph{Thirdly}, (\emph{b}) is ruled out by reference to the definition
of a straight line.

Hence (\emph{a}) only is true, and there is only \emph{one} point~$D$
such as described.]

(v)~Lastly, Saccheri concludes that the straight line $AXD$ so
determined ``is \emph{alone} a straight line, and the \emph{immediate}
prolongation from $A$ beyond $X$ to~$D$,'' relying again on the
definition of a straight line as ``lying evenly.''

Simson deduced the proposition that \emph{two straight lines cannot
  have a common segment} as a corollary from \prop{1}{11}; but his
argument is a complete \emph{petitio principii}, as shown by Todhunter
in his note on that proposition.

Proclus (p.~217, 10) records an ancient proof also based on the
proposition \prop{1}{11}.  Zeno, he says, propounded this proof and
then criticised it.

Suppose that two straight lines $AC$, $AD$ have a common segment $AB$,
and let $BE$ be drawn at right angles to~$AC$.

\sidefig{post_2c}

Then the angle $EBC$ is right.

If then the angle $EBD$ is also right, the two angles will be equal;
which is impossible.

If the angle $EBD$ is not right, draw $BF$ at right angles to~$AD$;
therefore the angle $FBA$ is right.

But the angle $EBA$ is right.

Therefore the angles $EBA$, $FBA$ are equal: which is impossible.

Zeno objected to this, says Proclus, because it assumed the later
proposition \prop{1}{11} for its proof.  Posidonius said that there
was no trace of such a proof to be found in the text-books of
Elements, and that it was only invented by Zeno for the purpose of
slandering contemporary geometers.  Posidonius maintains further that
even this proof has something to be said for it.  There must be some
straight line at right angles to each of the two straight lines $AC$,
$AD$ (the very definition of right angles assumes this):
``\emph{suppose then it happens to be the straight line we have set
  up}.''  Here then we have an ancient instance of a defence of
\emph{hypothetical construction}, but in such apologetic terms (``it
is possible to say something even for this proof'') that we may
conclude that in general it would not have been accepted by geometers
of that time as a legitimate means of proving a proposition.

Todhunter proposed to deduce \emph{two straight lines cannot have a
  common segment} from \prop{1}{13}.  But this will not serve either,
since, as before mentioned, the assumption is really required for
\prop{1}{4}.

It is best to make it a postulate.

\section*{Postulate 3}

\greek{Καὶ παντὶ κέντρῳ καὶ διαστήματι κύκλον γράφεσθαι.}

\emph{To describe a circle with any centre and distance.}

In this case Euclid's text has the passive of the verb: ``a circle can
be drawn'' ; Proclus however has the active (\greek{γράφαι}) as Euclid
has in the first two Postulates.

\emph{Distance}, \greek{διαστήματι}.  This word, meaning ``distance''
quite generally (cf.\ Arist.\ \emph{Metaph.}\ 1055~a~9 ``it is between
extremities that distance is greatest,'' \ibid~1056~a~36 ``things
which have something between them, that is, a certain distance''), and
also ``distance'' in the sense of ``dimension'' (as in ``space has
three dimensions, length, breadth and depth,'' Arist.\ \emph{Physics}
\r4.~1, 209~a~4), was the regular word used for describing a circle
with a certain \emph{radius}, the idea being that each point of the
circumference was at that \emph{distance} from the centre
(cf.\ Arist.\ \emph{Meteorologica} \r3.~5, 376~b~8: ``if a circle be
drawn…with distance~\greek{ΜΠ}'').  The Greeks had no word
corresponding to \emph{radius}: if they had to express it, they said
``(straight lines) drawn from the centre'' (\greek{αἱ ἐκ τοῦ κέντρου},
Eucl.\ \book{3}~Def.~1 and \hyperref[prop_III_26]{Prop.~26});
\emph{Meteorologica} \r2.~5, 362~b~1 has the full phrase \greek{αἱ ἐκ
  τοῦ κέντρου ἀγόμεναι γραμμαί}).

Mr~Frankland observes that it would be remarkable if, unlike
Postulates~\ref{post:1} and~\ref{post:2}, this Postulate implied
\emph{merely} what it says, that a circle can be drawn with any centre
and distance.  We may regard it, if we please, as helping to the
complete delineation of the Space which Euclid's geometry is to
investigate formally.  The Postulate has the effect of removing any
restriction upon the size of the circle.  It may (1)~be indefinitely
small, and this implies that space is \emph{continuous}, not discrete,
with an irreducible minimum distance between contiguous points in it.
(2)~The circle may be indefinitely large, which implies the
fundamental hypothesis of \emph{infinitude} of space.  This last
assumed characteristic of space is essential to the proof of
\prop{1}{16}, a theorem not universally valid in a space which is
unbounded in extent but finite in size.  It would however be unsafe to
suppose that Euclid foresaw the use to which his Postulate might thus
be put, or formulated it with such an intention.

\section*{Postulate 4}

\greek{Καὶ πάσας τὰς ὀρθὰς γωνίας ἴσας ἀλλήλαις εἶναι.}

\emph{That all right angles are equal to one another.}

While this Postulate asserts the essential truth that a right angle is
a \emph{determinate magnitude} so that it really serves as an
invariable standard by which other (acute and obtuse) angles may be
measured, much more than this is implied, as will easily be seen from
the following consideration.  If the statement is to be \emph{proved},
it can only be proved by the method of applying one pair of right
angles to another and so arguing their equality.  But this method
would not be valid unless on the assumption of the \emph{invariability
  of figures}, which would therefore have to be asserted as an
antecedent postulate.  Euclid preferred to assert as a postulate,
directly, the fact that all right angles are equal; and hence his
postulate must be taken as equivalent to the principle of
\emph{invariability of figures} or its equivalent, the
\emph{homogeneity of space}.

According to Proclus, Geminus held that this Postulate should not be
classed as a postulate but as an axiom, since it does not, like the
first three Postulates, assert the possibility of \emph{some}
construction but expresses an essential property of right angles.
Proclus further observes (p.~188,~8) that it is not a postulate in
Aristotle's sense either.  (In this I think he is wrong, as explained
above.)  Proclus himself, while regarding the assumption as axiomatic
(``the equality of right angles suggests itself even hy virtue of our
common notions''), is prepared with a proof, if such is asked for.

\sidefig{post_4a}

Let $ABC$, $DEF$ be two right angles.

If they are not equal, one of them must be the greater, say $ABC$.

Then, if we apply $DE$ to $AB$, $EF$ will fall within $ABC$, as~$BG$.

Produce $CB$ to~$H$.  Then, since $ABC$ is a right angle, so is $ABH$,
and the two angles are equal (a right angle being by definition equal
to its adjacent angle).

Therefore the angle $ABH$ is \emph{greater} than the angle $ABG$.

Producing $GB$ to~$K$, we have similarly the two angles $ABK$, $ABG$
both right and equal to one another; whence the angle $ABH$ is
\emph{less} than the angle~$ABG$.

But it is also greater: which is impossible.

Therefore etc.

A defect in this proof is the assumption that $CB$, $GB$ can each be
produced only in one way, and that $BK$ falls outside the angle~$ABH$.

\sidefig{post_4b}

Saccheri's proof is more careful in that he premises a third lemma in
addition to those asserting (1)~that two straight lines cannot enclose
a space and (2)~that two straight lines cannot have a common segment.
The third lemma is: \emph{If two straight lines $AB$, $CXD$ meet one
  another at an intermediate point~$X$, they do not \emph{touch} at
  that point, but \emph{cut one another}}.

Suppose now that $DA$ standing on $BAC$ makes the two angles $DAB$,
$DAC$ equal, so that each is a right angle by the definition.

Similarly, let $LH$ form with the straight line $FHM$ the right angles
$LHF$, $LHM$.

\sidefig{post_4c}

Let $DA$, $HL$ be equal; and suppose the whole of the second figure so
laid upon the first that the point $H$ falls on~$A$, and $L$ on~$D$.

Then the straight line $FHM$ will (by the third lemma) not
\emph{touch} the straight line $BC$ at~$A$; it will either

(\emph{a})~coincide exactly with $BC$, or

(\emph{b})~\emph{cut} it so that one of its extremities, as~$F$, will
fall above [$BC$] and the other, $M$, below it.

If the alternative (\emph{a}) is true, we have already proved the
exact equality of all rectilineal right angles.

Under alternative (\emph{b}) we prove that the angle $LHF$, being
equal to the angle $DAF$, is less than the angle $DAB$ or~$DAC$, and
\emph{a fortiori} less than the angle $DAM$ or $LHM$; which is
contrary to the hypothesis.

[Hence (\emph{a}) is the only possible alternative, so that all right
  angles are equal.]

Saccheri adds that it makes no difference if the. angle $DAF$ diverges
\emph{infinitely little} from the angle~$DAB$.  This would equally
lead to a conclusion contradicting the hypothesis.

\sidefig{post_4d}

It will be observed that Saccheri speaks of ``the exact equality of
all rectilineal right angles.''  He may have had in mind the remark of
Pappus, quoted by Proclus (p.~189,~11), that the converse of this
postulate, namely that an angle which is equal to a right angle is
also right, is not necessarily true, unless the former angle is
rectilineal.  Suppose two equal straight lines $BA$, $BC$ at right
angles to one another, and semi-circles described on $BA$, $BC$
respectively as $AEB$, $BDC$ in the figure.  Then, since the
semi-circles are equal, they coincide if applied to one another.
Hence the ``angles'' $EBA$, $DBC$ are equal.  Add to each the
``angle'' $ABD$; and it follows that the \emph{lunular} angle $EBD$ is
equal to the right angle $ABC$.  (Similarly, if $BA$, $BC$ be inclined
at an acute or obtuse angle, instead of at a right angle, we find a
\emph{lunular} angle equal to an acute or obtuse angle.)  This is one
of the curiosities which Greek commentators delighted in.

Veronese, Ingrami, and Enriques and Amaldi deduce the fact that
\emph{all right angles are equal} from the equivalent fact that all
flat angles are equal, which is either itself assumed as a postulate
or immediately deduced from some other postulate.

Hilbert takes quite a different line.  He considers that Euclid did
wrong in placing Post.~\ref{post:4} among ``axioms.''  He himself,
after his Group~\r3.\ of Axioms containing six relating to congruence,
proves several theorems about the congruence of triangles and angles,
and then deduces our Postulate.

As to the \emph{raison d'être} and the place of Post.~\ref{post:4} one
thing is quite certain.  It was essential from Euclid's point of view
that it should come before Post.~\ref{post:5}, since the condition in
the latter that a certain pair of angles are together less than two
right angles would be useless unless it were first made clear that
right angles are angles of determinate and invariable magnitude.

\section*{Postulate 5}

\greek{Καὶ ἐὰν εἰς δύο εὐθείας εὐθεῖα ἐμπίπτουσα τὰς ἐντὸς καὶ ἐπὶ τὰ
  αὐτὰ μέρη γωνίας δύο ὀρθῶν ἐλάσσονας ποιῇ, ἐκβαλλομένας τὰς δύο
  εὐθείας ἐπ’ ἅπειρον συμπίπτειν, ἐφ’ ἃ μέρη εἰσὶν αἱ τῶν δύο ὀρθῶν
  ἐλάσσονες.}

\emph{That, if a straight line falling on two straight lines make the
  interior angles on the same side less than two right angles, the two
  straight lines, if produeed indefinitely, meet on that side on whieh
  are the angles less than the two right angles.}

Although Aristotle gives a clear idea of what he understood by a
\emph{postulate}, he does not give any instances from geometry; still
less has he any allusion recalling the particular postulates found in
Euclid.  We naturally infer that the formulation of these postulates
was Euclid's own work.  There is a more positive indication of the
originality of Postulate~\ref{post:5}, since in the passage
(\emph{Anal.\ prior.}\ \r2.~16, 65~a~4) quoted above in the note on
the definition of parallels he alludes to some \emph{petitio
  principii} involved in the theory of parallels current in his time.
This reproach was removed by Euclid when he laid down this
epoch-making Postulate.  When we consider the countless successive
attempts made through more than twenty centuries to prove the
Postulate, many of them by geometers of ability, we cannot but admire
the genius of the man who concluded that such a hypothesis, which he
found necessary to the validity of his whole system of geometry, was
really indemonstrable.

From the very beginning, as we know from Proclus, the Postulate was
attacked as such, and attempts were made to prove it as a theorem or
to get rid of it by adopting some other definition of parallels; while
in modern times the literature of the subject is enormous.  Riccardi
(\emph{Saggio di una bibliografia Euclidea}, Part~\r4., Bologna, 1890)
has twenty quarto pages of titles of monographs relating to
Post.~\ref{post:5} between the dates 1607 and 1887.  Max Simon
(\emph{Ueber die Entwicklung der Elementar-geometrie im
  XIX.\ Jakrhundert}, 1906) notes that he has seen three new attempts,
as late as 1891 (a century after Gauss laid the foundation of
non-Euclidean geometry), to prove the theory of parallels
independently of the Postulate.  Max Simon himself (pp.~53–61) gives a
large number of references to books or articles on the subject and
refers to the copious information, as to contents as well as names,
contained in Schotten's \emph{Inhalt und Methode des planimetrischen
  Unterrichts}, \r2.~pp.~183–332.

This note will include some account of or allusion to a few of the
most noteworthy attempts to prove the Postulate.  Only those of
ancient times, as being less generally accessible, will be described
at any length; shorter references must suffice in the case of the
modern geometers who have made the most important contributions to the
discussion of the Postulate and have thereby, in particular,
contributed most towards the foundation of the non-Euclidean
geometries, and here I shall make use principally of the valuable
Article~8, \emph{Sulla teoria delle parallel e sulle geometrie
  non-euclidee} (by Roberto Bonola), in \emph{Questioni riguardanti le
  matematiche elementari}, \r1.~pp.~247–363.

Proclus (p.~191, 21~sqq.)\ states very clearly the nature of the first
objections taken to the Postulate.

``This ought even to be struck out of the Postulates altogether; for
it is a theorem involving many difficulties, which Ptolemy, in a
certain book, set himself to solve, and it requires for the
demonstration of it a number of definitions as well as theorems.  And
the converse of it is actually proved by Euclid himself as a theorem.
It may be that some would be deceived and would think it proper to
place even the assumption in question among the postulates as
affording, in the lessening of the two right angles, ground for an
instantaneous belief that the straight lines converge and meet.  To
such as these Geminus correctly replied that we have learned from the
very pioneers of this science not to have any regard to mere plausible
imaginings when it is a question of the reasonings to be included in
our geometrical doctrine.  For Aristotle says that it is as
justifiable to ask scientific proofs of a rhetorician as to accept
mere plausibilities from a geometer; and Simmias is made by Plato to
say that he recognises as quacks those who fashion for themselves
proofs from probabilities.  So in this case the fact that, when the
right angles are lessened, the straight lines converge is true and
necessary; but the statement that, since they converge more and more
as they are produced, they will sometime meet is plausible but not
necessary, in the absence of some argument showing that this is true
in the case of straight lines.  For the fact that some lines exist
which approach indefinitely, but yet remain non-secant
(\greek{ἀσύμπτωτοι}), although it seems improbable and paradoxical, is
nevertheless true and fully ascertained with regard to other species
of lines.  May not then the same thing be possible in the case of
straight lines which happens in the case of the lines referred to?
Indeed, until the statement in the Postulate is clinched by proof, the
facts shown in the case of other lines may direct our imagination the
opposite way.  And, though the controversial arguments against the
meeting of the straight lines should contain much that is surprising,
is there not all the more reason why we should expel from our body of
doctrine this merely plausible and unreasoned (hypothesis)?

``It is then clear from this that we must seek a proof of the present
theorem, and that it is alien to the special character of postulates.
But how it should be proved, and by what sort of arguments the
objections taken to it should be removed, we must explain at the point
where the writer of the Elements is actually about to recall it and
use it as obvious.  It will be necessary at that stage to show that
its obvious character does not appear independently of proof, but is
turned by proof into matter of knowledge.''

Before passing to the attempts of Ptolemy and Proclus to prove the
Postulate, I should note here that Simplicius says (in an-Nairīzī,
ed.\ Besthorn-Heiberg, p.~119, ed.\ Curtze, p.~65) that this Postulate
is by no means manifest, but requires proof, and accordingly
``Abthiniathus'' and Diodorus had already proved it by means of many
different propositions, while Ptolemy also had explained and proved
it, using for the purpose Eucl.\ \prop{1}{13}, \prop*{1}{15} and
\prop*{1}{16} (or \prop*{1}{18}).  The Diodorus here mentioned may be
the author of the \emph{Analemma} on which Pappus wrote a commentary.
It is difficult even to frame a conjecture as to who ``Abthiniathus''
is.  In one place in the Arabic text the name appears to be written
``Anthisathus ``(H.~Suter in \emph{Zeitschrift für Math.\ und Physik},
\r39., hist.\ litt.\ Abth.\ p.~194).  It has occurred to me whether he
might be Peithon, a friend of Serenus of Antinoeia (Antinoupolis) who
was long known as Serenus of \emph{Antissa}.  Serenus says (\emph{De
  sectione cylindri}, ed.\ Heiberg, p.~96): ``Peithon the geometer,
explaining parallels in a work of his, was not satisfied with what
Euclid said, but showed their nature more cleverly by an example; for
he says that parallel straight lines are such a thing as we see on
walls or on the ground in the shadows of pillars which are made when
either a torch or a lamp is burning behind them.  And, although this
has only been matter of merriment to every one, I at least must not
deride it, for the respect I have for the author, who is my friend.''
If Peithon was known as ``of Antinoeia'' or ``of Antissa,'' the two
forms of the mysterious name might perhaps be an attempt at an
equivalent; but this is no more than a guess.

Simplicity adds in full and word for word the attempt of his
``friend'' or his ``master Aganis'' to prove the Postulate.

Proclus returns to the subject (p.~365,~5) in his note on
Eucl.\ \prop{1}{29}.  He says that before his time a certain number of
geometers had classed as a theorem this Euclidean postulate and
thought it matter for proof, and he then proceeds to give an account
of Ptolemy's argument.

\subsubsection*{Noteworthy attempts to prove the Postulate}

\subsubsection{Ptolemy}

We learn from Proclus (p.~365, 7–11) that Ptolemy wrote a book on the
proposition that ``straight lines drawn from angles less than two
right angles meet if produced,'' and that he used in his ``proof''
many of the theorems in Euclid preceding \prop{1}{19}.  Proclus
excuses himself from reproducing the early part of Ptolemy's argument,
only mentioning as one of the propositions proved in it the theorem of
Eucl.\ \prop{1}{28} that, if two straight lines meeting a transversal
make the two interior angles on the same side equal to two right
angles, the straight lines do not meet, however far produced.

I.~From Proclus' note on \prop{1}{28} (p.~362, 14~sq.)\ we know that
Ptolemy proved this somewhat as follows.

Suppose that there are two straight lines $AB$, $CD$, and that $EFGH$,
meeting them, makes the angles $BFG$, $FGD$ equal to two right angles.
I say that $AB$, $CD$ are parallel, that is, they are non-secant.

\sidefig{post_5a}

For, if possible, let $FB$, $GD$ meet at~$K$.

Now, since the angles $BFG$, $FGD$ are
equal to two right angles, while the four
angles $AFG$, $BFG$, $FGD$, $FGC$ are together
equal to four right angles,

the angles $AFG$, $FGC$ are equal to two
right angles.

``\emph{If therefore $FB$, $GD$, when the interior angles are equal to
  two right angles, meet at~$K$, the straight lines $FA$, $GC$ will
  also meet if produced}; for the angles $AFG$, $CGF$ are also equal
to two right angles.

``Therefore the straight lines will either meet in both directions or
in neither direction, if the two pairs of interior angles are both
equal to two right angles.

``Let, then, $FA$, $GC$ meet at~$L$.

``Therefore the straight lines $LABK$, $LCDK$ enclose a space; which
is impossible.

``Therefore it is not possible for two straight lines to meet when the
interior angles are equal to two right angles.  Therefore they are
parallel.''

[The argument in the words italicised would be clearer if it had been
  shown that the two interior angles on one side of $EH$ are
  \emph{severally} equal to the two interior angles on the other,
  namely $BFG$ to $CGF$ and $FGD$ to $AFG$; whence, assuming $FB$,
  $GD$ to meet in~$K$, we can take the triangle $KFG$ and place it
  (e.g.\ by rotating it in the plane about~$O$ the middle point
  of~$FG$) so that $FG$ falls where $GF$ is in the figure and $GD$
  falls on~$FA$, in which case $FB$ must also fall on~$GC$; hence,
  since $FB$, $GD$ meet at $K$, $GC$ and $FA$ must meet at a
  corresponding point $L$, Or, as Mr~Frankland does, we may substitute
  for $FG$ a straight line $MN$ through the middle point of~$FG$ drawn
  perpendicular to one of the parallels, say~$AB$.  Then, since the
  two triangles $OMF$, $ONG$ have two angles equal respectively,
  namely $FOM$ to $GON$ (\prop{1}{15}) and $OFM$ to $OGN$; and one
  side $OF$ equal to one side $OG$, the triangles are congruent, the
  angle $ONG$ is a right angle, and $MN$ is perpendicular to both $AB$
  and~$CD$. Then, by the same method of application, $MA$, $NC$ are
  shown to form with $MN$ a triangle $MALCN$ congruent with the
  triangle $NDKBM$, and $MA$, $NC$ meet at a point~$L$ corresponding
  to~$K$.  Thus the two straight lines would meet at the two points
  $K$, $L$.  This is what happens under the Riemann hypothesis, where
  the axiom that two straight lines cannot enclose a space does not
  hold, but all straight lines meeting in one point have another point
  common also, and e.g.\ in the particular figure just used $K$, $L$
  are points common to all perpendiculars to $MN$.  If we suppose that
  $K$, $L$ are not distinct points, but one point, the axiom that two
  straight lines cannot enclose a space is not contradicted.]

II.~Ptolemy now tries to prove \prop{1}{29} without using our Postulate, and
then deduces the Postulate from it (Proclus, pp.~365, 14–367, 27).

The argument to prove \prop{1}{29} is as follows.

\sidefig{post_5b}

The straight line which cuts the parallels must make the sum of the
interior angles on the same side equal to, greater than, or less than,
two right angles.

``Let $AB$, $CD$ be parallel, and let $FG$ meet them.  I say (1)~that
$FG$ does not make the interior angles on the same side greater than
two right angles.

``For, if the angles $AFG$, $CGF$ are greater than two right angles,
the remaining angles $BFG$, $DGF$ are less than two right angles.

``But the same two angles are also greater than two right angles;
\emph{for $AF$, $CG$ are no more parallel than $FB$, $GD$, so that, if
  the straight line falling on $AF$, $CG$ makes the interior angles
  greater than two right angles, the straight line falling on $FB$,
  $GD$ will also make the interior angles greater than two right
  angles}.

``But the same angles are also less than two right angles; for the
four angles $AFG$, $CGF$, $BFG$, $DGF$ are equal to four right angles:
which is impossible.

``Similarly (2)~we can show that the straight line falling on the
parallels does not make the interior angles on the same side less than
two right angles.

``But (3), if it makes them neither greater nor less than two right
angles, it can only make the interior angles on the same side
\emph{equal} to two right angles.''

III.~Ptolemy deduces Post.~\ref{post:5} thus:

Suppose that the straight lines making angles with a transversa) less
than two right angles do not meet on the side on which those angles
are.

Then, \emph{a fortiori}, they will not meet on the other side on which
are the angles \emph{greater} than two right angles.

Hence the straight lines will not meet in either direction; they are
therefore \emph{parallel}.

But, if so, the angles made by them with the transversal are equal to
two right angles, by the preceding proposition (=~\prop{1}{29}).

Therefore the same angles will be both equal to and less than two
right angles: which is impossible.

Hence the straight lines will meet.

IV.~Ptolemy lastly enforces his conclusion that the straight lines
will meet \emph{on the side on which are the angles less than two
  right angles} by recurring to the \emph{a fortiori} step in the
foregoing proof.

Let the angles $AFG$, $CGF$ in the accompanying figure be together
less than two right angles.

\sidefig{post_5c}

Therefore the angles $BFG$, $DGF$ are greater than two right angles.

We have proved that the straight lines are not non-secant.

If they meet, they must meet either towards $A$, $C$, or towards
$B$,~$D$.

(1)~Suppose they meet towards $B$, $D$, at~$K$.

Then, since the angles $AFG$, $CGF$ are less than two right angles,
and the angles $AFG$, $GFB$ are equal to two right angles, take away
the common angle $AFG$, and

the angle $CGF$ is less than the angle~$BFG$;

that is, the exterior angle of the triangle $KFG$ is less than the
interior arid opposite angle $BFG$: which is impossible.

Therefore $AB$, $CD$ do not meet towards $B$,~$D$.

(2)~But they do meet, and therefore they must meet in one direction or
the other:

therefore they meet towards $A$, $B$, that is, on the side where are
the angles less than two right angles.

The flaw in Ptolemy's argument is of course in the part of his proof
of \prop{1}{29} which I have italicised.  As Proclus says, he is not
entitled to assume that, if $AB$, $CD$ are parallel, whatever is true
of the interior angles on one side of $FG$ (i.e.\ that they are
together equal to, greater than, or less than, two right angles) is
necessarily true at the same time of the interior angles on the other
side.  Ptolemy justifies this by saying that $FA$, $GC$ are no more
parallel in one direction than $FB$, $GD$ are in the other: which is
equivalent to the assumption that \emph{through any point only one
  parallel can be drawn to a given straight line}.  That is, he
assumes an equivalent of the very Postulate he is endeavouring to
prove.

\subsubsection{Proclus}

Before passing to his own attempt at a proof, Proclus (p.~368,
26~sqq.)\ examines an ingenious argument (recalling somewhat the
famous one about Achilles and the tortoise) which appeared to show
that it was \emph{impossible} for the lines described in the Postulate
to meet.

Let $AB$, $CD$ make with $AC$ the angles $BAC$, $ACD$ together less
than two right angles.

\sidefig{post_5d}

Bisect $AC$ at~$E$ and along $AB$, $CD$ respectively measure $AF$,
$CG$ so that each is equal to $AE$.

Bisect $FG$ at $K$ and mark off $FK$, $GL$ each equal to~$FH$; and so
on.

Then $AF$, $CG$ will not meet at any point on~$FG$; for, if that were
the case, two sides of a triangle would be together equal to the
third: which is impossible.

Similarly, $AB$, $CD$ will not meet at any point on~$KL$; and
``proceeding like this indefinitely, joining the non-coincident
points, bisecting the lines so drawn, and cutting off from the
straight lines portions equal to the half of these, they say they
thereby prove that the straight lines AB, CD will not meet anywhere.''

It is not surprising that Proclus does not succeed in exposing the
fallacy here (the fact being thai the process will indeed be endless,
and yet the straight lines will intersect within a finite distance).
But Proclus' criticism contains nevertheless something of value.  He
says that the argument will prove too much, since we have only to join
$AG$ in order to see that straight lines making \emph{some} angles
which are together less than two right angles do in fact meet, namely
$AG$, $CG$.  ``Therefore it is not possible to assert, without some
definite limitation, that the straight lines produced from angles less
than two right angles do not meet.  On the contrary, it is manifest
that \emph{some} straight lines, when produced from angles less than
two right angles, do meet, although the argument seems to require it
to be proved that this property belongs to all such straight lines.
For one might say that, the lessening of the two right angles being
subject to no limitation, \emph{with such and such an amount of
  lessening the straight lines remain non-secant, but with an amount
  of lessening in excess of this they meet} (p.~371, 2–10).''

[Here then we have the germ of such an idea as that worked out by
  Lobachewsky, namely that the straight lines issuing from a point in
  a plane can be divided with reference to a straight line lying in
  that plane into two classes, ``secant'' and ``non-secant,'' and that
  we may define as parallel the two straight lines which divide the
  secant from the non-secant class.]

Proclus goes on (p.~371,~10) to base his own argument upon ``an axiom
such as Aristotle too used in arguing that the universe is finite.
For, \emph{if from one point two straight lines forming an angle be
  produced indefinitely, the distance (\greek{διάστασις},
  Arist.\ \greek{διάστημα}) between the said straight lines produced
  indefinitely will exceed any finite magnitude}.  Aristotle at all
events showed that, if the straight lines drawn from the centre to the
circumference are infinite, the interval between them is infinite.
For, if it is finite, it is impossible to increase the distance, so
that the straight lines (the radii) are not infinite.  Hence the
straight lines, when produced indefinitely, will be at a distance from
one another greater than any assumed finite magnitude.''

This is a fair representation of Aristotle's argument in \emph{De
  caelo} \prop{1}{5}, 271~b~28, although of course it is not a proof
of what Proclus assumes as an axiom.

This being premised, Proclus proceeds (p.~371,~24):

I.~``I say that, \emph{if any straight line cuts one of two parallels,
  it will cut the other also}.

``For let $AB$, $CD$ be parallel, and let $EFG$ cut $AB$; I say that
it will cut $CD$ also.

\sidefig{bookId_21}

``For, since $BF$, $FG$ are two straight lines from one point~$F$,
they have, when produced indefinitely, a distance greater than any
magnitude, so that it will also be greater than the interval between
the parallels.  Whenever therefore they are at a distance from one
another greater than the distance between the parallels, $FG$ will
cut~$CD$.

``Therefore etc.''

II.~``Having proved this, we shall prove, as a deduction from it, the
theorem in question.

\sidefig{bookId_22}

``For let $AB$, $CD$ be two straight lines, and let $EF$ falling on
them make the angles $BEF$, $DFE$ less than two right angles.

``I say that the straight lines will meet on that side on which are
the angles less than two right angles.

``For, since the angles $BEF$, $DFE$ are less than two right angles,
let the angle $HEB$ be equal to the excess of two right angles (over
them), and let $HE$ be produced to~$K$.

``Since then $EF$ falls on $KH$, $CD$ and makes the two interior
angles $HEF$, $DFE$ equal to two right angles,

the straight lines $MX$, $CD$ are parallel.

``And $AB$ cuts $KH$; therefore it will also cut $CD$, by what was
before shown.

``Therefore $AB$, $CD$ will meet on that side on which are the angles
less than two right angles.

``Hence the theorem is proved.''

Clavius criticised this proof on the ground that the axiom from which
it starts, taken from Aristotle, itself requires proof.  He points out
that, just as you cannot assume that two lines which continually
approach one another will meet (witness the hyperbola and its
asymptote), so you cannot assume that two lines which continually
diverge will ultimately be so far apart that a perpendicular from a
point on one let fall on the other will be greater than any assigned
distance; and he refers to the \emph{conchoid} of Nicomedes, which
continually approaches its asymptote, and therefore continually gets
farther away from the tangent at the vertex; yet the perpendicular
from any point on the curve to that tangent will always be less than
the distance between the tangent and the asymptote.  Saccheri supports
the objection.

Proclus' first proposition is open to the objection that it assumes
that two ``parallels'' (in the Euclidean sense) or, as we may say, two
straight liius which have a common perpendicular, are (not necessarily
equidistant, but) so related that, when they are produced
indefinitely, the perpendicular from a point of one upon the other
remains finite.

This last assumption is incorrect on the hyperbolic hypothesis; the
``axiom'' taken from Aristotle does not hold on the elliptic
hypothesis,

\subsubsection{Naṣīraddīn aṭ-Ṭūsī}

The Persian-born editor of Euclid, whose date is 1201–1274, has three
lemmas leading up to the final proposition.  Their content is
substantially as follows, the first lemma being apparently assumed as
evident

\sidefig{post_5g}

I.~(\emph{a}) If $AB$, $CD$ be two straight lines such that successive
perpendiculars, as $EF$, $GH$, $KL$, from points on $AB$ to $CD$
always make with $AB$ unequal angles, which are always acute on the
side towards~$B$ and always obtuse on the side towards~$A$, then the
lines $AB$, $CD$, so long as they do not cut, approach continually
nearer in the direction of the acute angles and diverge continually in
the direction of the obtuse angles, and the perpendiculars diminish
towards $B$, $D$, and increase towards~$A$,~$C$.

(\emph{b})~Conversely, if the perpendiculars so drawn continually
become shorter in the direction of $B$, $D$, and longer in the
direction of $A$, $C$, the straight lines $AB$, $CD$ approach
continually nearer in the direction of $B$, $D$ and diverge
continually in the other direction; also each perpendicular will make
with $AB$ two angles one of which is acute and the other is obtuse,
and all the acute angles will lie in the direction towards $B$, $D$,
and the obtuse angles in the opposite direction.

[Saccheri points out that even the first part (\emph{a}) requires
  proof.  As regards the converse (\emph{b}) he asks, why should not
  the successive acute angles made by the perpendiculars with $AB$,
  while remaining acute, become greater and greater as the
  perpendiculars become smaller until we arrive at last at a
  perpendicular which is a common perpendicular to \emph{both} lines?
  If that happens, all the author's efforts are in vain.  And, if you
  are to assume the truth of the statement in the lemma without proof,
  would it not, as Wallis said, be as easy to assume as axiomatic the
  statement in Post.~\ref{post:5} without more ado?]

II.~\emph{If $AC$, $BD$ be drawn from the extremities of $AB$ at right
  angles to it and on the same side, and if $AC$, $BD$ be made equal
  to one another and $CD$ be joined, each of the angles $ACD$, $BDC$
  will be right, and $CD$ will be equal to~$AB$,}

\sidefig{bookId_24}

The first part of this lemma is proved by \emph{reductio ad absurdum}
from the preceding lemma.  If, e.g., the angle $ACD$ is not right, it
must either be acute or obtuse.

Suppose it is acute; then, by lemma~1, $AC$ is greater than $BD$,
which is contrary to the hypothesis. And so on.

The angles $ACD$, $BDC$ being proved to be right angles, it is easy to
prove that $AB$, $CD$ are equal.

[It is of course assumed in this ``proof'' that, if the angle $ACD$ is
  acute, the angle $BDC$ is obtuse, and vice versa.]

III.~\emph{In any triangle the three angles are together equal to two
  right angles.}

This is proved for a \emph{right-angled} triangle by means of the
foregoing lemma, the four angles of the quadrilateral $ABCD$ of that
lemma being all right angles.  The proposition is then true for
\emph{any} triangle, since any triangle can be divided into two
right-angled triangles

IV.~Here we have the final ``proof'' of Post.~\ref{post:5}.  Three
cases are distinguished, but it is enough to show the case where one
of the interior angles is right and the other acute.

\sidefig{post_5i}

Suppose $AB$, $CD$ to be two straight lines met by $FCE$ making the
angle $ECD$ a right angle and the angle $CEB$ an acute angle.

Take any point $G$ on~$EB$, and draw $GH$ perpendicular to~$EC$.

Since the angle $CEG$ is acute, the perpendicular $GH$ will fall on
the side of~$E$ towards~$D$, and will either coincide with $CD$ or not
coincide with it.  In the former case the proposition is proved.

If $GH$ does not coincide with $CD$ but falls on the side of it
towards~$F$, $CD$, being within the triangle formed by the
perpendicular and by $CE$, $EG$, must cut~$EG$.  [An axiom is here
  used, namely that, if $CD$ be produced far enough, it must pass
  \emph{outside} the triangle and therefore cut \emph{some} side,
  which must be $EB$, since it cannot be the perpendicular
  (\prop{1}{27}), or $CE$.]

Lastly, let $GH$ fall on the side of $CD$ towards~$E$.

Along $HC$ set off $HK$, $KL$ etc., each equal to~$EH$, until we get
the first point of division, as~$M$, beyond~$C$.

Along $GB$ set off $GN$, $NO$ etc., each equal to $EG$, until $EP$ is
the same multiple of $EG$ that $EM$ is of~$EH$.

Then we can prove that the perpendiculars from $N$, $O$, $P$ on~$EC$
fall on the points $K$, $L$, $M$ respectively.

For take the first perpendicular, that from~$N$, and call it~$NS$.

Draw $EQ$ right angles to~$EH$ and equal to~$GH$, and set off $SR$
along $SN$ also equal to $GH$.  Join $QG$,~$GR$.

Then (second lemma) the angles $EQG$, $QGH$ are right, and $QG = EH$.

Similarly the angles $SRG$, $RGH$ are right, and $RG=SH$.

Thus $RGQ$ is one straight line, and the vertically opposite angles
$NGR$, $EGQ$ are equal.  The angles $NRG$, $EQG$ are both right, and
$NG = GE$, by construction.

Therefore (prop{1}{26}) $RG = GQ$;

whence $SH = HE = KH$, and $S$ coincides with~$K$.

We may proceed similarly with the other perpendiculars.

Thus $PM$ is perpendicular to~$FE$.  Hence $CD$, being parallel
to~$MP$ and within the triangle $PME$, must cut $EP$, if produced far
enough.

\subsubsection*{John Wallis}

As is well known, the argument of Wallis (1616—1703) assumed as a
postulate that, \emph{given a figure, another figure is possible which
  is similar to the given one and of any sine whatever}.  In fact
Wallis assumed this for \emph{triangles} only.  He first proved
(1)~that, if a finite straight line is placed on an infinite straight
line, and is then moved in its own direction as far as we please, it
will always lie on the same infinite straight line, (2)~that, if an
angle be moved so that one leg always slides along an infinite
straight line, the angle will remain the same, or equal, (3)~that, if
two straight lines, cut by a third, make the interior angles on the
same side less than two right angles, each of the exterior angles is
greater than the opposite interior angle (proved by means of
\prop{1}{13}).

\sidefig{post_5j}

(4)~If $AB$, $CD$ make, with $AC$, the interior angles less than two
right angles, suppose $AC$ (with $AB$ rigidly attached to it) to move
along $AF$ to the position $\alpha \gamma$, such that $\alpha$
coincides with~$C$.  If $AB$ then takes the position $\alpha\beta$,
$\alpha\beta$ \emph{lies entirely outside} $CD$ (proved by means
of~(3) above).

(5)~With the same hypotheses, \emph{the straight line $\alpha\beta$,
  or $AB$, during its motion, and before $\alpha$ reaches~$C$, must
  cut the straight line~$CD$}.

(6)~Here is enunciated the postulate stated above.

(7)~Postulate~\ref{post:5} is now proved thus.

\sidefig{post_5k}

Let $AB$, $CD$ be the straight lines which make, with the infinite
straight line $ACF$ meeting them, the interior angles $BAC$, $DCA$
together less than two right angles.

Suppose $AC$ (with $AB$ rigidly attached to it) to move along $ACF$
until $AB$ takes the position of $\alpha\beta$ cutting $CD$ in~$\pi$.

Then, $\alpha C \pi$ being a triangle, we can, by the above postulate,
suppose a triangle drawn on the base $CA$ similar to the triangle
$\alpha C\pi$.

Let it be $ACP$.

[Wallis here interposes a defence of the hypothetical construction.]

Thus $CP$ and $AP$ meet at~$P$; and, as by the definition of similar
figures the angles of the triangles $PCA$, $\pi C \alpha$ are
respectively equal, the angle $PCA$ being equal to the angle $\pi C
\alpha$ and the angle $PAC$ to the angle $\pi\alpha C$ or $BAC$, it
follows that $CP$, $AP$ lie on $CD$, $AB$ produced respectively.

Hence $AB$, $CD$ meet on the side on which are the angles less than
two right angles.

[The whole gist of this proof lies in the assumed postulate as to the
  existence of similar figures; and, as Saccheri points out, this is
  equivalent to unconditionally assuming the ``hypothesis of the right
  angle,'' and consequently Euclid's Postulate~\ref{post:5}.]

\subsubsection*{Gerolamo Saccheri}

The book \emph{Euclides ab omni naevo vindicatus} (1733) by Gerolamo
Saccheri (1667—1733), a Jesuit, and professor at the University of
Pavia, is now accessible (1)~edited in German by Engel and Stäckel,
\emph{Die Theorie der Parallellinien von Euklid bis auf Gauss}, 1895,
pp.~41–136, and (2)~in an Italian version, abridged but annotated,
\emph{L'Euclide emendato del P.~Gerolamo Saccheri}, by G.\ Boccardini
(Hoepli, Milan, 1904).  It is of much greater importance than all the
earlier attempts to prove Post.~\ref{post:5} because Saccheri was the
first to contemplate the possibility of hypotheses other than that of
Euclid, and to work out a number of consequences of those hypotheses.
He was therefore a true precursor of Legendre and of Lobachewsky, as
Beltrami called him (1889), and, it might be added, of Riemann also.
For, as Veronese observes (\emph{Fondamenti di geometria}, p.~570),
Saccheri obtained a glimpse of the theory of parallels in all its
generality, while Legendre, Lobachewsky and G.~Bolyai excluded \emph{a
  priori}, without knowing it, the ``hypothesis of the obtuse angle,''
or the Riemann hypothesis.  Saccheri, however, was the victim of the
preconceived notion of his time that the sole possible geometry was
the Euclidean, and he presents the curious spectacle of a man
laboriously erecting a structure upon new foundations for the very
purpose of demolishing it afterwards; he sought for contradictions in
the heart of the systems which he constructed, in order to prove
thereby the falsity of his hypotheses.

\sidefig{post_5l}

For the purpose of formulating his hypotheses he takes a plane
quadrilateral $ABDC$, two opposite sides of which, $AC$, $BD$, are
equal and perpendicular to a third~$AB$. Then the angles at $C$
and~$D$ are easily proved to be equal.  On the Euclidean hypothesis
they are both right angles; but apart from this hypothesis they might
be both obtuse or both acute.  To the three possibilities, which
Saccheri distinguishes by the names (1)~\emph{the hypothesis of the
  right angle}, (2)~\emph{the hypothesis of the obtuse angle} and
(3)~\emph{the hypothesis of the acute angle} respectively, there
corresponds a certain group of theorems; and Saccheri's point of view
is that the Postulate will be completely proved if the consequences
which follow from the last two hypotheses comprise results
inconsistent with one another.

Among the most important of his propositions are the following:

(1)~\emph{If the hypothesis of the right angle, or of the obtuse
  angle, or of the acute angle is proved true in a single case, it is
  true in every other case.}  (Props.~\r5., \r6., \r7.)

(2)~\emph{According as the hypothesis of the right angle, the obtuse
  angle, or the acute angle is true, the sum of the three angles of a
  triangle is equal to, greater than, or less than two right angles.}
(Prop.~\r9.)

(3)~\emph{From the existence of a single triangle in which the sum of
  the angles is equal to, greater than, or less than two right angles
  the truth of the hypothesis of the right angle, obtuse angle, or
  acute angle respectively follows.} (Prop.~\r15.)

These propositions involve the following: \emph{If in a single
  triangle the sum of the angles is equal to, greater than, or less
  than two right angles, then any triangle has the sum of its angles
  equal to, greater than, or less than two right angles respectively},
which was proved about a century later by Legendre for the two cases
only where the sum is \emph{equal to} or \emph{less than} two right
angles.

The proofs are not free from imperfections, as when, in the proofs of
Prop.~\r12.\ and the part of Prop.~\r13. relating to the hypothesis of
the \emph{obtuse angle}, Saccheri uses Eucl.\ \prop{1}{18} depending
on \prop{1}{16}, a proposition which is only valid on the assumption
that \emph{straight lines are infinite in length}; for this assumption
itself does not hold under the hypothesis of the obtuse angle (the
Riemann hypothesis).

The hypothesis of the acute angle takes Saccheri much longer to
dispose of, and this part of the book is less satisfactory; but it
contains the following propositions afterwards established anew by
Lobachewsky and Bolyai, viz.:

(4)~\emph{Two straight lines in a plane (\emph{even on the hypothesis
    of the acute angle}) either have a common perpendicular, or must,
  if produced in one and the same direction, either intersect once at
  a finite distance or at least continually approach one another.}
(Prop.~\r23.)

(5)~\emph{In a cluster of rays issuing from a point there exist always
  (\emph{on the hypothesis of the acute angle}) two determinate
  straight lines which separate the straight lines which intersect a
  fixed straight line from those which do not intersect it, ending
  with and including the straight line which has a common
  perpendicular with the fixed straight line.} (Props.~\r30., \r31.,
\r32.)

\subsubsection*{Lambert}

A dissertation by G.~S. Klügel, \emph{Conatuum praecipuorum theoriam
  parallelarum demonstrandi recensio} (1763), contained an examination
of some thirty ``demonstrations'' of Post.~\ref{post:5} and is
remarkable for its conclusion expressing, apparently for the first
time, \emph{doubt as to its demonstrability} and observing that the
certainty which we have in us of the truth of the Euclidean hypothesis
is not the result of a series of rigorous deductions but rather of
experimental observations.  It also had the greater merit that it
called the attention of Johann Heinrich Lambert (1728—1777) to the
theory of parallels.  His \emph{Theory of Parallels} was written in
1766 and published after his death by G.~Bernoulli and
C.~F. Hindenburg; it is reproduced by Engel and Stäckel
(\emph{op.~sit.}\ pp.~151—208).

The third part of Lambert's tract is devoted to the discussion of the
same three hypotheses as Saccheri's, the hypothesis of the \emph{right
  angle} being for Lambert the \emph{first}, that of the \emph{obtuse
  angle} the \emph{second}, and that of the \emph{acute angle} the
\emph{third}, hypothesis; and, with reference to a quadrilateral with
\emph{three right angles} from which Lambert starts (that is, one of
the halves into which the median divides Saccheri's quadrilateral),
the three hypotheses are the assumptions that the fourth angle is a
right angle, an obtuse angle, or an acute angle respectively.

Lambert goes much further than Saccheri in the deduction of new
propositions from the \emph{second} and \emph{third} hypotheses.  The
most remarkable is the following.

\emph{The area of a plane triangle, under the \emph{second} and
  \emph{third} hypotheses, is proportional to the difference between
  the sum of the three angles and two right angles.}

Thus the numerical expression for the area of a triangle is, under the
\emph{third} hypothesis
\begin{equation}\label{eq:1}
\Delta = k (\pi - A - B - C) \tag{1}
\end{equation}
and under the second hypothesis
\begin{equation}\label{eq:2}
\Delta = k (A + B + C - \pi) \tag{2}
\end{equation}
where $k$ is a positive constant

A remarkable observation is appended (\S82): ``In connexion with this
it seems to be remarkable that the \emph{second} hypothesis holds if
\emph{spherical} instead of plane triangles are taken, because in the
former also the sum of the angles is greater than two right angles,
and the excess is proportional to the area of the triangle.

``It appears still more remarkable that what I here assert of
spherical triangles can be proved independently of the difficulty of
parallels.''

This discovery that the \emph{second} hypothesis is realised on the
surface of a sphere is important in view of the development, later, of
the Riemann hypothesis (1854).

Still more remarkable is the following prophetic sentence: ``\emph{I
  am almost inclined to draw the conclusion that the \emph{third}
  hypothesis arises with an imaginary spherical surface}''
(cf.\ Lobachewsky's \emph{Géométrie imaginaire}, 1837).

No doubt Lambert was confirmed in this by the fact that, in the
formula~\eqref{eq:2} above, which, for $k = r^2$, represents the area
of a spherical triangle, if $r \sqrt{-1}$ is substituted for~$r$, and
$r^2 = k$, we obtain the formula~\eqref{eq:1}.

\subsubsection*{Legendre}

No account of our present subject would be complete without a full
reference to what is of permanent value in the investigations of
Adrien Marie Legendre (1752—1833) relating to the theory of parallels,
which extended over the space of a generation.  His different attempts
to prove the Euclidean hypothesis appeared in the successive editions
of his \emph{Éléments de Géométrie} from the first (1794) to the
twelfth (1823), which last may be said to contain his last word on the
subject.  Later, in 1833, he published, in the \emph{Mémoires de
  l'Académie Royale des Sciences}, \r12.\ p.~367~sqq., a collection of
his different proofs under the title \emph{Réflexions sur différentes
  manières de démontrer la théorie des parallèles}.  His exposition
brought out clearly, as Saccheri had done, and kept steadily in view,
the essential connexion between the theory of parallels and the sum of
the angles of a triangle.  In the first edition of the \emph{Éléments}
the proposition that \emph{the sum of the angles of a triangle is
  equal to two right angles} was proved analytically on the basis of
the assumption that the choice of a \emph{unit of length} does not
affect the correctness of the proposition to be proved, which is of
course equivalent to Wallis' assumption of \emph{the existence of
  similar figures}.  A similar analytical proof is given in the notes
to the twelfth edition.  In his second edition Legendre proved
Postulate~\ref{post:5} by means of the assumption that, \emph{given
  three points not in a straight line, there exists a circle passing
  through all three}.  In the third edition (1800) he gave the
proposition that \emph{the sum of the angles of a triangle is not
  greater than two right angles}; this proof, which was geometrical,
was replaced later by another, the best known, depending on a
construction like that of Euclid \prop{1}{16}, the continued
application of which enables any number of successive triangles to be
evolved in which, while the sum of the angles in each remains always
equal to the sum of the angles of the original triangle, one of the
angles increases and the sum of the other two diminishes
continually. But Legendre found the proof of the equally necessary
proposition that the sum of the angles of a triangle is \emph{not
  less} than two right angles to present great difficulties.  He first
observed that, as in the case of spherical triangles (in which the sum
of the angles is greater than two right angles) the excess of the area
of the angles over two right angles is proportional to the area of the
triangle, so in the case of rectilineal triangles, if the sum of the
angles is less than two right angles by a certain \emph{deficit}, the
\emph{deficit} will be proportional to the area of the triangle.
Hence if, starting from a given triangle, we could construct another
triangle in which the original triangle is contained at least $m$
times, the \emph{deficit} of this new triangle will be equal to at
least $m$ times that of the original triangle, so that the sum of the
angles of the greater triangle will diminish progressively as $m$
increases, until it becomes zero or negative: which is absurd.  The
whole difficulty was thus reduced to that of the construction of a
triangle containing the given triangle at least twice; but the
solution of even this simple problem requires it to be assumed (or
proved) that \emph{through a given point within a given angle less
  than two-thirds of a right angle we can always draw a straight line
  which shall meet both sides of the angle}.  This is however really
equivalent to Euclid's Postulate. The proof in the course of which the
necessity for the assumption appeared is as follows.

It is required to prove that the sum of the angles of a triangle
cannot be \emph{less} than two right angles.

\sidefig{post_5m}

Suppose $A$ is the least of the three angles of a triangle
$ABC$. Apply to the opposite side $BC$ a triangle $DBC$, equal to the
triangle $ACB$, and such that the angle $DBC$ is equal to the angle
$ACB$, and the angle $DCB$ to the angle $ABC$; and \emph{draw any
  straight line through $D$ cutting $AB$, $AC$ produced in $E$, $F$}.

If now the sum of the angles of the triangle $ABC$ is less than two
right angles, being equal to $2 R - \delta$ say, the sum of the angles
of the triangle $DBC$, equal to the triangte $ABC$, is also $2 R –
\delta$.

Since the sum of the three angles of the remaining triangles $DEB$,
$FDC$ respectively cannot at all events be \emph{greater} than two
right angles [for Legendre's proofs of this see below], the sum of the
twelve angles of the four triangles in the figure \emph{cannot be
  greater} than
\[
    4R + (2R - \delta) + (2R - \delta), \quad\text{i.e.\ $8R - 2\delta$}.
\]

Now the sum of the three angles at each of the points $B$, $C$, $D$
is~$2R$.

Subtracting these nine angles, we have the result that the three
angles of the triangle $AEF$ \emph{cannot be greater} than $2R -
2\delta$.

Hence, if the sum of the angles of the triangle $ABC$ is less than two
right angles by~$\delta$, the sum of the angles of the larger triangle
$AEF$ is less than two right angles by \emph{at least~$2\delta$}.

We can continue the construction, making a still larger triangle from
$AEF$, and so on.

But, however small $\delta$ is, we can arrive at a multiple
$2^n\delta$ which shall exceed any given angle and therefore $2R$
itself; so that the sum of the three angles of a triangle sufficiently
large would be xzero or even less than zero: which is absurd.

Therefore etc.

The difficulty caused by the necessity of making the above-mentioned
assumption made Legendre abandon, in his ninth edition, the method of
the editions from the third to the eighth and return to Euclid's
method pure and simple.

But again, in the twelfth, he returned to the plan of constructing any
number of successive triangles such that the sum of the three angles
in all of them remains equal to the sum of the three angles of the
original triangle, but two of the angles of the new triangles become
smaller and smaller, while the third becomes larger and larger; and
this time he claims to prove in one proposition that the sum of the
three angles of the original triangle is \emph{equal} to two right
angles by continuing the construction of new triangles
\emph{indefinitely} and compressing the two smaller angles of the
ultimate triangle into nothing, while the third angle is made to
become a \emph{flat} angle at the same time. The construction and
attempted proof are as follows.

Let $ABC$ be the given triangle; let $AB$ be the greatest side and
$BC$ the least; therefore $C$ is the greatest angle and $A$ the least.

From $A$ draw $AD$ to the middle point of $BC$, and produce $AD$
to~$C'$, making $AC'$ equal to~$AB$,

Produce $AB$ to $B'$, making $AB'$ equal to twice $AD$.

The triangle $AB'C'$ is then such that the sum of its three angles is
equal to the sum of the three angles of the triangle $ABC$.

\infig{post_5n}

For take $AK$ along $AB$ equal to $AD$, and join $C'K$.

Then the triangles $ABD$, $AC'K$ havt two sides and the included
angles respectively equal, and are therefore equal in all respects;
and $C'K$ is equal to $BD$ or~$DC$.

Next, in the triangles $B'C'K$, $ACD$, the angles $B'KC'$, $ADC$ are
equal, being respectively supplementary to the equal angles $AKC'$,
$ADB$; and the two sides about the equal angles are respectively
equal;

therefore the triangles $B'C'K$, $ACD$ are equal in all respects.

Thus the angle $AC'B'$ is the sum of two angles respectively equal to
the angles $B$, $C$ of the original triangle; and the angle~$A$ in the
original triangle is the sum of two angles respectively equal to the
angles at $A$ and~$B'$ in the triangle~$ABC$.

It follows that the sum of the three angles of the new triangle
$AB'C'$ is equal to the sum of the angles of the triangle~$ABC$.

Moreover, the side $AC'$, being equal to $AB$, and therefore greater
than $AC$, is greater than $B'C'$ which is equal to~$AC$.

Hence the angle $C'AB'$ is less than the angle $AB'C'$; so that the
angle $C'AB'$ is less than~$\frac{1}{2} A$, where $A$ denotes the
angle $CAB$ of the original triangle.

[It will be observed that the triangle $AB'C'$ is really the same
  triangle as the triangle $AEB$ obtained by the construction of
  Eucl.\ \prop{1}{16}, but differently placed so that the longest side
  lies along~$AB$.]

By taking the middle point $D'$ of the side $B'C'$ and repeating the
same construction, we obtain a triangle $AB''C''$ such that (1)~the
sum of its three angles is equal to the sum of the three angles of
$ABC$, (2)~the sum of the two angles $CAB''$, $AB''C''$ is equal to
the angle $C'AB'$ in the preceding triangle, and is therefore less
than $\frac{1}{2} A$, and (3)~the angle $C”AB''$ is less than half the
angle $C'AB'$, and therefore less than~$\frac{1}{4}A$.

Continuing in this way, we shall obtain a triangle $Abc$ such that the
sum of two angles, those at $A$ and~$b$, is less than $\frac{1}{2^n}
A$, and the angle at~$c$ is greater than the corresponding angle in
the preceding triangle.

If, Legendre argues, the construction be continued indefinitely so
that $\frac{1}{2^n} A$ becomes smaller than any assigned angle, the
point~$c$ ultimately lies on $Ab$, and the sum of the three angles of
the triangle (which is equal to the sum of the three angles of the
original triangle) becomes identical with the angle at~$c$, which is
then a \emph{flat} angle, and therefore equal to two right angles.

This proof was however shown to be unsound (in respect of the final
inference) by J.~P.~W. Stein in Gergonne's \emph{Annales de
  Mathématiques} \r15., 1824, pp.~77—79.

We will now reproduce shortly the substance of the theorems of
Legendre which are of the most permanent value as not depending on a
particular hypothesis as regards parallels.

I.~\emph{The sum of the three angles of a triangle \emph{cannot} be
  greater than two right angles.}

This Legendre proved in two ways.

(1)~\emph{First proof} (in the third edition of the \emph{Éléments}).

Let $ABC$ be the given triangle, and $ACJ$ a straight line.

Make $CE$ equal to~$AC$, the angle~$DCE$ equal to the angle $BAC$, and
$DC$ equal to~$AB$. Join~$DE$.

Then the triangle $DCE$ is equal to the triangle $BAC$ in all respects.

\infig{post_5o}

If then the sum of the three angles of the triangle $ABC$ is greater
than $2R$, the said sum must be greater than the sum of the angles
$BCA$, $BCD$, $DCE$, which sum is \emph{equal} to~$2R$.

Subtracting the equal angles on both sides, we have the result that

the angle $ABC$ is \emph{greater} than the angle $BCD$.

But the two sides $AB$, $BC$ of the triangle $ABC$ are respectively
equal to the two sides $DC$, $CB$ of the triangle $BCD$.

Therefore the base $AC$ is \emph{greater} than the base $BD$
(Eucl.\ \prop{1}{14}).

Next, make the triangle $FEG$ (by the same construction) equal in all
respects to the triangle $BAC$ or~$DCE$; and we prove in the same way
that $CE$ (or~$AC$) is greater than~$DE$.

And, at the same time, $BD$ is equal to~$DF$, because the angles
$BCD$, $DEF$ are equal.

Continuing the construction of further triangles, however small the
difference between $AC$ and~$BD$ is, we shall ultimately reach some
multiple of this difference, represented in the figure by (say) the
difference between the straight line $AJ$ and the composite line
$BDFHK$, which will be greater than any assigned length, and greater
therefore than the sum of $AB$ and~$JK$.

Hence, on the assumption that the sum of the angles of the triangle
$ABC$ is greater than~$2R$, the broken line $ABDFHKJ$ may be less than
the straight line~$AJ$: which is impossible.

Therefore etc.

(2) \emph{Proof substituted later.}

If possible, let $2R + \alpha$ be the sum of the three angles of the
triangle $ABC$, of which $A$ is not greater than either of the others.

\sidefig{post_5p}

Bisect $BC$ at~$H$, and produce $AH$ to~$D$, making $HD$ equal
to~$AH$; join~$BD$.

Then the triangles $AHC$, $DHB$ are equal in all respects
(\prop{1}{4}); and the angles $CAH$, $ACH$ are respectively equal to
the angles $BDH$, $DBH$.

It follows that the sum of the angles of the triangle $ABD$ is equal
to the sum of the angles of the original triangle, i.e.\ to $2R +
\alpha$.

And one of the angles $DAB$, $ADB$ is either equal to or less than
half the angle~$CAB$.

Continuing the same construction with the triangle $ADB$, we find a
third triangle in which the sum of the angles is still $2R + \alpha$,
while one of them is equal to or less than $(\angle CAB)/4$.

Proceeding in this way, we arrive at a triangle in which the sum of
the angles is $2R + \alpha$, and one of them is not greater than
$(\angle CAB)/2^n$.

And, if $n$ is sufficiently large, this will be less than~$\alpha$; in
which case we should have a triangle in which two angles are together
greater than two right angles: which is absurd.

Therefore $\alpha$ is equal to or less than zero.

(It will be noted that in both these proofs, as in
Eucl.\ \prop{1}{16}, it is taken for granted that \emph{a straight
  line is infinite in length} and does not return into itself, which
is not true under the Riemann hypothesis.)

II.~On the assumption that the sum of the angles of a triangle is
\emph{less} than two right angles, \emph{if a triangle is made up of
  two others, the ``deficit'' of the former is equal to the sum of the
  ``deficits'' of the others}.

In fact, if the sums of the angles of the component triangles are $2R
- \alpha$, $2R - \beta$ respectively, the sum of the angles of the
whole triangle is
\[
    (2R-\alpha) + (2R-\beta) - 2R = 2R - (\alpha + \beta)
\]

III.~\emph{If the sum of the three angles of a triangle is
  \emph{equal} to two right angles, the same is true of all triangles
  obtained by subdividing it by straight lines drawn from a vertex to
  meet the opposite side.}

Since the sum of the angles of the triangle $ABC$ is equal to~$2R$, if
the sum of the angles of the triangle $ABD$ were $2R - \alpha$, it
would follow that the sum of the angles of the triangle~$ADC$ must be
$2R + \alpha$, which is absurd (by I.~above).

\sidefig{post_5q}

IV.~\emph{If in a triangle the sum of the three angles is equal to two
  right angles, a quadrilateral can always be constructed with four
  right angles and four equal sides exceeding in length any assigned
  rectilineal segment.}

Let $ABC$ be a triangle in which the sum of the angles is equal to two
right angles.  We can assume $ABC$ to be an isosceles right-angled
triangle because we can reduce the case to this by making subdivisions
of $ABC$ by straight lines through vertices (as in Prop.~III.\ above).

Taking two equal triangles of this kind and placing their hypotenuses
together, we obtain a quadrilateral with four right angles and four
equal sides.

Putting four of these quadrilaterals together, we obtain a new
quadrilateral of the same kind but with its sides double of those of
the first quadrilateral.

After $n$ such operations we have a quadrilateral with four right
angles and four equal sides, each being equal to $2^n$ times the
side~$AB$.

The diagonal of this quadrilateral divides it into two equal isosceles
right-angled triangles in each of which the sum of the angles is
equal to two right angles.

Consequently, from the existence of \emph{one} triangle in which the
sum of the three angles is equal to two right angles it follows that
there exists an isosceles right-angled triangle with sides greater
than any assigned rectilineal segment and such that the sum of its
three angles is also equal to two right angles.

V.~\emph{If the sum of the three angles of \emph{one} triangle is
  equal to two right angles, the sum of the three angles of \emph{any
    other} triangle is also equal to two right angles.}

\sidefig{post_5r}

It is enough to prove this for a \emph{right-angled triangle}, since
any triangle can be divided into two right-angled triangles.

Let $ABC$ be any right-angled triangle.

If then the sum of the angles of any one triangle is equal to two
right angles, we can construct (by the preceding Prop.)\ an isosceles
right-angled triangle with the same property and with its
perpendicular sides, greater than those of $ABC$.

Let $A'B'C'$ be such a triangle, and let it be applied to $ABC$, as in
the figure.

Applying then Prop.~\r3.\ above, we deduce first that the sum of the
three angles of the triangle $AB'C'$ is equal to two right angles, and
next, for the same reason, that the sum of the three angles of the
original triangle $ABC$ is equal to two right angles.

VI.~\emph{If in \emph{any one} triangle the sum of the three angles is
  less than two right angles, the sum of the three angles of \emph{any
    other} triangle is also less than two right angles.}

This follows from the preceding theorem.

(It will be observed that the last two theorems are included among
those of Saccheri, which contain however in addition the corresponding
theorem touching the case where the sum of the angles is
\emph{greater} than two right angles.)

We come now to the bearing of these propositions upon Euclid's
Postulate~\ref{post:5}; and the next theorem is

VII.~\emph{If the sum of the three angles of a triangle is equal to
  two right angles, through any point in a plane there can only be
  drawn one parallel to a given straight line.}

For the proof of this we require the following

\begin{lemma*}
It is always possible, through a point~$P$, to draw a straight line
which shall make, with a given straight line~($r$), an angle less than
any assigned angle.
\end{lemma*}

Let $Q$ be the foot of the perpendicular from $P$ upon~$r$.

\sidefig{post_5s}

Let a segment $QR$ be taken on~$r$, on either side of~$Q$, such that
$QR$ is equal to~$PQ$.

Join $PR$, and mark off the segment $RR'$ equal to~$PR$; join~$PR'$.

If $\omega$ represents the angle $QPR$ or the angle $QRP$, each of the
equal angles $RPR'$, $RR'P$ is not greater than $\omega/2$.

Continuing the construction, we obtain, after the requisite number of
operations, a triangle $P R_{n-1} R_n$ in which each of the equal
angles is equal to or less than~$\omega/2^n$.

Hence we shall arrive at a straight line $PR_n$, which, starting from
$P$ and meeting~$r$, makes with $r$ an angle as small as we please.

To return now to the Proposition.  Draw from~$P$ the straight line~$s$
perpendicular to~$PQ$.

Then any straight line drawn from~$P$ which meets $r$ in~$R$ will form
equal angles with $r$ and~$s$, since, by hypothesis, the sum of the
angles of the triangle $PQR$ is equal to two right angles.

And since, by the Lemma, it is always possible to draw through $P$
straight lines which form with $r$ angles as small as we please, it
follows that all the straight lines through~$P$, except~$s$, will
meet~$r$.  Hence $s$ is the only parallel to~$r$ that can be drawn
through~$P$.

The history of the attempts to prove Postulate~\ref{post:5} or
something equivalent has now been brought down to the parting of the
ways.  The further developments on lines independent of the Postulate,
beginning with Schweikart (1780—1857), Taurinus (1794—1874), Gauss
(1777—1855), Lobachewsky (1793—1850), J.~Bolyai (1802—1860), Riemann
(1826—1866), belong to the history of non-Euclidean geometry, which is
outside the scope of this work.  I may refer the reader to the full
article \emph{Sulla teoria delle parallele e sulla geometrie
  non-euclidee} by R.~Bonola in \emph{Questioni riguardanti le
  matematiche elementari}, \r1., of which I have made considerable use
in the above, to the same author's \emph{La geometria non-euclidea},
Bologna, 1906, to the first volume of Killing's \emph{Einführing in
  die Grundlagen der Geometrie}, Paderborn, 1893, to P.~Mansion's
\emph{Premiers principes de métagéométrie}, and P.~Barbarin's \emph{La
  géométrie ncn-Euclidienne}, Paris, 1902, to the historical summary
in Veronese's \emph{Fondamenti di geometria}, 1891, p.~565 sqq., and
(for original sources) to Engel and Stäckel, \emph{Die Theorie der
  Parallellinien von Euklid bis auf Gauss}, 1895, and \emph{Urkunden
  zur Geschichte der nicht-Euklidischen Geometrie},
\r1.\ (Lobachewsky), 1899, and \r2.\ (Wolfgang und Johann Bolyai).  I
will only add that it was Gauss who first expressed a conviction that
the Postulate could never be proved; he indicated this in reviews in
the \emph{Göttingische gelehrte Anzeigen}, 20 Apr.\ 1816 and 28
Oct.\ 1822, and affirmed it in a letter to Bessel of 27 January, 1829.
The actual indemonstrability of the Postulate was proved by Beltrami
(1868) and by Hoüel (\emph{Note sur l'impossibilité de démontrer par
  une construction plane le principe de la théorie des parallèles dit
  Postulatum d'Euclide} in \emph{Giornale di matematiche}, \r8., 1870,
pp.~84–89).

\subsubsection*{Alternatives for Postulate 5}

It may be convenient to collect here a few of the more noteworthy
substitutes which have from time to time been formally suggested or
tacitly assumed.

(1)~\emph{Through a given point only one parallel can be drawn to a
  given straight line} or, \emph{Two straight lines which intersect
  one another cannot both be parallel to one and the same straight
  line.}

This is commonly known as ``Playfair's Axiom,'' but it was of course
not a new discovery.  It is distinctly stated in Proclus' note to
Eucl.~\prop{1}{31}.

(1~\emph{a})~\emph{If a straight line intersect one of two parallels,
  it will intersect the other also} (Proclus).

(1~\emph{b})~\emph{Straight lines parallel to the same straight line
  are parallel to one another.}

The forms (1~\emph{a}) and (1~\emph{b}) are exactly equivalent to~(1).

(2)~\emph{There exist straight lines everywhere equidistant from one
  another} (Posidonius and Geminus); with which may be compared
Proclus' tacit assumption that \emph{Parallels remain, throughout
  their length, at a finite distance from one another}.

(3)~\emph{There exists a triangle in which the sum of the three angles
  is equal to two right angles} (Legendre).

(4)~\emph{Given any figure, there exists a figure similar to it of any
  size we please} (Wallis, Carnot, Laplace).

Saccheri points out that it is not necessary to assume so much, and
that it is enough to postulate that \emph{there exist two unequal
  triangles with equal angles}.

(5)~\emph{Through any point within an angle less than two-thirds of a
  right angle a straight line can always be drawn which meets both
  sides of the angle} (Legendre).

With this may be compared the similar axiom of Lorenz (\emph{Grundriss
  der reinen und angewandten Mathematik}, 1791): \emph{Every straight
  line through a point within an angle must meet one of the sides of
  the angle}.

(6)~\emph{Given any three points not in a straight line, there exists
  a circle passing through them} (Legendre, W.~Bolyai).

(7)~``\emph{If I could prove that a rectilineal triangle is possible
  the content of which is greater than any given area, I am in a
  position to prove perfectly rigorously the whole of geometry}''
(Gauss, in a letter to W.~Bolyai, 1799).

Cf.~the proposition of Legendre numbered \r4.\ above, and the axiom of
Worpitzky: \emph{There exists no triangle in which every angle is as
  small as we please}.

(8)~\emph{If in a quadrilateral three angles are right angles, the
  fourth angle is a right angle also} (Clairaut, 1741).

(9)~\emph{If two straight lines are parallel, they are figures
  opposite to (or the reflex of) one another with respect to the
  middle points of all their transversal segments} (Veronese,
\emph{Elementi}, 1904).

Or, \emph{Two parallel straight lines intercept, on every transiiersal
  which passes through the middle point of a segment included between
  them, another segment the middle point of which is the middle point
  of the first} (Ingrami, \emph{Elementi}, 1904).

Veronese and Ingrami deduce immediately Playfair's Axiom,

\chapter*{Axioms or \emph{Common Notions}}

In a paper \emph{Sur l'authenticité des axiomes d'Euclide} in the
\emph{Bulletin des sciences math.\ et astron.}\ 1884,
p.~162~sq. (\emph{Mémoires scientifiques}, \r2., pp.~48–63), Paul
Tannery maintained that the \emph{Common Notions} (including the first
three) were not in Euclid's work but were interpolated later.  The
following are his main arguments. (1)~If Euclid had set about
distinguishing between indemonstrable principles (\emph{a})~common to
all demonstrative sciences and (\emph{b})~peculiar to geometry, he
would, says Tannery, certainly not have placed the common principles
second and the special principles (the Postulates) first. (2)~If the
\emph{Common Notions} are Euclid's, this designation of them must be
his too; for he must have used \emph{some} name to distinguish them
from the Postulates and, if he had used another name, such as
\emph{Axioms}, it is impossible to imagine why that name was changed
afterwards for a less suitable one.  The word \greek{ἔννοια}
(\emph{notion}), says Tannery, never signified a notion in the sense
of a \emph{proposition}, but a notion of some object; nor is it found
in any technical sense in Plato and Aristotle.  (3)~Tannery's own view
was that the formulation of the \emph{Common Notions} dates from the
time of Apollonius, and that it was inspired by his work relating to
the Elements (we know from Proclus that Apollonius tried to prove the
\emph{Common Notions}).  This idea, Tannery thought, was confirmed by
a ``fortunate coincidence'' furnished by the occurrence of the word
\greek{ἔννοια} (\emph{notion}) in a quotation by Proclus (p.~100,~6):
``we shall agree with Apollonius when he says that we have a
\emph{notion} (\greek{ἔννοια}) of a line when we order the lengths,
only, of roads or walls to be measured.''

In reply to argument (1) that it is an unnatural order to place the
purely geometrical Postulates first, and the \emph{Common Notions},
which are not peculiar to geometry, last, it may be pointed out that
it would surely have been a still more awkward arrangement to give the
Definitions first and then to separate from them, by the interposition
of the \emph{Common Notions}, the Postulates, which are so closely
connected with the Definitions in that they proceed to postulate the
\emph{existence} of certain of the things defined, namely straight
lines and circles.

(2)~Though it is true that \greek{ἔννοια} in Plato and Aristotle is
generally a notion of an \emph{object}, not of a \emph{fact} or
proposition, there are instances in Aristotle where it does mean a
notion of a fact: thus in the \emph{Eth.\ Nic.}\ \r9.~11, 1171~a~32 he
speaks of ``the notion (or consciousness) \emph{that friends
  sympathise}'' (\greek{ἡ ἔννοια τοῦ συναλγεῖν τοὺς φίλους}) and
again, b~14, of ``the \emph{notion} (or consciousness) \emph{that they
  are pleased} at his good fortune.''  It is true that Plato and
Aristotle do not use the word in a technical sense; but neither was
there apparently in Aristotle's time any fixed technical term for what
we call ``axioms,'' since he speaks of them variously as ``the
so-called axioms in mathematics,'' ``the so-called common axioms,''
``the common (things)'' (\greek{τὰ κοινά}), and even ``the common
\emph{opinions}'' (\greek{κοιναὶ δόξαι}).  I see therefore no reason
why Euclid should not himself have given a technical sense to ``Common
Notions,'' which is at least a distinct improvement upon ``common
opinions.''

(3)~The use of \greek{ἔννοια} in Proclus' quotation from Apollonius
seems to me to be an unfortunate, rather than a fortunate, coincidence
from Tannery's point of view, for it is there used precisely in the
old sense of the notion of an \emph{object} (in that case a line).

No doubt it is difficult to feel certain that Euclid did himself use
the term \emph{Common Notions}, seeing that Proclus' commentary
generally speaks of \emph{Axioms}.  But even Proclus (p.~194,~8),
after explaining the meaning of the word ``axiom,'' first as used by
the Stoics, and secondly as used by ``Aristotle and the geometers,''
goes on to say: ``For in their view (that of Aristotle and the
geometers) \emph{axiom} and \emph{common notion} are the same thing.''
This, as it seems to me, may be a sort of apology for using the word
``axiom'' exclusively in what has gone before, as if Proclus had
suddenly bethought himself that he had described both Aristotle and
the geometers as using the one term ``axiom,'' whereas he should have
said that Aristotle spoke of ``axioms,'' while ``the geometers'' (in
fact Euclid), though meaning the same thing, called them \emph{Common
  Notions}.  It may be for a like reason that in another passage
(p.~76, 16), after quoting Aristotle's view of an ``axiom,'' as
distinct from a postulate and a hypothesis, he proceeds: ``For it is
not by virtue of a \emph{common notion} that, without being taught, we
preconceive the circle to be such and such a figure.''  If this view
of the two passages just quoted is correct, it would strengthen rather
than weaken the case for the genuineness of \emph{Common Notions} as
the Euclidean term.

Again, it is clear from Aristotle's allusions to the ``common axioms
in mathematics'' that more than one axiom of this kind had a place in
the text-books of his day; and as he constantly quotes the particular
axiom that, \emph{if equals be taken from equals, the remainders are
  equals} which is Euclid's \emph{Common Notion}~\ref{cn:3}, it would
seem that at least the first three \emph{Common Notions} were adopted
by Euclid from earlier textbooks.  It is, besides, scarcely credible
that, if the \emph{Common Notions} which Apollonius tried to prove had
not been introduced earlier (e.g.\ by Euclid), they would then have
been interpolated as axioms and not as propositions to be proved.  The
line taken by Apollonius is much better explained on the assumption
that he was directly attacking axioms which he found already admitted
into the \emph{Elements}.

Proclus, who recognised the five \emph{Common Notions} given in the
text, warns us, not only against the error of unnecessarily
multiplying the axioms, but against the contrary error of reducing
their number unduly (p.~196, 15), ``as Heron does in enunciating three
only; for it is also an axiom that \emph{the whole is greater than the
  part}, and indeed the geometer employs this in many places for his
demonstrations, and again \emph{that things which coincide are
  equal}.''

Thus Heron recognised the first three of the \emph{Common Notions};
and this fact, together with Aristotle's allusions to ``common
axioms'' (in the plural), and in particular to our \emph{Common
  Notion}~\ref{cn:3}, may satisfy us that at least the first three
\emph{Common Notions} were contained in the \emph{Elements} as they
left Euclid's hands.

\section*{Common Notion 1}

\greek{Τὰ τῷ αὐτῷ ἴσα καὶ ἀλλήλοις ἐστὶν ἴσα.}

\emph{Things which are equal to the same thing are also equal to one
  another.}

\sidefig{cn_1}

Aristotle throughout emphasises the fact that axioms are self-evident
truths, which it is impossible to demonstrate.  If, he says, any one
should attempt to prove them, it could only be through ignorance.
Aristotle therefore would undoubtedly have agreed in Proclus'
strictures on Apollonius for attempting to prove the axioms.  Proclus
gives (p.~194, 25), as a specimen of these attempted proofs by
Apollonius, that of the first of the \emph{Common Notions}.  ``Let $A$
be equal to~$B$, and the latter to~$C$; I say that $A$ is also equal
to~$C$.  For, since $A$ is equal to~$B$, it occupies the same space
with it; and since B is equal to~$C$, it occupies the same space with
it.

Therefore $A$ also occupies the same space with~$C$.''

Proclus rightly remarks (p.~194, 23) that ``the middle term is no more
intelligible (better known, \greek{γνωριμώτερον}) than the conclusion,
if it is not actually more disputable.''  Again (p.~195, 6), the proof
assumes two things, (1)~that things which ``occupy the same space''
(\greek{τόπος}) are equal to one another, and (2)~that things which
occupy the same space with one and the same thing occupy the same
space with one another; which is to explain the obvious by something
much more obscure, for space is an entity more unknown to us than the
things which exist in space.

Aristotle would also have objected to the proof that it is partial and
not general (\greek{καθόλου}), since it refers only to things which
can be supposed to occupy a space (or take up room), whereas the axiom
is, as Froclus says (p.~196,~1), true of numbers, speeds, and periods
of time as well, though of course each science uses axioms in relation
to its own subject-matter only.

\section*{Common Notions 2, 3}

2.~\greek{Καὶ ἐὰν ἴσοις ἴσα προστεθῇ, τὰ ὅλα ἐστὶν ἴσα.}

3.~\greek{Καὶ ἐὰν ἀπὸ ἴσων ἴσα ἀφαιρεθῇ, τὰ καταλειπόμενά ἐστιν ἴσα.}

2.~\emph{If equals be added to equals, the wholes are equal.}

3.~\emph{If equals he subtracted from equals, the remainders are equal.}

These two Common Notions are recognised by Heron and Proclus as
genuine.  The latter is the axiom which is so favourite an
illustration, with Aristotle.

Following them in the \textsc{mss.}\ and editions there came four
others of the same type as 1–3.  Three of these are given by Heiberg
in brackets; the fourth he omits altogether.

The three are:

(\emph{a}) \emph{If equals be added to unequals, the wholes are
  unequal.}

(\emph{b}) \emph{Things which are double of the same thing are equal
  to one another.}

(\emph{c}) \emph{Things which are halves of the same thing are equal
  to one another.}

The fourth, which was placed between (\emph{a}) and~(\emph{b}), was:

(\emph{d})~\emph{If equals be subtracted front unequals, the
  remainders are unequal.}

Proclus, in observing that axioms ought not to be multiplied,
indicates that all should be rejected which follow from the five
admitted by him and appearing in the text above (p.~155).  He mentions
the second of those just quoted~(\emph{b}) as one of those to be
excluded, since it follows from \emph{Common Notion}~\ref{cn:1}.
Proclus does not mention (\emph{a}), (\emph{c}) or~(\emph{d});
an-Nairīzī gives (\emph{a}), (\emph{d}), (\emph{b}) and~(\emph{c}), in
that order, as Euclid's, adding a note of Simplicius that ``three
axioms (sententiae acceptae) only are extant in the ancient
manuscripts, but the number was increased in the more recent.''

(\emph{a}) stands self-condemned because ``unequal'' tells us nothing.
It is easy to see what is wanted if we refer to \prop{1}{17}, where
the same angle is added to a \emph{greater} and a \emph{less}, and it
is inferred that the first sum is greater than the second.  So far
however as the wording of~(\emph{a}) is concerned, the addition of
equal to \emph{greater} and \emph{less} might be supposed to produce
\emph{less} and \emph{greater} respectively.  If therefore such an
axiom were given at all, it should be divided into two.  Heiberg
conjectures that this axiom may have been taken from the commentary of
Pappus, who had the axiom about equals added to unequais quoted
below~(\emph{e}); if so, it can only be an unskilful adaptation of
some remark of Pappus, for his axiom (\emph{e}) has some point,
whereas (\emph{a}) is useless.

As regards (\emph{b}), I agree with Tannery in seeing no sufficient
reason why, if we reject it (as we certainly must), the words in
\prop{1}{47} ``But things which are double of equals are equal to one
another ``should be condemned as an interpolation.  If they were
interpolated, we should have expected to find the same interpolation
in \prop{1}{42}, where the axiom is \emph{tacitly} assumed.  I think
it quite possible that Euclid may have inserted such words in one case
and left them out in another, without necessarily implying either that
he was quoting a formal \emph{Common Notion} of his own or that he had
\emph{not} included among his Common Notions the particular fact
stated as obvious.

The corresponding axiom (\emph{c}) about the halves of equals can
hardly be genuine if (\emph{b}) is not, and Proclus does not mention
it.  Tannery acutely observes however that, when Heiberg, in
\prop{1}{37}, \prop*{1}{38}, brackets words stating that ``the halves
of equal things are equal to one another'' on the ground that axiom
(\emph{c}) was interpolated (although before Theon's time), and
explains that Euclid used \emph{Common Notion}~\ref{cn:3} in making
his inference, he is clearly mistaken.  For, while axiom (\emph{b}) is
an obvious inference from \emph{Common Notion}~\ref{cn:2}, axiom
(\emph{c}) is not an inference from \emph{Common Notion}~\ref{cn:3}.
Tannery says, in a note, that (\emph{c}) would have to be established
by \emph{reductio ad absurdum} with the help of axiom~(\emph{b}), that
is to say, of \emph{Common Notion}~\ref{cn:2}.  But, as the hypothesis
in the \emph{reductio ad absurdum} would be that one of the halves is
\emph{greater} than the other, and it would therefore be necessary to
prove that the one whole is greater than the other, while axiom
(\emph{b}) or \emph{Common Notion}~\ref{cn:2} only refers to
\emph{equals}, a little argument would be necessary in addition to the
reference to \emph{Common Notion}~\ref{cn:2}.  I think Euclid would
not have gone through this process in order to prove~(\emph{c}), but
would have assumed it as equally obvious with~(\emph{b}).

Proclus (pp.~197, 6–198, 5) definitely rejects two other axioms of the
above kind given by Pappus, observing that, as they follow from the
genuine axioms, they are rightly omitted in most copies, although
Pappus said that they were ``on record'' with the others
(\greek{συναναγράφεσθαι}):

(\emph{e})~\emph{If unequals be added to equals, the difference
  between the wholes is equal to the difference between the added
  parts}; and

(\emph{f})~\emph{If equals be added to unequals, the difference
  between the wholes is equal to the difference between the original
  unequals.}

Proclus and Simplicius (in an-Nairīzī) give proofs of both.  The proof
of the former, as given by Simplicius, is as follows:

\sidefig{cn_3}

Let $AB$, $CD$ be equal magnitudes; and let $EB$, $FD$ be added to
them respectively, $EB$ being greater than~$FD$.

I say that $AE$ exceeds $CF$ by the same difference as that by which
$BE$ exceeds~$DF$.

Cut off from $BE$ the magnitude $BG$ equal to~$DF$.

Then, since $AE$ exceeds $AG$ by~$GE$, and $AG$ is equal to~$CF$ and
$BG$ to~$DF$,

$AE$ exceeds $CF$ by the same difference as that by which $BE$
exceeds~$DF$.

\section*{Common Notion 4}

\greek{Καὶ τὰ ἐφαρμόζοντα ἐπ’ ἄλληλα ἴσα ἀλλήλοις ἐστίν.}

\emph{Things which coincide with one another are equal to one another.}

The word \greek{ἐφαρμόζειν}, as a geometrical term, has a different
meaning according as it is used in the active or in the passive.  In
the passive, \greek{ἐφαρμόζεσθαι}, it means ``to be \emph{applied}
to'' without any implication that the applied figure will exactly fit,
or coincide with, the figure to which it is applied; on the other hand
the active \greek{ἐφαρμόζειν} is used intransitively and means ``to
fit exactly,'' ``to coincide with.'' In Euclid and Archimedes
\greek{ἐφαρμόζειν} is constructed with \greek{ἐπί} and the accusative,
in Pappus with the dative.

On \emph{Common Notion}~\ref{cn:4} Tannery observes that it is
incontestably geometrical in character, and should therefore have been
excluded from the \emph{Common Notions}; again, it is difficult to see
why it is not accompanied by its converse, at all events for straight
lines (and, it might be added, angles also), which Euclid makes use of
in \prop{1}{4}.  As it is, says Tannery, we have here a definition of
geometrical equality more or less sufficient, but not a real axiom.

It is true that Proclus seems to recognise this \emph{Common Notion}
and the next as proper axioms in the passage (p.~196, 15–21) where he
says that we should not cut down the axioms to the minimum, as Heron
does in giving only three axioms; but the statement seems to rest, not
upon authority, but upon an assumption that Euclid would state
explicitly at the beginning all axioms subsequently used and not
reducible to others unquestionably included.  Now in \prop{1}{4} this
\emph{Common Notion} is not quoted; it is simply inferred that ``the
base $BC$ will coincide with~$EF$, \emph{and will be equal to it}.''
The position is therefore the same as it is in regard to the statement
in the same proposition that, ``if…the base $BC$ does not coincide
with~$EF$, two straight lines will enclose a space: which is
impossible''; and, if we do not admit that Euclid had the axiom that
``two straight lines cannot enclose a space,'' neither need we infer
that he had \emph{Common Notion}~\ref{cn:4}.  I am therefore inclined
to think that the latter is more likely than not to be an
interpolation.

It seems clear that the Common Notion, as here formulated, is intended
to assert that superposition is a legitimate way of proving the
equality of two figures which have the necessary parts respectively
equal, or, in other words, to serve as an \emph{axiom of congruence}.

The phraseology of the propositions, e.g.\ \prop{1}{4} and
\prop{1}{8}, in which Euclid employs the method indicated, leaves no
room for doubt that he regarded one figure as actually \emph{moved}
and \emph{placed upon} the other.  Thus in \prop{1}{4} he says, ``The
triangle $ABC$ being applied (\greek{ἐφαρμοζομένου}) to the triangle
$DEF$, and the point $A$ being \emph{placed} (\greek{τιθεμένου}) upon
the point~$D$, and the straight line $AB$ on~$DE$, the point~$B$ will
also coincide with~$E$ because $AB$ is equal to~$DE''$; and in
\prop{1}{8}, ``If the sides $BA$, $AC$ do not coincide with $ED$,
$DF$, but \emph{fall beside them} (take a different position,
\greek{παραλλάξουσιν}), then'' etc.  At the same time, it is clear
that Euclid disliked the method and avoided it wherever he could,
e.g.\ in \prop{1}{26}, where he proves the equality of two triangles
which have two angles respectively equal to two angles and one side of
the one equal to the corresponding side of the other.  It looks as
though he found the method handed down by tradition (we can hardly
suppose that, if Thales proved that the diameter of a circle divides
it into two equal parts, he would do so by any other method than that
of superposition), and followed it, in the few cases where he does so,
only because he had not been able to see his way to a satisfactory
substitute.  But seeing how much of the \emph{Elements} depends on
\prop{1}{4}, directly or indirectly, the method can hardly be regarded
as being, in Euclid, of only subordinate importance, on the contrary,
it is fundamental.  Nor, as a matter of fact, do we find in the
ancient geometers any expression of doubt as to the legitimacy of the
method.  Archimedes uses it to prove that any spheroidal figure cut by
a plane through the centre is divided into two equal parts in respect
of both its surface and its volume; he also postulates in
\emph{Equilibrium of Planes}~\r1.\ that ``when equal and similar plane
figures coincide if applied to one another, their centres of gravity
coincide also.''

Killing (\emph{Einführung in die Grundlagen der Geometrie},
\r2.\ pp.~4,~5) contrasts the attitude of the Greek geometers with
that of the philosophers, who, he says, appear to have agreed in
banishing motion from geometry altogether.  In support of this he
refers to the view frequently expressed by Aristotle that mathematics
has to do with immovable objects (\greek{ἀκίνητα}), and that only
where astronomy is admitted as part of mathematical science is motion
mentioned as a subject for mathematics.  Cf.\ \emph{Metaph.}\ 989~b~32
``For mathematical objects are among things which exist apart from
motion, except such as relate to astronomy'';
\emph{Metaph.}\ 1064~a~30 ``Physics deals with things which have in
themselves the principle of motion; mathematics is a theoretical
science and one concerned with things which are \emph{stationary}
(\greek{μένοντα}) but not separable'' (sc.\ from matter); in
\emph{Physics}~\r2.~2, 193~b~34 he speaks of the subjects of
mathematics as ``in thought separable from motion.''

But I doubt whether in Aristotle's use of the words ``immovable,''
``without motion'' etc.\ as applied to the subjects of mathematics
there is any implication such as Killing supposes.  We arrive at
mathematical concepts by abstraction from material objects; and just
as we, in thought, eliminate the matter, so according to Aristotle we
eliminate the attributes of matter as such, e.g.\ qualitative change
and \emph{motion}.  It does not appear to me that the use of
``immovable'' in the passages referred to means more than this.  I do
not think that Aristotle would have regarded it as illegitimate to
\emph{move} a geometrical figure from one position to another; and I
infer this from a passage in \emph{De caelo} \r3.~1 where he is
criticising ``those who make up every body that has an origin by
putting together \emph{planes}, and resolve it again into
\emph{planes}.''  The reference must be to the \emph{Timaeus}
(54~\textsc{b}~sqq.)\ where Plato evolves the four elements in this
way.  He begins with a right-angled triangle in which the hypotenuse
is double of the smaller side; six of these put together in the proper
way produce one equilateral triangle.  Making solid angles with
(\emph{a})~three, (\emph{b})~four, and (\emph{c})~five of these
equilateral triangles respectively, and taking the requisite number of
these solid angles, namely four of~(\emph{a}), six of~(\emph{b}) and
twelve of~(\emph{c}) respectively, and putting them together so as to
form regular solids, he obtains (α)~a tetrahedron, (β)~an octahedron,
(γ)~an icosahedron respectively.  For the fourth element (earth), four
isosceles right-angled triangles are first put together so as to form
a square, and then six of these squares are put together to form a
cube.  Now, says Aristotle (299~b~23), ``it is absurd that planes
should only admit of being put together so as to touch in a
\emph{line}; for just as a line and a line are put together in both
ways, lengthwise and breadthwise, so must a plane and a plane.  A line
can be combined with a line in the sense of being a line
\emph{superposed}, and not \emph{added}''; the inference being that a
\emph{plane} can be superposed on \emph{plane}.  Now this is precisely
the sort of motion in question here; and Aristotle, so far from
denying its permissibility, seems to blame Plato for not using it.
Cf.\ also \emph{Physics} \r5.~4, 228~b~25, where Aristotle speaks of
``the spiral or other magnitude in which any part will not coincide
with any other part,'' an where superposition is obviously
contemplated.

\subsubsection*{Motion without deformation}

It is well known that Helmholtz maintained that geometry requires us
to assume the actual existence of rigid bodies and their free mobility
in space, whence he inferred that geometry is dependent on mechanics.

Veronese exposed the fallacy in this (\emph{Fondamenti di geometria},
pp.~\r35–\r36, 239–240 note, 615—7), his argument being as follows.
Since geometry is concerned with empty space, which is immovable, it
would be at least strange if it was necessary to have recourse to the
real motion of bodies for a definition, and for the proof of the
properties, of immovable space.  We must distinguish the intuitive
principle of motion in itself from that of motion without deformation.
Every point of a figure which moves is transferred to another point in
space. ``Without deformation'' means that the mutual relations between
the points of the figure do not change, but the relations between them
and other figures do change (for if they did not, the figure could not
move).  Now consider what we mean by saying that, when the figure~$A$
has moved from the position~$A_1$, to the position~$A_2$ it the
relations between the points of~$A$ in the position~$A_2$, are
unaltered from what they were in the position~$A_1$ are the same in
fact as if $A$ had not moved but remained at~$A_1$.  We can only say
that, judging of the figure (or the body with its physical qualities
eliminated) by the impressions it produces in us during its movement,
the impressions produced in us in the two different positions (which
are in time distinct) are equal.  In fact, we are making use of the
notion of equality between two distinct figures.  Thus, if we say that
two bodies are equal when they can be superposed by means of
\emph{movement without deformation}, we are committing a \emph{petitio
  principii}.  The notion of the equality of spaces is really prior to
that of rigid bodies or of motion without deformation.  Helmholtz
supported his view by reference to the process of measurement in which
the measure must be, at least approximately, a rigid body, but the
existence of a rigid body as a standard to measure by, and the
question how we discover two equal spaces to be equal, are matters of
no concern to the geometer.  The method of superposition, depending on
motion without deformation, is only of use as a \emph{practical} test;
it has nothing to do with the \emph{theory} of geometry.

Compare an acute observation of Schopenhauer (\emph{Die Welt als
  Wille}, 2~ed.\ 1844, 11. p.~130) which was a criticism in advance of
Helmholtz' theory: ``I am surprised that, instead of the eleventh
axiom [the Parallel-Postulate], the eighth is not rather attacked:
'Figures which coincide (sich decken) are equal to one another.'  For
\emph{coincidence} (das Sichdecken) is either mere tautology, or
something entirely empirical, which belongs, not to pure intuition
(Anschauung), but to external sensuous experience.  It presupposes in
fact the mobility of figures; but that which is movable in space is
matter and nothing else.  Thus this appeal to coincidence means
leaving pure space, the sole element of geometry, in order to pass
over to the material and empirical.''

Mr~Bertrand Russell observes (\emph{Encyclopaedia Britannica},
Suppl.\ Vol.~4, 1902, Art.\ ``Geometry, non-Euclidean'') that the
apparent use of motion here is deceptive; what in geometry is called a
motion is merely the transference of our attention from one figure to
another.  Actual superposition, which is nominally employed by Euclid,
is not required; all that is required is the transference of our
attention from the original figure to a new one defined by the
position of some of its elements and by certain properties which it
shares with the original figure.

If the method of superposition is given up as a means of defining
theoretically the equality of two figures, some other definition of
equality is necessary.  But such a definition can be evolved out of
\emph{empirical} or \emph{practical} observation of the result of
superposing two material representations of figures.  This is done by
Veronese (\emph{Elementi di geometria}, 1904) and Ingrami
(\emph{Elementi di geometria}, 1904). Ingrami says, namely (p.~66);

``If a sheet of paper be folded double, and a triangle be drawn upon
it and then cut out, we obtain two triangles \emph{superposed} which
we in practice call \emph{equal}.  If points $A$, $B$, $C$, $D$… be
marked on one of the triangles, then, when we place this triangle upon
the other (so as to coincide with it), we see \emph{that} each of the
particular points taken on the first is superposed on one particular
point of the second in such a way that the segments $AB$, $AC$, $AD$,
$BC$, $BD$, $CD$, … are respectively superposed on as many segments in
the second triangle and are therefore equal to them respectively.  In
this way we justify the following

``\textbf{Definition of equality.}

``Any two figures whatever will be called \emph{equal} when to the
points of one the points of the other can be made to correspond
\emph{univocally} [i.e.\ every \emph{one} point in one to \emph{one
    distinct} point in the other and \emph{vice versa}] in such a way
that the segments which join the points, two and two, in one figure
are respectively equal to the segments which join, two and two, the
corresponding points in the other.''

Ingrami has of course previously postulated as known the signification
of the phrase \emph{equal (rectilineal) segments}, of which we get a
\emph{practical} notion when we can place one upon the other or can
place a third movable segment successively on both.

\subsubsection*{New systems of Congruence-Postulates}

In the fourth Article of \emph{Questioni riguardanti le matematische
  etementari}, \r1., pp.~93–122, a review is given of three different
systems: (1)~that of Pasch in \emph{Vorlesungen über neuere
  Geometrie}, 1882, p.~101 sqq., (2)~that of Veronese according to the
\emph{Fondamenti di geometria}, 1891, and the \emph{Elementi} taken
together, (3)~that of Hilbert (see \emph{Grundlagen der Geometrie},
1903, pp.~7—15).

These systems differ in the particular conceptions taken by the three
authors as primary.  (1)~Pasch considers as primary the notion of
\emph{congruence} or \emph{equality} between \emph{any figures which
  are made up of a finite number of points only}.  The definitions of
congruent \emph{segments} and of congruent \emph{angles} have to be
\emph{deduced} in the way shown on pp.~102–103 of the Article referred
to, after which Eucl.\ \prop{1}{4} follows immediately, and
Eucl.\ \prop{1}{26}~(1) and \prop{1}{8} by a method recalling that in
Eucl.\ \prop{1}{7},~\prop*{1}{8}.

(2)~Veronese takes as primary the conception of congruence between
\emph{segments} (rectilineal).  The transition to congruent
\emph{angles}, and thence to \emph{triangles} is made by means of the
following postulate:

``Let $AB$, $AC$ and $A'B'$, $A'C'$ be two pairs of straight lines
intersecting at $A$, $A'$, and let there be determined upon them the
congruent segments $AB$, $A'B'$ and the congruent segments $AC$,
$A'C'$;

then, if $BC$, $B'C'$ are congruent, the two \emph{pairs of straight
  lines} are congruent.''

(3)~Hilbert takes as primary the notions of congruence between
\emph{both segments and angles}.

It is observed in the Article referred to that, from the theoretical
standpoint, Veronese's system is an advance upon that of Pasch, since
the idea of congruence between \emph{segments} is more simple than
that of congruence between \emph{any figures}; but, didactically, the
development of the theory is more complicated when we start from
Veronese's system than when we start from that of Pasch.

The system of Hilbert offers advantages over both the others from the
point of view of the teaching of geometry, and I shall therefore give
a short account of his system only, following the Article above
quoted.

\textbf{Hilbert's system.}

The following are substantially the Postulates laid down,

(1)~\emph{If one segment is congruent with another, the second is also
  congruent with the first.}

(2)~\emph{If an angle is congruent with another angle, the second
  angle is also congruent with the first.}

(3)~\emph{Two segments congruent with a third are congruent with one
  another.}

(4)~\emph{Two angles congruent with a third are congruent with one
  another.}

(5)~\emph{Any segment $AB$ is congruent with itself, independently of
  its sense.}  This we may express symbolically thus:
\[
    AB \equiv AB \equiv BA.
\]

(6)~\emph{Any angle $(ab)$ is congruent with itself, independently of
  its sense.}  This we may express symbolically thus:
\[
    (ab) \equiv (ab) \equiv (6a).
\]

(7)~\emph{On any straight line $r'$ starting from any one of its
  points $A'$, and on each side of it respectively, there exists one
  and only one segment congruent with a segment $AB$ belonging to the
  straight line~$r$.}

(8)~\emph{Given a ray~$a$, issuing from a point~$O$, in any plane which
  contains it and on each of the two sides of it, there exists one and
  only one ray~$b$ issuing from~$O$ such that the angle~$(ab)$ is
  congruent with a given angle~$(a'b')$.}

(9)~\emph{If $AB$, $BC$ are two consecutive segments of the same
  straight line~$r$ (segments, that is, having an extremity and no
  other point common), and $A'B'$, $B'C'$ two consecutive segments on
  another straight line~$r'$, and if $AB \equiv A'B'$, $BC \equiv
  B'C'$, then}
\[
    AC \equiv A'C'.
\]

(10)~\emph{If $(ab)$, $(bc)$ are two consecutive angles in the same
  plane~$\pi$ (angles, that is, having the vertex and one side
  common), and $(a'b')$, $(b'c')$ two consecutive angles in another
  plane~$\pi'$, and if $(ab) \equiv (a'b')$, $(bc) \equiv (b'c')$,
  then}
\[
    (ac) = (a'c').
\]

(11)~\emph{If two triangles have two sides and the included angles
  respectively congruent, they have also their third sides congruent
  as well as the angles opposite to the congruent sides respectively.}

\sidefig{cn_4a}

As a matter of fact, Hilbert's postulate corresponding to~(11) does
not assert the equality of the third sides in each, but only the
equality of the two remaining angles in one triangle to the two
remaining angles in the other respectively.  He proves the equality of
the third sides (thereby completing the theorem of Eucl.~\prop{1}{4})
by \emph{reductio ad absurdum} thus. Let $ABC$, $A'B'C'$ be the two
triangles which have the sides $AB$, $AC$ respectively congruent with
the sides $A'B'$, $A'C'$ and the included angle at~$A$ congruent with
the included angle at~$A'$.

Then, by Hilbert's own postulate, the angles $ABC$, $A'B'C'$ are
congruent, as also the angles $ACB$, $A'C'B'$.

If $BC$ is not congruent with $B'C'$, let $D$ be taken on $B'C'$ such
that $BC$, $B'D$ are congruent and join~$A'D$.

Then the two triangles $ABC$, $A'B'D$ have two sides and the included
angles congruent respectively; therefore, by the same postulate, the
angles $BAC$, $B'A'D$ are congruent.

But the angles $BAC$, $B'A'C'$ are congruent; therefore, by (4) above,
the angles $B'A'C'$, $B'A'D$ are congruent: which is impossible, since
it contradicts (8) above.

Hence $BC$, $B'C'$ cannot but be congruent.

Eucl.\ \prop{1}{4} is thus proved; but it seems to be as well to
include all of that theorem in the postulate, as is done in (11)
above, since the two parts of it are equally suggested by empirical
observation of the result of one superposition.

A proof similar to that just given immediately establishes
Eucl.\ \prop{1}{26}~(1), and Hilbert next proves that

\emph{If two angles $ABC$, $A'B'C'$ are congruent with one another,
  their supplementary angles $CBD$, $C'B'D'$ are also congruent with
  one another.}

We choose $A$, $D$ on one of the straight lines forming the first
angle, and $A'$, $D'$ on one of those forming the second angle, and
\infig{cn_4b}
again $C$, $C'$ on the other
straight lines forming the angles, so that $A'B'$ is congruent with
$AB$, $C'B'$ with $CB$, and $D'B'$ with~$DB$.

The triangles $ABC$, $A'B'C'$ are congruent, by (11) above; and $AC$
is congruent with $A'C'$, and the angle $CAB$ with the angle $C'A'B'$.

Thus, $AD$, $A'D'$ being congruent, by~(9), the triangles $CAD$,
$C'A'D'$ are also congruent, by~(11);

whence $CD$ is congruent with $C'D'$, and the angle $ADC$ with the
angle~$A'D'C'$.

Lastly, by~(11), the triangles $CDB$, $C'D'B'$ are congruent, and the
angles $CBD$, $C'B'D'$ are thus congruent.

Hilbert's next proposition is that

\emph{Given that the angle $(h, k)$ in the plane~$\alpha$ is congruent
  with the angle $(h', k')$ in the plane~$\alpha'$, and that $l$ is a
  half-ray in the plane a. starting from the vertex of the angle $(h,
  k)$ and lying within that angle, there always exists a half-ray~$l'$
  in the second plane~$\alpha'$, starting from the vertex of the angle
  $(h', k')$ and lying within that angle, such that}
\[
    (h,l) \equiv (h',l'),\quad\text{and}\quad (k, l) = (k',l').
\]

If $O$, $O'$ are the vertices, we choose points $A$, $B$ on $h$, $k$,
and points $A'$, $B'$ on $h'$, $k'$ respectively, such that $OA$,
$O'A'$ are congruent and also $OB$, $O'B'$.

\infig{cn_4c}

The triangles $OAB$, $O'A'B'$ are then congruent; and, if $l$ meets
$AB$ in~$C$, we can determine $C'$ on $A'B'$ such that $A'C'$ is
congruent with~$AC$.

Then $l'$ drawn from $O'$ through~$C'$ is the half-ray required.

The congruence of the angles $(h, l)$, $(h', l')$ follows from~(11)
directly, and that of $(k, l)$ and $(k', l')$ follows in the same way
after we have inferred by means of~(9) that, $AB$, $AC$ being
respectively congruent with $A'B'$, $A'C'$, the difference $BC$ is
congruent with the difference $B'C'$.

It is by means of the two propositions just given that Hilbert proves
that \emph{All right angles are congruent with one another}.

Let the angle $BAD$ be congruent with its adjacent angle $CAD$, and
likewise the angle $B'A'D'$ congruent with its adjacent angle $C'A'D$.
All four angles are then right angles.

\infig{cn_4d}

If the angle $B'A'D'$ is not congruent with the angle $BAD$, let the
angle with $AB$ for one side and congruent with the angle $B'A'D'$ be
the angle $BAD''$, so that $AD''$ falls either within the angle $BAD$
or within the angle $DAC$.  Suppose the former.

By the last proposition but one (about adjacent angles), the angles
$B'A'D'$, $BAD''$ being congruent, the angles $C'A'D'$, $CAD''$ are
congruent.

Hence, by the hypothesis and postulate~(4) above, the angles $BAD''$,
$CAD''$ are also congruent.

And, since the angles $BAD$, $CAD$ are congruent, we can find within
the angle $CAD$ a half-ray $CAD'''$ such that the angles $BAD''$,
$CAD'''$ are congruent, and likewise the angles $DAD''$, $DAD''$ (by
the last proposition).

But the angles $BAD''$, $CAD''$ were congruent (see above); and it
follows, by~(4), that the angles $CAD''$, $CAD'''$ are congruent:
which is impossible, since it contradicts postulate~(8),

Therefore etc.

Euclid \prop{1}{5} follows directly by applying the postulate~(11)
above to $ABC$, $ACB$ as distinct triangles.

Postulates (9), (10) above give in substance the proposition that
``the sums or differences of segments, or of angles, respectively
equal, are equal.''

\sidefig{cn_4e}

Lastly, Hilbert proves Eucl.\ \prop{1}{8} by means of the theorem of
Eucl.\ \prop{1}{5} and the proposition just stated as applied to
angles.

$ABC$, $A'B'C'$ being the given triangles with three sides
respectively congruent, we suppose an angle $CBA''$ to be determined,
on the side of $BC$ opposite to~$A$, congruent with the angle
$A'B'C'$, and we make $BA''$ equal to~$A'B'$.

The proof is obvious, being equivalent to the alternative proof often
given in our text-books for Eucl.\ \prop{1}{8}.

\section*{Common Notion 5}

\greek{καὶ τὸ ὅλοντοῦ μέρους μεῖζόν [ἐστιν].}

\emph{The whole is greater than the part.}

Proclus includes this ``axiom ``on the same ground as the preceding
one.  I think however there is force in the objection which Tannery
takes to it, namely that it replaces a \emph{different} expression in
Eucl.\ \prop{1}{6}, where it is stated that ``the triangle $DBC$ will
be equal to the triangle $ACB$, \emph{the less to the greater: which
  is absurd}.''  The axiom appears to be an abstraction or
generalisation substituted for an immediate inference from a
geometrical figure, but it takes the form of a sort of definition of
whole and part.  The probabilities seem to be against its being
genuine, notwithstanding Proclus' approval of it.

Clavius added the axiom that \emph{the whole is the equal to the sum
  of its parts}.

\chapter*{Other Axioms introduced after Euclid's time}

[9]~\emph{Two straight lines do not enclose (\emph{or} contain) a space.}

Proclus (p.~196, 21) mentions this in illustration of the undue
multiplication of axioms, and he points out, as an objection to it,
that it belongs to the subject matter of geometry, whereas axioms are
of a general character, and not peculiar to any one science.  The real
objection to the axiom is that it is unnecessary, since the fact which
it states is included in the meaning of Postulate~\ref{post:1}.  It
was no doubt taken from the passage in \prop{1}{4}, ``if…the base $BC$
does not coincide with the base $EF$, \emph{two straight lines will
  enclose a space: which is impossible}''; and we must certainly
regard it as an interpolation, notwithstanding that two of the best
\textsc{mss.}\ have it after Postulate~\ref{post:5}, and one gives it
as \emph{Common Notion}~9.

Pappus added some others which Proclus objects to (p.~198, 5) because
they are either anticipated in the definitions or follow from them.

(\emph{g})~\emph{All the parts of a plane, or of a straight line,
  coincide with one another.}

(\emph{h})~\emph{A point divides a line, a line a surface, and a
  surface a solid}x; on which Proclus remarks that everything is
\emph{divided} by the same things as those by which it is
\emph{bounded}.

An-Nairīzī (ed.\ Besthorn-Heiberg, p.~31, ed.\ Curtze, p.~38) in his version
of this axiom, which be also attributes to Pappus, omits the reference to
solids, but mentions planes as a particular case of surfaces.

``(α)~\emph{A Surface cuts a surface in a line;}

(β)~\emph{If two surfaces which cut one another are plane, they cut
  one another in a straight line;}

(γ)~\emph{A line cuts a line in a point} (this last we need in the
first proposition).''

(\emph{k})~\emph{Magnitudes are susceptible of the infinite (or
  unlimited) both by way of addition and by way of successive
  diminution, but in both cases potentially only (\greek{τὸἄπειρον ἐν
    τοῖς μεγέθεσίν ἐστιν καὶ τῇ προσθέσει καὶ τῇ ἐπικαθαιρέσει,
    δυνάμει δὲ ἑκάτερον}).}

An-Nairīzī's version of this refers to straight lines and plane
surfaces only: ``\emph{as regards the straight line and the plane
  surface, in consequence of their evenness, it is possible to produce
  them indefinitely}.''

This ``axiom'' of Pappus, as quoted by Proclus, seems to be taken
directly from the discussion of to \greek{τὸ ἄπειρον} in Aristotle,
\emph{Physics} \r3.~5—8, even to the wording, for, while Aristotle
uses the term \emph{division} (\greek{διαίρεσις}) most frequently as
the antithesis of \emph{addition} (\greek{σύνθεσις}), he occasionally
speaks of \emph{subtraction} (\greek{ἀφαίρεσις}) and \emph{diminution}
(\greek{καθαίρεσις}).  Hankel (\emph{Zur Geschichte der Mathematik im
  Alterthum und Mittelalter}, 1874, pp.~119–120) gave an admirable
summary of Aristotle's views on this subject; and they are stated in
greater detail in Görland, \emph{Aristoteles und die Mathematik},
Marburg, 1899, pp.~157–183.  The infinite or unlimited
(\greek{ἄπειρον}) only exists potentially (\greek{δυνάμει}), not in
actuality (\greek{ἐνεργείᾳ}). The infinite is so in virtue of its
endlessly changing into something else, like day or the Olympic Games
(\emph{Phys.}\ \r3.~6, 206~a~15–25).  The infinite is manifested in
different forms in time, in Man, and in the division of magnitudes.
For, in general, the infinite consists in something new being
continually taken, that something being itself always finite but
always different.  Therefore the infinite must not be regarded as a
particular thing (\greek{τόδε τι}), as man, house, but as being always
in course of becoming or decay, and, though finite at any moment,
always different from moment to moment.  But there is the distinction
between the forms above referred to that, whereas in the case of
magnitudes what is once taken remains, in the case of time and Man it
passes or is destroyed but the succession is unbroken.  The case of
addition is in a sense the same as that of division; in the finite
magnitude the former takes place in the converse way to the latter;
for, as we see the finite magnitude divided \emph{ad infinitum}, so we
shall find that addition gives a sum tending to a definite limit.  I
mean that, in the case of a finite magnitude, you may take a definite
fraction of it and add to it (continually) in the same ratio; if now
the successive added terms do not include one and the same magnitude
whatever it is [i.e.\ if the successive terms diminish in geometrical
  progression], you will not come to the end of the finite magnitude,
but, if the ratio is increased so that each term does include one and
the same magnitude whatever it is, you will come to the end of the
finite magnitude, for every finite magnitude is exhausted by
continually taking from it any definite fraction whatever.  Thus in no
other sense does the infinite exist, but only in the sense just
mentioned, that is, potentially and by way of diminution
(206~a~25–b~13).  And in this sense you may have potentially infinite
addition, the process being, as we say, in a manner, the same as with
division \emph{ad infinitum}: for in the case of addition you will
always be able to find something outside the total for the time being,
but the total will never exceed every definite (or assigned) magnitude
in the way that, in the direction of division, the result will pass
every definite magnitude, that is, by becoming smaller than it.  The
infinite therefore cannot exist even potentially in the sense of
exceeding every finite magnitude as the result of successive addition
(206~b~16—22).  It follows that the correct view of the infinite is
the opposite of that commonly held: it is not that which has nothing
outside it, but that which always has something outside it
(206~b~33–207~a~1).

Contrasting the case of number and magnitude, Aristotle points out
that (1)~in number there is a limit in the direction of smallness,
namely unity, but none in the other direction: a number may exceed any
assigned number however great; but (2)~with magnitude the contrary is
the case: you can find a magnitude smaller than any assigned
magnitude, but in the other direction there is no such thing as an
infinite magnitude (207~b~1—5).  The latter assertion he justified by
the following argument. However large a thing can be potentially, it
can be as large actually.  But there is no magnitude perceptible to
sense that is infinite.  Therefore excess over every assigned
magnitude is an impossibility; otherwise there would be something
larger than the universe (\greek{αὀρανός}) (207~b~17–21).

Aristotle is aware that it is essentially of physical magnitudes that
he is speaking.  He had observed in an earlier passage
(\emph{Phys.}\ \r3.~5, 204~a~34) that it ts perhaps a more general
inquiry that would be necessary to determine whether the infinite is
possible in mathematics, and in the domain of thought and of things
which have no magnitude; but he excuses himself from entering upon
this inquiry on the ground that his subject is physics and sensible
objects.  He returns however to the bearing of his conclusions on
mathematics in \r3.~7, 207~b~27: ``my argument does not even rob
mathematicians of their study, although it denies the existence of the
infinite in the sense of actual existence as something increased to
such an extent that it cannot be gone through (\greek{ἀδιεξίτητον});
for, as it is, they do not even need the infinite or use it, but only
require that the finite (straight line) shall be as long \emph{as they
  please}; and another magnitude of any size whatever can be cut in
the same ratio as the greatest magnitude.  Hence it will make no
difference to them for the purpose of demonstration.''

Lastly, if it should be urged that the infinite exists in
\emph{thought}, Aristotle replies that this does not involve its
existence in \emph{fact}.  A thing is not greater than a certain size
because it is conceived to be so, but because it is; and magnitude is
not infinite in virtue of increase in thought (208~a~16–22).

Hankel and Görland do not quote the passage about an infinite series
of magnitudes (206~b~3—13) included in the above paraphrase; but I
have thought that mathematicians would be interested in the distinct
expression of Aristotle's view that the existence of an infinite
series the terms of which are magnitudes is impossible unless it is
convergent, and (with reference to Riemann's developments) in the
statement that it does not matter to geometry if the straight line is
not infinite in length, provided that it is as long as we please.

Aristotle's denial of even the potential existence of a sum of
magnitudes which shall exceed every definite magnitude was, as he
himself implies, in conflict with the lemma or assumption used by
Eudoxus (as we infer from Archimedes) to prove the theorem about the
volume of a pyramid. The lemma is thus stated by Archimedes
(\emph{Quadrature of a parabola}, preface): ``The excess by which the
greater of two unequal areas exceeds the less can, if it be
continually added to itself, be made to exceed any assigned finite
area.''  We can therefore well understand why, a century later,
Archimedes felt it necessary to justify his own use of the lemma as he
does in the same preface; ``The earlier geometers too have used this
lemma: for it is by its help that they have proved that circles have
to one another the duplicate ratio of their diameters, that spheres
have to one another the triplicate ratio of their diameters, and so
on.  And, in the result, each of the said theorems has been accepted
no less than those proved without the aid of this lemma.''

\subsubsection*{Principle of continuity}

The use of actual construction as a method of proving the existence of
figures having certain properties is one of the characteristics of the
\emph{Elements}.  Now constructions are effected by means of straight
lines and circles drawn in accordance with
Postulates~\ref{post:1}–\ref{post:3}; the essence of them is that such
straight lines and circles determine by their intersections other
points in addition to those given, and these points again are used to
determine new lines, and so on.  This being so, the existence of such
points of intersection must be postulated or proved in the same way as
that of the lines which determine them.  Yet there is no postulate of
this character expressed in Euclid except Post.~\ref{post:5}.  This
postulate asserts that two straight lines meet if they satisfy a
certain condition.  The condition is of the nature of a
\greek{διορισμός} (\emph{discrimination}, or condition of possibility)
in a problem; and, if the existence of the point of intersection were
not granted, the solutions of problems in which the points of
intersection of straight lines are used would not in general furnish
the required proofs of the existence of the figures to be constructed.

But, equally with the intersections of straight lines, the
intersections of circle with straight line, and of circle with circle,
are used in constructions.  Hence, in addition to
Postulate~\ref{post:5}, we require postulates asserting the actual
existence of points of intersection of circle with straight line and
of circle with circle.  In the very first proposition the vertex of
the required equilateral triangle is determined as one of the
intersections of two circles, and we need therefore to be assured that
the circles will intersect.  Euclid seems to assume it as obvious,
although it is not so; and he makes a similar assumption in
\prop{1}{22}.  It is true that in the latter case Euclid adds to the
enunciation that two of the given straight lines must be together
greater than the third; but there is nothing to show that, if this
condition is satisfied, the construction is always possible.  In
\prop{1}{12}, in order to be sure that the circle with a given centre
will intersect a given straight line, Euclid makes the circle pass
through a point on the side of the line opposite to that where the
centre is.  It appears therefore as if, in this case, he based his
inference in some way upon the definition of a circle combined with
the fact that the point within it called the centre is on one side of
the straight line and one point of the circumference on the other,
and, in the case of two intersecting circles, upon similar
considerations.  But not even in Book~\book{3}, where there are
several propositions about the relative positions of two circles, do
we find any discussion of the conditions under which two circles have
two, one, or no point common.

The deficiency can only be made good by the \emph{Principle of
  Continuity}.

Killing (\emph{Einführung in die Grundlagen der Geometri}, \r2.~p.~43)
gives the following forms as sufficient for most purposes

(\emph{a})~Suppose a line belongs entirely to a figure which is
divided into two parts; then, if the line has at least one point
common with each part, it must also meet the boundary between the
parts; or

(\emph{b})~If a point moves in a figure which is divided into two
parts, and if it belongs at the beginning of the motion to one part
and at the end of the motion to the other part, it must during the
motion arrive at the boundary between the two parts.

In the \emph{Questioni riguardanti le matematiche elementari}, \r1.,
Art.~5, pp.~123—143, the principle of continuity is discussed with
special reference to the Postulate of Dedekind, and it is shown,
first, how the Postulate may be led up to and, secondly, how it may be
applied for the purposes of elementary geometry.

Suppose that in a segment $AB$ of a straight line a point~$C$
determines two segments $AC$, $CB$.  If we consider the point~$C$ as
belonging to only one of the two segments $AC$, $CB$, we have a
division of the segment $AB$ into two parts with the following
properties.

1.~Every point of the segment $AB$ belongs to \emph{one} of the two
parts.

2.~The point $A$ belongs to one of the two parts (which we will call
the \emph{first}) and the point~$B$ to the other; the point~$C$ may
belong indifferently to one or the other of the two parts according as
we choose to premise.

3.~Every point of the first part precedes every point of the second in
the order $AB$ of the segment.

(For generality we may also suppose the case in which the point~$C$
falls at~$A$ or at~$B$.  Considering~$C$, in these cases respectively,
as belonging to the first or second part, we still have a division
into parts which have the properties above enunciated, one part being
then a single point $A$ or~$B$.)

Now, considering carefully the inverse of the above proposition, we
see that it agrees with the idea which we have of the continuity of
the straight line.  Consequently we are induced to admit as a
postulate the following.

\emph{If a segment of a straight line $AB$ is divided into two parts
  so that}

(1)~\emph{every point of the segment $AB$ belongs to one of the parts,}

(2)~\emph{the extremity $A$ belongs to the first part and $B$ to the
  second, and}

(3)~\emph{any point whatever of the first part precedes any point
  whatever of the second part, in the order $AB$ of the segment,}

\emph{there exists a point~$C$ of the segment~$AB$ (which may belong
  either to one part or to the other) such that every point of~$AB$
  that precedes~$C$ belongs to the first part, and every point of~$AB$
  that follows~$C$ belongs to the second part in the division
  originally assumed.}

(If one of the two parts consists of the single point $A$ or~$B$, the
point~$C$ is the said extremity $A$ or~$B$ of the segment.)

This is the Postulate of Dedekind, which was enunciated by Dedekind
himself in the following slightly different form (\emph{Stetigkeit und
  irrationale Zahlen}, 1872, new edition 1905, p.~11).

``\emph{If all points of a straight line fall into two classes such
  that every point of the first class lies to the left of every point
  of the second class, there exists one and only one point which
  produces this division of all the points into two classes, this
  division of the straight line into two parts.}''

The above enunciation may be said to correspond to the intuitive
notion which we have that, if in a segment of a straight line two
points start from the ends and describe the segment in opposite
senses, they meet in a point.  The point of meeting might be regarded
as belonging to both parts, but for the present purpose we must regard
it as belonging to one only and subtracted from the other part.

\emph{Application of Dedckind's postulate to angles.}

If we consider an angle less than two right angles bounded by two rays
$a$, $b$, and draw the straight line connecting~$A$, a point on~$a$,
with~$B$, a point on~$b$, we see that all points on the finite
segment~$AB$ correspond univocally to all the rays of the angle, the
point corresponding to any ray being the point in which the ray cuts
the segment~$AB$; and if a ray be supposed to move about the vertex of
the angle from the position~$a$ to the position~$b$, the corresponding
points of the segment~$AB$ are seen to follow in the same order as the
corresponding rays of the angle~$(ab)$.

Consequently, if the angle $(ab)$ is divided into two parts so that

(1)~each ray of the angle $(ab)$ belongs to one of the two parts,

(2)~the outside ray~$a$ belongs to the first part and the ray~$b$ to
the second,

(3)~any ray whatever of the first part precedes any ray whatever of
the second part,

the corresponding points of the segment $AB$ determine two parts of
the segments such that

(1)~every point of the segment $AB$ belongs to one of the two parts,

(2)~the extremity~$A$ belongs to the first part and $B$ to the second,

(3)~any point whatever of the first part precedes any point whatever
of the second.

But in that case there exists a point $C$ of~$AB$ (which may belong to
one or the other of the two parts) such that every point of~$AB$ that
precedes~$C$ belongs to the first part and every point of~$AB$ that
follows~$C$ belongs to the second part.

Thus exactly the same thing holds of~$c$, the ray corresponding
to~$C$, with reference to the division of the angle~$(ad)$ into two
parts.

It is not difficult to extend this to an angle $(ab)$ which is either
flat or greater than two right angles; this is done (Vitali,
\emph{op.~cit.}\ pp.~126—127) by supposing the angle to be divided
into two, $(ad)$, $(db)$, each less than two right angles, and
considering the three cases in which

(1)~the ray~$d$ is such that all the rays that precede it belong to
the first patt and those which follow it to the second part,

(2)~the ray~$d$ is followed by some rays of the first part,

(3)~the ray~$d$ is preceded by some rays of the second part.

\emph{Application to circular arcs.}

If we consider an arc $AB$ of a circle with centre~$O$, the points of
the arc correspond univocally, and in the same order, to the rays from
the point~$O$ passing through those points respectively, and the same
argument by which we passed from a segment of a straight line to an
angle can be used to make the transition from an angle to an arc.

\subsubsection*{Intersections of a straight line with a circle}

It is possible to use the Postulate of Dedekind to prove that

\emph{If a straight line has one point inside and one point outside a
  circle, it has two points common with the circle.}

For this purpose it is necessary to assume (1)~the proposition with
reference to the perpendicular and obliques drawn from a given point
to a given straight line, namely that of all straight lines drawn from
a given point to a given straight line the perpendicular is the
shortest, and of the rest (the obliques) that is the longer which has
the longer projection upon the straight line, while those are equal
the projections of which are equal, so that for any given length of
projection there are two equal obliques and two only, one on each side
of the perpendicular, and (2)~the proposition that any side of a
triangle is less than the sum of the other two.

Consider the circle~($C$) with centre~$O$, and a straight line~($r$)
with one point~$A$ inside and one point~$B$ outside the circle.

\sidefig{miscI_1}

By the definition of the circle, if $R$ is the radius,
\[
    OA<R, \quad OB>R.
\]

Draw $OP$ perpendicular to the straight line~$r$.

Then $OP < OA$, so that $OP$ is always less than~$R$, and $P$ is
therefore within the circle~$C$.

Now let us fix our attention on the finite segment $AB$ of the
straight line~$r$.  It can be divided into two parts, (1)~that
containing all the points $H$ for which $OH < R$ (i.e.\ points
inside~$C$), and (2)~that containing all the points $K$ for which $OK
\geq R$ (points outside~$C$ or on the circumference of~$C$).

Thus, remembering that, of two obliques from a given point to a given
straight line, that is greater the projection of which is greater, we
can assert that all the points of the segment $PB$ which
\emph{precede} a point inside~$C$ are inside~$C$, and those which
\emph{follow} a point on the circumference of~$C$ or outside~$C$ are
outside~$C$.

Hence, by the Postulate of Dedekind, there exists on the segment $PB$
a point~$M$ such that all the points which precede it belong to the
first part and those which follow it to the second part.

I say that $M$ is common to the straight line~$r$ and the circle~$C$,
or
\[ OM=R.
\]

For suppose, e.g., that $OM < R$.

There will then exist a segment (or length) $\sigma$ less than the
difference between $R$ and~$OM$.

Consider the point $M'$, one of those which \emph{follow}~$M$, such
that $MM'$ is equal to~$\sigma$.

Then, because any side of a triangle is less than the sum of the Other
two,
\[
    OM' < OM + MM'.
\]

But
\[
    OM + MM' = OM + \sigma < R,
\]
whence
\[
    OM' < R,
\]
which is absurd.

A similar absurdity would follow if we suppose that $OM > R$.

Therefore $OM$ mist be equal to~$R$.

It is immediately obvious that, corresponding to the point $M$ on the
segment $PB$ which is common to~$r$ and~$C$, there is another point
on~$r$ which has the same property, namely that which is symmetrical
to~$M$ with respect to~$P$.

And the proposition is proved.

\subsubsection*{Intersections of two circles}

We can likewise use the Postulate of Dedekind to prove that

\emph{If in a given plane a circle~$C$ has one point~$X$ inside and
  one point~$Y$ outside another circle~$C'$, the two circles intersect
  in two points.}

We must first prove the following

\emph{Lemma.}

If $O$, $O'$ are the centres of two circles $C$, $C'$, and $R$, $R'$
their radii respectively, the straight line $OO'$ meets the circle~$C$
in two points $A$, $B$, one of which is inside~$C'$ and the other
outside it.

Now one of these points must fall (1)~on the prolongation of $O'O$
beyond~$O$ or (2)~on $OO'$ itself or (3)~xon the prolongation of $OO'$
beyond~$O$.

\sidefig{miscI_2}

(1) First, suppose $A$ to lie on $O'O$ produced.

Then
\begin{equation}\label{eq:alpha}\tag{$\alpha$}
    A0' = AO + 00' = R + OO'
\end{equation}
But, in the triangle $OO'Y$,
\[
    O'Y < OY + OO',
\]
and, since $0'Y > R'$, $OY=R$,
\[
    R' < R + OO'.
\]
It follows from \eqref{eq:alpha} that $A0'>R$; and therefore lies
\emph{outside}~$C'$.

\sidefig{miscI_3}

(2)~Secondly, suppose $A$ to lie on $00'$.

Then
\begin{equation}\label{eq:beta}\tag{$\beta$}
    OO' = OA + AO' = R + AO'
\end{equation}
From the triangle $OO'X$ we have
\[
    OO < OX + O'X,
\]
and, since $OX = R$, $O'X<R'$, it follows that
\[
    OO' < R + R'
\]
whence, by~\eqref{eq:beta}, $AO' < R$, so that $A$ lies
\emph{inside}$C'$.

\sidefig{miscI_4}

(3)~Thirdly, suppose $A$ to lie on $OO'$ produced.

Then
\begin{equation}\label{eq:gamma}\tag{$\gamma$}
    R = OA = OO' + O'A
\end{equation}
And, in the triangle $OO'X$,
\[
    OX < OO' + O' X
\]
that is
\[
    R < OO' + O'X,
\]
whence, by~\eqref{eq:gamma},
\[
    OO' + O'A < OO' + O'X,
\]
or
\[
    O'A < O'X,
\]
and $A$ lies \emph{inside}~$C$.

It is to be observed that one of the two points $A$, $B$ is in the
position of case~(1) and the other in the position of either case~(2)
or case~(3): whence we must conclude that one of the two points $A$,
$B$ is \emph{inside} and the other outside the circle~$C$.

\emph{Proof of theorem.}

The circle $C$ is divided by the points $A$, $B$ into two semicircles,
one of them, and suppose it to be described by a point moving from $A$
to~$B$.

\sidefig{miscI_5}

Take two separate points $P$, $Q$ on it and, to fix our ideas, suppose
that $P$ precedes~$Q$.

Comparing the triangles $OO'P$, $OO'Q$, we observe that one side $OO'$
is common, $OP$ is equal to~$OQ$, and the angle $POO'$ is less than
the angle $QOO'$.

Therefore $O'P < O'Q$.

Now, considering the semicircle $APQB$ as divided into two parts, so
that the points of the first part are inside the circle~$C$, and those
of the second part on the circumference of~$C$ or outside it, we have
the conditions necessary for the applicability of the Postulate of
Dedekind (which is true for arcs of circles as for straight lines);
whence \emph{there exists a point $M$ separating the two parts}.

I say that $O'M = R'$.

For, if not, suppose $O'M < R'$.

If then $\sigma$ signifies the difference between $R'$ and $O'M$,
suppose a point~$M'$, which \emph{follows}~$M$, taken on the
semicircle such that the chord $MM'$ is not greater than $\sigma$ (for
a way of doing this see below).

Then, in the triangle $O'MM'$,
\[
    O'M' < O'M + MM' < O'M + \sigma
\]
and therefore
\[
    OM' < R'.
\]

It follows that $M'$, a point on the arc $MB$, is inside the
circle~$C'$: which is absurd.

Similarly it may be proved that $O'M$ is not greater than~$R$.

Hence $O'M=R$.

[To find a point~$M'$ such that the chord $MM'$ is not greater
  than~$\sigma$, we may proceed thus.

Draw from $M$ a straight line $MP$ distinct from~$OM$, and cut off
$MP$ on it equal to $\sigma/2$.

Join $OP$, and draw another radius $OQ$ such that the angle $POQ$ is equal
to the angle $MOP$.

\sidefig{miscI_6}

The intersection, $M'$ of~$OQ$ with the circle satisfies the required
condition.

For $MM'$ meets $OP$ at right angles in~$S$.

Therefore, in the right-angled triangle $MSP$, $MS$ is not greater
than $MP$ (it is less, unless $MP$ coincides with $MS$, when it is
equal).

Therefore $MS$ is not greater than $\sigma/2$, so that $MM'$ is not
greater than~$\sigma$.]

\part{Book I. Propositions}

\begin{proposition}
\label{prop:I_1}

\begin{statement}
On a given finite straight line to construct an equilateral triangle.
\end{statement}

\begin{proof}

Let $AB$ be the given finite straight line.

\sidefig{propI_1}

Thus it is required to construct an equilateral triangle on the
straight line $AB$.

With centre $A$ and distance $AB$ let the circle $BCD$ be described;
\using{\rpost{3}}
\0 again, with centre $B$ and distance $BA$ let the circle $ACE$ be
described;\using{\rpost{3}}
\0 and from the point $C$, in which the circles cut one another, to
the points $A$, $B$ let the straight lines $CA$, $CB$ be joined.
\using{\rpost{1}}

Now, since the point $A$ is the centre of the circle $CDB$,
\C $AC$ is equal to $AB$.\using{\rdef{I_15}}

Again, since the point $B$ is the centre of the circle $CAB$,
\C $BC$ is equal to $BA$.\using{\rdef{I_15}}

But $CA$ was also proved equal to $AB$;
\0 therefore each of the straight lines $CA$, $CB$ is equal to $AB$.

And things which are equal to the same thing are also
equal to one another;\using{\rcn{1}}
\C therefore $CA$ is also equal to $CB$.\E

Therefore the three straight lines $CA$, $AB$, $BC$ are all equal to
one another.

Therefore the triangle $ABC$ is equilateral; and it has been
constructed on the given finite straight line $AB$.

\useqed{(Being) what it was required to do.}
\end{proof}

\begin{annotations}

1. \textbf{On a given finite straight line.} The Greek usage differs
from ours in that the definite article is employed in such a phrase as
this where we have the indefinite. \greek{ἐπὶ της δοθείσης εὐθείας
  πεπερασμένης}, ``on \emph{the} given finite straight line,''
i.e.\ the finite straight line which we choose to take.

3. \textbf{Let $AB$ be the given finite straight line.} To be strictly
literal we should have to translate in the reverse order ``let the
given finite straight line be the (straight line) $AB$''; but this
order is inconvenient in other cases where there is more than one
datum, e.g.\ in the sitting-out of I.~2, ``let the given point be $A$,
and the given straight Line $BC$,'' the awkwardness arising from the
omission of the verb in the second clause.  Hence I have, for
clearness' sake, adopted the other order throughout the book.

8. \textbf{let the circle $BCD$ be described.} Two things are here to
be noted, (1)~the elegant and practically universal use of the perfect
passive imperative in constructions, \greek{γεγράφθω} meaning of
course ``let it \emph{have been} described'' or ``suppose it
described,'' (2)~the impossibility of expressing shortly in a
translation the force of the words in their original order.
\greek{κύκλος γεγράφθω ὁ ΒΓΔ} means literally ``let a circle have been
described, the (circle, namely, which I denote by) $BCD$.'' Similarly
we have lower down ``let straight lines, (namely) the (straight Lines)
$CA$, $CB$, be joined,'' \greek{ἐπεζεύχθωσαν εὐθεῖαι αἰ ΓΑ, ΓΒ}. There
seems to be no practicable alternative, in English, but to translate
as I have done in the text.

13. \textbf{from the point $C$\dots.} Euclid is careful to adhere to the
phraseology of Postulate~\ref{post:1} except that he speaks of
``joining'' (\greek{ἐπεζεύχθωσαν}) instead of ``drawing
(\greek{γράφειν}). He does not allow himself to use the shortened
expression ``let the straight line $FC$ be joined'' (without mention
of the points $F$, $C$) until \prop{1}{5}.

20. \textbf{each of the straight lines $CA$, $CB$, \greek{ἐκατέρα τῶν
    ΓΑ, ΓΒ}} and 24.~\textbf{the three straight lines $CA$, $AB$,
  $BC$, \greek{αἱ τρεῖς αἱ ΓΑ, ΑΒ, ΒΓ}}.  I have, here and in all
  similar expressions, inserted the words ``straight lines'' which are
  not in the Greek. The possession of the inflected definite article
  enables the Greek to omit the words, but this it not possible in
  English, and it would scarcely be English to write ``each of $CA$,
  $CB$'' or ``the three $CA$, $AB$, $BC$.''

\end{annotations}

\begin{notes}

It is a commonplace that Euclid has no right to assume, without
premising some postulate, that the two circles \emph{will} meet in a
point~$C$. To supply what is wanted we must invoke the Principle of
Continuity (see note thereon above, p.~235\?). It is sufficient for
the purpose of this proposition and of \prop{1}{21}, where there is a
similar tacit assumption, to use the form of postulate suggested by
Killing. ``\emph{If a line} [in this case e.g.\ the circumference
  $ACE$] \emph{belongs entirely to a figure} [in this case a plane]
\emph{which is divided into two parts} [namely the part enclosed
  within the circumference of the circle $BCD$ and the part outside
  that circle], \emph{and if the line has at least one point common
  with each part, it must also meet the boundary between the parts}
     [i.e.\ the circumference $ACE$ must meet the circumference
       $BCD$].''

Zeno's remark that the problem is not solved unless it is taken for
granted that two straight lines cannot have a common segment has
already been mentioned (note on Post.~\ref{post:2}, p.~100\?). Thus,
if $AC$, $BC$ meet at $F$ before reaching~$C$, and have the part $EC$
common, the triangle obtained, namely $FAB$, will not be equilateral,
but $FA$, $FB$ will each be less than~$AB$. But Post.~\ref{post:2} has
already laid it down that two straight lines cannot have a common
segment.

Proclus devotes considerable space to this part of Zeno's criticism,
but satisfies himself with the bare mention of the other part, to the
effect that it is also necessary to assume that two
\emph{circumferences} (with different centres) cannot have a common
part. That is, for anything we know, there may be any number of points
$C$ common to the two circumferences $ACE$, $BCD$. It is not until
\prop{3}{10} that it is proved that two circles cannot intersect in
more points than two, so that we are not entitled to assume it
here. The most we can say is that it is enough for the purpose of this
proposition if \emph{one} equilateral triangle can be found with the
given base; that the construction only gives \emph{two} such triangles
has to be left over to be proved subsequently.  And indeed we have not
long to wait; for \prop{1}{7} clearly shows that on either side of the
base $AB$ only \emph{one} equilateral triangle can be described. Thus
\prop{1}{7} gives us the \emph{number of solutions} of which the
present problem is susceptible, and it supplies the same want in
\prop{1}{22} where a triangle has to be described with three sides of
given length; that is, \prop{1}{7} furnishes us, in both cases, with
one of the essential parts of a complete \greek{διορισμός}, which
includes not only the determination of the conditions of possibility
but also the number of solutions (\greek{ποσαχῶς ἐγχωρεῖ}, Proclus,
p.~202,~5). This view of \prop{1}{7} as supplying an equivalent for
\prop{3}{10} absolutely needed in \prop{1}{1} and \prop{1}{22} should
serve to correct the idea so common among writers of text-books that
\prop{1}{7} is merely of use as a lemma to Euclid's proof of
\prop{1}{8}, and therefore may be left out if an alternative proof of
that proposition is adopted.

\sidefig{propI_1a}

Agreeably to his notion that it is from \prop{1}{1} that we must
satisfy ourselves that isosceles and scalene triangles actually exist,
as well as equilateral triangles, Proclus shows how to draw, first a
particular isosceles triangle, and then a scalene triangle, by means
of the figure of the proposition. To make an isosceles triangle he
produces $AB$ in both directions to meet the respective circles in
$D$, $E$, and then describes circles with $A$, $B$ as centres and
$AE$, $BD$ as radii respectively. The result is an isosceles triangle
with each of two sides double of the third side. To make an isosceles
triangle in which the equal sides are not so related to the third side
but have any given length would require the use of \prop{1}{3}; and
there is no object in treating the question at all in advance of
\prop{1}{22}. An easier way of satisfying ourselves of the existence
of some isosceles triangles would surely be to conceive any two radii
of a circle drawn and their extremities joined.

\sidefig{propI_1b}

There is more point in Proclus' construction of a \emph{scalene}
triangle, Suppose $AC$ to be a radius of one of the two circles, and
$D$ a point on $AC$ lying in that portion of the circle with
centre~$A$ which is outside the circle with centre~$B$.  Then, joining
$BD$, as in the figure, we have a triangle which obviously has all its
sides unequal, that is, a scalene triangle.

The above two constructions appear in al-Nairīzī's commentary under
the name of Heron; Proclus does not mention his source.

In addition to the above construction for a scalene triangle
(producing a triangle in which the ``given'' side is greater than one
and less than the other of the two remaining sides), Heron has two
others showing the other two possible cases, in which the ``given''
side is (1)~less than, (2)~greater than, either of the other two
sides.

\end{notes}

\end{proposition}

\begin{proposition}
\label{prop:I_2}

\begin{statement}
To place at a given point (as an extremity) a straight line
equal to a given straight line.
\end{statement}

\begin{proof}

Let $A$ be the given point, and $BC$ the given straight line.

Thus it is required to place at the point $A$ (as an extremity) a
straight line equal to the given straight line~$BC$.

\sidefig{propI_2}

From the point~$A$ to the point~B let the straight line~$AB$ be joined;
\using{\rpost{1}}
\0 and on it let the equilateral triangle
$DAB$ be constructed. \using{\prop{1}{1}}

Let the straight lines $AE$, $BF$ be produced in a straight line
with~$DA$, $DB$; \using{\rpost{3}}
\0 with centre~$B$ and distance~$BC$ let the circle $CGH$ be described;
\using{\rpost{3}}
\0 and again, with centre~$D$ and distance $DG$ let the circle $GKL$ be
described, \using{\rpost{3}}

Then, since the point $B$ is the centre of the circle $CGH$,
\C $BC$ is equal to $BG$.\E

Again, since the point $D$ is the centre of the circle $GKL$,
\C $DL$ is equal to $DG$.\E

And in these $DA$ is equal to~$DB$;
\1 therefore the remainder $AL$ is equal to the remainder~$BG$. \using{\rcn{3}}

But $BC$ was also proved equal to~$BG$;
\1 therefore each of the straight lines $AL$, $BC$ is equal to~$BG$.\E

And things which are equal to the same thing are also
equal to one another; \using{\rcn{1}}
\C therefore $AL$ is also equal to~$BC$.\E

Therefore at the given point~$A$ the straight line $AL$ is placed
equal to the given straight line~$BC$.

\useqed{(Being) what it was required to do.}
\end{proof}

\begin{annotations}

1. \textbf{(as an extremity).}  I have inserted these words because
``to place a straight line at a given point'' (\greek{πρὸς τῷ δοθέντι
  σημείῳ}) is not quite clear enough, at least in English.

11. \textbf{Let the straight lines $AB$, $BF$ be produced….}  It will
be observed that in this first application of Postulate~\ref{post:1},
and again in \prop{1}{5}, Euclid speaks of the \emph{continuation} of
the straight line as that which is produced in such cases,
\greek{ἐκβεβλήσθωσαν} and \greek{προσεκβεβλήσθωσαν} meaning little
more than \emph{drawing} straight lines ``in a straight line with''
the given straight lines.  The first place in which Euclid uses
phraseology exactly corresponding to ours when speaking of a straight
line being produced is in \prop{1}{16}: ``let one side of it, $BC$, be
produced to~$D$'' (\greek{προσεκβεβλήσθω αὐτοῦ μλα πλευρὰ ἡ ΒΓ ἐπὶ τὸ
  Δ}).

23. \textbf{the remainder $AL$…the remainder $BG$.} The Greek
expressions are \greek{λοιπὴ ἡ ΑΛ} and \greek{λοιπῇ τῇ ΒΗ} and the
literal translation would be ``$AL$ (or $BG$) \emph{remaining},'' but
the shade of meaning conveyed by the position of the definite article
can hardly be expressed in English.

\end{annotations}

\begin{notes}

This proposition gives Proclus an opportunity, such as the Greek
commentators revelled in, of distinguishing a multitude of
\emph{cases}.  After explaining that those theorems and problems are
said to have \emph{cases} which have the same force, though admitting
of a number of different figures, and preserve the same method of
demonstration while admitting variations of position, and that cases
reveal themselves in the \emph{construction}, he proceeds to
distinguish the cases in this problem arising from the different
positions which the given point may occupy relatively to the given
straight line.  It may be (he says) either (1)~outside the line or
(2)~on the line, and, if (1), it may be either (\emph{a}) on the line
produced or (\emph{b}) situated obliquely with regard to it; if (2),
it may be either (\emph{a}) one of the extremities of the line or
(\emph{b}) an intermediate point on it.  It will be seen that Proclus'
anxiety to subdivide leads him to give a ``case,'' (2)~(\emph{a}),
which is useless, since in that ``case'' we are given what we are
required to rind, and there is really no problem to solve.  As Savile
says, ``qui quaerit ad \emph{β} punctum ponere rectam aequalem
\greek{γῇ βγ} rectae, quaerit quod datum est, quod nemo faceret nisi
forte insaniat,''

Proclus gives the construction for (2)~(\emph{b}) following Euclid's
way of taking $G$ as the point in which the circle with centre~$B$
intersects $DB$ \emph{produced}, and then proceeds to ``cases,'' of
which there are still more, which result from the different ways of
drawing the equilateral triangle and of producing its sides.

This last class of ``cases'' he subdivides into three according as
$AB$ is (1)~equal to, (2)~greater than or (3) less than~$BC$.  Here
again ``case'' (1) serves no purpose, since, if $AB$ is equal to $BC$,
the problem is already solved.  But Proclus' figures for the other two
cases are worth giving, because in one of them the point $G$ is on
$BD$ produced beyond $D$, and in the other it lies on $BD$ itself and
there is no need to produce any side of the equilateral triangle.
\infig{propI_2a}
A glance at these figures will show that, if they were used in the
proposition, each of them would require a slight modification in the
wording (1) of the construction, since $BD$ is in one case produced
beyond $D$ instead of~$B$ and in the other case not produced at all,
(2) of the proof, since $BG$, instead of being the difference between
$DG$ and DB, is in one case the sum of DG and DB and in the other the
difference between $DB$ and~$DG$.

Modern editors generally seem to classify the cases according to the
possible variations in the construction rather than according to
differences in the data.  Thus Lardner, Potts, and Todhunter
distinguish eight cases due to the three possible alternatives,
(1)~that the given point may be joined to either end of the given
straight line, (2)~that the equilateral triangle may then be described
on either side of the joining line, and (3)~that the side of the
equilateral triangle which is produced may be produced in either
direction.  (But it should have been observed that, where $AB$ is
greater than $BC$, the third alternative is between producing $DB$ and
not producing it at all.)  Potts adds that, when the given point lies
either on the line or on the line produced, the distinction which
arises from joining the two ends of the line with the given point no
longer exists, and there are only four cases of the problem (I think
he should rather have said \emph{solutions}).

To distinguish a number of cases in this way was foreign to the really
classical manner.  Thus, as we shall see, Euclid's method is to give
one case only, for choice the most difficult, leaving the reader to
supply the rest for himself.  Where there was a real distinction
between cases, sufficient to necessitate a substantial difference in
the proof, the practice was to give separate \emph{enunciations} and
proofs altogether as we may see, e.g., from the \emph{Conics} and the
\emph{De sectione rationis} of Apollonius.

Proclus alludes, in conclusion, to the error of those who proposed to
solve \prop{1}{2} by describing a circle with the given point as
centre and with a distance equal to~$BC$, which, as he says, is a
\emph{petitio principii}.  De Morgan puts the matter very clearly
(\emph{Supplementary Remarks on the first six Books of Euclid's
  Elements} in the \emph{Companion to the Almanac}, 1849, p.~6).  We
should ``insist,'' he says, ``here upon the restrictions imposed by
the first three postulates, which do not allow a circle to be drawn
with a compass-carried distance; suppose the compasses to dose of
themselves the moment they cease to touch the paper.  These two
propositions [\prop{1}{2}, \prop*{1}{3}] extend the power of
construction to what it would have been if all the usual power of the
compasses had been assumed; they are mysterious to all who do not see
that postulate~iii does not ask for \emph{every use of the
  compasses}.''

\end{notes}

\end{proposition}

\begin{proposition}
\label{prop:I_3}

\begin{statement}
Given two unequal straight lines, to cut off from the greater a
straight line equal to the less.
\end{statement}

\begin{proof}

\sidefig{propI_3}

Let $AB$, $C$ be the two given unequal straight lines, and let $AB$ be
the greater of them.

Thus it is required to cut off from $AB$ the greater a straight line
equal to $C$ the less.

At the point $A$ let $AD$ be placed equal to the straight line~$C$;
\using{\prop{1}{2}}
and with centre~$A$ and distance $AD$ let the circle $DEF$ be
described. \using{\rpost{3}}

Now, since the point $A$ is the centre of the circle $DEF$,
\C $AE$ is equal to $AD$. \using{\rdef{15}}

But $C$ is also equal to $AD$.

Therefore each of the straight lines $AE$, $C$ is equal to $AD$; so
that $AE$ is also equal to~$C$. \using{\rcn{1}}

Therefore, given the two straight lines $AB$, $C$, from $AB$ the
greater $AE$ has been cut off equal to $C$ the less.

\useqed{(Being) what it was required to do.}
\end{proof}

\begin{notes}

Proclus contrives to make a number of ``cases'' out of this
proposition also, and gives as many as eight figures.  But he only
produces this variety by practically incorporating the construction of
the preceding proposition, instead of assuming it as we are entitled
to do.  If Prop.~\prop*{1}{2} is assumed, there is really only one
``case'' of the present proposition, for Potts' distinction between
two cases according to the particular extremity of the straight line
from which the given length has to be cut off scarcely seems to be
worth making.

\end{notes}

\end{proposition}

\begin{proposition}
\label{prop:I_4}

\begin{statement}
If two triangles have the two sides equal to two sides respectively,
and have the angles contained by the equal straight lines equal, they
will also have the base equal to the base, the triangle will be equal
to the triangle, and the remaining angles s will be equal to the
remaining angles respectively, namely those which the equal sides
subtend.
\end{statement}

\begin{proof}

Let $ABC$, $DEF$ be two triangles having the two sides $AB$, $AC$
equal to the two sides $DE$, $DF$ respectively, namely $AB$ to $DE$
and $AC$ to $DF$, and the angle $BAC$ equal to the angle~$EDF$.

I say that the base $BC$ is also equal to the base $EF$, the triangle
$ABC$ will be equal to the triangle $DEF$, and the remaining angles
will be equal to the remaining angles respectively, namely those which
the equal sides subtend, that is is, the angle $ABC$ to the angle
$DEF$, and the angle $ACB$ to the angle $DFE$.

For, if the triangle $ABC$ be applied to the triangle $DEF$, and if
the point $A$ be placed on the point~$D$
and the straight line $AB$ on~$DE$, then the point~$B$ will also
coincide with~$E$, because $AB$ is equal to~$DE$.

Again, $AB$ coinciding with~$DE$,
the straight line $AC$ will also coincide with $DF$, because the angle
$BAC$ is equal to the angle $EDF$;

hence the point~$C$ will also coincide with the point$F$,
because $AC$ is again equal to~$DF$.

But $B$ also coincided with~$E$;
hence the base $BC$ will coincide with the base~$EF$.

[For if, when $B$ coincides with~$E$ and $C$ with~$F$, the base $BC$
  does not coincide with the base $EF$, two straight lines will
  enclose a space: which is impossible.

Therefore the base $BC$ will coincide with $EF$] and will be equal to
it. \using{\rcn{4}}

Thus the whole triangle $ABC$ will coincide with the whole triangle
$DEF$,

and will be equal to it.

And the remaining angles will also coincide with the remaining angles
and will be equal to them,

the angle $ABC$ to the angle~$DEF$,

and the angle $ACB$ to the angle~$DFE$,

Therefore etc.

\useqed{(Being) what it was required to prove.}
\end{proof}

\begin{annotations}

1—3. It is a fact that Euclid's enunciations not infrequently leave
something to be desired in point of clearness and precision.  Here he
speaks of the triangles having ``the angle equal to the angle, namely
the angle contained by the equal straight lines'' (\greek{τὴν γωνίαν
  τῇ γωνίᾳ ἴσην ἔχῃ τὴν ὑπὸ τῶν ἴσων εὐθειῶν περιεχομένην}), only one
of the two angles being described in the latter expression (in the
accusative), and a similar expression in the dative being left to be
understood of the other angle.  It is curious too that, after
mentioning two ``\emph{sides}'' he speaks of the angles contained by
the equal ``\emph{straight lines}'' not ``\emph{sides}.''  It may be
that he wished to adhere scrupulously, at the outset, to the
phraseology of the definitions, where the angle is the inclination to
one another of two \emph{lines} or \emph{straight lints}.  Similarly
in the enunciation of \prop{1}{5} he speaks of producing the equal
``straight lines'' as if to keep strictly to the wording of
Postulate~\ref{post:2}.

2.  \textbf{respectively.}  I agree with Mr H.~M. Taylor
(\emph{Euclid}, p.~ix) that it is best to abandon the traditional
translation of ``each to each,'' which would naturally seem to imply
that all the four magnitudes are equal rather than (as the Greek
\greek{ἑκατέρα ἑκατέρᾳ} does) that one is equal to one and the other
to the other.

3. \textbf{the base.} Here we have the word \emph{base} used for the
first time in the \emph{Elements}.  Proclus explains it (p.~136,
13—15) as meaning (1), when no side of a triangle has been mentioned
before, the side ``which is on a level with the sight'' (\greek{τὴν τῇ
  ὄφει κειμένην}), and (2), when two sides have already been
mentioned, the third side.  Proclus thus avoids the mistake made by
some modern editors who explain the term exclusively with reference to
the case where two sides have been mentioned before.  That this is an
error is proved (1)~by the occurrence of the term in the enunciations
of \prop{1}{37} etc.\ about triangles on the same base and equal
bases, (2)~by the application of the same term to the bases of
parallelograms in \prop{1}{35} etc.  The truth is that the use of the
term must have been suggested by the practice of drawing the
particular side horizontally, as it were, and the rest of the figure
above it.  The \emph{base} of a figure was therefore spoken of,
primarily, in the same sense as the base of anything else. e.g.\ of a
pedestal or column; but when, as in \prop{1}{5}, two triangles were
compared occupying other than the normal!  positions which gave rise
to the name, and when two sides had been previously mentioned, the
base was as Proclus says, necessarily the third side.

6. \textbf{subtend.} \greek{ὑποτείνειν ὑπό}, ``to stretch under,''
with accusative.

9. \textbf{the angle $BAC$.}  The full Greek expression would be
\greek{ἡ ὑπὸ τῶν ΒΑ, ΑΛ περιεχομένη γωνία}, ``the angle contained by
the (straight lines) $BA$, $AC$.''  But it was a common practice of
Greek geometers, e.g. of Archimedes and Apollonius (and Euclid too in
Books \r10.— \r13.), to use the abbreviation at \greek{αἱ ΒΑ\?} for
\greek{αἱ ΒΑ, Α\?}, ``the (straight lines) $BA$, $AC$.''  Thus, on
\greek{περιεχομένη} being dropped, the expression would become first
\greek{ἡ ὑπὸ τῶν ΒΑΓ γωνία}, then \greek{ἡ ὑπὸ ΒΑΓ γωνία}, and finally
\greek{ἡ ὑπὸ ΒΑΓ}, without \greek{γωνία}, as we regularly find it in
Euclid.

17. \textbf{if the triangle be applied to…}, 23. \textbf{coincide.}
The difference between the technical use of the passive
\greek{ἐφαρμόζεσθαι} ``to be \emph{applied} (to),'' and of the active
\greek{ἐφαρμόζειν} ``to \emph{coincide} (with) has been noticed above
(note on \emph{Common Notion}~\ref{cn:4}, pp.~224—5).

32. \emph{[For if, when B coincides…36. coincide with EF].}  Heiberg
(\emph{Paralipomena zu Euklid} in \emph{Hermes}, \r38., 1903, p.~56]
  has pointed out, as a conclusive reason for regarding these words as
  an early interpolation, that the text of an-Nairīzī (\emph{Codex
    Leidensis}~399, 1, ed.\ Besthorn-Heiberg, p.~55) does not give the
  words in this place but after the conclusion \textsc{q.e.d.}, which
  shows that they constitute a \emph{scholium} only.  They were
  doubtless added by some commentator who thought it necessary to
  explain the immediate inference that, since $B$ coincides with~$E$
  and $C$ with~$F$, the straight line $BC$ coincides with the straight
  line $EF$, an inference which really follows from the definition of
  a straight line and Post.~\ref{post:1}; and no doubt the Postulate
  that ``Two straight lines cannot enclose a space'' (afterwards
  placed among the \emph{Common Notions}) was interpolated at the same
  time.

44. \emph{Therefore etc.}  Where (as here) Euclid's conclusion merely
repeats the enunciation word for word, I shall avoid the repetition
and write ``Therefore etc.''\ simply.

\end{annotations}

\begin{notes}

In the note on Common Notion~\ref{cn:4} I have already mentioned that
Euclid obviously used the method of superposition with reluctance, and
I have given, after Veronese for the most part, the reason for holding
that that method is not admissible as a theoretical means of proving
equality, although it may be of use as a practical test, and may thus
furnish an empirical basis on which to found a postulate.  Mr~Bertrand
Russell observes (\emph{Principles of Mathematics} \r1.\ p.~405) that
Euclid would have done better to assume \prop{1}{4} as an axiom, as is
practically done by Hilbert (\emph{Grundlagen der Geometrie}, p.~9).
It may be that Euclid himself was as well aware of the objections to
the method as are his modern critics; but at all events those
objections were stated, with almost equal clearness, as early as the
middle of the 16th century.  Peletarius (Jacques Peletier) has a long
note on this proposition (\emph{In Eucidis Elementa geometrica
  demonstrationum libri sex}, 1557), in which he observes that, if
superposition of lines and figures could be assumed as a method of
proof, the whole of geometry would be full of such proofs, that it
could equally well have been used in \prop{1}{2}, \prop*{1}{3} (thus
in \prop{1}{2} we could simply have supposed the line taken up and
placed at the point), and that in short it is obvious how far removed
the method is from the dignity of geometry.  The theorem, he adds, is
obvious in itself and does not require proof; although it is
introduced as a theorem, it would seem that Euclid intended it rather
as a \emph{definition} than a theorem, ``for I cannot think that two
angles are equal unless I have a conception of what equality of angles
is.''  Why then did Euclid include the proposition among theorems,
instead of placing it among the axioms?  Peletarius makes the best
excuse he can, but concludes thus: ``Huius itaque propositionis
veritatem non aliunde quam a communi iudicio petemus: cogitabimusque
figuras figuris superponere, Mechanicum quippiam esse: intelligere
verò, id demum esse Mathematicum.''

Expressed in terms of the modern systems of Congruence-Axioms referred
to in the note on Common Notion~\ref{cn:4}, what Euclid really assumes
amounts to the following:

(1)~On the line $DE$, there is a point~$E$, on either side of~$D$,
such that $AB$ is equal to~$DE$.

(2)~On either side of the ray $DE$ there is a ray $DF$ such that the
angle $EDF$ is equal to the angle~$BAC$.

It now follows that on $DF$ there is a point such that $DF$ is equal
to~$AC$.

And lastly (3), we require an axiom from which to infer that the two
remaining angles of the triangles are respectively equal and that the
bases are equal.

I have shown above (pp.~\pageref{229—230}) that Hilbert has an axiom
stating the equality of the remaining angles simply, but proves the
equality of the bases.

Another alternative is that of Pasch (\emph{Vorlesungen über neuere
  Geometrie}, p.~109) who has the following ``Grundsatz'':

If two figures $AB$ and $FGH$ are given ($FGH$ not being contained in
a straight length), and $AB$, $FG$ are congruent, and if a plane
surface be laid through $A$ and~$B$, we can specify in this plane
surface, produced if necessary, two points $C$, $D$, neither more nor
less, such that the figures $ABC$ and $ABD$ are congruent with the
figure $FGH$, and the straight line $CD$ has with the straight line
$AB$ or with $AB$ produced one point common.

I pass to two points of detail in Euclid's proof:

(1)~The inference that, since $B$ coincides with $E$, and $C$
with~$F$, the bases of the triangles are wholly coincident rests, as
expressly stated, on the impossibility of two straight lines enclosing
a space, and therefore presents no difficulty.

But (2)~most editors seem to have failed to observe that at the very
beginning of the proof a much more serious assumption is made without
any explanation whatever, namely that, if $A$ be placed on~$D$, and
$AB$ on~$DE$, the point~$B$ will coincide with~$E$, because $AB$ is
equal to~$DE$.  That is, the \emph{converse} of \emph{Common
  Notion}~\ref{cn:4} is assumed for straight lines.  Proclus merely
observes, with regard to the converse of this Common Notion, that it
is only true in the case of things ``of the same form''
(\greek{ὁμοειδῆ}), which he explains as meaning straight lines, arcs
of one and the same circle, and angles ``contained by lines similar
and similarly situated'' (p.~241, 3—8).

Savile however saw the difficulty and grappled with it in his note on
the Common Notion.  After stating that all straight lines with two
points common are congruent between them (for otherwise two straight
lines would enclose a space), he argues thus.  Let there be two
straight lines $AB$, $DE$, and let $A$ be placed on~$D$, and $AB$
on~$DE$.  Then $B$ will coincide with~$E$.  For, if not, let $B$ fall
somewhere short of~$E$ or beyond~$E$; and in either case it will
follow that the less is equal to the greater, which is impossible.

Savile seems to assume (and so apparently does Lardner who gives the
same proof) that, if the straight lines be ``applied,'' $B$ will fall
somewhere on~$DE$ or $DE$ produced.  But the ground for this
assumption should surely be stated; and it seems to me that it is
necessary to use, not Postulate~\ref{post:1} alone, nor
Postulate~\ref{post:2} alone, but both, for this purpose (in other
words to assume, not only that \emph{two straight lines cannot enclose
  a space}, but also that \emph{two straight lines cannot have a
  common segment}).  For the only safe course is to place $A$ upon~$D$
and then turn $AB$ about~$D$ until \emph{some} point on~$AB$
intermediate between $A$ and~$B$ coincides with \emph{some} point
on~$DE$.  In this position $AB$ and~$DE$ have two points common.  Then
Postulate~\ref{post:1} enables us to infer that the straight lines
coincide \emph{between} the two common points, and
Postulate~\ref{post:2} that they coincide beyond the second common
point towards $B$ and~$E$.  Thus the straight lines coincide
throughout so far as \emph{both} extend; and Savile's argument then
proves that $B$ coincides with~$E$.

\end{notes}

\end{proposition}

\begin{proposition}
\label{prop:I_5}

\begin{statement}
In isosceles triangles the angles at the base are equal to one
another, and, if the equal straight lines be produced further, the
angles under the base will be equal to one another.
\end{statement}

\begin{proof}

Let $ABC$ be an isosceles triangle having the side $AB$
equal to the side $AC$;\0
and let the straight lines $BD$, $CE$ be produced further in a
straight line with $AB$, $AC$. \using{\rpost{2}}

I say that the angle $ABC$ is equal to the angle $ACB$, and the angle
$CBD$ to the angle $BCE$.

Let a point~$F$ be taken at random on~$BD$;\0
from $AE$ the greater let $AG$ be cut off equal to $AF$ the less;
\using{\prop{1}{3}}
and let the straight lines $FC$, $GB$ be joined.

Then, since $AF$ is equal to $AG$ and $AB$ to~$AC$,\0
the two sides $FA$, $AC$ are equal to the two sides $GA$, $AB$,
respectively;\0
and they contain a common angle, the angle $FAG$.

Therefore the base $FC$ is equal to the base $GB$,\0
and the triangle $AFC$ is equal to the triangle~$AGB$,\0
and the remaining angles will be equal to the remaining angles
respectively, namely those which the equal sides subtend,
\C that is, the angle $ACF$ to the angle $ABG$,\E
\C and the angle $AFC$ to the angle $AGB$. \using{\prop{1}{4}}

And, since the whole $AF$ is equal to the whole~$AG$,
\C and in these $AB$ is equal to $AC$,\E
the remainder $BF$ is equal to the remainder~$CG$.

But $FC$ was also proved equal to~$GB$;

therefore the two sides $BF$, $FC$ are equal to the two sides $CG$,
$GB$ respectively;

and the angle $BFC$ is equal to the angle $CGB$,

while the base $BC$ is common to them;

therefore the triangle $BFC$ is also equal to the triangle $CGB$, and
the remaining angles will be equal to the remaining angles
respectively, namely those which the equal sides subtend;

therefore the angle $FBC$ is equal to the angle $GCB$,

and the angle $BCF$ to the angle $CBG$.

Accordingly, since the whole angle $ABG$ was proved equal to the angle
$ACF$,

and in these the angle $CBG$ is equal to the angle~$BCF$, the
remaining angle $ABC$ is equal to the remaining angle~$ACB$;

and they are at the base of the triangle $ABC$.

But the angle $FBC$ was also proved equal to the angle $GCB$; and they
are under the base.

Therefore etc. \qedhere
\end{proof}

\begin{annotations}

1. \textbf{the equal straight lines} (meaning the equal \emph{sides}).
Cf. note on the similar expression in Prop.~\prop*{1}{4}, lines 2,~3.

10. \textbf{Let a point $F$ be taken at random on $BD$},
\greek{εἰλήφθω ἐπὶ τῆς ΒΔ τυχὸν σημεῖον τὸ Ζ}, where \greek{τυχὸν
  σημεῖον} means ``a chance point.''

17, \textbf{the two sides $FA$, $AC$ are equal to the two aides $GA$,
  $AB$ respectively}, \greek{δύο αἰ Α, ΑΓ δυσὶ παῖς ΗΑ, ΑΒ ἴσαι εἰσὶν
  ἑκατέρα ἑκατέρᾳ}.  Here, and in numberless later passages, I have
inserted the word ``sides'' for the reason given in the note on
\prop{1}{1}, line~10.  It would have been permissible to supply either
``straight lines'' or ``sides''; but on the whole ``sides'' seems to
be more in accordance with the phraseology of \prop{1}{4},

33. \textbf{the base $BC$ is common to them}, i.e., apparently, common
to the \emph{angles}, as the \greek{αὐτῶν} in \greek{βάσις αὐτῶν
  κοινὴ} can only refer to \greek{γωνία} and \greek{γωνίᾳ} preceding.
Simson wrote ``and the base $BC$ is common to the two triangles $BFC$,
$CGB$; Todhunter left out these words as being of no use and tending
to perplex a beginner.  But Euclid evidently chose to quote the
conclusion of \prop{1}{4} exactly; the first phrase of that conclusion
is that the bases (of the two triangles) are equal, and, as the equal
bases are here the \emph{same} base, Euclid naturally substitutes the
word ``common'' for ``equal.''

48. As ``(Being) what it was required to prove'' (or ``do'') is
somewhat long, I shall henceforth write the time-honoured
``\qedsymbol''\ and ``\qefsymbol''\ for \greek{ὅπερ ἔδει δεῖξαι} and \greek{ὅπερ
  ἔδει ποιῆσαι}.

\end{annotations}

\begin{notes}

According to Proclus (p.~250, 20) the discoverer of the fact that in any
isosceles triangle the angles at the base are equal was Thales, who however
is said to have spoken of the angles as being similar, and not as being equal.
(cf. ArisL De caelo iv. 4, 311 b 34 n-pos ifiolas ymvlat dxuVcrtu dxpoVcvov where
equal angles are meant.)

\subsection*{A pre-Euclidean proof of \prop{1}{5}}

One of the most interesting of the passages in Aristotle indicating
differences between Euclid's proofs and those with which Aristotle was
familiar, in other words, those of the text-books immediately
preceding Euclid's, has reference to the theorem of \prop{1}{5}.  The
passage (\emph{Anal.\ Prior.}\ \r1.~34, 41~b~13—21) is so important
that I must quote it in full.  Aristotle is illustrating the fact that
in any syllogism one of the propositions must be affirmative and
universal (\greek{καθόλου}). ``This,'' he says, ``is better shown in
the case of geometrical propositions ``(\greek{ἐν τοῖς διαγράμμασιν}),
e.g.\ the proposition that \emph{the angles at the bast of an
  isosceles triangle are equal}.

``For let $A$, $B$ be drawn [i.e.\ joined] to the centre.

``If, then, we assumed (1)~that the angle $AC$ [i.e.\ $A + C$] is
equal to the angle $BD$ [i.e.\ $B + D$ without asserting generally
  that \emph{the angles of semicircles are equal}, and again (2)~that
  the angle~$C$ is equal to the angle~$D$ without making the further
  assumption that \emph{the two angles of all segments art equal}, and
  if we then inferred, lastly, that, since the whole angles are equal,
  and equal angles are subtracted from them, the angles which remain,
  namely $E$, $F$, are equal, we should commit a \emph{petitio
    principii}, unless we assumed [generally] that, \emph{when equals
    are subtracted from equals, tht remainders are equal}.''

The language is noteworthy in some respects.

(1)~$A$, $B$ are said to be drawn (\greek{ἠγμέναι}) to the centre (of
the circle of which the two equal sides are radii) as if $A$, $B$ were
not the angular points but the sides or the radii themselves.  (There
is a parallel for this in Eucl.\ \prop{4}{4}.)

(2)~``The angle $AC''$ is the angle which is the sum of $A$ and~$C$,
and $A$ means here the angle at~$A$ of the isosceles triangle shown in
the figure, and afterwards spoken of by Aristotle as~$E$, while $C$ is
the ``mixed'' angle between $AB$ and the circumference of the smaller
segment cut off by it.

(3)~The ``angle of a semicircle'' (i.e. the ``angle'' between the
diameter and the circumference, at the extremity of the diameter) and
the ``angle \emph{of} a segment'' appear in Euclid \prop{3}{16} and
\book{3}.~Def.~\ref{def:III_7} respectively, obviously as survivals
from earlier text-books.

But the most significant facts to be gathered from the extract are
that in the text-books which preceded Euclid's ``mixed'' angles played
a much more important part than they do with Euclid, and, in
particular, that at least two propositions concerning such angles
appeared quite at the beginning, namely the propositions that
\emph{the (mixed) angles of semicircles are equal and that the two
  (mixed) angles of any segment of a circle are equal}.  The wording
of the first of the two propositions is vague, but it does not
necessarily mean more than that the two (mixed) angles in one
semicircle are equal, and I know of no evidence going to show that it
asserts that the angle of any one semicircle is equal to the angle of
any other semicircle (of different size).  It is quoted in the same
form, ``because the angles of semicircles are equal,'' in the Latin
translation from the Arabic of Heron's \emph{Catoptrica}, Prop.~9
(Heron, Vol.~\r2., Teubner, p.~334), but it is only inferred that the
different radii of one circle make equal ``angles'' with the
circumference; and in the similar proposition of the Pseudo-Euclidean
\emph{Catoptrica} (Euclid, Vol.~\r7., p.~394) angles of the same sort
in one circle are said to be equal ``because they are (angles) of a
semicircle.''  Therefore the first of the two propositions may be only
a particular case of the second.

But it is remarkable enough that the second proposition (that
\emph{the two ``angles of'' any segment of a circle are equal})
should, in earlier text-books, have been placed before the theorem of
Eucl.\ \prop{1}{5}.  We can hardly suppose it to have been proved
otherwise than by the superposition of the semicircles into which the
circle is divided by the diameter which bisects at right angles the
base of the segment; and no doubt the proof would be closely connected
with that of Thales' other proposition that any diameter of a circle
bisects it, which must also (as Proclus indicates) have been proved by
superposing one of the two parts upon the other.

It is a natural inference from the passage of Aristotle that Euclid's
proof of \prop{1}{5} was his own, and it would thus appear that his
innovations as regards order of propositions and methods of proof
began at the very threshold of the subject.

\subsection*{Proof without producing the sides}

In this proof, given by Proclus (pp.~248, 21—249, 19) $D$ and~$E$ are
taken on $AB$, $AC$, instead of on $AB$, $AC$ \emph{produced}, so that
$AD$, $AE$ are equal.  The method of proof is of course exactly like
Euclid's, but it does not establish the equality of the angles beyond
the base as well.

\subsection*{Pappus' proof}

Proclus (pp.~249, 20—250, 12) says that Pappus proved the theorem in a
still shorter manner without the help of any construction whatever.

This very interesting proof is given as follows:

``Let $ABC$ be an isosceles triangle, and $AB$ equal to~$AC$.

Let us conceive this one triangle as two triangles, and let us argue
in this way.

Since $AB$ is equal to $AC$, and $AC$ to~$AB$, the two sides $AB$,
$AC$ are equal to the two sides $AC$,~$AB$.

And the angle $BAC$ is equal to the angle $CAB$, for it is the same.

Therefore all the corresponding parts (in the triangles) are equal,
namely $BC$ to~$BC$, the triangle $ABC$ to the triangle $ABC$
(i.e.\ $ACB$), the angle $ABC$ to the angle $ACB$, and the angle $ACB$
to the angle~$ABC$, (for these are the angles subtended by the equal
sides $AB$,~$AC$.

Therefore in isosceles triangles the angles at the base are equal.''

This will no doubt be recognised as the foundation of the alternative
proof frequently given by modern editors, though they do not refer to
Pappus.  But they state the proof in a different form, the common
method being to suppose the triangle to be taken up, turned over, and
placed again upon itself, after which the same considerations of
congruence as those used by Euclid in \prop{1}{4} are used over again.
There is the obvious difficulty that it supposes the triangle to be
taken up and at the same time to remain where it is, (Cf.\ Dodgson's
humorous remark upon this, \emph{Euclid and his modern Rivals},
p.~47.)  Whatever we may say in justification of the proceeding
(e.g.\ that the triangle may be supposed to leave a \emph{trace}), it
is really equivalent to assuming the construction (hypothetical, if
you will) of another triangle equal in all respects to the given
triangle; and such an assumption is not in accordance with Euclid's
principles and practice.

It seems to me that the form given to the proof by Pappus himself is
by far the best, for the reasons (1)~that it assumes no construction
of a second triangle, real or hypothetical, (2)~that it avoids the
distinct awkwardness involved by a proof which, instead of merely
quoting and applying the \emph{result} of a previous proposition,
repeats, with reference to a new set of data, the \emph{process} by
which that result was established.  If it is asked how we are to
realise Pappus' idea of \emph{two} triangles, surely we may answer
that we keep to one triangle and merely view it in two aspects.  If it
were a question of helping a beginner to understand this, we might say
that one triangle is the triangle looked at in front and that the
other triangle is the same triangle looked at from \emph{behind}; but
even this is not really necessary.

Pappus' proof, of course, does not include the proof of the second
part of the proposition about the angles under the base, and we should
still have to establish this much in the same way as Euclid does.

\subsection*{Purpose of the second part of the theorem}

An interesting question arises as to the reason for Euclid's insertion
of the second part, to which, it will be observed, the converse
proposition \prop{1}{6} has nothing corresponding.  As a matter of
fact, it is not necessary for any subsequent demonstration that is to
be found in the original text of Euclid, but only for the interpolated
second case of \prop{1}{7}; and it was perhaps not unnatural that the
undoubted genuineness of the second part of \prop{1}{5} convinced many
editors that the second case of \prop{1}{7} must necessarily be
Euclid's also.  Proclus' explanation, which must apparently be the
right one, is that the second part of \prop{1}{5} was inserted for the
purpose of fore-arming the learner against a possible \emph{objection}
(\greek{ἔνστασις}), as it was technically called, which might be
raised to \prop{1}{7} as given in the text, with one case only.  The
\emph{objection} would, as we have seen, take the specific ground
that, as demonstrated, the theorem was not conclusive, since it did
not cover all possible cases.  From this point of view, the second
part of \prop{1}{5} is useful not only for \prop{1}{7} but, according
to Proclus, for \prop{1}{9} also.  Simson does not seem to have
grasped Proclus' meaning, for he says: ``And Proclus acknowledges,
that the second part of Prop.~\prop*{1}{5} was added upon account of
Prop.~\prop*{1}{7} but gives a ridiculous reason for it, 'that it
might afford an answer to objections made against the 7th,' as if the
case of the 7th which is left out were, as he expressly makes it, an
objection against the proposition itself.''

\end{notes}

\end{proposition}

\begin{proposition}
\label{prop:I_6}

\begin{statement}
If in a triangle two angles be equal to one another, the sides which
subtend the equal angles will also be equal to one another.
\end{statement}

\begin{proof}

Let $ABC$ be a triangle having the angle $ABC$ equal to
the angle~$ACB$;

I say that the side $AB$ is also equal to the
side~$AC$.

For, if $AB$ is unequal to~$AC$, one of them is
greater.

Let $AB$ be greater; and from $AB$ the
greater let $DB$ be cut off equal to $AC$ the less;

let $DC$ be joined.

Then, since $DB$ is equal to~$AC$, and $BC$ is common,

the two sides $DB$, $BC$ are equal to the two sides $AC$,
$CB$ respectively;

and the angle $DBC$ is equal to the angle~$ACB$;

therefore the base $DC$ is equal to the base~$AB$, and the triangle
$DBC$ will be equal to the triangle $ACB$, the less to the greater:
which is absurd.

Therefore $AB$ is not unequal to~$AC$;
it is therefore equal to it.

Therefore etc.
\end{proof}

\begin{notes}

Euclid assumes that, because $D$ is between $A$ and~$B$, the triangle
$DBC$ is less than the triangle~$ABC$.  Some postulate is necessary to
justify this tacit assumption; considering an angle less than two
right angles, say the angle $ACB$ in the figure of the proposition, as
a cluster of rays issuing from~$C$ and bounded by the rays $CA$, $CB$,
and joining $AB$ (where $A$, $B$ are any two points on $CA$, $CB$
respectively), we see that to each successive ray taken in the
direction from $CA$ to~$CB$ there corresponds one point on~$AB$ in
which the said ray intersects~$AB$, and that all the points on~$AB$
taken in order from $A$ to~$B$ correspond univocally to all the rays
taken in order from $CA$ to~$CB$, each point namely to the ray
intersecting $AB$ in the point.

We have here used, for the first time in the \emph{Elements}, the
method of \emph{reductio ad absurdum}, as to which I would refer to
the section above (pp.~\pageref{136}, \pageref{140}) dealing with this
among other technical terms.

This proposition also, being the \emph{converse} of the preceding
proposition, brings us to the subject of

\subsection*{Geometrical Conversion}

This must of course be distinguished from the \emph{logical}
conversion of a proposition.  Thus, from the proposition that all
isosceles triangles have the angles opposite to the equal sides equal,
\emph{logical} conversion would only enable us to conclude that
\emph{some} triangles with two angles equal are isosceles.  Thus
\prop{1}{6} is the geometrical, but not the logical, converse of
\prop{1}{5}.  On the other hand, as De Morgan points out
(\emph{Companion to the Almanac}, 1849, p.~7), \prop{1}{6} is a purely
logical deduction from \prop{1}{5} and \prop{1}{18} taken together, as
is \prop{1}{19} also.  For the general argument see the note on
\prop{1}{19}.  For the present proposition it is enough to state the
matter thus.  Let $X$ denote the class of triangles which have the two
sides other than the base equal, $Y$ the class of triangles which have
the base angles equal; then we may call non-$X$ the class of triangles
having the sides other than the base unequal non-$Y$ the class of
triangles having the base angles unequal.

Thus we have
\begin{align*}
&\text{All $X$ is $Y$,}\tag*{\using{\prop{1}{5}}}\\
&\text{All non-$X$ is non-$Y$;}\tag*{\using{\prop{1}{18}}}\\
\intertext{and it is a purely logical deduction that}
&\text{All $Y$ is $X$.}\tag*{\using{\prop{1}{6}}}
\end{align*}

According to Proclus (p.~252, 5 sqq.)\ two forms of \emph{geometrical
  conversion} were distinguished.

(1)~The leading form (\greek{προηγουμένη}). the conversion \emph{par
  excellence} (\greek{ἡ κυρίως ἀντιστροφή}), is the complete or simple
conversion in which the hypothesis and the conclusion of a theorem
change places exactly, the conclusion of the theorem being the
hypothesis of the converse theorem, which again establishes, as its
conclusion, the hypothesis of the original theorem.  The relation
between the first part of \prop{1}{5} and \prop{1}{6} is of this
character.  In the former the hypothesis is that two sides of a
triangle are equal and the conclusion is that the angles at the base
are equal, while the converse (\prop{1}{6}) starts from the hypothesis
that two angles are equal and proves that the sides subtending them
are equal.

(2)~The other form of conversion, which we may call \emph{partial}, is
seen in cases where a theorem starts from two or more hypotheses
combined into one enunciation and leads to a certain conclusion, after
which the converse theorem takes this conclusion in substitution for
one of the hypotheses of the original theorem and from the said
conclusion along with the rest of the original hypotheses obtains, as
its conclusion, the omitted hypothesis of the original theorem.
\prop{1}{8} is in this sense a converse proposition to \prop{1}{4};
for \prop{1}{4} takes as hypotheses (1)~ that two sides in two
triangles are respectively equal, (2)~that the included angles are
equal, and proves (3)~that the bases are equal, while \prop{1}{8}
takes (1) and~(3) as hypotheses and proves~(2) as its conclusion.  It
is clear that a conversion of the \emph{leading} type must be unique,
while there may be many \emph{partial} conversions of a theorem
according to the number of hypotheses from which it starts.

Further, of convertible theorems, those which took as their hypothesis
the \emph{genus} and proved a \emph{property} were distinguished as
the leading theorems (\greek{προηγούμενα}), while those which started
from the property as hypothesis and described, as the conclusion, the
genus possessing that property were the converse theorems.
\prop{1}{5} is thus the leading theorem and \prop{1}{6} its converse,
since the genus is in this case taken to be the isosceles triangle.

\subsection*{Converse of second part of \prop{1}{5}}

Why, asks Proclus, did not Euclid convert the \emph{second} part of
\prop{1}{5} as well?  He suggests, properly enough, two reasons:
(1)~that the second part of \prop{1}{5} itself is not wanted for any
proof occurring in the original text, but is only put in to enable
\emph{objections} to the existing form of later propositions to be
met, whereas the converse is not even wanted for this purpose;
(2)~that the converse could be deduced from \prop{1}{6}, if wanted, at
any time after we have passed \prop{1}{13}, which can be used to prove
that, if the angles formed by producing two sides of a triangle beyond
the base are equal, the base angles themselves are equal.

Proclus adds a proof of the converse of the second part of
\prop{1}{5}, i.e.\ of the proposition that, if the angles formed by
producing two sides of a triangle beyond the base are equal, the
triangle is isosceles; but it runs to some length and then only
effects a reduction to the theorem of \prop{1}{6} as we have it.  As
the result of this should hardly be assumed, a better proof would be
an independent one adapting Euclid's own method in \prop{1}{6}.  Thus,
with the construction of \prop{1}{5}, we first prove by means of
\prop{1}{4} that the triangles $BFC$, $CGB$ are equal in all respects,
and therefore that $FC$ is equal to~$GB$, and the angle $BFC$ equal to
the angle~$CGB$.  Then we have to prove that $AF$, $AG$ are equal.  If
they are not, let $AF$ be the greater, and from $FA$ cut off $FH$
equal to~$GA$.  Join~$CH$.

Then we have, in the two triangles $HFC$, $AGB$,
two sides $HF$, $FC$ equal to two sides $AG$, $GB$
and the angle $HFC$ equal to the angle~$AGB$.

Therefore (\prop{1}{4}) the triangles $HFC$, $AGB$ are equal.  But the
triangles $BFC$, $CGB$ are also equal.

Therefore (if we take away these equals respectively) the triangles
$HBC$, $ACB$ are equal: which is impossible.

Therefore $AF$, $AG$ are not unequal.

Hence $AF$ is equal to~$AG$ and, if we subtract the equals $BF$, $CG$
respectively, $AB$ is equal to~$AC$.

This proof is found in the commentary of an-Nairīzī
(ed.\ Besthornl-Heiberg, p.~61; ed.\ Curtze, p.~50).

\subsection*{Alternative proofs of \prop{1}{6}}

Todhunter points out that \prop{1}{6}, not being wanted till
\prop{2}{4}, could be postponed till later and proved by means of
\prop{1}{26}.  Bisect the angle $BAC$ by a straight line meeting the
base at~$D$.  Then the triangles $ABD$, $ACD$ are equal in all
respects.

Another method depending on \prop{1}{26} is given by an-Nairīzī after
that proposition.

Measure equal lengths $BD$, $CE$ along the sides $BA$, $CA$.  Join
$BE$, $CD$.

Then [\prop{1}{4}] the triangles $DBC$, $ECB$ are equal in all
respects;

therefore $EB$, $DC$ are equal, and the angles $BEC$, $CDB$ are equal.

The supplements of the latter angles are equal [\prop{1}{13}], and
hence the triangles $ABE$, $ACD$ have two angles equal respectively
and the side $BE$ equal to the side~$CD$.

Therefore [\prop{1}{26}] $AB$ is equal to~$AC$.

\end{notes}

\end{proposition}

\begin{proposition}
\label{prop:I_7}

\begin{statement}
Given two straight lines constructed on a straight line
(from its extremities) and meeting in a point, there cannot be
constructed on the same straight line (from its extremities),
and on the same side of it, two other straight lines meeting in
another point and equal to the former two respectively, namely
each to that which has the same extremity with it.
\end{statement}

\begin{proof}

For, if possible, given two straight lines $AC$, $CB$ constructed on
the straight line $AB$ and meeting at the point~$C$, let two other
straight lines

$AD$, $DB$ be constructed on the same straight line $AB$, on the same
side of it, meeting in another point~$D$ and equal to the former two
respectively, namely each to that which has the same extremity with
it, so that $CA$ is equal to $DA$ which has the same extremity $A$
with it, and $CB$ to~$DB$ which has the same extremity~$B$ with it;
and let $CD$ be joined.

Then, since $AC$ is equal to~$AD$, the angle $ACD$ is also equal to
the angle $ADC$; \using{\prop{1}{5}} therefore the angle $ADC$ is
greater than the angle $DCB$; therefore the angle $CDB$ is much
greater than the angle~$DCB$.

Again, since $CB$ is equal to~$DB$, the angle $CDB$ is also equal to
the angle $DCB$.  But it was also proved much greater than it: which
is impossible.

Therefore etc.
\end{proof}

\begin{annotations}

1—6. In an English translation of the enunciation of this proposition
it is absolutely necessary, in order to make it intelligible, to
insert some words which are not in the Greek.  The reason is partly
that the Greek enunciation is itself very elliptical, and partly that
some words used in it conveyed more meaning than the corresponding
words in English do.  Particularly is this the case with \greek{οὐ
  συσταθήσονται ἐπί} ``there shall not be constructed upon,'' since
\greek{συνίστασθαι} is the regular word for constructing a
\emph{triangle} in particular.  Thus a Greek would easily understand
\greek{συσταθήσονται ἐπί} as meaning the construction of two lines
\emph{forming a triangle on} a given straight line as base; whereas to
``construct two straight lines on a straight line'' is not in English
sufficiently definite unless we explain that they are drawn from the
\emph{ends} of the straight line to \emph{meet} at a point.  I have
had the less hesitation in putting in the words ``from its
extremities'' because they are actually used by Euclid in the somewhat
similar enunciation of \prop{1}{21}.

How impossible a literal translation into English is, if it is to
convey the meaning of the enunciation intelligibly, will be clear from
the following attempt to render literally: ``On the same straight line
there shall not be constructed two other straight lines equal, each to
each, to the same two straight lines, (terminating) at different
points on the same side, having the same extremities as the original
straight lines'' (\greek{ἐπὶ τῆς αὐτῆς εὐθείας δύο ταῖς αὐταῖς
  εὐθείαις ἄλλαι δύο εὐθεῖαι ἴσαι ἑκατέρα ἑκατέρᾳ οὐ συσταθήσονται
  πρὸς ἄλλῳ καὶ ἄλλῳ συμείῳ ἐπὶ τὰ αὐτὰ μέρη τὰ αὐτὰ πέρατα ἔχουσαι
  ταῖς ἐξ ἀρχῆς εὐθείαις}).

The reason why Euclid allowed himself to use, in this enunciation,
language apparently so obscure is no doubt that the phraseology was
traditional and therefore, vague as it was, had a conventional meaning
which the contemporary geometer well understood.  This is proved, I
think, by the occurrence in Aristotle (\emph{Meteorologica}~\r3.~5,
376~a~2 sqq.)\ of the very same, evidently technical, expressions.
Aristotle is there alluding to the theorem given by Eutocius from
Apollonius' \emph{Plane Loci} to the effect that, if $H$, $K$ be two
fixed points and $M$ such a variable point that the ratio of $MH$
to~$MK$ is a given ratio (not one of equality), the locus of~$M$ is a
circle.  (For an account of this theorem see note on \prop{6}{3}
below.)  Now Aristotle says ``The lines drawn up from $H$, $K$ in this
ratio cannot be constructed to two different points of the
semicircle~$A$ ``(\greek{αἰ οὖν ἀπὸ τῶν ΗΚ ἀναγόμεναι γραμμαὶ ἐν τούτῳ
  τῷ λόγῳ οὐ συσταθήσονται τοῦ ἐφ’ ᾦ Α ἡμικυκλίου πρὸς ἄλλο καὶ ἄλλο
  συμεῖον}).

If a paraphrase is allowed instead of a translation adhering as
closely as possible to the original, Simson's is the best that could
be found, since the fact that the straight lines form \emph{triangles}
on the same base is really conveyed in the Greek.  Simson's
enunciation is, \emph{Upon the same base, and on the same side of it,
  there cannot lie two triangles that have their sides which are
  terminated in one extremity of the base equal to one another, and
  likewise those which are terminated at the other extremity}.
Th.~Taylor (the translator of Proclus) attacks Simson's alteration as
``indiscreet'' and as detracting from the beauty and accuracy of
Euclid's enunciation which are enlarged upon by Proclus in his
commentary.  Yet, when Taylor says ``Whatever difficulty learners may
find in conceiving this proposition abstractedly is easily removed by
its exposition in the figure,'' he really gives his case away.  The
fact is that Taylor, always enthusiastic over his author, was nettled
by Simson's slighting remarks on Proclus' comments on the proposition.
Simson had said, with reference to Proclus' explanation of the bearing
of the second part of \prop{1}{5} on \prop{1}{7}, that it was not
``worth while to relate his trifles at full length,'' to which Taylor
retorts ``But Mr~Simson was no philosopher; and therefore the greatest
part of these Commentaries must be considered by him as trifles, from
the want of a philosophic genius to comprehend their meaning, and a
taste superior to that of a \emph{mere mathematician}, to discover
their beauty and elegance.''

20. It would be natural to insert here the step ``but the angle $ACD$
is greater than the angle $BCD$. [\rcn{5}.]''

21. \textbf{much greater}, literally ''greater by much'' (\greek{πολλῷ
  μείζων}).  Simson and those who follow him translate: ``\emph{much
  more then} is the angle $BDC$ \emph{greater} than the angle~$BCD$,''
but the Greek for this would have to be \greek{πολλῷ} (or
\greek{πολὺ}) \greek{μᾶλλόν ἐστι…μείζων}, however, though used by
Apollonius, is not, apparently, found in Euclid or Archimedes.

\end{annotations}

\begin{notes}

Just as in \prop{1}{6} we need a Postulate to justify theoretically
the statement that $CD$ falls within the angle $ACB$, so that the
triangle $DBC$ is less than the triangle $ABC$, so here we need
Postulates which shall satisfy us as to the relative positions of
$CA$, $CB$, $CD$ on the one hand and of $DC$, $DA$, $DB$ on the other,
in order that we may be able to infer that the angle $BDC$ is greater
than the angle $ADC$, and the angle $ACD$ greater than the
angle~$BCD$,

De Morgan (\emph{op.~cit.}\ p.~7) observes that \prop{1}{7} would be
made easy to beginners if they were first familiarised, as a common
notion, with ``if two magnitudes be equal, any magnitude greater than
the one is greater than any magnitude less than the other.''  I doubt
however whether a beginner would follow this easily; perhaps it would
be more easily apprehended tn the form ``if any magnitude~$A$ is
greater than a magnitude~$B$, the magnitude~$A$ is greater than any
magnitude equal to~$B$, and (\emph{a fortiori}) greater than any
magnitude less than~$B$.''

It has been mentioned already (note on \prop{1}{5}) that the second
case of \prop{1}{7} given by Simson and in our text-books generally is
not in the original text (the omission being in accordance with
Euclid's general practice of giving only one case, and that the most
difficult, and leaving the others to be worked out by the reader for
himself).  The second case is given by Proclus as the answer to a
possible \emph{objection} to Euclid's proposition, which should assert
that the proposition is not proved to be universally true, since the
proof given does not cover all possible cases.  Here the objector is
supposed to contend that what Euclid declares to be impossible may
still be possible if one pair of lines lie wholly within the other
pair of lines; and the second part of \prop{1}{5} enables the
objection to be refuted.

If possible, let $AD$, $DB$ be entirely within the triangle formed
by~$AC$, $CB$ with~$AB$, and let $AC$ be equal to~$AD$ and $BC$
to~$BD$.

Join $CD$, and produce $AC$, $AD$ to $E$ and~$F$.

Then, since $AC$ is equal to $AD$,
the triangle $ACD$ is isosceles,
and the angles $ECD$, $FDC$ under the base are equal.

But the angle $ECD$ is greater than the angle~$BCD$, therefore the
angle $FDC$ is also greater than the angle $BCD$.

Therefore the angle $BDC$ is greater by far than the angle $BCD$.

Again, since $DB$ is equal to~$CB$,
the angles at the base of the triangle BDC are equal, \using{\prop{1}{5}}
that is, the angle $BDC$ is equal to the angle $BCD$.

Therefore the same angle $BDC$ is both greater than and equal to the
angle $BCD$: which is impossible.

The case in which $D$ falls on $AC$ or $BC$ does not require proof.

I have already referred (note on \prop{1}{1}) to the mistake made by
those editors who regard \prop{1}{7} as being of no use except to
prove \prop{1}{8}.  What \prop{1}{7} proves is that if, in addition to
the base of a triangle, the length of the side terminating at each
extremity of the base is given, only one triangle satisfying these
conditions can be constructed on one and the same side of the given
base.  Hence not only does \prop{1}{7} enable us to prove \prop{1}{8},
but it supplements \prop{1}{1} and \prop{1}{22} by showing that the
constructions of those propositions give one triangle only on one and
the same side of the base.  But for \prop{1}{7} this could not be
proved except by anticipating \prop{3}{10}, of which therefore
\prop{1}{7} is the equivalent for Book~\book{1}.\ purposes.  Dodgson
(\emph{Euclid and his modern Rivals}, pp.~194—5) puts it in another
way. ``It [\prop{1}{7}] shows that, of all plane figures that can be
made by hingeing rods together, the \emph{three}-sided ones (and these
only) are \emph{rigid} (which is another way of stating the fact that
there cannot be \emph{two} such figures on the same base).  This is
analogous to the fact, in relation to solids contained by plane
surfaces hinged together, that \emph{any} such solid is rigid, there
being no maximum number of sides.  And there is a close analogy
between \prop{1}{7}, \prop*{1}{8} and \prop{3}{23}, \prop*{3}{24}.
These analogies give to geometry much of its beauty, and I think that
they ought not to be lost sight of.''  It will therefore be apparent
how ill-advised are those editors who eliminate \prop{1}{7} altogether
and rely on Philo's proof for \prop{1}{8}.

Proclus, it may be added, gives (pp.~268, 19—269, 10) another
explanation of the retention of \prop{1}{7}, notwithstanding that it
was apparently only required for \prop{1}{8}.  It was said that
astronomers used it to prove that three successive eclipses could not
occur at equal intervals of time, i.e.\ that the third could not
follow the second at the same interval as the second followed the
first; and it was argued that Euclid had an eye to this astronomical
application of the proposition.  But, as we have seen, there are other
grounds for retaining the proposition which are quite sufficient of
themselves.

\end{notes}

\end{proposition}

\begin{proposition}
\label{prop:I_8}

\begin{statement}
If two triangles have the two sides equal to two sides
respectively, and have also the base equal to the base, they will
also have the angles equal which are contained by the equal
straight lines.
\end{statement}

\begin{proof}

Let $ABC$, $DBF$ be two triangles having the two sides
$AB$, $AC$ equal to the two sides
$DE$, $DF$ respectively, namely
$AB$ to $DE$, and $AC$ to~$DF$ and
let them have the base $BC$ equal
to the base~$EF$;

I say that the angle $BAC$ is
also equal to the angle~$EDF$.

For, if the triangle $ABC$ be applied to the triangle $DEF$, and if
the point~$B$ be placed on the point~$E$ and the straight line $BC$
on~$EF$, the point~$C$ will also coincide with~$F$, because $BC$ is
equal to~$EF$.

Then, $BC$ coinciding with $EF$,
$BA$, $AC$ will also coincide with $ED$, $DF$;
for, if the base $BC$ coincides with the base $EF$, and the sides
$BA$, $AC$ do not coincide with $ED$, $DF$ but fall beside them
as $EG$, $GF$,

then, given two straight lines constructed on a straight
line (from its extremities) and meeting in a point, there will
have been constructed on the same straight line (from its
extremities), and on the same side of it, two other straight
lines meeting in another point and equal to the former
two respectively, namely each to that which has the same
extremity with it.

But they cannot be so constructed. \using{\prop{1}{7}}

Therefore it is not possible that, if the base $BC$ be applied
to the base $EF$, the sides $BA$, $AC$ should not coincide with
$ED$, $DF$;
they will therefore coincide,
so that the angle $BAC$ will also coincide with the angle
$EDF$, and will be equal to it.

If therefore etc.
\end{proof}

\begin{annotations}

19. \textbf{$BA$, $AC$}. The text has here ``$BA$, $CA$.''

21, \emph{fall beside them}.  The Greek has the future,
\greek{παραλλάξουσι}.  \greek{παραλλάττω} means ``to pass by without
touching,'' ``to miss'' or ``to deviate.''

\end{annotations}

\begin{notes}

As pointed out above (p.~\pageref{257}) \prop{1}{8} is a
\emph{partial} converse of \prop{1}{4}.

It is to be observed that in \prop{1}{8} Euclid is satisfied with
proving the equality of the vertical angles and does not, as in
\prop{1}{4}, add that the triangles are equal, and the remaining
angles are equal respectively.  The reason is no doubt (as pointed out
by Proclus and by Savile after him) that, when once the vertical
angles are proved equal, the rest follows from \prop{1}{4}, and there
is no object in proving again what has been proved already.

Aristotle has an allusion to the theorem of this proposition in
\emph{Meteorologia} \r3.~3, 373~a~5—16.  He is speaking of the rainbow
and observes that, if equal rays be reflected from one and the same
point to one and the same point, the points at which reflection takes
place are on the circumference of a circle.  ``For let the broken
lines $ACB$, $AFB$, $ADB$ be all reflected from the point~$A$ to the
point~$B$ (in such a way that) $AC$, $AF$, $AD$ are all equal to one
another, and the lines (terminating) at~$B$, i.e.\ $CB$, $FB$, $DB$,
are likewise all equal; and let $AEB$ be joined.  It follows that
\emph{the triangles are equal}; for they are upon the equal (base)
$AEB$.''

Heiberg (\emph{Mathematisches zu Aristoteles}, p.~18) thinks that the
form of the conclusion quoted is an indication that in the
corresponding proposition to Eucl.\ \prop{1}{8}, as it lay before
Aristotle, it was maintained that the \emph{triangles} were equal, and
not only the angles, and ``we see here therefore, in a clear example,
how the stones of the ancient fabric were recut for the rigid
structure of his \emph{Elements}.''  I do not, however, think that
this inference from Aristotle's language as to the form of the
pre-Euclidean proposition is safe.  Thus if we, nowadays, were arguing
from the data in the passage of Aristotle, we should doubtless infer
directly that the triangles are equal in all respects, quoting
\prop{1}{8} alone.  Besides, Aristotle's language is rather careless,
as the next sentences of the same passage show. ``Let
perpendiculars,'' he says, ``be drawn to $AEB$ from the angles, $CE$
from~$C$, $FE$ from~$F$ and $DE$ from~$D$.  These, then, are equal;
for they are all in equal triangles, and in one plane; for all of them
are perpendicular to $AEB$, and they meet at one point~$E$.  Therefore
the (line) drawn (through $C$, $F$, $D$) will be a circle, and its
centre (will be)~$E$.''  Aristotle should obviously have proved that
the three perpendiculars \emph{will} meet at one point~$E$ on $AEB$
before he spoke of drawing the perpendiculars $CE$, $FE$, $DE$.  This
of course follows from their being ``in equal triangles'' (by means of
Eucl.\ \prop{1}{26}); and then, from the fact that the perpendiculars
meet at one point on~$AB$, it can be inferred that all three are in
one plane.

\subsection*{Philo's proof of \prop{1}{8}}

This alternative proof avoids the use of \prop{1}{7}, and it is
elegant; but it is inconvenient in one respect, since three cases have
to be distinguished.  Proclus gives the proof in the following order
(pp.~266, 15—268, 14).

Let $ABC$, $DEF$ be two triangles having the sides $AB$, $AC$ equal to
the sides $DE$, $DF$ respectively, and the base $BC$ equal to the
base~$EF$.

Let the triangle $ABC$ be applied to the triangle $DEF$, so that $B$
is placed on~$E$ and $BC$ on~$EF$, but so that $A$ falls on the
opposite side of $EF$ from~$D$, taking the position~$G$.  Then $C$
will coincide with~$F$, since $BC$ is equal to~$EF$.

Now $FG$ will either be in a straight line with $DF$, or make an angle
with it, and in the latter case the angle will either be
\emph{interior} (\greek{κατὰ τὸ ἔντός}) to the figure or \emph{exterior}
(\greek{κατὰ τὸ ἐκτός}).

I. Let $FG$ be in a straight line with~$DF$.

Then, since $DE$ is equal to $EG$, and $DFG$ is a straight line,

$DEG$ is an isosceles triangle, and the angle at~$D$ is equal to the
angle at~$G$. \using{\prop{1}{5}}

II. Let $DF$, $FG$ form an angle interior to the figure.

Let $DG$ be joined.

Then, since $DE$, $EG$ are equal,
the angle $EDG$ is equal to the angle $EGD$.

Again, since $DF$ is equal to~$FG$,
the angle $FDG$ is equal to the angle~$FGD$.

Therefore, by addition,
the whole angle $EDF$ is equal to the
whole angle~$EGF$.

III. Let $DF$, $FG$ form an angle \emph{exterior} to the figure.

Let $DG$ be joined.

The proof proceeds as in the last case,
except that subtraction takes the place of
addition, and
the remaining angle $EDF$ is equal to the
remaining angle~$ECF$.

Therefore in all three cases the angle
$EDF$ is equal to the angle $EGF$, that is,
to the angle~$BAC$.

It will be observed that, in accordance with the practice of the Greek
geometers in not recognising as an ``angle'' any angle not less than
two right angles, the re-entrant angle of the quadrilateral $DEGF$
ignored and the angle $DFG$ is said to be \emph{outside} the figure.

\end{notes}

\end{proposition}

\begin{proposition}
\label{prop:I_9}

\begin{statement}
To bisect a given rectilineal angle.
\end{statement}

\begin{proof}

Let the angle $BAC$ be the given rectilineal angle.

Thus it is required to bisect it.

Let a point $D$ be taken at random on~$AB$;
let $AE$ be cut off from $AC$ equal to~$AD$; \using{\prop{1}{3}}
let $DE$ be joined, and on $DE$ let the equilateral
triangle $DEF$ be constructed;
let $AF$ be joined.

I say that the angle $BAC$ has been bisected by the
straight line~$AF$.

For, since $AD$ is equal to~$AE$,
and $AF$ is common,

the two sides $DA$, $AF$ are equal to the two sides
$EA$, $AF$ respectively.

And the base $DF$ is equal to the base~$EF$;

therefore the angle $DAF$ is equal to the angle~$EAF$. \using{\prop{1}{8}}

Therefore the given rectilineal angle $BAC$ has been
bisected by the straight line $AF$.
\qef
\end{proof}

\begin{notes}

It will be observed from the translation of this proposition that
Euclid does not say, in his description of the construction, that the
equilateral triangle should be constructed on the side of $DE$
opposite to~$A$; he leaves this to be inferred from his figure.  There
is no particular value in Proclus' explanation as to how we should
proceed in case any one should assert that he could not recognise the
existence of any space below~$DE$.  He supposes, then, the equilateral
triangle described on the side of $DE$ towards~$A$, and hence has to
consider three cases according as the vertex of the equilateral
triangle falls on~$A$, above~$A$ or below it.  The second and third
cases do not differ substantially from Euclid's.  In the first case,
where $ADE$ is the equilateral triangle constructed on~$DE$, take any
point~$F$ on $AD$, and from $AE$ cut off $AG$ equal to~$AF$.  Join
$DG$, $EF$ meeting in~$H$ and join $AH$.  Then $AH$ is the bisector
required.

Proclus also answers the possible \emph{objection} that might be
raised to Euclid's proof on the ground that it assumes that, if the
equilateral triangle be described on the side of~$DE$ opposite to~$A$,
its vertex~$F$ will lie within the angle~$BAC$.  The objector is
supposed to argue that this is not necessary, but that $F$ might fall
either on one of the lines forming the angle or outside it altogether.
The two cases are disposed of thus.

Suppose $F$ to fall as shown in the two figures below respectively.

Then, since $FD$ is equal to $FE$,
the angle $FDE$ is equal to the angle~$FED$.

Therefore the angle $CED$ is greater than the angle $FDE$; and, in the
second figure, \emph{a fortiori}, the angle $CED$ is greater than the
angle~$BDE$.

But, since $ADE$ is an isosceles triangle, and the equal sides are produced,
the angles under the base are equal,
i.e., the angle $CED$ is equal to the angle $BDE$.

But the angle $CED$ was proved greater: which is impossible.

Here then is the second case in which, in Proclus' view, the second
part of \prop{1}{5} is useful for refuting objections.

On this proposition Proclus takes occasion (p.~271, 15–19) to
emphasize the fact that the given angle must be rectilineal, since the
bisection of any sort of angle (including angles made by curves with
one another or with straight lines) is not matter for an elementary
treatise, besides which it is questionable whether such bisection is
always possible. ``Thus it is difficult to say whether it is possible
to bisect the so-called \emph{horn-like} angle ``(formed by the
circumference of a circle and a tangent to it).

\subsection*{Trisection of an angle}

Further it is here that Proclus gives us his valuable historical note
about the \emph{trisection} of any acute angle, which (as well as the
division of an angle in any given ratio) requires resort to other
curves than circles, i.e.\ curves of the species which, after Geminus,
he calls ``mixed.'' ``This,'' he says (p.~272, 1—12), ``is shown by
those who have set themselves the task of trisecting such a given
rectilineal angle.  For Nicomedes trisected any rectilineal angle by
means of the \emph{conchoidal} lines, the origin, order, and
properties of which he has handed down to us, being himself the
discoverer of their peculiarity.  Others have done the same thing by
means of the \emph{quadratrices} of Hippias and Nicomedes, thereby
again using 'mixed' curves.  But others, starting from the Archimedean
spirals, cut a given rectilineal angle in a given ratio.''

(\emph{a}) Trisection by means of the \emph{conchoid}.

I have already spoken of the \emph{conchoid} of Nicomedes (note on
Def.~\ref{def:I_2}, pp.~160–1); it remains to show how it could be
used for trisecting an angle.  Pappus explains this (\r4.\ pp.~274—5)
as follows.

Let $ABC$ be the given acute angle, and from any point~$A$ in~$AB$
draw $AC$ perpendicular to~$BC$.

Complete the parallelogram $FBCA$ and produce $FA$ to a point~$E$ such
that, if $BE$ be joined, \emph{$BE$ intercepts between $AC$ and $AE$ a
  length $DE$ equal to twice $AB$}.

I say that the angle $EBC$ is one-third of the angle $ABC$.

For, joining $A$ to~$G$, the middle point of~$DE$, we have the three
straight lines $AG$, $DG$, $EG$ equal, and the angle $AGD$ is double
of the angle $AED$ or~$EBC$.

But $DE$ is double of~$AB$;
therefore $AG$, which is equal to~$DG$, is equal to~$AB$.

Hence the angle $AGD$ is equal to the angle $ABG$.

Therefore the angle $ABD$ is also double of the angle~$EBC$;
so that the angle $EBC$ is one-third of the angle~$ABC$.

So far Pappus, who reduces the construction to the drawing of~$BE$ so
that $DE$ shall be equal to twice~$AB$.

This is what the conchoid constructed with~$B$ as \emph{pole}, $AC$ as
\emph{directrix}, and \emph{distance} equal to twice $AB$ enables us
to do; for that conchoid cuts $AE$ in the required point~$E$.

(\emph{b})~Use of the \emph{quadratrix}.

The plural \emph{quadratrices} in the above passage is a Hellenism for
the singular \emph{quadratrix}, which was a curve discovered by
Hippias of Elis about 420~\bc. According to Proclus (p.~356, 11)
Hippias proved its properties; and we are told (1)~in the passage
quoted above that Nicomedes also investigated it and that it was used
for trisecting an angle, and (2)~by Pappus (\r4.\ pp.~350, 33–252, 4)
that it was used by Dinostratus and Nicomedes and some more recent
writers for squaring the circle, whence its name.  It is described
thus (Pappus~\r4.\ p.~252).

Suppose that $ABCD$ is a square and $BED$ a quadrant of a circle with
centre~$A$.

Suppose (1)~that a radius of the circle moves
uniformly about~$A$ from the position~$AB$ to the
position~$AD$, and (2)~that in the same time the
line $BC$ moves uniformly, always parallel to itself,
and with its extremity~$B$ moving along~$BA$, from
the position $BC$ to the position~$AD$.

Then the radius $AE$ and the moving line $BC$ determine at any instant
by their intersection a point~$F$.

The locus of $F$ is the \emph{quadratrix}.

The property of the curve is that, if $F$ is any point, the arc $BED$
is to the arc $ED$ as $AB$ is to~$FH$.

In other words, if $\phi$ is the angle $FAD$, $\rho$ the radius vector
$AF$ and $a$ the side of the square,
\[
    (\rho \sin \phi)/a = \phi/\frac{1}{2}\pi.
\]

Now the angle $EAD$ can not only be \emph{trisected} but \emph{divided
  in any given ratio} by means of the quadratrix (Pappus~\r4.\ p.~286).

For let $FH$ be divided at $K$ in the given ratio.

Draw $KL$ parallel to~$AD$, meeting the curve in~$L$; join $AL$ and
produce it to meet the circle in~$N$.

Then the angles $EAN$, $NAD$ are in the ratio of $FK$ to~$KH$, as is
easily proved.

(\emph{c}) Use of the \emph{spiral of Archimedes}.

The trisection of an angle, or the division of an angle in any ratio,
by means or the \emph{spiral of Archimedes} is of course an equally
simple matter.  Suppose any angle included between the two radii
vectores $OA$ and~$OB$ of the spiral, and let it be required to cut
the angle $AOB$ in a given ratio.  Since the radius vector increases
proportionally with the angle described by the vector which generates
the curve (reckoned from the original position of the vector
coinciding with the initial line to the particular position assumed),
we have only to take the radius vector $OB$ (the greater of the two
$OA$, $OB$), mark off $OC$ along it equal to $OA$, cut $CB$ in the
given ratio (at $D$ say), and then draw the circle with centre $O$ and
radius $OD$ cutting the spiral in~$E$.  Then $OE$ will divide the
angle $AOB$ in the required manner.

\end{notes}

\end{proposition}

\begin{proposition}
\label{prop:I_10}

\begin{statement}
To bisect a given finite straight line.
\end{statement}

\begin{proof}

Let $AB$ be the given finite straight line.

Thus it is required to bisect the finite straight line~$AB$.

Let the equilateral triangle $ABC$ be constructed on it,
\using{\prop{1}{1}}

and let the angle $ACB$ be bisected by the straight line $CD$;
\using{\prop{1}{9}}

I say that the straight line $AB$ has been bisected at the point~$D$.

For, since $AC$ is equal to~$CB$, and $CD$ is common,

the two sides $AC$, $CD$ are equal to the two sides~$BC$, $CD$
respectively;

and the angle $ACD$ is equal to the angle $BCD$;

therefore the base $AD$ is equal to the base $BD$. \using{\prop{1}{4}}
Therefore the given finite straight line AB has been
bisected at~$D$.
\qef
\end{proof}

\begin{notes}

Apollonius, we are told (Proclus, pp.~279, 16—280,~4), bisected a
straight line $AB$ by a construction tike that of \prop{1}{1}.  With
centres $A$, $B$, and radii $AB$, $BA$ respectively, two circles are
described, intersecting in~$C$, $D$.  Joining $CD$, $AC$, $CB$, $AD$,
$DB$, Apollonius proves in two steps that $CD$ bisects~$AB$.

(1)~Since, in the triangles $ACD$, $BCD$, two sides $AC$, $CD$ are
equal to two sides $BC$, $CD$, and the bases $AD$, $BD$ are equal, the
angle $ACD$ is equal to the angle $BCD$.
\using{\prop{1}{8}}

(2)~The latter angles being equal, and $AC$ being equal to~$CB$, while
$CE$ is common, the equality of $AE$, $EB$ follows by \prop{1}{4}.

The objection to this proof is that, instead of \emph{assuming} the
bisection of the angle~$ACB$, as already effected by \prop{1}{9},
Apollonius goes a step further back and embodies a construction for
bisecting the angle.  That is, he unnecessarily does over again what
has been done before, which is open to objection from a theoretical
point of view.

Proclus (pp.~277, 25—279, 4) warns us against being moved by this
proposition to conclude that geometers assumed, as a preliminary
hypothesis, that a line is not made up of indivisible parts (\greek{ἐξ
  ἀμερῶν}).  This might be argued thus.  If a line is made up of
indivisibles, there must be in a finite line either an odd or an even
number of them.  If the number were odd, it would be necessary in
order to bisect the line to bisect an indivisible (the odd one).  In
that case therefore it would not be possible to bisect a straight
line, if it is a magnitude made up of indivisibles.  But, if it is not
so made up, the straight line can be divided \emph{ad infinitum} or
without limit (\greek{ἐπ’ ἄπειρον διαιρεῖται}).  Hence it was argued
(\greek{φασίν}), says Proclus, that the divisibility of magnitudes
without limit was admitted and assumed as a geometrical principle.  To
this he replies, following Geminus, that geometers did indeed assume,
by way of a common notion, that a continuous magnitude, i.e.\ a
magnitude consisting of parts connected together (\greek{συνημμένων}),
is divisible (\greek{διαιρετόν}).  But \emph{infinite} divisibility
was not assumed by them; it was \emph{proved} by means of the first
principles applicable to the case.  ``For when,'' he says, ``they
prove that the incommensurable exists among magnitudes, and that it is
not all things that are commensurable with one another, what else will
any one say that they prove but that every magnitude can be divided
for ever, and that we shall never arrive at the indivisible, that is,
the least common measure of the magnitudes?  This then is matter of
demonstration, whereas it is an \emph{axiom} that everything
continuous is divisible, so that a finite continuous line is
divisible.  The writer of the Elements bisects a finite straight line,
starting from the latter notion, and not from any assumption that it
is divisible without limit.''  Proclus adds that the proposition may
also serve to refute Xenocrates' theory of indivisible lines
(\greek{ἄτομοι γραμμαί}).  The argument given by Proclus to disprove
the existence of indivisible lines is substantially that used by
Aristotle as regards magnitudes generally (cf.\ \emph{Physics} \r6.~1,
231~a~21 sqq.\ and especially \r6.~2, 233~b~15—32).

\end{notes}

\end{proposition}

\begin{proposition}
\label{prop:I_11}

\begin{statement}
To draw a straight line at right angles to a given straight line from
a given point on it.
\end{statement}

\begin{proof}

Let $AB$ be the given straight line, and $C$ the given point on it.

\sidefig{propI_11}

Thus it is required to draw from the point~$C$ a straight line at
right angles to the straight line~$AB$.

Let a point D be taken at random on~$AC$;
let $CE$ be made equal to $CD$; \using{\prop{1}{3}}
on~$DE$ let the equilateral triangle $FDE$ be constructed,
\using{\prop{1}{1}}
and let $FC$ be joined;

I say that the straight line $FC$ has been drawn at right angles to
the given straight line $AB$ from~$C$ the given point on it.

For, since $DC$ is equal to~$CE$, and $CF$ is common,

the two sides $DC$, $CF$ are equal to the two sides $EC$, $CF$
respectively;

and the base $DF$ is equal to the base~$FE$;

therefore the angle $DCF$ is equal to the angle $ECF$ \using{\prop{1}{8}}
and they are adjacent angles.

But, when a straight line set up on a straight line makes
the adjacent angles equal to one another, each of the equal
angles is right; \using{\rdef{I_10}}
therefore each of the angles DCF, FCE is right.

Therefore the straight line $CF$ has been drawn at right angles to the
given straight line $AB$ from the given point~$C$ on it.
\qef
\end{proof}

\begin{annotations}

10. \textbf{let $CB$ be made equal to $CD$}.  The verb is
\greek{κείσθω} which, as well as the other parts of \greek{κεῖμαι}, a
constantly used for the passive of \greek{τίθημι} ``to \emph{place}'';
and the latter word it constantly used in the sense of \emph{making},
e.g., one straight line equal to another straight line.

\end{annotations}

\begin{notes}

De Morgan remarks that this proposition, which is ``to bisect the
angle made by a straight line and its continuation'' [i.e.\ a
  \emph{flat} angle], should be a particular case of \prop{1}{9}, the
constructions being the same.  This is certainly worth noting, though
I doubt the advantage of rearranging the propositions in consequence.

Apollonius gave a construction for this proposition (see Proclus,
p.~282,~8) differing from Euclid's in much the same way as his
construction for bisecting a straight line differed from that of
\prop{1}{10}.  Instead of assuming an equilateral triangle drawn
without repeating the process of~\prop{1}{1}, Apollonius takes $D$
and~$E$ equidistant from~$C$ as in Euclid, and then draws circles in
the manner of
\infig{propI_11a}
\prop{1}{1} meeting at~$F$.  This necessitates proving again that $DF$
is equal to~$FE$ whereas Euclid's assumption of the construction of
\prop{1}{1} in the words ``let the equilateral triangle $FDE$ be
constructed'' enables him to dispense with the drawing of circles and
with the proof that $DF$ is equal to~$FE$ at the same time.  While
however the substitution of Apollonius' constructions for \prop{1}{10}
and \prop*{1}{11} would show faulty arrangement in a theoretical
treatise like Euclid's, they are entirely suitable for what we call
practical geometry, and such may have been Apollonius' object in these
constructions and in his alternative for~\prop{1}{23}.

Proclus gives a construction for drawing a straight line at right
angles to another straight line but from one end of it, instead of
from an intermediate point on it, it being supposed (for the sake of
argument) that we are not permitted to \emph{produce} the straight
line.  In the commentary of an-Nairīzī (ed.\ Besthorn-Heiberg,
pp.~73—4; ed.\ Curtze, pp.~54—5) this construction is attributed to
Heron.

Let it be required to draw from~$A$ a straight line at right angles
to~$AB$.

On $AB$ take any point~$C$, and in the manner of the proposition draw
$CE$ at right angles to~$AB$.

\sidefig{propI_11b}

From $CE$ cut off $CD$ equal to $AC$, bisect the angle $ACE$ by the
straight line~$CF$, \using{\prop{1}{9}} and draw $DF$ at right angles
to~$CE$ meeting~$CF$ in~$F$.  Join~$FA$.

Then the angle $FAC$ will be a right angle.

For, since, in the triangles $ACF$, $DCF$, the two sides $AC$, $CF$
are equal to the two sides $DC$, $CF$ respectively, and the included
angles $ACF$, $DCF$ are equal,

the triangles are equal in all respects. \using{\prop{1}{4}}

Therefore the angle at~$A$ is equal to the angle at~$D$, and is
accordingly a right angle.

\end{notes}

\end{proposition}

\begin{proposition}
\label{prop:I_12}

\begin{statement}
To a given infinite straight line, from a given point which is not on
it, to draw a perpendicular straight line.
\end{statement}

\begin{proof}

Let $AB$ be the given infinite straight line, and $C$ the given point
which is not on it; Thus it is required to draw to the given infinite
straight line~$AB$, from the given point~$C$ which is not on it, a
perpendicular straight line.

For let a point $D$ be taken
at random on the other side of
the straight line $AB$, and with
centre~$C$ and distance~$CD$ let
the circle $EFG$ be described;
\using{\rpost{3}}
let the straight line $EG$
be bisected at~$H$, \using{\prop{1}{10}}
and let the straight lines $CG$, $CH$, $CE$ be joined.

\using{\rpost{1}} I say that $CH$ has been drawn perpendicular to the
given infinite straight line~$AB$ from the given point~$C$ which is
not on it.

For, since $GH$ is equal to~$HE$, and $HC$ is common,
the two sides $GH$, $HC$ are equal to the two sides
$EH$, $HC$ respectively;
and the base $CG$ is equal to the base~$CE$;
therefore the angle $CHG$ is equal to the angle~$EHC$.
\using{\prop{1}{8}}

And they are adjacent angles.

But, when a straight line set up on a straight line makes
the adjacent angles equal to one another, each of the equal
angles is right, and the straight line standing on the other is
called a perpendicular to that on which it stands.
\using{\rdef{I_10}}

Therefore $CH$ has been drawn perpendicular to the given infinite
straight line $AB$ from the given point~$C$ which is not on it
\qef
\end{proof}

\begin{annotations}

2. \textbf{a perpendicular straight line}, \greek{κάθετον εὐθεῖαν
  γραμμήν}.  This is the full expression for a \emph{perpendicular},
\greek{κάθετος} meaning \emph{let fall} or \emph{let down}, so that
the expression corresponds to our \emph{plumb-line}.  \greek{ἡ κάετος}
is however constantly used alone for a perpendicular, \greek{γραμμή}
being understood.

10. \emph{on the other side of the straight line $AB$}, literally
``towards the other parts of the straight line $AB$,'' \greek{ἐπι1 τὰ
  ἕτερα μέρη τῆς ΑΒ}.  Cf.\ ``on the same side'' (\greek{ἐπὶ τὰ αὐτὰ
  μέρη}) in Post.~\ref{post:5} and ``in both directions'' (\greek{ἐφ’
  ἑκάτερα τὰ μέρη}) in Def.~\ref{def:I_13}.

\end{annotations}

\begin{notes}

``This problem,'' says Proclus (p.~183, 7—10), ``was first
  investigated by Oenopides [5th cent~\bc], who thought it useful for
  astronomy. He however calls the perpendicular, in the archaic
  manner, (a line drawn) \emph{gnomon-wise} (\greek{κατὰ γνώμονα}),
  because the gnomon is also at right angles to the horizon.'' In this
  earlier sense the \emph{gnomon} was a staff placed in a vertical
  position for the purpose of casting shadows and so serving as a
  means of measuring time (Cantor, \emph{Geschichte der Mathematik},
  \r1\tsup{3}, p.~161). The later meanings of the word as used in
  Eucl.\ Book~\r2.\ and elsewhere will be explained in the note on
  Book~\r2.~Def.~\ref{def:II_2},

Proclus says that two kinds of perpendicular were distinguished, the
``plane'' (\greek{ἐπίπεδος}) and the ``solid'' (\greek{στερεά}), the
former being the perpendicular dropped on a line is a plane and the
latter the perpendicular dropped on a plane.  The term ``solid
perpendicular'' is sufficiently curious, but it may perhaps be
compared with the Greek term ``solid locus'' applied to a conic
section, apparently on the ground that it has its origin in the
section of a solid, namely a cone.

Attention is called by most editors to the assumption in this
proposition that, if only $D$ be taken on the side of~$AB$ remote
from~$C$, the circle described with~$CD$ as radius must necessarily
cut~$AB$ in two points.  To satisfy us of this we need, as in
\prop{1}{1}, some postulate of continuity, e.g.\ something like that
suggested by Killing (see note on the Principle of Continuity above,
p.~\pageref{235}): ``If a point [here the point describing the circle]
moves in a figure which is divided into two parts [by the straight
  line], and if it belongs at the beginning of the motion to one part
and at another stage of the motion to the other part, it must during
the motion cut the boundary between the two parts,'' and this of
course applies to the motion in \emph{two} directions from~$D$.

But the editors have not, as a rule, noticed a possible
\emph{objection} to the Euclidean statement of this problem which is
much more difficult to dispose of at this stage, i.e.\ without
employing any proposition later than this in Euclid's order.  How do
we know, says the supposed critic, that the circle does not cut $AB$
in \emph{three} or more points, in which case there would be not one
perpendicular but \emph{three} or more?  Proclus (pp.~286, 12—289,~6)
tries to refute this objection, and it is interesting to follow his
argument, though it will easily be seen to be inconclusive.  He takes
in order three possible suppositions.

1.~May not the circle meet $AB$ in a third point~$K$ between the
middle point of~$GE$ and either extremity of it, taking the form drawn
in the figure appended?

Suppose this possible.  Bisect $GE$ in~$H$. Join~$CH$, and produce it
to meet the circle in~$L$.  Join $CG$, $CK$, $CE$.

Then, since $CG$ is equal to~$CE$, and $CH$ is common, while the base
$GH$ is equal to the base~$HE$,

the angles $CHG$, $CHE$ are equal and, since they are adjacent, they
are both right.

Again, since $CG$ is equal to~$CE$, the angles at $G$ and~$E$ are
equal.

Lastly, since $CK$ is equal to~$CG$ and also to~$CE$, the angles
$CGK$, $CKG$ are equal, as also are the angles $CKE$, $CEK$.

Since the angles $CGK$, $CEK$ are equal, it follows that
the angles $CKG$, $CKE$ are equal and therefore both right.

Therefore the angle $CKH$ equal to the angle $CHK$, and $CH$ is equal
to~$CK$.

But $CK$ is equal to~$CL$, by the definition of the circle; therefore
$CH$ is equal to~$CL$: which is impossible.

Thus Proclus; but why should not the circle meet $AB$ in~$H$ as well
as~$K$?

2.~May not the circle meet $AB$ in the middle point of~$GE$ and take
the form shown in the second figure?

In that case, says Proclus, join $CG$, $CH$, $CE$ as before.  Then
bisect $ME$ at~$K$, join $CK$ and produce it to meet the circumference
at~$L$.

Now, since $HK$ is equal to~$KE$, $CK$ is common, and the base~$CH$ is
equal to the base~$CE$,

the angles at~$K$ are equal and therefore both right angles.

Therefore the angle $CHK$ is equal to the angle~$CKH$, whence $CK$ is
equal to~$CH$ and therefore to~$CL$: which is impossible.

So Proclus; but why should not the circle meet $AB$ in~$K$ as well
as~$H$?

3.~May not the circle meet $AB$ in two points besides $G$, $E$ and
pass, between those two points, to the side of~$AB$ towards~$C$, as in
the next figure?

Here again, by the same method, Proclus proves that, $K$, $L$ being
the other two points in which the circle cuts~$AB$, $CK$ is equal to
$CH$, and, since the circle cuts $CH$ in~$M$, $CM$ is equal to~$CK$
and therefore to~$CH$: which is impossible.

But, again, why should the circle not cut $AB$ in the point~$H$ as
well?

In fact, Proclus' cases are not mutually exclusive, and his method of
proof only enables us to show that, if the circle meets $AB$ in one
more point besides $G$, $E$, it must meet it in more points still.  We
can always find a new point of intersection by bisecting the distance
separating any two points of intersection, and so, applying the method
ad infinitum, we should have to conclude ultimately that the circle
with radius $CH$ (or~$CG$) coincides with~$AB$.  It would follow that
a circle with centre~$C$ and radius greater than $CH$ would not meet
$AB$ at all.  Also, since all straight lines from~$C$ to points
on~$AB$ would be equal in length, there would be an infinite number of
perpendiculars from~$C$ on~$AB$.

Is this under any circumstances possible?  It is not possible in
Euclidean space, but it is possible, under the Riemann hypothesis
(where a straight line is a ``closed series'' and returns on itself),
in the case where $C$ is the pole of the straight line~$AB$.

It is natural therefore that, for a proof that in Euclidean space
there is only one perpendicular from a point to a straight line, we
have to wait until \prop{1}{16}, the precise proposition which under
the Riemann hypothesis is only valid with a certain restriction and
not universally.  There is no difficulty involved by waiting until
\prop{1}{16}, since \prop{1}{12} is not used before that proposition
is reached; and we are only in the same position as when, in order to
satisfy ourselves of the number of possible solutions of \prop{1}{1},
we have to wait till~\prop{1}{7}.

But if we wish, after all, to prove the truth of the assumption
\emph{without} recourse to any later proposition than \prop{1}{12}, we
can do so by means of this same invaluable~\prop{1}{7}.

If the circle intersects $AB$ as before in $G$, $E$, let $H$ be the
middle point of $GE$, and suppose, if possible, that the circle also
intersects $AB$ in any other point~$K$ on~$AH$.

From~$H$, on the side of $AB$ opposite to~$C$, draw $HL$ at right angles
to $AB$, and make $HL$ equal to~$HC$.

Join $CG$, $EG$, $CK$, $LK$.

Now, in the triangles $CHG$, $LHG$, $CH$ is equal to $LH$, and $HG$ is
common.

Also the angles $CHG$, $LHG$, being both right, are equal.

Therefore the base $CG$ is equal to the base~$LG$.

Similarly we prove that $CK$ is equal to~$LK$.

But, by hypothesis, since $K$ is on the circle,
$CK$ is equal to~$CG$.

Therefore $CG$, $CK$, $LG$, $LK$ are all equal.

Now the next proposition, \prop{1}{13}, will tell us that $CH$, $HL$
are in a straight line; but we will not assume this.  Join~$CL$.

Then on the same base $CL$ and on the same side of it we have two
pairs of straight lines drawn from $C$, $L$ to~$G$ and $K$ such that
$CG$ is equal to~$CK$ and $LG$ to~$LK$.

But this is impossible [\prop{1}{7}].

Therefore the circle cannot cut $BA$ or $BA$ produced in any point
other than~$G$ on that side of $CL$ on which $G$ is.

Similarly it cannot cut $AB$ or $AB$ produced at any point other
than~$E$ on the other side of~$CL$.

The only possibility left therefore is that the circle might cut $AB$
in the same point as that in which $CL$ cuts it.  But this is shown to
be impossible by an adaptation of the proof of \prop{1}{7},

For the assumption is that there may be some point~$M$ on $CL$ such
that $CM$ is equal to $CG$ and $LM$ to~$LG$.

If possible, let this be the case, and produce $CG$ to~$N$.

Then, since $CM$ is equal to $CG$, the angle $NGM$ is equal to the
angle~$GML$ [\prop{1}{5}, part~2].

Therefore the angle $GML$ is greater than the angle~$MGL$.

Again, since $LG$ is equal to~$LM$, the angle $GML$ is equal to the
angle~$MGL$.

But it was also greater: which is impossible.

Hence the circle in the original figure cannot cut $AB$ in the point
in which $CL$ cuts it.

Therefore the circle cannot cut $AB$ in any point whatever except $G$
and~$E$.

[This proof of course does not prove that $CK$ is \emph{less}
  than~$CG$, but only that it is not equal to it.  The proposition
  that, of the obliques drawn from $C$ to~$AB$, that is less the foot
  of which is nearer to~$H$ can only be proved later.  The proof by
  \prop{1}{7} also fails, under the Riemann hypothesis, if $C$, $L$
  are the poles of the straight line $AB$, since the broken lines
  $CGL$, $CKL$ etc.\ become equal straight lines, all perpendicular
  to~$AB$.]

Proclus rightly adds (p.~289, 18 sqq.)\ that it is not
\emph{necessary} to take $D$ on the side of $AB$ away from~$A$ if an
objector ``says that there is no space on that side.''  If it is not
desired to trespass on that side of~$AB$, we can take $D$ anywhere
on~$AB$ and describe the arc of a circle between $D$ and the point
where it meets $AB$ again, drawing the arc on the side of $AB$ on
which $C$ is.  If it should happen that the selected point~$D$ is such
that the circle only meets $AB$ in \emph{one} point ($D$ itself), we
have only to describe the circle with $CD$ as radius, then, if $E$ be
a point on this circle, take $F$ a point further from~$C$ than $E$ is,
and describe with $CF$ as radius the circular arc meeting $AB$ in two
points.

\end{notes}

\end{proposition}

\begin{proposition}
\label{prop:I_13}

\begin{statement}
If a straight line set up on a straight line make angles, it wilt make
either two right angles or angles equal to two right angles.
\end{statement}

\begin{proof}

For let any straight line $AB$ set up on the straight line
$CD$ make the angles $CBA$, $ABD$;

I say that the angles $CBA$, $ABD$ are either two right angles or
equal to two right angles.

Now, if the angle $CBA$ is equal to the angle~$ABD$,

they are two right angles. \using{\rdef{I_10}}

But, if not, let $BE$ be drawn from the point~$B$ at right
angles to~$CD$; \using{\prop{1}{11}}

therefore the angles $CBE$, $EBD$ are two right angles.

Then, since the angle $CBE$ is equal to the two angles
$CBA$, $ABE$,
let the angle $EBD$ be added to each;
therefore the angles $CBE$, $EBD$ are equal to the three
angles $CBA$, $ABE$, $EBD$. \using{\rcn{2}}

Again, since the angle $DBA$ is equal to the two angles
$DBE$, $EBA$,
let the angle $ABC$ be added to each;
therefore the angles $DBA$, $ABC$ are equal to the three
angles $DBE$, $EBA$, $ABC$. \using{\rcn{2}}

But the angles $CBE$, $EBD$ were also proved equal to
the same three angles;
and things which are equal to the same thing are also
equal to one another; \using{\rcn{1}}
therefore the angles $CBE$, $EBD$ are also equal to the
angles $DBA$, $ABC$.

But the angles $CBE$, $EBD$ are two right angles;
therefore the angles $DBA$, $ABC$ are also equal to two
right angles.

Therefore etc.
\end{proof}

\begin{annotations}
17. \textbf{let the angle $EBD$ be added to each}, literally ``let the
angle $EBD$ be added (so as to be) common,'' \greek{κοινὴ προσκείσθω ἡ
  ὑπὸ ΕΒΔ}.  Similarly \greek{κοινὴ ἀφῃρήσθω} is used of subtracting a
straight line or angle from each of two others. ``Let the common angle
$EBD$ be added'' is clearly an inaccurate translation, for the angle
is not common before it is added, i.e.\ the \greek{κοινὴ} is
proleptic.  ``Let the common angle be \emph{subtracted}'' as a
translation of \greek{κοινὴ ἀφῃρήσθω} would be less unsatisfactory, it
is true, but, as it is desirable to use corresponding words when
translating the two expressions, it seems hopeless to attempt to keep
the word ``common,'' and I have therefore said ``to each'' and ``from
each'' simply.

\end{annotations}

\end{proposition}

\begin{proposition}
\label{prop:I_14}

\begin{statement}
If with any straight line, and at a point on it, two straight lines
not lying on the same side make the adjacent angles equal to two right
angles, the two straight lines will be in a straight line with one
another.
\end{statement}

\begin{proof}
For with any straight line $AB$, and at the point~$B$ on it, let the two
straight lines $BC$, $BD$ not lying on the same side make the adjacent
angles $ABC$, $ABD$ equal to two right angles;

I say that $BD$ is in a straight line with~$CB$.

For, if $BD$ is not in a straight line
with~$BC$, let $BE$ be in a straight line
with~$CB$.

Then, since the straight line $AB$
stands on the straight line $CBE$,
the angles $ABC$, $ABE$ are equal to two right angles.
\using{\prop{1}{13}}

But the angles $ABC$, $ABD$ are also equal to two right angles;
therefore the angles $CBA$, $ABE$ are equal to the angles
$CBA$, $ABD$. \using{\rpost{4} and \rcn{1}}

Let the angle $CBA$ be subtracted from each; therefore the remaining
angle $ABE$ is equal to the remaining angle $ABD$, \using{\rcn{3}}

the less to the greater: which is impossible.  Therefore $BE$ is not
in a straight line with~$CB$.  Similarly we can prove that neither is
any other straight line except~$BD$.

Therefore $CB$ is in a straight line with~$BD$.

Therefore etc.
\end{proof}

\begin{annotations}

1. \textbf{If with any straight line….}  There is no greater
difficulty in translating the works of the Greek geometers than that
of accurately giving the force of prepositions, \greek{πρός}, for
instance, is used in all sorts of expressions with various shades of
meaning.  The present enunciation begins \greek{Ἐὰν πρός τινι εὐθείᾳ
  καὶ τῷ πρὸς αὐτῇ συμείῳ}, and it is really necessary in this one
sentence to translate \greek{πρός} by three different words,
\emph{with}, \emph{at}, and \emph{on}.  The first \greek{πρός} must be
translated by \emph{with} because two straight lines ``make'' an angle
with one another.  On the other hand, where the similar expression
\greek{πρὸς τῇ δοθείσῃ εὐθείᾳ} occurs in \prop{1}{23}, but it is a
question of ``constructing'' an angle (\greek{συστήσασθαι}), we have
to say ``to construct \emph{on} a given straight line.''
\emph{Against} would perhaps be the English word coming nearest to
expressing all these meanings of \greek{πρὸς}, but it would be
intolerable as a translation.

17. Todhunter points out that for the inference in this line
Post.~\ref{post:4}, that all right angles are equal, is necessary as
well as the Common Notion that things which are equal to the same
thing (or rather, here, to \emph{equal things}) are equal.  A similar
remark applies to steps in the proofs of \prop{1}{15} and
\prop{1}{18}.

24.~\textbf{we can prove.}  The Greek expresses this by the future of
the verb, \greek{δειξομεν}, ``we shall prove,'' which however would
perhaps be misleading in English.

\end{annotations}

\begin{notes}

Proclus observes (p.~297) that two straight lines on the \emph{same}
side of another straight line and meeting it in one and the same point
may make with one and the same portion of the straight line terminated
at the point two angles which are together equal to two right angles,
in which case however the two straight lines would not be in a
straight line with one another.  And he quotes from Porphyry a
construction for two such straight lines in the particular case where
they form with the given straight line angles equal respectively to
half a right angle and one and a half right angles.  There is no
particular value in the construction, which will be gathered from the
annexed figure where $CE$, $CF$ are drawn at the prescribed
inclinations to~$CD$.

\end{notes}

\end{proposition}

\begin{proposition}
\label{prop:I_15}

\begin{statement}
If two straight lines cut one another, they make the vertical angles
equal to one another.
\end{statement}

\begin{proof}

For let the straight lines $AB$, $CD$ cut one another at the
point~$E$; I say that the angle $AEC$ is equal to the angle~$DEB$, and
the angle $CEB$ to the angle~$AED$.

For, since the straight line $AE$ stands on the straight line~$CD$,
making the angles $CEA$, $AED$, the angles $CEA$, $AED$ are equal to
two right angles. \using{\prop{1}{13}}

Again, since the straight line $DE$ stands on the straight line $AB$,
making the angles $AED$, $DEB$, the angles $AED$, $DEB$ are equal to
two right angles.  \using{\prop{1}{3}}

But the angles $CEA$, $AED$ were also proved equal to two right
angles;

therefore the angles $CEA$, $AED$ are equal to the
angles $AED$, $DEB$. \using{\rpost{4} and \rcn{1}}

Let the angle $AED$ be subtracted from each;
therefore the remaining angle $CEA$ is equal to the
remaining angle $BED$. \using{\rcn{3}}

Similarly it can be proved that the angles $CEB$, $DEA$
are also equal.

Therefore etc.
\end{proof}

\begin{porism*}
From this it is manifest that, if two straight lines cut one another,
they will make the angles at the point of section equal to four right
angles.
\end{porism*}

\begin{annotations}

1. \textbf{the vertical angles}.  The difference between
\emph{adjacent} angles (\greek{αἰ ἐφεξῆς γωνίαι}) and \emph{vertical}
angles (\greek{αἰ κατὰ κορυφὴν γωνίαι}) is thus explained by Proclus
(p.~398, 14—34).  The first term describes the angles made by two
straight lines when one only it divided by the other, i.e.\ when one
straight Line meets another at a point which is not either of its
extremities, but is not itself produced beyond the point of meeting.
When the first straight line is produced, so that the lines cross at
the point, they make two pairs of \emph{vertical} angles (which are
more clearly described as \emph{vertically opposite} angles), and
which are so called because their convergence is from opposite
directions to one point (the intersection of the lines) as vertex
(\greek{κορυφή}).

16. \textbf{at the point of section}, literally ``at the section,''
\greek{πρὸς τῇ τομῇ},

\end{annotations}

\begin{notes}

This theorem, according to Eudemus, was first discovered by Thales,
but found its scientific demonstration in Euclid (Proclus, p.~299,
3—6).

Proclus gives a converse theorem which may be stated thus.  \emph{If a
  straight line is met at one and the same point intermediate in its
  length by two other straight lines on different sides of it and such
  as to make the vertical angles equal, the latter straight lines are
  in a straight line with one another.}  The proof need not be given,
since it is almost self-evident, whether (1)~it is direct, by means of
\prop{1}{13}, \prop*{1}{14}, or (2)~indirect, by \emph{reductio ad
    absurdum} depending on~\prop{1}{15}.

The balance of \textsc{ms.}\ authority seems to be against the
genuineness of this \emph{Porism}, but Proclus and Psellus both have
it.  The word is not here used, as it is in the title of Euclid's lost
\emph{Porisms}, to signify a particular class of independent
propositions which Proclus describes as being in some sort
intermediate between theorems and problems (requiring us, not to bring
a thing into existence, but to \emph{find} something which we know to
exist).  \emph{Porism} has here (and wherever the term is used in the
\emph{Elements}) its second meaning; it is what we call a corollary,
i.e.\ an incidental result springing from the proof of a theorem or
the solution of a problem, a result not directly sought but appearing
as it were by chance without any additional labour, and constituting,
as Proclus says, a sort of windfall (\greek{ἔρμαιον}) and \emph{bonus}
(\greek{κέρδος}).  These Porisms appear in both the geometrical and
arithmetical Books of the \emph{Elements}, and may either result from
theorems or problems.  Here the Porism is geometrical, and springs out
of a theorem; \prop{7}{2} affords an instance of an arithmetical
Porism.  As an instance of a Porism to a problem Proclus cites ``that
which is found in the second Book'' (\greek{τὸ ἐν τῷ δευτέρῳ βιβλίῳ
  κείμενον}); but as to this see notes on \prop{2}{4}
and~\prop{4}{15}.

The present Porism, says Proclus, formed the basis of ``that
paradoxical theorem which proves that only the following three
(regular) polygons can fill up the whole space surrounding one point,
the equilateral triangle, the square, and the equilateral and
equiangular hexagon.''  We can in fact place round a point in this
manner six equilateral triangles, three regular hexagons, or four
squares. ``But only the angles of these regular figures, to the number
specified, can make up four right angles: a theorem due to the
Pythagoreans.''

Proclus further adds that it results from the Porism that, if any
number of straight lines intersect one another at one point, the sum
of all the angles so formed will still be equal to four right angles.
This is of course what is generally given in the text-boots as
Corollary~2.

\end{notes}

\end{proposition}

\begin{proposition}
\label{prop:I_16}

\begin{statement}
In any triangle, if one of the sides be produced, the exterior angle
is greater than either of the interior and opposite angles.
\end{statement}

\begin{proof}

Let $ABC$ be a triangle, and let one side of it $BC$ be produced
to~$D$;

I say that the exterior angle $ACD$ is greater than either of the
interior and opposite angles $CBA$, $BAC$.

Let $AC$ be bisected at~$E$ \using{\prop{1}{10}} and let $BE$ be
joined and produced in a straight line to~$F$;

let $EF$ be made equal to $BE$ \prop{1}{10} let $FC$ be joined
[\rpost{1}], and let $AC$ be drawn through to~$G$ [\rpost{2}].

Then, since $AE$ is equal to~$EC$, and $BE$ to~$EF$, the two sides
$AE$, $EB$ are equal to the two sides $CE$, $EF$ respectively; and the
angle $AEB$ is equal to the angle $FEC$, for they are vertical
angles. \using{\prop{1}{15}}

Therefore the base $AB$ is equal to the base~$FC$,
and the triangle $ABE$ is equal to the triangle~$CFE$,
and the remaining angles are equal to the remaining angles
respectively, namely those which the equal sides subtend; \using{\prop{1}{4}}
therefore the angle $BAE$ is equal to the angle~$ECF$.

But the angle $ECD$ is greater than the angle~$ECF$;
\using{\rcn{5}}
therefore the angle $ACD$ is greater than the angle~$BAE$.

Similarly also, if $BC$ be bisected, the angle $BCG$, that is, the
angle $ACD$ [\prop{1}{15}, can be proved greater than the angle $ABC$
  as well.

Therefore etc.
\end{proof}

\begin{annotations}

1. \textbf{the exterior angle}, literally ``the outside angle,''
\greek{ἡ ἐκτὸς γωνία}.

2. \textbf{the interior and opposite angles}, \greek{τῶν ἐντὸς καὶ
  ἀπεναντίον γωνιῶν}.

12. \textbf{let $AC$ be drawn through to~$G$.}  The word is
\greek{διήχθω}, a variation on the more usual \greek{ἐκβεβλήσθω},
``let it be \emph{produced}.''

21. \textbf{$CFE$}, in the text ``$FEC$.''

\end{annotations}

\begin{notes}

As is well known, this proposition is not universally true under the
Riemann hypothesis of a space endless in extent but not infinite in
size.  On this hypothesis a straight line is a ``closed series'' and
returns on itself; and two straight lines which have one point of
intersection have another point of intersection also, which bisects
the whole length of the straight line measured from the first point on
it to the same point again; thus the axiom of Euclidean geometry that
two straight lines do not enclose a space does not hold.  If $4\delta$
denotes the finite length of a straight line measured from any point
once round to the same point again, $2\delta$ is the distance between
the two intersections of two straight lines which meet.  Two points
$A$, $B$ do not determine one sole straight line unless the distance
between them is different from $2\delta$.  In order that there may
only be one perpendicular from a point~$C$ to a straight line $AB$,
$C$ must not be one of the two ``poles'' of the straight line.

Now, in order that the proof of the present proposition may be
universally valid, it is necessary that $CF$ should always fall within
the angle $ACD$ so that, the angle $ACF$ may be less than the angle
$ACD$.  But this will not always be so on the Riemann hypothesis.
For, (1)~if $BE$ is equal to~$\delta$, so that $BF$ is equal
to~$2\delta$, $F$ will be the second point in which $BE$ and $BD$
intersect; i.e.\ $F$ will lie on~$CD$, and the angle $ACF$ will be
equal to the angle $ACD$.  In this case the exterior angle $ACD$ will
be equal to the interior angle $BAC$.  (2)~If $BE$ is greater
than~$\delta$ and less than $2\delta$, so that eBF is greater than
$2\delta$ and less than $4\delta$, the angle $ACF$ will be
\emph{greater} than the angle $ACD$, and therefore the angle $ACD$
will be \emph{less} than the interior angle~$BAC$.  Thus, e.g., in the
particular case of a right-angled triangle, the angles other than the
right angle may be (1)~both acute, (2)~one acute and one obtuse, or
(3)~both obtuse according as the perpendicular sides are (1)~both less
than~$\delta$, (2)~one less and the other greater than~$\delta$,
(3)~both greater than~$\delta$.

Proclus tells us (p.~307, 1—12) that some combined this theorem with
the next in one enunciation thus: \emph{In any triangle, if one side
  be produced, the exterior angle of the triangle is greater than
  either of the interior and opposite angles, and any two of the
  interior angles are less than two right angles}, the combination
having been suggested by the similar enunciation of Euclid
\prop{1}{32}.  \emph{In any triangle, if one of the suits be produced,
  the exterior angle is equal to the two inierior and opposite angles,
  and the three interior angles of the triangle are equal to two right
  angles.}

The present proposition enables Proclus to prove what he did not
succeed in establishing conclusively in his note on \prop{1}{12},
namely that \emph{from one point there cannot be drawn to the same
  straight line three straight lines equal in length.}

For, if possible, let $AB$, $AC$, $AD$ be all equal, $B$, $C$, $D$
being in a straight line.

Then, since $AB$, $AC$ are equal, the angles $ABC$, $ACB$ are equal.

Similarly, since $AB$, $AD$ are equal, the angles $ABD$, $ADB$ are
equal.

Therefore the angle $ACB$ is equal to the angle $ADC$, i.e.\ the
exterior angle to the interior and opposite angle: which is
impossible.

Proclus next (p.~308, 14 sqq.)\ undertakes to prove by means of
\prop{1}{16} that, \emph{if a straight line falling on two straight
  lines make the exterior angle equal to the interior and opposite
  angle, the two straight lines will not form a triangle or meet}, for
in that case the same angle would be both greater and equal.

The proof is really equivalent to that of Eucl.\ \prop{1}{17}.  If
$BE$ falls on the two straight lines $AB$, $CD$ in such a way that the
angle $CDE$ is equal to the interior and opposite angle $ABD$, $AB$
and $CD$ cannot form a triangle or meet.  For, if they did, then (by
\prop{1}{16}) the angle $CDE$ would be greater than the angle $ABD$,
while by the hypothesis it is at the same time \emph{equal} to it.

Hence, says Proclus, in order that $BA$, $DC$ may form a triangle it
is necessary for them to approach one another in the sense of being
turned round one pair of corresponding extremities, e.g.\ $B$, $D$, so
that the other extremities $A$, $C$ come nearer.  This may be brought
about in one of three ways: (1)~$AB$ may remain fixed and $CD$ be
turned about~$D$ so that the angle $CDE$ increases; (2)~$CD$ may
remain fixed and $AB$ be turned about~$B$ so that the angle $ABD$
becomes smaller; (3)~both $AB$ and $CD$ may move so as to make the
angle $ABD$ smaller and the angle $CDE$ larger at the same time.  The
reason, then, of the straight lines $AB$, $CD$ coming to form a
triangle or to meet is (says Proclus) \emph{the movement of the
  straight lines}.

Though he does not mention it here, Proclus does in another passage
(p.~371, 2–10, quoted on p.~\pageref{207} above) hint at the
possibility that, while \prop{1}{16} may remain universally true,
either of the straight lines $BA$, $DC$ (or both together) may be
turned through any angle not greater than a certain finite angle and
yet may not meet (the Bolyai-Lobachewsky hypothesis).

\end{notes}

\end{proposition}

\begin{proposition}
\label{prop:I_17}

\begin{statement}
In any triangle two angles taken together in any manner are less than
two right angles.
\end{statement}

\begin{proof}

Let $ABC$ be a triangle;

I say that two angles of the triangle $ABC$ taken together in any
manner are less than two right angles.

For let $BC$ be produced to~$D$. \using{\rpost{2}}

Then, since the angle $ACD$ is an exterior angle of the triangle
$ABC$, it is greater than the interior and opposite angle $ABC$.

Let the angle $ACB$ be added to each; therefore the angles $ACD$,
$ACB$ are greater than the angles $ABC$, $BCA$.

\infig{propI_17}

But the angles $ACD$, $ACB$ are equal to two right angles.
\using{\prop{1}{13}}

Therefore the angles $ABC$, $BCA$ are less than two right angles.

Similarly we can prove that the angles $BAC$, $ACB$ are also less than
two right angles, and so are the angles $CAB$, $ABC$ as well.

Therefore etc.
\end{proof}

\begin{annotations}

1. \textbf{taken together in any manner}, \greek{πάντῃ
  μεταλαμβανόμεναι}, i.e.\ any pair added together.

\end{annotations}

\begin{notes}

As in his note on the previous proposition, Proclus tries to state the
\emph{cause} of the property.  He takes the case of two straight lines
forming right angles with a transversal and observes that it is the
convergence of the straight lines towards one another (\greek{σύνευσις
  τῶν εὐθειῶν}), the lessening of the two right angles, which produces
the triangle.  He will not have it that the fact of the exterior angle
being greater than the interior and opposite angle is the \emph{cause}
of the property, for the odd reason that ``it is not necessary that a
side should be produced, or that there should be any exterior angle
constructed…and how can what is not necessary be the cause of what is
necessary?''\ (p.~311, 17—21).

Agreeably to this view, Proclus then sets himself to prove the theorem
without producing a side of the triangle.

Let $ABC$ be a triangle.  Take any point~$D$ on $BC$, and join~$AD$.

Then the exterior angle $ADC$ of the triangle $ABD$ is greater than
the interior and opposite angle~$ABD$.

Similarly the exterior angle $ADB$ of the triangle $ADC$ is greater
than the interior and opposite angle $ACD$.

Therefore, by addition, the angles $ADB$, $ADC$ are together greater
than the angles $ABC$, $ACB$.

But the angles $ADB$, $ADC$ are equal to two right angles; therefore
the angles $ABC$, $ACB$ are less than two right angles.

Lastly, Proclus proves (what is obvious from this proposition) that
\emph{there cannot be more than one perpendicular to a straight line
  from a point without it}.  For, if this were possible, two of such
perpendiculars would form a triangle in which two angles would be
right angles: which is impossible, since any two angles of a triangle
are together less than two right angles.

\end{notes}

\end{proposition}

\begin{proposition}
\label{prop:I_18}

\begin{statement}
In any triangle the greater side subtends the greater angle.
\end{statement}

\begin{proof}

For let $ABC$ be a triangle having the side $AC$ greater
than~$AB$;

I say that the angle $ABC$ is also greater than the angle
$BCA$.

For, since $AC$ is greater than~$AB$, let $AD$ be made equal
to $AB$ [\prop{1}{3}], and let $BD$ be joined.

Then, since the angle $ADB$
is an exterior angle of the triangle~$BCD$,
it is greater than the interior
and opposite angle DCB. [\prop{1}{16}]

But the angle $ADB$ is equal
to the angle $ABD$,
since the side $AB$ is equal to~$AD$;
therefore the angle $ABD$ is also greater than the angle~$ACB$
therefore the angle $ABC$ is much greater than the angle~$ACB$.

Therefore etc.
\end{proof}

\begin{annotations}

In the enunciation of this proposition we have \greek{ὑποτείνειν}
(''subtend'') used with the simple accusative instead of the more
usual \greek{ὑπό} with accusative.  The latter construction is used in
the enunciation of \prop{1}{19} which otherwise only differs from that
of \prop{1}{18} in the order of the words.  The point to remember in
order to distinguish the two is that the \emph{datum} comes first and
the \emph{quaesitum} second, the \emph{datum} being in this
proposition the greater \emph{side} and in the next the greater
\emph{angle}.  Thus the enunciations are (\prop{1}{18}) \greek{παντὸς
  τριγώνου ἡ μείζων πλευρὰ τὴν μείζονα γωνίαν ὑποτείνει} and
(\prop{1}{19}) \greek{παντὸς τριγώνου ὑπὸ τὴν μείζονα γωνίαν ἡ μείζων
  πλευρὰ ὑποτείνει}.  In order to keep the proper order in English we
must use the passive of the verb in \prop{1}{19}.  Aristotle quotes
the result of \prop{1}{19}, using the exact wording, \greek{ὑπὸ γὰρ
  τὴν μείζω γωνίαν ὑποτείνει} (\emph{Meteorologica} \r3.~5, 376~a~12).

\end{annotations}

\begin{notes}

``In order to assist the student in remembering which of these two
  propositions [\prop{1}{18}, \prop*{1}{19}] is demonstrated directly
  and which indirectly, it may be observed that the order is similar
  to that in \prop{1}{5} and \prop{1}{6}'' (Todhunter).

An alternative proof of \prop{1}{18} given by Porphyry (see Proclus,
pp.~315, 11—316, 13) is interesting.  It starts by supposing a length
equal to $AB$ cut off from the other end of~$AC$; that is, $CD$ and
not~$AD$ is made equal to~$AB$.

Produce $AB$ to~$E$ so that $BE$ is equal to $AD$, and join~$EC$.

Then, since $AB$ is equal to $CD$, and $BE$ to $AD$, $AE$ is equal
to~$AC$,

Therefore the angle $AEC$ is equal to the angle $ACE$.

Now the angle $ABC$ is greater than the angle $AEC$,
\using{\prop{1}{16}} and therefore greater than the angle $ACE$.
Hence, \emph{a fortiori}, the angle $ABC$ is greater than the
angle~$ACB$.

\end{notes}

\end{proposition}

\begin{proposition}
\label{prop:I_19}

\begin{statement}
In any triangle the greater angle is subtended by the greater side.
\end{statement}

\begin{proof}

Let $ABC$ be a triangle having the angle $ABC$ greater than the
angle~$BCA$;

I say that the side $AC$ is also greater than the side~$AB$.

For, if not, $AC$ is either equal to~$AB$ or less.

Now $AC$ is not equal to~$AB$;
for then the angle $ABC$ would also have been
equal to the angle $ACB$; \using{\prop{1}{5}}
but it is not;
therefore $AC$ is not equal to~$AB$.

Neither is $AC$ less than~$AB$, for then the angle $ABC$ would also
have been less than the angle~$ACB$; \using{\prop{1}{18}} but it is
not; therefore $AC$ is not less than~$AB$.

And it was proved that it is not equal either.  Therefore $AC$ is
greater than~$AB$.

Therefore etc.
\end{proof}

\begin{notes}

This proposition, like \prop{1}{6}, can be proved by merely logical
deduction from \prop{1}{5} and \prop{1}{18} taken together, as pointed
out by De Morgan.  The general form of the argument used by De Morgan
is given in his \emph{Formal Logic} (1847), p.~25, thus:

``\emph{Hypothesis}.  Let there be any number of propositions or
assertions—three for instance, $X$, $Y$ and~$Z$—of which it is the
property that one or the other must be true, \emph{and one only}.  Let
there be three other propositions $P$, $Q$ and~$P$ of which it is also
the property that one, and one only, must be true.  Let it be a
connexion of those assertions that:

when $X$ is true, $P$ is true,

when $K$ is true, $Q$ is true,

when $Z$ is true, $R$ is true.

\emph{Consequence}: then it follows that,

when $P$ is true, $X$ is true,

when $Q$ is true, $Y$ is true,

when $R$ is true, $Z$ is true.''

To apply this to the case before us, let us denote the sides of the
triangle $ABC$ by $a$, $b$, $c$, and the angles opposite to these
sides by $A$, $B$, $C$ respectively, and suppose that $a$ is the base.

Then we have the three propositions,

when $b$ is equal to $c$, $B$ is equal to~$C$,

when $b$ is greater than~$c$, $B$ is greater than~$C$,

when $b$ is less than~$c$, $B$ is less than~$C$,

and it follows \emph{logically} that,

when $B$ is equal to $C$, $b$ is equal to~$c$,

when $B$ is greater than $C$, $b$ is greater than~$c$,

when $B$ is less than $C$, $b$ is less than~$c$.

\subsection*{Reductio ad absurdum by exhaustion}

Here, says Proclus (p.~318, 16–23), Euclid proves the impossibility
``by means of \emph{division}'' (\greek{ἐκ διαιρέσεως}).  This means
simply the separation of different hypotheses, each of which is
inconsistent with the truth of the theorem to be proved, and which
therefore must be successively shown to be impossible.  If a straight
line is not greater than a straight line, it must be either equal to
it or less; thus in a \emph{reductio ad absurdum} intended to prove
such a theorem as \prop{1}{19} it is necessary to dispose successively
of \emph{two} hypotheses inconsistent with the truth of the theorem.

\subsection*{Alternative (direct) proof}

Proclus gives a direct proof (pp.~319—321) which an-Nairīzī also has
and attributes to Heron.  It requires a lemma and is consequently open
to the slight objection of separating a theorem from its converse.
But the lemma and proof are worth giving.

\subsection*{Lemma}

\emph{If an angle of a triangle be bisected and the straight line
  bisecting it meet the base and divide it into unequal parts, the
  sides containing the angle will be unequal, and the greater will be
  that which meets the greater segment of the base, and the less that
  which meets the less.}

Let $AD$, the bisector of the angle~$A$ of the triangle $ABC$, meet
$BC$ in~$D$, making $CD$ greater than~$BD$.

I say that $AC$ is greater than~$AB$.

Produce $AD$ to~$E$ so that $DE$ is equal to $AD$.  And, since $DC$ is
greater than $BD$, cut off $DF$ equal to~$BD$.

Join $EF$ and produce it to~$G$.

Then, since the two sides $AD$, $DB$ are equal to the two sides $ED$,
$DF$, and the vertical angles at~$D$ are equal,

$AB$ is equal to~$EF$,
and the angle $DEF$ to the angle $BAD$,
i.e.\ to the angle $DAG$ (by hypothesis).

Therefore $AG$ is equal to~$EG$,
and therefore greater than $EF$, or~$AB$,
Hence, \emph{a fortiori}, $AC$ is greater than $AB$.

\subsection*{Proof of \prop{1}{19}}

Let $ABC$ be a triangle in which the angle $ABC$ is greater than the
angle $ACB$.

Bisect $BC$ at~$D$, join $AD$, and produce it to~$B$ so that $DE$ is
equal to~$AD$.  Join~$BE$.

Then the iwo sides $BD$, $DE$ are equal to the two sides $CD$, $DA$,
and the vertical angles at~$D$ are equal;
therefore $BE$ is equal to~$AC$,
and the angle $DBE$ to the angle at~$C$.

But the angle at~$C$ is less than the angle~$ABC$; therefore the angle
$DBE$ is less than the angle~$ABD$.

Hence, if $BF$ bisect the angle $ABE$, $BF$ meets $AE$ between $A$
and~$D$.  Therefore $EF$ is greater than~$FA$.

It follows, by the lemma, that $BE$ is greater than~$BA$, that is,
$AC$ is greater than~$AB$.

\end{notes}

\end{proposition}

\begin{proposition}
\label{prop:I_20}

\begin{statement}
In any triangle two sides taken together in any manner are greater
than the remaining one.
\end{statement}

\begin{proof}

For let $ABC$ be a triangle;
I say that in the triangle $ABC$ two sides taken together in
any manner are greater than the remaining one, namely

$BA$, $AC$ greater than $BC$,

$AB$, $BC$ greater than $AC$,

$BC$, $CA$ greater than $AB$.

For let $BA$ be drawn through to the point~$D$ let $DA$ be made equal
to~$CA$, and let $DC$ be joined.

Then, since $DA$ is equal to~$AC$,
the angle $ADC$ is also equal to the angle $ACD$; \using{\prop{1}{5}}
therefore the angle $BCD$ is greater than
the angle $ADC$. \using{\rcn{5}}

And, since $DCB$ is a triangle having the angle $BCD$ greater than the
angle~$BDC$,
and the greater angle is subtended by the greater side,
\using{\prop{1}{19}}
therefore $DB$ is greater than~$BC$.

But $DA$ is equal to~$AC$;
therefore $BA$, $AC$ are greater than~$BC$.

Similarly we can prove that $AB$, $BC$ are also greater
than~$CA$, and $BC$, $CA$ than~$AB$.

Therefore etc.
\end{proof}

\begin{notes}

It was the habit of the Epicureans, says Proclus (p.~322), to ridicule
this theorem as being evident even to an ass and requiring no proof,
and their allegation that the theorem was ``known'' (\greek{γνώριμον})
even to an ass was based on the fact that, if fodder is placed at one
angular point and the ass at another, he does not, in order to get to
his food, traverse the two sides of the triangle but only the one side
separating them (an argument which makes Savile exclaim that its
authors were ``digni ipsi, qui cum Asino foenum essent,'' p.~78).
Proclus replies truly that a mere perception of the truth of the
theorem is a different thing from a scientific proof of it and a
knowledge of the reason why it is true.  Moreover, as Simson says, the
number of axioms should not be increased without necessity.

\subsection*{Alternative Proofs}

Heron and Porphyry, we are told (Proclus, pp.~323—6), proved this
theorem in different ways as follows, without producing one of the
sides.

\emph{First proof.}

Let $ABC$ be the triangle, and let it be required to prove that the
sides $BA$, $AC$ are greater than~$BC$.

Bisect the angle $BAC$ by $AD$ meeting $BC$ in~$D$.

Then, in the triangle $ABD$,
the exterior angle $ADC$ is greater than the
interior and opposite angle~$BAD$, \using{\prop{1}{16}}
that is, greater than the angle $DAC$.

Therefore the side $AC$ is greater than the side~$CD$,
\using{\prop{1}{19}}

Similarly we can prove that $AB$ is greater than~$BD$.

Hence, by addition, $BA$, $AC$ are greater than~$BC$.

\emph{Second proof.}

This, like the first proof, is direct. There are several cases to be considered.

(1)~If the triangle is \emph{equilateral}, the truth of the
proposition is obvious.

(2)~If the triangle is \emph{isosceles}, the proposition needs no
proof in the case~(\emph{a}) where each of the equal sides is greater
than the base.

(\emph{b})~If the base is greater than either of the other sides, we
have to prove that the sum of the two equal sides is greater than the
base.  Let $BC$ be the base in such a triangle.

Cut off from $BC$ a length~$BD$ equal to~$AB$, and join~$AD$.

Then, in the triangle $ADB$, the exterior angle
$ADC$ is greater than the interior and opposite angle
$BAD$. \using{\prop{1}{16}}

Similarly, in the triangle $ADC$, the exterior angle $ADB$ is greater
than the interior and opposite angle $CAD$.

By addition, the two angles $BDA$, $ADC$ are together greater than the
two angles $BAD$, $DAC$ (or the whole angle $BAC$).

Subtracting the equal angles $BDA$, $BAD$, we have the angle $ADC$
greater than the angle $CAD$.

It follows that $AC$ is greater than $CD$; \using{\prop{1}{19}} and,
adding the equals $AB$, $BD$ respectively, we have $BA$, $AC$ together
greater than~$BC$.

(3)~If the triangle be \emph{scalene}, we can arrange the sides in
order of length.  Suppose $BC$ is the greatest, $AB$ the intermediate
and $AC$ the least side.  Then it is obvious that $AB$, $BC$ are
together greater than $AC$, and $BC$, $CA$ together greater than $AB$.

It only remains therefore to prove that $CA$, $AB$ are together
greater than~$BC$.

We cut off from $BC$ a length $BD$ equal to the adjacent side, join
$AD$, and proceed exactly as in the above case of the isosceles
triangle.

\emph{Third proof.}

This proof is by \emph{reductio ad absurdum}.

Suppose that $BC$ is the greatest side and, as before, we have to
prove that $BA$, $AC$ are greater than~$BC$.

If they are not, they must be either equal to~$A$ or less than~$BC$.

(1)~Suppose $BA$, $AC$ are together equal to~$BC$.

From $BC$ cut off $BD$ equal to~$BA$, and join~$AD$.

It follows from the hypothesis that $DC$ is equal to~$AC$.

Then, since $BA$ is equal to~$BD$, the angle $BDA$ is equal to the
angle~$BAD$.

Similarly, since $AC$ is equal to~$CD$, the angle~$CDA$ is equal to
the angle~$CAD$.

By addition, the angles $BDA$, $ADC$ are together equal to the whole
angle~$BAC$.

That is, the angle $BAC$ is equal to two right angles: which is impossible.

(2)~Suppose $BA$, $AC$ are together less than~$BC$.

From $BC$ cut off $BD$ equal to~$BA$, and from $CB$ cut off~$CE$ equal
to~$CA$.  Join $AD$,~$AE$.

In this case, we prove in the same way that the angle $BDA$ is equal
to the angle $BAD$, and the angle $CEA$ to the angle~$CAE$.

By addition, the sum of the angles~$BDA$, $AEC$ is equal to the sum of
the angles $BAD$, $CAE$.

Now, by \prop{1}{16}, the angle $BDA$ is greater than the angle $DAC$,
and therefore, \emph{a fortiori}, greater than the angle~$EAC$.

Similarly the angle $AEC$ is greater than the angle~$BAD$.

Hence the sum of the angles $BDA$, $AEC$ is greater than the sum of
the angles $BAD$, $EAC$.

But the former sum was also equal to the latter: which is impossible,

\end{notes}

\end{proposition}

\begin{proposition}
\label{prop:I_21}

\begin{statement}
If on one of the sides of a triangle, from its extremities, there be
constructed two straight lines meeting within the triangle, the
straight lines so constructed will be less than the remaining two
sides of the triangle, but will contain a greater angle.
\end{statement}

\begin{proof}

On $BC$, one of the sides of the triangle $ABC$, from its extremities
$B$, $C$, let the two straight lines $BD$, $DC$ be constructed meeting
within the triangle;

I say that $BD$, $DC$ are less than the remaining two sides of the
triangle $BA$, $AC$, but contain an angle $BDC$ greater than the
angle~$BAC$.

For let $BD$ be drawn through to~$E$.

Then, since in any triangle two
sides are greater than the remaining
one, \using{\prop{1}{20}}
therefore, in the triangle $ABE$, the
two sides $AB$, $AE$ are greater than~$BE$.

Let $EC$ be added to each;
therefore $BA$, $AC$ are greater than $BE$, $EC$.

Again, since, in the triangle $CED$,
the two sides $CE$, $ED$ are greater than~$CD$,
let $DB$ be added to each;
therefore $CE$, $EB$ are greater than $CD$, $DB$.

But $BA$, $AC$ were proved greater than $BE$, $EC$;
therefore $BA$, $AC$ are much greater than $BD$, $DC$.

Again, since in any triangle the exterior angle is greater
than the interior and opposite angle, \using{\prop{1}{16}}
therefore, in the triangle $CDE$,
the exterior angle BDC is greater than the angle CED.

For the same reason, moreover, in the triangle $ABE$ also,
the exterior angle $CEB$ is greater than the angle~$BAC$.

But the angle $BDC$ was proved greater than the angle $CEB$; therefore
the angle $BDC$ is much greater than the angle~$BAC$.

Therefore etc.
\end{proof}

\begin{annotations}

2. \textbf{be constructed…meeting within the triangle.}  The word
``meeting'' is not in the Greek, where the words are \greek{ἐντὸς
  συσταθῶσιν}.  \greek{συνίστασθαι} is the word used of constructing
two straight lines \emph{to a point} (cf.\ \prop{1}{7}) or so as to
form a triangle; but it is necessary in English to indicate that they
\emph{meet}.

3. \textbf{the straight lines so constructed.}  Observe the elegant
brevity of the Greek \greek{αἱ συσταθεῖσαι}.

\end{annotations}

\begin{notes}

The editors generally call attention to the fact that the lines drawn
within the triangle in this proposition must be drawn, as the
enunciation says, from the \emph{ends} of the side; otherwise it is
not necessary that their sum should be less than that of the remaining
sides of the triangle.  Proclus (p.~327, 12~sqq.)\ gives a simple
illustration.

Let $ABC$ be a right-angled triangle.  Take any point~$D$ on~$BC$,
join $DA$, and cut off from it $DE$ equal to~$AB$.  Bisect $AE$
at~$F$, and join~$FC$.

Then shall $CF$, $FD$ be together greater than $CA$, $AB$.
For $CF$, $FE$ are equal to $CF$, $FA$,
and therefore greater than~$CA$.

Add the equals $ED$, $AB$ respectively;
therefore $CF$, $FD$ are together greater than $CA$, $AB$.

Pappus gives the same proposition as that just proved, but follows it
up by a number of others more elaborate in character, selected
apparently from ``the so-called paradoxes'' of one Erycinus (Pappus,
\r3.\ p.~106 sqq.). Thus he proves the following:

1.~In any triangle, except an equilateral triangle or an isosceles triangle
with base less than one of the other sides, it is possible to construct on the
base and within the triangle two straight lines the sum of which is equal to
the sum of the other two sides of the triangle.

2.~In any triangle in which it is possible to construct two straight
lines on the base which are equal to the sum of the other two sides of
the triangle it is also possible to construct two others the sum of
which is \emph{greater} than that sum.

3.~Under the same conditions, if the base is greater than either of
the other two sides, two straight lines can be constructed in the
manner described which are \emph{respectively} greater than the other
two sides of the triangle; and the lines may be constructed so as to
be respectively equal to the two sides, if one of those two sides is
less than the other and each of them less than the base.

4.~The lines may be so constructed that their sum will bear to the sum
of the two sides of the triangle any ratio less than~2:1.

As a specimen of the proofs we will give that of the proposition which
has been numbered~(1) for the case where the triangle is isosceles
(Pappus, \r3.\ pp.~108–110).

Let $ABC$ be an isosceles triangle in which the base $AC$ is greater than
either of the equal sides $AB$, $BC$.

With centre~$A$ and radius $AB$ describe a circle meeting $AC$ in~$D$.

Draw any radius $AEF$ such that it meets $BC$ in a point~$F$ outside
the circle.

Take any point~$G$ on $EF$, and through it draw $GH$ parallel to~$AC$.
Take any point~$K$ on $GH$, and draw $KL$ parallel to $FA$ meeting
$AC$ in~$L$.

From $BC$ cut off $BN$ equal to~$EG$.

Thus $AG$, or~$LK$, is equal to the sum of $AB$, $BN$, and $CN$ is
less than~$LK$.

Now $GF$, $FH$ are together greater than~$GH$,
and $CH$, $HK$ together greater than~$CK$.

Therefore, by addition, $CF$, $FG$, $HK$ are together greater than
$CK$, $HG$.

Subtracting $HK$ from each side, we see that
$CF$, $FG$ are together greater than $CK$, $KG$;
therefore, if we add $AG$ to each,
$AF$, $FC$ are together greater than $AG$, $GK$, $KC$.

And $AB$, $BC$ are together greater than $AF$, $EC$. \using{\prop{1}{21}}

Therefore $AB$, $BC$ are together greater than $AG$, $GK$, $KC$.

But, by construction, $AB$, $BN$ are together equal to $AG$;
therefore, by subtraction, $NC$ is greater than $GK$, $KC$,
and \emph{a fortiori} greater than~$KC$.

Take on $KC$ produced a point~$M$ such that $KM$ is equal to~$NC$;
with centre~$K$ and radius $KM$ describe a circle meeting $CL$ in~$O$,
and join~$KO$.

Then shall $LK$, $KO$ be equal to $AB$, $BC$.

For, by construction, $LK$ is equal to the sum of $AB$, $BN$, and $KO$
is equal to~$NC$; therefore $LK$, $KO$ are together equal to $AB$,
$BC$.

It is after \prop{1}{21} at that (as remarked by De Morgan) the
important proposition about the perpendicular and obliques drawn from
a point to a straight line of unlimited length is best introduced:

\emph{Of all straight lines that can be drawn to a given straight line
  of unlimited length from a given point without it:}

(\emph{a})~\emph{the perpendicular is the shortest;}

(\emph{b})~\emph{of the obliques, that is the greater the fool of
  which is further from the perpendicular;}

(\emph{c})~\emph{given one oblique, only one other can be found of the
  same length, namely that the foot of which is equally distant with
  the foot of the given one from the perpendicular, but on the other
  side of it.}

Let $A$ be the given point, $BC$ the given straight line; let $AD$ be
the perpendicular from~$A$ on~$BC$, and $AE$, $AF$ any two obliques of
which $AF$ makes the greater angle with~$AD$.

Produce $AD$ to $A'$, making $A'D$ equal to~$AD$, and join $A'E$,
$AF$.

Then the triangles $ADE$, $A'DE$ are equal in all respects; and so are
the triangles $ADF$, $A'DF$.

Now (1)~in the triangle $AEA'$ the two sides $AE$, $EA'$ are-greater
than $AA'$ [\prop{1}{20}], that is, twice $AE$ is greater than
twice~$AD$.

Therefore $AE$ is greater than~$AD$.

(2)~Since $AE$, $A'E$ are drawn to~$E$, a point within the triangle
$AFA'$, $AE$, $EA'$ are together greater than $AE$, $EA'$,
\using{\prop{1}{21}} or twice $AF$ is greater than twice $AE$.

Therefore AE is greater than AE.

(3)~Along $DB$ measure off $DG$ equal to~$DF$, and join~$AG$.

The triangles $AGD$, $AFD$ are then equal in all respects, so that the
angles $GAD$, $FAD$ are equal, and $AG$ is equal to~$AF$.

\end{notes}

\end{proposition}

\begin{proposition}
\label{prop:I_22}

\begin{statement}
Out of three straight lines, which are equal to three given straight
lines, to construct a triangle: thus it is necessary that two of the
straight lines taken together in any manner should be greater than the
remaining one. \using{\prop{1}{20}}
\end{statement}

\begin{proof}

Let the three given straight lines be $A$, $B$, $C$, and of these let
two taken together in any manner be greater than the remaining one,
namely $A$, $B$ greater than~$C$,
$A$, $C$ greater than~$B$,
and $B$, $C$ greater than~$A$;
thus it is required to construct a triangle out of straight lines
equal to $A$, $B$, $C$.

\infig{propI_22}

Let there be set out a straight line $DE$, terminated at~$D$
but of infinite length in the direction of~$B$,
and let $DF$ be made equal to $A$, $FG$ equal to $B$, and $GH$
equal to $C$. \using{\prop{1}{3}}

With centre~$F$ and distance $FD$ let the circle $DKL$ be described;\0
again, with centre~$G$ and distance $GH$ let the circle $KLH$ be
described;\0
and let $KF$, $KG$ be joined;

I say that the triangle $KFG$ has been constructed out of
three straight lines, equal to $A$, $B$, $C$.

For, since the point $F$ is the centre of the circle $DKL$, $FD$ is
equal to~$FK$.

But $FD$ is equal to~$A$;
therefore $KF$ is also equal to~$A$.

Again, since the point~$G$ is the centre of the circle $LKH$, $GH$ is
equal to~$GK$.

But $GH$ is equal to~$C$;
therefore $KG$ is also equal to~$C$.

And $FG$ is also equal to~$B$; therefore the three straight lines
$KF$, $FG$, $GK$ are equal to the three straight lines $A$, $B$,~$C$.

Therefore out of the three straight lines $KF$, $FG$, $GK$,
which are equal to the three given straight lines $A$, $B$, $C$, the
triangle $KFG$ has been constructed.
\qef
\end{proof}

\begin{annotations}

2–4. This is the first cast in the \emph{Elements} of a
\greek{διορισμός} to a problem in the sense of a statement of the
conditions or limits of the possibility of a solution.  The criterion
is of course supplied by the preceding proposition.

2. \textbf{thus it is necessary.}  This is usually translated
(e.g.\ by Williamson and Simson) ``\emph{But} it is necessary,'' which
is however inaccurate, since the Greek is not \greek{δεῖ δέ} but
\greek{δεῖ δή}.  The words are the same as those used to introduce the
\greek{διορισμός} in the other sense of the ``definition'' or
``particular statement'' of a construction to be effected.  Hence, as
in the latter case we say ``thus it is required'' (e.g.\ to bisect the
finite straight line $AB$, \prop{1}{10}), we should here translate
``\emph{thus} it is necessary.''

4.~To this enunciation all the \textsc{mss.}\ and Boethius add, after
the \greek{διορισμός}, the words ``because in any triangle two sides
taken together in any manner are greater than the remaining one.''
But this explanation has the appearance of a gloss, and it is omitted
hy Proclus and Campanus.  Moreover there is no corresponding addition
to the \greek{διορισμός} of \prop{6}{28}.

\end{annotations}

\begin{notes}

It was early observed that Euclid assumes, without giving any reason,
that the circles drawn as described will meet if the condition that
any two of the straight lines $A$, $Β$, $C$ are together greater than
the third be fulfilled.  Proclus (p.~331, 8~sqq.)\ argues the matter
by means of \emph{reductio ad absurdum}, but does not exhaust the
possible hypotheses inconsistent with the contention.  He says the
circles must do one of three things, (1) cut one another, (1)~touch
one another, (2)~stand apart (\greek{διεστάναι}) from one another.  He
then considers the hypotheses (\emph{a})~of their touching
\emph{externally}, (\emph{b})~of their being separated from one
another by a space.  He should have considered also the hypothesis
(\emph{c}) of one circle touching the other \emph{internally} or lying
entirely within the other without touching.  These three hypotheses
being successively disproved, it follows that the circles must meet
(this is the line taken by Camerer and Todhunter).

Simson says: ``Some authors blame Euclid because he does not
demonstrate that the two circles made use of in the construction of
this problem must cut one another: but this is very plain from the
determination he has given, namely, that any two of the straight lines
$DF$, $FG$, $GH$ must be greater than the third.  For who is so dull,
though only beginning to learn the Elements, as not to perceive that
the circle described from the centre~$F$, at the distance~$FD$, must
meet $FH$ betwixt $F$ and~$H$, because $FD$ is less than $FH$; and
that, for the like reason, the circle described from the centre~$G$ at
the distance $GH$ must meet $DG$ betwixt $D$ and~$G$; and that these
circles must meet one another, because $FD$ and $GH$ are together
greater than~$FG$.''

We have in fact only to satisfy ourselves that one of the circles,
e.g.\ that with centre~$G$, has at least one point of its
circumference outside the other circle and also at least one point of
its circumference inside the same circle; and this is best shown with
reference to the points in which the first circle cuts the straight
line $DE$. For (1)~$FH$, being equal to the sum of $B$ and~$C$, is
greater than $A$, i.e.\ than the radius of the circle with centre~$F$,
and therefore $H$ is outside that circle. (2)~If $GM$ be measured
along $GF$ equal to $GH$ or~$C$, then, since $GM$ is either
(\emph{a})~less or (\emph{b})~greater than $GF$, $M$ will fall either
(\emph{a})~between $G$ and~$F$ or (\emph{b}) beyond $F$ towards~$D$;
in the first case (\emph{a}) the sum of $FM$ and $C$ is equal to $FG$
and therefore less than the sum of $A$ and~$C$, so that $FM$ is less
than $A$ or $FD$; in the second case (\emph{b}) the sum of $MF$ and
$FG$, i.e.\ the sum of $MF$ and $B$, is equal to $GM$ or~$C$, and
therefore less than the sum of $A$ and~$B$, so that $MF$ is less than
$A$ or $FD$; hence in either case $M$ falls within the circle with
centre~$F$.

It being now proved that the circumference of the circle with
centre~$G$ has at least one point outside, and at least one point
inside, the circle with centre~$F$, we have only to invoke the
Principle of Continuity, as we have to do in \prop{1}{1} (cf.\ the
note on that proposition, p.~\pageref{242}, where the necessary
postulate is stated in the form suggested by Killing).

That the construction of the proposition gives only \emph{two} points
of intersection between the circles, and therefore only two triangles
satisfying the condition, one on each side of~$FG$, is clear from
\prop{1}{7}, which, as before pointed out, takes the place, in
Book~\r1.\ of \prop{3}{10} proving that two circles cannot intersect
in more points than two.

\end{notes}

\end{proposition}

\begin{proposition}
\label{prop:I_23}

\begin{statement}
On a given straight line and at a point on it to construct a
rectilineal angle equal to a given rectilineal angle.
\end{statement}

\begin{proof}
Let $AB$ be the given straight line, $A$ the point on it, and the
angle $DCE$ the given rectilineal angle;

thus it is required to construct on the given straight line $AB$, and
at the point~$A$ on it, a rectilineal angle equal to the given
rectilineal angle $DCE$.

\infig{propI_23}

On the straight lines $CD$, $CE$ respectively let the points D, E be
taken at random; let $D$, $E$ be joined, and out of three straight
lines which are equal to the three straight lines $CD$, $DE$, $CE$ let
the triangle $AFG$ he. constructed in such a way that $CD$ is equal to
$AF$, $CE$ to $AG$, and further $DE$ to $FG$. \using{\prop{1}{22}}

Then, since the two sides $DC$, $CE$ are equal to the two sides $FA$,
$AG$ respectively, and the base $DE$ is equal to the base~$FG$,
the angle $DCE$ is equal to the angle $FAG$. \using{\prop{1}{8}}

Therefore on the given straight line $AB$, and at the point~$A$ on it,
the rectilineal angle $FAG$ has been constructed equal to the given
rectilineal angle~$DCE$.
\qef
\end{proof}

\begin{notes}

This problem was, according to Eudemus (see Proclus, p.~333, 5),
``rather the discovery of Oenopides,'' from which we must apparently
infer, not that Oenopides was the first to find any solution of it,
but that it was he who discovered the particular solution given by
Euclid.  (Cf.\ Bretschneider, p.~65.)

The editors do not seem to have noticed the fact that the construction
of the triangle assumed in this proposition is not exactly the
construction given in \prop{1}{22}.  We have here to construct a
triangle on a certain finite straight line $AG$ as base; in
\prop{1}{11} we have only to construct a triangle with sides of given
length without any restriction as to how it is to be placed.  Thus in
\prop{1}{22} we set out any line whatever and measure successively
three lengths along it beginning from the given extremity, and what we
must regard as the base is the intermediate length, not the length
beginning at the given extremity, of the straight line arbitrarily set
out.  Here the base is a given straight line abutting at a given point
Thus the construction has to be modified somewhat from
%
\infig{propI_23a}
%
that of the preceding proposition.  We must measure $AG$ along $AB$ so
that $AG$ is equal to~$CE$ (or~$CD$), and $GH$ along $GB$ equal
to~$DE$; and then we must produce $BA$, in the opposite direction,
to~$F$, so that $AF$ is equal to $CD$ (or $CE$, if $AG$ has been made
equal to $CD$).

Then, by drawing circles (1)~with centre~$A$ and radius~$AF$, (2)~with
centre~$G$ and radius $GH$, we determine~$K$, one of their points of
intersection, and we prove that the triangle $KAG$ is equal in all
respects to the triangle $DCE$, and then that the angle at~$A$ is
equal to the angle~$DCE$.

I think that Proclus must (though he does not say so) have felt the
same difficulty with regard to the use in \prop{1}{23} of the result
of \prop{1}{22}, and that this is probably the reason why he gives
over again the construction which I have given above, with the remark
(p.~334, 6) that ``you may obtain the construction of the triangle in
a more instructive manner (\greek{διδασκαλικώτερον}) as follows.''

Proclus objects to the procedure of Apollonius in constructing an
angle under the same conditions, and certainly, if he quotes
Apollonius correctly, the latter's exposition must have been somewhat
slipshod.

``He takes an angle $CDE$ at random,'' says Proclus (p.~335,
19~sqq.f), ``and a straight line $AB$, and with centre~$D$ and
distance $CD$ describes the circumference $CE$, and in the same way
with centre~$A$ and distance $AB$ the circumference $FB$.  Then,
cutting off $FB$ equal to $CE$, he joins $AF$.  And he declares that
the angles $A$, $D$ standing on equal circumferences are equal.''

In the first place, as Proclus remarks, it should be premised that
$AB$ is equal to $CD$ in order that the circles may be equal; and the
use of Book~\r3.\ for such an elementary construction is
objectionable.  The omission to state that $AB$ must be taken equal to
$CD$ was no doubt a slip, if it occurred.  And, as regards the equal
angles ``standing on equal circumferences,'' it would seem possible
that Apollonius said this in \emph{explanation}, for the sake of
brevity, rather than by way of proof.  It seems to me probable that
his construction was only given from the point of view of
\emph{practical}, not theoretical, geometry.  It really comes to the
same thing as Euclid's except that $DC$ is taken equal to~$DE$.  For
cutting off the arc~$BF$ equal to the arc~$CE$ can only be meant in
the sense of measuring the \emph{chord}~$CE$, say, with a pair of
compasses, and then drawing a circle with centre~$B$ and radius equal
to the chord~$CE$.  Apollonius' direction was therefore probably
intended as a practical short cut, avoiding the actual drawing of the
chords $CE$, $BF$, which, as well as a proof of the equality in all
respects of the triangles $CDE$, $BAF$, would be required to establish
\emph{theoretically} the correctness of the construction.

\end{notes}

\end{proposition}

\begin{proposition}
\label{prop:I_24}

\begin{statement}
If two triangles have the two sides equal to two sides respectively,
but have the one of the angles contained by the equal straight lines
greater than the other, they will also have the base greater than the
base.
\end{statement}

\begin{proof}

Let $ABC$, $DEF$ be two triangles having the two sides
$AB$, $AC$ equal to the two sides $DE$, $DF$ respectively, namely
$AB$ to~$DE$, and $AC$ to~$DF$, and let the angle at~$A$ be greater
than the angle at~$D$;

I say that the base $BC$ is also greater than the base~$EF$.

For, since the angle $BAC$
is greater than the angle $EDF$,
let there be constructed, on the
straight line $DE$, and at the
point~$D$ on it, the angle $EDG$
equal to the angle $BAC$; [\prop{1}{13}]
let $DG$ be made equal to either
of the two straight lines $AC$,
$DF$, and let $EG$, $FG$ be joined.

Then, since $AB$ is equal to $DE$, and $AC$ to $DG$, the two sides
$BA$, $AC$ are equal to the two sides $ED$, $DG$, respectively; and
the angle $BAC$ is equal to the angle $EDG$; therefore the base $BC$
is equal to the base $EG$. \using{\prop{1}{4}}

Again, since $DF$ is equal to~$DG$, the angle $DGF$ is also equal to
the angle $DFG$; \using{\prop{1}{5}}
therefore the angle $DFG$ is greater than the angle~$EGF$.

Therefore the angle $EFG$ is much greater than the angle~$EGF$.

And, since $EFG$ is a triangle having the angle $EFG$ greater than the
angle $EGF$,
and the greater angle is subtended by the greater side,
\using{\prop{1}{19}}
the side $EG$ is also greater than~$EF$.

But $EG$ is equal to~$BC$.

Therefore $BC$ is also greater than~$EF$.

Therefore etc.
\end{proof}

\begin{notes}

10.~I have naturally left out the well-known words added by Simson in
order to avoid the necessity of considering three cases: ``Of the two
sides $DE$, $DF$ let $DE$ be the side which is not greater than the
other.''  I doubt whether Euclid could have been induced to insert the
words himself, even if it had been represented to him that their
omission meant leaving two possible cases out of consideration.  His
habit and that of the great Greek geometers was, not to set out all
possible cases, but to give as a rule one case, generally the most
difficult, as here, and to leave the others to the reader to work out
for himself. We have already seen one instance in \prop{1}{7}.

Proclus of course gives the other two cases which arise if we do not
first provide that $DE$ is not greater than~$DF$.

(1)~In the first case~$G$ may fall on~$EF$ produced, and it is then
obvious that $EG$ is greater than~$EF$.

(2)~In the second case $EG$ may fall below~$EF$.

If so, by \prop{1}{21}, $DF$, $FE$ are together less than $DG$, $GE$.

But $DF$ is equal to $DG$; therefore $EF$ is less than~$EG$,
i.e.\ than~$BC$.

These two cases are therefore decidedly simpler than the case taken by
Euclid as typical, and could well be left to the ingenuity of the
learner.

If however after all we prefer to insert Simson's words and avoid the
latter two cases, the proof is not complete unless we show that, with
his assumption, $F$ must, in the figure of the proposition, fall
\emph{below}~$EG$.

De Morgan would make the following proposition precede: \emph{Every
  straight line drawn from the vertex of a triangle to the base is
  less than the greater of the two sides, or than either if they are
  equal}, and he would prove it by means of the proposition relating
to perpendicular and obliques given above, p.~\pageref{291}.

But it is easy to prove directly that $F$ falls below $EG$, if $DE$ is
not greater than $DG$, by the method employed by Pfleiderer, Lardner,
and Todhunter.

Let $DF$, produced if necessary, meet $EG$ in~$H$.

Then the angle $DHG$ is greater than the angle~$DEG$
\using{\prop{1}{16}}
and the angle $DEG$ is not less than the angle $DGE$;
\using{\prop{1}{18}}
therefore the angle $DHG$ is greater than the angle $DGH$.

Hence $DH$ is less than~$DG$, \using{\prop{1}{19}}
and therefore $DH$ is less than $DF$.

\subsection*{Alternative proof}

Lastly, the modern alternative proof is worth giving.

\infig{propI_24x}

Let $DH$ bisect the angle $FDG$ (after the triangle $DEG$ has been
made equal in all respects to the triangle $ABC$, as in the
proposition), and let $DH$ meet $EG$ in~$H$.  Join~$HF$.

Then, in the triangles $FDH$, $GDH$,
the two sides $FD$, $DH$ are equal to the two sides $GD$, $DH$,
and the included angles $FDH$, $GDH$ at equal;
therefore the base $HF$ is equal to the base $HG$.

Accordingly $EG$ is equal to the sum of $EH$, $HF$;
and $EH$, $HF$ are together greater than $EF$; \using{\prop{1}{20}}
therefore $EG$, or $BC$, is greater than~$EF$.

Proclus (p.~339, 11 sqq.)\ answers by anticipation the possible
question that might occur to any one on this proposition, viz.\ why
does Euclid not compare the areas of the triangles as he does in
\prop{1}{4}?  He observes that inequality of the areas does not follow
from the inequality of the angles contained by the equal sides, and
that Euclid leaves out all reference to the question both for this
reason and because the areas cannot be compared without the help of
the theory of parallels. ``But if,'' says Proclus, ``we must
anticipate what is to come and make our comparison of the areas at
once, we assert that (1)~\emph{if the angles $A$, $D$}—supposing that
our argument proceeds with reference to the figure in the
proposition—\emph{are (together) equal to two right angles, the
  triangles a« primed equal}, (2)~\emph{if greater than two right
  angles, that triangle which has the greater angle is less, and}
(3)~\emph{if they are less, greater}.''  Proclus then gives the proof,
but without any reference to the source from which he quoted the
proposition.  Now an-Nairīzī adds a similar proposition to
\prop{1}{38}, but definitely attributes it to Heron.  I shall
accordingly give it in the place where Heron put it.

\end{notes}

\end{proposition}

\begin{proposition}
\label{prop:I_25}

\begin{statement}
If two triangles have the two sides equal to two sides respectively,
but have the base greater than the base, they will also have the one
of the angles contained by the equal straight lines greater than the
other.
\end{statement}

\begin{proof}
Let $ABC$, $DEF$ be two triangles having the two sides $AB$, $AC$
equal to the two sides $DE$, $DF$ respectively, namely $AB$ to~$DE$,
and $AC$ to~$DF$; and let the base $BC$ be greater than the base~$EF$;

I say that the angle $BAC$ is also greater than the angle~$EDF$.

\infig{propI_25}

For, if not, it is either equal to it or less.

Now the angle $BAC$ is not equal to the angle $EDF$; for then the base
$BC$ would also have been equal to the base $EF$, but it is not;
therefore the angle $BAC$ is not equal to the angle $EDF$.

Neither again is the angle $BAC$ less than the angle $EDF$; for then
the base~$BC$ would also have been less than the base $EF$,
\using{\prop{1}{24}} but it is not; therefore the angle $BAC$ is not
less than the angle $EDF$.

But it was proved that it is not equal either; therefore the angle
$BAC$ is greater than the angle~$EDF$.

Therefore etc.
\end{proof}

\begin{notes}

De Morgan points out that this proposition (as also \prop{1}{8}) is a
purely \emph{logical} consequence of \prop{1}{4} and \prop{1}{24} in
the same way as \prop{1}{19} and \prop{1}{6} are purely \emph{logical}
consequences of \prop{1}{18} and \prop{1}{5}.  If $a$, $b$, $c$ denote
the sides, $A$, $B$, $C$ the angles opposite to them in a triangle
$ABC$, and $a'$, $b'$, $c'$, $A'$, $B'$, $C'$ the sides and opposite
angles respectively in a triangle $A'B'C$, \prop{1}{4} and
\prop{1}{24} tell us that, $b$, $c$ being respectively equal to $b'$,
$c'$.
\begin{enumerate}
\item if $A$ is equal to $A'$ then $a$ is equal to $a'$,
\item if $A$ is less than $A'$, then $a$ is less than $a'$,
\item if $A$ is greater than $A'$, then $a$ is greater than $a'$;
\end{enumerate}
and it follows \emph{logically} that,
\begin{enumerate}
\item if $a$ is equal   to   $a'$, the angle $A$ is equal to the angle
  $A'$,
\using{\prop{1}{8}}
\item if $a$ is less    than $a'$, $A$ is less than $A'$,
\item if $a$ is greater than $a'$, $A$ is greater than $A'$. \prop{1}{25}
\end{enumerate}

Two alternative proofs of this theorem are given by Proclus
(pp.~345—7), and they are both interesting.  Moreover both are
\emph{direct}.

\subsection*{I. Proof by Menelaus of Alexandria}

Let $ABC$, $DEF$ be two triangles having the two sides $BA$, $AC$
equal to the two sides $ED$, $DF$, but the base $BC$ greater than the
base~$EF$.

Then shall the angle at~$A$ be greater than the angle at~$D$.

From $BC$ cut off $BG$ equal to~$EF$.  At~$B$, on the straight line
$BC$, make the angle $GBH$ (on the side of $BG$ remote from~$A$) equal
to the angle~$FED$.

Make $BH$ equal to~$DE$; join $HG$, and produce it to meet $AC$
in~$K$.  Join~$AH$.

Then, since the two sides $GB$, $BH$ are equal to the two sides $FE$,
$ED$ respectively,
and the angles contained by them are equal,
$HG$ is equal to $DF$ or~$AC$,
and the angle $BHG$ is equal to the angle~$EDF$.

Now $HK$ is greater than $HG$ or~$AC$,
and \emph{a fortiori} greater than $AK$;
therefore the angle $KAH$  greater than the angle $KHA$.
And, since $AB$ is equal to $BH$,
the angle $BAH$ is equal to the angle $BHA$.

Therefore, by addition,
the whole angle $BAC$ is greater than the whole angle $BHG$,
that is, greater than the angle $EDF$.

\subsection*{II. Heron's proof}

Let the triangles be given as before.

Since $BC$ is greater than~$EF$, produce $EF$ to~$G$ so that $EG$ is
equal to~$BC$.

Produce $ED$ to~$H$ so that $DH$ is equal to~$DF$.  The circle with
centre~$D$ and radius $DF$ will then pass through~$H$.  Let it be
described, as~$FKH$.

Now, since $BA$, $AC$ are together greater than~$BC$,
and $BA$, $AC$ are equal to $ED$, $DH$ respectively,
while $BC$ is equal to~$EG$,
$EH$ is greater than~$EG$.

Therefore the circle with centre~$E$ and radius~$EG$ will cut~$EH$,
and therefore will cut the circle already drawn.  Let it cut that
circle in~$K$, and join $DK$, $KB$.

Then, since $D$ is the centre of the circle~$FKH$,
$DK$ is equal to $DF$ or~$AC$.

Similarly, since $E$ is the centre of the circle $KG$,
$EK$ is equal to $EG$ or~$BC$,

And $DE$ is equal to $AB$.

Therefore the two sides $BA$, $AC$ are equal to the two sides $ED$,
$DK$ respectively;
and the base $BC$ is equal to the base~$EK$;
therefore the angle $BAC$ is equal to the angle $EDK$

Therefore the angle $BAC$ is greater than the angle~$EDF$.

\end{notes}

\end{proposition}

\begin{proposition}
\label{prop:I_26}

\begin{statement}
If two triangles have the two angles equal to two angles respectively,
and one side equal to one side, namely, either the side adjoining the
equal angles, or that subtending one of the equal angles, they will
also have the remaining sides equal to the remaining sides and the
remaining angle to the remaining angle.
\end{statement}

\begin{proof}

Let $ABC$, $DEF$ be two triangles having the two angles
$ABC$, $BCA$ equal to the two angles $DEF$, $EFD$ respectively,
namely the angle $ABC$ to the angle $DEF$, and the angle
$BCA$ to the angle $EFD$; and let them also have one side
equal to one side, first that adjoining the equal angles, namely
$BC$ to~$EF$;

I say that they will also have the remaining sides equal
to the remaining sides respectively, namely $AB$ to $DE$ and
$AC$ to~$DF$, and the remaining angle to the remaining angle,
namely the angle $BAC$ to the angle~$EDF$.

For, if $AB$ is unequal to~$DE$, one of them is greater.

Let $AB$ be greater, and let $BG$ be made equal to~$DE$; and let $GC$
be joined.

Then, since $BG$ is equal to~$DE$, and $BC$ to~$EF$, the two sides
$GB$, $BC$ are equal to the two sides $DE$, $EF$ respectively; \0
and the angle $GBC$ is equal to the angle~$DEF$;
therefore the base $GC$ is equal to the base~$DF$,
and the triangle $GBC$ is equal to the triangle $DEF$,
and the remaining angles will be equal to the remaining angles,
namely those which the equal sides subtend; \using{\prop{1}{4}}
therefore the angle $GCB$ is equal to the angle~$DFE$.

But the angle $DFE$ is by hypothesis equal to the angle $BCA$;
therefore the angle $BCG$ is equal to the angle $BCA$,
the less to the greater: which is impossible.

Therefore $AB$ is not unequal to~$DE$,
and is therefore equal to it.

But $BC$ is also equal to~$EF$;
therefore the two sides $AB$, $BC$ are equal to the two
sides $DE$, $EF$ respectively,
and the angle $ABC$ is equal to the angle $DEF$;
therefore the base $AC$ is equal to the base~$DF$,
and the remaining angle $BAC$ is equal to the remaining
angle $EDF$. \using{\prop{1}{4}}

Again, let sides subtending equal angles be equal, as $AB$ to~$DE$;

I say again that the remaining sides will be equal to the remaining
sides, namely $AC$ to~$DF$ and $BC$ to~$EF$, and further the remaining
angle BAC is equal to the remaining angle $EDF$.

For, if $BC$ is unequal to~$EF$, one of them is greater.

Let $BC$ be greater, if possible, and let $BH$ be made equal to~$EF$;
let $AH$ be joined.

Then, since $BH$ is equal to~$EF$, and $AB$ to~$DE$, the two sides
$AB$, $BH$ are equal to the two sides $DE$, $EF$ respectively, and
they contain equal angles;
therefore the base $AH$ is equal to the base~$DF$,
and the triangle $ABH$ is equal to the triangle~$DEF$,
ss and the remaining angles will be equal to the remaining angles,
namely those which the equal sides subtend; \using{\prop{1}{4}}
therefore the angle $BHA$ is equal to the angle $EFD$.

But the angle $EFD$ is equal to the angle $BCA$;
therefore, in the triangle $AHC$, the exterior angle $BHA$ is
equal to the interior and opposite angle $BCA$:
which is impossible. \using{\prop{1}{16}}

Therefore $BC$ is not unequal to $EF$,
and is therefore equal to it.

But $AB$ is also equal to $DE$;
therefore the two sides $AB$, $BC$ are equal to the two sides
$DE$, $EF$ respectively, and they contain equal angles;
therefore the base $AC$ is equal to the base~$DF$,
the triangle $ABC$ equal to the triangle~$DEF$,
and the remaining angle $BAC$ equal to the remaining angle
$EDF$. \using{\prop{1}{4}}

Therefore etc.
\end{proof}

\begin{annotations}

2—3. \textbf{the side adjoining the equal angles}, \greek{πλευρα ν τὴν
  πρὸς ταῖς ἴσαις γωνίαις}.

29. \textbf{is by hypothesis equal}. \greek{ὑπόκειται ἴση}, according
to the elegant Greek idiom.  \greek{ὑπόκειμαι} is used for the passive
of \greek{ὑποτίθημι}, as \greek{κεῖμαι} is used for the passive of
\greek{τίθημι}, and to with the other compounds.
Cf.\ \greek{προσκεῖσθαι}, ``to be added.''

\end{annotations}

\begin{notes}

The alternative method of proving this proposition, viz.\ by applying
one triangle to the other, was very early discovered, at least so far
as regards the case where the equal sides are adjacent to the equal
angles in each.  An-Nairīzī gives it for this case, observing that the
proof is one which he had found, but of which he did not know the
author.

Proclus has the following interesting note (p.~352, 13—18): ``Eudemus
in his geometrical history refers this theorem to Thales.  For he says
that, in the method by which they say that Thales proved the distance
of ships in the sea, it was necessary to make use this theorem.''  As,
unfortunately, this information is not sufficient of itself to enable
us to determine how Thales solved this problem, there is considerable
room for conjecture as to his method.

The suggestions of Bretschneider and Cantor agree in the assumption
that the necessary observations were probably made from the top of
some tower or structure of known height, and that a right-angled
triangle was used in which the tower was the perpendicular, and the
line connecting the bottom of the tower and the ship was the base, as
in the annexed figure, where $AB$ is the tower and $C$ the ship.
Bretschneider (\emph{Die Geometrie and die Geometer vor Eukleides},
§30) says that it was only necessary for the observer to observe the
angle $CAB$, and then the triangle would be completely determined by
means of this angle and the known length $AB$.  As Bretschneider says
that the result would be obtained ``in a moment'' by this method, it
is not clear in what sense he supposes Thales to have ``observed'' the
angle $BAC$.  Cantor is more definite (\emph{Gesch.\ d.\ Math.}
\r1\tsub{3}, p.~145), for he says that the problem was nearly related
to that of finding the \emph{Seqt} from given sides.  By the
\emph{Seqt} in the Papyrus Rhind is meant the ratio to one another of
certain lines in pyramids or obelisks.  Eisenlohr and Cantor took the
one word to be equivalent, sometimes to the \emph{cosine} of the angle
made by the \emph{edge} of the pyramid with the co-terminous diagonal
of the base, sometimes to the \emph{tangent} of the angle of slope of
the \emph{faces} of the pyramid.  It is now certain that it meant one
thing, viz.\ the ratio of half the side of the base to the height of
the pyramid, i.e.\ the \emph{cotangent} of the angle of slope.  The
calculation of the \emph{Seqt} thus implying a sort of theory of
similarity, or even of trigonometry, the suggestion of Cantor is
apparently that the \emph{Seqt} in this case would be found from a
\emph{small} right-angled triangle $ADE$ having a common angle~$A$
with $ABC$ as shown in the figure, and that the ascertained value of
the \emph{Seqt} with the length $AB$ would determine $BC$.  This
amounts to the use of the property of similar triangles; and
Bretschneider's suggestion must apparently come to the same thing,
since, even if Thales measured the \emph{angle} in our sense (e.g.\ by
its ratio to a right angle), he would, in the absence of something
corresponding to a table of trigonometrical ratios, have gained
nothing and would have had to work out the proportions all the same.

Max C.~P. Schmidt also (\emph{Kulturhistorische Beiträge zur Kenntnis
  des griechischen und römischen Altertums}, 1906, p.~32) similarly
supposes Thales to have had a right angle made of wood or bronze with
the legs graduated, to have placed it in the position $ADE$ ($A$ being
the position of his eye), and then to have read off the lengths $AD$,
$DE$ respectively, and worked out the length of $BC$ by the rule of
three.

How then does the supposed use of similar triangles and their property
square with Eudemus' remark about \prop{1}{26}?  As it stands, it
asserts the \emph{equality} of \emph{two} triangles which have two
angles and one side respectively equal, and the theorem can only be
brought into relation with the above explanations by taking it as
asserting that, if two angles and \emph{one} side of one triangle are
given, the triangle is completely determined.  But, if Thales
practically used \emph{proportions}, as supposed, \prop{1}{26} is
surely not at all the theorem which this procedure would naturally
suggest as underlying it and being ``necessarily used''; the use of
proportions or of similar but not equal triangles would surely have
taken attention altogether away from \prop{1}{26} and fixed it on
\prop{4}{4}.

For this reason I think Tannery is on the right road when he tries to
find a solution using \prop{1}{26} as it stands, and withal as
primitive as any recorded solution of such a problem.  His suggestion
(\emph{La Géométrie grecque}, pp.~90—1) is based on the \emph{fluminis
  varatio} of the Roman agrimensor Marcus Junius Nipsus and is as
follows.

To find the distance from a point~$A$ to an inaccessible point~$B$.
From~$A$ measure along a straight line at right angles to $AB$ a
length $AC$ and bisect it at~$D$.  From $C$ draw $CE$ at right angles
to $CA$ on the side of it remote from~$B$, and let $E$ be the point on
it which is in a straight line with $B$ and~$D$.

Then, by \prop{1}{26}, $CE$ is obviously equal to~$AB$.

As regards the equality of angles, it is to be observed that those
at~$D$ are equal because they are vertically opposite, and, curiously
enough, Thales is expressly credited with the discovery of the
equality of such angles.

The only objection which I can see to Tannery's solution is that it
would require, in the case of the ship, a certain extent of free and
level ground for the construction and measurements.

I suggest therefore that the following may have been Thales' method.
Assuming that he was on the top of a tower, he had only to use a rough
instrument made of a straight stick and a cross-piece fastened to it
so as to be capable of turning about the fastening (say a nail) so
that it could form any angle with the stick and would remain where it
was put.  Then the natural thing would be to fix the stick upright (by
means of a plumb-line) and direct the cross-piece towards the ship.
Next, leaving the cross-piece at the angle so found, the stick could
be turned round, still remaining vertical, until the cross-piece
pointed to some visible object on the shore, when the object could be
mentally noted and the distance from the bottom of the tower to it
could be subsequently measured.  This would, by \prop{1}{36}, give the
distance from the bottom of the tower to the ship.  This solution has
the advantage of corresponding better to the simpler and more probable
version of Thales' method of measuring the height of the pyramids;
Diogenes Laertius says namely (\r1.~27, p.~6, ed.\ Cobet) on the
authority of Hieronymus of Rhodes (\bc~293—230), that he waited for
this purpose until the moment when \emph{our shadows are of the same
  length as ourselves}.

\subsection*{Recapitulation of congruence theorems}

Proclus, like other commentators, gives at this point (p.~347, 20
sqq.)\ a summary of the cases in which the equality of two triangles
in all respects can be established.  We may, he says, seek the
conditions of such equality by successively considering as hypotheses
the equality (1)~of sides alone, (2)~of angles alone, (3)!of sides and
angles combined.  Taking (1) first, we can only establish the equality
of the triangles in all respects if all three sides are respectively
equal; we cannot establish the equality of the triangles by any
hypothesis of class~(2), not even the hypothesis that all the three
angles are respectively equal; among the hypotheses of class~(3), the
equality of one side and one angle in each triangle is not enough, the
equality (\emph{a}) of one side and all three angles is more than
enough, as is also the equality (\emph{b}) of two sides and two or
three angles, and (\emph{c}) of three sides and one or two angles.

The only hypotheses therefore to be examined from this point of view
are the equality of

(\greek{α})~three sides [Eucl.\ \prop{1}{8}].

(\greek{β})~two sides and one angle [\prop{1}{4} proves one case of
  this, where the angle is that contained by the sides which are by
  hypothesis equal].

(\greek{γ})~one side and two angles [\prop{1}{26} covers all cases].

It is curious that Proclus makes no allusion to what we call the
\emph{ambiguous case}, that case namely of (\greek{β}) in which it is
an angle opposite to one of the two specified sides in one triangle
which is equal to the angle opposite to the equal side in the other
triangle.  Camerer indeed attributes to Proclus the observation that
in this case the equality of the triangles cannot, unless some other
condition is added, be asserted generally; but it would appear that
Camerer was probably misled by a figure (Proclus, p.~351) which looks
like a figure of the ambiguous case but is really only used to show
that in \prop{1}{26} the equal sides most be \emph{corresponding}
sides, i.e., they must be either adjacent to the equal angles in each
triangle, or opposite to corresponding equal angles, and that, e.g.,
one of the equal sides must not be adjacent to the two angles in one
triangle, while the side equal to it in the other triangle is opposite
to one of the two corresponding angles in that triangle.

\subsection*{The ambiguous case}

\emph{If two triangles have two sides equal to two sides respectively,
  and if the angles opposite to one pair of equal sides be also equal,
  then will the angles opposite the other pair of equal sides be
  either equal or supplementary; and, in the former ease, the
  triangles will be equal in all respects.}

Let $ABC$, $DEF$ be two triangles such that $AB$ is equal to~$DE$, and
$AC$ to~$DF$, while the angle $ABC$ is equal to the angle $DEF$ it is
required to prove that the angles $ACB$, $DFE$ are either equal or
supplementary.

Now (1), if the angle $BAC$ be equal to the angle $EDF$, it follows,
since the two sides $AB$, $AC$ are equal to the two sides $DE$, $DF$
respectively, that the triangles $ABC$, $DEF$ are equal in all
respects. \using{\prop{1}{4}}
and the angles $ACB$, $DFE$ are equal.

(2)~If the angles $BAC$, $EDF$ be not equal, make the angle $EDG$ (on
the same side of~$ED$ as the angle $EDF$) equal to the angle $BAC$.

Let $EF$, produced if necessary, meet $DG$ in~$G$.

Then, in the triangles $ABC$, $DEG$, the two angles $BAC$, $ABC$ are
equal to the two angles $EDG$, $DEG$ respectively, and the side $AB$
is equal to the side~$DE$; therefore the triangles $ABC$, $DEG$ are
equal in all respects, \using{\prop{1}{26}} so that the side $AC$ is
equal to the side $DG$, and the angle $ACB$ is equal to the angle
$DGE$.

Again, since $AC$ is equal to $DF$ as well as to~$DG$, $DF$ is equal
to $DG$, and therefore the angles $DFG$, $DGF$ are equal.

But the angle $DFE$ is supplementary to the angle $DFG$; and the angle
$DGF$ was proved equal to the angle $ACB$; therefore the angle $DFE$
is supplementary to the angle $ACB$.

If it is desired to avoid the ambiguity and secure that the triangles may
be congruent, we can introduce the necessary conditions into the enunciation,
on the analogy of Eucl.\ \prop{6}{7}.

\emph{If two triangles have two sides of the one equal to two sides of
  the other respectively, and the angles opposite to a pair of equal
  sides equal, then, if the angles opposite to the other pair of equal
  sides are both acute, or both obtuse, or if one of them is a right
  angle, the two triangles are equal in all respects.}

The proof of the three cases (by \emph{reductio ad absurdum}) was
given by Todhunter.

\end{notes}

\end{proposition}

\begin{proposition}
\label{prop:I_27}

\begin{statement}
If a straight line failing on two straight lines make the alternate
angles equal to one another, the straight lines will be parallel to
one another.
\end{statement}

\begin{proof}

For let the straight line $EF$ failing on the two straight lines $AB$,
$CD$ make the alternate angles $AEF$, $EFD$ equal to one another;

I say that $AB$ is parallel to~$CD$.

For, if not, $AB$, $CD$ when produced will meet either in the
direction of $B$, $D$ or towards $A$,~$C$.

Let them be produced and meet, in the direction of $B$, $D$, at~$G$.

Then, in the triangle $GEF$,
the exterior angle $AEF$ is equal to the interior and opposite
is angle $EFG$:
which is impossible. \using{\prop{1}{16}}

Therefore $AB$, $CD$ when produced will not meet in the direction of
$B$,~$D$.

Similarly it can be proved that neither will they meet towards
$A$,~$C$.

But straight lines which do not meet in either direction
are parallel; \using{\rdef{I_23}}
therefore $AB$ is parallel to~$CD$.

Therefore etc.
\end{proof}

\begin{annotations}

1. \textbf{filling on two straight lines}, \greek{εἰς δύο
  εὐθείασἐμτίπτουσα}, the phrase being the same as that used in
Post.~\ref{post:5}, meaning a \emph{transversal}.

2. \textbf{the alternate angles}, \greek{αἱ ἐναλλὰξ γωνίαι}. Proclus
(p.~357, 9) explains that Euclid uses the word \emph{alternate} (or,
more exactly, \emph{alternately}, \greek{ἐναλλα/ξ}) in two connexions,
(1)~of a certain transformation of a proportion, as in Book~\r5.\ and
the arithmetical Books, (2)~as here, of certain of the angles formed
by parallels with a straight line crossing them.  \emph{Alternate}
angles are, according to Euclid as interpreted by Proclus, those which
are not on the same side of the transversal, and are not adjacent, but
are separated by the transversal, both being within the parallels but
one ``above'' and the other ``below.''  The meaning is natural enough
if we imagine the four internal angles to be taken in cyclic order and
\emph{alternate} angles to be any two of them not successive but
separated by one angle of the four.

9. \textbf{in the direction of $B$, $D$ or towards $A$, $C$},
literally ``towards the \emph{parts} $B$, $D$ or towards $A$, $C$,''
\greek{ἐπὶ τὰ Β, Δ, μέρη ἢ ἐπὶ τὰ Δ, Γ}.

\end{annotations}

\begin{notes}

With this proposition begins the second section of the first Book.  Up
to this point Euclid has dealt mainly with triangles, their
construction and their properties in the sense of the relation of
their parts, the sides and angles, to one another, and the comparison
of different triangles in respect of their parts, and of their area in
the particular cases where they are congruent.

The second section leads up to the third, in which we pass to
relations between the areas of triangles, parallelograms and squares,
the special feature being a new conception of \emph{equality} of
areas, equality not dependent on \emph{congruence}.  This whole
subject requires the use of parallels.  Consequently the second
section beginning at \prop{1}{27} establishes the theory of parallels,
introduces the cognate matter of the equality of the sum of the angles
of a triangle to two right angles (\prop{1}{32}), and ends with two
propositions forming the transition to the third section, namely
\prop{1}{33}, \prop*{1}{34}, which introduce the parallelogram for the
first time,

\subsection*{Aristotle on parallels}

We have already seen reason to believe that Euclid's personal
contribution to the subject was nothing less than the formulation of
the famous Postulate~\ref{post:5} (see the notes on that Postulate and
on Def.~\ref{def:I_13}), since Aristotle indicates that the then
current theory of parallels contained a \emph{petitio principii}, and
presumably therefore it was Euclid who saw the defect and proposed the
remedy.

But it is clear that the propositions \prop{1}{27}, \prop*{1}{18} were
contained in earlier text-books. They were familiar to Aristotle, as
we may judge from two interesting passages.

(1)~In \emph{Anal.\ Post.}\ \r1.~5 he is explaining that a scientific
demonstration must not only prove a fact of every individual of a
class (\greek{κατὰ παντός}) but must prove it primarily and generally
true (\greek{πρῶτον καθόλου}) of the \emph{whole} of the class as one;
it will not do to prove it first of one part, then of another part,
and so on, until the class is exhausted.  He illustrates this
(74~a~13—16) by a reference to parallels: ``If then one were to show
that right (angles) do not meet, the proof of this might be thought to
depend on the fact that this is true of all (pairs of actual) right
angles.  But this is not so, inasmuch as the result does not follow
because (the two angles are) equal (to two right angles) \emph{in the
  particular way} [i.e.\ because each is a right angle], but by virtue
of their being equal (to two right angles) in any way whatever
[i.e.\ because the \emph{sum} only needs to be equal to two right
  angles, and the angles themselves may vary as much as we please
  subject to this].''

(2)~The second passage has already been quoted in the note on
Def.~\ref{def:I_23}: ``there is nothing surprising in different
hypotheses leading to the same false conclusion; e.g.\ the conclusion
that parallels meet might equally be drawn from either of the
assumptions (\emph{a})~that the interior (angle) is greater than the
exterior or (\emph{b})~that the sum of the angles of a triangle is
greater than two right angles'' (\emph{Anal. Prior.}\ \r2.~17,
66~a~11—15).

I do not quite concur in the interpretation which Heiberg places upon
these passages (\emph{Mathematisches zu Aristoteles}, pp.~18—19).  He
says, first, that the allusion to the ``interior angle'' being
``greater than the exterior'' in the second passage shows that the
reference in the first passage must be to Eucl.\ \prop{1}{28} and not
to \prop{1}{27}, and he therefore takes the words \greek{ὅτι ὡδὶ ἴσαι}
in the first passage (which I have translated ``because the two angles
are equal to two right angles in the particular way'') as meaning
``because the angles, viz.\ the exterior and the interior, are equal
in the particular way.''  He also takes \greek{αἱ ὀρθαὶ οὐ
  συμπίπτουσι} (which I have translated ``right angles do not meet,''
an expression quite in Aristotle's manner) to mean ``perpendicular
straight lines do not meet''; this is very awkward, especially as he
is obliged to supply \emph{angles} with \greek{ἴσαι} in the next
sentence.

But I think that the first passage certainly refers to \prop{1}{28},
although I do not think that the alternative (\emph{a})~in the second
passage suggests it.  This alternative may, I think, equally with the
alternative (\emph{b}) refer to \prop{1}{27}.  That proposition is
proved by \emph{reductio ad absurdum} based on the fact that, if the
straight lines do meet, they must form a \emph{triangle}, in which
case the exterior angle must be greater than the interior (while
according to the hypothesis these angles are equal).  It is true that
Aristotle speaks of the hypothesis that the \emph{interior} angle is
greater than the \emph{exterior}; but after all Aristotle had only to
state \emph{some} incorrect hypothesis.  It is of course only in
connexion with straight lines \emph{meeting}, as the hypothesis in
\prop{1}{27} makes them, that the alternative (\emph{b}) about the sum
of the angles of a triangle could come in, and alternative (\emph{a})
implies alternative~(\emph{b}).

It seems clear then from Aristotle that \prop{1}{27}, \prop*{1}{28} at
least are pre-Euclidean, and that it was only in \prop{1}{29} that
Euclid made a change by using his Postulate.

De Morgan observes that \prop{1}{27} is a \emph{logical} equivalent of
\prop{1}{16}.  Thus, if $A$ means ``straight lines forming a triangle
with a transversal,'' $B$ ``straight lines making angles with a
transversal on the same side which are together less than two right
angles,'' we have
\[
    \text{All $A$ is $B$,}
\]
and it follows \emph{logically} that
\[
    \text{All not-$B$ is not-$A$,}
\]

\end{notes}

\end{proposition}

\begin{proposition}
\label{prop:I_28}

\begin{statement}
If a straight line falling on two straight lines make the exterior
angle equal to the interior and opposite angle on the same side, or
the interior angles on the same side equal to two right angles, the
straight lines will be parallel to one another.
\end{statement}

\begin{proof}

For let the straight line $EF$ falling on the two straight lines $AB$,
$CD$ make the exterior angle $EGB$ equal to the interior and opposite
angle $GHD$, or the interior angles on the same side, namely $BGH$,
$GHD$, equal to two right angles;

I say that $AB$ is parallel to~$CD$.

For, since the angle $EGB$ is equal to the angle $GHD$, while the
angle $EGB$ is equal to the angle $AGH$, \using{\prop{1}{15}} the
angle $AGH$ is also equal to the angle $GHD$; and they are alternate;
therefore $AB$ is parallel to~$CD$. \using{\prop{1}{27}}

Again, since the angles $BGH$, $GHD$ are equal to two
right angles, and the angles $AGH$, $BGH$ are also equal to
two right angles, \using{\prop{1}{13}}
the angles $AGH$, $BGH$ are equal to the angles $BGH$, $GHD$.

Let the angle $BGH$ be subtracted from each; therefore the remaining
angle $AGH$ is equal to the remaining angle~$GHD$; and they are
alternate;
therefore $AB$ is parallel to $CD$. \using{\prop{1}{27}}

Therefore etc.
\end{proof}

\begin{notes}

One criterion of parallelism, the equality of alternate angles, is
given in \prop{1}{27}; here we have two more, each of which is easily
reducible, and is actually reduced, to the other.

Proclus observes (pp.~358—9) that Euclid could have stated six
criteria as well as three, by using, in addition, other pairs of
angles in the figure (not adjacent) of which it could be predicated
that the two angles are equal or that their sum is equal to two right
angles.  A natural division is to consider, first the pairs which are
on the same side of the transversal, and secondly the pairs which are
on different sides of it.

Taking (1)~the possible pairs on the \emph{same} side, we may have a
pair consisting of

(\emph{a})~two internal angles, viz.\ the pairs ($BGH$, $GHD$) and
($AGH$, $GHC$);

(\emph{b})~two external angles, viz.\ the pairs ($EGB$, $DHF$) and
($EGA$, $CHF$);

(\emph{c})~one external and one internal angle, viz.\ the pairs
($EGB$, $GHD$), ($FHD$, $HGB$), ($EGA$, $GHC$) and ($FHC$, $HGA$).

And (2)~the possible pairs on \emph{different} sides of the
transversal may consist respectively of

(\emph{a}) two internal angles, viz.\ the pairs ($AGH$, $GHD$) and
($CHG$, $HGB$);

(\emph{b}) two external angles, viz.\ the pairs ($AGE$, $DHF$) and
($EGB$, $CHF$);

(\emph{c}) one external and one internal, viz.\ the pairs ($AGE$,
$GHD$), ($EGB$, $GHC$), ($FHC$, $HGB$) and ($FHD$, $HGA$).

The angles are equal in the pairs (`)~(\emph{c}), (2)~(\emph{a}) and
(2)~(\emph{b}), and the sum is equal to two right angles in the case
of the pairs (1)~(\emph{a}), (1)(\emph{b}) and (2)~(\emph{c}).  For
his criteria Euclid selects the cases (2)~(\emph{a}) [\prop{1}{27}]
and (1)~(\emph{c}), (1)~(\emph{a}) [\prop{1}{28}], leaving out the
other three, which are of course equivalent but are not quite so
easily expressed.

From Proclus' note on \prop{1}{28} (p.~361) we iearn that one Aigeias
(?Aineias) of Hierapolis wrote an epitome or abridgment of the
\emph{Elements}.  This seems to be the only mention of this editor and
his work; and they are only mentioned as having combined
Eucl.\ \prop{1}{27}, \prop*{1}{28} into one proposition.  To do this,
or to make the three hypotheses the subject of three separate
theorems, would, Proclus thinks, have been more natural than to deal
with them, as Euclid does, in two propositions.  Proclus has no
suggestion for explaining Euclid's arrangement unless the ground were
that \prop{1}{27} deals with angles on different sides, \prop{1}{28}
with angles on one and the same side, of the transversal.  But may not
the reason have been one of convenience, namely that the criterion of
\prop{1}{27} is that actually used to prove parallelism, and is
moreover the basis of the construction of parallels in \prop{1}{31},
while \prop{1}{28} only reduces the other two hypotheses to that of
\prop{1}{27}, so that precision of reference, as well as clearness of
exposition, is better secured by the arrangement adopted?

\end{notes}

\end{proposition}

\begin{proposition}
\label{prop:I_29}

\begin{statement}
A straight line falling on parallel straight lines makes
the alternate angles equal to one another, the exterior angle
equal to the interior and opposite angle, and the interior angles
on the same side equal to two right angles.
\end{statement}

\begin{proof}

For let the straight line $EF$ fall on the parallel straight lines
$AB$, $CD$;

I say that it makes the alternate angles $AGH$, $GHD$
equal, the exterior angle $EGB$ equal to the interior and
opposite angle $GHD$, and the interior angles on the same
side, namely $BGH$, $GHD$, equal to two right angles.

For, if the angle $AGH$ is unequal
to the angle $GHD$, one of them is
greater.

Let the angle $AGH$ be greater.

Let the angle $BGH$ be added to
each;
therefore the angles $AGH$, $BGH$
are greater than the angles $BGH$,
$GHD$.

But the angles $AGH$, $BGH$ are equal to two right angles;
\using{\prop{1}{13}}
therefore the angles $BGH$, $GHD$ are less than two
right angles.

But straight lines produced indefinitely from angles less
than two right angles meet; \using{\rpost{5}}
therefore $AB$, $CD$, if produced indefinitely, will meet;
but they do not meet, because they are by hypothesis parallel.

Therefore the angle $AGH$ is not unequal to the angle $GHD$,
and is therefore equal to it.

Again, the angle $AGH$ is equal to the angle $EGB$;
\using{\prop{1}{15}}
 therefore the angle $EGB$ is also equal to the angle $GHD$.
\using{\rcn{1}}

Let the angle $BGH$ be added to each;
therefore the angles $EGB$, $BGH$ are equal to the
angles $BGH$, $GHD$. \using{\rcn{2}}

But the angles $EGB$, $BGH$ are equal to two right angles;
\using{\prop{1}{13}}
therefore the angles $BGH$, $GHD$ are also equal to
two right angles.

Therefore etc
\end{proof}

\begin{annotations}

23. \textbf{straight lines produced indefinitely from angles less than
  two right angles}, \greek{αἱ δὲ ἄπ ἐλασσόνων ἣ δύο ὁρθῶν
  ἐκβαλλόμεναι εἰς ἄπειρον συμπίπτουσιν}, a variation from the more
explicit language of Postulate~\ref{post:5}.  A good deal is left to
be understood, namely that the straight lines begin from points at
which they meet a transversal, and make with it internal angles on the
same aide the sum of which is less than two right angles.

26. \textbf{because they are by hypothesis parallel}, literally
``because they are supposed parallel,'' \greek{διὰ τὸ παραλλήλους
  αὐτὰς ὑποκεῖσθαι}.

\end{annotations}

\begin{notes}

\subsection*{Proof by ``Playfair's'' axiom}

If, instead of Postulate~\ref{post:5}, it is preferred to use
``Playfair's'' axiom in the proof of this proposition, we proceed
thus.

To prove that the alternate angles $AGH$, $GHD$ are equal.

If they are not equal, draw another straight line $KL$ through~$G$
making the angle $KGH$ equal to the angle $GHD$.

Then, since the angles $KGH$, $GHD$ are equal,
$KL$ is parallel to~$CD$. \using{\prop{1}{17}}

Therefore \emph{two straight lines $KL$, $AB$ intersecting at~$G$ are
  both parallel to the straight line $CD$: which is impossible} (by
the axiom).

Therefore the angle $AGH$ cannot but be equal to the angle $GHD$.

The rest of the proposition follows as in Euclid.

\subsection*{Proof of Euclid's Postulate~\ref{post:5} from ``Playfair's'' axiom}

Let $AB$, $CD$ make with the transversal $EF$ the angles $AEF$, $EFC$
together less than two right angles.

To prove that $AB$, $CD$ meet towards $A$,~$C$.

Through $E$ draw $GH$ making with $EF$ the angle $GEF$ equal (and
alternate) to the angle $EFD$.

Thus $GH$ is parallel to~$CD$, \using{\prop{1}{27}}

Then (1)~$AB$ must meet $CD$ in one direction or the other.

For, if it does not, $AB$ must be parallel to~$CD$; hence we have two
straight lines $AB$, $GH$ intersecting at~$E$ and both parallel
to~$CD$: which is impossible.

Therefore $AB$, $CD$ must meet.

(2)~Since $AB$, $CD$ meet, they must form a triangle with~$EF$.

But in any triangle any two angles are together less than two right
angles.

Therefore the angles $AEF$, $EFC$ (which are less than two right
angles), and not the angles $BEF$, $EFD$ (which are together greater
than two right angles, by \prop{1}{13}), are the angles of the
triangle; that is, $EA$, $FC$ meet in the direction of $A$, $C$, or on
the side of~$EF$ on which are the angles together less than two right
angles.

The usual course in modern text-books which use ``Playfair's'' axiom
in lieu of Euclid's Postulate is apparently to prove \prop{1}{9} by
means of the axiom, and then Euclid's Postulate by means of
\prop{1}{29}.

De Morgan would introduce the proof of Postulate~\ref{post:5} by means
of ``Playfair's'' axiom \emph{before} \prop{1}{29}, and would
therefore apparently prove \prop{1}{29} as Euclid does, without any
change.

As between Euclid's Postulate~\ref{post:5} and ``Playfair's'' axiom,
it would appear that the, tendency in modern text-books is rather in
favour of the latter.  Thus, to take a few noteworthy foreign writers,
we find that Rausenberger stands almost alone in using Euclid's
Postulate, while Hilbert, Henrici and Treutlein, Rouché and De
Comberousse, Enriques and Amaldi all use ``Playfair's'' axiom.

Yet the case for preferring Euclid's Postulate is argued with some
force by Dodgson (\emph{Euclid and his modern Rivals}, pp.~44–6).  He
maintains (1)~that ``Playfair's'' axiom in fact involves Euclid's
Postulate, but at the same time involves \emph{more} than the latter,
so that, to that extent, it is a needless strain on the faith of the
learner.  This is shown as follows.

Given $AB$, $CD$ making with $EF$ the angles $AEF$, $EFC$ together
less than two right angles, draw $GH$ through~$E$ so that the angles
$GEF$, $EFC$ are together equal to two right angles.

Then, by \prop{1}{28}, $GH$, $CD$ are ``separational.''

We see then that any lines which have the property~(\greek{α}) that
they make with a transversal angles less than two right angles have
also the property~(\greek{β}) that one of them intersects a straight
line which is ``separational'' from the other.

Now Playfair's axiom asserts that the lines which have
property~(\greek{β}) meet if produced: for, if they did not, we should
have two intersecting straight lines both ``separational'' from a
third, which is impossible.

We then argue that lines having property~(\greek{α}) meet because
lines having property~(\greek{α}) are lines having
property~(\greek{β}). But we do not know, until we have proved
\prop{1}{29}, that all pairs of lines having property~(\greek{β}) have
also property~(\greek{α}).  For anything we know to the contrary,
class~(\greek{β}) \emph{may} be greater than class~(\greek{α}).
Hence, if you assert anything of class~(\greek{β}), the logical effect
is more extensive than if you assert it of class~(\greek{α}); for you
assert it, not only of that portion of class~(\greek{β}) which is
known to be included in class~(\greek{α}), but also of the unknown
(but possibly existing) portion which is \emph{not} so included.

(2)~Euclid's Postulate puts before the beginner clear and
\emph{positive} conceptions, a pair of straight lines, a transversal,
and two angles together less than two right angles, whereas
``Playfair's'' axiom requires him to realise a pair of straight lines
which never meet though produced to infinity: a \emph{negative}
conception which does not convey to the mind any clear notion of the
relative position of the lines.  And (p.~68) Euclid's Postulate gives
a direct criterion for judging that two straight lines meet, a
criterion which is constantly required, e.g.\ in \prop{1}{44}.  It is
true that the Postulate can be \emph{deduced} from ``Playfair's''
axiom, but editors frequently omit to deduce it, and then tacitly
assume it afterwards: which is the least justifiable course of all.

\end{notes}

\end{proposition}

\begin{proposition}
\label{prop:I_30}

\begin{statement}
Straight lines parallel to the same straight line are also parallel to
one another.
\end{statement}

\begin{proof}

Let each of the straight lines $AB$, $CD$ be parallel to $EF$
I say that $AB$ is also parallel to $CD$.
For let the straight line $GK$ fall upon
them.

Then, since the straight line $GK$
has fallen on the parallel straight lines
$AB$, $EF$,
to the angle $AGK$ is equal to the
angle $GHF$. \using{\prop{1}{29}}

Again, since the straight line $GK$ has fallen on the parallel
straight lines $EF$, $CD$,
the angle $GHF$ is equal to the angle $GKD$. \using{\prop{1}{29}}

But the angle $AGK$ was also proved equal to the angle
$GHF$;
therefore the angle $AGK$ is also equal to the angle
$GKD$; \using{\rcn{1}}
and they are alternate.

Therefore $AB$ is parallel to~$CD$.
\end{proof}

\begin{annotations}

20. The usual \emph{conclusion} in general terms (``Therefore
etc.'')\ repeating the enunciation is, curiously enough, wanting at
the end of this proposition.

\end{annotations}

\begin{notes}

The proposition is, as De Morgan points out, the \emph{logical}
equivalent of ``Playfair's'' axiom.  Thus, $if$ X denote ``pairs of
straight lines intersecting one another,'' $Y$ ``pairs of straight
lines parallel to one and the same straight line,'' we have

No $X$ is $Y$,

and it follows logically that

No $Y$ is $X$.

De Morgan adds that a proposition is much wanted about parallels (or
perpendiculars) to two straight lines respectively making the same
angles with one another as the latter do.  The proposition may be
enunciated thus:

\emph{If the sides of one angle be respectively (1)~parallel or
  (2)~perpendicular to the sides of another angle, the two angles are
  either equal or supplementary.}

(1)~Let $DE$ be parallel to $AB$ and $GEF$ parallel to~$BC$.

To prove that the angles $ABC$, $DEG$ are equal and the angles $ABC$,
$DEF$ supplementary.

Produce $DE$ to meet $BC$ in~$H$.

Then [\prop{1}{29}] the angle $DEG$ is equal to the angle~$DHC$, and
the angle $ABC$ is equal to the angle~$DHC$.

Therefore the angle $DEG$ is equal to the angle~$ABC$; whence also the
angle $DEF$ is supplementary to the angle $ABC$.

(2)~Let $ED$ be perpendicular to $AB$, and $GEF$ perpendicular to~$BC$.

To prove that the angles $ABC$, $DEG$ are
equal, and the angles $ABC$, $DEF$ supplementary.

Draw $ED'$ at right angles to~$ED$ on the side of it opposite to~$B$,
and draw $EG'$ at right angles to $EF$ on the side of it opposite
to~$B$.

Then, since the angles $BDE$, $DED'$, being right angles, are equal,
$ED$ is parallel to~$BA$. \using{\prop{1}{27}}

Similarly $EG'$ is parallel to~$BC$.

Therefore [Part~(1)] the angle $D'EG'$ is equal to the angle $ABC$.

But, the right angle $DED'$ being equal to the right angle $GEG'$, if
the common angle $GED'$ be subtracted, the angle $DEG$ is equal to the
angle $D'EG'$.

Therefore the angle $DEG$ is equal to the angle $ABC$; and hence the
angle $DEF$ is supplementary to the angle~$ABC$.

\end{notes}

\end{proposition}

\begin{proposition}
\label{prop:I_31}

\begin{statement}
Through a given point to draw a straight line parallel to a given
straight line.
\end{statement}

\begin{proof}

Let $A$ be the given point, and $BC$ the given straight line; thus it
is required to draw through the point $A$ a straight line parallel to
the straight line $BC$.

Let a point~$D$ be taken at random on $BC$, and let $AD$ be joined; on
the straight line $DA$, and at the point~$A$ on it, let the angle
$DAE$ be constructed equal to the angle ADC [\prop{1}{23}]; and let
the straight line $AF$ be produced in a straight line with $EA$.

Then, since the straight line $AD$ falling on the two straight lines
$BC$, $EF$ has made the alternate angles $EAD$, $ADC$ equal to one
another,
therefore $EAF$ is parallel to~$BC$. \using{\prop{1}{17}}

Therefore through the given point~$A$ the straight line $EAF$ has been
drawn parallel to the given straight line~$BC$.
\qef
\end{proof}

\begin{notes}

Proclus rightly remarks (p.~376, 14—20) that, as it is implied in
\prop{1}{12} that only one perpendicular can be drawn to a straight
line from an external point, so here it is implied that only one
straight line can be drawn through a point parallel to a given
straight line.  The construction, be it observed, depends only upon
\prop{1}{27}, and might therefore have come directly after that
proposition.  Why then did Euclid postpone it until after \prop{1}{29}
and \prop{1}{30}?  Presumably because he considered it necessary,
before giving the construction, to place beyond all doubt the fact
that only one such parallel can be drawn.  Proclus infers this fact
from \prop{1}{30}; for, he says, if two straight lines could be drawn
through one and the same point parallel to the same straight line, the
two straight lines would be \emph{parallel}, though intersecting at
the given point: which is impossible.  I think it is a fair inference
that Euclid would have considered it necessary to justify the
assumption that only one parallel can be drawn by some such argument,
and that he deliberately determined that his own assumption was more
appropriate to be made the subject of a Postulate than the assumption
of the uniqueness of the parallel.

\end{notes}

\end{proposition}

\begin{proposition}
\label{prop:I_32}

\begin{statement}
In any triangle, if one of the sides be produced, the exterior angle
is equal to the two interior and opposite angles, and the three
interior angles of the triangle are equal to two right angles.
\end{statement}

\begin{proof}

Let $ABC$ be a triangle, and let one side of it $BC$ be produced
to~$D$;

I say that the exterior angle $ACD$ is equal to the two interior and
opposite angles $CAB$, $ABC$, and the three interior angles of the
triangle $ABC$, $BCA$, $CAB$ are equal to two right angles.

For let $CE$ be drawn through the point~$C$ parallel to the straight
line~$AB$. \using{\prop{1}{31}}

Then, since $AB$ is parallel to~$CE$, and $AC$ has fallen upon them,
the alternate angles $BAC$, $ACE$ are equal to one
another. \using{\prop{1}{29}}

Again, since $AB$ is parallel to~$CE$,
and the straight line $BD$ has fallen upon them,
the exterior angle $ECD$ is equal to the interior and opposite
angle~$ABC$. \using{\prop{1}{29}}

But the angle $ACE$ was also proved equal to the angle $BAC$;
therefore the whole angle $ACD$ is equal to the two interior and
opposite angles $BAC$, $ABC$.

Let the angle $ACB$ be added to each;
therefore the angles $ACD$, $ACB$ are equal to the three
angles $ABC$, $BCA$, $CAB$.

But the angles $ACD$, $ACB$ are equal to two right angles;
\using{\prop{1}{13}} therefore the angles $ABC$, $BCA$, $CAB$ are also
equal to two right angles.

Therefore etc.
\end{proof}

\begin{notes}

This theorem was discovered in the very early stages of Greek
geometry.  What we know of the history of it is gathered from three
allusions found in Eutocius, Proclus and Diogenes Laertius
respectively.

1.~Eutocius at the beginning of his commentary on the \emph{Conics} of
Apollonius (ed.\ Heiberg, Vol.~\r2.\ p.~170) quotes Geminus as saying
that ``the ancients (\greek{οἱ ἀρχαῖοι}) investigated the theorem of
the two right angles in each individual species of triangle, first in
the equilateral, again in the isosceles, and afterwards in the scalene
triangle, and later geometers demonstrated the general theorem to the
effect that in any triangle the three interior angles are equal to two
right angles.''

2.~Now, according to Proclus (p.~379, 2—5), Eudemus the Peripatetic
refers the discovery of this theorem to the Pythagoreans and gives
what he affirms to be their demonstration of it.  This demonstration
will be given below, but it should be remarked that it is general, and
therefore that the ``later geometers'' spoken of by Geminus were
presumably the Pythagoreans, whence it appears that the ``ancients''
contrasted with them must have belonged to the time of Thales, if they
were not his Egyptian instructors.

3.~That the truth of the theorem was known to Thales might also be
inferred from the statement of Pamphile (quoted by Diogenes Laertius,
\r1.~24–5, p.~6, ed.\ Cobet) that ``he, having learnt geometry from
the Egyptians, was the first to inscribe a right-angled triangle in a
circle and sacrificed an ox'' (on the strength of it); in other words,
he discovered that the angle in a semicircle is a right angle.  No
doubt, when this fact was once discovered (\emph{empirically}, say),
the consideration of the two isosceles triangles having the centre for
vertex and the sides of the right angle for bases respectively, with
the help of the theorem of Eucl.\ \prop{1}{5}, also known to Thales,
would easily lead to the conclusion that the sum of the angles of a
\emph{right-angled} triangle is equal to two right angles, and it
could be readily inferred that the angles of \emph{any} triangle were
likewise equal to two right angles (by resolving it into two
right-angled triangles).  But it is not easy to see how the property
of the angle in a semicircle could he \emph{proved} except (in the
reverse order) by means of the equality of the sum of the angles of a
right-angled triangle to two right angles; and hence it is most
natural to suppose, with Cantor, that Thales proved it (if he did
prove it) practically as Euclid does in \prop{3}{31}, i.e.\ by means
of \prop{1}{32} as applied to \emph{right-angled} triangles at all
events.

If the theorem of \prop{1}{32} was proved before Thales' time, or by
Thales himself, by the stages indicated in the note of Geminus, we may
be satisfied that the reconstruction of the argument of the older
proof by Hankel (pp.~96–7) and Cantor (\r1\tsub{3}, pp.~143–4) is not
far wrong.  First, it must have been observed that six angles equal to
an angle of an equilateral triangle would, if placed adjacent to one
another round a common vertex, fill up the whole space round that
vertex.  It is true that Proclus attributes to the Pythagoreans the
general theorem that only three kinds of regular polygons, the
equilateral triangle, the square and the regular hexagon, can fill up
the entire space round a point, but the practical knowledge that
equilateral triangles have this property could hardly have escaped the
Egyptians, whether they made floors with tiles in the form of
equilateral triangles or regular hexagons (Allman, \emph{Greek
  Geometry from Thales to Euclid}, p.~12) or joined the ends of
adjacent radii of a figure like the six-spoked wheel, which was their
common form of wheel from the time of Ramses~II.\ of the nineteenth
Dynasty, say 1300~\bc\ (Cantor, \r1\tsub{3}, p.~109).  It would then
be clear that six angles equal to an angle of an equilateral triangle
are equal to four right angles, and therefore that the three angles of
an equilateral triangle are equal to two right angles.  (It would be
as clear or clearer, from observation of a square divided into two
triangles by a diagonal, that an isosceles right-angled triangle has
each of its equal angles equal to half a right angle, so that an
isosceles right-angled triangle must have the sum of its angles equal
to two right angles.)  Next, with regard to the equilateral triangle,
it could not fail to be observed that, if $AD$ were drawn from the
vertex~$A$ perpendicular to the base~$BC$, each of the two
right-angled triangles so formed would have the sum of its angles
equal to two right angles; and this would be confirmed by completing
the rectangle $ADCE$, when it would be seen that the rectangle (with
its angles equal to four right angles) was divided by its diagonal
into two equal triangles, each of which had the sum of its angles
equal to two right angles.  Next it would be inferred, as the result
of drawing the diagonal of \emph{any} rectangle and observing the
equality of the triangles forming the two halves, that the sum of the
angles of any right-angled triangle is equal to two right angles, and
hence (the two congruent right-angled triangles being then placed so
as to form one isosceles triangle) that the same is true of \emph{any
  isosceles} triangle.  Only the last step remained, namely that of
observing that \emph{any} triangle could be regarded as the half of a
rectangle (drawn as indicated in the next figure), or simply that any
triangle could be divided into two right-angled triangles, whence it
would be inferred that in general the sum of the angles of any
triangle is equal to two right angles.

Such would be the probabilities if we could absolutely rely upon the
statements attributed to Pamphile and Geminus respectively. But in
fact there is considerable ground for doubt in both cases.

1.~Pamphiie's story of the sacrifice of an ox by Thales for joy at his
discovery that the angle in a semicircle is a right angle is too
suspiciously like the similar story told with reference to Pythagoras
and his discovery of the theorem of Eucl.\ \prop{1}{47} (Proclus,
p.~426, 6—9).  And, as if this were not enough, Diogenes Laertius
immediately adds that ''others, among whom is Apollodorus the
calculator (\greek{ὁ λογιστικος}), say it was Pythagoras'' (sc.\ who
``inscribed the right-angled triangle in a circle'').  Now Pamphile
lived in the reign of Nero (\ad~54—68) and therefore some 700 years
after the birth of Thales (about 640~\bc).  I do not know on what Max
Schmidt bases his statement (\emph{Kulturhistorische Beiträge zur
  Kenntnis des griechischen und römischen Altertums}, 1906, p.~31)
that ``other, \emph{much older}, sources name Pythagoras as the
discoverer of the said proposition,'' because nothing more seems to be
known of Apollodorus than what is stated here by Diogenes Laertius.
But it would at least appear that Apollodorus was only one of several
authorities who attributed the proposition to Pythagoras while
Pamphile is alone mentioned as referring it to Thales.  Again, the
connexion of Pythagoras with the investigation of the right-angled
triangle makes it \emph{a priori} more likely that it would be he who
would discover its relation to a semicircle.  On the whole, therefore,
the attribution to Thales would seem to be more than doubtful.

2.~As regards Geminus' account of the three stages through which the
proof of the theorem of \prop{1}{32} passed, we note, first, that it
is certainly not confirmed by Eudemus, who referred to the
Pythagoreans the \emph{discovery} of the theorem that the sum of the
angles of any triangle is equal to two right angles and says nothing
about any gradual stages by which it was proved.  Secondly, it must be
admitted, I think, that in the evolution of the proof as reconstructed
by Hankel the middle stage is rather artificial and unnecessary,
since, once it is proved that \emph{any right-angled} triangle has the
sum of its angles equal to two right angles, it is just as easy to
pass at once to any \emph{scalene} triangle (which is decomposable
into two \emph{unequal} right-angled triangles) as to the isosceles
triangle made up of two congruent right-angled triangles.  Thirdly, as
Heiberg has recently pointed out (\emph{Mathematisches zu
  Aristoteles}, p.~20), it is quite possible that the statement of
Geminus from beginning to end is simply due to a misapprehension of a
passage of Aristotle (\emph{Anal. Post.}\ \r1.~5, 74~a~25).  Aristotle
is illustrating his contention that a property is not scientifically
proved to belong to a class of things unless it is proved to belong
\emph{primarily} (\greek{πρῶτον}) and \emph{generally}
(\greek{καθόλου}) to the \emph{whole} of the class.  His first
illustration relates to parallels making with a transversal angles on
the same side together equal to two right angles, and has been quoted
above in the note on \prop{1}{27} (pp.~308—9).  His second
illustration refers to the transformation of a proportion
\emph{alternando}, which (he says) ``used at one time to be proved
separately'' for numbers, lines, solids, and times, although it admits
of being proved of all at once by one demonstration.  The third
illustration is: ``For the same reason, even \emph{if one should
  prove} (\greek{οὐδ’ ἄν τις δείξῃ}) with reference to each (sort of)
triangle, the equilateral, scalene and isosceles, separately, that
each has its angles equal to two right angles, either by one proofor
by different proofs, he does not yet know that \emph{the triangle},
i.e.\ the triangle \emph{in general}, has its angles equal to two
right angles, except in a sophistical sense, even though there exists
no triangle other than triangles of the kinds mentioned.  For he knows
it, not \emph{quâ} triangle, nor of \emph{every} triangle, except in a
numerical sense (\greek{κατ’ ἀριθμόν}); he does not know it
\emph{notionally} (\greek{κατ’ εἶδος}) of every triangle, even though
there be actually no triangle which he does not know.''

The difference between the phrase ``used at one time to be proved'' in
the second illustration and ``if any one should prove'' in the third
appears to indicate that, while the former referred to a historical
fact, the latter does not; the reference to a person who should prove
the theorem of \prop{1}{32} for the three kinds of triangle
separately, and then claim that he had proved it generally, states a
purely hypothetical case, a mere illustration.  Yet, coming after the
historical fact stated in the preceding illustration, it might not
unnaturally give the impression, at first sight, that it was
historical too.

On the whole, therefore, it would seem that we cannot safely go behind
the dictum of Eudemus that the discovery and proof of the theorem of
\prop{1}{32} in all its generality were Pythagorean.  This does not
however preclude its having been discovered by stages such as those
above set out after Hankel and Cantor.  Nor need it be doubted that
Thales and even his Egyptian instructors had advanced some way on the
same road, so far at all events as to see that in an equilateral
triangle, and in an isosceles right-angled triangle, the sum of the
angles is equal to two right angles.

\subsection*{The Pythagorean proof}

This proof, handed down by Eudemus (Proclus, p.~379, 2—15), is no less
elegant than that given by Euclid, and is a natural development from
the last figure in the reconstructed argument of Hankel.  It would be
seen, after the theory of parallels was added to geometry, that the
actual drawing of the perpendicular and the complete rectangle on $BC$
as base was unnecessary, and that the parallel to~$BC$ through $A$ was
all that was required.

Let $ABC$ be a triangle, and through~$A$ draw $DE$ parallel
to~$BC$. \using{\prop{1}{31}}

Then, since $BC$, $DE$ are parallel,
the alternate angles $DAB$, $ABC$ are equal, \using{\prop{1}{39}}
and so are the alternate angles $EAC$, $ACB$ also.

Therefore the angles $ABC$, $ACB$ are together equal to the angles
$DAB$, $EAC$.

Add to each the angle $BAC$; therefore the sum of the angles $ABC$,
$ACB$, $BAC$ is equal to the sum of the angles $DAB$, $BAC$, $CAE$,
that is, to two right angles.

\subsection*{Euclid's proof pre-Euclidean}

The theorem of \prop{1}{32} is Aristotle's favourite illustration when
he wishes to refer to some truth generally acknowledged, and so often
does it occur that it is often indicated by two or three words in
themselves hardly intelligible, e.g.\ \greek{τὸ δυσὶν ὀρθαῖσ}
(\emph{Anal. Post.}\ \r1.~24, 85~b~5) and \greek{ὑπάρχει παντὶ τριγώνῳ
  τὸ δύο} (\emph{ibid.}~85~b~11).

One passage (\emph{Metaph.}\ 1051~a~14) makes it clear, as Heiberg
(\emph{op.~cit.}\ p.~19) acutely observes, that in the proof as
Aristotle knew it Euclid's construction was used. ``Why does the
triangle make up two right angles?  Because \emph{the angles about one
  point} are equal to two right angles.  If then the parallel to the
side had been \emph{drawn up} (\greek{ἀνῆκτο}), the fact would at once
have been clear from merely looking at the figure.'' The words ``the
angles about one point'' would equally fit the Pythagorean
construction, but ``drawn \emph{upwards}'' applied to the parallel to
a side can only indicate Euclid's.

\subsection*{Attempts at proof independently of parallels}

The most indefatigable worker on these lines was Legendre, and a
sketch of his work has been given in the note on
Postulate~\ref{post:5} above.

One other attempted proof needs to be mentioned here because it has
found much favour. I allude to

\subsubsection*{Thibaut's method}

This appeared in Thibaut's \emph{Grundriss der reinen Mathematik},
Göttingen (2~ed.\ 1809, 3~ed.\ 1818), and is to the following effect.

Suppose $CB$ produced to~$D$, and let $BD$ (produced to any necessary
extent either way) revolve in one direction (say clockwise) first
about~$B$ into the position $BA$, then about~$A$ into the position
of~$AC$ produced both ways, and lastly about~$C$ into the position
$CB$ produced both ways.

The argument then is that the straight line $BD$ has revolved through
the sum of the three exterior angles of the triangle.  But, since it
has at the end of the revolution assumed a position in the same
straight line with its original position, it must have revolved
\emph{through four wight angles}.

Therefore the sum of the three exterior angles is equal to four right
angles; from which it follows that the sum of the three angles of the
triangle is equal to two right angles.

But it is to be observed that the straight line $BD$ revolves about
\emph{different points in it}, so that there is \emph{translation}
combined with \emph{rotatory} motion, and it is necessary to assume as
an axiom that the two motions are independent, and therefore that the
\emph{translation} may be neglected.

Schumacher (letter to Gauss of 3~May, 1831) tried to represent the
rotatory motion graphically in a second figure as mere motion round a
point; but Gauss (letter of 17~May, 1831) pointed out in reply that he
really assumed, without proving it, a proposition to the effect that
``If two straight lines (1) and (2) which cut one another make angles
$A$ , $A''$ with straight line (3) cutting both of them, and if a
straight line (4) in the same plane is likewise cut by (1) at an angle
$A'$, then (4) will be cut by (2) at the angle $A''$.  But this
proposition not only needs proof, but we may say that it is, in
essence, the very proposition to be proved ``(see Engel and Stäckel,
\emph{Die Theorie der Parallellinien von Euklid bis auf Gauss}, 1895,
p.~230).

How easy it is to be deluded in this way is plainly shown by Proclus'
attempt on the same lines.  He says (p.~384, 13—21) that the truth of
the theorem is borne in upon us by the help of ``common notions''
only.  For, if we conceive a straight line with two perpendiculars
drawn to it at its extremities, and if we then suppose the
perpendiculars to (revolve about their feet and) approach one another,
so as to form a triangle, we see that, to the extent to which they
converge, they diminish the right angles which they made with the
straight line, so that the amount taken from the right angles is also
the amount added to the vertical angle of the triangle, and the three
angles are necessarily made equal to two right angles,'' But a
moment's reflection shows that, so far from being founded on mere
``common notions,'' the supposed proof assumes, to begin with, that,
if the perpendiculars approach one another ever so little, they will
then form a triangle immediately, i.e., it assumes
Postulate~\ref{post:5} itself; and the fact about the vertical angle
can only be seen by means of the equality of the alternate angles
exhibited by drawing a perpendicular from the vertex of the triangle
to the base, i.e.\ a \emph{parallel} to either of the original
perpendiculars.

\subsubsection*{Extension to polygons}

The two important corollaries added to \prop{1}{32} in Simson's
edition are given by Proclus; but Proclus' proof of the first is
different from, and perhaps somewhat simpler than, Simson's.

1.~\emph{The turn of the interior angles of a convex rectilineal
  figure is equal to twice as many right angles as the figure has
  sides, less four.}

For let one angular point~$A$ be joined to all the other angular
points with which it is not connected already.

The figure is then divided into triangles, and mere inspection shows

(1)~that the number of triangles is two less than the number of sides
in the figure,

(2)~that the sum of the angles of all the triangles is equal to the
sum of all the interior angles of the figure.

Since then the sum of the angles of each triangle is equal to two
right angles the sum of the interior angles of the figure is equal to
$2(n - 2)$ right angles, i.e.\ $(2n - 4)$ right angles, where $n$ is
the number of sides in the figure.

3.~\emph{The exterior angles of any convex rectilineal figure are
  together equal to four right angles.}

For the interior and exterior angles together are equal to $2n$ right
angles, where $n$ is the number of sides.

And the interior angles are together equal to $(2n-4)$ right angles.

Therefore the exterior angles are together equal to four right angles.

This last property is already quoted by Aristotle as true of all
rectilineal figures in two passages (\emph{Anal.\ Post.}\ \r1.~24,
85~b~38 and \r2.~17, 99~a~10).

\end{notes}

\end{proposition}

\begin{proposition}
\label{prop:I_33}

\begin{statement}
The straight lines joining equal and parallel straight lines (at the
extremities which are) in the same directions (respectively) are
themselves also equal and parallel.
\end{statement}

\begin{proof}

Let $AB$, $CD$ be equal and parallel, and let the straight lines $AC$,
$BD$ join them (at the extremities which are) in the same directions
(respectively);

I say that $AC$, $BD$ are also equal and parallel.

Let $BC$ be joined.

Then, since $AB$ is parallel to~$CD$, and $BC$ has fallen upon them,
the alternate angles $ABC$, $BCD$
are equal to one another. \using{\prop{1}{29}}

And, since $AB$ is equal to~$CD$, and $BC$ is common, the two sides
$AB$, $BC$ are equal to the two sides $DC$, $CB$; and the angle $ABC$
is equal to the angle $BCD$; therefore the base $AC$ is equal to the
base $BD$, and the triangle $ABC$ is equal to the triangle $DCB$, and
the remaining angles will be equal to the remaining angles
respectively, namely those which the equal sides subtend;
\using{\prop{1}{4}} therefore the angle $ACB$ is equal to the angle
$CBD$.

And, since the straight line $BC$ falling on the two straight lines
$AC$, $BD$ has made the alternate angles equal to one another, $AC$ is
parallel to $BD$. \using{\prop{1}{27}}

And it was also proved equal to it.

Therefore etc.
\end{proof}

\begin{annotations}

1.~\textbf{joining…(at the extremities which are) in the same
  directions (respectively).}  I have for clearness' sake inserted the
words in brackets though they are not in the original Greek, which has
``joining…in the same directions'' or ``on the same sides,''
\greek{ε)πι` τα` αθ)τα` με'ρη ε)πιζεθγνη|'οθσαι}.  The expression
``towards the same parts,'' though usage has sanctioned it, is perhaps
not quite satisfactory.

15.~\textbf{$DC$, $CB$} and 18.~\textbf{$DCB$}.  The Greek has ``$BC$,
$CD$'' and ``eBCD'' in these places respectively.  Euclid is not
always careful to write in corresponding order the Letters denoting
corresponding points in congruent figures.  On the contrary, he
evidently prefers the alphabetical order, and seems to disdain to
alter it for the sake of beginners or others who might be confused by
it.  In the case of angles alteration is perhaps unnecessary; but in
the case of triangles and pairs of corresponding sides I have ventured
to alter the order to that which the mathematician of to-day expects.

\end{annotations}

\begin{notes}

This proposition is, as Proclus says (p.~385,~5), the connecting link
between the exposition of the theory of parallels and the
investigation of parallelograms.  For, while it only speaks of equal
and parallel straight lines connecting those ends of equal and
parallel straight lines which are in the same directions, it gives,
without expressing the fact, the construction or origin of the
parallelogram, so that in the next proposition Euclid is able to speak
of ``parallelogrammic areas'' without any further explanation.

\end{notes}

\end{proposition}

\begin{proposition}
\label{prop:I_34}

\begin{statement}
In parallelogrammic areas the opposite sides and angles are equal to
one another, and the diameter bisects the areas.
\end{statement}

\begin{proof}

Let $ACDB$ be a parallelogrammic area, and $BC$ its diameter;

I say that the opposite sides and angles of the parallelogram $ACDB$
are equal to one another, and the diameter~$BC$ bisects it.

For, since $AB$ is parallel to~$CD$, and the straight line $BC$ has
fallen upon them,
the alternate angles $ABC$, $BCD$
are equal to one another. \using{\prop{1}{29}}

Again, since $AC$ is parallel to $BD$, and $BC$ has fallen upon them,
the alternate angles $ACB$, $CBD$ are equal to one
another. \using{\prop{1}{29}}

Therefore $ABC$, $DCB$ are two triangles having the two
angles $ABC$, $BCA$ equal to the two angles $DCB$, $CBD$
respectively, and one side equal to one side, namely that
to adjoining the equal angles and common to both of them, $BC$;
therefore they will also have the remaining sides equal
to the remaining sides respectively, and the remaining angle
to the remaining angle; \using{\prop{1}{26}}
therefore the side $AB$ is equal to~$CD$,
and $AC$ to~$BD$,
and further the angle $BAC$ is equal to the angle $CDB$.
And, since the angle $ABC$ is equal to the angle $BCD$,
and the angle $CBD$ to the angle~$ACB$,
the whole angle $ABD$ is equal to the whole angle~$ACD$.
\using{\rcn{2}}

And the angle $BAC$ was also proved equal to the angle $CDB$.
Therefore in parallelogrammic areas the opposite sides and angles are
equal to one another.

I say, next, that the diameter also bisects the areas.
For, since $AB$ is equal to $CD$,
and $BC$ is common,
the two sides $AB$, $BC$ are equal to the two sides $DC$, $CB$
respectively;
and the angle $ABC$ is equal to the angle $BCD$;
therefore the base $AC$ is also equal to~$DB$,
and the triangle $ABC$ is equal to the triangle $DCB$. \using{\prop{1}{4}}

Therefore the diameter $BC$ bisects the parallelogram $ACDB$.
\end{proof}

\begin{annotations}

1.~It is to be observed that, when parallelograms have to be mentioned
for the first time, Euclid calls them ``\textbf{parallelogrammic
  areas}'' or, mare exactly, ``parallelogram'' areas
(\greek{παραλληλόγραμμα χωρία}).  The meaning is simply areas bounded
by parallel straight lines whh the further limitation placed upon the
term by Euclid that \emph{four-sided} figures are so called, although
of course there are certain regular polygons which have opposite sides
parallel, and which therefore might be said to be areas bounded by
parallel straight lines.  We gather from Proclus (p.~393) that the
word ``parallelogram'' was first introduced by Euclid, that its use
was suggested by \prop{1}{33}, and that the formation of the word
\greek{παραλληλόγραμμος} (parallel-lined) was analogous to that of
\greek{εὐθύγραμμος} (straight-lined or rectilineal).

17, 18, 40. \textbf{$DCB$} and 36.~\textbf{$DC$, $CB$}.  The Greek has
in these places ``$BCD''$ and ``$CD$, $BC$'' respectively. Cf.\ note
on \prop{1}{33}, lines 15,~18.

\end{annotations}

\begin{notes}

After specifying the particular kinds of parallelograms (squares and
rhombi) in which the diagonals bisect the angles which they join, as
well as the areas, and those (rectangles and rhomboids) in which the
diagonals do not bisect the angles, Proclus proceeds (pp.~390
sqq.)\ to analyse this proposition with reference to the distinction
in Aristotle's \emph{Anal.\ Post.} (\r1.~4, 5, 73~a~21—74~b~4) between
attributes which are only predicable of every individual thing
(\greek{κατὰ παντός}) in a class and those which are true of it
\emph{primarily} (\greek{τούτου πρώτου}) and generally
(\greek{καθόλου}).  We are apt, says Aristotle, to mistake a proof
\greek{κατὰ παντός} for a proof \greek{τούτου πρώτου καθόλου} because
it is either impossible to find a higher generality to comprehend all
the particulars of which the predicate is true, or to find a name for
it.  (Part of this passage of Aristotle has been quoted above in the
note on \prop{1}{32}, pp.~319–320.)

Now, says Proclus, adapting Aristotle's distinction to
\emph{theorems}, the present proposition exhibits the distinction
between theorems which are \emph{general} and theorems which are
\emph{not general}.  According to Proclus, the first part of the
proposition stating that the opposite sides and angles of a
parallelogram are equal is \emph{general} because the property is only
true of parallelograms; but the second part which asserts that the
diameter bisects the area is \emph{not general} because it does not
include all the figures of which this property is true, e.g.\ circles
and ellipses.  Indeed, says Proclus, the first attempts upon problems
seem usually to have been of this partial character
(\greek{μερικώτεραι}), and generality was only attained by degrees.
Thus ``the ancients, after investigating the fact that the diameter
bisects an ellipse, a circle, and a parallelogram respectively,
proceeded to investigate what was common to these cases,'' though ``it
is difficult to show what is common to an ellipse, a circle and a
parallelogram.''

I doubt whether the supposed distinction between the two parts of the
proposition, in point of ``generality,'' can be sustained.  Proclus
himself admits that it is presupposed that the subject of the
proposition is a \emph{quadrilateral}, because there are other figures
(e.g.\ regular polygons of an even number of sides) besides
parallelograms which have their opposite sides and angles equal;
therefore the second part of the theorem is, in this respect, no more
\emph{general} than the other, and, if we are entitled to the tacit
limitation of the theorem to quadrilaterals in one part, we are
equally entitled to it in the other.

It would almost appear as though Proclus had drawn the distinction for
the mere purpose of alluding to investigations by Greek geometers on
the general subject of \emph{diameters} of all sorts of figures; and
it may have been these which brought the subject to the point at which
Apollonius could say in the first definitions at the beginning of his
\emph{Conics} that ``In \emph{any bent line}, such as is in one plane,
I give the name \emph{diameter} to any straight line which, being
drawn from the bent line, bisects all the straight lines (chords)
drawn in the line parallel to any straight line.'' The term bent line
(\greek{καυπύλη γραμμή}) includes, e.g.\ in Archimedes, not only
curves, but any composite line made up of straight lines and curves
joined together in any manner.  It is of course clear that either
diagonal of a parallelogram bisects all lines drawn within the
parallelogram parallel to the other diagonal.

An-Nairīzī gives after \prop{1}{31} a neat construction for dividing a
straight line into any number of equal parts (ed.\ Curtze, p.~74,
ed.\ Besthorn-Heiberg, pp.~141—3) which requires only one measurement
repeated, together with the properties of parallel lines including
\prop{1}{33}, \prop*{1}{34}.  As \prop{1}{33}, \prop*{1}{34} are
assumed, I place the problem here.  The particular case taken is the
problem of dividing a straight line into \emph{three} equal parts.

Let $AB$ be the given straight line.  Draw $AC$, $BD$ at right angles
to it on opposite sides.

An-Nairīzī takes $AC$, $BD$ of the same length and then bisects $AC$
at~$E$ and $BD$ at~$F$.  But of course it is even simpler to measure
$AE$, $EC$ along one perpendicular equal and of any length, and $BF$,
$FD$ along the other also equal and of the same length.

Join $ED$, $CF$ meeting $AB$ in $G$, $H$ respectively.

Then shall $AG$, $GH$, $HB$ all be equal.

Draw $HK$ parallel to~$AC$, or at right angles to~$AB$.

Since now $EC$, $FD$ are equal and parallel, $ED$, $CF$ are equal and
parallel. \using{\prop{1}{33}}

And $HK$ was drawn parallel to~$AC$.

Therefore $ECHK$ is a parallelogram; whence $KH$ is equal as well as
parallel to~$EC$, and therefore to~$EA$.

The triangles $EAG$, $KHG$ have now two angles respectively equal and
the sides $AE$, $HK$ equal.

Thus the triangles are equal in all respects, and $AG$ is equal
to~$GH$.

Similarly the triangles $KHG$, $FBH$ are equal in all respects, and
$GH$ is equal to~$HB$.

If now we wish to extend the problem to the case where $AB$ is to be
divided into $n$ parts, we have only to measure $(n-1)$ successive
equal lengths along $AC$ and $(n— 1)$ successive lengths, each equal
to the others, along $BD$.  Then join the first point arrived at
on~$AC$ to the last point on~$BD$, the second on~$AC$ to the last but
one on~$BD$, and so on; and the joining lines cut~$AB$ in points
dividing it into $n$ equal parts.

\end{notes}

\end{proposition}

\begin{proposition}
\label{prop:I_35}

\begin{statement}
Parallelograms which are on the same base and in the same parallels
are equal to one another.
\end{statement}

\begin{proof}

Let $ABCD$, $EBCF$ be parallelograms on the same base $BC$ and in the
same parallels $AF$, $BC$; I say that $ABCD$ is equal to the
parallelogram $EBCF$,

For, since $ABCD$ is a parallelogram, $AD$ is equal
to~$BC$. \using{\prop{1}{34}}

For the same reason also
$EF$ is equal to~$BC$,
so that $AD$ is also equal to~$EF$; \using{\rcn{1}}
and $DE$ is common;
therefore the whole $AE$ is equal to the whole~$DF$.
\using{\rcn{2}}

But $AB$ is also equal to~$DC$; \using{\prop{1}{34}} therefore the two
sides $EA$, $AB$ are equal to the two sides $FD$, $DC$ respectively,
and the angle $FDC$ is equal to the angle $EAB$, the exterior to the
interior; \using{\prop{1}{39}}
therefore the base $EB$ is equal
to the base $FC$,
and the triangle $EAB$ will be equal to the triangle $FDC$.
\using{\prop{1}{4}}

Let $DGE$ be subtracted from each;
therefore the trapezium $ABGD$ which remains is equal to
the trapezium $EGCF$ which remains. \using{\rcn{3}}

Let the triangle $GBC$ be added to each;
therefore the whole parallelogram $ABCD$ is equal to the whole
parallelogram $EBCF$.  \using{\rcn{2}}

Therefore etc.
\end{proof}

\begin{annotations}

11. \textbf{$FDC$.} The text has ``$DFC$.''

22. \textbf{Let $DGE$ be subtracted.} Euclid speaks of the triangle
$DGE$ without any explanation that, in the case which he takes (where
$AD$, $EF$ have no point in common), $BE$, $CD$ must meet at a
point~$G$ between the two parallels.  He allows this to appear from
the figure simply.

\end{annotations}

\begin{notes}

\subsection*{Equality in a new sense}

It is important to observe that we are in this proposition introduced
for the first time to a new conception of equality between figures.
Hitherto we have had equality in the sense of congruence only, as
applied to straight lines, angles, and even triangles
(cf.\ \prop{1}{4}).  Now, without any explicit reference to any change
in the meaning of the term, figures are inferred to be equal which are
equal in \emph{area} or in \emph{content} but need not be of the same
form.  No definition of equality is anywhere given by Euclid; we are
left to infer its meaning from the few \emph{axioms} about ``equal
things.''  It will be observed that in the above proof the
``equality'' of two parallelograms on the same base and between the
same parallels is inferred by the successive steps (1)~of subtracting
one and the same area (the triangle $DGE$) from two areas equal in the
sense of \emph{congruence} (the triangles $AEB$, $DFC$), and inferring
that the remainders (the trapezia $ABGD$, $EGCF$) are ``equal'';
(2)~of adding one and the same area (the triangle $GBC$) to each of
the latter ``equal'' trapezia, and inferring the equality of the
respective sums (the two given parallelograms).

As is well known, Simson (after Clairaut) slightly altered the proof
in order to make it applicable to all the three possible cases.  The
alteration substituted \emph{one} step of subtracting congruent areas
(the triangles $AEB$, $DFC$) from one and the same area (the trapezium
$ABCF$) for the \emph{two} steps above shown of first subtracting and
then adding a certain area.

While, in either case, nothing more is explicitly used than the axioms
that, \emph{if equals be added to equals, the wholes are equal} and
that, \emph{if equals be subtracted from equals, the remainders are
  equal}, there is the further \emph{tacit} assumption that it is
indifferent to \emph{what part} or from \emph{what part} of the same
or equal areas the same or equal areas are added or subtracted.  De
Morgan observes that the postulate ``an area taken from an area leaves
the same area from whatever part it may be taken'' is particularly
important as the key to equality of non-rectilineal areas which could
not be cut into coincidence geometrically.

Legendre introduced the word \emph{equivalent} to express this wider
sense of equality, restricting the term \emph{equal} to things equal
in the sense of congruent; and this distinction has been found
convenient.

I do not think it necessary, nor have I the space, to give any account
of the recent developments of the theory of equivalence on new lines
represented by the researches of W.~Bolyai, Duhamel, De Zolt, Stolz,
Schur, Veronese, Hilbert and others, and must refer the reader to Ugo
Amaldi's article \emph{Sulla teoria dell' equivalenza} in
\emph{Questioni riguardanti le matematiche elementari},
\r1.\ (Bologna, 1912), pp.~145—198, and to Max Simon, \emph{Über die
  Entwicklung der Elmentar-geometrie im XIX.\ Jahrhundert} (Leipzig,
1906), pp.~115–120, with their full references to the literature of
the subject.  I may however refer to the suggestive distinction of
phraseology used by Hilbert (\emph{Grundlagen der Geometrie}, pp.~39,
40):

(1)~``Two polygons are called \emph{divisibly-equal}
(\emph{zerlegungsgleich}) if they can be divided into a \emph{finite}
number of triangles which are congruent two and two.''

(2)~``Two polygons are called \emph{equal in content}
(\emph{inhaltsgleich}) or \emph{of equal content} if it is possible to
add \emph{divisibly-equal} polygons to them in such a way that the two
combined polygons are \emph{divisibly-equal}.''

(Amaldi suggests as alternatives for the terms in (1) and~(2) the
expressions \emph{equivalent by sum} and \emph{equivalent by
  difference} respectively.)

From these definitions it follows that ``by combining
\emph{divisibly-equal} polygons we again arrive at
\emph{divisibly-equal} polygons; and, if we subtract
\emph{divisibly-equal} polygons from \emph{divisibly-equal} polygons,
the polygons remaining are \emph{equal in content}.''

The proposition also follows without difficulty that, ``if two
polygons are \emph{divisibly-equal} to a third polygon, they are also
\emph{divisibly-equal} to one another; and, if two polygons are
\emph{equal in content} to a third polygon, they are \emph{equal in
  content} to one another.''

\subsection*{The different cases}

As usual, Proclus (pp.~399—400), observing that Euclid has given only
the most difficult of the three possible cases, adds the other two
with separate proofs.  In the case where $E$ in the figure of the
proposition falls between $A$ and~$D$, he adds the congruent triangles
$ABE$, $DCF$ respectively to the smaller trapezium $EBCD$, instead of
subtracting them (as Simson does) from the larger trapezium $ABCF$.

\subsection*{An ancient ``Budget of Paradoxes.''}

Proclus observes (p.~396, 12 sqq.)\ that the present theorem and the
similar one relating to triangles are among the so-called paradoxical
theorems of mathematics, since the uninstructed might well regard it
as impossible that the area of the parallelograms should remain the
same while the length of the sides other than the base and the side
opposite to it may increase indefinitely.  He adds that mathematicians
had made a collection of such paradoxes, the so-called \emph{treasury
  of paradoxes} (\greek{ὁ παράδοξος τόπος})—cf.\ the similar
expressions \greek{τόπος ἀναλυομενος} (treasury of analysis) and
\greek{τόπος ἀστρονομουμενος}—in the same way as the Stoics with their
\emph{illustrations} (\greek{ὥσπερ οἱ ἀπὸ τῆς Στοᾶς ἐπὶ τῶν
  δειγμάτων}).  It may be that this \emph{treasury of paradoxes} was
the work of Erycinus quoted by Pappus (\r3.\ p.~107,~8) and mentioned
above (note on \prop{1}{21}, p.~290).

\subsection*{Locus-theorems and loci in Greek geometry}

The proposition \prop{1}{35} is, says Proclus (pp.~394—6), the first
\emph{locus-theorem} (\greek{τοπικὸν θεώρημα}) given by Euclid.
Accordingly it is in his note on this proposition that Proclus gives
us his view of the nature of a locus-theorem and of the meaning of the
word \emph{locus} (\greek{τόπος}); and great importance attaches to
his words because he is one of the three writers (Pappus and Eutocius
being the two others) upon whom we have to rely for all that is known
of the Greek conception of geometrical loci.

Proclus' explanation (pp.~394, 15—395, 2) is as follows. ``I call
those (theorems) \emph{locus-theorems} (\greek{τοπικά}) in which the
same property is found to exist on the whole of some locus
(\greek{πρὸς ὅλῳ τινὶ τόπῳ}), and (I call) a locus a position of a
line or a surface producing one and the same property (\greek{γραμμῆς
  ἣ ἐπιφανείας θέσιν ποιοῦσαν ἒν καὶ ταὐτὸν σύμπτωμα}). For, of
locus-theorems, some are constructed on lines and others on surfaces
(\greek{τῶν γὰρ τοπικοῶν τὰ μέν ἐστι πρὸς γραμμαῖς συνιστάμενα, τὰ δὲ
  πρὸς ἐπιφανείαις}). And, since some lines are plane
(\greek{ἐπίπεδοι}) and others solid (\greek{στερεαί})—those being
plane which are simply conceived of in a plane (\greek{ὦν ἐν ἐπιπέδῳ
  ἁπλῆ ἡ νόεσις}), and those solid the origin of which is revealed
from some section of a solid figure, as the cylindrical helix and the
conic lines (\greek{ὡς τῆς κυλινδρικῆς ἕλικος καὶ τῶν κωνικῶν
  γραμμῶν})–I should say (\greek{φαίην ἄν}) further that, of
locus-theorems on lines, some give a plane locus and others a solid
locus.''

Leaving out of sight for the moment the class of \emph{loci on
  surfaces}, we find that the distinction between \emph{plane} and
\emph{solid loci}, or \emph{plane} and \emph{solid lines}, was
similarly understood by Eutocius, who says (Apollonius, ed.\ Heiberg,
\r2.\ p.~184) that ``\emph{solid loci} have obtained their name from
the fact that the lines used in the solution of problems regarding
them have their origin in the section of solids, for example the
sections of the cone and several others.''  Similarly we gather from
Pappus that \emph{plane loci} were straight lines and circles, and
\emph{solid loci} were conics.  Thus he tells us (\r7.~p.~672, 20)
that Aristaeus wrote five books of \emph{Solid Loci} ``supplementary
to (literally, continuous with) the conics''; and, though Hultsch
brackets the passage (\r7.\ p.~662, 10–15) which says plainly that
\emph{plane loci} are straight lines and circles, while \emph{solid
  loci} are sections of cones, i.e.\ parabolas, ellipses and
hyperbolas, we have the exactly corresponding distinction drawn by
Pappus (\r3.\ p.~54, 7—16) between \emph{plane} and \emph{solid
  problems}, plane problems being those solved by means of straight
lines and circumferences of circles, and solid problems those solved
by means of one or more of the sections of the cone.  But, whereas
Proclus and Eutocius speak of other \emph{solid loci} besides conics,
there is nothing in Pappus to support the wider application of the
term.  According to Pappus (\r3.\ p.~54, 16–21) problems which could
not be solved by means of straight lines, circles, or conics were
\emph{linear} (\greek{γραμμικά}) because they used for their
construction lines having a more complicated and unnatural origin than
those mentioned, namely such curves as \emph{quadratrices}, conchoids
and cissoids.  Similarly, in the passage supposed to be interpolated,
\emph{linear loci} are distinguished as those which are neither
straight lines nor circles nor any of the conic sections
(\r7.\ p.~662, 13—15).  Thus the classification given by Proclus and
Eutocius is less precise than that which we find in Pappus; and the
inclusion by Proclus of the cylindrical helix among solid loci, on the
ground that it arises from a section of a solid figure, would seem to
be, in any case, due to some misapprehension.

Comparing these passages and the hints in Pappus about \emph{loci on
  surfaces} (\greek{τόποι πρὸς ἐπιφανείᾳ}) with special reference to
Euclid's two books under that title, Heiberg concludes that \emph{loci
  on lines} and \emph{loci on surfaces} in Proclus' explanation are
loci which \emph{are} lines and loci which \emph{are} surfaces
respectively.  Bui some qualification is necessary as regards Proclus'
conception of \emph{loci on lines}, because he goes on to say (p.~395,
5), with reference to this proposition, that, while the locus is a
\emph{locus on lines} and moreover \emph{plane}, it is ``the whole
space between the parallels'' which is the locus of the various
parallelograms on the same base proved to be equal in area.
Similarly, when he quotes \prop{3}{21} about the equality of the
angles in the same segment and \prop{3}{31} about the right angle in a
semicircle as cases where a circumference of a circle takes the place
of a straight line in a plane locus-theorem, he appears to imply that
it is the segment or semicircle as an area which is regarded as the
locus of an infinite number of triangles with the same base and equal
vertical angles, rather than that it is the \emph{circumference} which
is the locus of the angular \emph{points}.  Likewise he gives the
equality of parallelograms inscribed in ``the asymptotes and the
hyperbola'' as an example of a \emph{solid} locus-theorem, as if the
area included between the curve and its asymptotes was regarded as the
\emph{locus} of the equal parallelograms.  However this may be, it is
clear that the locus in the present proposition can only be either
(1)~a \emph{line}-locus of a \emph{line}, not a point, or (2)~an
\emph{area}-locus of an \emph{area}, not a point or a line; and we
seem to be thus brought to another and different classification of
loci corresponding to that quoted by Pappus (\r7.\ p.~660, 18
sqq.)\ from the preliminary exposition given by Apollonius in his
\emph{Plane Loci}.  According to this, loci in general are of three
kinds: (1)~\greek{ἐφεκτικοί}, \emph{holding-in}, in which sense the
locus of a point is a point, of a line a line, of a surface a surface,
and of a solid a solid, (2)~\greek{διεξοδικοί}, \emph{moving along}, a
line being in this sense a locus of a point, a surface of a line and a
solid of a surface, (3)~\greek{ἀναστροφικοί}, where a surface is a
locus of a point and a solid of a line.  Thus the locus in this
proposition, whether it is the space between the two parallels
regarded as the locus of the equal parallelograms, or the line
parallel to the base regarded as the locus of the sides opposite to
the base, would seem to be of the first class (\greek{ἐφεκτικός});
and, as Proclus takes the former view of it, a \emph{locus on lines}
is apparently not merely a locus which \emph{is} a line but a locus
\emph{bounded by lines} also, the locus being \emph{plane} in the
particular case because it is bounded by straight lines, or, in the
case of \prop{3}{21}, \prop*{3}{31}, by straight lines and circles,
but not by any higher curves.

Proclus notes lastly (p.~395, 13—21) that, according to Geminus,
``Chrysippus likened locus-theorems to the \emph{ideas}.  For, as the
ideas confine the genesis of unlimited (particulars) within defined
limits, so in such theorems the unlimited (particular figures) are
confined within defined \emph{places} or \emph{loci} (\greek{τόποι}).
And it is this boundary which is the cause of the equality; for the
height of the parallels, which remains the same, while an infinite
number of parallelograms are conceived on the same base, is what makes
them all equal to one another.''

\end{notes}

\end{proposition}

\begin{proposition}
\label{prop:I_36}

\begin{statement}
Parallelograms which are on equal bases and in the same
parallels are equal to one another.
\end{statement}

\begin{proof}

Let $ABCD$, $EFGH$ be parallelograms which are on
equal bases $BC$, $FG$ and in the same parallels $AH$, $BG$;

\infig{propI_36}

I say that the parallelogram $ABCD$ is equal to $EFGH$.

For let $BE$, $CH$ be joined.

Then, since $BC$ is equal to $FG$
while $FG$ is equal to $EH$,
$BC$ is also equal to $EH$. \using{\rcn{1}}

But they are also parallel.

And $EB$, $HC$ join them;
but straight lines joining equal and parallel straight lines (at
the extremities which are) in the same directions (respectively)
are equal and parallel.  \using{\prop{1}{33}}

Therefore $EBCH$ is a parallelogram. \using{\prop{1}{34}}

And it is equal to $ABCD$;
for it has the same base $BC$ with it, and is in the same
parallels $BC$, $AH$ with it. \using{\prop{1}{35}}

For the same reason also $EFGH$ is equal to the same
$EBCH$; \using{\prop{1}{35}}
so that the parallelogram $ABCD$ is also equal to $EFGH$.
\using{\rcn{1}}

Therefore etc.
\end{proof}

\end{proposition}

\begin{proposition}
\label{prop:I_37}

\begin{statement}
Triangles which are on the same base and in the same parallels are
equal to one another.
\end{statement}

\begin{proof}

Let $ABC$, $DBC$ be triangles on the same base $BC$ and
in the same parallels $AD$, $BC$;
I say that the triangle $ABC$ is equal to the triangle $DBC$.

\sidefig{propI_37}

Let $AD$ be produced in both
directions to $E$,~$F$;
through $B$ let $BE$ be drawn parallel
to~$CA$, \using{\prop{1}{31}}
and through $C$ let $CF$ be drawn
parallel to~$BD$. \using{\prop{1}{31}}

Then each of the figures
$EBCA$, $DBCF$ is a parallelogram;
and they are equal,

for they are on the same base $BC$ and in the same
parallels $BC$, $EF$. \using{\prop{1}{34}}

Moreover the triangle $ABC$ is half of the parallelogram
$EBCA$; for the diameter $AB$ bisects it. \using{\prop{1}{34}}

And the triangle $DBC$ is half of the parallelogram $DBCF$;
to for the diameter $DC$ bisects it. \using{\prop{1}{34}}

[But the halves of equal things are equal to one another.]

Therefore the triangle $ABC$ is equal to the triangle $DBC$.

Therefore etc.
\end{proof}

\begin{annotations}

21. Here and in the next proposition Heiberg brackets the words ``But
the halves of equal things are equal to one another'' on the ground
that, since the \emph{Common Notion} which asserted this fact was
interpolated at a very early date (before the time of Theon), it is
probable that the words here were interpolated at the same time.
Cf.\ note above (p.~\pageref{224}) on the interpolated \emph{Common
  Notion}.

\end{annotations}

\begin{notes}

There is a lacuna in the text of Proclus' notes to \prop{1}{36} and
\prop{1}{37}.  Apparently the end of the former and the beginning of
the latter are missing, the \textsc{mss.}\ and the \emph{editio
  princeps} showing no separate note for \prop{1}{37} and no lacuna,
but going straight on without regard to sense.  Proclus had evidently
remarked again in the missing passage that, in the case of both
parallelograms and triangles between the same parallels, the two sides
which stretch from one parallel to the other may increase in length to
any extent, while the area remains the same.  Thus the
\emph{perimeter} in parallelograms or triangles is of itself no
criterion as to their area.  Misconception on this subject was rife
among non-mathematicians; and Proclus (p.~403, 5~sqq.)\ tells us
(1)~of describers of countries (\greek{χωρογράφοι}) who drew
conclusions regarding the size of cities from their perimeters, and
(2)~of certain members of communistic societies in his own time who
cheated their fellow members by giving them land of greater perimeter
but less area than they took themselves, so that, on the one hand,
they got a reputation for greater honesty while, on the other, they
took more than their share of produce.  Cantor
(\emph{Gesch.\ d.~Math.}\ \r1\tsub{3}, p.~172) quotes several remarks
of ancient authors which show the prevalence of the same
misconception.  Thus Thucydides estimates the size of Sicily according
to the time required for circumnavigating it.  About 130~\bc\ Polybius
said that there were people who could not understand that camps of the
same periphery might have different capacities.  Quintilian has a
similar remark, and Cantor thinks he may have had in his mind the
calculations of Pliny, who compares the size of different parts of the
earth by adding their length to their breadth.

The comparison however of the areas of different figures of equal
contour had not been neglected by mathematicians.  Theon of
Alexandria, in his commentary on Book~\r1.\ of Ptolemy's
\emph{Syntaxis}, has preserved a number of propositions on the subject
taken from a treatise by Zenodorus \greek{περὶ ἰσομέτρων σχημάτων}
(reproduced in Latin on pp.~1190—1211 of Hultsch's edition of Pappus)
which was written at some date between, say, 200~\bc\ and 90~\ad, and
probably not long after the former date, Pappus too has at the
beginning of Book~\r5.\ of his \emph{Collection} (pp.~308 sqq.)\ the
same propositions, in which he appears to have followed Zenodorus
pretty closely while making some changes in detail.  The propositions
proved by Zenodorus and Pappus include the following: (1)~that,
\emph{of all polygons of the same number of sides and equal perimeter,
  the equilateral and equiangular polygon is the greatest in area},
(2)~that, \emph{of regular polygons of equal perimeter, that is the
  greatest in area which has the most angles}, (3)~that \emph{a circle
  is greater than any regular polygon of equal contour}, (4)~that,
\emph{of all circular segments in which the arcs are equal in length,
  the semicircle is the greatest}.  The treatise of Zenodorus was not
confined to propositions about plane figures, but gave also the
theorem that, \emph{of all solid figures the surfaces of which are
  equal, the sphere is the greatest in volume}.

\end{notes}

\end{proposition}

\begin{proposition}
\label{prop:I_38}

\begin{statement}
Triangles which are on equal bases and in the same Parallels are equal
to one another.
\end{statement}

\begin{proof}

\sidefig{propI_38}

Let $ABC$, $DEF$ be triangles on equal bases $BC$, $EF$ and
in the same parallels $BF$, $AD$;
I say that the triangle $ABC$ is
equal to the triangle $DEF$.

For let $AD$ be produced in
both directions to $G$, $H$
through $B$ let $BG$ be drawn
parallel to $CA$, \using{\prop{1}{31}}
and through $F$ let $FH$ be drawn parallel to~$DE$.

Then each of the figures $GBCA$, $DEFH$ is a parallelogram;
and $GBCA$ is equal to $DEFH$;
for they are on equal bases $BC$, $EF$ and in the same
parallels $BF$, $GH$. \using{\prop{1}{36}}

Moreover the triangle $ABC$ is half of the parallelogram
$GBCA$; for the diameter $AB$ bisects it. \using{\prop{1}{34}}

And the triangle $FED$ is half of the parallelogram $DEFH$;
for the diameter $DF$ bisects it.  \using{\prop{1}{34}}

[But the halves of equal things are equal to one another.]

Therefore the triangle $ABC$ is equal to the triangle $DEF$.

Therefore etc.
\end{proof}

\begin{notes}

On this proposition Proclus remarks (pp.~405—6) that Euclid seems to
him to have given in \prop{6}{1} one proof including all the four
theorems from \prop{1}{35} to \prop{1}{38}, and that most people had
failed to notice this.  When Euclid, he says, proves that triangles
and parallelograms of the same altitude have to one another the same
ratio as their bases, he simply proves all these propositions more
generally by the use of proportion; for of course to be of the same
altitude is equivalent to being in the same parallels.  It is true
that \prop{6}{1} generalises these propositions, but it must be
observed that it does not prove the propositions themselves, as
Proclus seems to imply; they ate in fact assumed in order to prove
\prop{6}{1}.

\subsection*{Comparison of areas of triangles of \prop{1}{34}}

The theorem already mentioned as given by Proclus on \prop{1}{34}
(pp.~340—4) is placed here by Heron, who also enunciates it more
clearly (an-Nairīzī, ed.\ Besthorn-Heiberg, pp.~155—161, ed.\ Curtze,
pp.~75—8).

\emph{If in two triangles two sides of the one be equal to two sides
  of the other respectively, and the angle of the one be greater than
  the angle of the other, namely the angles contained by the equal
  sides, then, (1)~if the sum of the two angles contained by the equal
  sides is equal to two right angles, the two triangles are equal to
  site another; (2)~if less than two right angles, the triangle which
  has the greater angle is also itself greater than the other; (3)~if
  greater than two right angles, the triangle which has the less angle
  is greater than the other triangle.}

\infig{propI_38a}

Let two triangles $ABC$, $DEF$ have the sides $AB$, $AC$ respectively
equal to $DE$,~$DF$.

(1)~First, suppose that the angles at $A$ and~$D$ in the triangles
$ABC$, $DEF$ are together equal to two right angles.

Heron's construction is now as follows.

Make the angle $EDG$ equal to the angle~$BAC$.

Draw $FH$ parallel to~$ED$ meeting $DG$ in~$H$.

Join $EH$.

Then, since the angles $BAC$, $EDF$ are equal to two right angles, the
angles $EDH$, $EDF$, equal to two right angles.

But so are the angles $EDH$, $DHF$.

Therefore the angles $EDF$, $DHF$ are equal.

And the alternate angles $EDF$, $DFB$ are equal. \using{\prop{1}{39}}

Therefore the angles $DHF$, $DFH$ are equal, and $DF$ equal
to~$DH$. \using{\prop{1}{6}}

Hence the two sides $ED$, $DH$ are equal to the two sides $BA$, $AC$;
and the included angles are equal.

Therefore the triangles $ABC$, $DEH$ are equal in all respects.

And the triangles $DEF$, $DEH$ between the same parallels are equal.
\using{\prop{1}{37}}

Therefore the triangles $ABC$, $DEF$ are equal.

[Proclus takes the construction of Eucl.\ \prop{1}{24}, i.e., he makes
  $DH$ equal to~$DF$ and then proves that $ED$, $FH$ are parallel.]

(2)~Suppose the angles $BAC$, $EDF$ together \emph{less} than two right
angles.

As before, make the angle $EDG$ equal to the angle $BAC$, draw $FH$
parallel to~$ED$, and join~$EH$.

\infig{propI_38b}

In this case the angles $EDH$, $EDF$ are together less than two right
angles, while the angles $EDH$, $DHF$ are equal to two right angles.
\using{\prop{1}{29}}

Hence the angle $EDF$, and therefore the angle $DFH$, is less than the
angle~$DHF$.

Therefore $DH$ is less than~$DF$. \using{\prop{1}{19}}

Produce $DH$ to~$G$ so that $DG$ is equal to $DF$ or~$AC$, and
join~$EG$.

Then the triangle $DEG$, which is equal to the triangle $ABC$, is
greater than the triangle $DEH$, and therefore greater than the
triangle $DEF$.

(3)~Suppose the angles $BAC$, $EDF$ together greater than two right
angles.

\infig{propI_38c}

We make the same construction in this case, and we prove in like
manner that the angle $DHF$ is less than the angle $DFH$, whence $DH$
is greater than $DF$ or~$AC$.

Make $DG$ equal to~$AC$, and join~$EG$.

It then follows that the triangle $DEF$ is greater than the triangle
$ABC$.

[In the second and third cases again Proclus starts from the
  construction in \prop{1}{24}, and proves, in the second case, that
  the parallel, $FH$, to $ED$ cuts $DG$ and, in the third case, that
  it cuts $DG$ produced.]

There is no necessity for Heron to take account of the position of~$F$
in relation to the side opposite~$D$.  For in the first and third
\infig{propI_38d} cases $F$ \emph{must} fall in the position in which
Euclid draws it in \prop{1}{24}, whatever be the relative lengths of
$AB$, $AC$.  In the second case the figure may be as annexed, but the
proof is the same, or rather the case needs no proof at all.

\end{notes}

\end{proposition}

\begin{proposition}
\label{prop:I_39}

\begin{statement}
Equal triangles which are on the same base and on the same side are
also in the same parallels.
\end{statement}

\begin{proof}

Let $ABC$, $DBC$ be equal triangles which are on the same
base $BC$ and on the same side of it;
[I say that they are also in the same parallels.]
And [For] let $AD$ be joined;
I say that $AD$ is parallel to~$BC$.

For, if not, let $AE$ be drawn through
the point~$A$ parallel to the straight line
$BC$ \using{\prop{1}{31}}
and let $EC$ be joined.

Therefore the triangle $ABC$ is equal
to the triangle $EBC$;
for it is on the same base $BC$ with it and in the same
arallels. \using{\prop{1}{37}}

But $ABC$ is equal to~$DBC$;
therefore $DBC$ is also equal to~$EBC$, \using{\rcn{1}}
the greater to the less: which is impossible.

Therefore $AE$ is not parallel to~$BC$.

Similarly we can prove that neither is any other straight
line except $AD$;
therefore $AD$ is parallel to~$BC$.

Therefore etc.
\end{proof}

\begin{annotations}

5.~\textbf{[I say that they are also in the same parallels.]}  Heiberg
has proved (\emph{Hermes}, \r38., 1903, p.~50) from a recently
discovered papyrus-fragment [\emph{Fayūm towns and their papyri},
  p.~50, No.~\r9.)\ that these words are an interpolation by some one
  who did not observe that the words ``And let $AD$ be joined'' are
  part of the \emph{setting-out} (\greek{ἔκθεσις}), but took them as
  belonging to the construction (\greek{κατασκευή}) and consequently
  thought that a \greek{διορισμός} or ``definition'' (of the thing to
  be proved) should precede.  The interpolator then altered ``And''
  into ``For'' in the next sentence.

\end{annotations}

\begin{notes}

This theorem is of course the \emph{partial} converse of \prop{1}{37}.
In \prop{1}{37} we have triangles which are (1)~on the same base,
(2)~in the same parallels, and the theorem proves (3)~that the
triangles are equal.  Here the hypothesis~(1) and the conclusion~(3)
are combined as hypotheses, and the conclusion is the hypothesis~(1)
of \prop{1}{37}, that the triangles are in the same parallels.  The
additional qualification in this proposition that the triangles must
be on the same side of the base is necessary because it is not, as in
\prop{1}{37}, involved in the other hypotheses.

Proclus (p.~407, 4—17) remarks that Euclid only converts \prop{1}{37}
and \prop{1}{38} relative to triangles, and omits the converses of
\prop{1}{35}, \prop*{1}{36} about parallelograms as unnecessary
because it is easy to see that the method would be the same, and
therefore the reader may properly be left to prove them for himself.

The proof is, as Proclus points out (p.~408, 5—21), equally easy on
the supposition that the assumed parallel $AE$ meets $BD$ or $CD$
produced beyond~$D$.

\end{notes}

\end{proposition}

\begin{proposition}
\label{prop:I_40}

\begin{statement}
Equal triangles which are on equal bases and on the same
side are also in the same parallels.
\end{statement}

\begin{proof}

Let $ABC$, $CDE$ be equal triangles on equal bases $BC$, $CE$ and on
the same side.

I say that they are also in the same parallels.

For let $AD$ be joined;
I say that $AD$ is parallel to~$BE$.

For, if not, let AF be drawn through $A$ parallel to~$BE$
\prop{1}{31}, and let $FE$ be joined.

Therefore the triangle $ABC$ is equal to the triangle $FCE$;
for they are on equal bases $BC$, $CE$ and in the same parallels
$BE$, $AF$. \using{\prop{1}{38}}

But the triangle $ABC$ is equal to the triangle $DCE$; therefore the
triangle $DCE$ is also equal to the triangle $FCE$, \using{\rcn{1}}
the greater to the less: which is impossible.  Therefore $AF$ is not
parallel to~$BE$.

Similarly we can prove that neither is any other straight
line except~$AD$;
therefore $AD$ is parallel to~$BE$.

Therefore etc.
\end{proof}

\begin{notes}
Heiberg has proved by means of the papyrus-fragment mentioned in the
last note that this proposition is an interpolation by some one who
thought that there should be a proposition following \prop{1}{39} and
related to it in the same way as \prop{1}{38} is related to
\prop{1}{37}, and \prop{1}{36} to \prop{1}{35}.
\end{notes}

\end{proposition}

\begin{proposition}
\label{prop:I_41}

\begin{statement}
If a parallelogram have the same base with a triangle and be in the
same parallels, the parallelogram is double of the triangle.
\end{statement}

\begin{proof}

For let the parallelogram $ABCD$ have the same base $BC$ with the
triangle $EBC$, and let it be in the same parallels $BC$, $AE$;

I say that the parallelogram $ABCD$ is double of the
triangle $BEC$.

For let $AC$ be joined.

Then the triangle $ABC$ is equal to
the triangle $EBC$;
for it is on the same base $BC$ with it
and in the same parallels $BC$, $AE$.
\using{\prop{1}{37}}

But the parallelogram $ABCD$ is double of the triangle $ABC$; for the
diameter $AC$ bisects it; \using{\prop{1}{34}} so that the
parallelogram $ABCD$ is also double of the triangle $EBC$.

Therefore etc.
\end{proof}

\begin{notes}

On this proposition Proclus (pp.~414, 15—415, 16), ``by way of
practice'' (\greek{γυμνασίας ἕνεκα}), considers the area of a
\emph{trapezium} (a quadrilateral with only one pair of opposite sides
parallel) in comparison with that of the triangles in the same
parallels and having the greater and less of the parallel sides of the
trapezium for bases respectively, and proves that the trapezium is
less than double of the former triangle and more than double of the
latter.

He next (pp.~415, 22–416, 14) proves the proposition that, \emph{If a
  triangle be formed by joining the middle point of either of the
  non-parallel sides to the extremities of the opposite side, the area
  of the trapezium is always double of that of the triangle}.

Let $ABCD$ be a trapezium in which $AD$, $BC$ are the parallel sides,
and $E$ the middle point of one of the non-parallel sides, say~$DC$.

Join $EA$, $EB$ and produce $BE$ to meet $AD$ produced in~$F$.

Then the triangles $BEC$, $FED$ have two angles
equal respectively, and one side $CE$ equal to one
side~$DE$;
therefore the triangles are equal in all respects.

Add to each the quadrilateral $ABED$;
therefore the trapezium $ABCD$ is equal to the triangle $ABF$,
that is, to twice the triangle $AEB$, since $BE$ is equal to~$EF$.
\using{\prop{1}{38}}

The three properties proved by Proclus may be combined in one enuncia-
tion thus:

\emph{If a triangle be formed by joining the middle point of one side
  of a trapezium to the extremities of the opposite side, the area of
  the trapezium is (1)~greater than, (2)~equal to, or (3)~less than,
  double the area of the triangle according as the side the middle
  point of which is taken is (1)~the greater of the parallel sides,
  (2)~either of the non-parallel sides, or (3)~the lesser of the
  parallel sides.}

\end{notes}

\end{proposition}

\begin{proposition}
\label{prop:I_42}

\begin{statement}
To construct, in a given rectilineal angle, a parallelogram equal to a
given triangle.
\end{statement}

\begin{proof}

Let $ABC$ be the given triangle, and $D$ the given rectilineal angle;
thus it is required to construct in the rectilineal angle~$D$ a
parallelogram equal to the triangle~$ABC$.

Let $BC$ be bisected at~$E$, and let $AE$ be joined; on the straight
line $EC$, and at the point~$E$ on it, let the angle $CEF$ be
constructed equal to the angle~$D$; \using{\prop{1}{23}} through $A$
let $AG$ be drawn parallel to $EC$, and \using{\prop{1}{31}} through
$C$ let $CG$ be drawn parallel to~$EF$,

Then $FECG$ is a parallelogram.

And, since $BE$ is equal to~$EC$,
the triangle $ABE$ is also equal to the triangle $AEC$,
for they are on equal bases $BE$, $EC$ and in the same parallels
$BC$, $AG$; \using{\prop{1}{38}}
therefore the triangle $ABC$ is double of the triangle~$AEC$.

But the parallelogram $FECG$ is also double of the triangle
$AEC$, for it has the same base with it and is in the same
parallels with it; \using{\prop{1}{41}}
therefore the parallelogram $FECG$ is equal to the
triangle $ABC$.

And it has the angle $CEF$ equal to the given angle~$D$.

Therefore the parallelogram $FECG$ has been constructed
equal to the given triangle $ABC$, in the angle $CEF$ which is
equal to~$D$.
\qef
\end{proof}

\end{proposition}

\begin{proposition}
\label{prop:I_43}

\begin{statement}
In any parallelogram the complements of the parallelograms about the
diameter are equal to one another.
\end{statement}

\begin{proof}

Let $ABCD$ be a parallelogram, and $AC$ its diameter;
and about $AC$ let $EH$, $FG$ be parallelograms, and $BK$, $KD$
the so-called complements;

I say that the complement $BK$ is equal to the complement~$KD$.

For, since $ABCD$ is a parallelogram, and $AC$ its diameter,
the triangle $ABC$ is equal to
to the triangle $ACD$. \using{\prop{1}{34}}

Again, since $EH$ is a parallelogram, and $AK$ is its diameter,
the triangle $AEK$ is equal to
the triangle $AHK$.

For the same reason
the triangle $KFC$ is also equal to $KGC$.
Now, since the triangle $AEK$ is equal to the triangle
$AHK$,
and $KFC$ to $KGC$,
the triangle $AEK$ together with $KGC$ is equal to the triangle
$AHK$ together with $KFC$. \using{\rcn{1}}

And the whole triangle $ABC$ is also equal to the whole~$ADC$;
therefore the complement $BK$ which remains is equal to the
complement $KD$ which remains. \using{\rcn{3}}

Therefore etc.
\end{proof}

\begin{annotations}

1.~\textbf{complements}, \greek{παραπληρώματα}, the figures put in to
fill up (interstices).

4.~\textbf{and about $AC$…}.  Euclid's phraseology here and in the
next proposition implies that the complements as well at the other
parallelograms are ``about'' the diagonal.  The words are here
\greek{περὶ δὲ τὴν ΑΓ παραλληλόγραμμα μὲν ἔστω τὰ ΕΘ, ΖΗ, τὰ δὲ
  λεγόμενα παραπληρώματα τὰ ΒΚ, ΚΔ}.  The expression ``the so-called
complements'' indicates that this technical use of
\greek{παραπληρώματα} was not new, though it might not be universally
known.

\end{annotations}

\begin{notes}

In the text of Proclus' commentary as we have it, the end of the note
on \prop{1}{41}, the whole of that on \prop{1}{42}, and the beginning
of that on \prop{1}{43} are missing.

Proclus remarks (p.~418, 15—20) that Euclid did not need to give a
formal definition of \emph{complement} because the name was simply
suggested by the facts; when once we have the two ``parallelograms
about the diameter,'' the complements are necessarily the areas
remaining over on each side of the diameter, which fill up the
complete parallelogram.  Thus (p.~417, 1 sqq.)\ the complements need
not be parallelograms.  They are so if the two ``parallelograms about
the diameter'' are formed by straight lines drawn through \emph{one
  point} of the diameter parallel to the sides of the original
parallelogram, but not otherwise.  If, as in the first of the
accompanying figures, the parallelograms have no common point, the
complements are five-sided figures as shown.  When the parallelograms
overlap, as in the second figure, Proclus regards the complements as
being the small parallelograms $FG$, $EH$.  But, if complements are
strictly the areas required to fill up the original parallelogram,
Proclus is inaccurate in describing $FG$, $EH$ as the complements.
The complements are really (1)~the parallelogram $FG$ \emph{minus} the
triangle $LMN$, and (2)~the parallelogram $EH$ \emph{minus} the
triangle $KMN$, respectively; the possibility that the respective
differences may be negative merely means the possibility that the sum
of the two parallelograms about the diameter may be together greater
than the original parallelogram.

In all the cases it is easy to show, as Proclus does, that the
complements are still equal.

\end{notes}

\end{proposition}

\begin{proposition}
\label{prop:I_44}

\begin{statement}
To a given straight line to apply, in a given rectilineal angle, a
parallelogram equal to a given triangle.
\end{statement}

\begin{proof}

Let $AB$ be the given straight line, $C$ the given triangle and $D$
the given rectilineal angle; thus it is required to apply to the given
straight line $AB$, in an angle equal to the angle~$D$, a
parallelogram equal to the given triangle~$C$.

Let the parallelogram $BEFG$ be constructed equal to the triangle~$C$,
in the angle $EBG$ which is equal to~$D$ \using*{\prop{1}{42}}; let it
be placed so that $BE$ is in a straight line with $AB$; let $FG$ be
drawn through to~$H$, and let $AH$ be drawn through~$A$ parallel to
either $BG$ or $EF$. \using{\prop{1}{31}}

Let $HB$ be joined.

\infig{propI_44}

Then, since the straight line $HF$ falls upon the parallels
is $AH$, $EF$,
the angles $AHF$, $HFE$ are equal to two right angles.
\using{\prop{1}{39}}

Therefore the angles $BHG$, $GFE$ are less than two right
angles;
and straight lines produced indefinitely from angles less than
two right angles meet; \using{\rpost{5}}
therefore $HB$, $FE$, when produced, wilt meet.

Let them be produced and meet at~$K$; through the point
$K$ let $KL$ be drawn parallel to either $EA$ or~$FH$, \using{\prop{1}{31}}
and let $HA$, $GB$ be produced to the points $L$, $M$.

hen $HLKF$ is a parallelogram,
$HK$ is its diameter, and $AG$, $ME$ are parallelograms, and
$LB$, $BF$ the so-called complements, about~$HK$;
therefore $LB$ is equal to~$BF$. \using{\prop{1}{43}}

But $BF$ is equal to the triangle~$C$;
therefore $LB$ is also equal to~$C$. \using{\rcn{1}}

And, since the angle $GBE$ is equal to the angle $ABM$,
\using{\prop{1}{15}}
while the angle $GBE$ is equal to~$D$,
the angle $ABM$ is also equal to the angle~$D$.

Therefore the parallelogram $LB$ equal to the given triangle~$C$ has
been applied to the given straight line $AB$, in the angle $ABM$ which
is equal to~$D$.
\qef
\end{proof}

\begin{annotations}

14.~\textbf{since the straight line $HF$ falls….}  The verb is in the
aorist (\greek{ἐνέπεσεν}) here and in similar expressions in the
following propositions.

\end{annotations}

\begin{notes}

This proposition wilt always remain one of the most impressive in all
geometry when account is taken (1)~of the great importance of the
result obtained, the transformation of a parallelogram of any shape
into another with the same angle and of equal area but with one side
of any given length, e.g.\ a \emph{unit} length, and (2)~of the
simplicity of the means employed, namely the mere application of the
property that the complements of the ``parallelograms about the
diameter'' of a parallelogram are equal.  The marvellous ingenuity of
the solution is indeed worthy of the ``godlike men of old,'' as
Proclus calls the discoverers of the method of ``application of
areas''; and there would seem to be no reason to doubt that the
particular solution, like the whole theory, was Pythagorean, and not a
new solution due to Euclid himself.

\subsection*{Application of areas}

On this proposition Proclus gives (pp.~419, 15—420, 23) a valuable
note on the method of ``application of areas'' here introduced, which
was one of the most powerful methods on which Greek geometry relied.
The note runs as follows:

``These things, says Eudemus (\greek{οἱ περὶ τὸν Εὔδημον}), are
ancient and are discoveries of the Muse of the Pythagoreans, I mean
the \emph{application of areas} (\greek{παραγολὴ τῶν χωρίων}) their
\emph{exceeding} (\greek{ὑπερβολή}) and their \emph{falling-short}
(\greek{ἔλλειψις}).  It was from the Pythagoreans that later geometers
     [i.e.\ Apollonius] took the names, which they again transferred
     to the so-called \emph{conic} lines, designating one of these a
     \emph{parabola} (application), another a \emph{hyperbola}
     (exceeding) and another an \emph{ellipse} (falling-short),
     whereas those godlike men of old saw the things signified by
     these names in the construction, in a plane, of areas upon a
     finite straight line.  For, when you have a straight line set out
     and lay the given area exactly alongside the whole of the
     straight line, then they say that you \emph{apply}
     (\greek{παραβάλλειν}) the said area; when however you make the
     length of the area greater than the straight line itself, it is
     said to \emph{exceed} (\greek{ὑπερβάλλειν}), and when you make it
     less, in which case, after the area has been drawn, there is some
     part of the straight line extending beyond it, it is said to
     \emph{fall short} (\greek{ἐλλείπειν}).  Euclid too, in the sixth
     book, speaks in this way both of \emph{exceeding} and
     \emph{falling-short}; but in this place he needed the application
     simply, as he sought to apply to a given straight line an area
     equal to a given triangle in order that we might have in our
     power, not only the \emph{construction} (\greek{σύστασις}) of a
     parallelogram equal to a given triangle, but also the
     \emph{application} of it to a finite straight line.  For example,
     given a triangle with an area of 12 feet, and a straight line set
     out the length of which is 4 feet, we apply to the straight line
     the area equal the triangle if we take the whole length of 4 feet
     and find how many feet the breadth must be in order that the
     parallelogram may be equal to the triangle.  In the particular
     case, if we find a breadth of 3 feet and multiply the length into
     the breadth, supposing that the angle set out is a right angle,
     we shall have the area.  Such then is the \emph{application}
     handed down from early times by the Pythagoreans.''

Other passages to a similar effect are quoted from Plutarch,
(1)~``Pythagoras sacrificed an ox on the strength of his proposition
(\greek{διάγραμμα}) as Apollodotus (?-rus) says…whether it was the
theorem of the hypotenuse, viz.\ that the square on it is equal to the
squares on the sides containing the right angle, or the problem about
the \emph{application of an area}.'' (\emph{Non posse suaviter vivi
  secundum Epicurum}, c.~\r2.) (2)~``Among the most geometrical
theorems, or rather problems, is the following: given two figures; to
\emph{apply} a third equal to the one and similar to the other, on the
strength of which discovery they say moreover that Pythagoras
sacrificed.  This is indeed unquestionably more subtle and more
scientific than the theorem which demonstrated that the square on the
hypotenuse is equal to the squares on the sides about the right angle
``(\emph{Symp.}~\r8.\ 2,~4).

The story of the sacrifice must (as noted by Bretschneider and Hankel)
be given up as inconsistent with Pythagorean ritual, which forbade
such sacrifices; but there is no reason to doubt that the first
distinct formulation and introduction into Greek geometry of the
method of \emph{application of areas} was due to the Pythagoreans.
The complete exposition of the \emph{application} of areas, their
\emph{exceeding} and their \emph{falling-short}, and of the
construction of a rectilineal figure equal to one given figure and
similar to another, takes us into the sixth Book of Euclid; but it
will be convenient to note here the general features of the theory of
\emph{application}, \emph{exceeding} and \emph{falling-short}.

The simple \emph{application} of a parallelogram of given area to a
given straight line as one of its sides is what we have in
\prop{1}{44} and \prop*{1}{45}; the general form of the problem with
regard to \emph{exceeding} and \emph{falling-short} may be stated
thus:

``To apply to a given straight line a rectangle (or, more generally, a
parallelogram) equal to a given rectilineal figure and
(1)~\emph{exceeding} or (2)~\emph{falling-short} by a square (or, in
the more general case, a parallelogram similar to a given
parallelogram).''

What is meant by saying that the applied parallelogram
(1)~\emph{exceeds} or (2)~\emph{falls short} is that, while its base
coincides and is coterminous \emph{at one end} with the straight line,
the said base (1)~overlaps or (2)~falls short of the straight line
\emph{at the other end}, and the portion by which the applied
parallelogram exceeds a parallelogram of the same angle and height on
the given straight line (exactly) as base is a parallelogram similar
to a given parallelogram (or, in particular cases, a square).  In the
case where the parallelogram is to \emph{fall short}, a
\greek{διορισμός} is necessary to express the condition of possibility
of solution.

We shall have occasion to see, when we come to the relative
propositions in the second and sixth Books, that the general problem
here stated is equivalent to that of solving geometrically a mixed
quadratic equation.  We shall see that, even by means of \prop{2}{5}
and~\prop*{2}{6}, we can solve geometrically the equations
\begin{align*}
ax \pm x^2 &= b^2,\\
 x^2 - ax  &= b^2
\end{align*}
but in \prop{6}{28}, \prop*{6}{29} Euclid gives the equivalent of the
solution of the general equations
\[
    ax \pm \frac{b}{c} x^2 = \frac{C}{m}.
\]

We are now in a position to understand the application of the terms
\emph{parabola} (application), \emph{hyperbola} (exceeding) and
\emph{ellipse} (falling-short) to conic sections.  These names were
first so applied by Apollonius as expressing in each case the
fundamental property of the curves as stated by him.  This fundamental
property is the geometrical equivalent of the Cartesian equation
referred to any diameter of the conic and the tangent at its extremity
as (in general, oblique) axes.  If the \emph{parameter} of the
ordinates from the several points of the conic drawn to the given
diameter be denoted by~$p$ ($p$ being accordingly, in the case of the
hyperbola and ellipse, equal to $d'^2/d$ , where $d$ is the length of
the given diameter and $d'$ that of its conjugate), Apollonius gives
the properties of the three conics in the following form.

(1)~For the \emph{parabola}, the square on the ordinate at any point
is equal to a rectangle applied to~$p$ as base with altitude equal to
the corresponding abscissa. That is to say, with the usual notation,
\[
    y^2 = px.
\]

(2)~For the \emph{hyperbola} and \emph{ellipse}, the square on the
ordinate is equal to the rectangle applied to~$p$ having as its width
the abscissa and \emph{exceeding} (for the hyperbola) or
\emph{falling-short} (for the ellipse) by a figure similar and
similarly situated to the rectangle contained by the given diameter
and~$p$.

That is, in the \emph{hyperbola}
\[
    y^2 = px + \frac{x^2}{d^2} pd,
\]
or
\[
    y^2 = px + \frac{p}{d} x^2,
\]
and in the \emph{ellipse}
\[
    y^2 = px - \frac{p}{d} x^2,
\]

The form of these equations will be seen to be exactly the same as
that of the general equations above given, and thus Apollonius'
nomenclature followed exactly the traditional theory of
\emph{application}, \emph{exceeding}, and \emph{falling-short}.

\end{notes}

\end{proposition}

\begin{proposition}
\label{prop:I_45}

\begin{statement}
To construct, in a given rectilineal angle, a parallelogram equal to a
given rectilineal figure.
\end{statement}

\begin{proof}

Let $ABCD$ be the given rectilineal figure and $E$ the given
rectilineal angle;\0
thus it is required to construct, in the given angle~$E$, a
parallelogram equal to the rectilineal figure $ABCD$.

\infig{propI_45}

Let $DB$ be joined, and let the parallelogram $FH$ be
constructed equal to the triangle $ABD$, in the angle $HKF$
which is equal to~$E$; \using{\prop{1}{42}}
let the parallelogram $GM$ equal to the triangle $DBC$ be
applied to the straight line $GH$, in the angle $GHM$ which is
equal to~$E$. \using{\prop{1}{44}}

Then, since the angle~$E$ is equal to each of the angles
$HKF$, $GHM$,
the angle $HKF$ is also equal to the angle $GHM$. \using{\rcn{1}}

Let the angle $KHG$ be added to each;
therefore the angles $FKH$, $KHG$ are equal to the angles
$KHG$, $GHM$.

But the angles $FKH$, $KHG$ are equal to two right angles;
\using{\prop{1}{29}}
therefore the angles $KHG$, $GHM$ are also equal to two right
angles.

Thus, with a straight line $GH$, and at the point~$H$ on it,
two straight lines $KH$, $HM$ not lying on the same side make
the adjacent angles equal to two right angles;
therefore $KH$ is in a straight line with~$HM$. \using{\prop{1}{14}}

And, since the straight line $HG$ falls upon the parallels
$KM$, $FG$, the alternate angles $MHG$, $HGF$ are equal to one
another. \using{\prop{1}{39}}

Let the angle $HGL$ be added to each;
therefore the angles $MHG$, $HGL$ are equal to the angles
$HGF$, $HGL$. \using{\rcn{2}}

But the angies $MHG$, $HGL$ are equal to two right angles;
therefore the angles $HGF$, $HGL$ are also equal to two right
angles. \using{\rcn{1}}

Therefore $FG$ is in a straight line with $GL$. \using{\prop{1}{14}}

And, since $FK$ is equal and parallel to $HG$, \using{\prop{1}{34}}
and $HG$ to $ML$ also,
$KF$ is also equal and parallel to~$ML$; \using{\rcn{1}; \prop{1}{30}}
and the straight lines $KM$, $FL$ join them (at their extremities);
therefore $KM$, $FL$ are also equal and parallel. \using{\prop{1}{33}}

Therefore $KFLM$ is a parallelogram.

And, since the triangle $ABD$ is equal to the parallelogram
$FH$ and $DBC$ to~$GM$,
the whole rectilineal figure $ABCD$ is equal to the whole
parallelogram $KFLM$.

Therefore the parallelogram $KFLM$ has been constructed equal to the
given rectilineal figure $ABCD$, in the angle $FKM$ which is equal to
the given angle~$E$.  \qef
\end{proof}

\begin{annotations}

2, 3, 6, 45. 48.  \textbf{rectilineal figure}, in the Greek
``rectilineal'' simply, without ``figure,'' \greek{εὐθύγραμμον} being
here used as a substantive, like the similarly formed
\greek{παραλληλόγραμμον}.

\end{annotations}

\begin{notes}

\subsection*{Transformation of areas}

We can now take stock of how far the propositions
\prop{1}{43}—\prop*{1}{45} bring us in the matter of
\emph{transformation of areas}, which constitutes so important a part
of what has been fitly called the \emph{geometrical algebra} of the
Greeks.  We have now learnt how to represent any rectilineal area,
which can of course be resolved into triangles, by a single
parallelogram having one side equal to any given straight line and one
angle equal to any given rectilineal angle.  Most important of all
such parallelograms is the rectangle, which is one of the simplest
forms in which an area can be shown.  Since a rectangle corresponds to
the product of two magnitudes in algebra, we see that
\emph{application} to a given straight line of a rectangle equal to a
given area is the geometrical equivalent of algebraical
\emph{division} of the product of two quantities by a third.  Further
than this, it enables us to \emph{add} or \emph{subtract} any
rectilineal areas and to represent the sum or difference by \emph{one}
rectangle with one side of any given length, the process being the
equivalent of obtaining a common factor.  But one step still remains,
the finding of a \emph{square} equal to a given rectangle, i.e.\ to a
given rectilineal figure; and this step is not taken till
\prop{2}{14}.  In general, the transformation of combinations of
rectangles and squares into other combinations of rectangles and
squares is the subject-matter of Book~\r2., with the exception of the
expression of the sum of two squares as a single square which appears
earlier in the other Pythagorean theorem \prop{1}{47}.  Thus the
transformation of rectilineal areas is made complete \emph{in one
  direction}, i.e.\ in the direction of their simplest expression in
terms of rectangles and squares, by the end of Book~\book{2}.  The
reverse process of transforming the simpler rectangular area into an
equal area which shall be similar to any rectilineal figure requires,
of course, the use of proportions, and therefore does not appear till
\prop{6}{25}.

Proclus adds to his note on this proposition the remark (pp.~422,
24—413, 6): ``I conceive that it was in consequence of this problem
that the ancient geometers were led to investigate the squaring of the
circle as well.  For, if a parallelogram can be found equal to any
rectilineal figure, it is worth inquiring whether it be not also
possible to prove rectilineal figures equal to circular.  And
Archimedes actually proved that any circle is equal to the
right-angled triangle which has one of its sides about the right angle
[the perpendicular] equal to the radius of the circle and its base
equal to the perimeter of the circle.  But of this elsewhere.''

\end{notes}

\end{proposition}

\begin{proposition}
\label{prop:I_46}

\begin{statement}
On a given straight line to describe a square.
\end{statement}

\begin{proof}

Let $AB$ be the given straight line;
thus it is required to describe a square
on the straight line $AB$.

Let $AC$ be drawn at right angles to the straight line $AB$ from the
point~$A$ on it \using*{\prop{1}{11}}, and let $AD$ be made equal to
$AB$; through the point $D$ let $DE$ be drawn parallel to~$AB$,
and through the point $B$ let $BE$ be drawn parallel to~$AD$.
\using{\prop{1}{31}}

Therefore $ADEB$ is a parallelogram;
therefore $AB$ is equal to $DE$, and $AD$ to $BE$. \using{\prop{1}{34}}

But $AB$ is equal to $AD$; therefore the four straight lines $BA$,
$AD$, $DE$, $EB$ are equal to one another; therefore the parallelogram
$ADEB$ is equilateral.

I say next that it is also right-angled.

For, since the straight line $AD$ falls upon the parallels $AB$, $DE$,
the angles $BAD$, $ADE$ are equal to two right angles.
\using{\prop{1}{29}}

But the angle $BAD$ is right;
therefore the angle $ADE$ is also right.

And in parallelogrammic areas the opposite sides and
as angles are equal to one another; \using{\prop{1}{34}}
therefore each of the opposite angles $ABE$, $BED$ is also
right.

Therefore $ADEB$ is right-angled.

And it was also proved equilateral.

Therefore it is a square; and it is described on the straight
line~$AB$.
\qef
\end{proof}

\begin{annotations}

1, 3, 30. Proclus (p.~423, 18 sqq.)\ notes the difference between the
word \emph{construct} (\greek{συστήσασθαι}) applied by Euclid to the
construction of a \emph{triangle} (and, he might have added, of an
\emph{angle}) and the words \emph{describe on} (\greek{ἀναγράφειν
  ἀπό}) used of drawing a square on a given straight line as one side.
The \emph{triangle} (or \emph{angle}) is, so to say, pieced together,
while the describing of a square on a given straight line is the
making of a figure ``from'' \emph{one} side, and corresponds to the
multiplication of the number representing the side by itself.

\end{annotations}

\begin{notes}

Proclus (pp.~424–5) proves that, \emph{if squares are described on
  equal straight lines, the squares are equal}; and, conversely that,
\emph{if two squares are equal, the straight lines are equal on which
  they are described}.  The first proposition is immediately obvious
if we divide the squares into two triangles by drawing a diagonal in
each.  The converse is proved as follows.

Place the two equal squares $AF$, $CG$ so that $AB$, $BC$ are in a
straight line.  Then, since the angles are right, $FB$, $BG$ will also
be in a straight line.  Join $AF$, $FC$, $CG$, $GA$.

Now, since the squares are equal, the triangles $ABF$, $CBG$ are
equal.

Add to each the triangle $FBC$; therefore the triangles $AFC$, $GFC$
are equal, and hence they must be in the same parallels.

Therefore $AG$, $CF$ are parallel.

Also, since each of the alternate angles $AFG$, $FGC$ is half a right
angle, $AF$, $CG$ are parallel.

Hence $AFCG$ is a parallelogram; and $AF$, $CG$ are equal.

Thus the triangles $ABF$, $CBG$ have two angles and one side
respectively equal; therefore $AB$ is equal to $BC$, and $BF$ to~$BG$.

\end{notes}

\end{proposition}

\begin{proposition}
\label{prop:I_47}

\begin{statement}
In right-angled triangles the square on the side subtending the right
angle is equal to the squares on the sides containing the right angle.
\end{statement}

\begin{proof}

Let $ABC$ be a right-angled triangle having the angle $BAC$ right;

I say that the square on $BC$ is equal to the squares on $BA$, $AC$.

For let there be described
on $BC$ the square $BDEC$,
and on $BA$, $AC$ the squares
$GB$, $HC$; \using{\prop{1}{46}}
through~$A$ let $AL$ be drawn
parallel to either $BD$ or~$CE$,
and let $AD$, $FC$ be joined.

Then, since each of the
angles $BAC$, $BAG$ is right,
it follows that with a straight
line $BA$, and at the point~$A$
on it, the two straight lines
$AC$, $AG$ not lying on the
same side make the adjacent
angles equal to two right
angles;
therefore $CA$ is in a straight line with~$AG$.

For the same reason
$BA$ is also in a straight line with $AH$.

And, since the angle $DBC$ is equal to the angle $FBA$: for each is
right:
let the angle $ABC$ be added to each;
therefore the whole angle $DBA$ is equal to the whole
angle $FBC$. \using{\rcn{2}}

And, since $DB$ is equal to~$BC$, and $FB$ to~$BA$, the two sides
$AB$, $BD$ are equal to the two sides $FB$, $BC$ respectively, and the
angle $ABD$ is equal to the angle $FBC$; therefore the base $AD$ is
equal to the base~$FC$, and the triangle $ABD$ is equal to the
triangle~$FBC$. \using{\prop{1}{4}} Now the parallelogram $BL$ is double
of the triangle $ABD$, for they have the same base $BD$ and are in the
same parallels $BD$, $AL$. \using{\prop{1}{41}}

And the square $GB$ is double of the triangle $FBC$, for they again
have the same base $FB$ and are in the same parallels $FB$,
$GC$. \using{\prop{1}{41}}

[But the doubles of equals are equal to one another,] Therefore the
parallelogram $BL$ is also equal to the square~$GB$.

Similarly, if $AE$, $BK$ be joined,
the parallelogram $CL$ can also be proved equal to the square~$HC$;
therefore the whole square $BDEC$ is equal to the two
squares $GB$, $HC$. \using{\rcn{2}}

And the square $BDEC$ is described on~$BC$,
and the squares $GB$, $HC$ on $BA$, $AC$.

Therefore the square on the side $BC$ is equal to the squares on the
sides $BA$, $AC$.

Therefore etc.
\end{proof}

\begin{annotations}

1. \textbf{the square on}, \greek{τὸ ἀπὸ…τετράγωνον}, the word
\greek{ἀναγραφέν} or \greek{ἀναγεγραμμένον} being understood.

\textbf{subtending the right angle}.  Here \greek{ὑποτεινούσης},
``subtending,'' is used with the simple accusative (\greek{τὴν ὀρθὴν
  γωνίαν}) instead of being followed by \greek{ὑπό} and the
accusative, which seems to be the original and more orthodox
construction. Cf.~\prop{1}{18}, note.

33. \textbf{the two sides $AB$, $BD$….}  Euclid actually writes
``$DB$, $BA$,'' and therefore the equal sides in the two triangles are
not mentioned in corresponding order, though be adheres to the words
\greek{ἐκατέρα ἐκατέρα} ``respectively.''  Here $DB$ is equal to $BC$
and $BA$ to~$FB$.

44.~\emph{[But the doubles of equals are equal to one another.]}
Heiberg brackets these words as an interpolation, since it quotes a
\emph{Common Notion} which is itself interpolated.  Cf.\ notes on
\prop{1}{37}, p.~337, and on interpolated \emph{Common Notions},
pp.~223—4.

\end{annotations}

\begin{notes}

``If we listen,'' says Proclus (p.~426, 6 sqq.), ``to those who wish
  to recount ancient history, we may find some of them referring this
  theorem to Pythagoras and saying that he sacrificed an ox in honour
  of his discovery.  But for my part, while I admire those who first
  observed the truth of this theorem, I marvel more at the writer of
  the Elements, not only because he made it fast (\greek{κατεδήσατο})
  by a most lucid demonstration, but because he compelled assent to
  the still more general theorem by the irrefragable arguments of
  science in the sixth Book.  For in that Book he proves generally
  that, in right-angled triangles, the figure on the side subtending
  the right angle is equal to the similar and similarly situated
  figures described on the sides about the right angle.''

In addition, Plutarch (in the passages quoted above in the note on
\prop{1}{44}), Diogenes Laertius (\r8.~12) and Athenaeus (\r10.~13)
agree in attributing this proposition to Pythagoras.  It is easy to
point out, as does G.~Junge (``Wann haben die Griechen das Irrationale
entdeckt?'' in \emph{Novae Symbolae Joachimicae}, Halle a.~S., 1907,
pp.~221—264), that these are late witnesses, and that the Greek
literature which we possess belonging to the first five centuries
after Pythagoras contains no statement specifying this or any other
particular great geometrical discovery as due to him.  Yet the distich
of Apollodorus the ``calculator,'' whose date (though it cannot be
fixed) is at least earlier than that of Plutarch and presumably of
Cicero, is quite definite as to the existence of \emph{one} ``famous
proposition'' discovered by Pythagoras, whatever it was.  Nor does
Cicero, in commenting apparently on the verses (\emph{De
  nat.\ deor.}\ \r3.\ c.~36, §88), seem to dispute the fact of the
geometrical discovery, but only the story of the sacrifice.  Junge
naturally emphasises the apparent uncertainty in the statements of
Plutarch and Proclus.  But, as I read the passages of Plutarch, I see
nothing in them inconsistent with the supposition that Plutarch
unhesitatingly accepted as discoveries of Pythagoras \emph{both} the
theorem of the square of the hypotenuse and the problem of the
application of an area, and the only doubt he felt was as to which of
the two discoveries was the more appropriate occasion for the supposed
sacrifice.  There is also other evidence not without bearing on the
question.  The theorem is closely connected with the whole of the
matter of Eucl.\ Book~\r2., in which one of the most prominent
features is the use of the \emph{gnomon}.  Now the gnomon was a
well-understood term with the Pythagoreans (cf.\ the fragment of
Philolaus quoted on p.~141 of Boeckh's \emph{Philolaos des
  Pythagoreers Lehren}, 1819).  Aristotle also (\emph{Physics} \r3.~4,
203~a~10—15) clearly attributes to the Pythagoreans the placing of odd
numbers as \emph{gnomons} round successive squares beginning with~1,
thereby forming new squares, while in another place
(\emph{Categ.}\ 14, 15~a~30) the word \emph{gnomon} occurs in the same
(obviously familiar) sense: ``e.g.\ a square, when a gnomon is placed
round it, is increased in size but is not altered in form.''  The
inference must therefore be that practically the whole doctrine of
Book~\r2.\ is Pythagorean.  Again Heron (?3rd cent.~\ad), like
Proclus, credits Pythagoras with a general rule for forming
right-angled triangles with rational whole numbers for sides.  Lastly,
the ``summary'' of Proclus appears to credit Pythagoras with the
discovery of the theory, or study, of irrationals (\greek{τὴν τῶν
  ἀλόγων πραγματείαν}).  But it is now more or less agreed that the
reading here should be, not \greek{τῶν ἀλόγων}, but \greek{τῶν
  ἀναλόγων}, or rather \greek{τῶν ἀνὰ λόγον} (``of proportionals''),
and that the author intended to attribute to Pythagoras a theory of
\emph{proportion}, i.e.\ the (arithmetical) theory of proportion
applicable only to commensurable magnitudes, as distinct from the
theory of Eucl.\ Book~\book{5}, which was due to Eudoxus.  It is not
however disputed that the \emph{Pythagoreans} discovered the
irrational (cf.\ the scholium No.~1 to Book~\book{10}).  Now
everything goes to show that this discovery of the irrational was made
with reference to~$\sqrt{2}$, the ratio of the diagonal of a square to
its side.  It is clear that this presupposes the knowledge that
\prop{1}{47} is true of an isosceles right-angled triangle; and the
fact that some triangles of which it had been discovered to be true
were \emph{rational} right-angled triangles was doubtless what
suggested the inquiry whether the ratio between the lengths of the
diagonal and the side of a square could also be expressed in whole
numbers.  On the whole, therefore, I see no sufficient reason to
question the tradition that, \emph{so far as Greek geometry is
  concerned} (the possible priority of the discovery of the same
proposition in India will be considered later), Pythagoras was the
first to introduce the theorem of \prop{1}{47} and to give a general
proof of it.

On this assumption, how was Pythagoras led to this discovery?  It has
been suggested and commonly assumed that the Egyptians were aware that
a triangle with its sides in the ratio 3, 4, 5 was right-angled.
Cantor inferred this from the fact that this was precisely the
triangle with which Pythagoras began, if we may accept the testimony
of Vitruvius (\r9.~2) that Pythagoras taught how to make a right angle
by means of three lengths measured by the numbers 3, 4, 5. If then he
took from the Egyptians the triangle 3, 4, 5, he presumably learnt its
property from them also.  Now the Egyptians must certainly be credited
from a period at least as far back as 2000~\bc\ with the knowledge
that $4^2 + 3^2 = 5^2$.  Cantor finds proof of this in a fragment of
papyrus belonging to the time of the 12th Dynasty newly discovered at
Kahun.  In this papyrus we have extractions of square roots:
e.g.\ that of 16 is 4, that of $1\frac{9}{16}$ is $1\frac{1}{4}$, that
of $6\frac{1}{2}$ is $2\frac{1}{2}$, and the following equations can
be traced:
\begin{align*}
    1^2 + \left(\frac{3}{4}\right)^2  &= \left(1\frac{1}{4}\right)^2\\[\jot]
    8^2 + 6^2                         &= 10^2\\[\jot]
    2^2 + \left(1\frac{1}{2}\right)^2 &= \left(2\frac{1}{2}\right)^2\\[\jot]
    16^2 + 12^2                       &= 20^2
\end{align*}
It will be seen that $4^2 + 3^2 = 5^2$ can be derived from each of
these by multiplying, or dividing out, by one and the same factor.  We
may therefore admit that the Egyptians knew that $3^2 + 4^2 = 5^2$.
But there seems to be no evidence that they knew that the triangle (3,
4, 5) is \emph{right-angled}; indeed, according to the latest
authority (T.~Eric Peet, \emph{The Rhind Mathematical Papyrus}, 1923),
nothing in Egyptian mathematics suggests that the Egyptians were
acquainted with this or any special cases of the Pythagorean theorem.

How then did Pythagoras discover the general theorem?  Observing that
3, 4, 5 was a right-angled triangle, while $3^2 + 4^2 = 5^2$, he was
probably led to
%
\infig{XXX}
%
consider whether a similar relation was true of the sides of
right-angled triangles other than the particular one.  The simplest
case (geometrically) to investigate was that of the \emph{isosceles}
right-angled triangle; and the truth of the theorem in this particular
case would easily appear from the mere construction of a figure.
Cantor (\r1\tsub{3}, p.~185) and Allman (\emph{Greek Geometry from
  Thales to Euclid}, p.~29) illustrate by a figure in which the
squares are drawn outwards, as in \prop{1}{47}, and divided by
diagonals into equal triangles; but I think that the truth was more
likely to be first observed from a figure of the kind suggested by
Bürk (\emph{Das Āpastamba-Śulba-Sūtra} in \emph{Zeitschrift der deuts
  hen morgenländ}.  \emph{Gesellschaft}, \r40., 1901, p.~557) to
explain how the Indians arrived at the same thing.  The two figures
are as shown above.  When the geometrical consideration of the figure
had shown that the isosceles right-angled triangle had the property in
question, the investigation of the same fact from the arithmetical
point of view would ultimately lead to the other momentous discovery
of the irrationality of the length of the diagonal of a square
expressed in terms of its side.

The \emph{irrational} will come up for discussion later; and our next
question is: Assuming that Pythagoras had observed the geometrical
truth of the theorem in the case of the two particular triangles, and
doubtless of other rational right-angled triangles, how did he
establish it generally?

There is no positive evidence on this point.  Two possible lines are
however marked out.  (1)~Tannery says (\emph{La Géométrie grecqe},
p.~105) that the geometry of Pythagoras was sufficiently advanced to
make it possible for him to prove the theorem by \emph{similar
  triangles}.  He does not say in what particular manner similar
triangles would be used, but their use must apparently have involved
the use of \emph{proportions}, and, in order that the proof should be
conclusive, of the theory of proportions in its complete form
applicable to incommensurable as well as commensurable magnitudes.
Now Eudoxus was the first to make the theory of proportion independent
of the hypothesis of commensurability; and as, before Eudoxus' time,
this had not been done, any proof of the general theorem by means of
proportions given by Pythagoras must at least have been inconclusive.
But this does not constitute any objection to the supposition that the
truth of the general theorem may have been discovered in such a
manner; on the contrary, the supposition that Pythagoras proved it by
means of an imperfect theory of proportions would better than anything
else account for the fact that Euclid had to devise an entirely new
proof, as Proclus says he did in \prop{1}{47}.  This proof had to be
independent of the theory of proportion even in its rigorous form,
because the plan of the \emph{Elements} postponed that theory to Books
\book{5} and~\book{6}, while the Pythagorean theorem was required as
early as Book~\book{2}.  On the other hand, if the Pythagorean proof
had been based on the doctrine of Books \book{1} and \book{2} only, it
would scarcely have been necessary for Euclid to supply a new proof.

The possible proofs by means of proportion would seem to be
practically limited to two.

(\emph{a})~One method is to prove, from the similarity of the
triangles $ABC$, $DBA$, that the rectangle $CB$, $BD$ is equal to the
square on $BA$, and, from the similarity of the triangles $ABC$,
$DAC$, that the rectangle $BC$, $CD$ is equal to the square on~$CA$;
whence the result follows by addition.

It will be observed that this proof is \emph{in substance} identical
with that of Euclid, the only difference being that the equality of
the two smaller squares to the respective rectangles is inferred by
the method of Book~\book{6} instead of from the relation between the
areas of parallelograms and triangles on the same base and between the
same parallels established in Book~\book{1}.  It occurred to me
whether, if Pythagoras' proof had come, even in substance, so near to
Euclid's, Proclus would have emphasised so much as he does the
originality of Euclid's, or would have gone so far as to say that he
marvelled more at that proof than at the original discovery of the
theorem.  But on the whole I see no difficulty; for there can be
little doubt that the proof by proportion is what suggested to Euclid
the method of \prop{1}{47}, and the transformation of the method of
proportions into one based on Book~\book{1} only, effected by a
construction and proof so extraordinarily ingenious, is a veritable
\emph{tour de force} which compels admiration, notwithstanding the
ignorant strictures of Schopenhauer, who wanted something as obvious
as the second figure in the case of the isosceles right-angled
triangle (p.~353), and accordingly (\emph{Sämmtliche Werke}, \r3.\ §39
and \r1.~§15) calls Euclid's proof ``a mouse-trap proof'' and ``a
proof walking on stilts, nay, a mean, underhand, proof'' (``Des
Eukleides stelzbeiniger, ja, hinterlistiger Beweis'').

(\emph{b})~The other possible method is this.  As it would be seen
that the triangles into which the original triangle is divided by the
perpendicular from the right angle on the hypotenuse are similar to
one another and to the whole triangle, while in these three triangles
the two sides about the right angle in the original triangle, and the
hypotenuse of the original triangle, are corresponding sides, and that
the sum of the two former similar triangles is identically equal to
the simitar triangle on the hypotenuse, it might be inferred that the
same would also be true of squares described on the corresponding
three sides respectively, because squares as well as similar triangles
are to one another in the duplicate ratio of corresponding sides.  But
the same thing is equally true of any similar rectilineal figures, so
that this proof would practically establish the extended theorem of
Eucl.\ \prop{6}{31}, which theorem, however, Proclus appears to regard
as being entirely Euclid's discovery.

On the whole, the most probable supposition seems to me to be that
Pythagoras used the first method (\emph{a}) of proof by means of the
theory of proportion as he knew it, i.e.\ in the defective form which
was in use up to the date of Eudoxus.

(2)~I have pointed out the difficulty in the way of the supposition
that Pythagoras' proof depended upon the principles of Eucl.\ Books
\book{1} and~\book{2} only.
%
\infig{XXX}
%
Were it not for this difficulty, the conjecture of Bretschneider
(p.~82), followed by Hankel (p.~98), would be the most tempting
hypothesis.  According to this suggestion, we are to suppose a figure
like that of Eucl.\ \prop{2}{4} in which $a$, $b$ are the sides of the
two inner squares respectively, and $a + b$ is the side of the
complete square.  Then, if the two complements, which are equal, are
divided by their two diagonals into four equal triangles of sides $a$,
$b$, $c$, we can place these triangles round another square of the
same size as the whole square, in the manner shown in the second
figure, so that the sides $a$, $b$ of \emph{successive} triangles make
up one of the sides of the square and are arranged in cyclic order.
It readily follows that the remainder of the square when the four
triangles are deducted is, in the one case, a square whose side
is~$c$, and in the other the sum of two squares whose sides are $a$,
$b$ respectively.  Therefore the square on~$c$ is equal to the sum of
the squares on $a$, $b$.  All that can be said against this
conjectural proof is that it has no specifically Greek colouring but
rather recalls the Indian method.  Thus Bhāskara (born 1114~\ad; see
Cantor, \r1\tsub{3}, p.~656) simply draws four right-angled triangles
equal to the original one inwards, one on each side of the square on
the hypotenuse, and says ``see!'', without even adding that inspection
shows that
\[
    c^2 = 4 \frac{ab}{2} + (a - b)^2 = a^2 + b^2.
\]

Though, for the reason given, there is difficulty in supposing that
Pythagoras used a general proof of this kind, which applies of course
to right-angled triangles with sides incommensurable as well as
commensurable, there is no objection, I think, to supposing that the
truth of the proposition in the case of the first \emph{rational}
right-angled triangles discovered, e.g.\ 3, 4, 5, was proved by a
method of this sort.  Where the sides are commensurable in this way,
the squares can be divided up into small (unit) squares, which would
much facilitate the comparison between them.  That this subdivision
was in fact resorted to in adding and subtracting squares is made
probable by Aristotle's allusion to odd numbers as \emph{gnomons}
placed round unity to form successive squares in \emph{Physics}
\r3.~4; this must mean that the squares were represented by dots
arranged in the form of a square and a gnomon formed of dots put
round, or that (if the given square was drawn in the usual way) the
gnomon was divided up into unit squares.  Zeuthen has shown
(``\emph{Théorème de Pythagore'' Origine de la Géométrie scientifique}
in \emph{Comptes rendus du II\tsup{me} Congrès international de
  Philosophie}, Genève, 1904), how easily the proposition could be
proved by a method of this kind for the triangle 3, 4, 5.  To admit of
the two smaller squares being shown side by side, take a square on a
line containing 7 units of length $(4 + 3)$, and divide it up into 49
small squares.  It would be obvious that the whole square could be
exhibited as containing four rectangles of sides 4, 3 cyclically
arranged round the figure with one unit square in the middle. (This
same figure is given by Cantor, \r1\tsub{3}, p.~680, to illustrate the
method given in the Chinese ``Chóu-pei''.)  It would be seen that

(i)~the. whole square ($7^2$) is made up of two squares $(3^2)$ and
$(4^2)$, and two rectangles 3, 4;

(ii)~the same square is made up of the square $EFGH$ and the halves of
four of the same rectangles 3, 4) whence the square $EFGH$, being
equal to the sum of the squares $3^2$ and $4^2$, must contain 25 unit
squares and its side, or the diagonal of one of the rectangles, must
contain 5 units of length.

Or the result might equally be seen by observing that

(i)~the square $EFGH$ on the diagonal of one of the rectangles is made
up of the halves of four rectangles and the unit square in the middle,
while

(ii)~the squares $3^2$ and $4^@$ placed at adjacent comers of the
large square make up two rectangles 3, 4 with the unit square in the
middle.

The procedure would be equally easy for any \emph{rational}
right-angled triangle, and would be a natural method of trying to
\emph{prove} the property when it had once been \emph{empirically}
observed that triangles like 3, 4, 5 did in fact contain a right
angle.

Zeuthen has, in the same paper, shown in a most ingenious way how the
property of the triangle 3, 4, 5 could be verified by a sort of
combination of the second possible method by similar triangles,
(\emph{b}) on p.~\pageref{354} above, with subdivision of rectangles
into similar small rectangles.  I give the method on account of its
interest, although it is no doubt too advanced to have been used by
those who first proved the property of the particular triangle.

Let $ABC$ be a triangle right-angled at~$A$, and such that the lengths
of the sides $AB$, $AC$ are 4 and 3 units respectively.

Draw the perpendicular $AD$, divide up $AB$, $AC$ into unit lengths,
complete the rectangle on $BC$ as base and with $AD$ as altitude, and
subdivide this rectangle into small rectangles by drawing parallels to
$BC$, $AD$ through the points of division of $AB$,~$AC$.

Now, since the diagonals of the small rectangles are all equal, each
being of unit length, it follows by similar triangles that the small
rectangles are all equal.  And the rectangle with $AB$ for diagonal
contains 16 of the small rectangles, while the rectangle with diagonal
$AC$ contains 9 of them.

But the sum of the triangles $ABD$, $ADC$ is equal to the
triangle~$ABC$.

Hence the rectangle with $BC$ as diagonal contains $9+16$ or 25 of the
small rectangles; and therefore $BC= 5$.

\subsection*{Rational right-angled triangles from the arithmetical standpoint}

Pythagoras investigated the \emph{arithmetical} problem of finding
rational numbers which could be made the sides of right-angled
triangles, or of finding square numbers which are the sum of two
squares; and herein we find the beginning of the \emph{indeterminate
  analysis} which reached so high a stage of development in
Diophantus.  Fortunately Proclus has preserved Pythagoras' method of
solution in the following passage (pp.~428, 7—429, 8). ``Certain
methods for the discovery of triangles of this kind are handed down,
one of which they refer to Plato, and another to Pythagoras.  [The
  latter] starts from odd numbers.  For it makes the odd number the
smaller of the sides about the right angle; then it takes the square
of it, subtracts unity, and makes half the difference the greater of
the sides about the right angle; lastly it adds unity to this and so
forms the remaining side, the hypotenuse.  For example, taking 3,
squaring it, and subtracting unity from the 9, the method takes half
of the 8, namely 4; then, adding unity to it again, it makes 5, and a
right-angled triangle has been found with one side 3, another 4 and
another 5.  But the method of Plato argues from even numbers.  For it
takes the given even number and makes it one of the sides about the
right angle; then, bisecting this number and squaring the half, it
adds unity to the square to form the hypotenuse, and subtracts unity
from the square to form the other side about the right angle.  For
example, taking 4, the method squares half of this, or 2, and so makes
4; then, subtracting unity; it produces 3, and adding unity it
produces 5.  Thus it has formed the same triangle as that which was
obtained by the other method.''

The formula of Pythagoras amounts, if $m$ be an odd number, to
\[
    m^2 + \left(\frac{m^2 - 1}{2}\right)^2 = \left(\frac{m^2 + 1}{2}\right)^2
\]
the sides of the right-angled triangle being $m$, $\frac{m^2 - 1}{2}$,
$\frac{m^2 - 1}{2}$.  Cantor (\r1\tsub{3}, pp.~185—6), taking up an
idea of Röth (\emph{Geschichte der abendländischen Philosophie},
\r2.\ 527), gives the following as a possible explanation of the way
in which Pythagoras arrived at his formula.  If $c^2 = a^2 + b^2$, it
follows that
\[
    a^2 = c^2 - b^2 = (c + b)(c - b)
\]
Numbers can be found satisfying the first equation if (1)~$c + b$ and
$c — b$ are either both even or both odd, and if further (2)~$c + b$
and $c — b$ are such numbers as, when multiplied together, produce a
square number.  The first condition is necessary because, in order
that $c$ and~$b$ may both be whole numbers, the sum and difference of
$c + b$ and $c - b$ must both be even.  The second condition is
satisfied if $c + b$ and $c-b$ are what were called \emph{similar
  numbers} (\greek{ὅμοιοι ἀριθμοί}); and that such numbers were most
probably known in the time before Plato may be inferred from their
appearing in Theon of Smyrna (\emph{Expositio rerum mathematicarum ad
  legendum Platonem utilium}, ed.\ Hiller, p.~36, 12), who says that
similar plane numbers are, first, all square numbers and, secondly,
such oblong numbers as have the sides which contain them proportional.
Thus 6 is an oblong number with length~3 and breadth~2; 24 is another
with length~6 and breadth~4.  Since therefore 6 is to 3 as 4 is to 2,
the numbers 6 and 24 are similar.

Now the simplest case of two similar numbers is that of 1 and $a^2$
and, since 1 is odd, the condition (1) requires that $a^2$ and
therefore $a$, is also odd.  That is, we may take 1 and $(2n + 1)^2$
and equate them respectively to $c-b$ and $c+6$, whence we have
\begin{align*}
    b &= \frac{(2n+1)^2 - 1}{2},\\[\jot]
    c &= \frac{(2n+1)^2 - 1}{2} + 1,\\
\intertext{while}
    a &= 2n + 1.
\end{align*}
As Cantor remarks, the form in which $c$ and~$b$ appear correspond
sufficiently closely to the description in the text of Proclus.

Another obvious possibility would be, instead of equating $c - b$ to
unity, to put $c - b = 2$, in which case the similar number $c + b$
must be equated to double of some square, i.e.\ to a number of the
form~$2n^2$, or to the half of an even square number, say
$\frac{(2n)^2}{2}$.  This would give
\begin{align*}
    a &= 2n\\
    b &= n^2 - 1\\
    c &= n^2 + 1,
\end{align*}
which is Plato's solution, as given by Proclus.

The two solutions supplement each other.  It is interesting to observe
that the method suggested by Röth and Cantor is very like that of
Eucl.~\book{10} (Lemma~1 following Prop.~\prop*{10}{28}).  We shall
come to this later, but it may be mentioned here that the problem is
\emph{to find two square numbers such that their sum is also a
  square}.  Euclid there uses the property of \prop{2}{6} to the
effect that, if $AB$ is bisected at~$C$ and produced to~$D$,
\[
    AD . DB + BC^2 = CD^2.
\]
We may write this
\[
    uv = c^2 - b^2,
\]
where
\[
    u = c + b, v = c-b.
\]
In order that $uv$ may be a square, Euclid points out that $u$ and~$v$
must be similar numbers, and further that $u$ and~$v$ must be either
both odd or both even in order that $b$ may be a whole number.  We may
then put for the similar numbers, say, $\alpha\beta^2$ and
$\alpha\gamma^2$, whence (if $\alpha\beta^2$, $\alpha\gamma^2$ are
either both odd or both even) we obtain the solution
\[
    \alpha \beta^2 . \alpha \gamma^2
        + \left(\frac{\alpha \beta^2 - \alpha \gamma^2}{2}\right)
= \left(\frac{\alpha \beta^2 + \alpha \gamma^2}{2}\right)
\]

But I think a serious, and even fatal, objection to the conjecture of
Cantor and Roth is the very fact that the method enables both the
Pythagorean and the Platonic series of triangles to be deduced with
equal ease.  If this had been the case with the method used by
Pythagoras, it would not, I think, have been left to Plato to discover
the second series of such triangles.  It seems to me therefore that
Pythagoras must have used some method which would produce his rule
\emph{only}; and further it would be some less recondite method,
suggested by direct \emph{observation} rather than by argument from
general principles.

One solution satisfying these conditions is that of Bretschneider
(p.~83), who suggests the following simple method Pythagoras was
certainly aware that the successive odd numbers are \emph{gnomons}, or
the differences between successive square numbers.  It was then a
simple matter to write down in three rows (\emph{a})~the natural
numbers, (\emph{b})~their squares, (\emph{c})~the successive odd
numbers constituting the differences between the successive squares
in~(\emph{b}), thus:

r 2 3 4 5 6 7 8 9 10 n 12 13 14
1 4 9 16 35 36 49 64 81 100 121 144 169 196
1357 9 11 13 15 17 19 21 *3 25 27

Pythagoras had then only to pick out the numbers in the third row
which are squares, and his rule would be obtained by finding the
formula connecting the square in the third line with the two adjacent
squares in the second line.  But even this would require some little
argument; and I think a still better suggestion, because making pure
observation play a greater part, is that of P.~Treutlein
(\emph{Zeitschrift für Mathematik und Physik}, \r28., 1883,
Hist.-litt.\ Abtheilung, pp.~209 sqq.).

We have the best evidence (e.g.\ in Theon of Smyrna) of the practice
of representing square numbers and other figured numbers,
e.g.\ oblong, triangular, hexagonal, by dots or signs arranged in the
shape of the particular figure.  (Cf.\ Aristotle,
\emph{Metaph.}\ 1092~b~12).  Thus, says Treutlein, it would be easily
seen that any square number can be turned into the next higher square
by putting a single row of dots round two adjacent sides, in the form
of a gnomon (see figures on next page).

If \emph{a} is the side of a particular square, the gnomon round it is
shown by simple inspection to contain $2a + 1$ dots or units.  Now, in
order that $2a + 1$ may itself be a square, let us suppose
\[
    2a + 1 = n,
\]
whence
\[
    a = \frac{1}{2} (n^2 - 1),
\]
and
\[
    a + 1 = \frac{1}{2} (n^2 + 1),
\]

In order that $a$ and $a + 1$ may be integral, $n$ must be odd, and we
have at once the Pythagorean formula
\[
    n^2 + \left(\frac{n^2 - 1}{2}\right)^2 =
        \left(\frac{n^2 + 1}{2}\right)^2
\]
I think Treutlein's hypothesis is shown to be the conect one by the
passage in Aristotle's \emph{Physics} already quoted, where the
reference is undoubtedly to the Pythagoreans, and odd numbers are
clearly identified with \emph{gnomons} ``placed round~1.''  But the
ancient commentaries on the passage make the matter clearer still.
Philoponus says: ``As a proof…the Pythagoreans refer to what
\infig{propI_47X} happens with the addition of numbers; for when the
odd numbers are successively added to a square number they keep it
square and equilateral….  Odd numbers are accordingly called
\emph{gnomons} because, when added to what are already squares, they
preserve the square form…Alexander has excellently said in explanation
that the phrase `when gnomons are placed round' means \emph{making a
  figure} with the odd numbers (\greek{τὴν κατὰ τοὺς περιττοὺς
  ἀριθμοὺς σχηματογραφίαν})…(for it is the practice with the
Pythagoreans to \emph{represent things in figures}
(\greek{σχηματογραφεῖν}).''

The next question is; assuming this explanation of the Pythagorean
formula, what are we to say of the origin of Plato's?  It could of
course be obtained as a particular case of the general formula of
Eucl.~\book{10} already referred to; but there are two simple
alternative explanations in this case also, (1)~Bretschneider observes
that, to obtain Plato's formula, we have only to double the sides of
the squares in the Pythagorean formula, for
\[
    (2n)^2 + (n^2-1)^2 = (n^2 + 1)^2,
\]
where however $n$ is not necessarily odd.

(2)~Treutlein would explain by means of an extension of the gnomon
idea.  As, he says, the Pythagorean formula was obtained by placing a
gnomon consisting of a single row of dots round two adjacent sides of
a square, it would be natural to try whether another solution could
not be found by placing round the square a gnomon consisting of a
\emph{double} row of dots.  Such a gnomon would equally turn the
square into a larger square; and the question would be whether the
double-row gnomon itself could be a square.  If the side of the
original square was~$a$, it would easily be seen that the number of
units in the double-row gnomon would be $4a + 4$, and we have only to
put
\[
    4a + 4 = 4n^2,
\]
whence
\begin{align*}
    a &= n^2 - 1,\\
a + 2 &= n^2 + 1,
\end{align*}
and we have the Platonic formula
\[
    (2n)^2 + (n^2 - 1)^2 = (n^2 + 1)^2.
\]
I think this is, in substance, the right explanation, but, in form,
not quite correct.  The Greeks would not, I think, have treated the
\emph{double} row as a gnomon. Their comparison would have been
between (1)~a certain square \emph{plus} a single-row gnomon and
(2)~the same square \emph{minus} a single-row gnomon.  As the
application of Eucl.~\prop{2}{4} to the case where the segments of the
side of the square are~$a$,~1 enables the Pythagorean formula to be
obtained as Treutlein obtains it, so I think that Eucl.~\prop{2}{8}
confirms the idea that the Platonic formula was obtained by comparing
a square \emph{plus} a gnomon with the same square \emph{minus} a
gnomon. For \prop{2}{8} proves that
\[
    4ab + (a-b)^2 = (a + b)^2,
\]
whence, substituting 1 for~$b$, we have
\[
    4a + (a-1)^2 = (a + 1)^2,
\]
and we have only to put $a = n^2$ to obtain Plato's formula.

\subsection*{The ``theorem of Pythagoras'' in India}

This question has been discussed anew in the last few years as the
result of the publication of two important papers by Albert Bürk on
\emph{Das Āpastamba-\emend{S}{Ś}ulba-Sūtra} in the \emph{Zeitschrift
  der deutschen morgenländischen Gesellschaft} (\r40., 1901,
pp. 543–591, and \r56., 1902, pp. 337—391).  The first of the two
papers contains the introduction and the text, the second the
translation with notes.  A selection of the most important parts of
the material was made and issued by G.~Thibaut in the \emph{Journal of
  the Asiatic Society of Bengal}, \r44., 1875, Part~1.\ (reprinted
also at Calcutta, 1875, as \emph{The Śulvasūtras}, by G.~Thibaut).
Thibaut in this work gave a most valuable comparison of extracts from
the three Śulvasūtras by Bāudhāyana, Āpastamba and Kātyāyana
respectively, with a running commentary and an estimate of the date
and originality of the geometry of the Indians.  Bürk has however done
good service by making the Āpastamba-Ś.-S.\ accessible in its entirety
and investigating the whole subject afresh.  With the natural
enthusiasm of an editor for the work he is editing, he roundly
maintains, not only that the Pythagorean theorem was known and proved
in all its generality by the Indians long before the date of
Pythagoras (about 580—500~\bc), but that they had also discovered the
irrational; and further that, so far from Indian geometry being
indebted to the Greek, the much-travelled Pythagoras probably obtained
his theory from India (\emph{loc.\ cit.}\ \r40., p.~575 note).  Three
important notices and criticisms of Bürk's work have followed, by
H.~G. Zeuthen (``\emph{Théorème de Pythagore},'' \emph{Origine de la
  Géométrie scientifique}, 1904, already quoted), by Moritz Cantor
(\emph{Über die älteste indische Mathematik} in the \emph{Archiv der
  Mathematik und Physik}, \r8., 1905, pp.~63—72) and by Heinrich Vogt
(\emph{Haben die alten Inder den Pythgoreischen Lehrsatz und das
  Irrationale gekannt?} in the \emph{Bibliotheca Mathematica},
\r8\tsub{3}, 1906, pp.~6—23.  See also Cantor's \emph{Geschichte der
  Mathematik}, \r1\tsub{3}, pp.~635–645.

The general effect of the criticisms is, I think, to show the
necessity for the greatest caution, to say the least, in accepting
Bürk's conclusions.

I proceed to give a short summary of the portions of the contents of
the Āpastamba-Ś.-S.\ which are important in the present connexion.  It
may be premised that the general object of the book is to show how to
construct altars of certain shapes, and to vary the dimensions of
altars without altering the form.  It is a collection of \emph{rules}
for carrying out certain constructions.  There are no proofs, the
nearest approach to a proof being in the rule for obtaining the area
of an isosceles trapezium, which is done by drawing a perpendicular
from one extremity of the smaller of the two parallel sides to the
greater, and then taking away the triangle so cut off and placing it,
the other side up, adjacent to the other equal side of the trapezium,
thereby transforming the trapezium into a rectangle.  It should also
be observed that Āpastamba does not speak of \emph{right-angled
  triangles}, but of two adjacent sides and the diagonal of a
\emph{rectangle}.  For brevity, I shall use the expression ``rational
rectangle'' to denote a rectangle the two sides and the diagonal of
which can be expressed in terms of rational numbers.  The references
in brackets are to the chapters and numbers of Āpastamba's work.

(1)~Constructions of right angles by means of cords of the following
relative lengths respectively :
\begin{gather*}
    \left\{
    \begin{array}{rrr@{\quad}l}
    3, &  4, &  5     & (\prop{1}{3}, \prop{5}{3})\\
   12, & 16, & 20     & (\prop{5}{3})\\
   15, & 20, & 25     & (\prop{5}{3})
    \end{array}
    \right.
\\
    \left\{
    \begin{array}{rrr@{\quad}l}
     5, & 12, & 13     & (\prop{5}{4})\\
    15, & 36, & 39     & (\prop{1}{2}, \prop{5}{2}, \prop*{5}{4})
    \end{array}
    \right.
\\
    \begin{array}{rrr@{\quad}l}
     8, & 15, & 17     & (\prop{5}{5})\\
    12, & 35, & 37     & (\prop{5}{5})
    \end{array}
\end{gather*}

(2)~A general enunciation of the Pythagorean theorem thus: ``The
diagonal of a rectangle produces [i.e. the square on the diagonal is
  equal to] the sum of what the longer and shorter sides separately
produce [i.e.\ the squares on the two sides].'' \using{\prop{1}{4}}

(3)~The application of the Pythagorean theorem to a \emph{square}
instead of a rectangle [i.e.\ to an \emph{isosceles} right-angled
  triangle]: ``The diagonal of a square produces an area double [of
  the original square].'' \using{\prop{1}{5}}

(4)~An approximation to the value of~$\sqrt{2}$; the diagonal of a
square is
$\left( 1 + \frac{1}{3} + \frac{1}{3 \cdot 4} -  \frac{1}{3 \cdot
  \cdot 34} \right)$
times the side. \using{\prop{1}{6}}

(5)~Application of this approximate value to the construction of a square
with side of any length. \using{\prop{1}{6}}

(6)~The construction of $a \sqrt{3}$, by means of the Pythagorean
theorem, as the diagonal of a rectangle with sides $a$ and $a
\sqrt{2}$. \using{\prop{2}{2}}

(7)~Remarks equivalent to the following:

(\emph{a}) $a \sqrt{\frac{1}{3}}$ is the side of $\frac{1}{9}(a
\sqrt{3})^2$, or $a\sqrt{\frac{1}{3}} = \frac{1}{3} a \sqrt{3}$.
\using{\prop{2}{3}}

{\emph{b}} A square on length of 1 unit gives 1 unit square
\using{\prop{3}{4}}

A square on length of 2 unit gives 4 unit squares
\using{\prop{3}{6}}

A square on length of 3 unit gives 9 unit squares
\using{\prop{3}{6}}

A square on length of $1\frac{1}{2}$ units gives $2\frac{1}{4}$ unit
squares \using{\prop{3}{8}}

A square on length of $2\frac{1}{2}$ units gives $6\frac{1}{4}$ unit
squares \using{\prop{3}{8}}

A square on length of $\frac{1}{2}$ units gives $\frac{1}{4}$ unit
squares \using{\prop{3}{10}}

A square on length of $\frac{1}{3}$ units gives $\frac{1}{9}$ unit
squares \using{\prop{3}{10}}

(\emph{c})~Generally, the square on any length contains as many rows
(of small, unit, squares) as the length contains units.
\using{\prop{3}{7}}

(8)~Constructions, by means of the Pythagorean theorem, of

(\emph{a})~the \emph{sum} of two squares as one square,
\using{\prop{2}{4}}

(\emph{b})~the \emph{difference} of two squares as one square.
\using{\prop{2}{5}}

(9)~A transformation of a rectangle into a square.
\using{\prop{2}{7}}

[This is not directly done as by Euclid in \prop{2}{14}, but the
  rectangle is first transformed into a gnomon, i.e.\ into the
  difference between two squares, which difference is then transformed
  into one square by the preceding rule.  If $ABCD$ be the given
  rectangle of which $BC$ is the longer side, cut off the square
  $ABEF$, bisect the rectangle $DE$ left over by $HG$ parallel
  to~$FE$, move the upper half $DG$ and place it on $AF$ as base in
  the position~$AK$.  Then the rectangle $ABCD$ is equal to the gnomon
  which is the difference between the square $LB$ and the square~$LF$.
  In other words, Āpastamba transforms the rectangle $ab$ into the
  difference between the squares $\left(\frac{a + b}{2}\right)^2$ and
  $\left(\frac{a - b}{2}\right)^2$.]

(10)~An attempt at a transformation of a square ($a^2$) into a rectangle
which shall have one side of given length ($b$).
\using{\prop{3}{1}}

[This shows no sign of such a procedure as that of Eucl.~\prop{1}{44},
  and indeed does no more than say that we must subtract $ab$ from
  $a^2$ and then adapt the remainder $a^2 - ab$ so that it may ``fit
  on'' to the rectangle~$ab$.  The problem is therefore only reduced
  to another of the same kind, and presumably it was only solved
  \emph{arithmetically} in the case where $a$, $b$ are given
  numerically.  The Indian was therefore far from the general,
  geometrical, solution.]

(11)~Increase of a given square into a larger square.
\using{\prop{3}{9}}

[This amounts to saying that you must add two rectangles ($a$, $b$)
  and another square ($b^2$) in order to transform a square $a^2$ into
  a square $(a + b)^2$.  The formula is therefore that of
  Eucl.~\prop{2}{4}, $a^2 + 2ab + b^2 = (a + b)^2$.]

The first important question in relation to the above is that of date.
Bürk assigns to the \emph{Āpastamba- Śulba-Sūtra} a date at least as
early as the 5th or 4th century~\bc.  He observes however (what is
likely enough) that the matter of it must have been much older than
the book itself.  Further, as regards one of the constructions for
right angles, that by means of cords of lengths 15, 36, 39, he show
that it was known at the time of the \emph{Tāittirīya-Saṃhitā} and the
\emph{Satapatha-Brāhmaṇa}, still older works belonging to the 8th
century~\bc\ at latest.  It may be that (as Bürk maintains) the
discovery that triangles with sides ($a$, $b$, $c$) in rational
numbers such that $a^2 + b^2 = c^2$ are right-angled was nowhere made
so early as in India.  We find however in two ancient Chinese
treatises (1)~a statement that the diagonal of the rectangle (3, 4)
is~5 and (2)~a rule for finding the hypotenuse of a ``right triangle''
from the sides, while tradition connects both works with the name of
Chou Kung who died 1105~\bc\ (D.~E. Smith, \emph{History of
  Mathematics}, \r1.\ pp.~30—33, \r2.\ p.~288).

As regards the various ``rational rectangles'' used by Āpastamba, it
is to be observed that two of the seven, viz.\ 8, 15, 17 and 12, 35,
37, do not belong to the Pythagorean series, the others consist of two
which belong to it, viz.\ 3, 4,~5 and 5, 12, 13, and multiples of
these.  It is true, as remarked by Zeuthen (\emph{op.\ cit.}\ p.~841),
that the rules of \prop{2}{7} and \prop{3}{9}, numbered (9) and~(11)
above respectively, would furnish the means of finding any number of
``rational rectangles.''  But it would not appear that the Indians had
been able to formulate any general rule; otherwise their list of such
rectangles would hardly have been so meagre.  Āpastamba mentions seven
only, really reducible to four (though one other, 7, 24, 25, appears
in the BāudhāyanaŚ-S., supposed to be older than Āpastamba).  These
are all that Āpastamba knew of, for he adds (\r5.~6): ``So many
\emph{recognisable} (erkennbare) constructions are there,'' implying
that he knew of no other ``rational rectangles'' that could be
employed.  But the words also imply that the theorem of the square on
the diagonal is also true of other rectangles not of the
``recognisable'' kind, i.e.\ rectangles in which the sides and the
diagonal are not in the ratio of integers; this is indeed implied by
the constructions for $\sqrt{2}$, $\sqrt{3}$ etc.\ up to $\sqrt{6}$
(cf.\ \prop{2}{2}, \prop{8}{5}).  This is all that can be said.  The
theorem is, it is true, enunciated as a general proposition, but there
is no sign of anything like a general proof; there is nothing to show
that the assumption of its universal truth was founded on anything
better than an imperfect induction from a certain number of cases,
discovered empirically, of triangles with sides in the ratio of whole
numbers in which the property~(1) that the square on the longest side
is equal to the sum of the squares on the other two sides was found to
be always accompanied by the property~(2) that the latter two sides
include a right angle.

It remains to consider Bürk's claim that the Indians had discovered
the \emph{irrational}.  This is based upon the approximate value of
$\sqrt{2}$ given by Āpastamba in his rule~\r1.~6 numbered~(4) above.
There is nothing to show how this was arrived at, but Thibaut's
suggestion certainly seems the best and most natural.  The Indians may
have observed that $17^2 = 289$ is nearly double of $12^2 = 144$.  If
so, the next question which would naturally occur to them would be, by
how much the side~17 must be diminished in order that the square on it
may be 288 \emph{exactly}.  If, in accordance with the Indian fashion,
a gnomon with unit area were to be subtracted from a square with 17 as
side, this would approximately be secured by giving the gnomon the
breadth $\frac{1}{34}$, for $2 \times 17 \times \frac{1}{34} = $.  The
side of the smaller square thus arrived at would be $17 - \frac{1}{34}
= 12 + 4 + - \frac{1}{34}$, whence, dividing out by~12, we have
\[
    \sqrt{2} = 1 + \frac{1}{3} + \frac{1}{3 \cdot 4} - \frac{1}{3
      \cdot 4 \cdot 34}, \quad\text{approximately}.
\]
But it is a far cry from this calculation of an approximate value to
the discovery of the \emph{irrational}.  First, we ask, is there any
sign that this value was known to be inexact?  It comes directly after
the statement (\r1.~6) that the square on the diagonal of a square is
double of that square, and the rule is quite boldly stated without any
qualification: ``lengthen the unit by one-third and the latter by
one-quarter of itself less one-thirty-fourth of this part.''  Further,
the approximate value is actually used for the purpose of constructing
a square when the side is given (\r2.~1).  So familiar was the formula
that it was apparently made the basis of a sub-division of measures of
length.  Thibaut observes (\emph{Journal of the Asiatic Society of
  Bengal}, \r49., p.~241) that, according to Bāudhāyana, the unit of
length was divided into 12 \emph{fingerbreadths}, and that one of two
divisions of the \emph{fingerbreadht} was into 34 \emph{sesame-corns},
and he adds that he has no doubt that this division, which he has not
elsewhere met, owes its origin to the formula for~$\sqrt{2}$.  The
result of using this sub-division would be that, in a square with side
equal to 12 \emph{fingerbreadths}, the diagonal would be 17
\emph{fingerbreadths} less 1 \emph{sesame-corn}.  Is it conceivable
that a sub-division of a measure of length would be based on an
evaluation known to be inexact?  No doubt the first discoverer would
be aware that the area of a gnomon with breadth $\frac{1}{34}$ and
outer side~17 is not exactly equal to~1 but less than it by the square
of~$\frac{1}{34}$ or by~$\frac{1}{1156}$ and therefore that, in taking
that gnomon as the proper area to be subtracted from $17^2$, he was
leaving out of account the small fraction~$\frac{1}{1156}$; as,
however, the object of the whole proceeding was purely practical, he
would, without hesitation, ignore this as being of no practical
importance, and, thereafter, the formula would be handed down and
taken as a matter of course without arousing suspicion as to its
accuracy.  This supposition is confirmed by reference to the sort of
rules which the Indians allowed themselves to regard as accurate.
Thus Āpastamba himself gives a construction for a circle equal in area
to a given square, which is equivalent to taking $\pi = 3.09$, and yet
observes that it gives the required circle ``\emph{exactly}'' (\r3.2),
while his construction of a square equal to a circle, which he equally
calls ``exact,'' makes the side of the square equal to
$\frac{13}{15}$th of the diameter of the circle (\r3.~3), and is
equivalent to taking $\pi = 3.004$.  But, even if some who used the
approximation for~$\sqrt{2}$ were conscious that it was not quite
accurate (of which there is no evidence), there is an immeasurable
difference between arrival at this consciousness and the discovery of
the irrational.  As Vogt says, three stages had to be passed through
before the irrationality of the diagonal of a square was discovered in
any real sense, (1)~All values found by direct measurement or
calculations based thereon have to be recognised as being inaccurate.
Next (2)~must supervene the conviction that it is \emph{impossible} to
arrive at an accurate arithmetical expression of the value.  And
lastly (3)~the impossibility must be proved.  Now there is no real
evidence that the Indians, at the date in question, had even reached
the first stage, still less the second or third.

The net results then of Bürk's papers and of the criticisms to which
they have given rise appear to be these.  (1)~It must be admitted that
Indian geometry had reached the stage at which we find it in Āpastamba
quite independently of Greek influence.  But (2)~the old Indian
geometry was purely empirical and practical, far removed from
abstractions such as the irrational.  The Indians had indeed, by trial
in particular cases, persuaded themselves of the truth of the
Pythagorean theorem and enunciated it in all its generality; but they
had not established it by scientific proof.

\subsection*{Alternative proofs}

I.~The well-known proof of \prop{1}{47} obtained by putting two
squares side by side, with their bases continuous, and cutting off
right-angled triangles which can then be put on again in different
positions, is attributed by an-Nairīzī to Thābit b.~Qurra
(826—901~\ad).

His actual construction proceeds thus.

Let $ABC$ be the given triangle right-angled at~$A$.

Construct on $AB$ the square~$AD$; produce $AC$ to~$F$ so that $EF$
may be equal to~$AC$.

Construct on $EF$ the square~$EG$, and produce $DH$ to~$K$ so that
$DK$ may be equal to~$AC$.

It is then proved that, in the triangles $BAC$, $CFG$, $KHG$, $BDK$,
the sides $BA$, $CF$, $KH$, $BD$ are all equal, and the sides $AC$,
$FG$, $HG$, $DK$ are all equal.

The angles included by the equal sides
are all right angles; hence the four triangles
are equal in all respects.  \using{\prop{1}{4}}

Hence $BC$, $CG$, $GK$, $KB$ are all equal.

Further the angles $DBK$, $ABC$ are equal;
hence, if we add to each the angle $DBC$,
the angle $KBC$ is equal to the angle $ABD$
and is therefore a right angle.

In the same way the angle $CGK$ is right;
therefore $BCGK$ is a square, i.e.\ the square on~$BC$.

Now the sum of the quadrilateral $GCLH$ and the triangle $LDB$
together with two of the equal triangles make the squares on $AB$,
$AC$, and together with the other two make the square on~$BC$.

Therefore etc.

II.~Another proof is easily arrived at by taking the particular case
of Pappus' more general proposition given below in which the given
triangle is right-angled and the parallelograms on the sides
containing the right angles are squares.  If the figure is drawn, it
will he seen that, with no more than one additional line inserted, it
contains Thābit's figure, so that Thābit's proof may have been
practically derived from that of Pappus.

III.~The most interesting of the remaining proofs seems to be that
shown in the accompanying figure.

It is given by J.~W. Müller, \emph{Systematische Zusammenstellung der
  wichtigsten bisher bekannten Beweise des Pythag.\ Lehrsatzes}
(Nürnberg, 1819), and in the second edition (Mainz, 1821) of
Ign.\ Hoffmann, \emph{Der Pythag.\ Lehrsatz mit 32 theils bekannten
  theils neuen Beweisen} [3 more in second edition].  It appears to
come from one of the scientific papers of Lionardo da Vinci
(1452—1519).

The triangle $HKL$ is constructed on the base $KH$ with the side $KL$
equal to $BC$ and the side $LH$ equal to~$AB$.

Then the triangle $HLK$ is equal in all respects to the triangle
$ABC$, and to the triangle~$EBF$.

Now $DB$, $BG$, which bisect the angles $ABE$, $CBF$ respectively, are
in a straight line.  Join~$BL$.

It is easily proved that the four quadrilaterals $ADGC$, $EDGF$,
$ABLK$, $HLBC$ are all equal.

Hence the hexagons $ADEFGC$, $ABCHLK$ are equal.

Subtracting from the former the two triangles $ABC$, $EBF$, and from
the latter the two equal triangles $ABC$, $HLK$, we prove that the
square $CK$ is equal to the sum of the squares $AE$,~$CF$.

\subsection*{Pappus' extension of \prop{1}{47}}

In this elegant extension the triangle may be \emph{any} triangle (not
necessarily right-angled), and \emph{any} parallelograms take the
place of squares on two of the sides.

Pappus (\r4.\ p.~177) enunciates the theorem as follows:

\emph{If $ABC$ be a triangle, and any parallelograms whatever $ABED$,
  $BCFG$ be described on $AB$, $BC$, and if $DE$, $FG$ be produced
  to~$H$, and $HB$ be joined, the parallelograms $ABED$, $BCFG$ are
  equal to the parallelogram contained by~$AC$, $HB$ in an angle which
  is equal to the sum of the angles $BAC$, $DHB$.}

Produce $HB$ to~$K$; through $A$, $C$ draw $AL$, $CM$ parallel
to~$HK$, and join~$LM$.

Then, since $ALHB$ is a parallelogram, $AL$, $HB$ are equal and
parallel.  Similarly $MC$, $HB$ are equal and parallel.

Therefore $AL$, $MC$ are equal and
parallel;
whence $LM$, $AC$ are also equal and parallel,
and $ALMC$ is a parallelogram.

Further, the angle $LAC$ of this parallelogram is equal to the sum of the
angles $BAC$, $DHB$, since the angle $DHB$ is equal to the angle~$LAB$.

Now, since the parallelogram $DABE$ is equal to the parallelogram
$LABH$ (for they are on the same base $AB$ and in the same parallels
$AB$, $DM$), and likewise $LABH$ is equal to $LAKN$ (for they are on
the same base~$LA$ and in the same parallels $LA$,~$HK$),
the parallelogram $DABE$ is equal to the parallelogram $LAKH$.

For the same reason,
the parallelogram $BGFC$ is equal to the parallelogram $NKCM$.

Therefore the sum of the parallelograms $DABE$, $BGFC$ is equal to the
parallelogram $LACM$, that is, to the parallelogram which is contained
by $AC$, $HB$ in an angle $LAC$ which is equal to the sum of the
angles $BAC$, $BHD$.

``And this is far more general than what is proved in the Elements
about squares in the case of right-angled (triangles).''

\subsection*{Heron's proof that $AL$, $BK$, $CF$ in Euclid's figure
  meet in a point}

The final words of Proclus' note on \prop{1}{47} (p.~429, 9—15) are
historically interesting.  He says: ``The demonstration by the writer
of the Elements being clear, I consider that it is unnecessary to add
anything further, and that we may be satisfied with what has been
written, since, in fact those who have added anything more, like
Pappus and Heron, were obliged to draw upon what is proved in the
sixth Book, for no really useful object.''  These words cannot of
course refer to the extension of \prop{1}{47} given by Pappus; but the
key to them, so far as Heron is concerned, is to be found in the
commentary of an-Nairīzī (pp.~175—185, ed.\ Besthorn-Heiberg;
pp.~78—84, ed.\ Curtze) on \prop{1}{47}, wherein he gives Heron's
proof that the lines $AL$, $FC$, $BK$ in Euclid's figure meet in a
point.  Heron proved this by means of three lemmas which would most
naturally be proved from the principle of similitude as laid down in
Book~\r6., but which Heron, as a \emph{tour de force}, proved on the
principles of Book~\book{1} only.  The \emph{first} lemma is to the
following effect

\emph{If, in a triangle $ABC$, $DE$ be drawn parallel to the base
  $BC$, and if $AF$ be drawn from the vertex~$A$ to the middle
  point~$F$ of $BC$, then $AF$ will also bisect~$DE$.}

This is proved by drawing $HK$ through~$A$ parallel, to $DE$ or $BC$,
and $HDL$, $KEM$ through $D$, $E$ respectively parallel to $AGE$, and
lastly joining $DF$, $EF$.

Then the triangles $ABE$, $AFC$ are equal (being on equal bases), and
the triangles $DBF$, $EFC$ are also equal (being on equal bases and
between the same parallels).

Therefore, by subtraction, the triangles $ADF$, $AEF$ are equal, and
hence the parallelograms $AL$, $AM$ are equal.

These parallelograms are between the same parallels $LM$, $HK$;
therefore $LF$, $FM$ are equal, whence $DG$, $GE$ are also equal.

The \emph{second} lemma is an extension of this to the case where $DE$
meets $BA$, $CA$ produced beyond~$A$.

The \emph{third} lemma proves the converse of Euclid \prop{1}{43},
that, \emph{If a parallelogram $AB$ it cut into four others $ADGE$,
  $DF$, $FGCB$, $CE$, so that $DF$, $CE$ are equal, the common
  vertex~$G$ will be on the diagonal~$AB$}.

Heron produces $AG$ till it meets $CF$ in~$H$.  Then, if we join $HB$,
we have to prove that $AHB$ is one straight line.  The proof is as
follows.  Since the areas $DF$, $EC$ are equal, the triangles $DGF$,
$ECG$ are equal.

If we add to each the triangle $GCF$,
the triangles $ECF$, $DCF$ are equal;
therefore $MD$, $CF$ are parallel.

Now it follows from \prop{1}{34}, \prop*{1}{29} and \prop*{1}{26} that
the triangles $AKE$, $GKD$ are equal in all respects;
therefore $EK$ is equal to~$KD$.

Hence, by the second lemma,
$CH$ is equal to~$HE$.

Therefore, in the triangles $FHB$, $CHG$,
the two sides $BF$, $FH$ are equal to the two sides $GC$, $CH$,
and the angle $BFH$ is equal to the angle $GCH$;
hence the triangles are equal in all respects,
and the angle $BHF$ is equal to the angle~$GHC$.

Adding to each the angle $GHF$, we find that the angles $BHF$, $FHG$ are
equal to the angles $CHG$, $GHF$,
and therefore to two right angles.

Therefore $AHB$ is a straight line.

Heron now proceeds to prove the proposition that, in the accompanying
figure, if $AKL$ perpendicular to~$BC$ meet $EC$ in~$M$, and if $BM$,
$MG$ be joined, $BM$, $MG$ are in one straight line.

Parallelograms are completed as shown in the figure, and the diagonals
$OA$, $FH$ of the parallelogram $FH$ are drawn.

Then the triangles $FAH$, $BAC$ are clearly equal in all respects;
therefore the angle $HFA$ is equal to
the angle $ABC$, and therefore to the angle
$CAK$ (since $AK$ is perpendicular to~$BC$).

But, the diagonals of the rectangle
$FH$ cutting one another in~$Y$,
$FY$ is equal to~$YA$,
and the angle $HFA$ is equal to the
angle~$OAF$.

Therefore the angles $OAF$, $CAK$ are
equal, and accordingly
$OA$, $AK$ are in a straight line.

Hence $OM$ is the diagonal of~$SQ$;

therefore $AS$ is equal to~$AQ$,
and, if we add $AM$ to each,
$FM$ is equal to~$MH$.

But, since $EC$ is the diagonal of the parallelogram~$FN$,
$FM$ is equal to $MN$.

Therefore $MH$ is equal to~$MN$;
and, by the third lemma, $BM$, $MG$ are in a straight line.

\end{notes}

\end{proposition}

\begin{proposition}
\label{prop:I_48}

\begin{statement}
If in a triangle the square on one of the sides be equal to the
squares on the remaining two sides of the triangle, the angle
contained by the remaining two sides of the triangle is right.
\end{statement}

\begin{proof}

For in the triangle $ABC$ let the square on one side $BC$
be equal to the squares on the sides $BA$, $AC$;

I say that the angle $BAC$ is right.

For let $AD$ be drawn from the point~$A$ at right angles to the
straight line $AC$, let $AD$ be made equal to~$BA$, and let $DC$ be
joined.

Since $DA$ is equal to~$AB$, the square on~$DA$ is also equal to the
square on~$AB$.

Let the square on $AC$ be added to each; therefore the squares on
$DA$, $AC$ are equal to the squares on $BA$,~$AC$.

But the square on $DC$ is equal to the squares on $DA$,
$AC$, for the angle $DAC$ is right; \using{\prop{1}{47}}
and the square on $BC$ is equal to the squares on $BA$, $AC$, for
this is the hypothesis;
therefore the square on $DC$ is equal to the square on~$BC$,
so that the side $DC$ is also equal to~$BC$.

And, since $DA$ is equal to $AB$, and $AC$ is common, the two sides
$DA$, $AC$ are equal to the two sides $BA$, $AC$; and the base $DC$ is
equal to the base~$BC$; therefore the angle $DAC$ is equal to the
angle $BAC$. \using{\prop{1}{8}}

But the angle $DAC$ is right;
therefore the angle $BAC$ is also right.

Therefore etc.
\end{proof}

\begin{notes}

Proclus' note (p.~430) on this proposition, though it does not mention
Heron's name, gives an alternative proof, which is the same as that
definitely attributed by an-Nairīzī to Heron, the only difference
being that Proclus demonstrates two cases in full, while Heron
dismisses the second with a ``similarly.'' The alternative proof is
another instance of the use of \prop{1}{7} as a means of answering
objections.  If, says Proclus, it be not admitted that the
perpendicular $AD$ may be drawn on the opposite side of $AC$ from~$B$,
we may draw it on the same side as $AB$, in which case it is
impossible that it should not coincide with $AB$.  Proclus takes two
cases, first supposing that the perpendicular falls, as $AD$, within
the angle $CAB$, and secondly that it falls, as~$AE$, outside that
angle.  In either case the absurdity results that, on the same
straight line $AC$ and on the same side of it, $AD$, $DC$ must be
respectively equal to $AB$, $BC$, which contradicts~\prop{1}{7}.

Much to the same effect is the note of De Morgan that there is here
``an appearance of avoiding indirect demonstration by drawing the
triangles on different sides of the base and appealing to \prop{1}{8},
because drawing them on the same side would make the appeal to
\prop{1}{7} (on which, however, \prop{1}{8} is founded).''

\end{notes}

\end{proposition}

\part{Book II}

\chapter*{Definitions}

\begin{enumerate}

\item\label{def:II_1} Any rectangular parallelogram is said to be
  contained by the two straight lines containing the right angle.

\item\label{def:II_2} And in any parallelogrammic area let any one
  whatever of the parallelograms about its diameter with the two
  complements be called a \textbf{gnomon}.

\end{enumerate}

\section*{Definition 1}

\greek{Πᾶν παραλληλόγραμμον ὀρθογώνιον περιέχεσθαι λέγεται ὑπὸ δύο τῶν
  την ὀρθὴν γωνίαν περιεχουσῶν εὐθειῶν.}

As the full expression in Greek for ``the angle $BAC''$ is ``the angle
contained by the (straight lines) $BA$, $AC$,'' \greek{ἡ ὑπὸ τῶν ΒΑ,
  ΑΓ περιεχομένη γωνία}, so the full expression for ``the rectangle
contained by BA, AC'' is \greek{τὸ ὑπὸ τῶν ΒΑ, ΑΓ περιεχόμενον
  ὀρθογώνιον}.  In this case too \greek{ΒΑ}, \greek{ΑΓ} is commonly
abbreviated by the Greek geometers into \greek{ΒΑΓ}.  Thus in
Archimedes and Apollonius \greek{τὸ ὑπὸ ΒΑΓ} or \greek{τὸ ὑπὸ τῶν ΒΑΓ}
means \emph{the rectangle $BA$, $AC$}, just as \greek{ἡ ὑπὸ ΒΑΓ} means
\emph{the angle $BAC$}; the gender of the article shows which is meant
in each case.  In the early Books Euclid uses the full expression
\greek{τὸ ὑπὸ τῶν ΒΑ, ΑΓ}; but the shorter form \greek{τὸ ὑπὸ τῶν ΒΑΓ}
is found from Book~\book{10} onwards.  Cf.\ \prop{12}{11}, where
\greek{τὰ (τμήματα) ἐπὶ τῶν ΘΟΕ, ΕΠΖ, ΖΠΗ, ΗΣΘ} means the segments on
the eight straight lines \greek{ΘΟ}, \greek{ΟΕ}, \greek{ΕΠ},
\greek{ΠΖ}, \greek{ΖΠ}, \greek{ΠΗ}, \greek{ΗΣ}, \greek{ΣΘ}.

\section*{Definition 2}

\greek{Παντὸς δὲ παραλληλογράμμου χωρίου το›ν περὶ τὴν διάμετρον αὐτοῦ
  παραλληλογράμμων ἓν ὁποιονοῦν σὺν τοῖς δυσὶ παραπληρώμασι γνώμων
  καλείσθω.}

Meaning literally a thing enabling something to be \emph{known},
\emph{observed} or \emph{verified}, a \emph{teller} or \emph{marker},
as we might say, the word \emph{gnomon} (\greek{γνώμων}) was first
used in the sense~(1) in which it appears in a passage of Herodotus
(\r2.~109) stating that ``the Greeks learnt the \greek{πόλος}, the
\emph{gnomon} and the twelve parts of the day from the Babylonians.''
According to Suidas, it was Anaximander (611—545~\bc) who introduced
the \emph{gnomon} into Greece.  Whatever may be the details of the
construction of the two instruments called the \greek{πόλος} and the
\emph{gnomon}, so much is certain, that the gnomon had to do with the
measurement of time by shadows thrown by the sun, and that the word
signified the placing of a staff perpendicular to the horizon.  This
is borne out by the statement of Proclus that Oenopides of Chios, who
first investigated the problem (Eucl.~\prop{1}{12}) of drawing a
perpendicular from an external point to a given straight line, called
the perpendicular a straight line drawn ``\emph{gnomon-wise}''
\sidefig{defII_1a}
(\greek{κατὰ γνώμονα}).  Then (1)~we find the term used of a
mechanical instrument for drawing right angles, as shown in the figure
annexed.  This seems to be the meaning in Theognis 805, where it is
said that the envoy sent to consult the oracle at Delphi should be
``straighter (\greek{ἰθύτερος}) than the \greek{τόρνος}, the
\greek{στάθμη} and the \emph{gnomon},'' and all three words evidently
denote appliances, the \greek{τόρνος} being an instrument for drawing
a circle (probably a string stretched between a fixed and a moving
point), and the \greek{στάθμη} a plumb-line.  Next (3)~it was natural
that the \emph{gnomon}, owing to its shape, should become the figure
which remained of a square when a smaller square was cut out of one
corner (or the figure, as Aristotle says, which when added to a square
increases its size but does not alter its form).  We have seen (note
on \prop{1}{47}, p.~351) that the Pythagoreans used the term in this
sense, and further applied it, by analogy, to the series of odd
numbers as having the same property in relation to square numbers.
The earliest evidence for this is the fragment of Philolaus
(\emph{c.}~460~\bc) already mentioned (see Boeckh, \emph{Philolaos des
  Pythagoreers Lehren}, p.~141) where he says that ``number makes all
things knowable and mutually agreeing (\greek{ποτάγορα ἀλλάλοις}) in
the way characteristic of the gnomon ``(\greek{κατὰ γνόμονος φύσιν}).
As Boeckh says (p.~144), it would appear from the fragment that the
connexion between the gnomon and the square to which it is added was
regarded as symbolical of union and agreement, and that Philolaus used
the idea to explain the knowledge of things, making the knowing
embrace and grasp the known as the gnomon does the square.
Cf.\ Scholium~\r2\ No.~11 (Euclid, ed.\ Heiberg, Vol.~\r5. p.~225),
which says ``It is to be noted that the gnomon was discovered by
geometers with a view to brevity, while the name came from its
incidental property, namely that from it the whole is known, whether
of the whole area or of the remainder, when it is either placed round
or taken away.  In sundials too its sole function is to make the
actual time of day known.

The geometrical meaning of the word is extended in the definition of
\sidefig{defII_1b}
\emph{gnomon} given by Euclid, where (4)~the gnomon has the same
relation to \emph{any parallelogram} as it before had to a square.
From the fact that Euclid says ``\emph{let}'' the figure described
``be \emph{callled} a gnomon'' we may infer that he was using the word
in the wider sense for the first time.  Later still~(5) we find Heron
of Alexandria defining a \emph{gnomon in general} as any figure which,
when added to any figure whatever, makes the whole figure similar to
that to which it is added.  In this definition of Heron (Def.~58)
Hultsch brackets the words which make it apply to any \emph{number} as
well; but Theon of Smyrna, who explains that plane, triangular,
square, solid and other kinds of numbers are so called after the
likeness of the areas which they measure, does make the term in its
most general sense apply to numbers. ``All the successive numbers
which [by being successively added] produce triangles or squares or
polygons are called gnomons (p.~37, 11—13, ed.\ Hilier).  Thus the
successive odd numbers added together make square numbers; the gnomons
in the case of triangular numbers are the successive numbers 1, 2, 3,
4…; those for pentagonal numbers are the series 1, 4, 7, 10… (the
common difference being~3), and so on.  In general, the successive
\emph{gnomonic} numbers for any polygonal number, say of $n$ sides,
have $n — 2$ for their common difference (Theon of Smyrna,
p.~34. 13–15).

\section*{Geometrical Algebra}

We have already seen (cf.\ part of the note on \prop{1}{47} and the
above note on the \emph{gnomon}) how the Pythagoreans and later Greek
mathematicians exhibited different kinds of numbers as forming
different geometrical figures.  Thus, says Theon of Smyrna (p.~36,
6—11), ``plane numbers, triangular, square and solid numbers, and the
rest, are not so called independently (\greek{κυρίως}) but in virtue
of their similarity to the areas which they measure; for~4, since it
measures a square area, is called square by adaptation from it, and 6
is called obiong for the same reason,'' A ``plane number'' is
similarly described as a number obtained by multiplying two numbers
together, which two numbers are sometimes spoken of as ``sides,''
sometimes as the ``length'' and ``breadth'' respectively, of the
number which is their product.

The \emph{product} of two numbers was thus represented geometrically
by the \emph{rectangle} contained by the straight lines representing
the two numbers respectively.  It only needed the discovery of
incommensurable or irrational straight lines in order to represent
geometrically by a rectangle the product of any two quantities
whatever, rational or irrational; and it was possible to advance from
a geometrical arithmetic to a geometrical \emph{algebra}, which indeed
by Euclid's time (and probably long before) had reached such a stage
of development that it could solve the same problems as our algebra so
far as they do not involve the manipulation of expressions of a degree
higher than the second.  In order to make the geometrical algebra so
generally effective, the theory of proportions was essential.  Thus,
suppose that $x$, $y$, $z$ etc.\ are quantities which can be
represented by straight lines, while $\alpha$, $\beta$, $\gamma$
etc.\ are coefficients which can be expressed by ratios between
straight lines.  We can then by means or Book~\book{6} find a single
straight line~$d$ such that
\[
    \alpha x + \beta y + \gamma z + \cdots = d.
\]
To solve the simple equation in its general form
\[
    \alpha x + a = b,
\]
where $a$ represents any ratio between straight lines also requires
recourse to the sixth Book, though, e.g., if $\alpha$ is $\frac{1}{2}$
or $\frac{1}{3}$ or any submultiple of unity, or if $\alpha$ is 2, 4
or any power of~2, we should not require anything beyond Book~\book{1}
for solving the aquation.  Similarly the general form of a quadratic
equation requires Book~\book{6} for its geometrical solution, though
particular quadratic equations may be so solved by means of
Book~\book{2} alone.

Besides enabling us to solve geometrically these particular quadratic
equations, Book~\book{2} gives the geometrical proofs of a number of
algebraical formulae.  Thus the first ten propositions give the
equivalent of the several identities
\begin{enumerate}

\item $ a(b + c + d + \cdots) = ab + ac + ad + \cdots $,

\item $ (a+b) a + (a+b) b = (a+b)^2 $,

\item $ (a+b) a = ab + a^2$,

\item $ (a+b)^2 = a^2 + b^2 + 2ab $.

\item $ab + \left( \frac{a+b}{2} - b\right) ^2 = \left(\frac{a+b}{2}\right)^2$,

or $(\alpha + \beta) (\alpha - \beta) + \beta^2 = \alpha^2$,

\item $ (2a + b) b + a^2 = (a+b)^2$,

or $ (\alpha + \beta) (\beta - \alpha) + \alpha^2 = \beta^2$,

\item $(a + b)^2 + a^2 = 2 (a + b) a + b^2$,

or $\alpha^2 + \beta^2 = 2\alpha \beta + (\alpha - \beta)^2$,

\item $4 (a + b) a + b^2 = \{(a+b) + a\}^2$,

or $4\alpha\beta + (\alpha - \beta)^2 = (\alpha + \beta)^2$,

\item $a^2 + b^2 = 2 \left\{
    \left(\frac{a+b}{2}\right)^2
  + \left(\frac{a+b}{2} - b\right)^2
\right\}$,

or $(\alpha + \beta)^2 + (\alpha - \beta)^2 = 2(\alpha^2 + \beta^2)$,

\item
$(2a + b)^2 + b^2 = 2 \{ a^2 + (a + b)^2 \}$,

or
$(\alpha + \beta)^2 + (\beta - \alpha)^2 = 2 (\alpha^2 + \beta^2)$.

\end{enumerate}
The form of these identities may of course be varied according to the
different symbols which we may use to denote particular portions of
the lines given in Euclid's figures.  They are, for the most part,
simple identities, but there is no reason to suppose that these were
the only applications of the geometrical algebra that Euclid and his
predecessors had been able to make.  We may infer the very contrary
from the fact that Apollonius in his \emph{Conics} frequently states
without proof much more complicated propositions of the kind.

It is important however to bear in mind that the whole procedure of
Book~\book{2} is \emph{geometrical}; rectangles and squares are shown
in the figures, and the equality of certain combinations to other
combinations is proved by those figures.  We gather that this was the
classical or standard method of proving such propositions, and that
the \emph{algebraical} method of proving them, with no figure except a
line with points marked thereon, was a later introduction.
Accordingly Eutocius' method of proving certain lemmas assumed by
Apollonius (\emph{Conics}, \r2.~23 and \r3.~29) probably represents
more nearly than Pappus' proof of the same the point of view from
which Apollonius regarded them.

It would appear that Heron was the first to adopt the
\emph{algebraical} method of demonstrating the propositions of
Book~\book{2}, beginning from the second, without figures, as
consequences of the first proposition corresponding to
\[
    a (b + c + d) = ab + ac + ad,
\]

According to an-Nairīzī (ed.\ Curtze, p.~89), Heron explains that it
is not possible to prove \prop{2}{1} without drawing a number of lines
(i.e.\ without actually drawing the rectangles), but that the
following propositions up to \prop{2}{10} inclusive can be proved by
merely drawing one line.  He distinguishes two varieties of the
method, one by \emph{dissolutio}, the other by \emph{compositio} by
which he seems to mean \emph{splitting-up} of rectangles and squares,
and \emph{combination} of them into others.  But in his proofs he
sometimes combines the two varieties.

When he comes to \prop{2}{11}, he says that it is not possible to do
without a figure because the proposition is a problem, which
accordingly requires an operation and therefore the drawing of a
figure.

The algebraical method has been preferred to Euclid's by some English
editors; but it should not find favour with those who wish to preserve
the essential features of Greek geometry as presented by its greatest
exponents, or to appreciate their point of view.

It may not be out of place to add a word with reference to the
geometrical equivalent of the algebraical operations.  The addition
and subtraction of quantities represented in the geometrical algebra
by lines is of course effected by producing the line to the required
extent or cutting off a portion of it.  The equivalent of
multiplication is the construction of the rectangle of which the given
lines are adjacent sides.  The equivalent of the division of one
quantity represented by a line by another quantity represented by a
line is simply the statement of a \emph{ratio} between lines on the
principles of Books \book{5} and~\book{6}.  The division of a product
of two quantities by a third is represented in the geometrical algebra
by the finding of a rectangle with one side of a given length and
equal to a given rectangle or square.  This is the problem of
\emph{application of areas} solved tn \prop{1}{44}, \prop*{1}{45}.
The addition and subtraction of products is, in the geometrical
algebra, the addition and subtraction of rectangles or squares; the
sum or difference can be transformed into a single rectangle by means
of the \emph{application of areas} to any line of given length,
corresponding to the algebraical process of finding a common measure.
Lastly, the extraction of the square root is, in the geometrical
algebra, the finding of a square equal to a given rectangle, which is
done in \prop{2}{14} with the help of~\prop{1}{47}.

\part{Book II. Propositions}

\begin{proposition}
\label{prop:II_1}

\begin{statement}
If there be two straight lines, and one of them be cut into any number
of segments whatever, the rectangle contained by the two straight
lines is equal to the rectangles contained by the uncut straight line
and each of the segments.
\end{statement}

\begin{proof}

Let $A$, $BC$ be two straight lines, and let $BC$ be cut at random at
the points $D$, $E$;
\0 I say that the rectangle contained by $A$, $BC$ is equal to the
rectangle contained by $A$, $BD$, that contained by $A$, $DE$ and that
contained by $A$,~$EC$.

For let $BF$ be drawn from~$B$ at right angles to~$BC$;
\using{\prop{1}{11}}
\0 let $BG$ be made equal to~$A$, \using{\prop{1}{3}}
\0 through $G$ let $GH$ be drawn is parallel to~$BC$,
\using{\prop{1}{31}}
\0 and through $D$, $E$, $C$ let $DK$, $EL$, $CH$ be
drawn parallel to~$BG$.

Then $BH$ is equal to $BK$, $DL$, $EH$.

Now $BH$ is the rectangle $A$, $BC$, for it is contained by $GB$,
$BC$, and $BG$ is equal to~$A$;

$BK$ is the rectangle $A$, $BD$, for it is contained by $GB$, $BD$,
and $BG$ is equal to $A$;

and $DL$ is the rectangle $A$, $DE$, for $DK$, that is $BG$
\using*{\prop{1}{34}}, is equal to~$A$.

Similarly also $EH$ is the rectangle $A$, $EC$.

Therefore the rectangle $A$, $BC$ is equal to the rectangle~$A$, $BD$,
the rectangle $A$, $DE$ and the rectangle $A$,~$EC$.

Therefore etc.
\end{proof}

\begin{annotations}

20. \textbf{the rectangle $A$, $BC$.}  From this point onward I shall
translate thus in cases where Euclid leaves out the word
\emph{contained} (\greek{περιεχόμενον}).  Though the word
``rectangle'' is also omitted in the Greek (the neuter article being
sufficient to show that the rectangle is meant), it cannot he
dispensed with in English.  De Morgan advises the use of the
expression ``the rectangle \emph{under} two lines.''  This does not
seem to me a very good expression, and, if used in a translation from
the Greek, it might suggest that \greek{ὑπό} in \greek{τὸ ὑπό} meant
\emph{under}, which it does not.

\end{annotations}

\begin{notes}

This proposition, the geometrical equivalent of the algebraical
formula
\[
    a(b + c + d + \cdots) = ab + ac + ad + \cdots,
\]
can, of course, easily be extended so as to correspond to the more
general algebraical proposition that the product of an expression
consisting of any number of terms added together and another
expression also consisting of any number of terms added together is
equal to the sum of all the products obtained by multiplying each term
of one expression by all the terms of the other expression, one after
another.  The geometrical proof of the more general proposition would
be effected by means of a figure showing all the rectangles
corresponding to the partial products, in the same way as they are
shown in the simpler case of \prop{2}{1}; the difference would be that
a series or parallels to $BC$ would have to be drawn as well as the
series of parallels to~$BF$.

\end{notes}

\end{proposition}

\begin{proposition}
\label{prop:II_2}

\begin{statement}
If a straight line be cut at random, the rectangle contained
by the whole and both of the segments is equal to the square on
the whole.
\end{statement}

\begin{proof}

For let the straight line $AB$ be cut at random at the
point~$C$;
\0 I say that the rectangle contained by $AB$, $BC$ together with
the rectangle contained by $BA$, $AC$ is equal
to the square on~$AB$.

For let the square $ADEB$ be described on $AB$ \using*{\prop{1}{46}},
and let $CF$ be drawn through C parallel to either AD or
BE. \using{\prop{1}{31}}

Then $AE$ is equal to $AF$, $CE$.

Now $AE$ is the square on $AB$;

$AF$$ is$ the rectangle contained by $BA$,
$AC$, for it is contained by $DA$, $AC$, and
$AD$ is equal to~$AB$;

and $CE$ is the rectangle $AB$, $BC$, for $BE$ is equal to~$AB$.

Therefore the rectangle $BA$, $AC$ together with the rectangle $AB$,
$BC$ is equal to the square on~$AB$.

Therefore etc.
\end{proof}

\begin{notes}

The \emph{fact} asserted in the enunciation of this proposition has
already been used in the proof of \prop{1}{47}; but there was no
occasion in that proof to observe that the two rectangles $BL$, $CL$
making up the square on~$BC$ are the rectangles contained by~$BC$ and
the two parts, respectively, into which it is divided by the
perpendicular from~$A$ on~$BC$.  It is this fact which it is necessary
to state in this proposition, in accordance with the plan of
Book~\book{2}.

The second and third propositions are of course particular cases of
the first.  They were no doubt separately enunciated by Euclid in
order that they might be immediately available for use hereafter,
instead of having to be deduced for the particular occasion from
\prop{2}{1}.  For, if they had not been thus separately stated, it
would scarcely have been practicable to quote them later without
explaining at the same time that they are included in \prop{2}{1} as
particular cases.  And, though the propositions are not used by Euclid
in the later propositions of Book~\book{2}, they are used afterwards
in \prop{13}{10} and \prop{9}{15} respectively; and they are of
extreme importance for geometry generally, being constantly used by
Pappus, for example, who frequently quotes the third proposition by
the Book and number.

Attention has been called to the fact that \prop{2}{1} is never used
by Euclid; and this may seem no less remarkable than the fact that
\prop{2}{2}, \prop*{2}{3} are not again used in Book~\book{2}.  But it
is important, I think, to observe that the proof of all the first ten
propositions of Book~\book{2} are practically independent of each
other, though the results are really so interwoven that they can often
be deduced from each other in a variety of ways.  What then was
Euclid's intention, first in inserting some propositions not
immediately required, and secondly in making the proofs of the first
ten practically independent of each other?  Surely the object was to
show the power of the \emph{method} of geometrical algebra as much as
to arrive at results.  From the point of view of illustrating the
\emph{method}, there can be no doubt that Euclid's procedure is far
more instructive than the semi-algebraical substitutes which seem to
rind a good deal of favour; practically it means that, instead of
relying on our memory of a few standard formulae, we can use the
machinery given us by Euclid's method to prove immediately \emph{ab
  initio} any of the propositions taken at random.

Let us contrast with Euclid's plan the semi-algebraical alternative.
One editor, for example, thinks that, as \prop{2}{1} is not used by
Euclid afterwards, it seems more logical to deduce from it those of
the subsequent propositions which can be readily so deduced.  Putting
this idea into practice, he proves \prop{2}{2} and \prop*{2}{3} by
quoting \prop{2}{1} then proves \prop{2}{4} by means of \prop{2}{1}
and~\prop*{2}{3}, \prop{2}{5} and~\prop*{2}{6} by means of
\prop{2}{1}, \prop*{2}{3} and~\prop*{2}{4}, and so on.  The result is
ultimately to deduce the whole of the first ten propositions from
\prop{2}{1}, which Euclid does not use at all; and this is to give an
importance to \prop{2}{1} which is altogether disproportionate and, by
starting with such a narrow foundation, to make the whole structure of
Book~\book{2} top-heavy.

Editors have of course been much influenced by a desire to make the
proofs of the propositions of Book~\book{2} easier, as they think, for
schoolboys.  But, even from this point of view, is it an improvement
to deduce \prop{2}{2} and~\prop*{2}{3} from \prop{2}{1} as
corollaries?  I doubt it.  For, in the first place, Euclid's figures
\emph{visualise} the results and so make it easier to grasp their
meaning; the truth of the propositions is made clear even to the eye.
Then, in the matter of brevity, to which such an exaggerated
importance is attached, Euclid's proof positively has the advantage.
Counting a capital letter or a collocation of such as one word, I
find, e.g., that Mr~H.~M. Taylor's proof of \prop{2}{2} contains 120
words, of which 8 represent the construction.  Euclid's as above
translated has 126 words, of which 22 are descriptive of the
construction; therefore the actual \emph{proof} by Euclid has 8 words
fewer than Mr Taylor's, and the extra words due to the construction in
Euclid are much more than atoned for by the advantage of picturing the
result in the figure.

The advantages then which Euclid's method may claim are, I think,
these: in the case of \prop{2}{2}, \prop*{2}{3} it produces the result
more easily and clearly than does the alternative proof by means of
\prop{2}{1}, and, in its general application, it is more powerful in
that it makes us independent of any recollection of results.

\end{notes}

\end{proposition}

\begin{proposition}
\label{prop:II_3}

\begin{statement}
If a straight line be cut at random, the rectangle contained
by the whole and one of the segments is equal to the rectangle
contained by the segments and the square on the aforesaid
segment.
\end{statement}

\begin{proof}

For let the straight line AB be cut at random at C;
I say that the rectangle contained by AB, BC is equal to the
rectangle contained by AC, CB together
with the square on BC.

For let the square CDEB be de-
scribed on CB; [1. 46]
let ED be drawn through to F,
and through A let AF be drawn parallel
to either CD or BE. [1. 31]

Then AE is equal to AD, CE.

Now AE is the rectangle contained by AB, BC, for it is
contained by AB, BE, and BE is equal to BC;

AD is the rectangle AC, CB, for DC is equal to CB;

and DB is the square on CB.
Therefore the rectangle contained by AB, BC is equal to
the rectangle contained by AC, CB together with the square
on BC.

Therefore etc.
\end{proof}

\begin{notes}

If we leave out of account the contents of Book 11. itself and merely look
to the applicability of propositions to general use, this proposition and the
preceding are, as already indicated, of great importance, and particularly so to
the semi-algebraical method just described, which seems to have found its first
exponents in Heron and Pappus. Thus the proposition that the difference of
the squares on two straight Urns is equal to the rectangle contained by the sum
and the differtna of the straight lines, which is generally given as equivalent to
ii. 5, 6, can be proved by means of ii. i, z, 3, as shown

by Lardner. For suppose the given straight lines are ft C P

A B, BC, the latter being measured along BA.

, Then, by n. 2, the square on AB is equal to. the sum of the rectangles
AB, BC and AB, AC.

By 11. 3, the rectangle AB, BC is equal to the sum of the square on BC
and the rectangle AC, CB.

Therefore the square on AB is equal to the square BC together with the
sum of the rectangles AC, AB and AC, CB.

But, by 11. 1, the sum of the latter rectangles is equal to the rectangle
contained by AC and the sum of A3, BC, i.e.\ the rectangle contained by the
sum and difference of AB, BC.

Hence the square en AB is equal to the square on BC and the rectangle
contained by the sum and difference of AB, BC

that is, the difference of the squares on AB, BCis equal to the rectangle
contained by the sum and difference of AB, BC.

\end{notes}

\end{proposition}

\begin{proposition}
\label{prop:II_4}

\begin{statement}
If a straight line be cut at random, the square on the whole
is equal to the squares on the segments and twice the rectangle
contained by the segments.
\end{statement}

\begin{proof}

For let the straight line AB be cut at random at C;

s I say that the square on AB is equal to the squares on A C,
CB and twice the rectangle contained
by AC, CB.

For let the square ADEB be de-
scribed on AB, [1. 46]

10 let BD be joined;
through C let CF be drawn parallel to
either AD or EB,

and through G let HK be drawn parallel
to either AB or DE. [1. 31]

is Then, since CF is parallel to AD,
and BD has fallen on them,

the exterior angle CGB is equal to the interior and opposite
angle ADB. [t 19]

But the angle ADB is equal to the angle ABD,

io since the side BA is also equal to AD; [f. 5]

therefore the angle CGB is also equal to the angle GBC,
so that the side BC is also equal to the side CG. [1. 6]

But CB is equal to GK, and CG to KB; [i. 34]

therefore GK is also equal to KB;
*s therefore CGKB is equilateral.

I say next that it is also right-angled.
For, since CG is parallel to BK,

the angles KBC, GCB are equal to two right angles.

[l. 2 9 ]

But the angle KBC is right;
3° therefore the angle BCG is also right,

so that the opposite angles CGK, GKB are also right.

b- 34]

Therefore CGKB is right-angled;
and it was also proved equilateral;

therefore it is a square;
35 and it is described on CB.
For the same reason

HF is also a square;
and it is described on HG, that is AC. [1. 34]

Therefore the squares HF, AX'' are the squares on AC, CB.
4° Now, since A G is equal to GE,
and A G is the rectangle A C, CB, for GC is equal to CB,
therefore GE is also equal to the rectangle AC, CB.
Therefore AG, GE are, equal to twice the rectangle AC,
CB.
+s But the squares HF, CK are also the squares on AC, CB;
therefore the four areas HF, CK, AG, GE are equal to
the squares on AC, CB and twice the rectangle contained by
AC, CB.

But HF, CK, AG, GE are the whole A DEB,
50 which is the square on AB.

Therefore the square on AB is equal to the squares on
AC, CB and twice the rectangle contained by AC, CB.
Therefore etc.\ Q.E.D.-
\end{proof}

\begin{annotations}

1. twice the rectangle contained by the segments. By a carious idiom ihjs is in
Greek ``the rectangle Mi contained by the segments.'' Similarly ``twice the rectangle
contained by AC, CB'' is expressed as ``the rectangle twin contained by AC, CB'' (rilli
vri>* Ar, TB ittpaxbfAtirw Apfhy4vtQi'),

35, 38. described, jo, 43. the squares (before ``on''). These words are not in the
Greek, which limply says that the squares ``are on ``(tlalr iri) their respective sides.

46. areas. It is necessary to supply some substantive (the Greek leaves it to be under-
stood); and I prefer ``areas ``to ``figures.''

\end{annotations}

\begin{notes}

The editions of the Greek text which preceded that of E. F. August
(Berlin, 1826 — 9) give a second proof of this proposition introduced by the
usual word dXAun or ``otherwise thus.'' Heiberg follows August in omitting
this proof, which is attributed to Theon, and which is indeed not worth
reproducing, since it only differs from the genuine proof in that portion of it
which proves that CGKB is a square. The proof that CGKB is equilateral
is rather longer than Euclid's, and the only interesting point to notice is that,
whereas Euclid still, as in 1. 46, seems to regard it as necessary to prove that
all the angles of CGKB are right angles before he concludes that it is right-
angled, Theon says simply ``And it also has the angle CBK right; therefore
CK is a square.'' The shorter form indicates a legitimate abbreviation of the
genuine proof; because there can be no need to repeat exactly that part of the
proof of 1. 46 which shows that all the angles of the figure there constructed
are right when one is.

There is also In the Greek text a Porism which is undoubtedly interpolated:
``From this it is manifest that in square areas the parallelograms about the
diameter are squares.'' Heiberg doubted its genuineness when preparing his
edition, and conjectured that it too may have been added by Theon; but the
matter is placed beyond doubt by a papyrus-fragment referred to already (see
Heiberg, Paralipomena zu Euklid, in Hermes xxxvm., 1903, p.~48) in which
the Porism was evidently wanting. It is the only Porism in Book 11., but
does not correspond to Proclus* remark (p.~304, 2) that ``the Porism found in
the second book belongs to a problem.'' Heiberg regards these words as
referring to the Porism to iv. 15, the correct reading having probably been not

itmifnf but 8', i.e.\ TwapTiii.

The semi -algebraical proof of tins proposition is very easy, and is of course
old enough, being found in Clavius and in most later editions. It proceeds
thus.

By 11, 1, the square on AB is equal to the sum of the rectangles AB, AC
and AB, CB.

But, by 11. 3, the rectangle AB, AC is equal to the sum of the square on
AC and the rectangle AC, CB;

while, by 11, 3, the rectangle AB, CB is equal to the sum of the square on
BC and the rectangle AC, CB.

Therefore the square on AB is equal to the sum of the squares on
AC, CB and twice the rectangle AC, CB.

The figure of the proposition also helps to visualise, in the orthodox
manner, the proof of the theorem deduced above from 11. 1 — 3, viz.\ that the
difference of the squares on two given straight lines is equal to the rectangle
contained by the sum and the difference of the lines.

For, if the lines be AB, BC respectively, the shorter of the lines being
measured along BA, the figure shows that

the square AE is equal to the sum of the square CK and the rectangles
AF,FK,

that is, the square on A B is equal to the sum of the square on BC and
the rectangles AB, AC and AC, BC.

But the rectangles AB, AC and BC, AC ait, by 11. 1, together equal to
the rectangle contained by Cand the sum of AB, BC,
i.e to the rectangle contained by the sum and difference of AB, BC.

Whence the result follows as before.

The proposition 11. 4 can also be extended to the case where a straight
line is divided into any number of segments; for the figure will show in like
manner that the square on the whole line is equal to the sum of the squares
on all the parts together with twice the rectangles contained by every pair of
the parts.

\end{notes}

\end{proposition}

\begin{proposition}
\label{prop:II_5}

\begin{statement}
If a straight line be cut into equal and unequal segments,
the rectangle contained by the unequal segments of the whole
together with the square on the straight line between the
points of section is equal to the square on the half
\end{statement}

\begin{proof}

For let a straight line AB be cut into equal segments
at C and into unequal segments at D;

I say that the rectangle contained by AD, DB together with
the square on CD is equal to the square on CB.

For let the square CEFB be described on CB, [1. 46]

and let BE be joined;

through D let DG be drawn parallel to either CE or BF,
through H again let KM be drawn parallel to either AB or
EF t

and again through A let AK be drawn parallel to either CL
or BM. [1.31]

Then, since the complement CH is equal to the comple-
ment HF   [1. 43]
let DM be added to each;

therefore the whole CM is equal to the whole DF.

But CM is equal to AL,

since AC is also equal to CB; • [1. 3 6 ]

therefore AL is also equal to DF.
Let CH be added to each;

therefore the whole AH is equal to the gnomon NOP.

But AH is the rectangle AD, DB, for DH is equal to
DB,

therefore the gnomon NOP is also equal to the rectangle
AD, DB.

Let LG, which is equal to the square on CD, be added to
each;

therefore the gnomon NOP and LG are equal to the
rectangle contained by AD, DB and the square on CD.

But the gnomon NOP and LG are the whole square
CEFB, which is described on CB;

therefore the rectangle contained by AD, DB together
with the square on CD is equal to the square on CB.

Therefore etc.\
\end{proof}

\begin{annotations}

3. between the points of section, literally ``between the icct/mr,'' ihc word being
the same (rofii) is that used of  conic irriim.

It will be observed that the gnomon is indicated in the figure by three separate letters
and a dotted carve. This is no doubt a clearer way of showing what exactly the gnomon is
than the method usual in our text -books. In this particular case the figure of the ttss. has
hiv M's in it, the gnomon being MN2. I have corrected the lettering to avoid confusion.

\end{annotations}

\begin{notes}

It is easily seen that this proposition and [he next give exactly the
theorem already alluded to under the last propositions, namely that the
difference of the squares on two straight lines is equal te the rectangle contained
by their sum and difference. The two given lines are, in 11. 5, the lines CB
and CD, and their sum and difference are respectively equal to AD and DB.
To show that 11. 6 gives the same theorem we have only to make CD the
greater line and CB the less, i.e.\ to
draw CD' equal to CB, measure . cob

CB along it equal to CD, and then '

produce B C to A', making A'C equal a| gj p' O *

to BC\ whence it is immediately clear

that A' D' on the second line is equal

to AD on the first, while DB is also equal to DB, so that the rectangles

AD, DB and A'D'', DB are equal, while the difference of the squares on

CB, CD is equal to the difference of the squares on CD, CB.

Perhaps the most important fact about 11. 5, 6 is however their bearing on
the

Geometrical solution of a quadratic equation.

Suppose, in the figure of 11. 5, that AB = a, DB = x;
then «*-*= the rectangle AH

= the gnomon NOP.

Thus, if the area of the gnomon is given (=*', say), and if a is given
(= AB), the problem of solving the equation

is, in the language of geometry, To a given straight lint (a) to apply a rectangle
which shall Si equal to a given square (c5*) and shall fall snort by a square figure,
Le. to construct the rectangle AH at the gnomon NOP,

Now we are told by Proclus (on 1. 44) that ``these propositions are ancient
and the discoveries of the Muse of the Pythagoreans, the application of
areas, their exceeding and their falling-short'' We can therefore hardly
avoid crediting the Pythagoreans with the geometrical solution, based upon
ti. 5, 6, of the problems corresponding to the quadratic equations which
are directly obtainable from them. It is certain that the Pythagoreans solved
the problem in n. 1 1, which corresponds to the quadratic equation

a (a — *) = ar 1 ,

and Simson has suggested the following easy solution of the equation now in
question,

ax~x*=P,
on exactly similar lines.

Draw CO perpendicular to AB and equal to b; produce OC to A 7 so
that ON= CB (or a); and with O as centre
and radius ON describe a circle cutting CB
in D.

Then DB (or x) is found, and therefore
the required rectangle AH.

For the rectangle AD, DB together with
the square on CD is equal to the square on
CB, [n. S ]

i.e.\ to the square on OD,
i.e.\ to the squares on OC, CD; [l. 47]
whence the rectangle AD, DB is equal to the square on OC,
or ax - x* = b*.

It is of course a necessary condition of the possibility of a real solution
that P must not be greater that (Ja)'. This condition itself can easily be
obtained from Euclid's proposition; for, since the sum of the rectangle AD,
DB and the square on CD is equal to the square on CB, which is constant,
it follows that, as CD diminishes, i.e.\ as D moves nearer to C, the rectangle
AD, DB increases and, when D actually coincides with C, so that CD
vanishes, the rectangle AD, DB becomes the rectangle AC, CB, i.e.\ the
square on CB, and is a maximum. It wilt be seen also that the geometrical
solution of the quadratic equation derived from Euclid does not differ from
our practice of solving a quadratic by completing the square on the side
containing the terms in x* and x.

But, while in this case there are two geometrically real solutions (because
the circle described with ON as radius will not only cut CB in D but will
also cut AC in another point E), Euclid's fig a re corresponds to one only of
the two solutions. Not that there is any doubt that Euclid was aware that the
method of solving the quadratic gives two solutions; he could not fail to see
that x = BE satisfies the equation as well as x = BD. If however he hud
actually given us the solution of the equation, he would probably have
omitted to specify the solution * = BE because the rectangle found by means
of it, which would be a rectangle on the base AE (equal to BD) and with
altitude EB (equal to AD), is really an equal rectangle to that corresponding
to the other solution x = BD; there is therefore no real object in distinguishing
two solutions. This is easily understood when we regard the equation as a
statement of the problem of finding two magnitudes when their sum (a) and
product (b*) are given, i.e.\ as equivalent to the simultaneous equations

x+y = a,
xy = b*.

These symmetrical equations have really only one solution, as the two apparent
solutions are simply the result of interchanging the values of x and y. This
form of the problem was known to Euclid, as appears from the Data, Prop.
85, which states that, If two straight lines contain a parallelogram given in
magnitude in a given angle, and if the sum of them be given, then shall each
of them be given.

This proposition then enables us to solve the problem of finding a
rectangle the area and perimeter of which are both given; and it also enables
us to infer that, of all rectangles of given perimeter, the square has the
greatest area, while, the more unequal the sides are, the less is the area.

If in the figure of 11. 5 we suppose that AD=a, BD=b, we find that
CB = (« + *)/ 2 and CD = (a—b)li, and we may state the result of the
proposition in the following algebraical form

ffl-ffi-*

This way of stating it (which could hardly have escaped the Pythagoreans)
gives a ready means of obtaining the two rales, respectively attributed to the
Pythagoreans and Plato, for finding integral square numbers which are the
sum of two other integral square numbers. We have only to make ab a
perfect square in the above formula. The simplest way in which this can be
done is to put a = n', b=i, whence we have

and in order that the first two squares may be integral a 1 , and therefore n,
must be odd Hence the Pythagorean rule.
Suppose next that a = in*, b- , and we have
(«*+.)'-(«*-i)' = 4A
whence Plato's rale starting from an even number in.

\end{notes}

\end{proposition}

\begin{proposition}
\label{prop:II_6}

\begin{statement}
If a straight line be bisected and a straight line be added
to ii in a straight line, the rectangle contained by the whole
with the added straight line and the added straight line together
with the square on the half is equal to the square on the
straight line made up of the half and the added straight
line.
\end{statement}

\begin{proof}

For let a straight line AB be bisected at the point C, and
let a straight line BD be added to it in a straight line;

I say that the rectangle contained by AD, DB together
with the square on CB is equal to the square on CD.

For let the square CEFD be described on CD, [1. 46]

and let DE be joined;

through the point B let BG be drawn parallel to either EC or
DF,
through the point H let KM be drawn parallel to either A 3
or EF,

and further through A let y4A*
be drawn parallel to either CL
or DM. [i. 31]

Then, since 4C is equal
toC,

j4Z is also equal to CH. [1. 36]
But CH is equal to HF Ql 43]
Therefore AL is also equal
toJKE
Let CM be added to each;

therefore the whole AM is equal to the gnomon NOP.
But /f4f is the rectangle AD, DB,

for DM is equal to Z?/?;

therefore the gnomon NOP is also equal to the rectangle
AD, DB.

Let LG, which is equal to the square on BC, be added
to each;

therefore the rectangle contained by AD, DB together
with the square on CB is equal to the gnomon NOP and LG.

But the gnomon NOP and LG are the whole square
CEFD, which is described on CD;

therefore the rectangle contained by AD, DB together
with the square on CB is equal to the square on CD.

Therefore etc.
\end{proof}

\begin{notes}

In this case the rectangle AD, DB is ``a rectangle applied to a given
straight line (AB) but exceeding by a square (the side of which is equal to
BD) ``; and the problem suggested by 11. 6 is to rind a rectangle of this
description equal to a given area, which we will, for convenience, suppose to
be a square; Le., in the language of geometry, to apply to a given straight
line a rectangle which shall be equal to a given square and shall exceed by a
square figure.

We suppose that in Euclid's figure AB = a, BD=x; then, if the given
square be F, the problem is to solve geometrically the equation

ax + jc* = •.
The solution of a problem theoretically equivalent to the solution of a
quadratic equation of this kind is presupposed in the fragment of Hippocrates'
Quadrature of lunes preserved in a quotation by Simplicius (Comment, in
Aristot. Phys. pp.~61 — 68, ed. Diels) from Eudemus' History of Geometry, In
this fragment Hippocrates (5th cent. b,c.) assumes the following construction.

AB being the diameter and O the centre of a semicircle, and C being the
middle point of OB and CD at right
angles to AB, a straight line of length
such that its square is \ times the square
on the radius (i.e.\ of length aj, where
a is the radius) is to be so placed, as EF,
between CD and the circumference AD
Jiat it ``verges towards B,'' that is, EF
when produced passes through B.

Now the right-angled triangles BFC,
BAE are similar, so that

BF:BC=BA .BE,

and therefore the rectangle BE, BF= rect BA, BC

= sq. on BO.

In other words, EF ( = o ,/f) being given in length, BF ( = x, say) has
to be found such that

(7t « + *)=**;
or the quadratic equation

,/£ ax + ar* = a''
has to be solved.

A straight line of length ajl would easily be constructed, for, in the
figure, CD*=AC. CB = a\ or CD=aJs, and aj\ is the diagonal of
a square of which CD is t,he side.

There is no doubt that Hippocrates could have solved the equation by
the geometrical construction given below, but he may have contemplated, on
this occasion, the merely mcehanital process of placing the straight line of the
length required between CD and the circumference AD and moving it until
E, F, B were in a straight line. Zeuthen (Die Lehre von den Kegelschnittm
im Altertum, pp.~370, 27 r) thinks this probable because, curiously enough,
the fragment speaks immediately afterwards of ``joining B to F.''

To solve the equation

we have to find the rectangle AH, or the
gnomon NOP, which is equal in area to £* and
has one of the sides containing the inner right
angle equal to CB or a. Thus we know
(Ja)* and £*, and we have to find, by \prop{1}{47},
a square equal to the sum of two given
squares.

To do this Simson draws BQ at right
angles to AB and equal to b, joins CQ and,
with centre C and radius CQ, describes a
circle cutting AB produced in D. Thus
BD, or x, is found.

Now the rectangle AD, DB together with the square on CB
is equal to the square on CD,
i.e.\ to the square on CQ,
i.e.\ to the squares on CB, BQ.

Therefore the rectangle AD, DB is equal to the square on BQ, that is,

jx + x* — fi.

From Euclid's point of view there would only be one solution in this case.
This proposition enables us also to solve the equation
x* — ax-
in a similar manner.

We have only to suppose that AB = a, and AD (instead of BD) = x; then
.» x*—ax = the gnomon.

To find the gnomon we have its area (P) and the area, CB 1 or (|o)*, by
which the gnomon differs from CD 1 . Thus we can find D (and therefore
AD or x) by the same construction as that just given.

Converse propositions to 11. 5, 6 are given by Pappus (vii. pp.~948—950)
among his lemmas to the Conits of Apollonius to the effect that,
(1) if D be a point dividing AB unequally, and C another point on AB
such that the rectangle AD, DB together with the square on CD is
equal to the square on AC, then

Cis equal to CB;

(a) if D be a point on AB produced, and C a point on AB such that the
rectangle AD, DB together with the square on CB is equal to the
square on CD, then

AC is equal to CB.

\end{notes}

\end{proposition}

\begin{proposition}
\label{prop:II_7}

\begin{statement}
If a straight line be cut at random, the square on the
whole and that on one of the segments both together are equal
to twice the rectangle contained by the whole and the said
segment and the square on the remaining segment.
\end{statement}

\begin{proof}

For let a straight line AB be cut at random at the point C;

I say that the squares on AB, BC are equal to twice the
rectangle contained by AB, BC and the
square on CA.

For let the square ADEB be
described on AB, [1. 46]

and let the figure be drawn.

Then, since AG is equal to GE, [i. 43]
let CF be added to each;

therefore the whole AF is equal to
the whole CE,

Therefore AF, CE are double of
AF.

But AF, CF are the gnomon KLM and the square CF;
therefore the gnomon KLM and the square CF are double
of AF.

But twice the rectangle AB, BC is also double of AF;
for BF is equal to BC;

therefore the gnomon KLM and the square CF are equal to
twice the rectangle AB, BC.

Let DG, which is the square on AC, be added to each;
therefore the gnomon KLM and the squares BG, GD are
equal to twice the rectangle contained by AB, BC and the
square on AC.

But the gnomon KLM and the squares BG, GD are the
whole ADEB and CF,

which are squares described on AB, BC;
therefore the squares on AB, BC are equal to twice the
rectangle contained by AB, BC together with the square on
AC.

Therefore etc.

q. £. D.
\end{proof}

\begin{notes}

An interesting variation of the form of this proposition may be obtained by
regarding AB, BC as two given straight lines of which AS is the greater, and
AC as the difference between the two straight lines. Thus the proposition
shows that the squares on two straight lines are together equal to twice the
rectangle contained by them and the square on their difference. That is, the
square en the different of two straight lines is equal to the sum of the squares on
the straight lines diminished by twite the rectangle contained by them. In other
words, just as 11. 4 is the geometrical equivalent of the identity

(a + b) , d , + b t + 2ab,
so 11. 7 proves that

(a -t) , = a* + P-tab.
The addition and subtraction of these formulae give the algebraical equivalent
of the propositions it. 9, 10 and 11. 8 respectively; and we have accordingly
a suggestion of alternative methods of proving those propositions.

\end{notes}

\end{proposition}

\begin{proposition}
\label{prop:II_8}

\begin{statement}
If a straight line be cut at random, four times the rectangle
contained by the whole and one of the segments together with
the square on the remaining segment is equal to the square
described on the whole and the aforesaid segment as on one
straight line.
\end{statement}

\begin{proof}

For let a straight line AB be cut at random at the point C;

1 say that four times the rectangle contained by AB, BC
together with the square on AC is equal to the square
described on AB, BC as on one straight line.

For let [the straight line] .5Z> be produced in a straight
line [with AB, and let BD be
made equal to CB;
let the square A FFD be described
on AD, and let the figure be
drawn double.

Then, since CB is equal to BD,
while CB is equal to GK, and
BD to AW,
therefore <7T is also equal to KN.

For the same reason
QR is also equal to RP.

And, since BC is equal to BD, and 6l'' to KN,
therefore CK is also equal to KD, and GA to RN [i. 36]

But CK is equal to AjV, for they are complements of the
parallelogram CP; [1. 43]

therefore KD is also equal to GR

therefore the four areas DK, CK, GR, RN are equal to one
another.

Therefore the four are quadruple of CK.

Again, since CB is equal to BD,
while BD is equal to BK, that is CG,
and CB is equal to GK, that is GQ,

therefore CG is also equal to GQ.

And, since CG is equal to GQ, and QR to RP,

AG'xs also equal to MQ, and £>Z. to RF. [1. 36]

But J/0 is equal to QL, for they are complements of the
parallelogram ML; [1. 43)

therefore AG is also equal to RF;
therefore the four areas AG, MQ, QL, RF are equal to one
another.

Therefore the four are quadruple of AG.
But the four areas CK, KD, GR, RN were proved to be
quadruple of CK;

therefore the eight areas, which contain the gnomon
STU, are quadruple of AK,

Now, since AK is the rectangle AB, BD, for BK is equal
to BD,

therefore four times the rectangle AB, BD is quadruple of
AK.

But the gnomon STU was also proved to be quadruple
otAK;

therefore four times the rectangle AB, BD is equal to the
gnomon STU.

Let OH, which is equal to the square on AC, be added
to each;

therefore four times the rectangle AB, BD together with
the square on AC is equal to the gnomon STU and OH.

But the gnomon STU and OH are the whole square
AEFD,

which is described on AD •
therefore four times the rectangle AB, BD together with
the square on AC is equal to the square on AD

But BD is equal to BC;
therefore four times the rectangle contained by AB, BC
together with the square on AC is equal to the square on
AD, that is to the square described on AB and BC as on
one straight line.

Therefore etc.
\end{proof}

\begin{notes}

This proposition is quoted by Pappus (p.~418, ed. Hultsch) and is used
also by Euclid himself in the Data, Prop.~86. Further, it is of decided use
in proving the fundamental property of a parabola.

Two alternative proofs are worth giving.

The first is that suggested hy the consideration mentioned in the last
note, though the proof is old enough, being given by Clavius and others. It
is of the semi-algebraical type.

Produce AB to D (in the figure of the pro-
position), so that BD is equal to BC.

By 11. 4, the square on AD is equal to the
squares on AB, BD and twice the rectangle AB,
BD, i.e.\ to the squares on AB, BC and twice
the rectangle AB, BC.

By 11. 7, the squares on AB, BC are equal to
twice the rectangle AB, BC together with the
square on AC

Therefore the square on AD is equal to four
times the rectangle AB, BC together with the ``

square on AC.

The second proof is after the manner of Euclid but with a difference.
Produce BA to D so that AD is equal to BC On BD construct the square
BEFD.

Take BG, Elf, FK each equal to BC or AD, and draw ALP, HNM
parallel to BE and GML, KPW parallel to BD.

Then it can be shown that each of the rectangles BL, AK, FN, EM is
equal to the rectangle AB, BC, and that PM is equal to the square on AC.

Therefore the square on BD is equal to four times the rectangle AB,
BC together with the square on AC.

\end{notes}

\end{proposition}

\begin{proposition}
\label{prop:II_9}

\begin{statement}
If a straight line be cut into equal and unequal segments,
the squares on the unequal segments of the whole are double
of the square on the half and of the square on the straight line
between the points of section.
\end{statement}

\begin{proof}

For let a straight line AB be cut into equal segments
at C, and into unequal segments at D

I say that the squares on AD, DB are double of the
squares on AC, CD.

For let CE be drawn from
C at right angles to AB,
and let it be made equal to
either AC at CB;
let EA, EB be joined,
let DF be drawn through D
parallel to EC,

and FG through F parallel to
AB,
and let AF be joined.

Then, since AC is equal to CE,
the angle EAC is also equal to the angle A EC.

And, since the angle at C is right,

the remaining angles EAC, AEC are equal to one
right angle. ['• 3*]

And they are equal;

therefore each of the angles CEA, CAE is half a right
angle.

For the same reason

each of the angles CEB, EBC is also half a right angle;

therefore the whole angle AEB is right
And, since the angle GEF is half a right angle.

and the angle EGF is right, for it is equal to the interior and
opposite angle ECB, [l ag]

the remaining angle EFG is half a right angle; [1. 3 2 ]
therefore the angle GEF is equal to the angle EFG,

so that the side EG is also equal to GF. [1. 6]

Again, since the angle at B is half a right angle,

and the angle FDB is right, for it is again equal to the interior

and opposite angle ECB, [i- *9]

the remaining angle BFD is half a right angle; [1. 3*]

therefore the angle at B is equal to the angle DFB,

so that the side FD is also equal to the side DB. [1. 6]
Now, since AC is equal to CE,

the square on AC is also equal to the square on CE;
therefore the squares on AC, CE are double of the square
on AC.

But the square on EA is equal to the squares on AC, CE t
for the angle A CE is right; [1. 47]

therefore the square on EA is double of the square on A C.
Again, since EG is equal to GF,
the square on EG is also equal to the square on GF;

therefore the squares on EG, GF are double of the square on
GF.

But the square on EF is equal to the squares on EG, GF;
therefore the square on EF is double of the square on GF.

But GF is equal to CD; [j. 34)

therefore the square on EF is double of the square on CD.

But the square on EA is also double of the square on AC;

therefore the squares on AE, EFre double of the squares
on AC, CD.

And the square on AF is equal to the squares on AE, EF,
for the angle AEF is right; [1. 47]

therefore the square on AF is double of the squares on AC,
CD.

But the squares on AD, DF are equal to the square on
AF, for the angle at D is right; [1. 47]

therefore the squares on AD, DF are double of the squares
on AC, CD.

And DF is equal to DB;
therefore the squares on AD, DB are double of the squares
on AC, CD.

Therefore etc.
\end{proof}

\begin{notes}

It is noteworthy that, while the first eight propositions of Book it. are
proved independently of the Pythagorean theorem i. 47, all the remaining
propositions beginning with the 9th are proved by means of it. Also the 9th
and 10th propositions mark a new departure in another respect; the method
of demonstration by showing in the figures the various rectangles and squares
to which the theorems relate is here abandoned.

The 9th and 10th propositions are related to one another in the same way
as the 5th and 6th; they really prove the same result which can, as in the
earlier case, be comprised in a single enunciation thus: The sum of the squares
on the sum and difference of two given straight lines is equal to twice the sum of
the squares on the lines.

The semi-algebraical proof of Prop, 9 is that suggested by the remark on
the algebraical formulae given at the end of the note on 11. 7. It applies
with a very slight modification to both u. 9 and 11. 10. We will put in
brackets the variations belonging to 11. 10.

The first of the annexed lines is the figure  COB

for 11. 9 and the second for ti. 10. '

By 11. 4, the square on AD is equal to a C a D

the squares on AC, CD and twice the ¥ >

rectangle AC, CD.

By 11. 7, the squares on CB, CD (CD, C£) are equal to

twice the rectangle CB, CD together with the square on BD.

By addition of these equals crosswise,
the squares on AD, DB together with twice the rectangle CB, CD are
equal to the squares on AC, CD, CB, CD together with twice
the rectangle AC, CD.

But AC, CB are equal, and therefore the rectangles AC, CD and CB,
CD are equal.

Taking away the equals, we see that

the squares on AD t DB are equal to the squares on AC, CD, CB, CD,

i.e.\ to twice the squares on AC, CD.
To show also that the method of geometrical algebra illustrated by
11. 1 — 8 is still effective for the purpose of
proving 11. 9, 10, we will now prove 11. 9 in
that manner.

Draw squares on AD, DB respectively
as shown in the figure. Measure DH along
DE equal to CD, and HL along HE also
equal to CD.

Draw HK, LNO parallel to EF, and
CNM parallel to DE.

Measure NP along NO equal to CD, F ``Q M E

and draw PQ parallel to DB.

Now, since AD, CD are respectively equal to DE, DH,
HE is equal to AC or CB;
and, since HL is equal to CD, LE is equal to DB.

Similarly, since each of the segments EM, MQ is equal to CD,
EQ is equal to EL or BD.

Therefore OQ is equal to the square on DB.

We have to prove that the squares on AD, DB are equal to twice the
squares on AC, CD.

Now the square on AD includes KM (the square on AC) and CH, HN
(that is, twice the square on CD).

Therefore we have to prove that what is left over of the square on AD
together with the square on DB is equal to the square on AC.

The parts left over are the rectangles CK and NE, which are equal to
KJV, PM respectively.

But the latter with the square on DB are equal to the rectangles KN,
BMand the square OQ,

i.e.\ to the square KM, or the square on AC.

Hence the required result follows.

\end{notes}

\end{proposition}

\begin{proposition}
\label{prop:II_10}

\begin{statement}
If a straight line be bisected, and a straight line be added
to it in a straight line, the square on the whole with the added
straight line and the square n the added straight line both
together are double of the square on the half and of the square
described on the straight line made up of the half and the
added straight line as on one straight line.
\end{statement}

\begin{proof}

For let a straight line AB be bisected at C, and let a
straight line BD be added to it in a straight line;

I say that the squares on AD, DB are double of the
squares on AC, CD.

For let CE be drawn from
the point C at right angles to
AB [1. 11], and let it be made
equal to either AC ox CB [1. 3];

let EA, EB be joined;

through E let EF be drawn
parallel to AD,

and through D let FD be drawn
parallel to CE. [1. 31]

Then, since a straight line EF falls on the parallel straight
lines EC, FD,

the angles CEF, EFD are equal to two right angles; [i. a]
therefore the angles FEB, EFD are less than two right
angles.

But straight lines produced from angles less than two
right angles meet; [i. Post 5]

therefore EB, FD, if produced in the direction B, D, will
meet.

Let them be produced and meet at G,
and let AG be joined.

Then, since A C is equal to CE,
the angle EA C is also equal to the angle AEC; [1. 5]

and the angle at C is right;

therefore each of the angles EAC, AEC is half a right
angle. [1. 32]

For the same reason

each of the angles CEB, EBC is also half a right angle;
therefore the angle AEB is right

And, since the angle EBC is half a right angle,
the angle DBG is also half a right angle. [1. 15]

Rut the angle BDG is also right,
for it is equal to the angle DCE, they being alternate; [1. 19]

therefore the remaining angle DGB is half a right angle;

['• 3'']
therefore the angle DGB is equal to the angle DBG,

so that the side BD is also equal to the side GD, [1. 6]

Again, since the angle EGF is half a right angle,
and the angle at F is right, for it is equal to the opposite
angle, the angle at C, [1. 34]

the remaining angle FEG is half a right angle; [1. 3*]

therefore the angle EGF is equal to the angle FEG,

so that the side GF is also equal to the side EF. [1. 6]

Now, since the square on EC is equal to the square on
CA,
the squares on EC, CA are double of the square on CA.

But the square on EA is equal to the squares on EC, CA;

b- «]

therefore the square on EA is double of the square on A C.

[a k 1]

Again, since FG is equal to EF,
the square on FG is also equal to the square on FE;
therefore the squares on GF, FE are double of the square on
EF

But the square on EG is equal to the squares on GF, FE;

[>• 47]
therefore the square on EG is double of the square on EF.

And EF is equal to CD; [i- 34]

therefore the square on EG is double of the square on CD.
But the square on EA was also proved double of the square
on AC;

therefore the squares on AE, EG are double of the squares
on AC, CD.

And the square on AG is equal to the squares on AE,
EG; [i. 47]

therefore the square on AG is double of the squares on AC,
CD.

But the squares on AD, DG are equal to the square on AG;

[•47]

therefore the squares on AD, DG are double of the squares
on AC, CD.

And DG is equal to DB;
therefore the squares on AD, DB are double of the squares
on AC, CD.

Therefore etc.
\end{proof}

\begin{notes}

The alternative proof of this proposition by means of the principles
exhibited in n. i — 8 follows the lines of that
which I have given for the preceding proposition.

It is at once obvious from the figure that the
square on AD includes within it twice the square
on AC together with once the square on CD.
What is left over is the sum of the rectangles AH,
KE. These, which are equivalent to BH, GK,
make up the square on CD less the square on
BD. Adding therefore the square BG to each
side, we have the required result.

Another alternative proof of the theorem which
includes both n. 9 and 10 is worth giving. The
theorem states that the sum of the squares on the

sum and difference of two given straight lines is equal to twite the sum of the
squares on the lines.

Let AD, DB be the two given straight lines (of which AD is the greater),
placed so as to be in one straight line. Make AC equal to DB and com-
pJete the figure as shown, each of the segments CG
and DH being equal to AC or DB. AC B

Now, AD, DB being the given straight lines, AB
is their sum and CD is equal to their difference.

Also AD is equal to BC.

And AE is the square on AB, GK is equal to
the square on CD, AK or Fffia the square on AD,
and BL the square on CB, while each of the small
squares AG, BH, EK, FL is equal to the square on
ACazDB.

We have to prove that twice the squares on AD,
DB are equal to the squares on AB, CD.

Now twice the square on AD is the sum of the squares on AD, CB,
which is equal to the sum of the squares BL, FH , and the figure shows
these to be equal to twice the inner square GK and once the remainder of
the large square AE excluding the two squares AG, KE, which latter squares
are equal to twice the square on AC 01 DB.

Therefore twice the squares on AD, DB are equal to twice the inner
square GK together with once the remainder of the large square AE, that is,
to the sum of the squares AE, GK, which are the squares on AB, CD.

``Side'' and ``diagonal ``numbers giving successive approxi-
mations to J2.

Zeuthen pointed out (Dii Lthre von den Kegehcknitten im Alia- turn, rS86,
pp.~37, 38) that 11. 9, 10 have great interest

in connexion with a problem of indeterminate: g g B

analysis which received much attention from

the ancient Greeks. If we take the straight line AB divided at C and D as

in 11. 9, and if we put CD = x, DB=y, the result obtained by Euclid, namely:

AEP + DB* lAC + i CD*,

or AD t ~iAC = iCD'-DB',

becomes the formula

(tx+y)* — a(x*yf = tx*-jf.

If therefore x, y be numbers which satisfy one of the two equations

2x* —y* — ± 1,

the formula gives us two higher numbers, x+y and 2x +y, which satisfy the
other of the two equations.

Euclid's propositions thus give a general proof of the very formula used
for the formation of the succession of what were called ``side ``and ``diagonal
numbers.''

As is well known, Theon of Smyrna (pp.~43, 44, ed. Hiller) describes this
system of numbers. The unit, being the beginning of all things, must be
potentially both a side and a diameter. Consequently we begin with two units,
the one being the first side and the other the first diameter, and (a) from the
sum of them, (£) from the sum of twice the first unit and once the second, we
form two new numbers

1.1 + 1 = *, 1.1 + 1 = 3.

Of these new numbers the first is a side- and the second a diagonal- umber,
or (as we may say)

flj=2, d,= 3.

In the same way as these numbers were formed from a I = 1, d, = 1, successive
pairs of numbers are formed from o,, d   , and so on, according to the formula

o«+i = o* + Jn rf»+i = **.. + <?»•
Thus a,= a + 3 = 5, d, = 8.3 + 3 = 7,

«*=5 + 7 = ``. rf,= 3. 5 + 7 = 17,
and so on.

Theon states, with reference to these numbers, the general proposition that
d n *=2a n i ± I,
and he observes (i) that the signs alternate as successive d's and n's are taken,
d* - as,' being equal to — 1, /4* - 't equal to + r, df - ia, s equal to I, and
so on, (2) that the sum of the squares of all the d's will be double of the sum
of the squares of all the a's. [If the number of successive terms in each
series is finite, it is of course necessary that the number should be even.]
The proof, no doubt omitted because it was well known, may be put
algebraically thus

rf„* - io,* = (ao,., + <C)' ``* ( a -> + rf «-0*

'= + tJ - an-A m 'ike manner,
ana so on, while d,* - 30,'  - i. Thus the theorem is established.

Euclid's propositions enable us to establish the theorem geometrically;
and this fact might well be thought to confirm the conjecture that the
investigation of the indeterminate equation 2-y'-±i in the manner
explained by Theon was no new thing but began at a period long before
Euclid's time. No one familiar with the truth of the proposition stated by
Theon could have failed to observe that, as the corresponding side- and
diagonal-numbers were successively formed, the value of rf„*/ a »'' would
approach more and more nearly to 3, and consequently that the successive
fractions dja n would give nearer and nearer approximations to the value of

/, mill II il

It is fairly clear that in the famous passage of Plato's Republk (546 c)
about the ``geometrical number'' some such system of approximations is
hinted at Plato there contrasts the ``rational diameter of five' (/Jijti) StafMrpot
tt)<; vtftwdSfK) with the ``irrational ``(diameter). This was certainly taken
from the Pythagorean theory of numbers (cf.\ the expression immediately
preceding, 546 B, C ntn xptxnjyopa no! /Sirra irpw aXkijka dviifnprav, with the
phrase wavta yv<iwra mi rordyopa dWdkois diripytifcrrat in the fragment of

Philolaus). The reference of Plato is to the fol lowin g consideration. If the
square of side 5 be taken, the diagonal is -J 2. 35 or V50. This is the
Pythagorean ``irrati onal d iameter'' of 5; and the ``rational diameter'' was
the approximation V50- 1, or 7.

But the conjecture of Zeuthen, and the attribution of the whole theory of
side- and (amtaAnumbers to the Pythagoreans, have now been fully confirmed
by the publication of Kroll's edition of Prodi Diadochi in Platonis rempuilieam
commentarii (Teubner), Vol. 11., 1901. The passages (cc. 33 and 37, pp.~34,
15 and 2 j — 39) which there saw the light for the first time describe the same
system of forming side- and diagen a /-numbers and definitely attribute it, as
well as the distinction between the ``rational ``and ``irrational diameter,'' to
the Pythagoreans. Procl us further says (p.~27, 16 — 22) that the property of the
side- and diagonal-naxabtts ``is proved graphically (ypafifitxmt) in the second
book of the Elements by 'Aim' (aV iiciirov). For, if a straight line be bisected
and a straight line be added to it, the square on the whole line including the
added straight line and the square on the latter atone are double of the square on
the half of the original straight line and of the square on the straight line made
up of the half and of the added straight line.'' And this is simply Eucl.\ 11. 10.
Proclus then goes on to show specifically how this proposition was used to
prove that, with the notation above used, the diameter corresponding to the
side a -t-dixa + d. Let AB be a side and BC equal to it, while CD is the
diameter corresponding to AB, i.e.\ a straight line such that the square on it is
double of the square on AB. (I use the figure supplied by Hultsch on p.~397
of KroU's Vol 11.)

Then, by the theorem of Eucl.\ 11. 10, the squares on AD, DC are double
of the squares on AB, BD,

But the square on DC (i.e.\ BE) is double of the square on AB; therefore,
by subtraction, the square on AD Is double of the square on BD.

And the square on DF, the diagonal corresponding to the side BD, is
double the square of BD.

Therefore the square on DP is equal to the square on AD, so that jPis
equal to AD.

That is, while the side BD is, with our notation, a + d, the corresponding
diagonal, being equal to AD, is 1a + d.

In the above reference by Proclus to 11. 10 dw' imivav ``by him'' must
apparently mean w BinXxiSov, ``by Euclid,'' although Euclid's name has not
been mentioned in the chapter; the phrase would be equivalent to saying
``in the second Book of the famous Elements.'' But, when Proclus says ``this
is firwed in the second Book of the Elements,'' he does not imply that it had
not been proved before; on the contrary, it is clear that the theorem had
been proved by the Pythagoreans, and we have therefore here a confirmation
of the inference from the part played by the gnomon and by 1. 47 in Book 11.
that the whole of the substance of that Book was Pythagorean. For further
detailed explanation of the passages of Proclus reference should be made to
Hultsch's note in KroU's Vol. 11. pp.~393 — 400, and to the separate article,
also by Hultsch, in the Bibliotheca Mathematica I,, 1900, pp.~8 — 12.

p.~Bergh has an ingenious suggestion (see \ZMP~

xxxi. Hist-titL Abt. p.~135, and Cantor, Geschithte der Mathematik, i„ p.~437)

as to the way in which the formation of the successive

side- and diagonal-numbers may have been discovered,

namely by observation from a very simple geometrical

figure. Let ABC be an isosceles triangle, right-angled at

A, with sides o.i, «„-,, d n , respectively. If now the

two sides AS, AC about the right angle be lengthened

by adding </„, to each, and the extremities D, E be

joined, it is easily seen by means of the figure (in which

BF, CG are perpendicular to DE) that the new diagonal

d m a equal to aa„- y + f/„, , while the equal sides a m are, by construction, equal

to a,-! + rf,-!.

Important deductions from II. g, 10.

I. Pappus (vii. pp.~856 — 8) uses 11. 9, 10 for proving tne well-known
theorem that

The sum of the squares on two sides of a triangle is equal to twice the square
oh half the base together with twice the square on the straight lint Joining the
Middle point of the base to the opposite vertex.

Let ABC be the given triangle and D the middle point of the base BC.
Join AD, and draw AE perpendicular to BC (produced if necessary).

C E

Now, by 11. 9, 10,
the squares on BE, EC are equal to twice the squares on BD, DE.
Add to each twice the square on AE.
Then, remembering that

the squares on BE, EA are equal to the square on BA,
the squares on AE, EC are equal to the square on A C,
and the squares on AE, ED are equal to the square on AD,
we find that

the squares on BA, AC are equal to twice the squares on AD, BD,
The proposition is generally proved by means of 11. 12, 13, but not, I
think, so conveniently as by the method of Pappus.

II. The inference was early made by Gregory of St. Vincent (1584–166;)
and Vivian i (1622–1703) that In any parallelogram the squares on the diagonals
are together equal to the squares on the sides, or to twice the squares on adjacent
sides.

III. It appears that Leonhard Euler (1 707–83) was the first to discover
the coiresponding theorem with reference to any quadrilateral, namely that
In any quadrilateral the sum of the squares on the sides is equal to the sum of the
squares on the diagonals and four limes the square on the line Joining the middle
points of the diagonals. Euler seems however to have proved the property
from the corresponding theorem for parallelograms just quoted (cf.\ Camerer's
Euclid, Vol. I. pp.~468, 469) and not from the property of the triangle, though
the latter brings out the result more easily.

\end{notes}

\end{proposition}

\begin{proposition}
\label{prop:II_11}

\begin{statement}
To cut a given straight line so that the rectangle contained
by the whole and one of the segments is equal to the square on
the remaining segment.
\end{statement}

\begin{proof}

Let AB be the given straight line;

thus it is required to cut AB so that the rectangle contained

by the whole and one of the segments is

equal to the square on the remaining ft

segment.

For let the square ABDC be described
on AB; [1. 46]

let AC be bisected at the point E, and let
BE be joined;

let CA be drawn through to F, and let EF
be made equal to BE;

let the square FH be described on AF, and
let Gbe drawn through to K.

I say that AB has been cut at H so as to make the
rectangle contained by AB, BH equal to the square on AH.

For, since the straight line AC has been bisected at E,
and FA is added to it,

the rectangle contained by CF, FA together with the
square on AE is equal to the square on EF. [n. 6]

But EF is equal to EB;

therefore the rectangle CF, FA together with the square
on AE is equal to the square on EB.

But the squares on BA, AE are equal to the square on
EB, for the angle at A is right: [1. 47]

therefore the rectangle CF, FA together with the square
on AE is equal to the squares on BA, AE.

Let the square on AE be subtracted from each;

therefore the rectangle CF, FA which remains is equal to
the square on AB,

Now the rectangle CF, FA is FK, for AF is equal to
FG
and the square on AB is AD;

therefore FK is equal to AD,

Let AKbe subtracted from each;

therefore FH which remains is equal to HD.

And HD is the rectangle AB, BH, for AB is equal to
BD;
and /! is the square on AH;

therefore the rectangle contained by AB, BH is equal
to the square on HA.

therefore the given straight line AB has been cut at H
so as to make the rectangle contained by AB, BH equal to
the square on HA.

Q.E.F.
\end{proof}

\begin{notes}

As the solution of this problem is necessary to that of inscribing a regular
pentagon in a circle (Eucl.\ iv. 10, n), we must necessarily conclude that it
was solved by the Pythagoreans, or, in other words, that they discovered the
geometrical solution of the quadratic equation

*(«-*) = *.
or * J + ax = «*.

The solution in 11. 11, too, exactly corresponds to the solution of the more
general equation

x* + ax = £*,
which, as shown above (pp.~387 — 8), Simson based upon 11. 6. Only Sijnson's
solution, if applied here, gives us the point F cm CA produced and does not
directly find the point //. It takes £ the middle point of CA, draws AB at
right angles to CA and of length equal to CA, and then describes a circle
with EB as radius cutting EA produced in F. The only difference between
the solution in this case and in the more general case is that AB is here equal
to CA instead of being equal to another given straight line b.

As in the more general case, there is, from Euclid's point of view, only one
solution.

The construction shows that CF is also divided at A in the manner
described in the enunciation, since the rectangle CF, FA is equal to the
square on CA.

The problem in 11. 11 reappears in vi. 30 in the form of cutting a given
straight tint in extreme and mean ratio.

\end{notes}

\end{proposition}

\begin{proposition}
\label{prop:II_12}

\begin{statement}
In obtuse-angled triangles the square on the side subtending
the obtuse angle is greater than the squares on the sides con-
taining ike obtuse angle by twice the rectangle contained by one
of the sides about the obtuse angle, namely that on which the
perpendicular falls, and the straight line cut off outside by the
perpendicular towards the obtuse angle.
\end{statement}

\begin{proof}

Let ABC be an obtuse-angled triangle having the angle
JiAC obtuse, and let BD be drawn from the point B per-
pendicular to CA produced;

I say that the square on BC is greater than the squares
on BA, AC by twice ihe rectangle con-
tained by CA, AD.

For, since the straight line CD has
been cut at random at the point A,
the square on DC is equal to the
squares on CA, AD and twice the rect-
angle contained by CA, AD. [«. 4]

Let the square on DB be added to
each;

therefore the squares on CD, DB are equal to the squares on
CA, AD, DB and twice the rectangle CA, AD.

But the square on CB is equal to the squares on CD, DB,
for the angle at D is right; [1. 47]

and the square on AB is equal to the squares on AD,
DB; [1.47]

therefore the square on CB is equal to the squares on CA, AB
and twice the rectangle contained by CA,-AD;

so that the square on CB is greater than the squares on
CA, AB by twice the rectangle contained by CA. AD.

Therefore etc.\
\end{proof}

\begin{notes}

Since in this proposition and the next we have to do with the squares on
the sides of triangles, the particular form of graphic representation of areas
which we have had in Book n. up to this point does not help us to visualise
the results of the propositions in the same way, and only two lines of proof
are possible, (1) by means of the results of certain earlier propositions in
Book il combined with the result of i. 47 and (2) by means of the procedure
in Euclid's proof of 1. 47 itself. The alternative proofs of ii. 12, 13 after the
manner of Euclid's proof of 1. 47 are therefore alone worth giving.

These proofs appear in certain modern text-books (e.g.\ Mehler, Henrict and
Treutlein, H. M. Taylor, Smith and Bryant). Smith and Bryant are not
correct in saying (p.~142) that they cannot be traced further back than
Lardner's Euclid (i8a8); they are to be found in Gregory of St Vincent's
work (published in 1647) Opus geemttricum quadraturtu circuK ei sectionum
eoni, Book 1. Pt 2, Props. 44, 45 (pp.~3r, 31).

To prove 11. 12, take an obtuse-angled triangle ABC in which the angle at
A is the obtuse angle

Describe squares on BC, CA, AB, as BCED, CAGF, ABKH.

Draw AL, BM, CN, perpendicular to BC, CA, AB (produced if neces-
sary), and produce them to meet the further
sides of the squares on them in P, Q, R re-
spectively.

Join AD, CK.

Then, as in t. 47, the triangles KBC, ABD
are equal in all respects;
therefore their doubles, the parallelograms in
the same parallels' respectively, are equal;

that is, the rectangle BP is equal to the
rectangle BR.

Similarly the rectangle CP is equal to the
rectangle CQ.

Also, if BG, CH be joined, we see that
the triangles BAG, HAC are equal in
all respects;
therefore their doubles, the rectangles AQ, AR, are equal.

Now the square on BCis equal to the sum of the rectangles BP, CP,

ie. to the sum of the rectangles BR, CQ,

i.e.\ to the sum of the squares BH, CG and

the rectangles AR, A Q.

But the rectangles AR, AQ are equal, and they are respectively the
rectangle contained by BA, ANacaA the rectangle contained by CA, AM.

Therefore the square on BCis equal to the squares on BA, AC together
with twice the rectangle BA, AN~ar CA, AM.

Incidentally this proof shows that the rectangle BA, AN is equal to, the
rectangle CA, AM: a result which will be seen later on to be a particular
case of the theorem in in, 35.

Heron (in an-NairlsI, ed. Curtze, p.~109) gives a ``converse'' of il u
related to it as 1. 48 is related to 1. 47.

In any triangle, if the square on one of the sides is greater than the squares
on the other two sides, the angle contained by the latter is obtuse.

Let ABC be a triangle such that the square on BC is greater than the
squares on BA, AC.

Draw AD at right angles to AC and
of length equal to AB.

Join DC.

Then, since DAC is a right angle,
the square on DC is equal to the squares
on DA, AC, [1. 47]

i.e.\ to the squares on BA, A C.

But the square on BC is greater than
the squares on BA, AC; therefore the square on BC is greater than the
square on DC.

Therefore BC is greater than DC.

Thus, in the triangles BAC, DAC, *
the two sides BA, AC are equal to the two sides DA, A C respectively,
but the base BC h greater than the base DC.

Therefore the angle BAC is greater than the angle DAC; [1. »s]

that is, the angle BA C is obtuse.

\end{notes}

\end{proposition}

\begin{proposition}
\label{prop:II_13}

\begin{statement}
In acute-angled triangles the square on the side subtending
the acute angle is less than the squares on the sides containing
the acute angle by twice the rectangle contained by one of the
sides about the acute angle, namely that on which the per-
pendicular falls, and the straight line cut off within by the
perpendicular towards the acute angle.
\end{statement}

\begin{proof}

Let ABC be an acute- angled triangle having the angle
at B acute, and let AD be drawn from the point A perpen-
dicular to BC;

I say that the square on AC is less than the squares on
CB, BA by twice the rectangle contained
by CB, BD,

For, since the straight line CB has
been cut at random at D,

the squares on CB, BD are equal to
twice the rectangle contained by CB, BD
and the square on DC. [». 7]

Let the square on DA be added to
each;

therefore the squares on CB, BD, DA are equal to twice
the rectangle contained by CB, BD and the squares on AD,
DC.

But the square on AB is equal to the squares on BD,
DA, for the angle at D is right; [i. 47]

and the square on AC is equal to the squares on AD, DC;
therefore the squares on CB, BA are equal to the square on
A C and twice the rectangle CB, BD,

so that the square on AC alone is less than the squares
on CB, BA by twice the rectangle contained by CB, BD.

Therefore etc.
\end{proof}

\begin{notes}

As the text stands, this proposition is unequivocally enunciated of aeutt-
angttd triangles; and, as if to obviate any doubt as to whether the restriction
was fully intended, the enunciation speaks of the rectangle contained by one
of the sides containing the acute angle and the straight line intercepted
within by the perpendicular towards the acute angle. On the other hand, it
is curious that it speaks of tbe square on the side subtending the acute angle;
and again the setting-out begins ``let ABC lie an acute-angled triangle having
the angle at B acute,'' though the last words have no point if all the angles of
the triangle are necessarily acute.

It was however very early noticed, not only by Isaacus Monachus,
Cam pan us, Peletarius, Clavius, Commandinus and the rest, but by the Greek
scholiast (Heiberg, VoL v, p.~253), that the relation between the sides of a
triangle established by this theorem is true of the side opposite to, and the
sides about, an acute angle respectively in any sort of triangle whether acute-
angled, right-angled or obtuse-angled. The scholiast tries to explain away the
word ``acute-angled'' in the enunciation: ``Since in the definitions he calls
acute-angled the triangle which has three acute angles, you must know that he
does not mean that here, but calls all triangles acute-angled because all have
an acute angle, one at least, if not all The enunciation therefore is: In any
triangle the square on the side subtending the acute angle is less than the
squares on the sides containing the acute angle by twice the rectangle, etc' ``

We may judge too by Heron's enunciation of his ``converse'' of the
proposition that he would have left the word ``acute-angled ``out of the
enunciation. His converse is: In any triangle in which the square on one of
the sides is less than the squares on the other two sides, the angle contained by the
latter sides is acute.

If the triangle that we take is a right-angled triangle, and the perpendicular
is drawn, not from the right angle, but from the acute angle
not referred to in the enunciation, the proposition reduces
to 1. 47, and this case need not detain us.

The other cases can be proved, like 11. is, after the
manner of 1. 47.

Let us take first the case where all the angles of the
triangle are acute.

e p

As before, if we draw ALP, BMQ, CNR perpendicular to BC, CA, AB
and meeting the further sides of the squares on BC, CA, AB in P, Q, R, and
if we join KC, AD, we have

the triangles KBC, ABD equal in all respects,
and consequently the rectangles BP, BR equal to one another.
Similarly the rectangles CP, CQ are equal to one another.

Next, by joining BG, CH, we prove in like manner that the rectangles AR,
AQaie equal.

Now the square on BC is equal to the sum of the rectangles BF, CP,

i.e.\ to the sum of the rectangles BR, CQ,

i.e.\ to the sum of the squares BH, CG diminished by the rectangles
AR, AQ.
But the rectangles AR, AQ are equal, and they are respectively the
rectangles contained by BA, AN and by CA, AM.

Therefore the square on JC is less than the squares on BA, AC by
twice the rectangle BA, AN or CA, AM.

Next suppose that we have to prove the theorem in the case where the
triangle has an obtuse angle at A.

Take B as the acute angle under considera-
tion, so that AC is the side opposite to it.

Now the square on CA is equal to the
difference of the rectangles CQ, AQ,

i.e.\ to the difference between CP and

AQ,
Le. to the difference between the square
BE and the sum of the rectangles
BP, AQ,
i.e.\ to the difference between the square
BE and the sum of the rectangles
BP, AR,

i.e.\ to the difference between the sum of
the squares BE, BH and the sum
of the rectangles BP, BR

(since AR is the difference between BR and BH).

But BP, BR are equal, and they are respectively the rectangles CB, BL
and A B, BN.

Therefore the square on CA is less than the squares on AB, BC by twice
the rectangle CB, BL or AB. BN.

Heron's proof of his converse proposition (an-NairlzI, ed, Curtze, p.~1 10),
which is also given by the Greek scholiast above quoted,
is of course simple. For let ABC be a triangle in which
the square on AC is less than the squares on AB, BC.

Draw BD at right angles to BC and of length equal
to BA.

Join DC.

Then, since the angle CBD is right,
the square on DC is equal to the squares on DB, BC,
Le. to the squares on AB, BC. [1. 47)

But the square on AC is less than the squares on
AB, BC.

Therefore the square on AC is less than the square on DC.

Therefore AC is less, than DC.

Hence in the two triangles DBC, ABC the sides about the angles DBC,

ABC are respectively equal, but the base DC is greater than the base AC.

Therefore the angle DBC (a right angle) is greater than the angle ABC
[l. 25], which latter is therefore acute.

It may be noted, lastly, that 11. 13, 13 are supplementary to 1. 47 and
complete the theory of the relations between the squares on the sides of any
triangle, whether right-angled or not.

\end{notes}

\end{proposition}

\begin{proposition}
\label{prop:II_14}

\begin{statement}
To construct a square equal to a given rectilineal figure.
\end{statement}

\begin{proof}

Let A be the given rectilineal figure;
thus it is required to construct a square equal to the rectilineal
figure A.

s For let there be constructed the rectangular parallelogram
BD equal to the rectilineal figure A. [1. 45]

Then, if BE is equal to ED, that which was enjoined
will have been done; for a square BD has been constructed
equal to the rectilineal figure A.
10 But, if not, one of the straight lines BE, ED is greater.
Let BE be greater, and let it be produced to F
let EF be made equal to ED, and let BF be bisected at G.

With centre G and distance one of the straight lines GB,
GF let the semicircle BHF be described; let DE be produced
is to H, and let GH be joined.

Then, since the straight line BF has been cut into equal
segments at G, and into unequal segments at E,

the rectangle contained by BE, EF together with the
square on EG is equal to the square on GF. [u. 5]

m But GF is equal to GH;
therefore the rectangle BE, EF together with the square on
GE is equal to the square on GH.

But the squares on HE, EG are equal to the square on
GH; [1. 47]

as therefore the rectangle BE, EF together with the square on
GE is equal to the squares on HE, EG.

Let the square on GE be subtracted from each;
therefore the rectangle contained by BE, EF which
remains is equal to the square on EH.
30 But the rectangle BE, EF is BD, for EF is equal to ED;

therefore the parallelogram BD is equal to the square on
HE.

And BD is equal to the rectilineal figure A.

Therefore the rectilineal figure A is also equal to the square
3j which can be described on EH.

Therefore a square, namely that which can be described
on EH, has been constructed equal to the given rectilineal
figure A. Q.E.F.
\end{proof}

\begin{annotations}

7. that which wan enjoined will have been done, literally ``would have been
done,'' >Fy»it t> ttj) tA iinTaxfr.

35, 36- which can be described, expressed by the future passive participle, iyvypaij.

\end{annotations}

\begin{notes}

Heiberg (Mathematisches zu AristoteUs, p.~»o) quotes as bearing on this
proposition Aristotle's remark (De ant ma 11. 2, 413 a 19: cf.\ Metaph. 996 b zi)
that ``squaring ``(Terpavuii'Krftoe) is better defined as the ``finding of the mean
(proportional) ``than as ``the making of an equilateral rectangle equal to a
riven oblong,'' because the former definition states the cause, the latter the
:ondusion only. This, Heiberg thinks, implies that in the text-books which were
in Aristotle's hands the problem of 11. 14 was solved by means of proportions.
As a matter of fact, the actual construction is the same in 11. 14 as in vi. 13;
and the change made by Euclid must have been confined to substituting in
the proof of the correctness of the construction an argument based on the
principles of Books 1. and 11. instead of Book vi.

As n. t2, 13 are supplementary to 1. 47, so 11. 14 completes the theory of
transformation of areas so far as it can be carried without the use of proportions.
As we have seen, the propositions 1. 42, 44, 45 enable us to construct a
parallelogram having a given side and angle, and equal to any given rectilineal
figure. The parallelogram can also be transformed into an equal triangle with
the same given side and angle by making the other side about the angle twice
the length. Thus we can, as a particular case, construct a rectangle on a
given base (or a right-angled triangle with one of the sides about the right
angle of given length) equal to a given square. Further, 1. 47 enables us
to make a square equal to the sum of any number of squares or to the
difference between any two squares. The problem still remaining unsolved is
to transform any rectangle (as representing an area equal to that of any
rectilineal figure) into a square of equal area. The solution of this problem,
given in 11. 14, is of course the equivalent of the extraction of the square root,
or of the solution of the pure quadratic equation

x* = ai.

Simson pointed out that, in the construction given by Euclid in this case,
it was not necessary to put in the words ``Let BE be greater,'' since the
construction is not affected by the question whether BE or ED is the greater.
This is true, but after all the words do little harm, and perhaps Euclid may
have regarded it as conducive to clearness to have the points B, G, E, F in
the same relative positions as the corresponding points A, C, D, B in the
figure of 11. 5 which he quotes in the proof.

\end{notes}

\end{proposition}

\part{Excursus I}

\chapter*{Pythagoras and the Pythagoreans}

The problem of determining how much of the Pythagorean discoveries in
mathematics can be attributed to Pythagoras himself is not only
difficult; it may be said to be insoluble.  Tradition on the subject
is very meagre and uncertain, and further doubt is thrown upon it by
the well-known tendency of the later Pythagoreans to ascribe
everything to the Master himself (\greek{αὐτὸς ἔφα}, \emph{Ipse
  dixit}).  Pythagoras himself left no written exposition of his
doctrines, nor did any of his immediate successors, not even Hippasus,
about whom the different stories ran (1)~that he was expelled from the
school because he published doctrines of Pythagoras, and (2)~that he
was drowned at sea for revealing the construction of the dodecahedron
in the sphere and claiming it as his own, or (as others have it) for
making known the discovery of the irrational or incommensurable.  Nor
is the absence of any written record of Pythagorean doctrines down to
the time of Philolaus to be put down to a pledge of secrecy binding
the school; at all events this did not apply to their mathematics or
their physics; and it may be that the supposed secrecy was invented to
account for the absence of documents. The fact seems to be that oral
communication was the tradition of the school, while their doctrines
would in the main be too abstruse to be understood by the generality
of people outside.  Even Aristotle felt the difficulty; he evidently
knew nothing for certain about any ethical or physical doctrines going
back to Pythagoras himself; when he speaks of the Pythagorean system,
he always refers it to ``the Pythagoreans,'' sometimes even to ``the
so-called Pythagoreans.''

Since my note on Eucl.\ \prop{1}{47} was originally written the part
of Pythagoras in the Pythagorean mathematical discoveries has been
further discussed and every scrap of evidence closely, and even
meticulously, examined in two long articles by Heinrich Vogt, ``Die
Geometrie des Pythagoras ``(\emph{Biblioteheca Mathematica}
\r9\tsub{3}, 1908/9, pp.~15—54) and ``Die Entstehungsgeschichte des
Irrationalen nach Plato und anderen Quellen des 4.\ Jahrhunderts''
(\emph{Bibliotheca Mathematica} \r9\tsub{3}, 1910, pp.~97—155).  These
papers would not indeed have enabled me to modify greatly what I have
written regarding the supposed discoveries of Pythagoras and the early
Pythagoreans, because I have throughout been careful to give the
traditions on the subject for what they are worth and no more, and not
to build too much upon them.  It is right however to give, in a
separate note, a few details of Vogt's arguments.

G.~Junge had, in his paper ``Wann haben die Griechen das Irrationale
entdeckt?''\ mentioned above (p.~351), tried to prove that Pythagoras
himself could not have discovered the irrational; and the object of
Vogt's papers is to go further on the same lines and to show (1)~that
it was only the later Pythagoreans who (before 410~\bc) recognised the
incommensurability of the diagonal with the side of a square, (2)~that
the theory of the irrational was first discovered by Theodorus, to
whom Plato refers (\emph{Theaetetus} 141~\textsc{d}), and (3)~that
Pythagoras himself could not have been the discoverer of any one of
the things specifically attributed to him, namely (\emph{a})~the
theorem of Eucl.\ \prop{1}{47}, (\emph{b})~the construction of the
five regular solids in the sense in which they are respectively
constructed in Eucl.~\book{13}, (\emph{c})~the application of an area
in its widest sense, equivalent to the solution of a quadratic
equation in its most general form.

Vogt's main argument as regards (\emph{a})~the theorem of \prop{1}{47}
is based on a new translation which he gives of the well-known passage
of Proclus' note on the proposition (p.~426, 6–9), \greek{Τῶν μὲν
  ἱστορεῖν τὰ ἀρχαῖα βουλομένων ἀκούοντας τὸ θεώρημα τοῦτο εἰς
  Πψθαγόραν ἀναπεμπόντων ἐστὶν εὑρεῖν καὶ βουθυτην λεγόντων αὐτὸν ἐπὶ
  τῇ εὑρέσει.} Vogt translates this as follows: ``Unter denen, welche
das Altertum erforschen wollen, kann man einige finden, welche denen
Geho|r geben, die dieses Theorem auf Pythagoras zurückführen und ihn
als Stieropferer bei dieser Gelegenheit bezeichnen,'' ``Among those
who have a taste for research into antiquity, we can find some who
give ear to those who refer this theorem to Pythagoras and describe
him as sacrificing an ox on the strength of the discovery.''
According to this version the words \greek{τῶν…βουλομένων} and the
words \greek{ἀναπεμπόντων…καὶ…λεγόντων} refer respectively to two
different sets of persons, in fact two different generations; the
latter are older authorities who are supposed to be cited by the
former; the former are a later generation, perhaps contemporaries of
Proclus, some of whom accepted the view of the older authorities while
others did not.  But this would have required the article \greek{τῶν}
before \greek{ἀναπεμπόντων}, or some such expression as \greek{ἄλλων
  τινῶν οἳ ἀναπέμπουσι} instead of \greek{ἀναπεμπόντων}.  Vogt's
interpretation is therefore quite inadmissible.  The persons denoted
by \greek{ἀναπεμπόντων} are \emph{some of} the persons denoted by
\greek{τῶν βουλομένων}; hence Tannery's translation, to which mine
(p.~350 above) is equivalent, is the only possible one, namely ``Si
l'on écoute ceux qui veulent raconter l'histoire des anciens temps, on
peut en trouver qui attribuent ce théorème Pythagore et lui font
sacrifier un bœuf après sa découverte ``(\emph{La Géométrie grecque},
p.~103). \greek{ἀκούοντας} agrees with the assumed \emph{subject} of
\greek{εὑρεῖν}; \greek{ἀναπεμπόντων} and \greek{λεγόντων} should,
strictly speaking, have been \greek{ἀναπέμποντας} and \greek{λέγοντας}
agreeing with \greek{τινὰς} (the direct \emph{object} of
\greek{εὑρεῖν}) understood, but are simply attracted into the case of
\greek{βουλομένων}; the construction is quite intelligible.  I agree
with Vogt that Eudemus' history contained nothing attributing the
theorem to Pythagoras.  The words of Proclus imply this; but I do not
think that they imply (as Vogt maintains) any pronouncement by Proclus
himself \emph{against} such attribution.  In my opinion, Proclus is
simply determined not to commit himself to any view; his way of
evading a decision is the sentence following, \greek{ἐγὼ δὲ θαυμάζω
  μὲν καὶ τοὺς πρώτους ἐπιστάντας τῇ τοῦδε τοῦ θεωρήματος ἀληθείᾳ,
  μειζόνως δὲ ἄγαμαι τὸν στοιχειωτήν…}; the plural \greek{τοὺς πρώτους
  ἐπιστάντας} is, I hold, used for the very purpose of making the
statement as vague as possible; he will not even allow it to be
inferred that he attributed the discovery to any single person.
Returning to \greek{ἡ τῶν ἀλόγων πραγματεία}. (Proclus, p.~65,~19), we
may now concede (following Diels) that we should read \greek{τῶν ἀνὰ
  λόγον} (``proportionals'') instead of \greek{τῶν ἀλόγων}
(``irrationals'') and that the author intended to attribute to
Pythagoras a theory of proportion (the arithmetical theory applicable
to commensurable magnitudes only) rather than the theory of
irrationals.  But I do not agree in Vogt's contention that the theory
of the irrational was first discovered by Theodorus.  It seems to me
that we have evidence to the contrary in the very passage of Plato
referred to.  Plato (\emph{Theaetetus} 147~\textsc{d}) mentions
$\sqrt{3}$, $\sqrt{5}$, …  up to $\sqrt{17}$ as dealt with by
Theodorus, but \emph{omits}~$\sqrt{2}$.  This fact, along with Plato's
allusions elsewhere to the irrationality of~$\sqrt{2}$, and to
approximations to it, in the expressions \greek{ἄρρητος} and
\greek{ῥητὴ διάμετρος τῆς πεμπάδος}, as if those expressions had a
well-known signification, implies that the discovery of the
irrationality of~$\sqrt{2}$ had been made before the time of
Theodorus. The words \greek{ἡ τῶν ἀλόγων πραγματεία} might well be
used even if the reference is only to~$\sqrt{2}$, because the first
step would be the most difficult, and \greek{πραγματεία} need not mean
the establishment of a complete theory of anything more than
``investigation'' of a subject.

Junge and Vogt hold that the theory of the irrational was not
discovered by the early Pythagoreans any more than Pythagoras because,
if it had been so discovered, an impossibly long period would
intervene between the investigation of the particular case
of~$\sqrt{2}$ and the extension of the theory by Theodorus to the
cases of $\sqrt{3}$, $\sqrt{5}$ etc.  But might not this well be due
to the fact that in the meantime the minds of geometers were engrossed
by other problems of importance, namely the quadrature of the circle
(Hippocrates of Chios and his quadratures of lunes), the trisection of
any angle (Hippias of Elis and his curve, afterwards known as the
\emph{quadratrix}), and the doubling of the cube (reduced by
Hippocrates to the problem of finding two mean proportionals in
continued proportion between two given straight lines), the last of
which problems, which meant finding geometrically the equivalent
of~$\sqrt{2}\? 3$, would naturally follow the investigation
of~$\sqrt{2}$?  Now Hippias was probably born about 460~\bc, while
Hippocrates seems to have been in Athens during a considerable portion
of the second half of the fifth century, perhaps from about 450 to
430~\bc.  Moreover Vogt has to get over the fact that Democritus (born
470/469~\bc) wrote a book \greek{περὶ ἀλόγων γραμμῶν καὶ ναστῶν};
\emph{On irrational lines and solids} (or \emph{atoms}).  This
difficulty he seeks to overcome by maintaining that \greek{ἀλόγων}
does not here mean ``irrational'' at all, but ``without ratio''
(``verhältnislos''), in the sense that any two straight lines are
``without ratio'' because they both contain an infinite number of the
indivisible (or atomic) lines, and therefore their ratio, being of the
form $\infty/\infty$, is indeterminate.  But, if these were so,
\emph{all} lines (including commensurable lines) would be ``without a
ratio'' to one another, whereas the title of Democritus' work clearly
implies that \greek{ἄλογοι γραμμαί} are a class or classes of lines
distinguished from other lines.  The fact is that Democritus was too
good a mathematician to have anything to do with ``indivisible
lines.''  This is confirmed by a scholium to Aristotle's \emph{De
  caelo} (p.~469 b~14, Brandis) which implicitly denies to Democritus
any theory of indivisible lines: ``of those who have maintained the
existence of indivisibles, some, as for example Leucippus and
Democritus, believe in indivisible bodies, others, like Xenocrates, in
indivisible lines.''  Moreover Simpiicius tells us that, according to
Democritus himself, even the atoms were, in a mathematical sense,
divisible further and in fact \emph{ad infinitum}.

Coming now to~(\emph{b}) the construction of the cosmic figures,
\greek{ἡ τῶν κοσμικῶν σχημάτων σ’στασις} (Proclus, p.~65, 20), I agree
with Vogt to the following extent.  It is unlikely that Pythagoras or
even the early Pythagoreans ``constructed'' the five regular solids in
the sense of a complete theoretical construction such as we find, say,
in Eucl.~\book{13}; and it is possible that Theaetetus was the first
to give these constructions, whether \greek{ἔγραψε} in Suidas' notice,
\greek{προ›τος δὲ τὰ πέντε καλούμενα στερεὰ ἔγραψε}, means
``constructed'' or ``wrote upon.''  But \greek{σύστασις} in the above
phrase of Proclus may well mean something less than the theoretical
constructions and proofs of Eucl.~\book{13}; it may mean, as Vogt
says, simply the ``putting together'' of the figures in the same way
as Plato puts them together in the \emph{Timaeus}, i.e.\ by bringing a
certain number of angles of equilateral triangles and of regular
pentagons together at one point.  There is no reason why the early
Pythagoreans should not have ``constructed'' the five regular solids
in this sense; in fact the supposition that they did so agrees well
with what we know of their having put angles of certain regular
figures together round a point (in connexion with the theorem of
Eucl.\ \prop{1}{32}) and shown that only three kinds of such angles
would fill up the space \emph{in one plane} round the point.  But I do
not agree in the apparent refusal of Vogt to credit the Pythagoreans
with the knowledge of the theoretical construction of the regular
pentagon as we find it in Eucl.\ \prop{4}{10}, \prop*{4}{11}.  I do
not know of any reason for rejecting the evidence of the
Scholia~\r4.\ Nos.~2 and 4 which say categorically that ``this Book
``(Book~\book{4}) and ``the whole of the theorems'' in it (including
therefore Props.~\prop*{4}{10}, \prop*{4}{11}) are discoveries of the
Pythagoreans.  And the division of a straight line in extreme and mean
ratio, on which the construction of the regular pentagon depends,
comes in Eucl.\ \book{2} (Prop.~\prop*{2}{11}), while we have
sufficient grounds for regarding the whole of the substance of
Book~\book{2} as Pythagorean.

I will permit myself one more criticism on Vogt's first paper. I think
he bases too much on the fact that it was left for Oenopides (in the
period from, say, 470 to 450~\bc) to discover two elementary
constructions (with ruler and compasses only), namely that of a
perpendicular to a straight line from an external point
(Eucl.\ \prop{1}{12}), and that of an angle equal to a given
rectilineal angle (Eucl.\ \prop{1}{23}). Vogt infers that geometry
must have been in a very rudimentary condition at the time.  I do not
think this follows; the explanation would seem to be rather that, the
restriction of the instruments used in constructions to the ruler and
compasses not having been definitely established before the time when
Oenopides wrote, it had not previously occurred to anyone to
substitute new constructions based on that principle for others
previously in vogue. In the case of the perpendicular, for example,
the construction would no doubt, in earlier days, have been made by
means of a set square.

\part{Excursus II}

\chapter*{Popular Names for Euclidean Propositions}

Although some of these time-honoured names are familiar to most
educated people, it seems to be impossible to trace them to their
original sources, or to say who applied them for the first time
respectively.  It may be that they were handed down by oral tradition
for long periods in each case before they found their way into written
documents.

We begin with

\section*{I.~5}

1.~This proposition is in this country universally known as the
\emph{Pons Asinorum}, ``Asses' Bridge.''  Even in this case opinion is
not unanimous as to the exact implication of the term.  Perhaps the
more general view is that taken in the \emph{Stanford Dictionary of
  Anglicised Words and Phrases} (by C.~A.~M. Fennell) where the
description is: ``Name of the fifth proposition of the first Book of
Euclid, suggested by the figure and the difficulties which poor
geometricians find in mastering it.'' This is certainly the equivalent
of what I gathered, in my early days at school, from a former Fellow
of St John's, the Reverend Anthony Bower, who was a high Wrangler in
1846 and a friend of Todhunter's.  The ``ass'' on this interpretation
is a synonym for ``fool.''  But there is another view (as I have
learnt lately) which is more complimentary to the ass.  It is that,
the figure of the proposition being like that of a trestle-bridge,
with a ramp at each end which is the more practicable the flatter the
figure is drawn, the bridge is such that, while a horse could not
surmount the ramp, an ass could; in other words, the term is meant to
refer to the surefootedness of the ass rather than to any want of
intelligence on his part. (I may perhaps mention that Sir George
Greenhill is a strong supporter of this view.)

An epigram of 1780 is the earliest reference to the term in Murray's
English Dictionary:
\begin{verse}
``If this be rightly called the bridge of asses,\\
He's not the fool that sticks but he that passes.''
\end{verse}
The writer's own view is not too clear.  He seems to imply that, while
the inventor of the name msant that only the fool finds the bridge
difficult to pass, the more proper view would be that, since the ass
can get over, and ``ass'' is synonymous with ``fool,'' therefore it
must be the fool who can get over; in other words, he seems to object
to the phrase as being a contradiction in terms.

But we have also to take account of the fact that the French apply the
term to \prop{1}{47}.  Now in Euclid's figure for \prop{1}{47} there
is no suggestion of a bridge, and the reference can only be to the
nature of the theorem, its difficulty or otherwise.  It is curious
that the French dictionaries give two different explanations of
\emph{Pont aux ânes}, Littré makes it ``ce que personne ne doit ni ne
peut ignorer; ce qui est si facile que tout le monde doit y reússir.''
Now no intelligent person could have applied the name to
Eucl.\ \prop{1}{47} for this reason, namely that it was so easy that
even a fool could not help knowing it.  Larousse is better informed;
there we find ``\emph{Pont aux ânes}, certaine difficulté, certaine
question qui n'arrête que les ignorants, et qui sert de critérium pour
juger l'intelligence de quelqu'un, et particulièrement d'un écolier.
C'est ainsi que, dans les classes de mathématiques, on ne manque
jamais dt dire que le carré de l'hypoténuse est le \emph{pont aux
  ânes} de la géométrie.  La plupart des dictionnaires entendent par
ce mot une chose si simple, si facile, que personne ne doit l'ignorer:
c'est une erreur évidente.''  Larousse is clearly right.  But it will
be observed that, so far as it goes, Larousse's interpretation rather
supports the first of the two alternative explanations of the meaning
of ``Asses' Bridge'' as applied to \prop{1}{5}, namely that it is
difficult for the fool (= ``ass'') to master.

In the \emph{Stanford Dictionary} it is added that ``in logic the term
was in the 16~\textsc{c}.\ applied to the conversion of propositions
by the aid of a difficult diagram for finding middle terms''; and if
the mathematicians borrowed the term from logic, this again would be
rather in favour of the first explanation of its use for \prop{1}{5}.

If it is permitted \emph{desipere in loco}, I would add for the
benefit of future generations (in the hope that they will still be
able to appreciate the joke or, in the alternative, will be tempted to
discuss learnedly what could possibly have been meant) a very topical
allusion in a recent \emph{Punch} (14~Oct.\ 1935): ``When they film
Euclid, as is suggested, we shall no doubt see a very thrilling rescue
over the burning Pons Asinorum.''—And yet it is safe to prophesy that
the ``Asses' Bridge'' will outlive the ``film''!

2. \emph{Elefuga}.

This name for Eucl.\ \prop{1}{5} is mentioned by Roger Bacon (about
1250), who also gives an explanation of it (\emph{Opus Tertium},
c.~vi).  He observes that in his day people in general, finding no
utility in any science such as geometry, for example, recoiled from
the idea of studying it unless they were boys forced to it by the rod,
so that they would hardly learn so much as three or four propositions.
Hence it is, he says, that the fifth proposition is called ``Elefuga,
id est, fuga miserorum; elegia enim Graece dicitur, Latine miseria; et
elegi sunt miseri.''  That is, according to Roger Bacon, Elefuga is
``flight of the miserable.''  This explanation no doubt accounts for
the verses about \emph{Dulcarnon} in Chaucer's \emph{Troilus and
  Criseyde}, \r3, 11.~933–5:
\begin{verse}
\llap{``}Dulcarnon called is `fleminge of wrecches';\\
It seemeth hard, for wrecches wol not lere\\
For verray slouthe or othere wilful tecches'';
\end{verse}
since ``fleminge of wrecches,'' ``banishment of the miserable,'' is a
translation of ``fuga miserorum.'' Only Dulcarnon is there wrongly
taken to be the same proposition as Elefuga, i.e.\ \prop{1}{5},
whereas, as we shall see, Dulcarnon was really the name for the
Pythagorean theorem \prop{1}{47}.

Etymologically, Roger Bacon's explanation leaves something to be
desired.  The word would really seem to be an attempt to compound the
two Greek words \greek{ἔλεος}, pity (or the object of pity), and
\greek{φυγή}, flight (cf.\ note \emph{ad loc.}\ in Skeat's edition of
Chaucer).  Notwithstanding the confusion of tongues, the object seems
to be a play upon the two words \emph{Elementa} and \greek{ἔλεος},
which both begin with the same three letters, and the implication is
that ``escape from the Elements'' (which normally came when
Prop.~\prop*{1}{5} was reached) was equivalent to ``escape from
misery'' or ``trouble.''  A better form for the word would perhaps be
Eleufuga; and this form actually occurs in Alanus'
\emph{Anticlaudianus}, \r3, c.~6 (cited by Du Cange,
\emph{Glossarium}, s.v.).  The word also occurs, according to Skeat's
note, in Richard of Bury's \emph{Philobiblon}, c.~xiii, where it was
somewhat oddly translated by J.~B. Inglis in 1832 ``How many scholars
has the Helleflight of Euclid repelled!''

\section*{I.~47}

The Pythagorean proposition about the square on the hypotenuse has
taken even a deeper hold of the minds of men, and has been known by a
number of names.

1.~\emph{The Theorem of the Bride} (\greek{θεώρημα τῆς νύμφης}).

This name is found in a textsc(ms.)\ of Georgius Pachymeres
(1242–1310) in the Bibliothèque Nationale at Paris; there is a note to
this effect by Tannery (\emph{La Géométrie grecque}, p.~105), but, as
he says nothing more, it is probable that the passage gives the mere
name without any explanation of it.  We have, however, much earlier
evidence of the supposed connexion of the proposition with marriage.
Plutarch (born about 46~\ad) says (\emph{De Iside et Osiride} 56,
p.~373~\textsc{f}) ``We may imagine the Egyptians (thinking of) the
most beautiful of triangles (and) likening the nature of the All to
this triangle most particularly, for it is this same triangle which
Plato is thought to have employed in the \emph{Republic}, when he put
together the Nuptial Figure (\greek{γαμήλιον
  διάγραμμα})''—\greek{διάγραμμα}, though literally meaning
``diagram'' or ``figure,'' was commonly used in the sense of
``proposition''—``and in that triangle the perpendicular side is~3,
the base~4, and the hypotenuse, the square on which is equal to the
sum of the squares on the sides containing (the right angle),~5.  We
must, then, liken the perpendicular to the male, the base to the
female and the hypotenuse to the offspring of both…. For 3 is the
first odd number and is perfect, 4 is the square on an even side,~2,
while the 5 partly resembles the father and partly the mother, being
the sum of 3 and~2.''

Plato used the three numbers 3, 4, 5 of the Pythagorean triangle in
the formation of his famous Geometrical Number; but Plato himself does
not call the triangle the Nuptial Triangle nor the number the Nuptial
Number.  It is later writers, Plutarch, Nicomachus and Iamblichus, who
connect the passage about the Geometrical Number with marriage;
Nicomachus (\emph{Introd.\ Ar.}, \r2, 24,~11) merely alludes to ``the
passage in the Republic connected with the so-called Marriage,'' while
Iamblichus (\emph{In Nicom.}, p.~82, 20 Pistelli) only speaks of ``the
Nuptial Number in the Republic''

It would appear, then, that the name ``Nuptial Figure'' or ``Theorem
of the Bride'' was originally used of one particular right-angled
triangle, namely (3, 4, 5).  A late Arabian writer Behā-ad-dīn
(1547–1622) seems to have applied the term ``Figure of the Bride'' to
the same triangle; the Arabs therefore seemingly followed the Greeks.
The idea underlying the use of the term, first for the triangle (3, 4,
5), and then for the general theorem of \prop{1}{47}, seems to be
roughly that of the two parties to a marriage becoming one, just as
the two squares on the sides containing the right angle become the one
square on the hypotenuse in the said theorem.

2.\emph{The ``Bride's Chair.''}

The origin of this name is more obscure.  It must presumably have been
suggested by a supposed resemblance between the figure of the
proposition and such a chair.  D.~E. Smith (\emph{History of
  Mathematics}, \r2, pp.~289–90) remarks that the ``Bride's Chair''
may be so-called ``because the Euclid figure is not unlike the chair
which a slave carries on his back and in which the Eastern bride is
sometimes transported to the ceremony,'' and he cites a note from
Edouard Lucas' \emph{Récréations Mathématiques}, \r2, p.~130: ``La
démonstration que nous venons de donner du théorème de Pythagore sur
le carré de l'hypoténuse ne diffèr pas essentialement de la
démonstration hindoue, connue sous le nom de la \emph{Chaise de la
  petite mariée}, que l'on rencontre dans l'ouvrage de Bhascara
(Bija-Ganita, §146).''  The figure of Bhāskara is not that of Euclid
but that shown at the top of p.~\pageref{355} above; I have however
not been able to find the name ``Bride's Chair'' in Colebrooke's
translation of the work of Bhāskara.

Notwithstanding the apparent frivolity of the setting, I venture to
suggest that light may be thrown on the question by a very modern
version of the ``Bride's Chair'' which appeared during or since the
War in \emph{La Vie Parisienne}.  The illustration represents Euclid's
figure for \prop{1}{47} and, drawn over it, as on a frame, a
\emph{poilu} in full fighting kit carrying on his back his bride and
his household belongings.  Roughly speaking, the soldier is standing
(or rather walking) in the middle of the large square, his head and
shoulders are bending to the right within the contour of one of the
small squares, while the lady, with mirror and powder-puff in action,
is sitting with her back to him in the right angle between the two
smaller squares ($HAG$ in the figure on p.~\pageref{349}
above)\footnote{Old Cambridge men will recall a picture in some
  respects not unlike, though less artistic than, the cartoon in
  \emph{La Vie Parisienne}, I mean the painting of ``The Man Loaded
  with Mischief'' which used to be over the door of the former inn of
  that name on the St Neots Road, a short distance from Cambridge.}.
I am informed by Sir George Greenhill that there was also an earlier
version ``showing the chair as it is in use to-day in Cairo and Egypt,
the earliest version of a taxi-chair, a pattern as early as Euclid and
suggesting the nickname of the proposition.''  This recalls to my mind
the remark of a friend to whom I mentioned the subject and showed the
figure of the proposition; he observed at once on seeing it ``But I
should have said it was more like a sedan chair,'' the large square
suggesting to him the actual chair and the two smaller squares the two
bearers,

3.~\emph{Dulcarnon}.

This name for \prop{1}{47} appears, as above mentioned, in Chaucer's
\emph{Troilus and Criseyde}, \r3, ll.~930–3, where Criseyde says:
\begin{verse}
'I am, til God me bettre minde sende,\\
At dulcarnon, right at my wittes ende.'\\
Quod Pandarus, 'ye, nece, wol ye here?\\
Dulcarnon called is ``fleminge of wrecches.''\,'
\end{verse}
Billingsley, too, in his edition of Euclid (1570) observes of
\prop{1}{47} that ``it hath bene commonly called of barbarous writers
of the latter time Dulcarnon.''

\emph{Dulcarnon} (see Skeat's note \emph{ad loc.})\ seems to represent
the Persian and Arabic \emph{du 'lkarnayn}, lit.\ \emph{two-horned},
from Pers.\ \emph{du}, two, and \emph{karn}, horn.  The name was
applied to \prop{1}{47} because the two smaller squares stick up like
two horns and, as the proposition is difficult, the word here takes
the sense of ``puzzle''; hence Criseyde was ``at dulcarnon'' because
she was perplexed and at her wit's end.

4.~\emph{Francisci tunica} = ``Franciskaner Kutte,'' ``Franciscan's
cowl.''

This name is quoted by Weissenborn (\emph{Die Uebersetzungen des
  Euklid durch Campano und Zamberti}, p.~42) as given in a
\emph{Geometrie} by one Kunze.  The name is quite appropriate, one of
the squares representing the hood thrown back.

\section*{III.~7, 8}

I have already mentioned the names ``Goose's Foot'' (\emph{Pes
  anseris}) and ``Peacock's Tail'' (\emph{Cauda pavonis}) applied,
suitably enough, to these propositions respectively.  They come from
Luca Paciuolo's edition of Euclid published in 1509 (\emph{vide}
Weissenborn, \ibid).

\end{document}

\part{Book III}

\chapter*{Definitions}

\begin{enumerate}

\item\label{def:III_1} Equal circles are those the diameters of which are
equal, or the radii of which are equal.

\item\label{def:III_2} A straight iine ts said to touch a circle
  which, meeting the circle and being produced, does not cut the
  circle,

\item\label{def:III_3} Circles are said to touch one another which,
  meeting one another, do not cut one another.

\item\label{def:III_4} In a circle straight lines are said to be
  equally distant from the centre when the perpendiculars drawn to
  them from the centre are equal.

\item\label{def:III_5} And that straight line is said to be at a
  greater distance on which the greater perpendicular falls.

\item\label{def:III_6} A segment of a circle is the figure contained
  by a straight line and a circumference of a circle.

\item\label{def:III_7} An angle of a segment is that contained by a
  straight line and a circumference of a circle.

\item\label{def:III_8} An angle in a segment is the angle which, when
a point is taken on the circumference of the segment and
straight lines are joined from it to the extremities of the
straight line which is the base of the segment, is contained
by the straight lines so joined.

\item\label{def:III_9} And, when the straight lines containing the angle cut
off a circumference, the angle is said to stand upon that
circumference,

\item\label{def:III_10} A sector of a circle is the figure which, when
  an angle is constructed at the centre of the circle, is contained by
  the straight lines containing the angle and the circumference cut
  off by them.

\item\label{def:III_11} Similar segments of circles are those which
  admit equal angles, or in which the angles are equal to one another.

\end{enumerate}

\section*{Definition 1}

Iiroi kvkXkh turivi wy at Stdfitrpoi urai tltrtVj  mv al fK rur HtrTptnv ivai tUrir,

Many editors have held that this should not have been included among
deAnitions. Some, e.g. Tartaglia, would call it apos/u/aU; others, e.g. Borelli
and Playfair, would cl it an axiom ; others again, as Billingsley and Clavius,
while admitting it as a definitien, add explanations based on the mode of
constructing a circle ; Simson and Pfleiderer hold that it is a tfuoretn, I
think however that Euclid would have maintained that it is a definition in
the proper sense of the term ; and certainly it satisfies Aristotle's requirement
that a ``definitional statement'' (opurTdtM AoyM) should not only state the
fait (to iri) but should indicate the cause as well (De aitima ii. i, 413 a
13). The equality ot circles with equal radii can of course be proved by
superposition, but, as we have seen, Euclid avoided this method wherever he
could, and there is nothing technically wrong in saying `` By equal circles 1
mean circles with equal radii.'' No flaw is thereby introduced into the system
of the Elements ; for the definition could only be objected to if it could be
proved that the equality predicated of the two circles in the definition was
not the same thing as the equality predicated of other equal figures in the
Elements on the lis of the Congruence- Axiom, and, ntless to say, this
cannot be proved because it is not true. The existence of equal circles (in
the sense of the definition) follows from the existence of equal straight lines
and I. Post. 3,

The Greeks had no distinct word for radius, which is with them, as here,
the (straight line drawn) from the centre 7 Jk r™ mVrpou ((Wiln) ; and so
definitely was the expression appropriated to the radius that in tov Kitpou
was used without the article as a predicate, just as if it were one word. Thus,
eg., in III. I JK KtvTpov yap means `` for they are radii `` : cf, Archimedes, On
the Sphere and Cylinder i. z, ij BE Ik rm xiirpoa iarl Tm,,,KiKjm, BM is it
radius of the circle.

\section*{Definition 2}

Euclid's phraseology here shows the regular distinction between ainvrvii
and its compound limaOai, the former meaning ``to tnett'' and the latter
``to touch.'' The distinction was generally observed, by Greek geometers
from Euclid onwards. There are however exceptions so far as hrrfaBm is
concerned; thus it means ``to touch'' in Eucl.\ iv. Def 5 and sometimes in
Archimedes On the other band, c/iairTccrdat is used by Aristotle in certain

cases where the orthodox geometrical term would be airrifrtfot. Thus in
Meleerohgica m. 5 (376 b 9) he says a certain circle will pass through all the
angles (ajTotrulK inutTM tv yioi'tiui'), atid (376 a 6) M will lie on a given
(circular) circumference (iihofian) trtptitptiiK c'icrat to M). We shall find
awrviBai used in these senses in Book iv. Deff. 2, 6 and Deff. 1, 3 respectively.
The latter of the two expressions (quoted from Aristotle means that thi locus
of M is a given drde, just as in Pappus otrai to armiiov Oia-n StSo/tiyrft
tinat means that th locus the point is a straight line given in position.

\section*{Definition 3}

Todhunter remarks that different opinions have been held as to what is,
or should be, included in this definition, one opinion being that it only means
that the circles do not cut in the neighbourhood of the point of contact,
and that it must be shown that they do not cut elsewhere, while another
opinion is that the definition means that the circles do not cut at all
Todhunter thinks the latter opinion correct. I do not think this is proved ;
and I prefer to read the definition as meaning simply that the circles meet
at a point but do not cut at that point. I think this interpretation
preferable for the reason that, although Euclid does practically assume in
III. ti — :3, without stating, the theorem that circles touching at one point
do not intersect anywhere else, he has given us, before reaching that
point in the Book, means for proving for ourselves the truth of that
statement. In particular, he has given us the propositions in. 7, 8 which,
taken as a whole, give us more information as to the general nature of a
circle than any other propositions that have preceded, and which can be used,
as will be seen in the sequel, to solve any doubts arising out of Euclid's
unproved assumptions. Now, as a matter of fact, the propositions are not used
in any of the genuine proofs of the theorems in Book in. ; in. 8 is required
for the second proof of ni. 9 which Simson selected in preference to the first
proof, but the first proof only is regarded by Heibecg as genuine. Hence it
would not be easy to account for the appearance of in. 7, 8 at all unless as
affording means of answering possible objections (cf. Proclus' explanation of
Euclid's reason for inserting the second part of i. 5).

External and internal contact are not distinguished in Euclid until 111.
II, 12, though thew of in. 6 (not the enunciation in the original text)
represents the case of internal contact only. But the definition of touching
circles here given must be taken to imply so much about internal and external
contact respectively as that (a) a circle touching another internally must,
immediately before `` meeting `` it, have passed through points within the
circle that it touches, and (b) a circle touching another externally must,
immediately before meeting it, have passed through points outside the circle
which it touches. These facts must indeed be admitted if internal and
external are to have any meaning at all in this connexion, and they constitute
a minimum admission necessary to the proof of in. 6.

\section*{Definition 4}

'Ev kukA urof Ltck-w airo roC Kci'Tpou cftuii AryotTui, orat' at a.vh TOv
Ktyrpov iv aitras Kti0tTQi oiyOfmt i(7u4 ciKTti.

\section*{Definition 5}

\section*{Definition 6}

T/t/ia KuicXov hm to wtpitofifyoy (rj/ia iro n (vttat Hat kvkXov

\section*{Definition 7}

TfHj/iOTOS 8c yMwla tirrlv jj ircpit)(onfirr) v'lni t* iJS«w »«ii KUtkov wtfyiptlat,
ThU definition is only interesting historically. The an of a segment,
being the `` angle `` formed by a straight line and a `` circumference,'' is of the
kind described by Proclus as `` mixed.'' A particular `` angle `` of this sort is
the ``angle of a semicircle,'' which we meet with again in ui. i6, along with
the so-called ``horn -like angle'' (jMparmtSijs), the supposed ``angle'' between
a tangent to a circle and the circle itself. The `` angle of a semicircle `` occurs
once in Pappus (vii. p. 670, 19), but tt there means scarcely more than the
corner of a semicircle regarded as a point to which a straight line is directed.
Heron does not give the definition of the att£k of a segment, and we may
conclude that the mention of it and of the angle of a umieircle in Euclid is a
survival from earlier text-books rather than an indication that Euclid considered
either to be of importance in elementary geometry (cf. the note on iii- i6
below).

We have however, in the note on i. s above (Vol. 1, ppi 252—3), seen evi-
detice that the a» aj<Mf had played some part in geometrical proofs up
to Euclid's time. It would appear from the passage of Aristotle there quoted
(Anal, prior, i, 24, 41 b 13 sqq.) that the theorem of 1. 5 was, in the text-books
immediately preceding Euclid, proved by means of the equality of the two
`` angles of'' any one segment. This latter property must therefore have been
regarded as more elementary (for whatever reason) than the theorem of i. 5 ;
indeed the definition as given by Euclid practically implies the same thing,
since it speaks of only one `` angle of a segment,'' namely ``/At angle contained
by a straight line and a circumference of a circle,'' Euclid abandoned the
actual use of the ``angle'' in question, but no douht thought it unnecessary
to bieak with tradition so far as to strike the definition out also.

\section*{Definition 8}

tnffutoy nal air aoi> ittI Tft irara tt tvOtiti, 7f itrrt ovif TOtJ TfiT/jfjarof

\section*{Definition 9}

'Orav Bt at vepiij(owTat rv yuiviav eotiai dirakafitii'tiMri xim irtpntpiiav,

\section*{Definition 10}

To/uvt Si icvKm/ itrriv, Smv irpof ry ithirp rov NtLncXou murra yaria,
TO n*pttxfirvotf <r)mfia into re Tmv TTfr ytaviav vtpttovrv (uttwv tcai n
djroXa/AtLvoftivjjt vtt* airrSv vtpu*p<itK.

A scholiast says that it was the shoemaket's knife, trtarrvrtiuA roiitit,
which suested the name rat for a. sector of a circle. The derivation of
the name from a resemblance of shape is parallel to the u of apirXos (also
a sMetmakfr't knife) to denote the well known figure of the Book of Lemmas
partly attributed to Archimedes.

A wider definition of a sector than that given by Euclid b found in a
Greek scholiast (Heiberg's Euclid, Vol. v. p. 260) and in an-Nairizi (ed. Curtze,
p. hi). ``There are two varieties of sectors ; the one kind have the angular
vertices at the centres, the other at the circumferences. Those others which
have their vertices neither at the circumferences nor at the centres, but at
some other points, are for that reason not called sectors but sector-like
figures (td/m«(8ij v-xpartk),'' The exact agreement between the scholiast and
an-Nairīzī suggests that Heron was the authority for this explanation.

The Mctor-Hkt figure bounded by an arc of a circle and two lines drawn
from its extremities to meet at any point actually appears in Euclid's book On
divisions (trtpl Sinifiwtuiv) discovered in an Arabic MS. and edited by
Woepcke (cf. Vol. 1. pp. 8—10 above). This treatise, alluded to by Proclus,
had for its object the division of figures such as triangles, trapezia,
([uadrilaterals and circles, by means of straight lines, into parts equal or
in given ratios. One proposition e.g. is, Ta divide a triangle into two equal
parts by a straight lint passing through a given point on one side. The
proposition (28) in which the quasi-udor occurs is, To divide suth a figure by a
straight line into two equal parts. The solution in this case is given by Cantor
(Gesck d. Math, u, pp. aS;— 8).

If ABCD be the given figure, E the middle point
of BD and EC at right angles to BD,
the broken line AEC clearly divides the figure into
two equal parts.

Join AC, and draw EF parallel to it meeting
ABn F.

Join CF, when it is seen that CF divides the
figure into two equal parts.

\section*{Definition 11}

De Morgan remarks that the use of the word similar in ``similar
segments `` is an anticipation, and that similarity of form is meant. He adds
that the definition is a theorem, or would be if `` similar `` had taken its final
meaning.

\part{Book III. Propositions}

\begin{proposition}\label{prop:III_1}

\begin{statement}
To find the centre of a given circle.
\end{statement}

\begin{proof}
Let ABC be the given circle ;
thus it is required to find the centre of the circle ABC.

Let a straight line AB be drawn
s through it at random, and let it be bisected
at the point D ;

from D let DC be drawn at right angles
to AB and let it be drawn through to E ;
let CE be bisected at F
``o I say that F is the centre of the circle
ABC.

For suppose it is not, but, if possible,
let G be the centre,

and let GA, GD, GB be joined. ,., , .. ,

IS Then, since AD is equal to DB,
and DG is common,

the two sides AD, DG are equal to the two sides
BD, DG respectively ;

and the base GA is equal to the base GB, for they are
20 radii ;

therefore the angle ADG is equal to the angle GDB. [i. 8]

But, when a straight line set up on a straight line makes

the adjacent angles equal to one another, each of the equal

angles is right ; [i. Def, 10]

35 therefore the angle GDB is right.

But the angle FDB is also right ;
therefore the angle FDB is equal to the angle GDB, the
greater to the less : which is impossible.

Therefore G is not the centre of the circle ABC.
30 Similarly we can prove that neither is any other point
except F.

Therefore the point F is the centre of the circle ABC.

, PoRiSM. From this it is manifest that, if in a circle a

straight line cut a straight line into two equal parts and at

35 right angles, the centre of the circle is on the cutting straight
line.

Q.E.F.
\end{proof}

\begin{annotations}

1. For auppose It is not. Tliis is expressetl in the Greek by the two wocds Hi yif,
but Biicb an EllLpticaJ phrase is impossible in English.

17. the two sides AD, DG are etjual to the two aides BD, DO respectively.
As before observed, Euclid 13 not always oarerul to put the equals in correspond Itig order.
The (est here has `` CZ>, D3.''

\end{annotations}

\begin{notes}

'Fodhunter observes that, when, in the construction, DC is said to be
productd to J?, it is assumed that D is within the circle, a fact which Euclid
first demonstrates in in. 2. This is no doubt true, although the word iaiw,
`` let it be drawn through' is used instead of iKijiKitrSm, `` let it bercJuad.''
And, although it is not necessary to assume that I> is within the circle, it is
necessary for the success of the construction that the straiglit line drawn
through jD at right angles to AB shall meet the circle ir two points (and no
more): an assumption which we are not entitled to make on the basis of what
has gone before only.

Hence there is much to be said for the alternative procedure recommended
by De Morgan as preferable to that of Euclid. De Morgan would first prove
the fundamental theorem that ``the line which bisects a chord perpendicularly
must contain the centre,'' and then make ni. i, iii. 25 and iv. 5 immediate
corollaries of it. The fundamental theorem is a direct consequence of the
theorem that, if P is any point equidistant from A
and .5, then P lies on the straight line bisecting AJ3
perpendicularly. We then take any two chords AB,
j4Cof the given circle and draw £>0, EO bisecting
them perpendicularly. Unless BA AC are in one
straight line, the straight lines DO, EO must meet
in some f)oint O (see note on iv. 5 for possible
methods of proving this). And, since both DO,
EO must contain the centre, must be the centre.

This method, which seems now to be generally
preferred to Euclid's, has the advantage of showing

that, in order to find the centre of a circle, it is sufficient to know three points
on the circumference. If therefore two circles have three points in common,
they must have the same centre and radius, so that two circles cannot have
three points in common without coinciding entirely. Also, as indicated by
De Morgan, the same construction enables us (i) to draw the complete circle
of which a segment or arc only is given (ill. 25), and (2) to circumscribe a
circle to any triangle (iv. 5).

But, if the Greeks had used this construction for finding the centre of a
circle, they would have considered it necessary to add a proof that no other
point than that obtained by the construction can be the centre, as is clear
both from the similar rtduetio ad abturdum in iii i and also from the fact
that Euclid thinks it necessary to prove as a separate theorem (ui. 9) that, if
a point within a circle be such that three straight lines (at least) drawn from it
to the circumference are equal, that point must be the centre. In fact,
honrever, the proof amounts to no more than the remark that the two
perpendicular bisectors can have no more than one point common.

And even in De Morgan's method there is a yet unproved assumption.
In order that DO, EO may meet, it is necessary that AB, AC should not be
in one straight line or, in other words, that BC should not pass through A.
This results from iii. 2, which therefore, stKctly speaking, should precede.

To return to Euclid's own proposition HI. i, it will be observed that the
demonstration only shows that the centre of the circle cannot lie on either
side of CD, so that it must lie on CD or CD produced. It is however taken
for granted rather than pioved that the centre must be the middle point of
CE. The proof of this by rtduetio ad absurdum is however so obvious as to
be scarcely worth giving. The same consideration which would prove it may
be used to show that a circle cannot have more than one ctntre, a proposition
which, if thought necessary, may be added to iii. i as a corollary.

Simson orved that the proof of [ii. i could not but be by reductio ad
aksurdum. At the beginning of Book in. we have nothing more to base the
proof upon than the dejinitton of a circle, and this cannot be made use of
unless we assume some point to be the centre. We cannot however assume
that the point found by the construction is the centre, because that is the
thing to be proved. Nothing is therefore left to us but to assume that some
other point is the centre and then to prove that, whatever other point is
taken, an absurdity results; whence we can infer that the point found is
the centre.

The Porism to in. i is inserted, as usual, parenthetically before the words
Svtp I5» T«7<7at, which of course refer to the problem itself.

\end{notes}

\end{proposition}

\begin{proposition}\label{prop:III_2}

\begin{statement}
If on the circumference of a circle two points be taken at
random, the straight line joining the points will fall within
the circle.
\end{statement}

\begin{proof}

Let ABC be a circle, and let two points A, B x. taken
at random on its circumference ; .

I say that the straight line joined from
. to Z? will fall within the circle.

For stippose it does not, but, if
possible, let it fall outside, as AEB ;
let the centre of the circle ABC be
taken [in. 4 and let it be Z? ; let DA,
DB be joined, and let DFE be drawn
through.

• Then, since DA is equal to DB,

the angle DAE is also equal to the angle DBE. [i. 5]
And, since one side AEB of the triangle DAE is produced,
the angle DEB is greater than the angle DAE. [i. 16]
But the angle DAE is equal to the angle DBE ;
therefore the angle DEB is greater than the angle DBE.
And the greater angle is subtended by the greater side ; [i. 19]
therefore DB is greater than DE.
But DB is equal to DF\ ~ -

therefore DF is greater than DE,

the less than the greater : which is impossible.
Therefore the straight Une joined from A to  will not
fall outside the circle.

Similarly we can prove that neither will it fall on the
circumference itself;

therefore it will fall within. ,< ,
Therefore etc.

• •
\end{proof}

\begin{notes}

The nduitio ad absurdum form of proof is not really necessary in this case,
and it has the additional disadvantage that it requires the destruction of two
hypotheses, namely (hat the chord is (i) outside, (i) on
ihe circle. To prove the proposition directly, we have
only to show that, if  be any point on the straight line
AB between A and B, DE is less than the radius of the
circle. This may be done by the method shown above,
under i. 24, for proving what is assumed in that
proposition, namely that, in the hgurc of the proposition,
/''falJs beiow EG if DE is not greater than DF. The
assumption amounts to the following proposition, which
De Morgan would make to precede 1. 4 ; `` Every
straight line drawn from the vertex of a triangle to the base is less than
the greater of the two sides, or than either if they be equal.'' The case
here Is that in which the two sides are equal ; and, since the angle DAB is
equal to the angle DBA, while the exterior angle DEA is greater than the
interior and opposite angle DBA, it follows that the angle DEA is greater
than the angle DAE, whence DE must be less than DA or DB.

Camerer points out that we may add to this proposition the further
statement that all points on AB produced in either direction are outside the
circle. This follows from the proposition (also proved by means of the
theorems that the exterior angle of a triangle ts greater than either of the
interior and opposite angles and that the greater angle is subtended by
the greater side) which De Morgan proposes to introduce after i. 3 1, namely,

`` The perpendicular is the shortest straight line that can be drawn from a
given point to a given straight line, and of others that which is nearer to the
perpendicular is less than the more lemote, and the converse ; also not more
than two equaJ straight lines can be drawn from the point to the line, one on
each side of the perpendicular.''

The fact that not more than two equal straight lines can be drawn from a
given point to a given straight line not passing through it is proved by Proclus
on ], 1 6 (see the note to that proposition) and can alternatively be proved by
means of i. 7, as shown above in the note on I. 1 2. It follows that

A straight line cannot cut a circle in Men than two points
a proposition which De Morgan would introduce here after in. a. The proof
given does not apply to a straight line passing through the centre j but that
!iiich a line only cuts the circle in two points is self evident

\end{notes}

\end{proposition}

\begin{proposition}\label{prop:III_3}

\begin{statement}
If in a circle a straight line through the centre bisect a
straight line not through the centre, it also cuts it at right
angles : and if it cut it at right angles, it also bisects it.
\end{statement}

\begin{proof}

Let ABC be a circle, and in it let a straight line CD
J throi]gh the centre bisect a straight line
AB not through the centre at the point
F;

I say that it also cuts it at right angles.
For let the centre of the circle ABC
10 be taken, and let it be .£''; let EA, EB
be joined.

Then, since AF is equal to FB,
and FE is common,

two sides are equal to two sides ;
IS and the base EA is equal to the base EB ;

therefore the angle AFE is equal to the angle BFE. [1.8]

Biit, when a straight line set up on a straight line makes

the adjacent angles equal to one another, each of the equal

angles is right ; [i- Def. 10]

20 therefore each of the angles AFE, BFE is right.

Therefore CD, which is through the centre, and bisects
AB which is not through the centre, also cuts it at right
angles.

Again, let CD cut AB at right angles ;
«5 I say that it also bisects it. that is, that AFis equal to FB.

in. 3. 4] PROPOSITIONS 2—4 11

For, with the same construction,

since £A is equal to EB,
the angle EAF is also equal to the angle EBF. [1. 5]

But the right angle AFE is equal to the right angle BEE,
JO therefore EAF, EBF are two triangles having two angles
equal to two angles and one side equal to one side, namely
EF, which is common to them, and subtends one of the equal
angles ;

therefore they will also have the remaining sides equal to
35 the remaining sides ; [i- 26]

therefore AF is equal to FB.
Therefore etc.
\end{proof}

\begin{annotations}

26. \emph{with the eame construction}, Ttav ntW-up jrnracrjrFLacrrrup .

\end{annotations}

\begin{notes}

This proposition asserts the two \emph{partial} converses (cf.\ note
on \prop{1}{6} of the Porism to \prop{3}{1}.  De Morgan would place it
next to \prop{3}{1}.

\end{notes}

\end{proposition}

\begin{proposition}\label{prop:III_4}

\begin{statement}
If in a circle two straight lines cut one another which are
not through the centre, they do not bisect one another.
\end{statement}

\begin{proof}

Let A BCD be a circle, and In it let the two straight lines
AC, BD, which are not through the
centre, cut one another at E
I say that they do not bisect one
another.

For, if possible, let them bisect one
another, so that AE is equal to EC,
and BE to ED ;

let the centre of the circle ABCD be
taken [in, 1], and let it h F\ let FE be
joined.

Then, since a straight line FE through the centre bisects
a straight line AC not through the centre,

it also cuts it at right angles ; [ni, 3]

therefore the angle FEA is right.
Again, since a straight line FE bisects a straight line BD,
it also cuts it at right angles ; fm. 3]

therefore the angle FEB is right.

But the angle FEA was also proved right ;
therefore the angle FEA is equal to the angle FEB,
the less to the greater : which is impossible.

Therefore AC, BD do not bisect one another.
Therefore etc.

y. K. D.
\end{proof}

\end{proposition}

\begin{proposition}\label{prop:III_5}

\begin{statement}
If two circles cut one another, they will not have ike same
centre.
\end{statement}

\begin{proof}

For let the circles ABC, CDG cut one another at the
points B, C

I say that they will not have the same
centre.

For, if possible, let it be E \ let EC
be joined, and let EFG be drawn
through at random.

Then, since the point E is the
centre of the circle ABC,

EC is equal to EF. [i. Def. 15]

Again, since the point E is the centre of the circle CDG,
EC is equal to EG.

But EC was proved equal to iS'/'also ;

therefore EF is also equal to EG, the less to the
greater : which is impossible.

Therefore the point E is not the centre of the circles
ABC, CDG.

Therefore etc. . .
\end{proof}

\begin{notes}
The propositions nt. 5, 6 could be combined in one. It makes no
difference whether the circles cut, or meet without cutting, so long as they do
not coincide altogether; in either case they cannot have the same centre.
The two cases are covered by the enunciatiorv ; If the circumferexces of two
ctTctti meet at a point th (annot have the same centre. On the other hand, If
two circles have the same autre and one point in their circumferences common,
they mitst coincide altogether.
\end{notes}

\end{proposition}

\begin{proposition}\label{prop:III_6}

\begin{statement}
If two circles touch one another, they will not have the
same centre.
\end{statement}

\begin{proof}

For let the two circles ABC, CDE touch one another
at the point C

I say that they will not have the
same centre.

For, if possible, let it be F; let
FC be joined, and let FEB be drawn
through at random.

Then, since the point F is the
centre of the circle ABC,

FC is equal to FB.

Again, since the point F is the
centre of the circle CDE,

FC is equal to FE. „

But FC was proved equal to FB ;

therefore FE is also equal to FB, the less to the greater:
which is impossible.

Therefore F is not the centre of the circles ABC, CDE.
Therefore etc,
\end{proof}

\begin{notes}

The English editions enunciate this propusltion of circles touching
inttmaliy, but the word (<vt<k) is a mere interpolation, which was no doutrt
made because Euclid's figure showed only the case of internal contact. The
fact is that, in his usual manner, he chose for demonstration the more difficult
case, and left the other case (that of external contact) to the intelligence of
the reader. It is indeed sufficiently self-evident that circles touching extemally
cannot have the same centre ; but Euclid's proof can really be used for thia
case too.

Camerer remarks that the proof of iti. 6 seems to assume tacitly that the
points E and B cannot coincide, or that circles which touch internally at C
cannot meet in any other point, whereas this fact is not proved by Euclid till
jii. 13. But no such general assumption is necessary here; it is only
necessary that one Une drawn from the assumed common centre should meet
the circles in different points; and the very notion of internal contact requires
that, before one circle metis the other on its inner side, it must have passed
through points within the latter circle.

\end{notes}

\end{proposition}

\begin{proposition}\label{prop:III_7}

\begin{statement}
If on the diameter of a circle a point be taken which is not
the centre of the circle, and front the point straight lines fall
upon the circle, (hat will be greatest on which the centre is, tlie
remainder of the same diameter will be least, and of ike rest
5 the nearer to the straight line through the centre is always
greater than the more remote, and only two equal straight
lines will fall from the point on the circle, one on each side
of ike least straight line.
\end{statement}

\begin{proof}

Let ABCD be a circle, and let AD be a diameter of it ;
10 on v4Z) let a point F be taken which is not the centre of the
circle, let E be the centre of the circle,

and from F let straight lines FB, FC, FG fall upon the circle
ABCD;

I say that FA is greatest, FD is least, and of the rest FB is
IS greater than FC, and FC than FG.
For let BE, CE, GE be joined.
Then, since in any triangle two
sides are greater than the remaining
one, [i- 20]

ao EB, EF are greater than BF.

But AE is equal to BE ;
therefore AF  greater than BF.
Again, since BE is equal to CE
and FE is common, `` '

25 the two sides BE, EF are equal to the two sides CE, EF.
But the angle BEFh also greater than the angle CEF;
therefore the base BF is greater than the base CF. [i. 24]

For the same reason

CF is also greater than FG, , , ,.

30 Again, since GF, FE are greater than EG,
and EG is equal to ED, w- - • .

GF, FE are greater than ED.
Let EE be subtracted from each ;

therefore the remainder GF is greater than the remainder
i FD.

Therefore FA is greatest, FD is least, and FB is greater
than FC, and FC than FG.

1 say also that from the point F only two equal straight
lines will fall on the circle A BCD, one on each side of the
40 least FD,

For on the straight line EF, and at the point E on it, let
the angle // be constructed equal to the angle GEF- aal.
and let FH be joined.

Then, since GE is equal to EH,
4i and EF is common,

the two sides GE, EF are equal to the two sides HE, EF;
and the angle GEF is equal to the angle HEF ;

therefore the base FG is equal to the base FH. [i- 4]
I say again that another straight line equal to FG will no;
so fall on the circle from the point F.
For, if possible, let FK so fall.
Then, since FK is equal to FG, and FH to FG,
' tJ,ii. • FK is also equal to FH, -n-i.

the nearer to the straight line through the centre being
ss thus equal to the more remote : which is impossible.

Therefore another straight line equal to GF m\ not fall
from the point F upon the circle ;

therefore only one straight line will so fall.
Therefore etc.
''
\end{proof}

\begin{annotations}

4, of the same diameter. I have iriseite<i Ihese words Tot clearness* sake. The text
Kas simply Xaxri A ij X«ir, `` and the remaining (straight line) least.''

7, 39. one on CAcb side. The word *' one * is not in the Greek, but is necessary to
g:iv< the force of t' hdrtpa r iKaxirnp, literally `` on both sides,'' or `` on each nrtlie two
sides, of the leajit.''

\end{annotations}

\begin{notes}

De Morgan points out that there is an unproved assumption in this
tietnonstration. We draw straight lintjs from F, as FB, FC, such that the
angle DFB is greater than the angle DFC and then assume, with respect to
the straight lines drawn from the centre E to B, C, that
the angle DEB is greater than the angle DEC. This
Is most easily pitived, I think, by means of the converse
of part of the theorem about the lengths of different
straight lines drawn to a given straight line from an
external point which was mentioned above in the note
on III. J. This converse would be to the effect that, 1/
two unequal straight lines be drawn from a point to a
gitxn straight line whieh are not perpendicular to the
straight line, tht greater of the hm is tht further from the perfettdicular from the
point to the given straight line. This can either be proved from its converse by
rtductio ad absurdum, or established directly by means of i. 47. Thus, in the
accompanying figure, FB must cut .£C in some point M, since the angle BFE
is less than the angle CFE.

Therefore EM is less than EC, and therefore than EB. tuff

Hence the point B in which FB meets the cticie is further from the foot
of the petpendicuUr from E on FB than i£'  ;

therefore the angle BEF  greater than the angle CEF,

Another way of enunciating the first part of the proposition is that of
Mr H. M. Taylor, viz. `` Of all straight lines drawn to a circle from an internal
point not the centre, the one which passes through the centre is the greatest,
and the one which when produced passes through the centre is the least; and
of any two others the one which mbUnds the greater angle at the centre is the
greater.'' The substitution of the angle subtended at the antre as the criterion
no doubt has the effect of avoiding the necessity of dealing with the unproved
assumption in Euclid's proof referred to above, and the similar substitution in
the enunciation of the first part of i[i. 8 has the effect of avoidmg the necessity
tor dealing with like unproved assumptions in Euclid's proof, as well as the
complication caused by the distinction in Euclid's enunciation between lines
falling from an external point on the convex eircumfercnce and on the ccneave
dreumfe'rence of a circle respectively, terms which are not defined but taken as
understood.

Mr Nixon (Euclid Eevised) similarly substitutes as the criterion the angle
subtended at the centre, but gives as his reason that the words `` nearer `` and
`` more remote `` in Euclid's enunciation are scarcely clear enough without
some definition of the sense in which they are used, Smith and Bryant make
the substitution in iii. 8, but follow Euclid in lii. 7.

On the whole, 1 think that Euclid's plan of taking straight lines drawn from
the point which is not the centre direct to the circumference and making
greater or less angles at that point with the straight line containing it and the
centre b the more instructive and useful of the two, since it is such lines
drawn in any manner to the circte from the point which are immediately useful
in the proofs of later propositions or in resolving difficulties connected with
those proofs.

Heton again (an-Nairizi, ed, Curtze, pp. 1145)  * ``o' o'' ''*'*
proposition which is curious. He first of alt says that Euclid proves that lines
nearer the centre are greater than those more remote fi'om it. This is a
different view of the question from that taken in Euclid's proposition as we
have it, in which the lines are not nearer to and more remote from the centre
but from the line through the centre. Euclid takes lines inclined to the latter
line at a greater or less angle ; Heron introduces distance from the centre in
the sense of Deff. 4, 5, i.e. in the sense of the length of the perpendicular drawn
to the line from the centre, which Euclid does not use till iii, t4, 15. Heron
then obsen'es that in Euclid's proposition the lines compared are all drawn on
one side of the line through the centre, and sets himself to prove the same
truth of lines on opposite sides which are more or less distant iww the centre.
The new point of view necessitates a quite different line of proof, anticipating
the methods of later propositions.

The first case taken by Heron is that of two straight lines such that the
perpendiculars from the centre on them fall on the lines themselves and not
in either case on the line produced.

Let A be the given point, D the centre, and let
AE be nearer the centre than AF, so that the
perpendicular DG on AE is less than the perpen-
dicular DIfoa AF.

Then sqs. on DG, G£ = sq . on DH, HF,
and sqs. on DG, GA = sqs. on DH, HA.

But sq. on I>G < sq. on DH.

Therefore sq. on GE > sq. on MJl

iind sq. on GA > sq. on ffA,

whence G£ > J/F,

GAHA. ,.- .

Therefore, by addition, AE > AF.
The other case taken by Heron is that where
one perpendicular fails on the line produced, as in
the annexed figure. In this case we prove in like
manner that GE > HF,

and GA > AH,

Thus AE is greater than the sum of HF, AH,
whence, \emph{a fortiori}, AE is greater than the difference
of HF, AH, i.e. than AF.

Heron does not give the third possible case, that, namely, where both
perpendiculars fall on the lines produced, The fact
is that, in this case, the foregoing method breaks
down. Though AE be nearer to the centre than
AF'in the serjse that DG is less than DH,
AE is not greater but less than AF.
Moreover this cannot be proved by the same
method as before.

For, while we can prove that

GE> HF,
GA > AH,
we ca.nnot make any inference as to the comparative length of AE, AF.

To judge by Heron's corresponding note to in. 8, he would, to prove this
case, practically prove iii. 35 first, i.e. prove that, if EA be produced to K
and FA to Z,

rect. FA, AL = lect. £A, AK,
from which he would infer that, since AK AL by the first case,

AE <AF.
An excellent moral can, I think, be drawn from the note of Heroa
Having the appearance of supplementing, or giving an alternative for, Euclid's
proposition, it cannot be said to do more than confuse the subject. Nor was
It necessary to find a new proof for the cast where the two lines which are
compared are on epposiit sides of the diameter, since Euclid shows that for each
line from the point to the circumference on one side of the diameter there is
another of the same length equally inclined to it on the other side.
\end{notes}

\end{proposition}

\begin{proposition}\label{prop:III_8}

\begin{statement}
If a point be taken outside a circle and front the point
straight lines be drawn through to the circle, one of which
is through the centre and the others are drawn at random,
then, of the straight lines which fall on the concave circum-
ference, that through the centre is greatest, while of the rest
ike nearer to that through the centre is always greater than
the more remote, but, of the straight lines falling on the convex
circumference, that between the point and the diameter is least,
while of the rest the nearer to the least is always less than the
more remote and only two equal straight lines will fall on the
circle from the point, one on each side of the least.
\end{statement}

\begin{proof}

Let ABC be a circle, and let a point D be taken outside
ABC; let there be drawn through
from it straight lines DA, DE, DF,
DC, and let DA be through the centre ;
I say that, of the straight lines falling
on the concave circumference AEFC,
the straight line DA through the centre
is greatest,

while DE is greater than DF and DF
than DC;

but, of the straight lines falling on the
convex circumference HLKG, the
straight line DG between the point
and the diameter AG is least; and
the nearer to the least DG is always
less than the more remote, namely DK
than DL, and DL than DH.

For let the centre of the circle ABC be taken [m. i], and
let xthM; let ME, MF, MC, MK, ML, MH be joined.

Then, since AM is equal to EM,
let MD be added to each ;

therefore AD is equal to EM, MD.

But EM, MD are greater than ED ; [i. lo]

therefore AD is also greater than ED.

Again, since ME is equal to MF,

and MD is common,
therefore EM, MD are equal to FM, MD ;

and the angle EMD is greater than the angle FMD ;

therefore the base ED is greater than the base FD.

[..«4]
Similarly we can prove that FD is greater than CD ;

therefore DA is greatest, while DE is greater than DF,

and DF than DC.

Next, since MK, KD are greater than MD, [i. 30]

and MG is equal to MK,

therefore the remainder KD is greater than the remainder
GD,

so that GD is less than KD,

And, since on MD, one of the sides of the triangle MLD,
two straight lines MK, KD were constructed meeting within
the triangle,

therefore MK, KD are less than ML, LD \ ., [i. 21]

and MK is equal to ML ;

therefore the remainder DK is less than the remainder
DL.

Similarly we can prove that DL is also less than DH ;
therefore DG is least, while DK is less than DL, and
DL than DH.

I say also that only two equal straight lines will fall from
the point D on the circle, one on each side of the least DG,

On the straight line MD, and at the point M on it,
let the angle DMB be constructed equal to the angle KMD,
and let DB be joined.

Then, since MK is equal to MB,
and MD is common,

the two sides KM, MD are equal to the two sides BM,
MD respectively ;
and the angle KMD is equal to the angle BMD ;

therefore the base DK is equal to the base DB. [i. 4]

I say that no other straight line equal to the straight line
DK will fall on the circle from the point D.

For, if possible, let a straight line so fall, and let it be DN,
Then, since DK is equal to DN,

while DK is equal to DB,

DB is also equal to DN,
that is, the nearer to the least DG equal to the more remote:
which was proved impossible.

Therefore no more than two equal straight lines will fall
on the circle ABC from the point D, one on each side ot
ZJ£? the least.

Therefore etc.
\end{proof}

\begin{notes}

As De Morgan points out, there are here two assumptions similar to
that tacitly made in the proof of iii. 7, nameSy that
K falls within the triangle DLM and E outside
the triangle DFM. These facts can be proved
in the same way as the assumption in iii. 7. Let
DE meet FM in K and LM in Z Then, as
before, MZ is less than ML and therefore than
MK, Therefore K lies further than Z from
the foot of the perpendicular from M on DE.
Similarly E lies further than Y from the foot of the
same perpendicular.

Heron deals with lines on opposite sides of the
diameter through the external point in a manner similar to that adopted in
his previous note.

For the case where E, F sk the seeond points in
which AE, AF meet the circle the method answers
well enough.

If AE is nearer the centre D than AF is,

sqs. on DG, GE = sqs. on DH, HF
and sqs, on DG GA = sqs, on DH, HA,

whence, since
it follows that

and
so that, by addition.

DG < DM,
GE>HF,
AG>AIf,

AE > AF.-
But, if , Z be the points in which AE, A F first
meet the circle, the method fails, and Heron is reduced to proving, in the first
instance, the property usually deduced from 111. 36. He argues thus :
AKD being an obtuse angle,
sq, on AD = sum of sqs. on AK, KD and twice rect AK, KG. [». 11]
ALD is also an obtuse angle, and it follows that

sum of sqs. on AK, KD and twice rect, AK, KG is equal to

sum of sqs. on AL, LD and twice rect. AL, LB. '``

Therefore, the squares on KD, LD being eqjal,
sq on AK ATid. twice rect AK, KG = sq. on AL and twice rect. AL, LH,
or sq on AKm6 rect. AK, £ = sq. on AL and rect. AL, LF,

i.e. rect. AK, AE = reci. AL, AF.

But, by the first part, AE > AF.

Therefore AK<AL.

in. 7, 8 deal with the lengths of the several lines drawn to the circum-
ference of a circle (1) from a point within it, (2) from a point outside it; but a
similar proposition is true of straight lines drawn from a point on the
circumference itself: If any point be taken on the circumference of a circle
then, ofalltht straight lines which can be drawn from it to the circumference, the
greatest is that in which the centre is ; of any others that which is nearer to the
straight line which passes through the centre is greater than one more remote ;
and from the same point there can be drawn to the circumference two straight
lines, and only tvo, which are equal fo one another, one on each side of the
greatest line.

The converses of in, 7, 8 and of the proposition just given are also true
and can easily be proved by reducHo ad ahurdum. They could be employed
to throw light on such questions as that of internal contact, and the relative
position of the centres of circles so touching. This is clear when part of the
converses is stated : thus (i) if from any point in the plane of a circle a
number of straight lines be drawn to the circumference of the circle, and one
of these is greater than any other, the centre of the circle must lie on that one,
(1) if one of them is less than any other, then, (a) if the point is within the
circle the centre is on the minimum straight line produced iemi the point,
(i) if the point is outside the circle, the centre is on the minimum straight line
prioduced btyond the point in which if meets the drck.

\end{notes}

\end{proposition}

\begin{proposition}\label{prop:III_9}

\begin{statement}
If a point be taken within a circle, and more than two
equal straight lines fall from the point on the circle, the point
taken is the centre of the circle.
\end{statement}

\begin{proof}

Let ABC be a circle and D a point within it, and from
D let more than two equal straight
lines, namely DA, DB, DC, fall on
the circle ABC ;

I say that the point D is the centre
of the circle ABC.

For let AB, BC be joined and
bisected at the points B, F, and let
ED, FD be joined and drawn through
to the points G, K, H, L.

Then, since AE is equal to EB,
and ED is common,

the two sides AF, ED are equal to the two sides BE, ED ;

and the base DA is equal to the base DB ;

therefore the angle AED is equal to the angle BED.

[1.8]
Therefore each of the angles AED, BED is right ;

[i. Def. lo]
therefore GK cuts AB into two equal parts and at right
angles.

And since, if in a circle a straight line cut a straight line
into two equal parts and at right angles, the centre of the
circle is on the cutting straight line, [ni, i. Pot.)

the centre of the circle is on GK.

For the same reason

the centre of the circle ABC is also on HL.

And the straight lines GK, HL have no other point
common but the point D ;

therefore the point D is the centre of the circle ABC.

Therefore etc.
\end{proof}

\begin{notes}

The result of this proposition is quoted by Aristotle, MettorolegUa in, 3,
373 a 13 — 16 (cf, note on i. 8).

III. 9 is, as De Morgan remarks, a loguai equivalent of part of in. 7,
where it is proved that every (w>«-centra.l point is not a point from which three
equal straight lines can be drawn to the circle. Thus 111. 7 says that every
nht-A is not-B, and in. 9 states the equivalent fact that every B   A.
Mr H. M. Taylor does in effect make a logical inference of the theorem that,
If from a point three equal straight linei tan be drawn (0 a circle that point is
the centre, by making it a corollary to his proposition which includes the part of
in. 7 referred to. Euclid does not allow himself these logical inferences, as we
shall have occasion to observe elsewhere also.

Of the two proofs of this proposition given in earlier texts of Euclid,
August and Heiberg regard that translated above as genuine, relegating the
other, which Simson gave alone, to a place in an Appendix. Camerer remarks
that the genuine proof should also have contemplated the case in which one
or other of the straight lines AB, BC passes through D. This would however
have been a departure from Euclid's manner of taking the most obscure case
for proof and leaving others to the reader.

The other proof, that selected by Simson, is as follows :

`` For let a point D be taken within the circle ABC, and from D let more
than two equal straight lines, namely AD, DB, DC,
fall on the circle ABC ;

I say that the point D so taken is the centrt: of the
circle ABC.

For suppose it is not ; but, if possible, let it be
£, and let D£ be joined and carried through to the
points J, G.

Therefore fV is a diameter of the circle ABC.

Since, then, on the diameter FG of the circle
ABC a point has been taken which is not the centre
of the circle, namely D,

DG n greatest, and DC is greater than DB, and DB than DA, .

But the latter are also equal : which is impossible

Therefore E is not the centre of the circle.

Similarly we can prove that neither is any other point except D;
therefore the point D is the centre of the circle ABC. `` '

.... r ..I Q.E.D.''

On this Todhunter correctly points out that the point E might be
supposed to fall within the angle ADC. It cannot then be shown that DC
is greater than DB and DB than DA, but only that either i?C or DA is [ess
than DB ; this however is sufficient for establishing the proposition.

\end{notes}

\end{proposition}

\begin{proposition}\label{prop:III_10}

\begin{statement}
A circle does not cut a circle at more points than two.
\end{statement}

\begin{proof}

For, if possible, let the circle ABC cut the circle DBF
at more points than two, namely
B, C, /*, Ii

let BH, BG be joined and
bisected at the points K, L,
and from K, L let KC, LM be
drawn at right angles to BH,
BG and carried through to the
points A, E.

Then, since in the circle
ABC a straight line AC cuts a
straight line BH into two equal
parts and at right angles,

the centre of the circle ABC is on AC. [in- i, For.]

Again, since in the same circle ABC a straight line NO
cuts a straight line BG into two equal parts and at right
angles,

the centre of the circle ABC is on NO.

But it was also proved to be on AC, and the straight
lines AC, NO meet at no point except at P ;

therefore the point P is the centre of the circle ABC.

Similarly we can prove that P is also the centre of the
circle DEF

therefore the two circles ABC, DBF which cut one
another have the same centre P : which is impossible, [in- s]

Therefore etc.
\end{proof}

\begin{annotations}

I. The won) circle (niiXii)) ii here employed in the uousual (Case of the eireum/tremt
(nptptM] of « drck. Cf. note on i. Der. ii,.

\end{annotations}

\begin{notes}

There is nothing in the demonstration of this proposition which assumes
that the circles cul one another ; it proves that two circles cannot mtet at mor
than two points, whether they cut or meet without cutting, i.e. iouch one
another,

Hete again, of two demonstrations given in the earlier texts, Simson chos«
the second, which Au(;u3t and Keilicrg relegate to an Appendix and which is
as follows :

`` For again let the circle ABC cut the circle DEF at more points than
two, namely B, G, H, F

let the centre K of the circle ABC be taken, and let KB, KG KF be
joined.

Since then a point K has been taken within the circle DEF,
and from K more than two straight lines, namely
KB, KF, KG, have fallen on the circle DEF,
the point A' is the centre of the circle DEF. [in. 9]

But K is also the centre of the circle ABC.

Therefore two circles cutting one another have
the same centre K : which is impossible, [111. 5]

Therefore a circle does not cut a circle at more
points than two.

Q.E.D.''

This demonstration is claimed by Heron (see an-Nairīzī, ed, Curtie,
pp. I JO — i). It is incomplete because it assumes that the point K which is
taken as the centre of the circle ABC is within the circle DEF. It can
however be completed by means of hi. 8 and the corresponding proposition
with reference to a point on the circumference of a circle which was enunciated
in the note on m. 8. For (i) if the point K is en the circumference of the
circle DEF, we obtain a contradiction of the latter proposition which asserts
that only two equal straight lines can be drawn from K to the circumference
of the circle DEF; (i) if the point K is outside the circle DEF, we obtain a
contradiction of the corresponding part of [ii. 8.

Euclid's proof contains an unproved assumption, namely that the lines
bisecting BG, BH at right angles will meet in a point P. For a discussion
of this assumption see note on ir. 5.

\end{notes}

\end{proposition}

\begin{proposition}\label{prop:III_11}

\begin{statement}
If tivo circles touch one another internally, and their centres
be taken, the straight line joining their centres, if it be also
produced, will fall on the point of contact of the circles.
\end{statement}

\begin{proof}

For let the two circles ABC, ADE touch one another
internally at the point A, and lei
the centre F of the circle ABC, and
the centre G of ADE, be taken ;
I say that the straight line joined
from G Xo F and produced will fall
on A.

For suppose it does not, but,
if possible, let it fall hs, FGH, and
let A F, AG he joined.

Then, since JG, 6''/ are greater
than FA, that is, than FH,

let FG be subtracted from each ;

therefore the remainder AG is greater than the remainder
GH.

But AG is equal to GD ;
therefore GD is also greater than G//,
the less than the greater : which is impossible.
Therefore the straight line joined from F to G will not
fall outside ;

therefore it will fall at A on the point of contact.
Therefore etc,
\end{proof}

\begin{annotations}

1. the straight line joining their centres, literally ``the straight line joined to their

3. point of contact is here trtira, and in the enunciation ur the next
proposition

\end{annotations}

\begin{notes}

Again August and Heiberg give in an Appendix the additional or
alternative proof, which however shows little or no variation from the genuine
proof and can therefore well be dispensed with.

The genuine proof is beset with difficulties in consequence of what tt
tacitly assumes in the figure, on the ground, probably, of its being obvious to
the eye, Camerer has set out these difficulties in a most careful ote, the
heads of which tnay be given as follows :

He observes, first, that the straight line joining the centres, when produced,
must necessarily (though this is not stated by Euclid) he produced in the
dirtdion of the centre of the circle which touches the ether inltrnally. (For
brevity, I shall call this circle the `` inner circle,'' though I shall imply nothing
by that term except thai it is the circle which touches the other on the inner
side of the latter, and therefore that, in accordance with the definition of
touching, points on it in the immediate neighbourhood of the point of contact
are necessarily within the circle which it touches.) Camerer then proceeds by
the following steps.

T. The two circles, touching at the given point, cannot intersect at any
f>oint. For, since points on the ``inner'' in the immediate neighbourhood of
the point of contact are within the ``outer'' circle, the inner circle, if it
intersects the other anywhere, must pass outside it and then return. This is
only possible (o) if it passes out at one point and returns at another point, or
(b) if it passes out and returns through one and the same point (a) is impossible
because it would require two circles to have three common points ; (i) would
require that the inner circle should have a node at the point where it passes
outside the other, and this is proved to be impossible by drawing any radius
cutting both loops.

*. Since the circles cannot intersect, one must be entire within the
other.

3, Therefore the outer circle must be greater than the inner, and the
radius of the outer greater than that of the inner.

4. Now, if /  be the centre of the greater and G of the inner circle, and
if FG produced beyond G does not pass throth A, the given point of
contact, then there are three possible hypotheses. ;. ,

(a) A may lie on GF produced beyond F.     .

() A may lie outside the line FG altogether, in which case JG produced

beyond G must, in consequence of result 3 above, either
(i) meet the circles in a point common to both, or
(ii) meet the cirdes in two points, of which that which is on the inner

circle is nearer to G than the other is.

(a) is then proved to be impossible by means of the fact that the radius of the

inner circle is less than the radius of the outer.

() (ii) is Euclid's case ; and his proof holds equally of () (i), the hypothois,

namely, that £> and Jf in the figure coincide-
Thus all alternative hypotheses are successively shown to be impossible,

and the proposition is completely established.

I think, however, that this procedure may be somewhat shortened in the
following manner.

In order to make Euclid's proof absolutely conclusive we have only (i) to
take care to produce /G beyond G, the centre of the `` inner `` circle, and then
(a) to prove that the point in which JG so produced meets the *' inner `` circle
is nai further from G than is the point in which it meets the other circle.
Euclid's proof is equally valid whether the first point is nearer to G than the
second or the first point and the second coincide.

If FG produced beyond G does not pass through A, there are two

conceivable hypotheses : (a) A may lie on GF produced beyond F, or (i) A
may be outside FG produced either way. In either case, if FG produced
meets the `` inner `` circle in D and the other in H, and if GD is greater than
GH, then the `` inner `` circle must cut the `` outer `` circle at some point
between A and D, say X.

But if two circles have a common point X lying on one side of the line of
centres, they must have another conesponding point on the other side of the
line of centres. This is clear from in. 7, 8 ; for the point is determined by
drawing from F and G, on the opposite side to that where X is, straight
lines FY, G Y making with FD angles equal to the angles DFX, DGX
respectively.

Hence the two circles will have at least three points common : which is
impossible.

Therefore GD cannot be greater than GH; accordingly GD must be
either equal to, or less than, GH, and Euclid's proof is valid.

The particular hypothesis in which FG is supposed to be in the same
straight line with A but G is on the side of Fway from A is easily disposed
of, and would in any case have been left to the reader by Euclid.

For GD is either equal to or less than GH.

Therefore GD is less than Fff, and therefore less than FA,

But GD is equal to GA, and therefore greater than FA : which is
impossible.

Subject to the same preliminary investigation as that required by Euclid's
proof, the proposition can also be proved directly from iii. 7.

For, by iii. 7, GH  the shortest straight line that can be drawn from G
to the circle with centre F;

therefore GHs less than GA, '

and therefore less than GD : which is absurd.

This proposition is the crucial one as regards circles which touch internally;
and, when it is once established, the relative position of the circles can be
completely elucidated by means of it and the propositions which have preceded
it. Thus, in the annexed fure, if  be the centre
of the outer circle and G the centre of the inner,
and if any radius FQ of the outer circle meet the
two circles in Q, P respectively, it follows, from
III. 7, in. 8, or the corresponding theorem with
reference to a point on the circumference, that FA
is the maximum straight line from .to the circum-
ference of the inner circle, FP is less than FA,
and FP diminishes in length as FQ moves round
from FA until FP reaches its minimum length
FB. Hence the circles do not meet at any other
point than A, and the distance PQ cut off between them on any radius FQ
of the outer circle becomes greater and greater as FQ moves round from FA
to FC and is a maximum when FQ coincides with FC, after which it
diminishes again on the other side of FC.

The same consideration gives the partial converse of in. 11 which forms
the 6th lemma of Pappus to the first book of the Tactioms of Apollonius
(Pappus, vn. p. 826). This is to the effect that, if h', AC art in ont straight
lim, and on ont side of A, tht cirda described on AB, AC as diameters touch
(internally at the point A). Pappus concludes this from the fact that the
circles have a common tangent at A ; but the truth of it is clear from the fact
that FP diminishes as FQ moves away from FA on either side ; whence the
circles meet at A hut do not cut one another.

Pappus' 5th lemma (vn. p. 824) is another partial converse, namely that,
pven two circles touching internally at A, and a lint ABC drawn from A cutting
both, then, if the centre of the outer circle lies on ABC, so does the centre -of the
inner. Pappus himself proves this, by means of the common tangent to the
circles at A, in two ways, (i) The tangent is at right angles to .C and
therefore to AB'. therefore the centre Qi the inner circle lies on AB. (2) By
in. 32, the angles in the alternate segnients of both circles are right angles, so
that ABC is a diameter of both.

\end{notes}

\end{proposition}

\begin{proposition}\label{prop:III_12}

\begin{statement}
If two circles touch one another externally, the straight
line joining their centres will pass through the point of
contact.
\end{statement}

\begin{proof}

For let the two circles ABC, ADE touch one another
S externally at the point A, and let the centre Foi ABC, and
the centre G of ADE, be taken ;

I say that the straight line joined from F to G will pass
through the point of contact at A.
For suppose it does not,
'`` but, if possible, let it pass as
FCDG, and let AF, AG be
joined.

Then, since the point F is
the centre of the circle ABC,
IS FA is equal to FC.

Again, since the point G is
the centre of the circle ADE,
GA is equal to GD.
But FA was also proved equal to FC ;
» therefore FA, AG are equal to FC, GD,

so that the whole FG is greater than FA, AG ;
but it is also less [i. zo] : which is impossible,

Therefore the straight line joined from F to G will not
fail to pass through the point of contact at A ;
n therefore it will pass through it.

Therefore etc. >< • Q.E.D.j
\end{proof}

\begin{annotations}

23. win not fall lo pang. The Greek has the doubk negaliye, o(ic dpe V-.'Wwli...
adK iXedrirai, Literally the straight line... will not ttef-as,,.,''
\end{annotations}

\begin{notes}

Heron says on iii, 1 1 : `` Euclid in proposition 1 1 has suppostKl the two
circles to touch internally, made his proposition deal with this case and proved
what was sought in it, Buf I will show how if is to be proved if the contact is
external.'' He then gives substantially the proof and figure of ill, i». It
seems clear that neither Heron nor an-Nairiit had ni. 1 2 in this place,

Campanus and the Arabic edition of Naslraddin at-fQsI have nothing more
of III. 12 than the following addition to 111. 11. ``In the case of external
contact the two lines ae and eb will be greater than ai, whence ad and cb will
be greater than the whole ai, which is false.'' (The points a, b, c, d, e cor-
respond respectively to G, P, C, D, A in the above figure.) It is most
probable that Theon or some other editor added Heron's prt>of in his edition
and made Prop. 12 out of it (an-Nairlit, ed, Curtze, pp. 121 — 2). An-NairM
and Campanus, conformably with what has been said, number Prop. 13 of
Hei berg's text Prop, i z, and so on through the Book.

What was said in the note on the last proposition applies, mutatis mutandis,
to this, Camerer proceeds in the same manner as before ; and we may use
the same alternative argument in this case also.

Euclid's proof is valid provided only that, if FG, joining the assumed
centres, meets the circle with centre F in C and the other circle in D, C is
not within the circle ADE and D is not within the circle ABC. (The proof
is equally valid whether C, D coincide or the successive points are, as drawn
in the figure, in the order F, C, D, G.) Now, if C is within the circle ADE
and D within the circle ABC, the circles must have cut between A and C
and between A and D. Hence, as before, they must also have another
corresponding point common on the other side of CO. That is, the circles
must have three common points : which is impossible.

Hence Euclid's proof is valid W F, A, G form a triangle, and the only
hypothesis which has still to be disproved is the
hypothesis which he would in any case have left to
the reader, namely that A does not lie on FG but
on FG produced in either direction. In this case, as
before, either C, D must coincide or C is nearer
/''than D is. Then the radius FC must be equal
to FA : which is impossible, since FC cannot be
greater than FD, and must therefore be less than
FA.

Given the same preliminaries, in, u can be proved by means of 111. 8,

Again, when the proposition in. 12 is once proved, in, S helps us to prove
at once that the circles He entirely outside each other and have no other
common point than the point of contact.

Among Pappus' lemmas to Aptollonius' Tactiones are the two partial
converses of this proposition corresponding to those given in the last note.
Lemma 4 (vii. p. 824) is to the effect that, tf AB, AC be in one straight tint, B
and C bang on opposite sides 0/ A, the circles drawn on AB, AC as diameters
touih externally at A. Lemma 3 (vii. p. 822) states that, 1/ two circles touch
externally at A and BAC is drawn through h cutting both circles and containing
the centre of one, BAC will also contain the centre of the other. The proofs, as
before, use the common tangent at A.

Mr H. M, Taylor gets over the difficulties involved by in. 11, 12 in a
manner which is most ingenious but not Euclidean. He first proves that, jrtco
circles meet at a point not in the same straight line with their centres, the circles
intersect at thai point ; this is very easily established by means of in. 7, 8 and
the third similar theorem. Then he gives as a corollary the statement that, if
two circles toueh, the point of contact is in the same straight line with their
centres. It is not explained how this is inferred from the substantive
proposition ; it seems, however, to be a logical inference simply. By the
proposition, every A (circles meeting at a point not in the same straight line
with the centre) is B (circles which intersect); therefore every not- is not' A,
i.e. circles which do not intersect do not meet at a point not in the same
straight line with the centres. Now non-intersecting circles may either meet
(i.e. touch) or not meet. In the former case they must meet en the line of
centres ; for, if they met at a point not in that line, they would intersect. But
such a purely logical inference is foreign to Euclid's manner. As De Moian
says, *' Euclid may have been ignorant of the identity of Every X is Y' and
' Every not- Y is noi-X,' for anything that appears in his writings ; he makes
the one follow from the other by a new proof each time `` (quoted in Keynes'
Formal Legie, p. 8r),

There is no difficulty in proving, by means of i. 20, Mr Taylor's next
profMJsition that, if two circles meet at a point which lies in the same straight
line as their centres and is between the centres, the circles touch at that point, and
each circle lies without the ether. But the similar proof, by means of e. so, of
the corresponding theorem for internal contact seems to be open to the same
objection as Euclid's proof of in. 11 in that it assumes without proof that the
circle which has its centre nearest to the point of meeting is the ``inner''
circle. Lastly, in order to prove that, if two circles hm'e a point of contact, they
do not mat at any other point, Mr Taylor uses the qtiestionable corollary.
Therefore in any case his alternative procedure doet not seem preferble to
Euclid's.

The altcjrnative to Eucl.\ HI. ii — 13 which finds most favour in modern
continental text-books (e.g. Lendre, Baltzer, Henrici and Treutlein,
Veronese, Ingrami, Enriques and Amaldi) connects the number, position and
nature of the coincidences between points on two circles with the relation in
which the distance between their centres stands to the length of their radii.
Enriqties and Amaldi, whose treatment of the different cases is typical, give
the following propositions (Veronese gives them in the converse form).

I . If the distance between the centres of two circles is greater than the stint
of the radii, the two circles have no point common and are external to one
another.

Let O, ff be the centres of the circles (which we will call `` the circles
0, O ``), r, r their radii respectively.

Since then OO >r-- r', \emph{a fortiori} OO ->r, and O is therefore exterior to
the circle O.

Next, the circumference of the circle intersects OG in a point A, and
since 0O>r-¥r\ AO>r', and A is
external to the circle O.

But (7A is less than any straight
line, as OB, drawn to the circum-
ference of the circle O [in. 8] ; hence
all points, as B, on the circumference
of the circle are external to the circle

- . . \ yo

Lastly, if C be any point internal

to the circle 0, the sum of (7C, fC is '

greater than (/O, and \emph{a fortiori} grtaXr than r--r'.

But OC is less than r: therefore OC is greater than t', or C is external
to 0.

Similarly we prove that any point on or within the circumference of the
circle O is external to the circle 0-

a. If the distance between the centres of two unequal circles is less than the
difference of the radii, the two circumferences have no eontnton point and the lesser
circle is entirety within the greater.

Let 0, C be the centres of the two circles, r, r' their radii respectively

Since Off <.r — r, \emph{a fortiori} Off < r*, so that is
internal to the circle O.

If A, A' be the points in which the straight line
00 intersects respectively the circumferences of the
circles 0, O,

00 is less than ffA'-OA,
so that (7(3 + OA, or ffA, is less than OA',
and therefore A is internal to the circle ff.

But, of all the straight lines from O to the circumference of the circle O,
OA passing through the centre O is the greatest [in, 7] ;
whence all the points of the circumference of are internal to the circle O.

A similar argument to the preceding will show that all points within the
circle O are internal to the circle O.

3. If the diitance btilvan the centra of two cirdti is equal to tht tuM of Iht

radii, tht two drcumfirtnces have one point (ommon and one onfy, and that point
is on the Um of an f res. Each circle is externa/ to the other.

Let O, C be the centres, r, r the radii of the circles, so that OO i equal
ar*-r'.

Thus 00 is greater than r, so that O
is external to the circle O, and the circum-
ference of the circle O cuts OO in a
point A.

And, since OO is equal to /  + r*, and
OA to r, it follows that 0A is equal to r
so that A belongs also to the circumference
of the circle O.

The proof that all other points on, and
all points within, the circumference of the circle O are external to the cincle O
follows the similar proof of prop. 1 above. And similarly all points (except A)
on, and all points within, the circumference of the circle O are external to the
circle O.

The two circles, having one common point only, touch at that point, which
lies, as shown, on the line of centres. And, since the circles are external to
one another, they touch externally.

4. If the distance between the centres of two unequal circles is equal to tht
difference between the radii, the two circumferences have one point and one only in
common, and that point lies on the line of centres. The lesser circle it within tha
other.

The proof is that of prop. 2 above, mutatis mutandis.

The circles here touch internally at the point on the line of centres.

5. If the distance between the centres of two circles is less than the sum, and
greater than the dierence, of the radii, the two circumferences hive two common
points symmetrically situated with respect to the line of centres but not lying on
that line.

Let O, O x the centres of the two circles, r, r their radii, *' being the
greater, so that

r'-r<Oa <r + *'.

It follows that in any case 00 + /•> r', so that, if DM be taken on ffO
produced equal to r (so that M is on the circumference of the circle 0), At is
external to the circle ff.'

We have to use the same Postulate as in Eucl.\ I, 1 that

An arc of a circle which has one extremity within and the other without a
given circle has one point common with the
latter and only one ; from which it follows,
if we consider two such arcs making a
complete circumference, that, if a circum-
ference of a circle passes through one point
internal to, and one point external to a
given circle, it cuts the latter circle in two
points.

We have then to prove that the circle O,
besides having one point M of its circum-
ference external to the circle ff, has one other point of its circumference (Z)
internal to the latter circle.

Three cases have to be distinguished according as 07 is greater than, equal
to, or less than, the radius r of the lesser circle.

(1) 00' > r, (See the preceding figure.)

Measure OL along Off equal to r, so that
Z lies on the circumference of the circle O.

Then, since Off < r + , OL will be less
than r, so that L is within the circle ff.

(2) Offr.

In this case the circumference of the circle
passes through ff, or L coincides with ff.

(3) Off<r.

If we measure OL along Off equal to r, the point L will lie on the
circumference of the circle O.

Then OLr- Off,
so that O'L < r, and \emph{a fortiori} ffL < r\ so that Z
lies within the circle ff.

Thus, in all three cases, since the circumference
of O passes through one point (M) external to, and
one point (L) internal to, the circle ff, the two
circumferences intersect in two points A, B [Post. J

And A, B cannot lie on the line of centres OO,
since this straight line intersects the circle O in
L, M only, and of these points one is inside, the other outside, the circle O.

Since ABa. common chord of both circles, the straight line bisecting it
at right angles passes through both centres, i.e. is identical with Off

And again by means of 111. 7, 8 we prove that all points except A, B on
the arc ALB lie within the circle ff, and all points except A B on the arc
A MB outside that circle ; and so on.
\end{notes}

\end{proposition}

\begin{proposition}\label{prop:III_13}

\begin{statement}
A circle does not touch a circle at more points t/ian one,
whether it touch it internally or externally.
\end{statement}

\begin{proof}

For, if possible, let the circle ABDC touch the circle
EBFD, first internally, at more
5 points than one, namely D, B.
Let the centre G of the circle
ABDC, and the centre H of
EBFD, be taken.

Therefore the straight line
10 joined from G to /f will fall on
B, D. [in. 11]

Let it so fall, as BGHD.
Then, since the point G is
the centre of the circle A BCD,
IS BG is equal to GD ;
therefore BG is greater than ZfD ; » -- '

therefore B// is much greater than HD.
Again, since the point // is the centre of the circle
EBFD,
ao B// is equal to //D ;

but it was also proved much greater than it : which is
impossible.

Therefore a circle does not touch a circle internally at
more points than one. .

*s I say further that neither does it so touch it externally.
For, if possible, let the circle ACK touch the circle
ABDC at more points than one, namely A, C,
and let AC be joined.

Then, since on the circumference of each of the circles

JO ABDC, A CK two points A , C have been taken at random,

the straight line joining the points will fall within each

, circle ; [m.

but it fell within the circle ABCD and outside ACK

[ni. Def, 3] : which is absurd.

js Therefore a circle does not touch a circle externally at
more points than one.

And it was proved that neither does it so touch it
internally.

Therefore etc.
\end{proof}

\begin{annotations}

3, 7, [4, 37, 30. 33. ABDC Euclid writes ABCD (hew and in the next proposition),
notwithstiuiding the order in which the points are placed in the iieure.

tj, iT- does it so touch it. It is necessary to supply these words which the Gr«ek
(Dri uliti licTin and fri oiSi itTij) leaves to be understood.

\end{annotations}

\begin{notes}

The difficulties which have been felt in regard to the proofs of this
proposition need not trouble us now, because they have already been disposed
gf in the discussion of the more crucial propositions in. 1 1, 11.

Euclid's proof of the first part of the proposition differs from Simson's ;
and we will deal with Euclid's first. On this Cannerer remarks that it is
assumed that the supposed second point of contact lies on the line of centres
productd beyond the centre of tht ``outer'' circle, whereas all that is proved in
III, 1 1 is that the line of centres produced beyond the centre of the `` inner'`` circle
passes through a point of contact. But, by the same argument as that given
on ni, 11, we show that the circles cannot have a point of contact, or even
any common point, outside the line of centres, because, if there were such a
point, there would be a corresponding common point on the other side of the
line, and the circles would have three common points. Hence the only
hypothesis left is that the second point of contact may be on the line of
centres but in the direction of the centre of the ``tfwr'' circle; and Euclid's
proof disposes of thb hypothesis.

Heron (in an-Nairīzī, ed. Curtze, pp. m — 4), curiously enough, does not
question Euclid's assumption chat the line of centres passes through both
points of contact (if double contact is possible) ; but he devotes some space to
proving that the centre of the ``outer'' circle must lie within the ``inner'' circle, a
fact which he represents Euclid as asserting (`` sicut dixit Euclides ``), though
there is no such assertion in our text. The proof of the fact is of course easy.
If the line of centres passes through fe/A points of contact, and the centre of
the ``outer'' circle lies either on or outside the ``inner'' circle, the line of
centres must cut the ``inner'' circle in /hrei points in all: which is impossible,
as Heron shows by the lemma, which he places here (and proves by i. 16),
that a straight line cannot cut the circumfertnct of a circle in mere points
than two.

Simson's proof is as follows (there is no real need for giving two figures as
he does).

`` If it be possible, let the circle EBF touch the circle ABC in more
points than one, and first on the inside, in the
points B, J?; join BD, and draw G/f bisecting
B£> at right angles.

Therefore, because the points B, D are in the
circumference of each of the circles, the straight
line BD falls within each of them : And their
centres are in the straight line GH which bisects
BD at right angles :

Therefore GH passes through the Doint of
ccMitact [ill. I ij ; but it does not pass through it,
because the points B, D are without the straight line GH: which is absurd.

Therefore one circle cannot touch another on the inside in more points
than one.''

On this Camerer remarks that, unless ill. 11 be more completely elucidated
than it is by Euclid's demonstration, which Slmson has, it is not sufficiently
clear that, besides the point of contact in which GH meets the circles, they
cannot have another point of contact either (1) on GH or (i) outside it.
Here again the latter supposition (2) is rendered im possible because in that
case there would be a third common point on the opposite side of GH ; and
the former supf)osition,(i) is that which Euclid's proof destroys.

Simson retains Euclid's proof of the second part of the proposition, though
his own proof of the first part would apply to the second part also if a
reference to iii. 12 were substituted for the reference to in. 11. Euclid might
also have proved the second part by the same method as that which he
employs for the first part.

\end{notes}

\end{proposition}

\begin{proposition}\label{prop:III_14}

\begin{statement}
In a circle equal straight lines are equally distant from
the centre, and those which are equally distant from the centre
are equal to one another.
\end{statement}

\begin{proof}

Let ABDC be a circle, and let ABt CD be equal straight
lines in it ;

I say that AB, CD are equally distant from the centre,

For let the centre of the circle ABDC be taken [jh.
and let it be E\ from E let EF, EG be drawn perpendicular
to AB, CD, and let AE, EC be joined.

Then, since a straight line EF through
the centre cuts a straight line AB not through
the centre at right angles, it also bisects it.

Therefore AF is equal to FB ;
I therefore AB is double of AF,

For the same reason
I CD is also double of CG ;

and AB is equal to CD ;

therefore AF is also equal to CG.
' And, since AE is equal to EC,

the square oxs. AE\ also equal to the square on EC.

But the squares on AF, EFr equal to the square on AE,
for the angle at F is right ;

and the squares on EG, GC are equal to the square on EC,
for the angle at G is right ; [r. 47]

therefore the squares on AF, FE are equal to the
squares on CG, GE,

of which the square on AF is equal to the square on CG,
for AFis equal to CG ;

therefore the square on FE which remains is equal to
the square on EG,

therefore EF is equal to EG
But in a circle straight lines are said to be equally distant
from the centre when the perpendiculars drawn to them from
the centre are equal ; [ni. Def. 4]

therefore AB, CD are equally distant from the centre.

Next, let the straight hnes AB, CD be equally distant
from the centre ; that is, let EF be equal to EG.

I say that AB is also equal to CD. ~ '

For, with the same construction, we can prove, similarly,
that AB is double of AF, and CD of CG.

And, since AE is equal to CE,

the square on AE is equal to the square on CE,
But the squares on EF, FA are equal to the square on AE,
and the squares on EG, GC equal to the square on CE. [i. 47]
Therefore the squares on £/, FA are equal to the
squares on £G, GC, i

of which the square on EF is equal to the square on EG,
for EF is equal to EG

therefore the square on AF which remains is equal to the
square on CG ; , ,

therefore  is equal to CC ``

And y4 is double of AF, and CD double of CG ;

therefore AB is equal to CD.
Therefore etc.
\end{proof}

\begin{notes}

Heron (an-NairixI, pp. 125 — 7) has an elaborate addition to this proposition
in which he proves, first by redtuiio ad aiiurdum, and then directly, that the
centre of the circle falls between the two chords.

\end{notes}

\end{proposition}

\begin{proposition}\label{prop:III_15}

\begin{statement}
Of straight lines in a circle the diameter is greatest,
and of the rest the nearer to the centre is always greater than
the more remote.
\end{statement}

\begin{proof}

Let ABCD be a circle, let AD be its diameter and E
the centre ; and let BC be nearer to the , ,
diameter AD, and FG more remote ;
I say that AD is greatest and BC
greater than FG.

For from the centre E let EH, EK
be drawn perpendicular to BC, FG.

Then, since BC is nearer to the
centre and FG more remote, EK is
greater than EH. [in. Def. 5]

Let EL be made equal to EH,
through L let LM be drawn at right
angles to EK and carried through to N, and let ME, EN,
FE, EG be joined.

Then, since EH is equal to EL,

BC is also equd to MN. [m. 14]

Again, since AE is equal to EM, and ED to EN,
AD is equal to ME, EN.

But ME, EN are greater than MN, [1. ao]

and MN is equal to BC\ '

therefore AD is greater than BC.

And, since the two sides ME, EN are equal to the two
sides FE, EG,

and the angle MEN greater than the angle FEG,

therefore the base MN is greater than the base FG, [i. 14]

But MN was proved equal to BC.

Therefore the diameter AD is greatest and BC greater
than FG.

Theretore etc. g. e. d,
\end{proof}

\begin{annotations}

1. Of straight lines. The Greek leaves these words to be understood.

5. Nearer to the diameter AD, As BC, FG are not In general parallel to AD,
Euclid should have said nearer to the centre.''

\end{annotations}

\begin{notes}

It will be observed that Euclid's proof differs from that given in our text-books (which is Simson's) in that Euclid introduces another line MN, which
is drawn so as to be equal to BC but at right angles to EK and therefore
parallel to FG. Simson dispenses with MNvA tases his proof on a similar
proof by Theodosius (Spkatrica i, 6). He proves that the sum of the squares
on EH, HB is equal to the sum of the squares on EK, KF\ whence he
infers that, since the square on EH'  less than the square on EK, the square
on BH is greater than the square on FK. It may be that Euclid would have
regarded this as too complicated an inference to make without explanation or
without an increase in the number of his axioms. But, on the other hand,
Euclid himself assumes that the angle subtended at the centre by MN is
greater than the angle subtended by FG, or, in other words, that M, N both
fall outside the triangle FEG. This is a similar assumption to that made in
lit, 7, 8, as already noticed; and its truth is obvious because EM, EN, being
r<idii of the circle, are greater than the distances from E to the points in which
MN cuts EF, EG, and therefore the latter points are nearer than M, Nit to
Z, the foot of the perpendicular from E to MN.

Simson adds the converse of the proposition, proving it in the same way
as he proves the proposition itself.

\end{notes}

\end{proposition}

\begin{proposition}\label{prop:III_16}

\begin{statement}
TAe slraighl line dragon ai right angles to the diameter
of a circle from its extremity will fall outside the circle, and
into the space between the straight line and the circumference
another straight line cannot be interposed ; further the angle
of the semicircle is greater, and the remaining angle less, than
any acute rectilineal angle.
\end{statement}

\begin{proof}

Let ABC be a circle about D as centre and AB as
diameter ;

I say that the straight line drawn from A at right angles
to AB from its extremity will fall - ,
outside the circle.

For suppose it does not, but,
if possible, let it fall within as CA,
and let DC be joined.

Since DA is equal to DC,

the angle DAC is also equal to
the angle A CD, [i. s]

But the angle DAC is right ;
therefore the angle ACD is also right :
thus, in the triangle ACD, the two angles DAC, ACD are
equal to two right angles : which is impossible. [i. 1 7]

Therefore the straight line drawn from the point A at
right angles to BA will not fall within the circle.

Similarly we can prove that neither will it fall on the
circumference ;

therefore it will fall outside.

Let it fall as AE ;
I say next that into the space between the straight line AE
and the circumference CHA another straight line cannot be
interposed.

For, if possible, let another straight line be so interposed,
as EA, and let DG be drawn from the point D perpendicular
to EA. -,

Then, since the anrie A GD is right, '

and the angle DA G is less than a right angle,

AD is greater than DG. [i. 19]

But DA is equal to DH ;

therefore DH is greater than DG, the less than the
greater : which is impossible.

Therefore another straight line cannot be interposed into
the space between the straight line and the circumference,

I say further that the angle of the semicircle contained by
the straight line BA and the circumference CHA is greater
than any acute rectilineal angle,

and the remaining angle contained by the circumference CHA

and the straight line AE is less than any acute rectilineal angle.

For, if there is any rectilineal angle greater than the

angle contained by the straight line BA and the circumference

III. 16] PROPOSITION i6 39

CHA, and any rectilineal angle less than the angle contained
by the circumference CHA and the straight line AE, then
into the space between the circumference and the straight line
AE a straight line will be interposed such as will make an
angle contained by straight lines which is greater than the
angle contained by the straight line BA and the circumference
CHA, and another angle contained by straight lines which
is less than the angle contained by the circumference CHA
and the straight line AE.

But such a straight line cannot be interposed ;

therefore there will not be any acute angle contained by
straight lines which is greater than the angle contained by
the straight line BA and the circumference CHA, nor yet
any acute angle contained by straight lines which is less than
the angle contained by the circumference CHA and the
straight line AE. —

PoKiSM. From this it is manifest that the straight line
drawn at right angles to the diameter of a circle from its
extremity touches tne circle.
\end{proof}

\begin{annotations}

,. cannot be Interposed, Ut«ir>lly `` will not fall in between'' (od nptikwaCrat).

\end{annotations}

\begin{notes}

This proposition is historically interesting because of the controversies to
which the last part of it gave rise from the 13th to the 17th centuries.
History was here repeating itself, for it is certain that, in ancient Greece, both
before and after Euclid's time, there had been a great deal of the same sort
of contention about the nature of the `` angle of a semicircle `` and the
``remaining angle'' between the circumference of the semicircle and the
tangent at its extremity. As we have seen (note on i. Def. 8), the latter angle
had a recognised name, iMparoitSij? yuivii, hern-tike or eomitu!ar angle ;
though this term does not appear in Euclid, it is often used by Proclus,
evidently as a term well understood. While it is from Proclus that we get the
best idea of the ancient controversies on this subject, we may, I thinl infer
their prevalence in Euclid's time from this solitary appearance of the two
`` angles `` in the Elements. Along with the definition of the angle 0/ a
segment, it seemi. to show that, although these angles are only mentioned to
be dropped again immediately, and are of no use in elementary geometry, or
even at all, Euclid thought that an allusion to them would be expected of
him ; it is as if he merely meant to guard himself against appearing to ignore
a subject which the geometiers of his time regarded with interest. If this
conjecture b right, the mention of these angles would correspond to the
insertion of definitions of which he makes no use, e.g. those of a rhombus and
a rhomboid.

Proclus has no hesitation in speaking of the `` angle of a semicircle `` and
the ``hom.like angle'' as true angles. I'hus he says that ``angles are contained
by  i. straight line and a circumference in two ways ; for they are either
contained by a straight line and a cunve.Y circumference, like, that of the setni-
circle, or by a straight line and a concave circumference, like the mparodSift ``
(p. 127, II — 14). ``There are mixed lines, as spirals, and angles, as the angle
of a semicircle and the ntpaTOfiSij! `` (p. 104, 16—18). The difficulty which
the ancients felt arose from the very fact which Euclid embodies in this
proposition. Since an angle can be divided by a line, it would seem to be a
magnitude; ``but if it is a magnitude, and all homogeneous magnitudes which
are finite have a ratio to one another, then alt homogeneous angles, or rather
all those on surfaces, will have a ratio to one another, so that the cornitular
will also have a ratio to the rectilineal. But tilings which have a latio to one
another can, if multiplied, exceed one another. Therefore the cornteular
angle will also sometime exceed the rectilineal ; which ts impossible, for it is
proved that the former is less than any rectilineal angle'' (Proclus, p. lai,
24 — I2Z, 6). The nature of contact between straight lines and circles was
also involved in the question, and that this was the subject of controversy
before Euclid's time is clear from the title of a work attributed to Democritus
(13. 420 — 400 B.C.) irtpi £((uapi9 yvuoi/ot  irfpl axaxjOK icuicAtnf fmx trtpaiptjif
On a differenu in a gnomon or on eontaei of a drcU and a sphert. There is,
however, another reading of the first words of this title as given by Diogenes
Iaertius (ex. 47), namely iripl Stapi yfilfLiTt. On a difference of opinion, etc.
May it not be that neither reading is correct, but that the words should be
TTtpt Stopi ytuci't/t t) jrtpt aiicTMx mJicXou not <r<upt)!. On a difference in an
angle or on contact with a circle and a sphere) There would, of course,
hardly be any ``angle'' in conne>tion with the sphere; but I do not think that
this constitutes any difficulty, because the sphere might easily be tacked on as
a kindred subject to tiie circle. A curiously similar collocation of words
appears in a passage of Proclus, though this may be an accident. He sas
(p. 5** 4) `` i y*aviv BiaopOr ktyofJ4v nai aiffiftf auruit' ... and then, in
the next hne but one, tt i rav as rtay kukXiov  rmv cwfuiiv, `` In what
sense do we speak of differences of angles and of increases of than . . . and in
what sense of the contaets (or meetings) of circles or of straight lines ? ``
I cannot help thinking that this subject of comicular angles would have had
a fascination for Democritus as being akin to the question of infinitesimals,
and very much of the same character as the other question which Plutarch
(On Common Notions, xxxix. 3) says that he raised, namely that of the
relation between the base of a cone and a section of it by a plane parallel to
the base and apparenrty, to judge by the context, infinitely near to it : `` if
a cone were cut by a plane parallel to its base, what must we think of the
surfaces of the sections, that they are equal or unequal? For, if they are
unequal, they will make the cone irregular, as having many indentations like
steps, and unevennesses ; but, if they are equal, the sections will be equal,
and the cone will appear to have the property of the cylinder, as being made
up of equal and not unequal circles, which is the height of absurdity.''

The contributions by Democritus to such investigations are further attested
by a passage in the Method of Archimedes discovered by Heiberg In 1906
(Archimedes, ed. Heiberg, Vol. ii. 191 3, p. 430; T. L. Heath, Tl'e Method
of Archimedes, 1912, p. 13), which says that, though Eudoxus was the first to
discover the scientific proof of the propositions (attributed to him) that the
cone and the pyramid are one-third of the cylinder and prism respectively
which have the same base and equal height, they were first stated, without
proof, by Democritus.

A full history of the later controversies about the cornicuiar `` angle ``
cannot be given here ; more on the subject will be found in Camerer's
Euclid (Excursus iv. on 11 1. 16) or in Cantor's Gtschichte der Maihematik.
Vol. 11. (see Contingenzwmkti in the index). But the following short note
about the attitude of certain well-known mathematicians to the question will
perhaps not be out of place, Johannes Campanus, who edited Euclid in
the 13th century, inferred from [ti. 16 that there was a flaw in the principle
that the iransitien from the less to the greater, or vice vers A, fakes place through
all intermediate quantities and therefore through the egxial. If a diameter of a
circle, he says, be moved about its extremity until it takes the position of the
tangent to that circle, then, as lon as it cuts the circle, it makes an acute
angle less than the `` angle of a semicircle `` ; but the moment it ceases to cut,
it niakes a right angle greater than the same `` angle of a semicircle.'' The
rectilineal angle is never, during the transition, egual to the `` angle of a semi-
circle.'' There is therefore an apparent inconsistency with x. 1, and Campanus
could only observe (as he does on that proposition), in explanation of the
paradox, that `` these are not angles in the same sense (univoce), for the
curved and the straight are not things of the same kind without qualification
(simpliciter).'' The argument assumes, of course, that the right angle is
greater than the ``angle of a semicircle.''

Very similar is the statement of the paradox by Cardano (1501 — 1576),
who observed that a quantity may eoutifiually increase without limit, and
another diminish •without limit ; and yet the firsts however iTiereased, may be lest
than the second, however diminished. The first quantity is of course the angle
of contact, as he calls it, which may be `` increased `` indefinitely by drawing
smaller and smaller circles touching the same straight line at the same point,
but will always be less than any acute rectilineal angle however small.

We next come to the French geometer, Peletier (Peletarius), who edited the
Elements in r 557, and whose views on this subject seem to mark a great advance.
Peletier's opinions and arguments are most easily accessible in the account of
them given by Clavius (Christoph Klau[?], 1537 — 1612) in the 1607 edition of
his Euclid. The violence of the controversy between the two will be understood
from the fact that the arguments and counter-aiuments (which sometimes run
into other matters than the particular question at issue) cover, in that book,
xt pages of small print. Peletier held that the `` angle of contact `` was not an
angle at all, that the ``contact of two circles,'' i.e. the ``angle'' between the
circumferences of two circles touching one another internally or externally, is
not a quantity, and that the `` contact of a straight line with a circle `` is not a
quantity either; that angles contained by a diameter and a circumference
whether inside or outside the circle are right angles and equal to rectilineal
right angles, and that angles contained by a diameter and the circumference
in all circles are equal The proof which Peletier gave of the latter pro-
position in a letter to Cardano is sufficiently ingenious. If a greater and
a less semicircle be placed with their diameters terminating at a common
point and lying in a straight line, then (i) the angle the larger obviously
cannot be less than the angle of the smaller. Neither (*) can the former be
greater than the latter ; for, if it were, we could obtain another angle of a
semicircle greater still by drawing a still larger semicircle, and so on, until we
should ultimately have an angle of a. semicircle greater than a right angle ; which
is imp)OSsible. Hence the angles semicircles must all be equal, and the dif-
ferences between them nothing. Having satisfied himself that all angles of
contact are JftfAangles, noqu an titles, and therefore nothings, Peletier holds the
difficulty about x. i to be at an end. He adds the interesting remark that
the essence of an angle is in cutting, not contact, and that a tangent is not
inclined to the circle at the point of contact but is, as it were, immersed in it at
that point, just as much as if the circle did not diverge from it on either side.

The reply of Claviua need not detain us. He argues,' evidently appealing
to the eye, that the angle of contact qan be divided by the arc of a circle
greater than the given one, that the angles of two semicircles of different sizes
cannot be equal, since they do not coincide if they are applied to one another,
that there is nothing to prevent angia of coniact from being quantities, it being
only necessary, in view of x. i, to admit that they are not of the same kind as
rectilineal angles ; lastly that, if the angle of contact had been a nothing,
Euclid would not have given himself so much trouble to prove that it is less
than any acute angle. (The word is dtsudasset, which is certainly an
exaggeration as applied to what is little more than an obiter dictum in in. id.)

Vieta (1540 — 1603) ranged himself on the side of Peletter, maintaining
that the angle of contact is no angle ; only he uses a new method of proof.
The circle, he says, may be regarded as a plane figure with an infinite number
of sides and angles ; but a straight line touching a straight line, however short
it may be, will coincide with that straight line and wilt not make an angle.
Never before, says Cantor (ii,, p. 540), had it been so plainly dccKired what
exactly was to be understood by contact,

Gahleo Galilei (1564 — 164*) seems to have held the same view as Vieta
and to have supported it by a very similar argument derived from the com-
parison of the circle and an inscribed polygon with an infinite number of
sides.

The last writer on the question who must be mentioned is John Wall is
(1616— 1703). He published in 1656 a paper entitled De angttlo contactus et
semicireuli traetatus in which he also maintained that the so-cailed angle was
not a true angle, and was not a quantity, Vincent Leotaud (1555—1672)
took up the cudgels for Clavius in his Cyclomathia which appeared in 1663,
This brought a reply from Wallis in a letter to Leotaud dated 17 February,
1667, but not apparently published till it appeared in A defense of the treatise
of the angle of contact which, with a separate title-page, and date 1G84, was
included in the English edition of his Algebra dated 1685, The essence of
Wallis' position may be put as follows. According to Euclid's definition, a
plane angle is an inclination of two lines; therefore two lines forming an angle
must incline to one another, and, if two lines meet without being inclined to
one another at the point of meeting (which is the case when a circumference
is touched by a straight line), the lines do not form an angle. The `` angle of
contact `` is therefore no angle, because at the point of contact the straight line
is not inclined to the circle but lies on it (1kA.lv<j9, or is coincident with it.
Again, as a point is not a line but a heginning of a line, and a line is not a
surface but a beginning oi a surface, so an angle is not the distance between
two lines, but their initial tendency towards separation : Angulus (seu gradus
divaricatianis) Distantia nen est sed Incef/thius distantiae. How far lines, which
at their point of meeting do not fomt an angle, separate from one another as
they pass on depends on the degree of curvature (gradus curvitatis), and it is
the latter which has to be compared in the case of two lines so meeting. The
arc of a smaller circle is more curved as having as much curvature in a lesser
length, and is therefore curved in a greater degree. Thus what Clavius called
angulus contactus becomes with Wallis gradus curvitatis, the use of which
expression shows that curvature and curvature can be compared according to
one and the same standard. A straight line has the least possible curvature ;
but of the ``angle'' made by it with a curve which it touches we cannot say that
it is greater or less than the `` angle `` which a second curve touching the same
straight line at the same point makes with the first curve ; for in both ca.<«s
there is no true angle at all (cf. Cantor m,, p. 24).

The words usually given as a part of the corollary ``and that a straight line
touches a circle at one point only, since in fact the straight line meeting it in
two points was proved to fall within it `` are omitted by Hetberg as being an
undoubted addition of Theon's. It was Simson who added the further remark
that ``it is evident that there can be but one straight line which touches the
circle at the same point''

\end{notes}

\end{proposition}

\begin{proposition}\label{prop:III_17}

\begin{statement}
From a given point to draw a straight line touching a
given circle.
\end{statement}

\begin{proof}

Let A be the given point, and BCD the given circle ;
thus it is required to draw from the point A a straight line
touching the circle BCD.

For let the centre E of the circle
be taken ; [m, i]

let AE be joined, and with centre E
and distance EA let the circle AEG
be described ;

from D Jet DF be drawn at right
angles to EA,
and let £F, AB h joined ;
I say that AB has been drawn from
the point A touching the circle BCD.

For, since E is the centre of the circles BCD, AFG,
EA is equal to EF, and ED to EB ;
therefore the two sides AE, EB are equal to the two sides
FE, ED :
and they contain a common angle, the angle at E ;

therefore the base DF is equal to the base AB,
.1 and the triangle DEE is equal to the triangle BE A,

and the remaining angles to the remaining angles ; [i. 4]
therefore the angle EDF is equal to the angle EBA.

But the angle EDF is right ;
therefore the angle EBA is also right.

Now EB is a radius ; ,
and the straight line drawn at right angles to the diameter
of a circle, from its extremity, touches the circle ; [in. 16, Por,]
therefore AB touches the circle BCD.

Therefore from the given point A the straight line AB
has been drawn touching the circle BCD.
\end{proof}

\begin{notes}

The construction shows, of course, that two straight lines can be drawn
from a givttn external point to touch a ven circle ; and it is equally obvious
that these two straight lines are equal in length and equally inclined to the
abaight line joining the exiemai point to the centre of the given circle.
These facts are given by Heron (an-Nairlzl, p. 130).

It is true that Euclid leaves out the case where the given point lies oit the
circumference of the circle, doubtless because the construction is so directly
indicated by iii. 16, For. as to be scarcely worth a separate statement.

An easier solution is of course possible as soon as we know (ici. 31) that
the angle in a semicircle is a right angle ; for we have only to describe a
circle on AE as diameter, and this circle cuts the given circle in the two points
of contact.

\end{notes}

\end{proposition}

\begin{proposition}\label{prop:III_18}

\begin{statement}
// a straight line touch a circle, and a straight line be
jained from the centre to the point of cop fact, the straight line
so joined wilt be perpendicular to the tangent.
\end{statement}

\begin{proof}

For let a straight line D£ touch the circle ABC at the
point C let the centre F of the
circle ABC be taken, and let FC
be joined from Fo C;
I say that FC is perpendicular to
DE.

For, if not, let FG be drawn
from F perpendicular to DE.

Then, since the angle FGC is
right,

the angle FCG is acute ; [i. 1 7]
and the greater angle is subtended
by the greater side ; [1. 19]

therefore FC is greater than FG.
But FC is equal to FB ;
therefore FB is also greater than FG,

the less than the greater: which is impossible.
Therefore FG is not perpendicular to DE.

Similarly we can prove that neither is any other straight
line except FC ;

therefore FC is perpendicular to DE.
Therefore etc.
\end{proof}

\begin{annotations}

3. the tangent, 4 ifuwreiUr.
\end{annotations}

\begin{notes}

Just as 111. 3 contains two partial converses of the ForUm to lit. i, so
the present proposition and the next give Jwo partial converses of the
corollary to iii, 16, We may show their relation thus: suppose three things,
( r) a tangent at a jwint of a circle, (2) a straight line drawn from the centre to
the point of contact, (t) right angles made at the point of contact [with (i) or
(1) as the case may bej. Then the corollary to in. 16 asserts that (t) and (3)
together give (i), iii. 18 that (t) and (i) give (3), and iii. rg that (t) and (3)
give (1), i.e. that the straight line drawn from the point of contact at right
angles to the tangent passes through the centre.

\end{notes}

\end{proposition}

\begin{proposition}
\label{prop:III_19}

\begin{statement}
If a straight line iottch a circle, and from the point of
contact a straight line be drawn at right angles to the tangent,
the centre of the circle will lie on the straight line so dratvn.
\end{statement}

\begin{proof}

For let a straight line DE touch the circle ABC at the
point C, and from C let CA be
drawn at right angles to DE ;
I say that the centre of the circle
is on A C.

For suppose it is not, but, if
possible, let F be the centre,
and let CF be joined.

Since a straight line DE touches
the circle ABC,

and FC has been joined from the
centre to the point of contact,

FC is perpendicular to DE ; [111, 18]

therefore the angle FCE is right

But the angle ACE is also right ;

therefore the angle FCE is equal to the angle ACE,
the less to the greater : which is impossible. <

Therefore F is not the centre of the circle ABC.

Similarly we can prove that neither is any other point
except a point on AC.

Therefore etc.
\end{proof}

\begin{notes}

We may abo regard iii, 19 as a partial converse of in. 18. Thus suppose
(t) a straight line through the centre, (s) a straight line through the point of
contact, and suppose (3) to mean perpendicular to the tangent ; then iii. tS
asserts that (i) and (2) combined produce (3), and 111. 19 that (1) and (3)
produce (i); while again we may enundate a second partial converse of iii, 18,
corresponding to the statement that (i) and (3) produce (2), to the effect that
a straight line drawn through the centre perpendicular to the tangent passes
through the point of contact.

We may add at this poin or even after the Porism to ill, 16, the theorem
that ttuo circles which touch om another internally or externally have a common
tangent at their point of cotttaei. For the line joining their centres, produced
if necessary, passes through their point of contact, and a straight line diawn
through that point at right angles to the line of centres is a tangent to both
circles.

\end{notes}

\end{proposition}

\begin{proposition}
\label{prop:III_20}

\begin{statement}
In a circle the angle at ihe centre is double of the angle
at the circumference, when the angles have the same circum-
ference as base.
\end{statement}

\begin{proof}

Let ABC be a circle, let the angle BEC be an angle
sat its centre, and the angle BAC an
angle at the circumference, and let
them have the same circumference BC
as base ;

I say that the angle BEC is double of
10 the angle BAC.

For let AE be joined and drawn
through to F.

Then, since EA is equal to EB,
the angle EAB is also equal to the
IS angle EBA ; [1. 5]

therefore the angles EAB, EBA are double of the angle
EAB.

But the angle BEF is equal to the angles EAB, EBA ;

['  3–1
therefore the angle BEF is also double of the angle
taEAB.
' For the same reason

the angle FEC is also double of the angle EAC.
Therefore the whole angle BEC is double of the whole
angle BAC.
as Again let another straight line be inflected, and let there
be another angle BDC\ let DE be joined and produced
to G,

Similarly then we can prove that the angle GEC is
double of the angle EDC,
» of which the angle GEB is double of the angle EDB ;

therefore the angle BEC which remains is double of the
angle BDC.

Therefore etc.
\end{proof}

\begin{annotations}

25. let another straight line be inflected, atM.tin ik ir(Ui> (without (Wcia). The
verb jfXdw (to brtak off) was the regular technical term for drawlnr from a point a (broken)
straight line which hfst tneeis another straight line or curve and is then htnt lnuk ham it
to anmher point, or (in other words) for drawing .straight lines from two points meeting at a
point on a curve or another straight line. Ki>tAff9vt Li one of the geometrical terms he
definition of which must according to Aristotle be assumed [AtuiL Rat, t. o, 76 b 9).

\end{annotations}

\begin{notes}

The early editors, Tartaglia, Commandinus, Peletarius, Clavius and others,
gave the extension of (his proposition to the case where the segment is less
than a semicircle, and where accordingly the `` angle `` corresponding to
Euclid's `` atigle at the centre `` is greater than two right angles. The
convenience of the extension is obvious, and the proof of it is the same as the
first part of Euclid's proof. By means of the extension in. 2 1 is demonstrated
without making two cases; Jti. zz will follow immediately from the fact that
the sum of the `` angles at the centre `` for two segments making up a whole
circle is equal to four right angles; also 111. 31 follows immediately from the
extended proposition.

But all the editors referred Xq were forestalled in this matter by Heron, as
we now learn from the commentary of an-Naitizi (ed. Curtxe, p. 131 sqq.).
Heron gives the extension of Euclid's pro[>osition which, he says, it had been
left for him to make, but which is necessary in order that the caviller may not
be able to say that the next proposition (about the equality of the angles
in any segment) is not established generally, i.e. in the case of a segment less
than a semicircle as well as in the case of a segment greater than a semicircle,
inasmuch as lit. no, as given 'ay Euclid, only enables us to prove it in the
latter case. Heron's enunciation is imt>ortaiit as showing how he describes
what we should now call an `` angle `` greater than two right angles. (The
language of Gherard's translation is, in other respects, a little obscure ; but
tht: meaning is made clear by what follows.)

``The angle,'' Heron says, ``which is at the centre of any circle is double
of the angle which is at the circumference of it vrhen one arc is tk( base of bolk
angles; and the remaining angles whieh are at I he centre, and fill up the four
right angles, are double of the angle at the circumference of the ate which is
subtended by the [original] angle which is at the ceiitre,''

Thus the `` angle greater than two right angles `` is for Heron the sum of
certain ``angles'' in the Euclidean sense of angles less than two right angles.
The particular method of splitting up which Heron adopts will be seen from
his proof, which is in substance as follows.

r Let CDB be an angle at the centre, CAB that at the circumference.
' Produce SD, CDto F,G;

take any point jE on BC and join BE, EC, ED.

Then any angle in the segment BAC is half of the angle SDC; and
tht turn of the angles BDG, GDF, FDC is double of any angle in the
segment BEC.
I'ttof. Since CZ7 is equal, to £Z7, I't./  • ,tuiM v i -

the angles DCE, DEC art equal.

Therefore the exterior angle GDE is equal to
twice the angle DEC.

Similarly the exterior angle EDE is equal to
twice the angle DEB.

By addition, the angles GDE, EDE are double
of the angle BEC.

But
the angle BDC is equal to the angle EDG,

therefore the sum of the anglts BDG, GDF, FDC
is doubU of the angle BEC.

And Euclid has proved the first part of the
proposition, namely that the angle BDC is double
of the angle BAC.

Now, says Heron, BAC is any angle in the segment BAC, and therefore
any angle in the segment BAC is half of the angle BDC.

Therefore all the angles in the segment BAC are equal.

Again, BEC is any angle in the segment BEC and is equal to half the
sum of the angles BDG, GDF, FDC.

Therefore all the angles in (he segment BEC are equal

Hence in. 2 1 is proved generally.

Lastly, says Heron,
since the sum of the angles BUG, GDF, FDC is double of the angle BEC,
and the angle BDC is double of the angle BAC,

therefore, by addition, the sum of four right angles is double of tKe sum of
the angles ..C, BEC.

Hence the angles BAC, BEC are together equal to two right angles, and
III. 12 is proved.

The above notes of Heron show conclusively, if proof were wanted, that
Euclid had no idea of in. zo applying in terms (either as a matter of
enunciation or proof) to the case where the angle at the circumference, or the
angle in the segmenl is oituse. He would not have recognised the `` angle ``
greater than two right angles or the so-called ``straight angle'' as being an
angle at all. This is indeed clear from his definition of an angle as the
ittciinatien ic.r.i,, and from the language used by other later Greek mathe-
maticians where there would be an opportunity for introducing the extension.
Thus Proclus' notion of a ``four-sided triangle'' (cf the note above on the
definition of a triangle) shows that he did not count a re-entrant angle as an
angle, and Zenodorus' application to the same figure of the word ``hollow-
angled `` shows that in that case it was the exterior angle only which he would
have called an angle. Further it would have been inconvenient to have
introduced at the beginning of the Elements an ``angle'' equal to or greater
than two right angles, because other definitions, e;g. that of a right angle,
would have needed a qualification. If an ``angle'' might be equal to two
right angles, one straight line in a straight line with another would have
satisfied Euclid's definition of a right angle. This is noticed by Dodgson
(p. 160), but it is pmctically brought out Dy Proclus on i, 13. ``For he did
not merely say that any straight line standing on a straight line either makes
two right angles or angles equal to two right angles but if it make angles.'
If it stand on the straight line at it extremity and make one angle, is it
possible for this to be equal to two right angles ? It is of course impossible ;
jbr every rectUineai angle is iess than two right angles, as every solid angle is
less than four right angles (p. 291, 13 — 20).'' [It is (rue that it has been
generally held that the meaning of `` angle `` is tacitly extended in vi. 33, but
there is no real ground for this view. See the note on the proposition.!

It will be observed that, following his usual habit, Euclid omits the
demonstration of the case which some editors, e.g. Clavius, have thought it
necessary to give separately, the case namely where one of the lines forming
the angle in the sment passes through the centre. Euclid's proof gives so
obviously the means of proving this that it is properly left out.

Tod hunter observes, what Clavius had also remarked, that there are two
assumptions in the proof of 111. zo, namely that, if A is double of B and C
double of D, then the sum, or difference, of A and C is equal to double the
sum, or difference, of B and D respectively, the assumptions being particular
cases of v. i and v. 5. But of course it is easy to satisfy ourselves of the
correctness of the assumption without any recourse to Book v.

\end{notes}

\end{proposition}

\begin{proposition}
\label{propIII_21}

\begin{statement}
In a circle the angles in ike same segment are equal to one
another.
\end{statement}

\begin{proof}

Let A BCD be a circle, and let the angles BAD, BED
be angles in the same segment BAED ;
I say that the angles BAD, BED are
equal to one another.

For let the centre of the circle
ABCD be taken, and let it be .f ; let
BE, ED be joined.

Now, since the angle BED is at
the centre,

and the angle BAD at the circum-
ference,

and they have the same circumference BCD as base,
therefore the angle BED is double of the angle BAD. [m. 20]

For the same reason

the angle BFD is also double of the angle BED ;
therefore the angle BAD is equal to the angle BED.
/ Therefore etc,

,
\end{proof}

\begin{notes}

Under the restriction that the `` angle at the centre `` used in iii, *o must
be less than two right angles, Euclid's proof of this proposition only applies
to the case of a segment greater than a semicircle, and the case of a segment
equal to or less than a semicircle has to be considered separately. The
simplest proof, of many, seems to be that of Simson.

`` But, if the segment BARD be not greater than a semicircle, let SAD,
BED be angles in it: these also are equal to one
another.

Draw AJlQ the centre, and produce it to C, and
join CE.

Therefore the segment BADC is greater than a
semicircle, and the angles in it SAC, EEC are equal,
by the first case.

For the same reason, because CBED is greater
than a semicircle,

the angl CAD, CED are equal.

Therefore the whole angle BAD is equal to the whole angle BED.''

We can prove, by means of redudio ad absurdiim, the important converse
of this proposition, namely that, if there be any hoo Iriangies on the same base
and on the same side of it, and with equal vertical angles, the circle passing
through the extremities of the base and the vertex of one triangle will pass
through the vertex of the other triangle also. That a circle can be thus
described about a triangle is clear from Euclid's construction in itt, 9, which
shows how to draw a circle passing through any three points, though it is
in tv. 5 only that we have the problem stated. Now,
suppose a circle BAC drawn through the angular
points of a triangle BAC, and let BDC be another
triangle with the same base BC and on the same side
of it, and having its vertical angle D equal to the
angle A. Then shal! the circle pass through D.

For, if it does not, it must pass through some point
E on BD or on BD produced. If then EC be
joined, the angle BEC is equal to the angle BAC,
by in.  , and therefore equal to the angle BDC.
Therefore an exterior angle of a triangle is equal to
the interior and opposite angle; which is impossible, by 1. 16.

Therefore D lies on the circle BA C.

Similarly for any other triangle on the base BC and with vertical angle
equal to A. Thus, if any number of triangles be constructed on the same base
and on the same side of it, with equal vertical angles, the vertices will all lie on
the circumfererue of a segmetit of a circle.

A useful theorem derivable from ill. 21 is given by Serenus (ZV sectUme
coni. Props. 5?, 53).

If ADB be any segment of a circle, and C l>e such a point on the
circumference that AC i equal to CB, and if
there be described with C as centre and radius
CA or CB the circle AI/B, then, ADB being
any other angle in the segment ACB, and BD
being produced to meet the outer segment in
E, the sum of AD, DB is equal to BE.

If BC be produced to meet the outer
segment in F, and FA be joined,

CA, CB, CEaie by hypothesis equal.

Therefore the angle EAC is equal to the
angle AEC.

Also, by Itt. It, the angles ACB, ADB are equal ;
therefore their supplements, the angles jiC/, AJ)£, are equal

Further, by m. ai, the angles AEB, AFB are equal.

Hence in the triangles ACF, ADE two angles are respectively equal ;

therefore the third angles EAD, FAC are equal.

But the angle FAC is equal to the angle AFC, and therefore equal to the
angle AED.

Therefore the angles AED, EAD are equal, or the triangle DEA is
isosceles,

and AD is equal to DE.

Adding BD to both, we see that

BE is equal to the sum of AD and DB.

Now, £F being a diameter of the circle of which the outer segment is
a part,

BF is greater than BE ;

therefore AC, CB are together greater than AD, DB.

And, generally, of ail trianglts oh tht same bas€ and on the same side of it
whkh hat>e tqttai vertical angles, the isosales triangle is that whieh has the
greatest perimeter, and of the others that has the lesser perimeter which is
further from being isosceles.

The theorem of Serenus gives us the means of solving the following
problem given in.Todhunter's Euclid, p 314.

To find a point in the eirmmferenee of a given segment of a circle such that
the straight Una which Jain the point to the extremities of the straight line on
which tlu segment stands may  together equal to a given straight line (the
length of which is of course subject to limits).

Let A CB in the above figure be the given segment. Find, by bisecting
AB at right angles, a point C on it such that  C is equal to CB.

Then with centre C and radius CA or CB describe the segment of a
ctccle AHB on the satne side of AB.

Lastly, with 4 or i' as centre and radius equal to the given straight line
describe a circle. This circle will, if the given straight line be greater than
AB and less than twice AC, meet the outer segment in two points, and if we
join those points to the centre of the circle last drawn (whether A or B\ the
joining straight lines will cut the inner segment in points satisfying the given
condition. If the given straight line be eguai to twice AC, C is of course
the required point. If the given straight line be greater than twice .JC, there
is no possible solution.

\end{notes}

\end{proposition}

\begin{proposition}
\label{propIII_22}

\begin{statement}
The opposite angles of qiutdrilaterah in circles are equal
to two right angles.
\end{statement}

\begin{proof}

Let ABCD be a circle, and let ABCD be a quadrilateral
in it;
1 say that the opposite angt.3 are equal to two right angles.

Let AC, BD be joined.

Then, since in any triangle the three angles are equal to
two right angles, [1. 31]

the three angles CAB, ABC, BCA of the triangle ABC
are equal to two right angles.

But the angle CAB is equal to the
angle BDC, for they are in the same
segment BADC; [m. ii]

and the angle ACS is equal to the angle
ADB, for they are in the same segment
ADCB',

therefore the whole angle ADC is equal
to the angles BAC, ACB.

Let the angle ABC be added to each ; i

therefore the angles ABC, BAC, ACB are equal to the
angles ABC, ADC.

But the angles ABC, BAC, ACB are equal to two right
angles ;

therefore the angles ABC, ADC are also equal to two right
angles.

Similarly we can prove that the angles BAD, DCB are
also equal to two right angles.

Therefore etc,
\end{proof}

\begin{notes}

As Todhunter remarks, the converse of this proposition is true and very
important : if hvo opposite anglts of a quadrilaitral bt togeShtr equal to two
right angin, a dnk may de (ircumsmbed about the quadrilateral. We can, by
the method of in. 9, or by iv. 5, circumscribe 3 circle about the triangle
ABC; and we can then prove, by reductio ad nbsurdum, that the circle
passes through the fourth angular fioint D.

\end{notes}

\end{proposition}

\begin{proposition}
\label{propIII_23}

\begin{statement}
On the same straight line there cannot be constructed two
similar and unequal segments of circles on the same side.
\end{statement}

\begin{proof}

For, if possible, on the same straight line AB let two
similar and unequal segments of circles
ACB, ADB be constructed on the same
side ; -ii °

let A CD be drawn through, and let CB,
DB be joined.

Then, since the segment ACB is
similar to the segment ADB,
and similar segments of circles are those which admit equal

angles, [m. Def. u]

the angle ACB is equal to the angle ADB, the exterior
to the interior : which is impossible. , [i. 16]

Therefore etc.
\end{proof}

\begin{annotations}

I. cannot be conatnicled, ti nwrorrcu, the stmt phnae is in [. 7.

\end{annotations}

\begin{notes}

Clavius and the other early editors point out that, while the words ``on
the same side `` in the enunciation are necessary for Euclid's proof, it is
equally true that neither can there be two similar and unequal segments on
apposite sides of the same straight line ; this is at once made clear by causing
one of the segments to revolve round the base till it is on the same side with
the other.

Simson observes with reason that, while Euclid in the following proposition,
III. 24, thinks It necessary to dispose of the hypothesis that, if two similar
segments on equal bases are applied to one another with the bases coincident,
the segments cannot cut in any other jwint than the extremities of the base
(since otherwise two circles would cut one another in more points than two),
this remark is an equally necessary preliminary to iii. 23, in order that we
may be justified in drawing the segments as being one inside the other-
Simson accordingly begins his proof of in. 23 thus :

``Then, because the circle ACB cuts the circle ADB in the two points
A, B, they cannot cut one another in any other point :

One of the segments must therefore fall within the other.

Let ACS fall within ADB and draw the straight line ACI), etc.''

Simson has also substituted ``not coinciding with one another'' for
``unequal'' in Euclid's enunciation.

Then in i[i. 24 Simson leaves out the words referring to the hypothesis
that the segment AEB when applied to the other CFD may be `` otherwise
placed as CGD'' \ in fact, after stating that AS must coincide with CD, he
merely adds words quoting the result of ni. aj : ``Therefore, the straight line
j4. coinciding with CD, the segment AEB must coincide with the segment
CFD, and is therefore equal to it.''

\end{notes}

\end{proposition}

\begin{proposition}
\label{prop:III_24}

\begin{statement}
Similar segments of circles on equal straight lines are equal
to one another.
\end{statement}

\begin{proof}

For let AEB, CFD be similar segments of circles on
eqtial straight lines AB, CD ;
s I say that the segment AEB is equal to the segment CFD.

For, if the segment AEB be applied to CFD, and if the
point A be placed on C and the straight line AB on CZ?,

the point B will also coincide with the point D, because
AB\ equal to CD ;

10 and,  coinciding with CD, • ; .• .   1

. , •. !;'r(j

the segment AEB will also coincide with CFD. r

For, if the straight line AB coincide with CD but the
segment AEB do not coincide with CFD,

it will either fall within it, or outside it ; • i-

IS or it will fall awry, as CGD, and a circle cuts a circle at more
points than two : which is impossible. [m. 10]

Therefore, if the straight line AB be applied to CD, the
segment AEB will not fail to coincide with CFD also ;

therefore it will coincide with it and will be equal to it.

20 Therefore etc,

' • •
\end{proof}

\begin{annotations}

rj . fftU awiy, TopoWdfc, the same v/ord a used in tfae Uke case in [. S. The word
impHes that the applied figure will partly fall short of, and partly overlap, the Aguie to
which it is applied

\end{annotations}

\begin{notes}

Compare the note on the last proposition. I have put a semicolon instead
of the comma which the Greek text has after ``outside it,'' in order the better
to indicate that the inference ``and a circle cuts a. circle in more points than
two `` only refers to the third hypothesis that the applied segment is ``otherwise
placed (ifapaAXftfft) as CGD.'' The first two hypotheses are disposed of by
a tacit reference to the preceding proposition in. 23,

\end{notes}

\end{proposition}

\begin{proposition}
\label{prop:III_25}

\begin{statement}
Given a segment 0/ a circle, to describe the complete circle
ofiuhich it is a segment.
\end{statement}

\begin{proof}

Let ABC be the given segment of a circle ;

thus it is required to describe the complete circle belonging
to the segment ABC, that is, of which it is a segment.

For let AC he. bisected at D, let DB be drawn from the
point D at right angles to AC, and let AB, be joined ;

the angle ABD is then greater than, equal to, or less
than the angle BAD.

First let it be greater ; ``

and on the straight line BA, and at the point A on it, let
the angle BAE be constructed equal to
the angle ABD; let DB be drawn through
to E, and let EC be joined.

Then, since the angle ABE is equal to
the angle BAE,

the straight line EB is also equal to
EA. [1. 6]

And, since AD is equal to DC,
and DE is common, ., . -< 1

the two sides AD, DE are equal to the two sides CD, DE
respectively ;

and the angle ADE is equal to the angle CDE, for each is

right ; . ,,

therefore the base AE is equal to the base C£. •

But AE was proved equal to BE ;

therefore BE is also equal to CE ;

therefore the three straight lines AE, EB, EC are equal to
one another.

Therefore the circle drawn with centre E and distance
one of the straight lines AE, EB, EC will also pass through
the remaining points and will have been completed, [ni. 9]

Therefore, given a segment of a circle, the complete circle
has been described.

And it is manifest that the segment ABC is less than a
semicircle, because the centre E happens to be outside it.

Similarly, even if. the angle ABD be equal to the angle
BAD,

AD being equal to each of the two BD, DC,

the three straight lines DA, DB, DC will
be equal to one another,

D will be the centre of the completed circle,

and ABC will clearly be a semicircle.

But, if the angle ABD be less than the angle BAD,
and if we construct, on the straight line BA
and at the point A on it, an angle equal to
the angle ABD, the centre will fall on DB
within the segment ABC, and the segment
ABC will clearly be greater than a semi-
circle.

Therefore, given a segment of a circle,
the complete circle has been described.

Q.E.F.
\end{proof}

\begin{annotations}

1. to deaciibe the complete circle itpatravaypdpat tov kAxXov, Utenlly ``Lo describe
the circle tm fo it. *

\end{annotations}

\begin{notes}

It will be remembered that Simson takes first the case in which the angles
ABD, BAD are equal to one another, and then takes the other two cases
together, telling us to ``produce BD, if necessary.'' This is a little shorter
than Euclid's procedure, though Euclid does not repeat the proof of the first
case in giving the third, but only refers to it as equally applicable.

Campanus, Petetarius and others give the solution of this problem in
which we take two chords not parallel and bisect each at rjht angles by
straight lines, which must meet in the centre, since each contains the centre
and they only intersect in one point. Clavius, Billingsley, Barrow and others
give the rather simpler solution in which the two chords have one extremity
common (cf. Euclid's proofs of lit. g, ro). This method De Morgan favours,
and (as noted on in. i above) would make iii, i, this proposition, and
IV, 5 all coroilaries of the theorem that `` the line which bisects a chord
perpendicularly must contain the centre,'' Mr H. M. Taylor practically
adopts this order and method, though he finds the centre of a circle by
means of any two non -parallel chords ; but he finds the ctntre of the circle of
whkh a given art is a part (his proposition corresponding to in. 15) by
bisecting at right angles first the base and then the chord joining one extremity
of the base to the point in which the line bisecting the base at right angles
meets the circumference of the segment. Under De Morgan's alternative the
relation between Euclid in. i and the Porism to it would be reversed, and
Euclid's notion of a Porism or corollary would have (o be considerably
extended.

If the problem is solved fter the manner of iv. 5, it is still desirable to
state, as Euclid does, after proving AE, EB, EC to be all equal, that ``the
circle drawn with centre E and distance one of the straight lines AE, EB,
EC will also pass through the remaining points of the segment'' [ni. 9], in
order to show that part of the circle described actually coincides with the
given segment. This is not so clear if the centre is determined as the
intersection of the straight lines bisecting at right angles chords which join
pairs of four different points.

\end{notes}

\end{proposition}

\begin{proposition}
\label{prop:III_26}

\begin{statement}
fn equal arcles equal angles stand on equal arcumferenees,
wheUier they stand at tlie centres or at the circumferences.
\end{statement}

\begin{proof}

Let ABC, DEF be equal circles, and in them let there
be equal angles, namely at the centres the angles BGCt
EHF, and at the circumferences the angles BAC, EDF
I say that the circumference BKC is equal to the circum-
ference ELF.

 1,1

For let i?C,  be joined. • .

Now, since the circles ABC, DEFz.re equal,

the radii are equal.
Thus the two straight lines BG, GC are equal to the
two straight lines EH, HF;

and the angle at G is equal to the angle at H;
therefore the base BC is equal to the base EF. [1. 4]
And, since the angle at A is equal to the angle at D,
the segment BAC is similar to the segment EDF;

[hi. Def. 11]
and they are upon equal straight lines.

But similar segments of circles on equal straight lines are
equal to one another ; [in. 34]

therefore the segment BAC is equal to EDF.
But the whole circle ABC is also equal to the whole circle
DEF:

therefore the circumference BKC which remains is equal to
the circumference j£'Z/*'. ,.-.,...

Therefore etc.
\end{proof}

\begin{notes}

As in HI. 21, if Euclid's ptoof is to cover all cases, it requires us to take
cognisance of `` angles at the centre `` which are equal w or greater than two
 right angles. Otherwise we must deal separately with the cases where the
angle at the circumference is equal to or greater than a right angle. The
case of an ebtust angle at the circumference can of course be reduced by
means of ni. iz to the case of an acute angle at the circumference; and, in
case the angle at the circumference is right, it is readily proved, by drawing
the radii to the vertex of the angle and to the other extremities of the lines
containing it, that the latter two radii are in a straight line, whence they make
equal bases in the two circles as in Euclid's proof.

Lordner has another way of dealing with the right angle or obtuse angte
at the circumference. In either case, he says, ``bisect them, and the halves
of them are equal, and it can be proved, as above, that the arcs upon which
these halves stand are equal, whence it follows that the arcs on which the
given angles stand are equal.''

\end{notes}

\end{proposition}

\begin{proposition}
\label{prop:III_27}

\begin{statement}
/« egua circles angles standing on equal circumferences
are equal ta one another, whether they stand at the centres or
at the circumferences.
\end{statement}

\begin{proof}

For in equal circles ABC, DBF, on equal circumferences
BC, EF, let the angles BGC, EHF stand at the centres G,
H, and the angles BAC, EDF 3X the circumferences ;

I say that the angle BGC is equal to the angle EHF,

and the angle BAC is equal to the angle EDF.

For, if the angle BGC is unequal to the angle EHF,

one of thetn is greater.

Let the angle BGC be greater : and on the straight line BG,

and at the point G on it, let the angle BGK be constructed

equal to the angle EHF. [1. as)

Now equal angles stand on equal circumferences, when
they are at the centres ; [m. a6]

therefore the circumference BK is equal to the circum-
ferencer EF.

But EF is equal to BC ;

therefore BK is also equal to BC, the less to the
greater : which is impossible.

Therefore the angle BGC is not unequal to the angle
EHF;

therefore it is equal to it

And the angle at A is half of the angle BGC,
and the angle at D half of the angle EHF\ [m. ao]

therefore the angle at A is also equal to the angle at D.
Therefore etc.
\end{proof}

\begin{notes}

This proposition is the converse of the preceding one, and the remarks
about the method of treating the different cases apply here also.

\end{notes}

\end{proposition}

\begin{proposition}
\label{prop:III_28}

\begin{statement}
In egtial circles equal straight lines cut off equal circum-
ferences, the greater equal to the greater and Ike less to tlie
less, . ;,
\end{statement}

\begin{proof}

Let ABC, DEF be equal circles, and in the circles let
AB, DE be equal straight lines cutting off ACS, DEE as
greater circumferences and AGB, DHE as lesser;
I say that the greater circumference ACB is equal to the
greater circumference DFE, and the less circumference Gjff
to DHE.

For let the centres K, L of the circles be taken, and let
 A'', , Z?Z., Z. be joined. . -, v, -i ./ j-

Now, since the circles are equal, . .

the radii are also equal ;
therefore the tjvo sides AK, KB are equal to the two
sides DL, LE ;
and the base AB is equal to the base DE ;

therefore the angle A KB is equal to the angle DLE.

[I. 8]
But equal angles stand on equal circumferences, when
they are at the centres ; [m. 26]

therefore the circumference AGB is equal to DHE.

6ai « BOOK. Ill ``;• [HI. a8, J9

And the whole circle ABC is also equal to the whole
circle DEF

therefore the circumference ACB which remains is also equal
to the circumference DFE which remains.

Therefore etc.
\end{proof}

\begin{notes}

Euclid's proof does not in terms cover the particular case in which the
chord in one circle passes through its centre ; but indeed this was scarcely
worth giving, as the proof can easily be supplied. Since the chord in one
circle passes through its centre, the chord in the second circle must also be a
diameter of that circle, for equal circles are those which have equal diameters,
and all other chords in any circle are less than its diameter [in. 15]; hence
the segments cut off in each circle are semicircles, and these must be equal
because the circles are equal.

\end{notes}

\end{proposition}

\begin{proposition}
\label{prop:III_29}

\begin{statement}
In equal circles equal circumferences are subtended by equal
straight lines.
\end{statement}

\begin{proof}

Let ABC, DBF be equal circles, and in them let equal
circumferences BGC, EHF be cut off; and let the straight
lines BC, EF be joined ;
I say that BC is equal to EF.

For let the centres of the circles be taken, and let them
be a: Z ; let BK, KC, EL, Z/ be joined.

Now, since the circumference BGC is equal to the
circumference EHF,

the angle BKC is also equal to the angle .£'Z/ [m. 17]
And, since the circles ABC, DEF are equal,

the radii are also equal ;
therefore the two sides BK KC are equal to the two sides
EL, LF; and they contain equal angles ;

therefore the base BC is equal to the base EF. [i. 4]
Therefore etc.
\end{proof}

\begin{notes}

The particular case of this converse of ill. 28 in which the given arcs are
ares of semicircles is even easier than the corresponding case of in, 18 itself.

The propositions in, z6 — 29 are of couise equaliy true if the same circle
is taken instead of iwo equal circles.

\end{notes}

\end{proposition}

\begin{proposition}
\label{prop:III_30}

\begin{statement}
To bisect a given circumference.
Let ADB be the given circumference ;
thus it is required to bisect the circumference ADB.
\end{statement}

\begin{proof}

Let AS h joined and bisected at
C ; from the point C let CD be drawn o

at right angles to the straight line AB,
and let AJJ, DB be joined.

Then, since ACb equal to CB,
and CD is common,

the two sides A C, CD are equal to the two sides BC, CD ;

and the angle ACD is equal to the angle BCD, for each is
right ;

therefore the base AD is equal to the base DB. [1. 4]

But equal straight lines cut off equal circumferences, the
greater equal to the greater, and the less to the less ; [in- «8]

and each of the circumferences AD, DB is less than a
semicircle ;

therefore the circumference AD is equal to the circum-
ference DB,

Therefore the given circumference has been bisected at
the point D.

Q.E.F.
\end{proof}

\end{proposition}

\begin{proposition}
\label{prop:III_31}

\begin{statement}
In a circle the angle in the semicircle is right, that in a
greater segment less than a right angle, and that in a less
segment greater than a right angle ; and further the angle of
the greater segment is greater than a right angle, and the angle
of the less segment less than a right angle.
\end{statement}

\begin{proof}

- Let ABCD be a circle, let BC be its diameter, and E its
centre, and let BA, AC, AD, DC
be joined ;

I say that the angle BAC in the
semicircle BAC is right,
the angle ABC in the segment -C
greater than the semicircle is less
than a right angle,
and the angle ADC in the segment
ADC less than the semicircle is
greater than a right angle.

Let AE be joined, and let BA ,„ ... , ;

be carried through to . , ., , . i

Then, since BE is equal to EA

the angle A BE is also equal to the angle BAE, [1. s]
; Again, since CE is equal to EA,  ..-i   /r*

the angle ACE is also equal to the angle CAE. [1. s]

Therefore the whole angle BAC is equal to the two angles
ABC, ACB.

But the angle EAC exterior to the triangle ABC is also
equal to the two angles ABC, ACB ; [i. 31]

therefore the angle BAC is also equal to the angle EAC;

therefore each is right ; [1. Def. 10]

therefore the angle BAC in the semicircle BAC is right.

Next, since in the triangle ABC the two angles ABC,
BAC are less than two right angles, [i. 17]

and the angle BAC is a right angle,

the angle ABC is less than a right angle ;
and it is the angle in the segment ABC greater than the
semicircle.

Next, since ABCD is a quadrilateral in a circle,
and the opposite angles of quadrilaterals in circles are equal
to two right angles, [iil »]

while the angle ABC is less than a right angle,
therefore the angle ADC which remains is greater than a
right angle ;

and it is the angle in the segment ADC less than the semi-
circle.

I say further that the angle of the greater segment, namely
that contained by the circumference ABC and the straight
line AC, is greater than a right angle ;

and the angle of the less segment, namely that contained by
the circumference ADC and the straight line AC, is less than
a right angle.

This is at once manifest.
For, since the angle contained by the straight lines BA, AC
is right,

the angle contained by the circumference ABC and the
straight line AC is greater than a right angle.

Again, since the angle contained by the straight lines
AC, AFis right,

the angle contained by the straight line CA and the
circumference ADC is less than a right angle.

Therefore etc.
\end{proof}

\begin{notes}

As already stated, this proposition is immediately deducible from in. 20 if
that theorem Is extended so as to include the case where the segment is equal
to or less than a semicircle, and where consequently the `` angle at the centre''
is equal to two right angles or greater than two right angles respectively.

There are indications in Aristotle that the proof of the first part of the
theorem in use before Euclid's time proceeded on different lines. Two
passages of Aristotle refer to the proposition that the angle in a semicircle
IS a right angle. The first passage is Anal. Fast 11. 11, 94 a 38: ``Why is
the angle in a semicircle a right arvgle? Or what makes it a right angle?
(tivo! ovrtK opij;) Suppose 4 to be a right angle, B half of two right
angles, C the angle in a semicircle. Then B is the cause of j4, the right
angle, being an attribute of C, the angle in the semicircle. For £ is equal to
/*, and CtoB; for C is half of two right angles. Therefore it is in virtue of
£ being half of two right angles that A is an attribute of C ; and the latter
means the fact that the angle in a semicircle is right.'' Now this passage
by itself would be consistent with a proof like Euclid's or the alterrmtive
interpolated proof next to be mentioned. But the second passage throws a
different light on the subject. This is Metaph, 1051 a 26 ; ``Why is the angle
in a semicircle a right angle invariably (dafloAou) ? Because, if there be three
straight lines, two forming tkt base, and the third iet uf at right angles at its
middle point, the fact is obvious by simple inspection to any one who knows
the property referred to'' (Ikuvo is the property that the angles of a triangle
are together equal to two right angles, mentioned two
lines before). That is to say, the anle at the middle
point of the circumference of the semicircle was taken
and proved, by means of the two isosceles right-angled
triangles, to be the sum of two angles each equal to
one-fourth of the sum of the angles of the large triangle
in the figure, or of two right angles; and the proof . ?. -'.t

must have been completed by means of the theorem of lit. zi (that angles
in the same sment are equal), which Euclid's more general proof does
not need.

In the Greek texts before that of August there is an alternative proof
that the angle BAC (in a semicircle) is right. August and Heiberg rel;ate
it to an Appendix.

`` Since the angle AEC is double of the angle BAE (for it is equal to the
two interior and opposite angles), while the angle AEB b also double of the
angle EAC,

the angles AEB, AECatft double of the angle SAC.

But the angles AEB, AEC are equal to two right angles J •  • 1
therefore the angle BAC is right.''

Lardner gives a slightly different proof of the second part of the theorem.
If ABC be a segment greater than a semicircle,
draw the diameter AD and join CH, CA.

Then, in the triangle ACD, the angle ACD is right
(being the angle in a semicircle) ;

therefore the angle ADCs acute.
But the angle ADC i equal to the angle ABC in
the same segment ;

therefore the angle ABC b acute.

Euclid's references in this proposition to the angle of a sment greater
or less than a semicircle respectively seem, like the part of 111. 16 relating to
the angle 0/ a semicircle, to be a survival of ancient controversies and not to
be put in deliberately as being an essential part of elementary geometry. Cf.
the notes on 111, Def. 7 and in. 16.

The corollary ordinarily attached to this proposition is omitted by Heibetg
as an interpolation of date later than Theon. It is to this effect ; `` From
this it is manifest that, if one angle of a triangle be equal to the other two,
the first angle is right because the exterior angle to it is also equal to the
same angles, and if the adjacent angles be equal, they are right.'' No doubt
the corollary is rightly suspected, because there is no necessity for it here, and
the words oirip itti Siifai come before it, not after tt, as is usual with Euclid.
But, on the other hand, as the fact stated does appear in the proof of 111. 31,
the Porism would be a Porism after the usual type, and I do not quite follow
Hei berg's argument that, ``if Euclid had wished to add it, he ought to have
placed It after 1. 3*.''

It has already been mentioned above (p. 44) that this proposition supplies
us with an alternative construction for the problem in 111, 1 7 of drawing the
two tangents to a circle from an external point.

Two theorems of some historical interest which follow directly from in. 3r
may be mentioned.

The first is a lemma of Pappus on `` the
14th problem `` of the second Book of Apol-
lonius' lost treatise on vcvVtit (Pappus vii.
p. 811) and is to this effect. If a circle, as
DEF, pass through D, the centre of a circle
ABC, and if through F, the other point in
which the line of centres meets the circle
DEF, any straight line be drawn (and produced
if necessary) meeting the circle DEF in E and the circle ABC in B, G,
then E is the middle point of £G. For, if UE be joined, the angle I>EJ
(in a. semicircle) is a right angle [iii. 31] ; and DE, being at right angles to
the chord BG of the circle A£C, also bisects it [m. 3].

The second is a proposition in the Zir Asiumpiarttm, attributed (no
doubt erroneously as regards much of it) to Archimedes, which has reached
us through the Arabic (Archimedes, ed, Heiberg, 11. pp. 52© — -5 21)'

If two chords AB, CD iit a circle infersicl at right angles in a point O,
thtn the sum of the squares on AG, BO, CO, DO is equal to the square on the
diameter.

For draw the diameter CE, and join AC, CB, AD, BE. , jj

Then the angle CAO is equal to the angle CES. (This follows, in the
first figure, from iii. 31 and, in the second, from 1. 13 and ill. 22.) Also the
angle COA, being right, is equal to the angle CBE which, being the angle in a
semicircle, is also right [iii. 31].

Therefore the triangles AOC, EBChve two angles equal respectively;
whence the third angles A CO, £CJ5 are equal. (In the second figure the
angle A CO is, by i. 13 and 111. aa, equal to the angle ABD, and therefore
the angles ABD, ECB are equal)

Therefore, in both figures, the arcs AD, BE, and consequently the chords
AD, BE subtended by them, are equal. [111. 36, 29]

Now the squares on AO, DO are equal to the square on AD. 47), that
is, to the square on BE.

And the squares on CO, £0 Mt equal to the square on BC.

Therefore, by addition, the squares on AO, BO, CO, DO are equal to the
squares on EB, BC, i.e. to the square on CE, [1. 47J

\end{notes}

\end{proposition}

\begin{proposition}
\label{prop:III_32}

\begin{statement}
If a straight line touch a circle, and from the point of
contact there be drawn across, in the circle, a straight line
cutting the circle, the angles which it makes with the tangent
will be equal to the angles in the alternate segments of the
circle.
\end{statement}

\begin{proof}

For let a straight line EF touch the circle A BCD at
the point B, and from the point B let there be drawn across,
in the circle ABCD, a straight line BD cutting it ;
I say that the angles which BD makes with the tangent EF
will be equal to the angles in the alternate segments of the
circle, that is, that the angle FBD is equal to the angle
constructed in the segment BAD, and the angle EBD is
equal to the angle constructed iii the
segment DCB.

For let BA be drawn from B at
right angles to EF,
let a point C be taken at random on
the circumference BD,
and let AD, DC, CB be joined.

Then, since a straight line EF
touches the circle A BCD at B,
and BA has been drawn from the point
of contact at right angles to the tangent,
the centre of the circle ABCD is on BA. [m. 19]

Therefore BA is a diameter of the circle ABCD ;

therefore the angle ADB, being an angle in a semicircle,
is right. [ill. 31]

Therefore the remaining angles BAD, ABD are equal to
one right angle. [1. 32]

But the angle ABF is also right ;
therefore the angle ABF is equal to the angles BAD, ABD.

Let the angle ABD be subtracted from each ;
therefore the angle DBF which remains is equal to the angle
BAD in the alternate segment of the circle.

Next, since ABCD is a quadrilateral in a circle,
its opposite angles are equal to two right angles. [iii. a»]

But the angles DBF, DBF are also equal to two right
angles ;

therefore the angles DBF, DBF are equal to the angles
BAD, BCD,

of which the angle BAD was proved equal to the angle
DBF;

therefore the angle DBF which remains is equal to the
angle DCB in the alternate segment DCB of the circle.

Therefore etc.
\end{proof}

\begin{notes}

The converse of this theorem is true, namely that, If a straight iine
drawn through one txtraniiy of a chord of a circle make with that chord
angles equal respectively to the angles in the alternate segments of the ctrde,
the straight line so drawn touches the circle.

This can, as Camerer and Tod hunter remark, be proved indirectly ; or we
may prove it, with Clavius, directly. Let BD be the given chord, and let £F
be drawn through B so that it makes with BD angles equal to the angles in
the alternate segments of the circle respectively.

Let BA be the diameter through B, and let C be any point on the
circumference of the segment DCB which does not contain A. Job A£>f
DC, CB.

Then, since, by hypothesis, the angle FBD is equal to the angle BAD,
let the angle ABI> be added to both;

therefore the angle ABFi equal to the angles AJSD, BAD.

But the angle BDA, being the angle in a semicircle, is a right angle ;

therefore the remaining angles ABD, BAD in the triangle ABD are
equal to a right angle.

Therefore the angle ABFi right ;
hence, since BA is the diameter through B,

£i touches the circle at A [cii. 16, Por,]

Pappus assumes in one place (iv, p. 196) the consequence of this
proposition that, If two eircks touch, any straight line drawn through the point
of (oniad and terminated by both cireiei mis off segments in each which are
respediwly similar. Pappus also shows how to prove this (vii, p, 8i6) by
drawing the cominon tangent at the point of contact and using thb proposition.

\end{notes}

\end{proposition}

\begin{proposition}
\label{prop:III_33}

\begin{statement}
On a given straight line to describe a segment of a circle
admitting an angle equal to a given rectilineal angle.
\end{statement}

\begin{proof}

Let AB be the given straight line, and the angle at C the
given rectilineal angle; . . rv -

thus it is required to describe
on the given straight line
AB a segment of a circle ad-
mitting an angle equal to the
angle at C.

The angle at C is then
acute, or right, or obtuse.

First let it be acute,
and, as in the first figure, on
the straight line AB, and at the point A, let the angle BAD
be constructed equal to the angle at C ;

therefore the angle BAD is also acute.

Let AE h drawn at right angles to DA, let AB be
bisected at /, let FG be drawn from the point F at right
angles to AB, and let GB be joined.

Then, since Af is equal to fB,
and FG is common,

the two sides AF, FG are equal to the two sides BF, FG ;
and the angle AFG is equal to the angle BFG ;

therefore the base AG is equal to the base BG. [i. 4]

Therefore the circle described with centre G and distance
GA will pass through B also.

Let it be drawn, and let it be ABE ;
let EB be joined.

Now, since AD is drawn from A, the extremity of the
diameter AE, at right angles to AE,

therefore AD touches the circle ABE. . 16, Por.]

Since then a straight line AD touches the circle ABE,
and from the point of contact at A a straight line AB is
drawn across in the circle ABE,

the angle DAB is equal to the angle AEB in the alternate
segment of the circle. [ni. 31]

But the angle DAB is equal to the angle at C;
therefore the angle at C is also equal to the angle AEB.

Therefore on the given straight line AB the segment
AEB of a circle has been described admitting the angle AEB
equal to the given angle, the angle at C 1

Next let the angle at C be right ;

and let it be again required to describe on AB a segment
of a circle admitting an angle equal to the right angle at C.

Let the angle BAD be constructed equal to the right
angle at C, as is the case in the second figure ;

let j4B h bisected at /, and with centre / and distance
either IA or /B let the circle AEB be described.

Therefore the straight line AD touches the circle ABE,
because the angle at A is right. [m. i6j Por.]

And the angle BAD is equal to the angle in the segment
AEB, for the latter too is itself a right angle, being an
angle in a semicircle. [ni, 31)

But the angle BAD is also equal to the angle at C.

Therefore the angle AEB is also equal to the angle at C.

Therefore again the segment AEB of a circle has been
described on AB admitting an angle equal to the angle at C.

Next, let the angle at C be obtuse ;

and on the straight line AB, and at the point A, let the
angle BAD be constructed equal to it, as is the case in the
third figure ;

let AE be drawn at right angles to AD, let AB be again
bisected at F, let FG be drawn at right angles to AB, and
let GB be joined.

Then, since AF is again equal to FB,
and FG is common,

the two sides AF, FG are equal to the two sides BF, FG ;
and the angle AFG is equal to the angle BFG ;

therefore the base AG is equal to the base BG. [i- 4]

Therefore the circle described with centre G and distance
GA will pass through B also ; let it so pass, as AEB.

Now, since AD is drawn at right angles to the diameter
AE from its extremity,

AD touches the circle AEB. [m. 16, Por.]

And AB has been drawn across from the point of contact
at W ;

therefore the angle BAD is equal to the angle constructed
in the alternate segment AHB of the circle. [m. 31]

But the angle BAD is equal to the angle at C
Therefore the angle in the segment A MB is also equal to

the angle at C:

Therefore on the given straight line AB the segment

AHB of a circle has been described admitting an angle equal

to the angle at C.

Q.E.F.
\end{proof}

\begin{notes}

Simson remarks truly that the first and third cases, those namely in which
the given angle is acute and obtuse respectively, have exactly the same
construction and demonstration, so that there is no advantage in repeating
them. Accordingly he deals with the cases as one, merely drawing two
different figures. It is also true, as Simson says, that the demonstration of
the second case in which the gi-en angle is a right angle `` is done in a round-
about way,'' whereas, as Clavius showed, the problem can be more easily
solved by merely bisecting AB and describing a semicircle on it. A glance
at Euclid's figure and proof will however show a more curious fact, namely
that he does not, in the proof of the second case, use the angle in the
alternate stgmint, as he does in the other two cases. He might have done so
after proving that AD touches the circle; this would only have required his
point .£ to be placed on the side of AB opposite to D. Instead of this, he
uses III. 31, and proves that the angle AEB is equal to the angle C, because
the former is an angle in a sanicirde, and is therefore a right angle as C is.

The difference of procedure is no doubt owing to the fact that he has not,
in III. 31, distinguished the case in which the cutting and touching straight
lines are at right angles, i.e. in which the two alternate segments are semicircles.
To prove this case would also have required in. 31, so that nothing would
have been gained by stating it separately in in. 32 and then quoting the
result as part of 111. 32, instead of referring directly to in. 31.

It is assumed in Euclid's proof of the first and third cases that AE and
FG will meet; but of course there is no difficulty in satisfying ourselves
of this. J

\end{notes}

\end{proposition}

\begin{proposition}
\label{prop:III_34}

\begin{statement}
From a given circle to cut off a segment admitting an angle
tqual to a given rectilineal angle.
\end{statement}

\begin{proof}

Let ABC be the given circle, and the angle at D the
given rectilineal angle ;

thus it is required to cut off from the circle ABC a segment
admitting an angle equal to the given rectilineal angle, the
angle at D.

Let EFi drawn touching ABC at the point B, and on
the straight line FB, and at the point B on it, let the angle
FBC be constructed equal to the angle at D. [1. 23]

' . Then, since a straight line EF touches the circle ABC
and BC has been drawn across from the point of contact
at,

the angle FBC is equal to the angle constructed in the alternate
segment BAC, [iti. 37]

But the angle FBC is equal to the angle at D ;

therefore the angle in the segment BAC is equal to the
angle at D.

Therefore from the given circle ABC the segment BAC,
has been cut off'' admitting an angle equal to the given recti-
lineal angle, the angle at D.

Q. E, F.
\end{proof}

\begin{notes}

An alternative construction here would be to make an ``angle at the
centre `` (in the extended sense, if necessary) double of the given angle ; and,
if the given angle is right, it is only necessary to draw a diameter of the circle.

\end{notes}

\end{proposition}

\begin{proposition}
\label{prop:III_35}

\begin{statement}
Jf in a circle two straight lines cut one another, the
rectangle contained by the segments of the one is equal to the
rectangle contained by the segments of the other.
\end{statement}

\begin{proof}

For in the circle ABCD let the two straight lines AC,
BD cut one another at the point E ;

I say that the rectangle contained hy AB,
EC is equal to the rectangle contained by
DE, EB.

If now AC, BD are through the centre,
so that E is the centre of the circle ABCD,

it is manifest that, AE, EC, DE, EB
being equal,

the rectangle contained by AE, EC is also equal to the
rectangle contained by DE, EB.

Next let AC, DB not be through the centre ;
let the centre of ABCD be taken, and
let it be F

from F let FG, FH be drawn perpen-
dicular to the straight lines AC, DB,
and let FB, FC, FE be joined.

Then, since a straight line GF
through the centre cuts a straight line
AC not through the centre at right
angles,

it also bisects it ; [in. 3]

therefore AG is equal to GC.

Since, then, the straight line AC has been cut into equal
parts at G and into unequal parts at E,

the rectangle contained by AE, EC together with the square
on EG is equal to the square on GC ; [11. 5]

Let the square on GF be added ;
therefore the rectangle AE, EC together with the squares
on GE, GF is equal to the squares on CG, GF.

But the square on FE is equal to the squares on EG, GF,
and the square on FC Is equal to the squares on CG, GF

therefore the rectangle AE, EC together with the square
on FE is equal to the square on FC.

And FC is equal to FB ;
therefore the rectangle AE, EC together with the square on
EF is equal to the square on FB.

For the same reason, also,
the rectangle DE, EB together with the square on FE is
equal to the square on FB.

But the rectangle AE, EC together with the square on
FE was also proved equal to the square on FB ;
therefore the rectangle AE, EC together with the square on
FE is equal to the rectangle DE, EB together with the
square on FE.

Let the square on FE be subtracted from each ;
therefore the rectangle contained by AE, EC which remains
is equal to the rectangle contained by DE, EB,

Therefore etc.
\end{proof}

\begin{notes}

In addition to the two cases in Euclid's text, Simson (following Campanus)
gi.ves two intermediate cases, namely (i) that in which one chord passes through
the centre and bisects the other which does not pass through the centre at right
angles, and (a) that in which one passes through the centre and cuts the other
which does not pass through the centre but not at right angles Simson then
reduces Euclid's second case, the most general one, to the second of the two
intermediate cases by drawing the diameter through £. His note is as
follows : ``As the 25th and 33rd propositions are divided into more cases,
so this 35th is divided into fewer cases than are necessary. Nor can it be
supposed that Euclid omitted them because they are easy ; as he has given
the case which by far is the easiest of them all, viz, that in which both the
straight lines pass through the centre ; And in the following proposition he
separately demonstrates the case in which the straight line passes through the
centre, and that in which it does not pass through the centre: So that it
seems Theon, or some other, has thought them too long to insert : But cases
that require different demonstrations should not be left out in the Elements,
as was before taken notice of: These cases are in the translation from the
Arabic and are now put into the text.'' Notwithstanding the ingenuity of the
argument based on the separate mention by Euclid of the simplest case of
all, I think the conclusion that Euclid himself gave /our cases is unsafe ; in
fact, in giving the simplest and most difficult cases only, he seems to be
following quite consistently his habit of avoiding Aw reai multiplicity of cases,
while not ignoring their existence.

The deduction from the next proposition (in, 36) which Simson, following
Clavius and others, gives as a corollary to it, namely that, IJ from any point
without a drcU then be drawn two straight lines cutting it, the rectangles
contained by the whole lines and the parts of them without the circle are equal t
one another, can of course be combined with ill. 35 in one enunciation.

As remarked by Todhunter, a large portion of the proofs of in, 35, 36
amounts to proving the proposition, If any point be taken on the bast, or the
base produced, of an isosceles triangle, the rectangle contained by the segments of
the base (i.e. the respective distances of the ends of the base from the point) is
equal to the difference betiveen the square on the straight line joining the point to
the vertex and the square on one of the equal sides of the triangle. This is of
course an immediate consequence of 1, 47 combined with ii. 5 or 11. 6,

The converse of in, 35 and Simson's corollary to lu. 36 may be stated
thus. If two straight lines AB, CXi, produced if necessary, intersect at O, and if
the rectangle AO, OB be equal to the rectangle CO, OD, the circumference of a
circle will pass through the four points A, B, C, D. The proof is indirect.
We describe a circle through three of the points, as A, B, C (by the method
used in Euclid's proofs of tii. 9, 10), and then we prove, by the aid of in. 35
and the corollary to in. 36, that the circle cannot but pass through D also,

\end{notes}

\end{proposition}

\begin{proposition}
\label{prop:III_36}

\begin{statement}
If a point be taken outside a circle and from it there fall
on the circle two straight lines, and if one of tliem cut the
circle and the other touch ii, the rectangle contained by the
whole of the straight line which cuts the circle and the straight
line intercepted on it outside between the point and the convex
circumference will be equal to the square on tke tangent.
\end{statement}

\begin{proof}

For let a point D be taken outside the circle ABC,
and from D let the two straight lines DC A,
DB fall on the circle ABC; let DCA cut
the circle ABC and let BD touch it ;
I say that the rectangle contained by AD,
DC is equal to the square on DB.

Then DCA is either through the centre
or not through the centre.

First let it be through the centre, and
let F be the centre of the circle ABC;
let FB be joined ;

therefore the angle FBD is right. [m. 18]

And, since AC has been bisected at F, and CD is added
to it,

the rectangle AD, DC together with the square on FC is
equal to the square on FD. [11. 6]

But FC is equal to FB ;
therefore the rectangle AD, DC together with the square on
FB is equal to the square on FD.

And the squares on FB, BD are equal to the square on
FD ; [i. 47]

therefore the rectangle AD, DC together with the square on
FB is equal to the squares on FB, BD.

Let the square on FB be subtracted from each ;
therefore the rectangle AD, DC which remains is equal to
the square on the tangent DB.

Again, let DCA not be through the centre of the circle
ABC;

let the centre E be taken, and from E
let EF be drawn perpendicular x.o AC;
let EB, EC, ED be joined.

Then the angle EBD is right.

[ill. 18]
And, since a straight line EF
through the centre cuts a straight line
AC not through the centre at right angles,

it also bisects it ; [in. 3]

therefore AF is equal to FC.

Now, since the straight line Chas been bisected at the
point F, and CD is added to it,

the rectangle contained by AD, DC together with the square
on FC is equal to the square on FD. [11, 6]

Let the square on FE be added to each ;
therefore the rectangle AD, DC together with the squares
on CF, FE is equal to the squares on FD, FE.

But the square on EC is equal to the squares on CF, FE,
for the angle EEC is right ; [1. 47]

and the square on ED is equal to the squares on DF, FE ;
therefore the rectangle AD, DC together with the square on
EC is equal to the square on ED.

And EC is equal to EB ;
therefore the rectangle AD DC together with the square on
EB is equal to the square on ED.

But the squares on EB, BD are equal to the square on
ED, for the angle EBD is right ; [i. 47]

therefore the rectangle AD, DC together with the square on
EB is equal to the squares on EB, BD.

Let the square on EB be subtracted from each ;
therefore the rectangle AD, DC which remains is equal to
the square on DB.

Therefore etc,
\end{proof}

\begin{notes}
Cf. note on the preceding proposition. Observe that, whereas it would
be natural with us to prove first that, if A is an external point, and two
straight lines AEB, AFC cut the circle in E, B and F, C respectively, the
rectangle BA, AE h equal to the rectangle CA, AF, and thence ihat, the
tangent from A being a straight line likt AEB in its limiting position when
E and B coincide, either rectangle is equal to the square on the tangent
(cf. Mr H. M. Taylor, p, 153), Euclid and the Greek geometers generally did
not allow themselves to infer the truth of a proposition in a limiting case
directly from the general case including it, but preferred a separate proof of
the limiting case (cf. Apollonius of Perga, p. 40, 139 — 140). This accounts for
the form of 11 r. 36.

\end{notes}

\end{proposition}

\begin{proposition}
\label{prop:III_37}

\begin{statement}
If a point be taken outside a iireie and from the point
there fall on the circle two straight lines, if one of them cut
the circle, and the other fall on it, and if further the rect-
angle contained by tlte whole of the straight line which cuts
Ike circle and the straight line intercepted on it outside
between the point and the convex circumference be equal to
the square on the straight line which falls on the circle, the
straight line which falls on it will touch the circle.
\end{statement}

\begin{proof}

For let a point D be taken outside the circle ABC;
from D let the two straight lines
DCA, DB fall on the circle ACB;
let DC A cut the circle and DB
fall on it ; and let the rectangle AD,
DC be equal to the square on DB.

I say that DB touches the circle
ABC.

For let DE be drawn touching
ABC ; let the centre of the circle ABC be taken, and let it
be F\ let FE, FB, FD be joined.

Thus the angle FED is right. [m. i8]

' Now, since DE touches the circle ABC, and DC A cuts it,
the rectangle AD, DC is equal to the square on DE, [m. 36]

But the rectangle AD, DC vt.s also equal to the square
onDB;
therefore the square on DE is equal to the square on DB ;

therefore DE is equal to DB.
  And FE is equal to FB ; 1

therefore the two sides DE, EF are equal to the two sides
DB, BF;
and FD is the common base of the triangles ;

therefore the angle DEF is equal to the angle DBF.

[l 8]
But the angle DEF is right ;

therefore the angle DBF is also right.
And FB produced is a diameter ;
and the straight line drawn at right angles to the diameter
of a circle, from its extremity, toucnes the circle ; [iir. 16, For.]
therefore DB touches the circle.
Similarly this can be proved to be the case even if the
centre be on C. 1

Therefore etc. • <- Q.E.D. '
\end{proof}

\begin{notes}

De Morgan observes that there is here the same defect as in i. 48, i.e. an
apparent avoidance of indirect demonstration by drawing the tangent DE on
the 0()OSite side of DF from UB. The case is similar to the appartnily
direct proof which Campanus gave. He drew the straight line from D
passing through the centre, and then (without drawing a second tangent)
proved by the aid of n. 6 that the square on DP is equal to the sum of the
squares on DB, BF\ whence (by t. 48) the angle DBF is a right angle.
But this proof uses I. 48, the very proposition to which De Morgan's original
remark relates.

The undisguised indirect proof is easy. If DB does not touch the circle,
it must cut it if produced, and it follows that the square on DB must be
equal to the rectangle contained by DB and a longer line ; which is absurd.

\end{notes}

\end{proposition}

\part{Book IV}

\chapter*{Definitions}

\begin{enumerate}

\item\label{def:IV_1} A rectilineal figure is said to be inscribed in
  a rectilineal figure when the respective angles of the inscribed
  figure lie on the respective sides of that in which it is inscribed.

\item\label{def:IV_2} Similarly a figure is said to be circumscribed
  about a figure when the respective sides of the circumscribed figure
  pass through the respective angles of that about which it is
  circumscribed.

\item\label{def:IV_3} A rectilineal figure is said to be inscribed in
  a circle when each angle of the inscribed figure lies on the
  circumference of the circle.

\item\label{def:IV_4}  A rectilineal figure is said to be circumscribed
about a circle, when each side of the circumscribed figure
touches the circumference of the circle.

\item\label{def:IV_5}  Similarly a circle is said to be inscribed in a figure
when the circumference of the circle touches each side of the
figure in which it is inscribed.

\item\label{def:IV_6}  A circle is said to be circumscribed about a figure
when the circumference of the circle passes through each
angle of the figure about which it is circumscribed.

\item\label{def:IV_7}  A straight line is said to be fitted into a circle when
its extremities are on the circumference of the circle.

\end{enumerate}

\section*{Definitions 1—7}

I append, as usual, the Greek text of the definitions.

1. xijxa (vvypafjLfiov tU (T/lii tddvypafiftov tyypdifntTdat AryCTCU, orAy
ttaimj TiSv Tol fyypafttKOu ojtaTOt ymyitar iiida-rft wXtvpa rou, tit S
iyyparTtLif airnfrai.

2. 2;/xa lituypofifiav tl kvkXov IffpovStui Xtycrac, rav cffairn ytaviti
ToS tffpatjiivav aiTTTfrat Ts tow kvicXod wtpitpiia. .
tXcitu tdS wtpiypaftivov iTmjrai riji tou kvkXou vtpitptiat.

3. KvitXot j4 (tt cr;(i7/ta o/UKOff lyypd()>i<rBai iyirat, Srav 1) Tov kvkXou

4. KujfXo M irpl vxfjta TTfpiypdtrSai Xrycrac, Jrav 17 tov icXov iripiipiia
iitwmj 'yw»'(a¥ tou, wtpi S vtpiypTat, aTmjrtii.

7. Eutftra (ft KuKXer Jra/ijufffO'dai Xrytrai, oral' tu ircaTa avrf JTrt ts
wtptfptuL  Toi) fcuxXou,

In the Rrst two definitions an English translation, if il is to be clear, must
depart slightly from the exact words used in the Greek, where ``each side'' of
one figure is said to pass through `` each angle `` of another, or `` each angle ``
(I.e. angular point) of one ties on `` each side `` of another (Udimi rktvpd,
iiiaiTD/ ytiiyia).

It is also necessary, in the five definitions 1, 1, 3, 5 and 6, to translate
the same Greek word aTrrToi in three different ways. It was observed on
tit. Def. 2 that the usual meaning of arrurSai in Euclid is to metf, in contra-
distinction to liTrrt<r9ai, which means to fauA. Exceptionally, as in Def. 5,
iimaSoi has the meaning of iaucA. But two new meanings of the word appear,
the first being to /ie on, zs in DefT. i atid 3, the second to pass through, as in
DefT. 3 and 6; ``each angle'' lies on (airriTat) a side or on a circle, and
`` each side,'' or a circle, passes through (aurrrai) an angle or `` each angle,''
The first meaning of lying an is exemplified in the phrase of Pappus af ctoi tJ
a)pM.w (cr<( hi oplyrrp (Mtiat, ``will lie on a Straight line given in position'';
the meaning of passing through seems to be much rarer (I have not seen it in
Archimedes or Pappus), but, as pointed out on itt. Def. 2, Aristotle uses the
compound l-miahai. in this sense.

Simson proposed to read iimjrat in the case (Def. 5) where an-njTot
means touches. He made the like suggestion as regards the Greek text of ttl.
II, 12, ij, 18, 19; in the first four of these cases there seems to be ms.
authority for the compound verb, and in the fifth He! berg adopts Slmson's
correction.

\part*{Book IV. Propositions}

\begin{proposition}
\label{propIV_1}

\begin{statement}
fn/o a £iven circle to fit a straight line equal to a given
straight line which is not greater than the diameter of the
circle.
\end{statement}

\begin{proof}

Let ABC be the given circle, and D the given straight
line not greater than the diameter
of the circle ;

thus it is required to fit into the
circle ABC a straight line equal
to the straight line D.

Let a diameter BC of the
circle ABC be drawn.

Then, if BC is equal to D,
that which was enjoined will have

been done ; for BC has been fitted into the circle ABC equal
to the straight line D.

But, if BC is greater than D,

let CE be made equal to D, and with centre C and distance
CE let the circle EAR be described ;

let CA be joined.

Then, since the point C is the centre of the circle EAF,

CA is equal to CE.

But CE is equal to /? ;

therefore D is also equal to CA.

Therefore into the given circle ABC there has been fitted
CA equal to the given straight line D.
\end{proof}

\begin{notes}

Of this problem as it stands there are of course an infinite number of
solutions; and, if a particular point be chosen as one extremity of the chord
to be ``fitted in,'' there are two solutions. More difficult cases of ``fitting
into `` a circle a chord of given length are arrived at by adding some further
condition, e.g. (i) that the chord is to be parallel to a given straight line, or
(2) that the chord, produced if necessary, shall pass through a given point.
The former problem is solved by Pappus (in. p. rja); instead of drawing the
chord as a tangent to a circle concentric with the given circle and having as
radius a straight line the square on which is equal to the difference between
the squares on the radius of the given circle and on half the given length, he
merely draws the diameter of the circle which is parallel to the given direction,
measures from the centre along it in each direction a length equal to half the
given length, and then draws, on one side of the diameter, perpendiculars to it
through the two points so determined.

The second problem of drawing a chord of given length, being less than
the diameter of the circle, and passing through a given point, is more
important as having been one of the problems discussed by Apollonius in his
work entitled vnxriit, now lost. Pappus states the problem thus (vii. p. 670):
``A circle being given in position, to fit into it a straight line given in
magnitude and verging (vtvoixrac) towards a given (point).'' To do this we
have only to place any chord HK in the given
circle (with centre O) equal to the given length,
take Z the middle point of it, with O as centre and
OL as radius describe a circle, and lastly through
the given point C draw a tangent to this circle
meeting the given circle in j4, B, AB is then one
of two chords which can be drawn satbfying the
given conditions, if C is outside the inner circle ; if
C Bn the inner circle there is one solution only ;
and, if C is within the inner circle, there is no
solution. Thus, if C is within the outer (given)

circle, besides the condition that the given length must not be greater than the
diameter of the circle, there is another necessary condition of the possibility
of a solution, viz. that the given length must not be Itss than double of the
straight line the square on which is equal to the difference between the squares
(i) on the radius of the given circle and (2) on the distance between its
centre and the given point.

\end{notes}

\end{proposition}

\begin{proposition}
\label{prop:IV_2}

\begin{statement}
In a given circle to inscribe a triangle equiangular with a
given triangle.
\end{statement}

\begin{proof}

Let ABC be the given circle, and DBF the given
triangle ;

thus it is required to inscribe in the circle ABC a triangle
equiangular with the triangle DEF.

Let GHx. drawn touching the circle ABC at A [m. i6,Por.];
on the straight line AH, and at the point A on it, let the
angle HAC be constructed eoual to the angle DEF,
and on the straight line AG, and at the point A on it, let
the angle GAB be constructed equal to the angle DFE ;

let BC be joined.

Then, since a straight line AH touches the circle ABC,
and from the point of contact at A the straight line C is
drawn across in the circle,

therefore the angle HA C is equal to the angle ABC in the
alternate segment of the circle. fin. 3*]

But the angle HA C is equal to the angle DEF ;
therefore the angle ABC is also equal to the angle DEF.

For the same reason

the angle ACB is also equal to the angle DFE ;
therefore the remaining angle BAC is also equal to the
remaining angle EDF. [i. 3*]

Therefore in the given circle there has been inscribed a
triangle equiangular with the given triangle. Q.E.F.
\end{proof}

\begin{notes}

Here again, since any point on the circle niay be taken as an angular
point of the triangle, there are an infinite number of solutions. Even when a
particular point has been chosen to form one angular point, the required
triangle may be constructed in six ways. For any one of the three angles
may be placed at the point ; and, whichever is placed there, the positions of
the two others relatively to it may be interchanged. The sides of the triangle
will, in all the different solutions, be of the same length respectively ; only
their relative positions will be different

This problem can of course be reduced (as it was by Borelli) to nt. 34,
namely the problem of cutting off from a given circle a segment containing an
angle equal to a given angle. It can also be solved by the alternative method
applicable to ni. 34 of drawing `` angles at the centre `` equal to double the
angles of the given triangle respectively ; and by this method we can easily
solve this problem, or . 34, with the further condition that one aide of the
required triangle, or the base of the required segment, respectively, shall be
parallel to a given straight line.

As a particular case, we can, by the method of this proposition, describe
an tguilaterai triangle in any circle after we have first constructed any
equilateral triangle by the aid of i. i. The possibility of this is assumed in
IV. t6. It is of course equivalent to dividing the circumference of a circle
into I Arte equttl parti. As De Morgan says, the idea of dividing a revolution
into equal parts should be kept prominent in considering Book iv. ; this
aspect of the construction of regular polygons is obvious enough, and the
reason why the division of the circle into fh-et equal parts is not given by
Euclid is that it happens to be as easy to divide the circle into three parts
which are in the ratio of the angles of any triangle as to divide it into three
equal parts.

\end{notes}

\end{proposition}

\begin{proposition}
\label{prop:IV_3}

\begin{statement}
About a given circle to circumscribe a triangle equiangular
with a given triangle.
\end{statement}

\begin{proof}

Let ABC be the given circle, and DEF the given
triangle ;

1 thus it is required to circumscribe about the circle ABC a
triangle equiangular with the triangle DEF.

Let EF be produced in both directions to the points
G, H,
let the centre K of the circle ABC be taken [in. r], and let

10 the straight line KB be drawn across at random ;
on the straight line KB and at the point K on it, let the
angle SKA be constructed equal to the angle DEG,
and the angle BKC equal to the angle DFH ; [i. J3]

and through the points A, B, C let LAM, MEN, NCL be

15 drawn touching the circle ABC. [in. 16, For]

Now, since LM, MN, NL touch the circle ABC at the
points A, B, C,

and KA, KB, KC have been joined from the centre K to
the points A, B, C,
ao therefore the angles at the points A, B, C are right. [iii. i8]

And, since the four angles of the quadrilateral AMBK
are equal to four right angles, inasnnuch as AMBK is in fact
divisible into two triangles,

and the angles KAM, KBM are right,

25 therefore the remaining angles A KB, A MB are equal to two

right angles.

But the angles DEG, DEF are also equal to two right

angles ; [1. 13]

therefore the angles A KB, A MB are equal to the angles
30 DEG, DEF,

of which the angle A KB is equal to the angle DEG ;

therefore the angle AMB which remains is equal to the
angle Z?/ which remains,

Similarly it can be proved that the angle LNB is also
3S equal to the angle DFE

therefore the remaining angle MLN is equal to the

angle EDF. . 33]

Therefore the triangle LMN is equiangular with the

triangle DEF; and it has been circumscribed about the

40 circle ABC.

Therefore about a given circle there has been circum-
scribed a triangle equiangular with the given triangle.

Q.E.F.
\end{proof}

\begin{annotations}

ii». at raodom, Uterslly `` ax it ma; chance,'' in trt/x''- The same etpression is used
in ][]. I and commonly.

11. Is In fact dlviaible, lol SuupttTai, literally `` is actually divided.''

\end{annotations}

\begin{notes}

The remarks as to the number of ways in which Prop, a can be solved
apply here also.

Euclid leaves us to satisfy ourselves that the three tangents )t>t7i meet and
form a triangle. This follows easily from the fact that each of the artgles
AB, BKC, CKA is less than two right angles. The first two are so by
construction, being the supplements of two angles of the given triangle re-
spectively, and, since ail three angles round K are together equal to four
right angles, it follows that the third, the angle AKC, is equal to the sum
of the two angles E, Foi the triangle, i.e. to the supplement of the angle D,
and is therefore less than two right angles.

Peletarius and Borelli gave an alternative solution, flrst inscribing a triangle
equiangular to the given triangle, by iv. 2, and then drawing tangents to the
circle parallel to the sides of the inscribed triangle respectively. This method
will of course give two solutions, since two tangents can be drawn parallel to
each of the sides of the inscribed triangle.

If the three pairs of parallel tangents be drawn and produced far enough,
they will form I'Af triangles, two of which are the triangles ctrcumKribed to
the circle in the manner required in the proposition. The other six triangles
are so related to the circle that the circle touches two of the sides in each
produced, i.e. the circle is an escribed circle to each of the six triangles.

\end{notes}

\end{proposition}

\begin{proposition}
\label{prop:IV_4}

\begin{statement}
In a given triangle to inscribe a circle.
\end{statement}

\begin{proof}

Let ABC be the given triangle ;
thus it is required to inscribe a circle in the triangle ABC.
Let the angles ABC, ACB
S be bisected by the straight Hnes
BD, CD [1. 9], and let these meet
one another at the point D ;
from D let DE, DF, DG be
drawn perpendicular tothestraight
10 lines AB, BC, CA.

Now, since the angle ABD
is equal to the angle CBD,

and the right angle BED is also equal to the right angle
BFD,
IS EBD, FBD are two triangles having two angles equal to two
angles and one side equal to one side, namely that subtending
one of the equal angles, which is BD common to the
triangles ;

therefore they will also have the remaining sides equal to
20 the remaining sides ; [i. 36]

therefore DE is equal to DF,
For the same reason

DG is also equal to DF.
Therefore the three straight lines DE, DF, DG are ec;ual
as to one another ;

therefore the circle described with centre D and distance

one of the straight lines DE, DF, DG will pass also

through the remaining points, and will touch the straight

lines AB, BC, CA, because the angles at the points E, F, G

30 are right.

For, if it cuts them, the straight line drawn at right angles
to the diameter of the circle from its extremity will be found
to fall within the circle : which was proved absurd ; [in. 16]
therefore the circle described with centre D and distance
35 one of the straight lines £>£, BF, DG will not cut the
straight lines AB, BC, CA ;

therefore it will touch them, and will be the circle inscribed
in the triangle ABC. [``v. Def. s]

Let it be inscribed, as FGE.
4° Therefore in the given triangle ABC the circle EFG has
been inscribed. . - .  .. ,. , ,
\end{proof}

\begin{annotations}
26, 34. and distance one of the (straight lines D)E, (D)F, (D)G. The wonte
«nd letters here shown in brarkets are put in to fill out tbe ralher carelcK hngiiige o( ihe
Greek- Hete and in several other places in Book IV. Euclid says lilemliy ``and w I ih distance
one of the (points) E,F,G'' (lai Juunttan M rwc E, Z, H) and the like. In one case (1 v. 13)
he actually has `` with distance one of the pttintt G, If, IC, L, M'' (tiuHmaTi irl rSr H, ©,
K, A, H irit;i(lup). Heiberg notes'' Craecam loculionem satis miram et negligenlem,'' but,
in view of its frequent occurrence in good M3S., does not venture to correct it.

\end{annotations}

\begin{notes}

Euclid does not think it necessary to prove that ££>, CD will meet ; this
is indeed obvious, for the angles DEC, DCB are together half of the angles
ABC, ACB, which themselves are tcether less than two right angles, and
therefore the two bisectors of the angles B, C must meet, by Post. 5.

It follows from the proof of this proposition that, if the bisectors of two
angles B, C q( b. triangle meet in D, the line joining D ia A also bisects the
third angle A, or the bisectors of the three angles. of a* triangle meet in
a point

It will be observed that Euclid uses the indirect form of proof when
showing that the circle touches the three sides of the triangle. Simson proves
it directly, and points out that Euclid does the same in 111. 17, 33 and 37,
whereas in iv. 8 and 13 as well as here he uses the indirect form. The
difference is unimportant, being one of fonn and not of substance; the
indirect proof refers back to in. 16, whereas the direct refers back to the
Porism to that proposition.

We may state this problem in the moie general form : Te describe a circle
touching three given straight lines which do net all meet in one point, and of
which not mere than two are parallel.

In the case (i) where two of the straight lines are parallel and the third
cuts them, two pairs of interior angles are formed, one on each side of the
third straight line. If we bisect each of the interior angles on one side, the
bisectors will meet in a point, and this point will be the centre of a circle
which can be drawn touching each of the three straight lines, its radius being
the perpendicular from the point on any one of the three. Since the alternate
angles are equal, two equal circles can be drawn in this manner satisfying the
given condition.

In the case (2) where the three straight lines form a triangle, suppose each
straight line produced indefinitely. Then each straight line will make two
pairs of interior angles with the other two, one pair forming two angles of the
triangle, and the other pair being their supplements. By bisecting each angle
of either pair we obtain, in the manner of the proposition, two circles
satisfying the conditions, one of them tieing the inscribed circle of the triangSe
and the other being a circle eseriied to it, i.e. touching one side and the other
two sides ptvduftd. Next, taking the pairs of interior angles formed by a
second side with the other two produced indefinitely, we get two circles
satisfying the conditions, one of which is the same inscribed circle that we had
before, while the other is a second escribed circle. Similarly with the third side.
Hence we have the inscribed circle, and three escribed circles (one opposite
each angle of the triangle), i.e. four circles in all, satisfying the conditions of
the probiem.

it ntay perhaps not be inappropriate to give at this point Heron's elegant
proof of the formula for the area of a triangle in terms of the sides, which we
usually write thus :

A=Js(f-a)(s-b)(s-c),

although it requires the theory of proportions and uses some ungeometrical
expressions, e.g. the product of two areas and the ``side `` of such a product,
where of course the areas are so many square units of length. The proof is
given in the Metrica, i. 8, and in the Dioptra, 30 (Heron, Vol. iii., Teubner,
190J, pp. *o— i4 and pp. j8o — 4, or Heron, ed. Hultsch, pp. 235 — 7).

Suppose the side-s of the triangle ABCo be given in length.

Inscribe the circle DEF, and let G be its centre.

If  !

Join AG, BG, CG, DG, EG, FG.

Then EC. EG =2. A BGC,

CA. FG=i.£:.ACG,

AB.DG=2.£.ABG.

Therefore, by addition,

p.EGi.CiABC,
where/ is the perimeter.

Produce CB to H, so that BH AD.
Then, since AD = AF, DB = BE, FC = CE,

CH= y.

Hence CH. EG=t, ABC.

as BOOK IV [iv. 4, 5

But CH .EG is the ``side'' of the product CH . EC, that b
JCH.EG;

therefore (iABC)=Cir.EG

Draw GL at right angles to CG, and BL at right angles to CB, meeting
at L. Join CL.

Then, since each of the angles CGL, CBL is right, CGBL is a quadri-
lateral in a circle.

Therefore the angles CGB, CLB are equal to two right angles.
Now the angles CGB, AGD are equal to two right angles, since AG, BG,
CG bisect the angles at G, and the angles CGB, AGD are equal to the
angles AGC, DGB, while the sum of all four is equal to four right angles.
Therefore the angles j4 CZJ, CZZf are equal.
So are the right angles ADG, CBL.
Therefore the triangles AGD, CLB are similar.
Hence BC: BL = AD-.DG

= BH: EG,
and, alternately, CB : BH = BL : EG

= BK: KE,
whence, tomponende, CH: HB = BE : EK.

It follows that CH-'' : CH . HB  BE . EC.CE . EK

' BE. EC: EG*
Therefore

(A ABC) = CH' . EG'= CH. HB . CE . EB

P(p-BC)(p-AB)(p-AO.

\end{notes}

\end{proposition}

\begin{proposition}
\label{prop:IV_5}

\begin{statement}
About a ven triangle to circumscribe a circle.
\end{statement}

\begin{proof}

Let ABC be the given triangle ;
thus it is required to circumscribe a circle about the given
triangle ABC.

Let the straight lines AB, AC h bisected at the points
D, E [i. 10], and from the points D, E let DF, EF be drawn
at right angles to AB, AC ;

they will then meet within the triangle ABC or on the
straight line BC, or outside BC.

First let them meet within at /  and let FB, FC, FA be
joined.

Then, since AD is equal to DB,   . f

and DF is common and at right angles,
therefore the base AFis equal to the base FB. [!  4]

Similarly we can prove that

CF is also equal to ; '

so that FB is also equal to FC ;
 therefore the three straight Hnes FA, FB, FC are equal
to one another,

Therefore the circle described with centre F and distance
one of the straight lines FA, FB, FC will pass also through
the remaining points, and the circle will have been circum-
scribed about the triangle ABC.

Let it be circumscribed, as ABC.

Next, let DF, EF meet on the straight line BC at F,
as is the case in the second figure ; and let AF be joined.

Then, similarly, we shall prove that the point F is the
centre of the circle circumscribed about the triangle ABC.

Again, let DF, EF meet outside the triangle ABC at F,
as is the case in the third figure, and let AF, EF, CF be
joined.

Then again, since AD is equal to DB,

and DF'is common and at right angles,

therefore the base AF is equal to the base BF. [i. 4]

Similarly we can prove that

C/ is also equal to v/;
so that BF is also equal to FC ;

therefore the circle described with centre F and distance one
of the straight lines FA, FB, FC will pass also through
the remaining points, and will have been circumscribed about
the triangle ABC.

Therefore about the given triangle a circle has been
circumscribed,

Q.E.F.

And it is manifest that, when the centre of the circle falls
within the triangle, the angle BAC, being in a segment
greater than the semicircle, is less than a right angle ;
when the centre falls on the straight line BC, the angle BAC,
being in a semicircle, is right ;

and when the centre of the circle falls outside the triangle,
the angle BAC, being in a segment less than the semicircle,
is greater than a right angle, [m. 31]
\end{proof}

\begin{notes}

Simson points out that Euclid does not prove that DF, EFmW meet, and
he inserts in the text the following argument to supply the omission.

`` DF, /i'' produced meet one another. For, if they do not meet, they
are parallel, wherefore AB, AC, which are at right angles to them, are
parallel [or, he should have added, in a straight line] : which is absurd.''

This assumes, of course, that straight lines which are at right angles to two
parallels are themselves parallel ; but this is an obvious deduction from J. 28.

On the assumption that DF, EF will meet Todhunter has this note : `` It
has been proposed to show this in the following way; join DE\ then the
angles EDFi.nA. DEFax together less than the angles ADFand AEF, that
is, they are together less than two right angles ; and therefore DF and Efi
will meet, by Axiom 1 2 \using{\rpost{5}}. This assumes that ADE and AED are
acute angles ; it may, however, be easily shown that DE is parallel to BC, so
that the triangle ADE is equiangular to the triangle ABC; and we must
therefore select the two sides AB and A C such that ABC and ACB may he
acute angles.''

This is, however, unsatisfactory, Euchd makes no such selection in tti. 9
and III. 10, where the same assumption is tacitly made; and it is unnecessary,
because it is easy to prove that the straight lines DF, EF meet in all cases,
by considering the different possibilities separately and drawing a separate
figure for each case.

Simson thinks that Euclid's demonstration had been spoiled by some
unskilful hand both because of the omission to prove that the perpendicular
bisectors meet, and because ``without any reason he divides the proposition
into three cases, whereas one and the same construction and demonstmtion
serves for them all, as Cam pan us has observed,'' However, up to the usual
words awiji (S(i jrmtjcrat there seems to be no doubt about the text. Heiberg
suggests that Euclid gave separately the case where /''falls on BC because, in
that case, only -Z'' needs to be drawn and not BF, CF a.s well.

The addition, though given in Simson and the text-books as a ``corollary,''
has no heading jropio-/t« in the best mss. ; it is an explanation like that which
is contained in the penultimate paragraph of iii. 25.

The Greek text has a further addition, which is rejected by Heiiwrg as not
genuine, ``So that, further, when the given angle happens to be less than a
right angle, DF, EF will fall within the triangle, when it is right, on BC, and,
when it is greater than a right angle, outside BC. (being) what it was required
to do.'' Simson had already observed that the text here is vitiated `` where
mention is made of a given angle, though there neither is, nor can he, any-
thing in the proposition relating to a given angle.''

\end{notes}

\end{proposition}

\begin{proposition}
\label{prop:IV_6}

\begin{statement}
In a given circle to inscribe a square. • '
\end{statement}

\begin{proof}

Let A BCD be the given circle ;
thus ii is required to inscribe a square in the circle A BCD.

Let two diameters AC, BD of the
circle ABCD be drawn at right angles
to one another, and let AB, BC, CD,
DA be joined.

Then, since BE is equal to ED, for
E is the centre,

and EA is common and at right angles,
therefore the base AB is equal to the
base AD. [[. 4]

For the same reason
each of the straight lines BC, CD is also equal to each of
the straight lines AB, AD ;

therefore the quadrilateral ABCD is equilateral,

I say next that it is also right-angled.

For, since the straight line BD is a diameter of the circle
ABCD, •

therefore BAD is a semicircle ;

therefore the angle BAD is right. [ni. 31]

For the same reason
each of the angles ABC, BCD, CD A is also right ;

therefore the quadrilateral ABCD is right-angled.

But it was also proved equilateral ;
therefore it Is a square ; [1. Def. 2»]

and it has been inscribed in the circle ABCD.

Therefore in the given circle the square ABCD has been

inscribed.   ``

Q.E.F.
\end{proof}

\begin{notes}

Euclid here proceeds to consider problems conespondtng to those in
Props. 2 — s with reference to figures of four or more sides, but with the
difference that, whereas he dealt with triangles of any fomn, he confines
himlf henceforth to regular figures. It happened to be as easy to divide a
circle into thrct parts which are in the ratio of the angles, or of the supplements
of the angles, of a triangle as into three cwn/ parts. But, when it is required to
inscribe in a circle a figure equiangular to a given quadrUattral, this can only be
done provided (hat the quadritateral has either pair of opposite angles equal
to two right angles. Moreover, in this case, tlie problem may be solved in the
same way as that of iv, z, i.e. by simply inscribing; a triangle equiangular to one
of the triangles into which the quadrilateral is divided by either diagonal, and
then drawing on the side corresponding to the diagonal as base another
triangle equiangular Co the other triangle contained in the quadrilateral. But
this is not the on/y solution ; there are an infinite
number of other solutions in which the inscribed
quadrilateral will, unlike that found by this particular
method, not he of the same /arm as the given quadri-
lateral For suppose A BCD to be the quadrilateral 11/ •' llry
inscribed in the circle by the method of iv. 2. Take '``
any point ff on AB, join AB'', and then make the
angle DAD (measured towards AC) equal to the
angle BAff. Join ffC, CU. Then AECD is also
tquiangular to the given quadrilateral, but not of the
same form. Hence the problem is indeterminate in the case of the general
quadrilateral. It is equally so if the given quadrilateral is a rectangle ; and it
is determinate only when the given quadrilateral is a square.

\end{notes}

\end{proposition}

\begin{proposition}
\label{prop:IV_7}

\begin{statement}
Aboui a given circle to circumscribe a square.
\end{statement}

\begin{proof}

Let ABCD be the given circle ;

thus it is required to circumscribe a square about the circle
ABCD.

Let two diameters AC, BD of the
circle ABCD be drawn at right angles
to one another, and through the points
A, B, C, D let FG, GH, HK, KF be
drawn touching the circle ABCD.

[III. i6,Por.]

Then, since FG touches the circk
ABCD,

and EA has been joined from the centre
E to the point of contact at A,

therefore the angles at A are right. [11 1. 18]

For the same reason

the angles at the points B, C, D are also right.

Now, since the angle AEB is right,
and the angle EBG is also right,

therefore GH is parallel to AC. [i. 18

IV. 7] PROPOSITIONS 6, 7 93

For the same reason

AC is also parallel to FK,
so that G// is also parallel to FK. . . ['• 3°]

Similarly we can prove that

each of the straight lines GF, HK is parallel to BED,

Therefore GK, GC, AK, FB, BK are parallelograms ;
therefore GF is equal to HK, and GH to FK. [i. 34]

And, since AC is equal to BD,
and AC is also equal to each of the straight lines G//, FA',

while BO is equal to each of the straight lines GF, HK,

[' 34]
therefore the quadrilateral FGHK is equilateral.

I say next that it is also right-angled.

For, since GBEA is a parallelogram,
and the angle AEB is right,
therefore the angle AGB is also right.  [i. 34]

Similarly we can prove that

the angles at H, K, F are also right.

Therefore FGHK is right-angled.

But it was also proved equilateral ;

therefore it is a square ;
and it has been circumscribed about the circle ABCD.

Therefore about the given circle a square has been
circumscribed.

.  '-'' Q.E.F.
\end{proof}

\begin{notes}

It is just as easy to describe about a given circle a polygon equiangular to
any given polygon as it is to describe a square about a given circle. We have
only to use the method of iv. 3, i.e. to take any radius of the circle, to
measure round the centre successive angles in one and the same direction
equal to the supplements of the successive angles of the given polygon and,
lastly, to draw tangents to the circle at the extremities of the several radii so
detemiined ; but again the polygon would in general not be of the same form
as the given one ; it would only be so if the given polygon happened to be
such that a circle couid be inscribed in it. To take the case of a quadrilateral
only : it is easy to prove that, if a quadrilateral be described about a circle,
the sum of one pair of opjxtsite sides must be equal to the sum of the other
pair. It may be proved, conversely, tliat, if a quadrilateral has the sums of the
pairs of opposite sides equal, a circle can be inscribed in it. If then a given
quadrilateial has the sums of the pairs of opposite sides equal, a quadrilateral
can be described about any given circle not only equiangular with it but
having the iaxaform or, in the words of Book vi., similar to it.

\end{notes}

\end{proposition}

\begin{proposition}
\label{prop:IV_8}

\begin{statement}
In a given square to inscribe a circle.
\end{statement}

\begin{proof}

Let ABCD be the given square ;
thus it is required to inscribe a circle in the given square
ABCD.

Let the straight lines AD AB be
bisected at the points E, F respectively

[»• to].
through E let EH be drawn parallel
to either AB or CD, and through
F let FK be drawn parallel to either
AD or BC; [1.31]

therefore each of the figures AK, KB,
AH, HD, AG, GC, BG, GD is a parallelogram,
and their opposite sides are evidently equal.

Now, since AD is equal to AB,
and  is half of AD, and AF half of AB,

therefore AE is equal to AF,
so that the opposite sides are also equal ; • - -
therefore FG is equal to GE.

Similarly we can prove that each of the straight lines GH,
GK is equal to each of the straight lines FG, GE ;

therefore the four straight lines GE, GF, GH, GK are
equal to one another.

Therefore the circle described with centre G and distance
one of the straight lines GE, GF, . GH, GK will pass also
through the remaining points.

And it will touch the straight lines AB, BC, CD, DA,
because the angles at E, F, H, K are right.

For, if the circle cuts AB, BC, CD, DA, the straight
line drawn at right angles to the diameter of the circle from
its extremity will fall within the circle : which was proved
absurd ; [iii, ifi)

therefore the circle described with centre G and distance
one of the straight lines GE, GF, GH, GK will not cut
the straight lines AB, BC, CD, DA.

Therefore it will touch them, and will have been inscribed
in the square ABCD.

Therefore in the given square a circle has been inscribed.
\end{proof}

\begin{notes}

As was remarked in the Iftst note, a circle can be inscribed in any
uadrilatiral -vihKh has the sum of one pair ofoppcite sides equal to the sum
of the other pair. In particular, il follows that a circle can be inscribed in a
tfuare or a rhsmbus but not in a rectangle or a rhomboid.

\end{notes}

\end{proposition}

\begin{proposition}
\label{prop:IV_9}

\begin{statement}
About a given square to circumscribe a circle.
\end{statement}

\begin{proof}

Let A BCD be the given square ;

thus it is required to ctrcu inscribe a circle about the square

A BCD.

For let AC, BD be joined, and let them
cut one another at E,

Then, since DA is equal to AB
and AC is common,

therefore the two sides DA, AC are equal
to the two sides BA, AC;
and the base DC is equal to the base BC ;
therefore the angle DAC is equal to
the angle BAC. [i. 8]

Therefore the angle DAB is bisected by AC.

Similarly we can prove that each of the angles ABC,
BCD, CDA is bisected by the straight lines AC, DB.

Now, since the angle DAB is equal to the angle ABC,
and the angle BAB is half the angle DAB,
and the angle BBA half the angle ABC, 'V

therefore the angle BAB is also equal to the angle £BA ;
so that the side £A is also equal to BB. [i. 6]

Similarly we can prove that each of the straight lines
BA, BB is equal to each of the straight lines EC, ED.

Therefore the four straight lines EA, BB, BC, BD are
equal to one another.

Therefore the circle described with centre E and distance
one of the straight lines EA, EB, EC, ED will pass also
through the remaining points ;
and it will have been circumscribed about the square ABCD.

Let it be circumscribed, as ABCD.

Therefore about the given square a circle has been
circumscribed.
\end{proof}

\end{proposition}

\begin{proposition}
\label{prop:IV_10}

\begin{statement}
Ta construct an isosceles triangle Itaving each of the angles
at the base double of the remaining one.
\end{statement}

\begin{proof}

Let any straight line AB be set out, and let it be cut at
the point C so that the rectangle
contained by AB, BC is equal to
the square on CA\ [•' ``]

with centre A and distance AB let
the circle BDE be described,

and let there be fitted in the circle
BDE the straight line BD equal to
the straight line AC which is not
greater than the diameter of the
circle BDE, [iv. ,]

Let AD, DC be joined, and let
the circle ACD be circumscribed about the triangle A CD.

['V. S)
Then, since the rectangle AB, BC is equal to the square
on AC,

and AC is equal to BD,

therefore the rectangle AB, BC is equal to the square on BD.

And, since a point B has been taken outside the circle
ACD,

and from B the two straight lines BA, BD have fallen on
the circle ACD, and one of them cuts it, while the other falls
on it,

and the rectangle AB, BC is equal to the square on BD,

therefore BD touches the circle ACD. [in. 37]

Since, then, BD touches it, and DC is drawn across
from the point of contact at D,

therefore the angle BDC is equal to the angle DAC in the
alternate segment of the circle. [in. 3*]

Since, then, the angle BDC is equal to the angle DAC,
let the angle CDA be added to each ;

therefore the whole angle BDA is equal to the two angles
CDA, DAC.

But the exterior angle BCD is equal to the angles CD A,
DAC; [1.3a]

therefore the angle BDA is also equal to the angle BCD.

But the angle BDA is equal to the angle CBD. since the
side AD is also equal to AB ; [i. s]

so that the angle DBA is also equal to the angle BCD.

Therefore the three angles BDA, DBA, BCD are equal
to one another.

And, since the angle DBC is equal to the angle BCD,

the side BD is also equal to the side DC. [i. 6]

But BD is by hypothesis equal to CA ;
therefore CA is also equal to CD,

so that the angle CD A is also equal to the angle DA C ;

[i-S]
therefore the angles CD A, DAC are double of the angle DA C.

But the angle BCD is equal to the angles CD A, DAC;

therefore the angle BCD is also double of the angle CAD.

But the angle BCD is equal to each of the angles BDA,
DBA ,

therefore each of the angles BDA, DBA is also double of
the angle DAB.

Therefore the isosceles triangle ABD has been constructed
having each of the angles at the base DB double of the
remaining one.

Q.E.F.
\end{proof}

\begin{notes}

There is every reason to conclude that the connexion of the triangle
constructed in this proposition with the regular pentagon, and the construction
of the triangle itself, were the discovery of the Pythagoreans. In the first
place the Scholium iv. No. a (Heiberg, Vol. v. p. 273) says `` this Book is the
discovery of the Pythagoreans.'' Secondly, the summary in Proclus (p. fi, to)
says that Pythagoras discovered ``the construction of the cosmic figures,'' by
which rnust be understood the five regular solids. Thirdly, lamblichus (yn.
Pyth, c. 18, s. 38) quotes a story of Hippasus, `` that he was one of the Pytha-
goreans but, owing to his being the first to publish and write down (the con-
struction of) the sphere arising from the twelve pentagons (rrpi in tuv hitxa
ir<kraywfijf), perished by shipwreck for his inipiety, having got credit for the
discovery all the same, whereas everything belonged to HIM (intivcmrov dkSpot),
for it is thus that they refer to Pythagoras, and they do not call him by his
name.'' Cantor has (i,, pp. 176 sqq.) collected notices which help us to form
an idea how the discovery of the Euclidean construction for a regular pentagon
may have been arrived at by the Pythagoreans.

Plato puts into th''. mouth of Timaeus a description of the formation from
right-angled triangles of the figures which are the faces o( the first four regular
solids. The face of the cube is the S(]uare which is formed from isosceles
right-angled triangles by placing four of these triangles contiguously so that
the four right angles are in contact at the centre. The
equilateral triangle, however, which is the form of the faces of
the tetrahedron, the octahedron and the icosahedron, cannot
be constructed from isosceles right-angled triangles, but is
constructed from a particular scalene right-angled triangle
which Timaeus (54 a, h) regards as the most b<:autiful of all
scalene right-angled triangles, namely that in which the square on one of the
sides about the right angle is three times the square on the other. This is, of
course, the triangle forming half of an equilateral triangle bisected by the
perpendicular from one angular point on the opposite side. The Platonic
Timaeus does not construct his equilateral triangle from two such triangles
but from six, by placing th« latter contiguously round a
point so that the hypotenuses and the smaller of the sides
about the right angles respectively adjoin, and all of them
meet at the common centre, as shown in the figure
(T/macHs, 54 d, k.). The probability that this exposition
was Pythagorean is confirmed by the independent testimony
of Proclus (pp. 304—5), who attributes to the Pythagoreans
the theorem that six equilateral triangles, or three hexagons, or four squar
placed contiguously with one angular point of each at a common point, will
just fill up the four right angles round that point, and that no other regular
polygons in any numbers have this property.

How then would it be proposed to split up into triangles, or to make up
out of triangles, the face of the remaining solid, the dodecahedron ? It would
easily be seen that the pentagon could not be constructed by means of tlie
two right-angled triangles which were used for constructing the square and the
equilateral triangle respectively. But attempts would naturally be made to
split up the pentagon into elementary triangles, and traces of such attempts
are actually forthcoming. Plutarch has in two passages spoken of the division
of the faces of the dodecahedron into triangles, remarking in one place
(Quaes/. Platon. v. i) that each of the twelve faces is made up of 30 elemen-

tary scalene triangles, so that, taken together, they give 360 such triangles,
and in another (Dt deftctu oramlorum, c. 33) that the elementary triangle of
the dodecahedron must be different from that of the tetrahedron, octahedron
and icosahedron. Another writer of the and cent,, Alcinous, has, in his
introduction to the study of Plato (De doctrina Platonis, c. 11), spoken
similarly of the 360 elements which are produced when every one of the
pentagons is divided into 5 isosceles triangles, and each of the latter into
6 scalene triangles. Now, if we proceed to draw lines in a pentagon sejmrating
it into this number of small triangles as shown in the above figure, the figure
which stands out most prominently in the mass of lines is the ``star-pentagon,''
as drawn separately, which then (if the consecutive comers be joined) suggests
the drawing, as part of a pentagon, of a triangle of a definite character. Now
we are expressly told by Lucian and the scholiast to the Clouds of Aristophanes
(see Bretschneider, pp. 85 — 86) that the triple interwoven triangle, the penta-
gram (to TpurXflEf Tpiycuvov, TO Zi oAAijAaic, to TroTaypttfiftov), was used by the
Pythagoreans as a symbol of recognition between the members of the same
school (avohjjf ipos Tous ofioSo'fou! ixf™'''°)i *''d '``'** called by them Health.
There seems lo be therefore no room for doubt that the construction of a
pentagon by means of an isosceles triangle having each of its base angles
double of the vertical angle was due to the Pythagoreans.

The construction of this triangle depends upon ii. 1 1, or the problem of
dividing a straight line so that the rectangle contained by the whole and one
of the parts is equal to the square on the other part. This problem of course
appears again in Eucl.\ vi. 30 as the problem of cutting a given straight line in
exirtme and mean ratio, i.e. the problem of the goldtn section, which is nc-
doubt `` the section `` referred to in the passage of the summary given by
Proclus (p. 67, 6) which says that Eudoxus ``greatly added to the number
of the theorems which Plato originated regarding the section.'' This idea that
Plato began the study of the `` golden section `` as a subject in itself is not in
the least inconsistent with the supposition that the problem of Eucl ii. 1 1 was
solved by the Pythagoreans. The very fact that Euclid places it among other
propositions which are clearly Pythagorean in origin is significan|, as is also
the fact that its solution is effected by `` applying to a straight line a rectangle
equal to a given square and exceeding by a square,'' while Proclus says plainly
(p. 419, 15) that, according to Eudemus, ``the application of areas, their
txcuding and their falling short, are ancient and discoveries of the Muse of
the Pythagoreans.''

We may suppose the construction of iv. to to have been arrived at by
analysis somewhat as follows (Todhunter's Euclid, p. 325).

Suppose the problem solved, i.e. let ABD be an isosceles triangle having
each of its base angles double of the vertical angle.

Bisect the angle ADB by the straight line DC meeting AB in C. [1, 9]

Therefore the angle BDC is equal to the angle BAD ; and the angle
CDA is also equal to the angle BAD,

so that DC is equal to CA,

Again, since, in the triangles BCD, BDA,

the angle BDC is equal to the angle BAD,
and the angle B is common,

the third angle BCD is equal to the third angle BDA, and therefore to
the angle DEC.

Therefore Z'C is equal to DB.

Now, if a circle be described about the triangle ACD [iv. 5), since the
angle BDC is equal to the angle in the segment CAD,

BD must touch the circle [by the converse of [ii. 32 easily proved from it
by riduttio ad aiiurdum

Hence [11 [. 36] the square oa BD and therefore the square on CD, or
AC, is equal to the rectangle AS, BC.

Thus the problem is reduced to that of cutting AB at C so that the
rectangle AB, BC is equal to the square on AC, [ii, 11]

t99 BOOK IV [iv. lo, It

When this is done, we have only to draw a circle with centre A and radius
A£ and place in it a chord SU equal in length to AC. [iv. i]

Since each of the angles ABD, ADB is double of the angle BAD, the
latter is equal to one-fifth of the sum of all three, i.e. is one-fifth of two right
angles, or two-fifths of a right angle, and each of the base angles is four-fifths
of a right angle.

If we bisect the angle BAD, we obtain an angle equal to one-fifth of a
right angle, so that the proposition enables us to divide a right angle into five
equal parts.

It will be observed that BD is the side of a regular dtcagen inscribed in
the larger circle.

Proclus, as retnarked above (Vol. i, p. 130), gives iv. 10 as an instance in
which two of the six formal divisions of a proposition, the idting-out and the
``definition'' are left out, and explains that they are unnecessary because
there is no datum in the enunciation. This is however no more than formally
true, because Euclid does begin bis proposition by tetling out `` any straight
line AB'' and he constructs an isosceles triangle having AB for one of its
equal sides, i.e. he does practically imply a datum in the enunciation, and a
corresponding setting-out and ``definition '' in the proposition itself.

\end{notes}

\end{proposition}

\begin{proposition}
\label{prop:IV_11}

\begin{statement}
In a given circle to inscribe an equilateral and uiangular
pentagon.
\end{statement}

\begin{proof}

Let ABCDE be the given circle ;
thus it is required to inscribe in the circle ABCDE an equi-
lateral and equiangular pentagon.

Let the isosceles triangle FGH
be set out having each of the angles
at G, H double of the angle at F

[iv. 10]

let there be inscribed in the circle
ABCDE the triangle ACD equi-
angular with the triangle FGH, so

that the angle CAD is equal to the angle at F and the angles
at G, respectively equal to the angles ACD, CD A ; [iv. a]
therefore each of the angles ACD, CD A is also double of the
angle CAD.

Now let the angles ACD, CD A be bisected respectively
by the straight lines CE, DB [1. 9], and let AB, BC, DE, EA
be joined.

Then, since each of the angles ACD, CD A is double of
the angle CAD,
and they have been bisected by the straight lines CE, DB,

IV. ii] PROPOSmONS 10, II i«i

therefore the five angles DAC, ACE, BCD, CDS, BDA
are equal to one another.

But equal angles stand on equal circumferences ; [iii. a6]

therefore the five circumferences AB, BC, CD, DE, EA are
equal to one another.

But equal circumferences are subtended by equal straight
lines ; [m. 19]

therefore the five straight lines AB, BC, CD, DE, EA are
equal to one another ;

therefore the pentagon ABCDE is equilateral.

I say next that it is also equiangular.
For, since the circumference AB is equal to the circum-
ference DE, let BCD be added to each ;

therefore the whole circumference ABCD is equal to the
whole circumference EDCB.

And the angle A ED stands on the circumference ABCD,
and the angle BAE on the circumference EDCB ;

therefore the angle BAE is also equal to the angle AED.

[iiL a;]
For the same reason *

each of the angles ABC, BCD, CDE is also equal to each
of the angles BAE, AED ;

therefore the pentagon ABCDE is equiangular.

But it was also proved equilateral ;

therefore in the given circle an equilateral and equi-
angular pentagon has been inscribed.

Q.E.F.
\end{proof}

\begin{notes}

De Morgan remarks that `` the method of iv. 11 is not  a natuial as
malting a direct use of the angle obtained in the last.'' On the other hand,
if we look at the figure and notice that it shows the whole of the pmtagrafo-
star except one line (that connecting B and E), I think we shall conclude
that the method is nearer to that used by the Pythagoreans, and therefore of
much more historical interest.

Another method would of course be to use iv. 10 to describe a decagtnt in
the circle, and then to join any vertex to the next alternate one, the tatter to
the next alternate one, and so on.

Mr H. M. Taylor gives ``a complete geometrical construction for in-
scribing a regular decagon or pentagon in a given circle,'' as follows.

`` Find O the centre.

Draw two diameters AOC, BOD at right g

angles to one another.

Bisect OD in £.

Draw A£ and cut off E£ equal to 0£.

Place round the circle ten chords equal
to AF.

These chords will be the sides of a regular
decagon. Draw the chords joining Hve alternate
vertices of the decagon ; they will be the sides
of a regular pentagon.''

The construction is of course only a com-
bination of those in ]i. ii and iv. i ; and the
proof would have to follow that in iv. lo.

\end{notes}

\end{proposition}

\begin{proposition}
\label{prop:IV_12}

\begin{statement}
About a given circle to circumscribe an equilateral and
equiangular pentagon.
\end{statement}

\begin{proof}

Let y4 .5 CZ?.£' be the given circle ; >-

thus it is required to circumscribe an equilateral and equi-
angular pentagon about the circle
ABCDE.

Let A, B, C, D, E be conceived to
be the angular points of the inscribed
pentagon, so that the circumferences
AB, EC, CD, DE, EA are equal ;

through A, B, C, D. E let G//, HK,

KL, LM, MG be drawn touching the

circle ; [in. 16, Por.]

let the centre F of the circle ABCDE be taken [m. 1], and

let FB, FK, FC, FL, FD be joined.

Then,since the straight line KL touches the circle ABCDE
at C,

and FC has been joined from the centre F to the point of
contact at C,

therefore FC is perpendicular to KL ; , [in. 18]

therefore each of the angles at C is right. •

For the same reason

the angles at the points B, D are also right

And, since the angle FCK is right,
therefore the square on FK is equal to the squares on FC, CK.

For the same reason [1. 47]

the square on FK is also equal to the squares on FB, BK ;

so that the squares on FC, CK are equal to the squares
on FB, BK,

of which the square on FC is equal to the square on FB ;
therefore the square on CK which remains is equal to the
square on BK.

Therefore BK is equal to CK.

And, since 5 is equal to TC • •

and FK common,

the two sides BF, FK are equal to the two sides CF, FK
and the base BK equal to the base CK ;

therefore the angle BFK is equal to the angle KFC, [i. 8]

and the angle BKF to the angle FKC.
Therefore the angle BFC is double of the angle KFC,

and the angle BKC of the angle FKC.

For the same reason

the angle CFD is also double of the angle CFL,

and the angle DLC of the angle FL C.

Now, since the circumference BC is equal to CD,
the angle BFC is also equal to the angle CFD. [in. 17]

And the angle BFC is double of the angle KFC, and the
angle DFC of the angle LFC ;

therefore the angle KFC is also equal to the angle LFC.

But the angle FCK is also equal to the angle FCL ;
therefore FKC, FLC are two triangles having two angles
equal to two angles and one side equal to one side, namely
FC which is common to them ;

therefore they will also have the remaining sides equal to the
remaining sides, and the remaining angle to the remaining
angle ; [i- *6)

therefore the straight line KC is equal to CL,
and the angle FKC to the angle FLC,

And, since KC is equal tO CZ, ,

therefore KL is double of KC , .

For the same reason it can be proved that

I/J is also double of S/C. 1

And A' is equal to C;

therefore I/K is also equal to KL.

Similarly each of the straight lines //G, GM, ML can
also be proved equal to each of the straight lines //K, KL ;

therefore the pentagon GHKLM is equilateral.
. - I say next that it is also equiangular.

For, since the angle FKC is equal to the angle FLC,
and the angle HKL was proved double of the angle FKC,

and the angle KLM double of the angle FLC,
therefore the angle HKL is also equal to the angle KLM.

Similarly each of the angles KHG, HGM, GML can also
be proved equal to each of the angles HKL, KLM;
therefore the five angles GHK, HKL, KLM, LMG, MGH
are equal to one another.

Therefore the pentagon GHKLM is equiangular.

And it was also proved equilateral ; and it has been
circumscribed about the circle ABCDE.

Q.E.F.
\end{proof}

\begin{notes}

De Morgan remarks that iv. 12, 13, 14 supply the pkce of the following :
Having given a regular polygon of any number of sides inscribed in a circle, lo
describe the same about ike circle; and, having given the polygon, lo inscribe and
circumscribe a circle. For the method can be applied generally, as indeed
Euclid practically says in the Porism to iv. 15 about the regular hexagon and
in the remark appended to iv. 16 about the regular fifteen-angled figure.

The conclusion of this proposition, `` therefore about the given circle an
equilateral and equiangular pentagon bas been circumscnbed,'' is omitted in
the Mss.

\end{notes}

\end{proposition}

\begin{proposition}
\label{prop:IV_13}

\begin{statement}
In a given pentagon, wkick is equilateral and equiangular,
to inscribe a circle.
\end{statement}

\begin{proof}

Let ABCDE be the given equilateral and equiangular
pentagon ;

thus it is required to inscribe a circle in the pentagon
ABCDE.

For let the angles BCD, CDE be bisected by the
straight lines CF, Z?J respectively ; and from the point F, at
which the straight lines CF, DF meet one another, let the
straight lines FB, FA, FE be joined. 1 • •

Then, since BC is equal to CD,
and CV common,

the two sides BC, CF are equal to the
two sides DC, CF

and the angle BCF is equal to the
angle DCF;

therefore the base BF is equal
to the base DF,

and the triangle BCF is equal to the
triangle DCF,

and the remaining angles will be equal to the remaining angles,
namely those which tne equal sides subtend. [1. 4]

Therefore the angle CBF is equal to the angle CDF.

And, since the angle CDE is double of the angle CDF,
and the angle CDE is equal to the angle ABC,
while the angle CDF is equal to the angle CBF
therefore the angle CBA is also double of the angle CBF)
therefore the angle ABF is equal to the angle FBC ;
therefore the angle ABC has been bisected by the straight
line BF.

Similarly it can be proved that
the angles BAE, AED have also been bisected by the straight
lines FA, FE respectively.

Now let EG, FH, FK, FL, FMh drawn from the point
F perpendicular to the straight lines AB, BC, CD, DE, EA.

Then, since the angle HCF is equal to the angle KCF,
and the right angle FHC is also equal to the angle FKC,
FHC, FKC are two triangles having two angles equal to two
angles and one side equal to one side, namely EC which is
common to them and subtends one of the equal angles ;
therefore they will also have the remaining sides equal to the
remaining sides ; [i. i6]

therefore the perpendicular FH is equal to the perpendicular
FK.

Similarly it can be proved that
each of the straight lines FL, FM, EG is also equal to each
of the straight lines EH, FK ;
therefore the five straight lines FG, FH, FK, FL, FM are
equal to one another.

Therefore the circle described with centre F and distance
one of the straight lines FG, FH, FK, FL, FM will pass
also through the remaining points ;

and it will touch the straight lines AB, BC, CD, DE, EA,
because the angles at the points G, N, K, L, M aire right.

For, if it does not touch them, but cuts them,

it will result that the straight line drawn at right angles to
the diameter of the circle from its extremity falls within the
circle : which was proved absurd. [ui. 16]

Therefore the circle described with centre F and distance
one of the straight lines FG, FN, FK, FL, FM will not
cut the straight lines AB, BC, CD, DE, EA ;

therefore it will touch them.

Let it be described, as GHKLM.

Therefore in the given pentagon, which is equilateral and
equiangular, a circle has been inscribed,

Q.E.F.
\end{proof}

\end{proposition}

\begin{proposition}
\label{prop:IV_14}

\begin{statement}
About a given pentagon, which ts equilateral and equi-
angular, to circumscribe a circle.
\end{statement}

\begin{proof}

Let ABCDE be the given pentagon, which is equilateral
and equiangular ;

thus it is required to circumscribe a circle
about the pentagon ABCDE.

Let the angles BCD, CDE be bisected
by the straight lines CF, OF respectively,
and from the point F, at which the straight
lines meet, let the straight lines FB, FA,
FE be joined to the points B, A, E.

Then in manner similar to the pre-
ceding it can be proved that the angles
CBA, BAE, AED have also been bisected by the straight
lines FB, FA, FE respectively. •

Now, since the angle BCD is equal to the angle CDE,
and the angle FCD is half of the angle BCD,
and the angle CDF half of the angle CDE,
therefore the angle FCD is also equal to the angle CDF,  !

so that the side FC is also equal to the side FD. [i- 6]

Similarly it can be proved that
each of the straight lines FB, FA, FE is also equal to each
of the straight lines FC, FD ;

therefore the five straight lines FA, FB, FC, FD, FE are
equal to one another.

Therefore the circle described with centre F and distance
one of the straight lines FA, FB, FC, FD, FE will pass
also through the remaining points, and will have been
circumscribed. , ,

Let it be circumscribed, and let it be ABCDE.

Therefore about the given pentagon, which is equilateral
and equiangular, a circle has been circumscribed.

Q. E, F.
\end{proof}

\end{proposition}

\begin{proposition}
\label{prop:IV_15}

\begin{statement}
In a given, circle to inscribe an equilateral and equiangular
hexagon.
\end{statement}

\begin{proof}

Let ABCDEF be the given circle ; ,...::;

thus it is required to inscribe an equilateral and equiangular
hexagon in the circle ABCDEF.

Let the diameter AD of the circle
ABCDEF be drawn ;
let the centre G of the circle be taken, and
with centre D and distance DG let the
circle EGCH be described ;
let EG, CG be joined and carried through
to the points B, F,

and let AB, EC, CD, DE, EF, FA be
joined.

I say that the hexagon ABCDEF is
equilateral and equiangular.

For, since the point G is the centre of the circle ABCDEF,
GE is equal to GD.    v. •• v

•»*S BOOK IV 1 `` [iv, IS

Again, since the point D is the centre of the circle GCH,

DE is equal to DG,   ':-'

„ But GE was proved equal to GD ;

therefore GE is also equal to ED ;

therefore the triangle EGD is equilateral ;

and therefore its three angles EGD, GDE, DEG are equal
to one another, inasmuch as, in isosceles triangles, the angles
at the base are equal to one another. [i. s]

And the three angles of the triangle are equal to two
right angles ; [i. 31

therefore the angle EGD is one-third of two right angles.

Similarly, the angle DGC can also be proved to be one-
third of two right angles.

And, since the straight line CG standing on EB makes
the adjacent angles EGC, CGB equal to two right angles,

therefore the remaining angle CGB is also one-third of two
right angles.

Therefore the angles EGD, DGC, CGB are equal to one
another ;

so that the angles vertical to them, the angles SGA, AGF,
FGE are equal, [i. 15]

Therefore the six angles EGD, DGC, CGB, BGA, AGF,
FGE are equal to one another.

But equal angles stand on equal circumferences ; (m- a6]
therefore the six circumferences AB, BC, CD, DE, EF, FA
are equal to one another.

And equal circumferences are subtended by equal straight
lines ; [m. 29]

therefore the six straight lines are equal to one another;

therefore the hexagon ABCDEF is equilateral,

I say next that it is also equiangular.

For, since the circumference FA is equal to the circum-
ference ED,

let the circumference ABCD be added to each ;

therefore the whole FA BCD is equal to the whole
EDCBA ;

IV, is] proposition is 109

and the angle FED stands on the circumference FA BCD,

and the angle AFE on the circumference EDCBA ;

therefore the angle AFE is equal to the angle DEF,

[m. 27]

Similarly it can be proved that the remaining angles of
the hexagon ABCDEF are also severally equal to each of
the angles AFE, FED ;

therefore the hexagon ABCDEF is equiangular.

But it was also proved equilateral ;

and it has been inscribed in the circle ABCDEF.

Therefore in the given circle an equilateral and equiangular
hexagon has been inscribed.

Q.E.F.

PoRiSH. From this it is manifest that the side of the
hexagon is equal to the radius of the circle.

And, in like manner as in the case of the pentagon, if
through the points of division on the circle we draw
tangents to the circle, there will be circumscribed about the
circle an equilateral and equiangular hexagon in conformity
with what was explained in the case of the pentagon.

And further by means similar to those explained in the
case of the pentagon we can both inscribe a circle in a given
hexagon and circumscribe one about it, ,

Q.E.F.
\end{proof}

\begin{notes}

Heiberg, I think with good reason, considers the Porism to this proposition
to be referred to in the instance which Proclus (p. 304, a) gives of a porism
following a problem. As the text of Proclus stands, `` the (poristn) found
in the second Book (td ii ir ry Smrifnf ijSXi'u xtiiititor) is a porism to a
problem `` ; but this is not true of the only porism that we find in the second
Book, namely the jwrism to it. 4. Hence Heibeig thinks that for rif
StvfifMf fiifiXiif should be read 1 £' fiiKuf, i.e. the fourth Book. Moreover
Proclus speaks of tAe porism in the particular Book, from which we gather
that there was only arte porism in BooJt iv. as he knew it, and therefore that
he did not regard as a porism the addition to iv. 5. Cf. note on that
proposition.

It appears that Theon substituted for the first words of the Porism to
IV. 15 ``And in like manner as in the case of the pentagon'' (d/iouiit Si
ToTs M. rol leivTaymrm) the simple word `` and `` or `` also `` (Wj, apparently
thinking that the words had the same meaning as the similar words lower
down. This is however not the case, the meaning being that `` if, as in the
case of the pentagon, we draw tangents, we can prove, also as was done in
the case of the pentagon, that the figure so formed is a circumscribed rular
hexagon.''

\end{notes}

\end{proposition}

\begin{proposition}
\label{prop:IV_16}

\begin{statement}
In a given circle to inscribe a fifteen-angled figure which
shall be both equilateral and equiangular.
\end{statement}

\begin{proof}

Let ABCD be the given circle ;
thus it is required to inscribe in the circle ABCD a fifteen-
angled figure which shall be
both equilateral and equi-
angular.

In the circle ABCD let
there be inscribed a side AC
of the equilateral triangle
inscribed in it, and a side AB
of an equilateral pentagon ;
therefore, of the equal seg-
ments of which there are
fifteen in the circle ABCD,
there will be five in the cir-
cumference ABC which is
one-third of the circle, and
there will be three in the cir-
cumference AB which is one-fifth of the circle ;

therefore in the remainder BC there will be two of the
equal segments.

Let BC be bisected at E ; [m. 30]

therefore each of the circumferences BE, EC is a fifteenth
of the circle ABCD,

If therefore we join BE, EC and fit into the circle ABCD
straight lines equal to them and in contiguity, a fifteen-angled
figure which is both equilateral and equiangular will have been
inscribed in it.

'   Q.E.F.

And, in like manner as in the case of the pentagon, if
through the points of division on the circle we draw
tangents to the circle, there will be circumscribed about the
circle a fifteen-angled figure which is equilateral and equi-
angular.

And further, by proofs similar to those in the case of the
pentagon, we can both inscribe a circle in the given fifteen-
angled figure and circumscribe one about it.
\end{proof}

\begin{notes}

Here, as in ii[. lo, we have the term ``circle'' used by Euclid in its
exceptional sense of the drcumjerena of a circle, instead of the ``plane figurt
contained by one hne'' of i. l)ef. 15. Cf. the note on that definition (Vol. i.
pp. 184—5).

Proclus (p. 269) refers to this proposition in illustratiotv of his statement
that Euclid gave proofs of a number of propositions with an eye to their use
in astronomy. `` With regard to the last proposition in the fourth Book in
which he inscribes the side of the fifteen-angled figure in a circle, for what
object does anyone assert that he propounds it except for the reference of this
problem to astronomy ? For, when we have inscribed the fifteen -angled figure
in the circle through the poles, we have the distance from the poles both of
the equator and the zodiac, since they are distant from one another by the
side of the fifteen-angled figure,'' This agrees with what we know from other
sources, namely that up to the time of Eratosthenes (circa 2iJ4 -204 B.C.) 24
was generally accepted as the correct measurement of the obliquity of the
ecliptic. This measurement, and the construction of the fifteen-angled figure,
were probably due to the Pythagoreans, though it would appear that the
former was not known to Oenopides of Chios (fl. circa 460 B.C.), as we learn
from Theon of Smyrna (pp. 198 — 9, ed, Hiller), who gives Dercy Hides as his
authority, that Eudemus (H. circa 32a B.C.) stated in his dcrTp«A.<ry('<u that,
while Oenopides discovered certain things, and Thales, Anaximander and
Anaximenes others, it was the rest (01 AoHrm) who added other discoveries
to these and, among them, that `` the axes of the fixed stars and of the planets
respectively are distant from one another by the side of a fifteen-angled figure.''
Eratosthen« evaluated the angle to Jrds of 180°, i.e. about 23' 51' 10'',
which measurement was apparently not improved upon in antiquity (cf. Ptolemy,
Syataxii, ed. Heiberg, p. 68).

Euclid has now shown how to describe regular polygons with 3, 4, 5, £
and 15 sides. Now, when any regular polygon is given, we can construct a
regular polygon with twice the number of sides by first describing a circle
about the given polygon and then bisecting all the smaller arcs subtended by
the sides. Applying this process any number of times, we see that we can by
Euclid's methods construct regular polygons with 3,1'', 4-a*, 5,2'', 15.2'' sides,
where « is zero or any positive integer.

\end{notes}

\end{proposition}

\part{Book V}

\chapter*{Introductory Note}

The anonymous author of a scholium to Book v. (Euclid, ed. Heiberg,
Vol. V. p. 280), who is perhaps Eoclus, tells us that ``some say'' thb Book,
containing the general theory of proportion which ts equally applicable to
geometry, arithmetic, music, and all mathematical science, ``is the discovery
of Eudoxus, the teacher of Plata'' Not that there had been no theory of
proportion developed before his time j on the contrary, it is certain that the
Pythagoreans had worked out such a theory with regard to numbtrs, by which
must be understood commensurable and even whole numbers (a number
being a `` multitude made up of units,'' as defined in Eucl.\ vii). Thus we
are told that the Pythagoreans distinguished three sorts of means, the
arithmetic, the geometric and the harmonic mean, the geometric mean
being called proportion (amXoyui) par exallenee; and further lamblichus
speaks of the ``most perfect proportion consisting of four terms and specially
called harmonU'' in other words, the proportion

a + b xab ,

       ``''IT-y'

which was said to be a discovery of the Babylonians and to have been Rrst
introduced into Greece by Pythagoras (lamblichus, Comm. en Nimtachas,
p. ti8). Now the principle of similitude is one which is presupposed by all
the arts of design from their very beginnings ; it was certainly known to the
Egyptians, and it must certainly have been thoroughly familiar to Pythagoras
and his school. This consideration, together with the evidence of the
employment by him of the gemetric proportion, makes it indubitable that the
Pythagoreans used the theory of proportion, in the form in which it was
known to them, i.e. as applicable to commensurables only, in their geometry.
But the discovery, also by the Pythagoreans, of the incommensurable would
of course be seen to render the proofs which depended on the theory of
proportion as then understood inconclusive ; as Tannery observes (Xrr
Giomftrie grecqui, p. 98), *' the discovery of incommensurability must have
caused a veritable logical scandal in geometry and, in order to avoid it, they
were obliged to restrict as far as possible the use of the principle of similitude,
pending the discovery of a means of establishing it on the basis of a theory of
proportion independent of commensurability.'' The glory 0/ the latter dis-
covery belongs then most probably to Eudoxus. Certain it is that the com.
plete theory was already familiar to Aristotle, as we shall see later.

It seems probable, as indicated by Tannery (lot. a'i.), that the theory
of proportions and the principle of similitude took, in the earliest Greek
geometry, an earlier place than they do in Euclid, but that, in consequence
of the discovery of the incommensurable, the treatment of the subject was
fundamentally remodelled in the period between Pythagoras and Eudoxus,
An indication of this is afforded by the clever device used in Euclid i. 44
for applying to a given straight line a parallelogram equal to a given triangle ;
the equality of the ``complements'' in a parallelciam is there used for doing
what is practically finding a fourth proportional to three given straight lines.
Thus Euclid was no doubt following for the subject-matter of Books t. — iv.
what had become the traditional method, and this is probably one of the
reasons why proportions and similitude are postponed till as late as Books
v., VI,

It is a remarkable fact that the theory of proportions is twice treated in
Euclid, in Book v. with reference to magnitudes in general, and in Book vci.
with reference to the particular case of numbers. The latter exposition
referring only to commensurable may be taken to represent fairly the theory
of proportions at the stage which it had reached before the great extension of
it made by Eudoxus. The differences between the definitions etc. in Books v.
and VII. will appear as we go on ; but the question naturally arises, why did
Euclid not save himself so much repetition and treat numbers merely as a
particular case of magnitude, referring back to the corresponding more
general propositions of Book v. instead of proving the same propositions
over again for numbers? It could not have escaped him that numbers
fall under the conception of magnitude. Aristotle had plainly indicated
that magnitudes may be numbers when he observed (Anal. post. t. 7,
75 b 4) that you cannot adapt the arithmetical method of proof to the
properties of magnitudes if the magnitudes are not numbers. Further
Aristotle had remarked (Anal. post. 1. 5, 74 a 17) that the proposition that
the terms of a proportion can be taken alternately was at one time proved
•eparately for numbers, lines, solids and times, though it was possible to prove
it for all by one demonstration ; but, because there was no common tuime
comprehending them all, namely numbers, lengths, times and solids, and their
character was different, they were taken separately. Now however, he adds,
the proposition is proved generally. Yet Euclid says nothing to connect
the two theories of proportion even when he comes in x. 5 to a proportion
two terms of which are magnitudes and two are numbers (`` Com mensurable
magnitudes have to one another the ratio which a number has to a number'').
The probable explanation of the phenomenon is that Euclid simply followed
tradition and gave the two theones as he found them. This would square
with the remark in Pappus (vii. p. 678) as to Euclid's fairness to others and
his readiness to give them credit for their work.

\chapter*{Definitions}

\begin{enumerate}

\item\label{def:V_1} A magnitude is a part of a magnitude, the less of
  the greater, when it measures the greater.

\item\label{def:V_2} The greater is a multiple of the less when it is
  measured by the less.

\item\label{def:V_3} A ratio is a sort of relation in respect of size
  between two magnitudes of the same kind.

\item\label{def:V_4} Magnitudes are said to have a ratio to one another
which are capable, when multiplied, of exceeding one another.

\item\label{def:V_5} Magnitudes are said to be in the same ratio, the
  first to the second and the third to the fourth, when, if any
  equimultiples whatever be taken of the first and third, and any
  equimultiples whatever of the second and fourth, the former
  equimuhiples alike exceed, are alike equal to, or alike fall short
  of, the latter equimultiples respectively taken in corresponding
  order.

\item\label{def:V_6} Let magnitudes which have the same ratio be
  called proportional.

\item\label{def:V_7} When, of the equimultiples, the multiple of the
  first magnitude exceeds the multiple of the second, but the multiple
  of the third does not exceed the multiple of the fourth, then the
  first is said to have a greater ratio to the second than the third
  has to the fourth.

\item\label{def:V_8} A proportion in three terms is the least
  possible.

\item\label{def:V_9} When three magnitudes are proportional, the first
  is said to have to the third the duplicate ratio of that which it
  has to the second.

\item\label{def:V_10} When four magnitudes are < continuously >
  propor- tional, the first is said to have to the fourth the
  triplicate ratio of that which it has to the second, and so on con-
  tinually, whatever be the proportion.

\item\label{def:V_11} The term corresponding magnitudes is used of
  antecedents in relation to antecedents, and of consequents in
  relation to consequents.

\item\label{def:V_12} Alternate ratio means taking the antecedent in
  relation to the antecedent and the consequent in relation to the
  consequent.

\item\label{def:V_13} Inverse ratio means taking the consequent as
  antecedent in relation to the antecedent as consequent.

\item\label{def:V_14} Composition of a ratio means taking the
  antecedent together with the consequent as one in relation to the
  consequent by itself.

\item\label{def:V_15} Separation of a ratio means taking the excess
by which the antecedent exceeds the consequent in relation
to the consequent by itself.

\item\label{def:V_16} Conversion of a ratio means taking the ante-
cedent in relation to the excess by which the antecedent
exceeds the consequent.

\item\label{def:V_17} A ratio ex aequali arises when, there being several
magnitudes and another set equal to them in multitude which
taken two and two are in the same proportion, as the first is
to the last among the first magnitudes, so is the first to the
last among the second magnitudes ;

Or, in other words, it means taking the extreme terms
by virtue of the. removal of the intermediate terms.

\item\label{def:V_18} A perturbed proportion arises when, there being
  three magnitudes and another set equal to them in multitude, as
  antecedent is to consequent among the first magnitudes, so is
  antecedent to consequent among the second magnitudes, while, as the
  consequent is to a third among the first magnitudes, so is a third
  to the antecedent among the second magnitudes.

\end{enumerate}

\section*{Definition 1}

The word/ar/ (fii'pot) is here used in the restricted sense of a submtiitipU
or an aliquot part as distinct from the more general sense in which it is used
in the Common Notion (;) which says that ``the whole is greater than the
part.'' It is used in th same restricted sense in vii. Def, 3, which is the same
definition as this with ``number'' (opifl/iot) substituted for ``magnitude.''
VII. Def. 4, keeping up the restriction, says that, when a number does not
measure another (>umoer, it is farfs (in the plural), not b part of it. Thus,
I, a, or 3, is a part of 6, bat 4 is not a pari of 6 but parts. The same
distinction between the restricted and the more general sense of the word
part appears in Aristotle, Mdaph. 1023 b is: ``In one sense a part is
that into which quantity (to irocrov) can anyhow be divided ; for that which is
taken away from quantity, guA quatitity, is always called a 'part' of it, as
e.g. two is said to be in a sense a part of three. But in another sense a
'part' ill only what mtasura (ra Karo/wTpovtra) such quantities. Thus two
b in one sense said to be a part of three, in the other not.''
noXXa)rX(i(rtov Si to «i£oy roC lAttrnivaft orov narajitTpTai vn> raC
IXaTTOfot.

\section*{Definition 2}

\section*{Definition 3}

AvytK ivrl Suo /iicyfPur ofiOytvif  Kara mXifconra vota <rw(.

The best explanation of the definitions of ratio miA proportion that I have
seen is that of De Moigan, which will be found in the articles under those
titles in the Penny Cyclopaedia, Vol xix. (1841) ; and in the following notes
I shall draw largely from these articles. Very valuable also aie the notes on
the definitions of Book v. given by Hanlcel (fragment on Euclid published as
an appendix to his work Zur GeschiehU der Mathimaiik in AUtrthum und
Mittdalier, 1874).

There has been controversy as to what is the proper translation of the
word in)Xucar))s in the definition, irxitrit 'ara n-iXtxcTTifTii has generally been
translated `` relation in respect of quantify.'' Upon this De Morgan remarks
that it makes nonsense of the definition ; ``for magnitude has hardly a
different meaning from quantity, and a relation of magnitudes with respect to
quantity may give a clear idea to those who want a word to convey a notion
of architecture with respect to building or of battles with respect to fighting,
and to no others.'' The true interpretation De Morgan, following Wallis and
Gregory, takes to be guantuplidty, referring to the number of times one
magnitude is contained in the other. For, he says, we cannot describe
magnitude in language without quantuplicitative reference to other magni-
tude; hence he supposes that the definition simply conveys the fact that the
mode of expressing quantity in terms of quantity is entirely based upon the
notion of quantuphcity or that relation of which we take cognizance when we
find how many times one is contained in the other. While all the rest of
De Morgan's observations on the definition are admirable, it seems to me
that on ttiis question of the proper translation of infAtKo'Tijt he is in error. He
supports his view tnainly by reference (i) to the definition of a compounded
ratio usually given as the 5th definition of Book vi., which speaks of the
TiiKiKonfrti of two ratios being multiplied together, and (t) to the comments
of Eutocius and a scholiast on this definition. Eutocius says namely
(Archimedes, ©d. Heiberg, iii, p. wo) that ``the term njKuainp is evidently
used of the number from which the given ratio is called, as (among others)
Nicomachus says in his first book on music and Heion in his commentary
on the Introduction to Arithmetic.'' But it now appears certain that this
definition is an interpolation ; it is never used, it is not found in Campanus,
and Peyrard's MS. only has it in the margin. At the same time it is clear
that, if the definition is admitted at all, any commentator would be obliged to
explain it in the way that Eutocius does, whether the explanation was consistent
with the proper meaning of mfXutorijv or not. Hence we must look elsewhere
for the meaning of mXiKot and nrXtitttnTt. If we do this, I think we shall find
no case in which the words have the sense attributed to them by De Morgan.
The teal meaning of irufXi'itos is how great. It is so used by Aristotle, e.g. in
Eth, Me. V. to, 1134 b 11, where he speaks of a man's child being as it were
a part of him so long as he is of a certain age (lutt av  mXtKcn'), Ag»in
Nicomachus, to whom Eutocius appeals, himself (i, 2, 5, p. Si ed. Hoche)
distinguishes ttjjXikov as referring to magnitude, while h-octo's refers to multitude.
So does lamblichus in his commentary on Nicomachus (p. 8, 3 — 5) ; besides
which lamblichus distinguishes irqdKov as the subject of geometry, being am-
tinucus, and irocraf as the subject of arithmetic, being discrele, and speaks of a
point being the origin of Tnjkinov as a unit is of iroo-oV, and so on. Similarly,
Ptolemy (Syntaxis, ed, Heiberg, p. 31) speaks of the sise (injXijtdr)) of the
chords in a circle (jr<pi njt mjXntoTifros tw jv rifi kuhXj ti6nwv). Consequently
I think we can only translate wiitK6n)t in the definition as size. This
corresponds to Hankel's translation of it as `` GrOsse,'' though he uses this
same word for a concrete `` magnitude `` as well ; size seems to me to give
the proper distinction between injXiitontt and ii,iyt0o, as size is the attribute,
and a magnitude (in its ordinary mathematical sense) is the thing which
possesses the attribute of siie.

The view that `` relation in respect of iize `` is meant by the words in the
text is also confirmed, I think, by a later remark of De Morgan himself,
tiamely that a synonym for the word raiis may be found in the more in-
telligible term relative magniiude. In fact axvm in the definition corresponds
to relative and in/XotoTj) to magnitude. (By magnitude De Moan here
means the attribute and not the thing possessing it.)

Of the definition as a whole Simson and Hankel express the opinion that
it is an interpolation. Hankel points to the fact that it is unnecessary and
moreover so vague as to be of no practical use, while the very use of the
expression na™ mjXtitorijTa seems to him suspicious, since the only other
place in which the word wrjkiKoxTp occurs in Euclid is the 5th definition of
Book VI., which is admittedly not genuine. Yet the definition of ratio appears
in all the MSS., the only variation being that some add the words npm oAAifXo,
``to one another,'' which are rejected by Heiberg as an interpolation of
Theon ; and on the whole there seems to be no sufficient ground for regarding
it as other than genuine. The true explanation of its presence would appear
to be substantially that given by Barrow (Lectiones Cantabrig., London, 1684,
Lect. Ill, of 1666), namely that Euclid bserted it for completeness' sake, mote
for ornament than for use, intending to give the learner a general notion of
ratio by means of a metaphysical, rather than a mathematical definition ; `` for
metaphysical it is and not, properly speaking, mathematical, since nothing
depends on it or is deduced from it by mathematicians, nor, as I think, can
anything be deduced.'' This is confirmed by the fact that there is no
definition of Xo'yot in Book vii., and it could equally have been dispensed
with here. Similarly De Morgan observes that Euclid never attempts this
vague sort of definition except when, dealing with a well-known term of
common life, he wishes to bring it into geometry with something like an
expressed meaning which may aid the conception of the thing, though it does
not furnish a perfect criterion. Thus we may compare the definition with
that of a straight line, where Euclid merely calls the reader's attention to the
well-known term tiStia ypa/iij, tries how far he can present the conception
which accompanies it in other words, and trusts for the correct use of the
term to the axioms (or postulates) which the universal conception of a straight
line makes self-evident.

We have now to trace as clearly as passible the development of the
conception of Xcfyoi, ratio, or relative magnitude. In its primitive sense
Xff/os was only used of a ratio between com mensu rabies, i.e. a ratio which
could be expressed, and the manner of expressing it is indicated in the
proposition, Eucl.\ x. 5, which proves that commensurate magnitudes have to
one another the ratio whieh a numl>er has to a number. That this was the
primitive meaning of Aoyoi is proved by the use of the term uAoyoi for the
mcom mensurable, which means irrational in the sense of not having a ratio
to something taken as rational (TTot). , , , .

- 1 Euclid himself shows us how we are to set about finding the ratio, or
relative magiiitude, of two commensurable magnitudes. He gives, in x. 3,
practically our ordinary method of finding the greatest common measure.
If ,  be two magnitudes of which B h the less, we cut off from A a part
equal to B, from the remainder a part equal to B, and so on, until we leave a
remainder less than B, say Ji,. We measure off , frcn S in the same way
until a remainder X., is left which is less than fii- We repeat the process
with 1, B,, and so on, until we find a remainder which is contained in the
preceding remainder a certain number of times exactly. If account is taken
of the number of times each magnitude is contained (with something over,
except at the last) in that upon which it is measured, we can calculate how
many times the last remainder is contained in A and how many times the
last remainder is contained in B ; and we can thus express the ratio of A to
B as the ratio of one number to another.

But it may happen that the two mttudes have no common measure,
i.e. are incommensurable, in which case the process described would never
come to an end and the means of expression would fail ; the magnitudes
would then Aave na ratio in the primitive sense. But the word Aoyos, ratio,
acquires in Euclid, Book v., a wider sense covering the relative magnitude of
incommensurabies as well as commensurables ; as stated in Euclid's 4th
definition, ``magnitudes are said to have a ratio to one another which can,
when multiplied, exceed one another,'' and finite incommensurabies have this
property as much as commetisurables. De Morgan explains the manner of
transition from the narrower to the wider signification of ratio as follows,
``Since the relative magnitude of two quantities is always shown by the
quantuplicitative mode of expression, when that is possible, and since pro-
portional quantities (pairs which have the same relative magnitude) are pairs
which have the same mode (if possible) of expression by means of each other ;
in all such cases sameness of relative magnitude leads to sameness of mode of
expression ; or proportion is sameness of ratios (in the primitive sense). But
sameness of relative magnitude may exist where quantuplicitative expression
is impossible ; thus the diagonal of a larger square is the same compared with
its side as the diagonal of a smaller square compared with its side. It is an
easy transition to speak of sameness of ratio even in this case ; that is, to use
the term ratio in the sense of relative magnitude, that word having originally
only a reference to the mode of expressing relative magnitude, in cases which
allow of a particular mode of expression. The word irraiional (SXsrpsi) does
not make any corresponding change but continues to have its primitive
meaning, namely, incapable of quantuplicitative expression.''

It remains to consider how we are to describe the relative magnitude of
two incommensurabies of the same kind. That they have a definite relation
is certain. Suppose, for precision, that S is the side of a square, D its
diagonal ; then, if .S is given, any alteration in D or any error in D would
make the figure cease to be a square. At the same time, a person altogether
ignorant of the relative magnitude of D and 5 might say that drawing two
straight lines of length .S so as to form a right angle and joining the ends by
a straight line, the length of which would accordingly be D, does not help
him to realise the relative magnitude, but that he would like to know how
many diagonals make an exact number of sides. We should have to reply
that no number of diagonals whatever makes an exact number of sides ; but
that he may mtaition any fraction of the side, a hundredth, a thousandth or
a millionth, and that we will then express the diagonal with an error not so
great as that fraction. We then teU him that 1,000,000 diagonals exceed

1,414,113 sides but fall short of 1,414,214 sides; consequently the diagonal
lies between t '41 41 13 and i '4 142 14 times the side, and these differ only by
one-millionth of the side, so that the error in the diagonal is less still. To
enable him to continue the firocess further, we show him how to perform the
arithmetical operation of approximating to the value of J 2. This gives the
means of carrying the approximation to any degree of accuracy that may be
desired. In the power, then, of carrying approximations of this kind as far as
we please lies that of expressing the ratio, so far as expression is possible, and
of comparing the ratio with others as accurately as if expression had been
possible

Euclid was of course aware of this, as were probably others before him ;
though the actual approximations to the values of ratios of incommensurabies
of which we find record in the works of the great Greek geometers are very
few. The history of such approximations up to Archimedes is, so far as
material was available, sketched in 7 e Works of Archimides (pp. Ixxvti and
following); and it is sufficient here to note the facts (i) that Plato, and,e*'en
the PythagoreaiK, were familiar with J as an approximfvtion to .j, (2) that
the method of finding any number of successive approximations by the system
of side- and iftajfo/taZ-numbere described by Theon of Smyrna was also
Pythagorean (cf. the note above on Euclid, n. 9, 10), (3) that Archimedes,
without a word of preliminary «tplanation, gives out that

gives approximate values for the square roots of several large numbers, and
proves that the ratio of the circumference of a circle to its diameter is less
than 3t but greater than j-rii (4) 't the first approach to the rapidity with
which the decimal system enables us to approximate to the value of surds
was furnished by the method of sexagesimal fractions, which was almost as
convenient to work with as the methoid of decimals, and which appears fully
developed in Ptolemy's avyravi. A number consisting of a whole number
and any fraction was under this system represented as so many units, so
many of the fractions which we should denote by , so many of those which
we should write (jj)', (A)'> *''<1 ° "  Theon of Alexandria shows us how
to extract the square root of 4500 in this sexagesimal system, and, to show
how effective it was, it is only necessary to mention that Ptolemy gives

-—5 + j + ~ as an approximation to 3, which approximation is equi«ilent

to 17320509 in the ordinary decimal notation and is therefore correct to
6 places.

Between Def. 3 and Def, 4 two manuscripts and Campanus insert `` Pro-
portion is the sameness of ratios'' (avoAoyta St jJ ruf AoyuiK rnvTonft), and even
the best ms. has it in the margin. It would be altogether out of place, since
it is not till Def, 5 that it is explained what sameness of ratios is. The words
are an interpolation later than Theon (Heiberg, Vol. v, pp. xxxv, Ixxxix),
and are no doubt taken from arithmetical works (cf Nicomachus and Theon
of Smyrna). It is true that Aristotle says similarly, `` Proportion is equality
of ratios'' (Eth. Nic. v. 6, 1131 a 31), and he appear to be quoting from
the Pythagoreans ; but the reference is to numbers.

Similarly two mss. (inferiorX insert after Def 7 ``Proportion is the similarity
(juartp) of ratios.'' Here too we have a mere interpolation.

\section*{Definition 4}

iymr f(v irpot SXXTjXa. furftdTj Kiytna, A St/rarai inAkiarXiuriaiiiMXi

This definition supplements the last one. De Morgan says that it amounts
to saying that the magnitudes are of the same species. But this can hardly
be aU ; the definition seems rather to be meant, on the one hand, to exclude
the relation of a finite magnitude to a magnitude of the same kind which is
either infinitely great or infinitely small, and, even more, to emphasise the
fact that the term ra/t'e, as defined in the preceding definition, and about to
be used throughout the book, includes the relation between any two t'ncom-
nunsurable as well as between any two commensurable finite mnitudes of
the same kind. Hence, while De Morgan seems to regard the extension of
the meaning of ratio to include the relative magnitude of incommensurables
as;, so to speak, taking place between Def. 3 and I>ef. Si the 4th definition
appears to show that it is ratio in its extended sense that is being defined in
Def. 3- .

\section*{Definition 5}

TrfapTOf, otar to, toC TpajTOU KoX TptVov laaKK iroAAairAotrio tw toB Sniripoo
xot rtrdpfTov utokk TroKXwrXairitiiv Kau ottoiovovv TroXXairAao'UKr/io iKartpcv

In my translation of this definition I have compromised between an
attempted literal translation and the more expanded version of Simson. 71ie
difiiculty in the way of an exactly literal translation is due to the fact that the
words (KaS* ijToiovoIi' jroXAaTrXao-iarr/iov) signifying that the equimultiples in
eiuh ease are any equimultiples wAa/evcr occur only once in the Greek, though
they apply ieiA to Ta....'uTdKK wokkan-kajria in the nominative and Tuv...Uri*is
u-oXAmrAno-i'ui' in the genitive. I have preferred ``alike `` to `` simultaneously''
as a translation of a/ia because `` simultaneously `` might suggest that time was
of the essence of the matter, whereas what is meant is that any particular
comparison made between the equimultiples must be made between (At same
equimultiples of the two pairs respectively, not that they need to be compared
at the same time,

Aristotle has an allusion to a definition of `` the same ratio `` in Tcfiia
VIII. 3, 158 b 29 ; `` In mathematics too some things appear to be not easy to
prove (ytidfa6ai) for want of a definition, e.g. that the parallel to the side
which cuts a plane [a parallelogram] divides the straight hne [the other side]
and the area similarly. But, when the definition is expressed, the said property
is immediately manifest ; for the areas and the straight lines Aave the same
di'Tai'tn'p«ri9, and this is the definition of 'the same ratio.'`` Upon this
passage Alexander says similarly, `` This is the definition of proportionals
which the ancients used : magnitudes are proportional to one another which
have (or show) ihe same mSvtupttrvst and Aristotle has called the latter
ivravalpKrit.'' Heiberg (Mathematisehes zu Aristofeles, p. 2 a) thinks that
Aristotle is alluding to the fact that the proposition referred to could not be
rigorously proved so long as the Pythagorean definition applicable to com-
mensurable magnitudes only was adhered to, and is (quoting the definition
belonging to the complete theory of Eudoxus ; whence, m view of the positive
statement of Aristotle that the definition quoted is the definition of ``the same
ratio,'' it would appear that the Euclidean definition (which Heiberg describes
as a careful and exact paraphrase of d-vTovalpftrn) is Euclid's own. I do not
feel able to subscribe to this view, which seems to me to involve very grave
difiScuUies. The Euclidean definition is regularly appealed to in Book v. as
the criterion of magnitudes being in proportion, and the use of it would appear
to constitute the whole essence of the new general theory of proportion; if then
this theory is due to Eudoxus, it seems impossible to believe that the definition
was not also due to him. Certainly the definition given by Aristotle would
be no substitute for it; dvfiv4>aipft7K and dnavoiptaK are words almost a:!
vague and `` metaphysical `` (as Barrow would say) as the words used to define
raifa, and it is difficult to see how any mathematical facts could be deduced
from such a definition. Consider for a moment the etymology of the words.
woifKo-ts or dvattFii means `` removal,'' `` taking away ``or `` destruction `` of
a thing; and the prefix om indicates that the ``taking away'' from one
magnitude answers to, corresponds with, alternates with, the `` taking away ``
from the other. So fai» therefore as the etymology goes, the word seems
rather to suggest the `` taking away `` of corresponding fractions, and therefore
to suit the old imperfect theory of proportion rather than the new one. Thus
Waitz (ad lac.) paraphrases the definition as meaning that `` as many parts as
are taken from one magnitude, so many are at the same time taken from the
other as well,'' A possible explanation would seem to be that, though
Eudoxus had formulated the new definition, the old one was still current in
the text-books of Aristotle's time, and was taken by him as being a good
enough illustration of what he wished to bring out in the pas.sage of the
Ibpia referred to.

From the revival of learning in Europe onwards the Euclidean definition
of proportion was the subject of much criticism. Campanus had failed to
understand it, had in fact misinterpreted it altogether, and he may have
misled others such as Ramus (1515 — 72), always a violently hostile critic of
Euclid. Among the objectors to it was no less a person than Galileo. For
particulars of the controversies on the subject down to Thomas Simpson
Elem. of Geometry, Lond. i8oo) the reader is referred to the Excursus at the
end of the second volume of Camerer's Euclid (1825). For us it is interesting
to note that the unsoundness of the usual criticisms of the definition was
never better exposed than by Barrow. Some of the objections, he pointed out
(tttt. Cantabr. vn.ofi665),areduetom isconception onthepartoftheir authors
as to the nature of a definition. Thus Euclid is required by these objectors
(e.g. Tacquet) to do the impossible and to show that what is predicated in the
definition is true of the thing defined, as if any one should be required to
show that the name ``circle'' was applicable to those figures alone which
have their radii all equal ! As we are entitled to assign to such figures and
such figures only the name of ``circle,'' so Euclid is entitled (``quamvis non
temere nee imprudenter at certii de causis iustis illis et idoneis'') to describe
a certain property which four magnitudes may have, and to call magnitudes
possessing that property magnitudes ``in the same ratio.'' Others had argued
from the occurrence of the other definition of proportion in vii. Def. so that
Euclid was dissatisfied with the present one ; Barrow pointed out that, on the
contrary, it was the fact that vu. Def. 3o was not adequate to cover the case
of incommensurables which made Euclid adopt the present definition here.
Lastly, he maintains, gainst those who descant on the ``obscurity'' of v.
Def. 5, that the supposed obscurity is due, partly no doubt to the inherent
difficulty of the subject of incommensurables, but also to faulty translators,
and most of all to lack of effort in the learner to grasp thoroughly the meaning
of words which, in themselves, are as clearly expressed as they could be.

To come now to the merits of the case, the best defence and explanation
of the definition that I have seen is that given by De Morgan, He first
translates it, observes that it applies equally to commensurable or incom-
mensurable quantities because no attempt is made to measure one by an
aliquot part of another, and then proceeds thus,

``The two questions which must be asked, and satisfactorily answered,
previously to its [the definition's] reception, are as follows :

1. What right had Euclid, or any one else, to expect that the preceding
most prolix and unwieldy statement should be received by the beginner as
the definition of a relation the perception of which is one of the most common
acts of his mind, since it is performed on every occasion where similarity or
dissimilarity of figure is looked for or presents itself F

2. If the preceding question should be clearly answered, how can the
definition of proportion ever be used ; or how is it possible to compare every
one of the infinite number of multiples of ji with every one of the multiples
of?

To the first question we reply that not only is the test proposed by
Euclid tolerably simple, when more closely examined, but that it is, or might
be made to appear, an easy and natural consequence of those (iandamental
perceptions with which it may at first seem difficult to compare it.''

To elucidate this De Morgan gives the following illustration.

Suppose there is a straight colonnade composed of equidistant columns
(which we will understand to mean the vertical lines forming the axes of the
columns respectively), the first of which is at a distance from a bounding wall
equal to the distance between consecutive columns. In front of the colonnade
let there be a straight row of equidistant railings (regarded as meaning their
axes), the first being at a distance from the bounding wall equal to the
distance between consecutive railings. Let the columns be numbered from
the wall, and also the railings. We suppose of course that the column distance
(say, C) and the railing distance (say, Ji) are different and that they may bear
to each other any ratio, commensurable or incommensurable ; i.e, that there
need not go any exact number of railings to any exact number of columns.

I t a 4 B 6 T 8

fl 10 tl 12 la 14 tfl la 17 la

If the construction be supposed carried on to any extent, a spectator can,
by mere inspection, and without measurement, compare C with Ji to any
degree of accuracy. For example, since the loth railing falls between the 4th
and 5th columns, 10 is greater than 4C and less than C, and therefore Jl
lies between -yVhs of C and yjjths of C. To get a more accurate notion, the
ten-thousandth railing may be talcen ; suppose it falls between the 4674th and
4675th columns. Therefore io,ooo.ff lies between 46 74 C and 4675 C, or  hes
between tVuVTy rVtyVs  There is no limit to the degree of accuracy
thus obtainable ; and the ratio of  to C is determined when the order of
distribution of the railings among the columns is assigned arf infinitum ; or, in
other words, when the position of any giver railing can be found, as to the
numbers of the columns between which it lies. Any alteration, however
small, in the place of the first railing must at last affect the order of
distribution. Suppose e.g. that the first railing is moved from the wall by one
part in a thousand of the distance between the columns ; then the second
railing is pushed forward by x7n(irC, the third by nnnri and so on, so that
the railings after the thousandth are pushed forward by more than C; i.e. the
order with respect to the columns is disarranged.

Now let it be proposed to make a model of the preceding construction in
which c shall be the column distance and r the railing distance. It needs no
definition of proportion, nor anything more than the conception which we
have of that term prior to definition (and with which we must show the agree-
ment of any definition that we may adopt), to assure us that C must be to J
in the same prof>ortion as  to r if the model be truly formed. Nor is it
drawing too largely on that conception of proportion to assert that the
distribution of the railings among the columns in the model must be every-
where the same as in the original ; for example, that the model would be out
ef proportion if its 37th railing fell between the i8th and 19th columns, while
the 37th rathng of the original fell between the 17th and iSth columns. Thus
the dependence of EucHd's definition upon common notions is settled; for the
obvious relation between the construction and its model which has just been
described contains the collection of conditions, the fulfilment of which,
according to Euclid, constitutes proportion. According to Euclid, whenever
mC exceeds, equals, or falls short of nR, then tttc must exceed, equal, or fall
short of nr; and, by the most obvious property of the constructions, according
as the wth column comes after, opposite to, or before the nth railing in the
original, the ffith column must come after, opposite to, or before the ffth
railing in the correct model.

Thus the test proposed by Euclid is necessary. It is also sufficient. For
admitting that, to a given original with a given column-distance in the model,
there is one correct model railing distance (which must therefore be that
which distributes the railings among the columns as in the original), we have
seen that any other railing distance, however slightly different, would at last
give a difTerent distribution ; that is, the correct distance, and the correct
distance only, satisfies all the conditions required by Euclid's definition.

The use of the word diitribtition having been well learnt, says De Morgan,
the following way of stating the definition will be found easier than that of
Euclid. `` Four magnitudes, A and B of one kind, and C and D of the same
or another kind, are proportional when all the multiples of A can be
distributed among the multiples of Bm the same intervals as the correspond-
ing multiples of C among those of D.'' Or, whatever numbers m, n may be,
if mA lies between ttB and (n + i)B, mC lies between nD and (« + i)£>.

It is important to note that, if the test be always satisfied from and after
any given multiples of A and C, it must be satisfied before those multiples. For
instance, let the test be always satisfied from and after oaA and looC; and
let f)A and 5C be instances for examination. Take any multiple of 5 which
will exceed 100, say 50 times five ; and let it be found on examination that
2504 lies between 678 and 67g.fi ; then 150 c lies between 678ZJ and
f>1D. Divide by 50, and it follows that A lies between   B and tzWB,
and ajortiori between iB and B. Similarly, 5 dies between islZ? and
13JJ/), and therefore between  D and 14Z?. Or A lies in the same
interval among the multiples of B in which 5 C lies among the multiples of D,
And so for any multiple of A, C less than 100, looC.

There remains the second question relating to the infinite character of the
definition ; four magnitudes A, B, C, D are not to be called proportional
until it b shown that every multiple of A falls in the same intervals among
the multiples of 5 in which the same multiple of C is found among the
multiples of D. Suppose that the distribution of the raihngs among the
columns should be found to agree in the model and the original as far as
the millionth railing. This proves only that the railing distance of the model
does not err by the millionth part of the corresponding column distance. We
can thus fix limits to the disproportion, if any, and we may make those limits
as small as we please, by carrying on the method of observation; but we
cannot obscrue an infinite number of cases and so enable ourselves to affirm
proportion absolutely. Mathematical methods however enable us to avoid
the difficulty. We can take any multipks whatever and work with them as if
they were particular multipJes. De Morgan gives, as an instance to show that
the definition of proportion can in practice be used, notwithstanding its
infinite character, tiie following proof of a proposition to the same effect as
EucL VI. 3.

o. A, oj At

``Let OAB be a triangle to one side AB of which ab is drawn parallel, and
on OA produced set off A At, AAf etc. equal to OA, and aa a,£i, etc. equal
to Oii.

Through every one of the points so obtained draw parallels to AB,
meeting OB produced in b, B, etc.

Then it is easily proved that W„ bj, etc. are severally equal to Ob, and
BB, BBi etc. to OB.

Consequently a distribution of the multiples of OA among the multiples
of Oa is made on one line, and of OB among those of Ob on the other.

The examination of this distribution in all its extent (which is impossible,
and hence the apparent difficulty of using the definition) is rendered
unnecessary by the known property of parallel lines. For, since At lies
between a, and a„ B must he between b and ,j for, if not, the line AB
would cut either a, or a.

Hence, without inquiring where A, doei fall, we know that, if it fall
between a, and a,„ B, must fall between b„ and ,+1 ; or, if m . OA fall in
magnitude between n.Oa and (n + i)(?a, then m.OB must fall between
n.Ob and («+i)Oi.''

Max Simon remarks (Euclid und die seeks planimeirischen Buchtr, p. no),
after Zeuthen, that Euclid's definition of equal ratios is word for word the
same as Weierstrass' definition of equal numbers. So far from agreeing in
the usual view that the Greeks saw in the irrational no number, Simon thinks
it is clear from Eucl.\ v. that they possessed a notion of number in all its
generality as clearly defined as, nay almost identical with, Weierstrass' con-
ception of it

Certain it is that there is an exact correspondence, almost coincidence,
between Euclid's definition of equal ratios and the modern theory of irrationals
due to Dedekind. Premising the ordinal arrangement of natural numbers in
ascending order, then enlarging the sphere of numbers by including
(i) negative numbers as well as positive, (2) fractions, as ajb, where a, b may
be any natural numbers, provided that i is not zero, and arranging the
fractions ordinally among the other numbers according to the definition :

let 1 be < = > J according as a;/ is < = > ii:,
e a

Dedekind arrives at the following definition of an irrational number.

An inatianal number a is defined whenever a law is stated which will
assign every given rational number to one and only one of two classes A and
B such that (i) every number in A precedes every number in j5, and (2) there
is no last number in A and no first number in B ; the definition of a being
that it is the one number which lies between all numbers in A and all
numbers in B.

Now let xly and ar'/y be equal ratios in Euclid's sense.

Then ~ will divide all rational numbers into two groups A and B ;
—, „ „ „ A' and B''.

Let -; be any rational number in A, so that

tax ,.

This means that ay <bx.

But Euclid's definition asserts that in that case af-cbs! also.

Hence also 7 < -1 ;

b y

therefore every member of group A is also a member of group A'.
Similarly every member of group  is a member of group B''.

For, if T belong to £,

ax

which means that ay > bx.

But in that case, by Euclid's definition, «y > bx' ;

therefore also i> -j-

y

Thus, in other words, A and B are coextensive with A' and S
respectively ;

therefore - = — , according to Dedekind, as well as according to Euclid.

If x(y, :iy happen to be rational,
then one of the groups, say A, includes xjy,
and one of the groups, say A', includes x'jy'. •

d . X • !.,-'''

In this case r might mncide with - ; ,..-.,

X

that is r = - 1

b y'

which means that aybx.

Therefore, by Euclid's definition, ay = e' ; -i-'wi-'* timu .,i.. •-

SO that T-5.

y

Thus the groups are again coextensive.

In a woid, Euclid's definition divides all rational numbers into two
coextensive classes, and therefore defines equal ratios in a manner exactly
corresponding to Dedekind's theory.

Alternativea for Eucl.\ V. Dcf. 5.

Saccheri records in his Evclides ob omni noivo vindicatus that a distinguished
geometer of his acquaintance proposed to substitute for Euclid's the following
definition :

``A first magnitude has to a second the same ratio that a third has to a
fourth when the first contains the aliquot parts of the second, auording to any
number [i.e. with any denominator] whatever, the same number of times as
the number of times the third contains the same aliquot parts of the fourth `` ;
on which Saccheri remarks that he sees no advantage in this definition, which
presupposes the notion of division, over that of Euclid which uses multiplication
and the notions o( greater, equal, and less.

This definition was, however, practically adopted by Faifofer [Elementi it
geometria, 3 ed., iSSsi) in the following form 1

`` Four infinitudes taken in a certain order form a proportion when, by
measuring the first and the third respectively by any equi-submultiples
whatever of the second and of the fourth, equal quotients are obtained,''

Ingrami (Elementi di geometria, 1904) takes multiples of the first and third
instead of submultiples of the second and fourth :

`` Given four magnitudes in predetermined order, the first two homogeneous
with one another, and likewise also the last two, the magnitudes are said to
form a proportion (or to be in proportion) when any multiple of the first
contains the second the same number of times that the equimultiple of the
third contains the fourth.''

Veronese's definition (Elementi di geometria, PL 11., 1905) is like that of
Faifofer; Enriques and Amaldi (Elemtnti di geomria, 1905) adhere to
Euclid's:

Proportionals of VII. Def. ao a particular case.

It has already been observed that Euclid has nowhere proved (though the
fact cannot have escaped him) that the proportion of numbers is included in
the proportion of magnitudes as a special case. This is proved by Simson as
being necessary to the 5th and 6th propositions of Book x. Simson's proof is
contained in his propositions C and D inserted in the text of Book v, and in
the notes thereon. Proposition C and the note on it prove that, if four
magnitudes are proportionals according to vii. Def. 20, they are also proportionals
according to v. Def. 5. Prop, D and the note prove the partial converse,
namely that, if four magnitudes are proportionals according to the Sth definition
of Book v., and if the first be any multiple, or any part, or parts, of the second,
the third is the same multiple, part, or parts, of the fourth. The proofs use
certain results obtained in Book V.

Prop. C is as follows ;

If the first be the same multiple of the second, or the same part of it, that tht
third is qfthe fourth, the first is to the second as the third to the fourth.

V. DEF. s] NOTE ON DEFINITION s nf

Let the (list A be the same multiple of B the second that C the third is of
the fourth D

 is to  as C is to 27.

A e

B O

C F

D H   '

Take of A C any equimultiples whatever E, F\ and of B, D any
equimultiple whatever G, H.

Then, because -4 is the same multiple of B that C is of 27, 'i- `` ; '``
and E is the same multiple o( A that F  of C,

E is the same multiple of B that F  of D. [v, 3]

Therefore E, Ek the same multiples of B, D.

But G, H d,K equimultiples of , D;

therefore, if £ be a greater multiple of B than G is, F'  a greater multiple of
JD than  is of i? ;

that is, if £ be greater than G, Fis greater than If.
In like manner,

if E be equal to G, or less, is equal to J/, or less than it.

But E, Fate equimultiples, any whatever, of , C;
and G, H any equimultiples whatever of B, D.

Therefore  is to  as C is to Z>. [v, Def. 5]

Next, let the first A be the same pari of the second B that the third C is
of the fourth D ;

j4 is to i? as C is to ZJ. A

For B is the same multiple of A that 2> is of C; B

wherefore, by the preceding case, q

 is to j4 as /J is to C; O

and, imxrsefy, A is to B as C is to D.

[For this last inference Sirason refers to his Proposition B. That
proposition is very simply proved by taking any equimultiples E, F of B, Z)
and any equimultiples G, Hoi A, C and then arguing as follows :

Since A to B  C is to Z>,

G, If are simuUamously greater than, equal to, or less than E, F
respectively ; so that

E, F are HmultantauUy less than, equal to, or greater than G, H
respectively,

and therefore [Def. 5]  is to 4 as i? is to C]

We have now only to add to Prop. C the case where AB contains the
same parts of CD that EFAaes of GH:

in this case likewise AB is to CD as EFto GIf.

Let CJC be a part of CD, and GL the same part of GIf; let AB be the
same multiple of CAT that EFis of GL.

iaS '- BOOK V [v. T>KF. 5

Therefore, by Prop. C, ``    '``• — -

A£ is to Cas £Fu> GL,

B E-

G-

c— R

And CD, GH ire equimultiples of CK, GL, the second and fourth.

Therefore AB is to CD as EF to G [Simson's Cor, to v. 4, which
however is the particular case of V. 4 in which the `` equimultiples `` of one
pair are the pair itself, i.e. the pair multiplied by unity].

To prove the partial converse we begin with Prop. D.

If the first be to the second as the third to the fourth, and if the first be a
multiple or part of the second, the third is the same mulliple or the same part of
the fourth.

Let v4 be to  as C is to i> ;

and, first, let /i be a multiple of B ;

C is the same multiple of S.

Take E equal to A, and whatever multiple A 01 E s of B, malce F the
same multiple of Z>.

Then, because A is Xo B a Cs 10 D, < '

and of B the second and D the fourth equimultiples have been taken E
and F,

/i is to £ as C is to  [v. 4, Cor.]

But A is equal to E ;

therefore C is equd to F.

[In support of this inference Simson cites his Prop. A, which however we
can directly deduce from v. Def. 5 by taking any, but the same, equimultiples
of all four magnitudes.]

A C-

B D-

e— F-

Now ,is the same multiple of Z> that Aisol B;

therefore C is the same multiple of D that A is of B.
Next, let the first /i be a part of the second B ;

C the third is the same part of the fourth D,
Because i is to ,5 as C is to D,

inversely, J is to -4 as Z) is to C. [Prop. B]

But A'ya. part of J; therefore ,5 is a multiple of >€;

and, by the preceding case, D is the same multiple of C,

that is, C is the same part of D that A is of B.
We have, again, only to add to Prop D the case where AB contains any
parts of CD, and AB is to CD as EFio GH;

then shall EF zontain the same parts of GB that AB does of CD.

?  DWr. 5—7] NOTES ON DEFINITIONS g— 7 199

For let CIC be a part of CJ), and G'Z the same part of Gff; and let j4£

be a multiple of CX.

jff shall be the same multiple of GL.

Take M the same multiple of GL that AB h of CA'';
therefore ji£ is to CAT as  is to GL. [P''*p. C]

A

B

E

— F

C tr-

G- L

H

M

And CI>, G/fte equimultiples of CA'', GL;
therefore /4B is to CD as j)/ is to GIL

But, by hypothesis, A£ is to CZJ as £is to GB'';

therefore M is equal to £J\ [v. 9]

and consequently £i3 the same multiple of GL that A£ is of Cff.

Definition 6. '

t

Tb Si TOV airoy Ijdktu 6yov /ityeTj avaXjoyov KoXturBa, x >

'A,vdXoryoy, though usually written in one word, is equivalent to aVi Xoyot', /»
proportion. It comes however in Greek mathematics to be used practically as
an indeclinable adjective, as here ; cf. oi Tto-o-opt? euflttdi avaktyfov laovtiu.,
`` the four straight lines will be proportional,'' tflyw/a. rat irXm/Mi aVoAoyov
Ix''** ``triangles having their sides proportional.'' Sometimes it is used
adverbially : akoXoyov o kiv ci>c 17 BA irpo? rv AF, outq ? HA irpo Tf AZ,
``proportionally therefore, as BA is to AC, so is CZJ to £>F''\ so too, ap-
parently, in the expression ij /lAnj aniXoyoi/ (*ufl<ut), `` the mean proportional.''
I do not follow the objection of Max Simon (Euclid, p. no) to ``proportional''
as a translation of oniAoyoi'. ``We ask,'' he says, ``in vain, what is proportional
to what? We say e.g. that weight is proportional to price because double, treble
etc. weight corresponds to double, treble etc. price. But here the meaning must
be 'standing in a relation of proportion.'`` Yet he admits that the Latin word
preportionalis is an adequate expression. He transl.ttes by ``in proportion''
in the text of this definition. But I do not see that ``in proportion `` is better
than ``proportional.'' The fact is that both expressions are elliptical when
used of four magnitudes `` in proportion `` ; but there is surely no harm in
using either when the meaning is so well understood.

The use of the word naXtiV, `` let magnitudes having the same ratio be
called proportional,'' seems to indicate that this definition is Euclid's own.

Definition 7.

TOv Tov Btvripov iroAXarXaatou, to Sc to€ rpirov TroWaTrKaior ij (ijrtpi)fff rov
ro3 rtropTou troXXavXairltni, rott to rpajrov irpoi to Stvrtpor fWifuwi Xoyoi'' •X'``'
Xfycrat, tp ra rpiVcn vpo to TtToprov.

As De Moian observes, the practical test of disproportion is simpler than
that of proportion. For, whereas no examination of individual cases, however
extensive, will enable an observer of the construction and its model (the
illustration by means of columns and railings described above) to affinn
proportion or deny disproportion, and all it enables us to do is to fix limits
(as small as we please) to the disproportion (if any), a single instance may
enable us to deny proportion or affirm disproportion, and also to slate which
way the disproportion lies. Let the igth railing in the original fall beyond
the nth column, while the 15th railing of the (so-called) model does not
come up to the nth column. It follows from this one instance that the
railing distance of the model is too small relatively to the column distance, or
that the column distance is too great relatively to the railing distance. That
is, the ratio of /• to r is less than that of /? to C, or the ratio of f to r is greater
than that of C to jR.

Saccheri (<7/. at,) remarks (as Commandinus had done) that the ratio of
the first magnitude to the second will also be greater than that of the third to
the fourth if, while the multiple of the first is efuai to the multiple of the
second, the multiple of the third is Uss than that of the fourth : a case not
mentioned in Euclid's definition. Saccheri speaks of this case being included
in Clavius' interpretation of the definition. 1 have, however, failed to find a
reference to the case in Clavius, though he adds, as a sort of corollary, in his
note on the definition, that if, on the other hand, the multiple of the first is
iess than the multiple of the second, while the multiple of the third is nt>i las
than that of the fourth, the ratio of the first to the second is kss than that of
the third to the fourth.

Euclid presumably left out the second possible criterion for a greater ratio,
and the definition of a less ratio, because he was anxious to reduce the
definitions to the minimum necessary for his purpose, and to leave the rest to
be inferred as soon as the development of the propositions of Book v. enabled
this to be done without difficulty.

Saccheri tried to reduce the second possible criterion for a greater ratio to
that ven by Euclid in his definition without recourse to anything coming
later in the Book, but, in order to do this, he has to use ``multiples'' produced
by multipliers which are not integral numbers, but integral numbers //uJ proper
fractions, so that Euclid's Def. 7 becomes inapplicable.

De Morgan notes that `` proof should be given that the same pair of
magnitudes can never offer both tests [i.e. the test in the definition for a
greater ratio and the corresponding test for a less ratio, with ``less'' substituted
for ``greater'' in the definition] to another pair; that is, the test of greater
ratio from one set of multiples, and that of less ratio from another.'' In other
words, if m, n, p, q are integers and A, B, C, D four magnitudes, none of the
pairs of equations

(i) mA->HB, mC=ai <nD,
(a) mA = nB, mC < nD
can be satisfied simultaneously with any one of the pairs of equations

(3) pAqB, pC>qD,

(4) pA < qB, y*C > or = qD.

There is no difficulty in proving this with the help of two simple
assumptions which are indeed obvious.

We need only take in illustration one of the numerous cases. Suppose, if
possible, that the following pairs of equations are simultaneously true :

(l) viA>nB, mC<nD
and (2) pA <qB, pC>qD.

V. DEFF. r, 8] NOTES ON DEFINITIONS ;, 8 iji

Multiply (i) by q and (z) by n.

(We need here to assume that, whet« rX, rK are any equimultiples of any
magnitudes X ¥,

according as X>- = < Y, rX> = <rY.

This is contained in Simson's Axioms i and 3.)

We have then the pairs of equations

my A > ngB, mqC < nD,

npA<nqB, npC>nqD,

From the second equations in each pair it follows that

mqC < npC.

(We now need to assume that, if rX, sX are any multiples of X, and
rY, sY the same multiples of Y, then,

according as rX >-< sX, rY-> = < sY.

Simson uses this same assumption in hb proof of v. i3.)

Therefore mqA <npA,   ,

But it follows from the first equations in each pair that

mqA > npA :
which is impossible.

Nor can Euclid's criterion for a greater ratio coexist with that for equal

ratios.

\section*{Definition 8}

 kvoXvfia, Si iv fptaXv opoit ISaxurryi hrriv.

This is the reading of Heiberg and Camerer (who follow Peyrard's Ms,)
and is that translated above. The other reading has Aaxiorcni, which can
only be translated ``consists in three terms a/ hast.'' Hankel regards the defi-
nition as a later interpolation, because it is superfluous, and because the word
ojxK for a term in a proportion is nowhere else used by Euclid, though it is
common in later writers such as Nicomachus and Theon of Smyrna. The
genuineness of the definition is however supported by the fact that Aristotle
not only uses Spot in this sense (Eth. Nic. v. 6, 7, 1131 b 5, 9), but has a similar
remark (ibid. 1131 a 31) that a ``proportion is in fQiir terms at least'' The
difference from Euclid is only formal ; for Aristotle proceeds : `` The diicrett
(Sipfiini) (proportion) is clearly in four (terms), but so also is the continuous
(awtxv'). For it uses one as two and mentions it twice, e.g. (in stating) that,
as a is to j3, so also is j3 to y ; thus j3 is mentioned twice, so that, if /9 be twice
put down, the proportionals are four.'' The disrinction between discrete and
C07ttinuous seems to have been Pythagorean (cf. Nicomachus, 11. 11, 5; 23,
a, 3; where however o-un/fioTj is used instead of Tvvtxq); Euclid does not
use the words Stjpit|/t(n; and awtxi in this connexion.

So far as they go, the first words of the next definition (g), ``When three
magnitudes are proportionals,'' which seemingly refer to Def. 8, also support
the view that the latter is, at least in substance, genuine.

\section*{Definitions 9, 10}

9. 'Oral' Si Tpi'a fityirj ivaXayov   to irpwroi' jrpos ri TpiTOf fcirXnirtova
Xoyov <;(<£i Xcycrai tttc/i irput to BtvTtpov.

10. 'Orair Sc THTO'aML fttyiTq aVaXoyoi  to vfitarav irpm To rireifrTOV
Tpurkatrlova Awyoy fx** 'tyriM irep irpos li MvTtpoy, nai dtt ift opHuii, wt

Here, and fn connexion with the definitions of duplicate, triplicate, etc.
ratios, would be the place to expecta definition of ``compaunii ratio.'' None
such is however forthcoming, and the only ``definition'' of it that we find is
that forming vi. Def. Si which is an interpolation made, perhaps, even before
Theon's time. According to the interpolated definition, `` A ratio is said to
be compounded of ratios when the sizes (jnfAtKo'nTret) of the ratios multiplied
together make some (? ratio).'' But the multiplication of the lis (or
magnitudes) of two ratios of incommensurable, and even of commensurable,
magnitudes is an operation unknown to the classical Greek geometers.
Eutocius (Archimedes, ed. Heiberg, iii. p. lao) is driven to explain the
definition by making irijAwonj mean the number from which the given ratio
is called, or, in other words, the number which multiplied into the consequent
of the ratio gives the antecedent. But he is only able to work out his idea with
reference to ratios between numbers, or between commensurable magnitudes ;
and indeed the definition is quite out of place in Euclid's theory of
proportion.

There is then only one statement in Euclid's text as we have it indicating
what is meant by compound ratio ; this is in vi: 23, where he says abruptly
``But the ratio of KXa M is compounded of the ratio of JT to Z and that of
L to M.'' Simson accordingly gives a defitiition (A of Book v.) of compound
ratio directly suggested by the statement in vi. 23 just quoted.

`` When there are any number of magnitudes of the same kind, the first
is said to have to the last of them the ratio compounded of the ratio which
the first has to the second, and of the ratio which the second has to the third,
and of the ratio which the third has to the fourth, and so on unto the last
magnitude.

For example, if A B, C, D b four maitudes of the same kind, the
first A is said to have to the last D the ratio compounded of the ratio of
A to B, and of the ratio of B to C, and of the ratio of C to Z) ; or the ratio
o A 10 Dm said to be compounded of the ratios of A ta S, B to C, and
C to D.

And if .(4 has to .ff the same ratio which E has to F; and .ff to C the
same ratio that G has to /f ; and C to Z> the same that A'' has to L; then,
by this definition, A is said to have to D the ratio compounded of ratios
which are the same with the ratios of E io F, G to H, and A'' to Z : and the
same thing is to be understood when it is more briefly expressed, by saying,
A has to D the ratio compounded of the ratios of £  to  G to If, and
JT to Z,

In like manner, the same things being supposed, if M has to N the
same ratio which A has to D ; then, for shortness' sake, M is said to have to
Athe ratio compounded of the ratios of -£  to  G to H, and A'' to L.''

De Morgan has some admirable remarks on compound ratio, which
uot only give a very clear view of what is meant by it but at the same time
supply a. plausible explanation of the origin of the term. ``Treat ratio,'' says
De Morgan, ''as an engine of operation. Let that of j4 to .ff surest the
power of altering any magnitude in that ratio.'' (It is true chat it is not yet
proved that, B being any magnitude, and /'' and Q two magnitudes of the
same kind, there does exist a magnitude ji which is to JS in the same ratio
OS /* to Q. It is not till vi. 1 3 that this is proved, by construction, in the
particular case where the three magnitudes are straight lines. The proof in the
Greek text of v, 18 which assumes the truth of the more general proposition
is, by reason of that assumption, open to objection ; see the note on that
proposition.) Now ``every alteration of a magnitude is alteration in some
ratio, two or more successive alterations are jointly equivalent to but one, and
the ratio of the initial magnitude to the terminal one is as properly said to be
the compound ratio of alteration as 13 to be the compound addend in* lieu of
8 and 5, or 28 the compound multiple for 7 and 4. Competition is used
here, as elsewhere, for the process of detecting one single alteration which
produces the joint effect of two or more. The composition of the ratios of
P a R, R .o S, 7'' to 6 is performed by assuming A, altering it in the first
ratio into B, altering B in the second ratio into C, and C in the third ratio
into D. The joint effect turns A into D, and the ratio o( A to D is the
compounded ratia''

Another word for (ompouncUd ratio is crvnjKwot (cruraTmu) which '
common in Archimedes and later writers.

It is clear that diiplicate ratio, triplicate ratio etc. defined in v. Deff. 9
and 10 are merely particular cases of compound ratio, being in fact the
ratios compounded of two, three etc tqual ratios. The use which the Greek
geometers made of compounded, duplicate, triplicate ratios etc. is well
illustrated by the discovery of Hippocrates that the problem of the duplication
of the cube (or, more generally, the construction of a cube which shall be to
a given cube in any given ratio) reduces to that of finding ``two mean
proportionals in continued proportion.'' This amounted to seeing that, if
X, y are two mean proportionals in continued proportion between any two
lines a, b, in other words, if a is to jc as a; to, and a: is toj* as jf to , then a
cube with side a is to a cube with side .x as a is to b\ and this is equivalent
to saying that a has to b the triplicate ratio of a to .-t*.

Euclid is careful to use the forms SorXatri'iuK, tpntXaaiioy, etc. to express what
we translate as dupiicait, tripliatte etc. ratios ; the Greek mathematicians,
however, commonly used StirXoffios Xoyg, ``double ratio,'' TpiirAao-ios Xoyw,
``triple ratio `` etc, in the sense of the ratios of i to i, 3 to i etc. The effort,
if such it was, to keep the one form for the one signification and the other for
the other was only partially successful, as there are several instances of the
contrary use, e,g. in Archimedes, Nicomachus and Pappus.

The expression for having the ratio which is `` duplicate (triplicate) of that
which it has to the second'' is curious — S«rAacrio™ (Tfjin-Aao-i'oi'a) irfov Ixtw
rcp 7rp« TO ieuTtr — -Trtp being used as if SiwXafftova or TpiTrXtwtoi'fl were a
sort of comparative, in the same way as it is used after ftfifom or tAno-o-ovo.
Another way of expressing the same thing is to say XiJyiK Sn-Xatrtwi' (fsatXasrlani)
TO if,  v lyy.... the ratio ``duplicate of that (ratio) which,,.'' The explanation
of both constructions would seem to be that StirXoo-io; or S(7rAacrw»' is, as
Hultsch translates it in his edition of Pappus (cf p. 59, 17), duplo maior,
where the ablative duplo implies not a difference but a proportion.

The four magnitudes in Def, 10 must of course be in continued proportion
(Kara to <rv>'(;(n). The Greek text as it stands does not state this.

'0/Moya fuyiTi KiyiTot To fihi TJytyv/ittva Toij tf)N)ti/i(ViH9 ra Si jira;ic)« Tott

It is difficult to expre the meaning of the Greek in as few words. A
translation more literal, but conveying less, would be, ``Antecedents are called
wrrtsponding magnitudes to antecedents, and consequents to consequents.''

I have preferred to translate iXoyo* by `` corresponding `` rather than by
`` homologous.'' I do not agree with Max Simon when he says (Euclid, p. 1 11)
that the technical term ``homologous'' is not the adjective i/uSAnyo!, and does
not mean ``corresponding,'' ``agreeing,'' but ``like inrespect of the proportion''
(``ahnlich in Bezug auf das Verhaltniss''), The definition seems to me to be
for the purpose of appropriating to a technical use precisely the ordinary
adjective A/ioAoyos, ``agreeing'' or ``corresponding.''

Atttt(tdtnts, TJyov/JLtva, are literally ``leoiiing (terms),'' and cotistijuents ,
iini/itvo, ``following (terms).''

\section*{Definition 12}

``EvaAAoi Aifyo* foTt Xiji/fii toB jj-you/atrou irpos To -tfyttityvi/ mi tou hmvow
irpoc TO hmvGV

We now come to a number of expressions for the transformation of ratios
or proportions. The first is ifoAAiif, alUrnaldy, which would be better
described with reference to a proportion of four terms than with reference to
a ratio. Bui probably Euclid defined all the terms in DeiT. li — 16 with
reference to ratios because to define them with reference to proportions would
look like assuming what ought to be proved, namely the legitimacy of the
various transformations of proportions (cf. v, 16, 7 Por., 18, 17, 19 For,). The
word iraXXof is of course a common term which has no exclusive reference to
mathematics. But this same use of it with reference to proportions already
occurs in Aristotle: Anal. post. t. 5, 74 a 18, i«u *o dvoAcryoi' ort ™AAi»f,
``and that a proportion (is true) alternately, or alttrnando'' Used with Aoyos,
as here, the adverb ivaXka has the sense of an adjective, ``alternate''; we
have already had it similarly used of `` alternate angles `` (at JvoAAof yuiruu) in
the theory of parallels.

\section*{Definition 13}

AvamAu' Aayot itrrX Aii tou iiro/io'ou wc vyov/nrwiu irpo? to ytmiitvm/ wc
Jro/MVOf.

'AvdiraXiv, `` inversely,'' `` the other way about,'' is also a general term with
no exclusive reference to mathematics. For this use of it with reference to
proportion cf, Aristotle, De Cado i. 6, i73b32TiJi' ovaAoyiV v to jSiipTj Ix'',
oE xpwoi avatrakai tfoiwiv, `` the proportion which the weights have, the times
will have invtrsely.'' As here used with Aoyo?, avdvaXw is, exceptionally,
adjectival.

\section*{Definition 14}

``Xwvtfri Aoyou i<n\ Xi4.f TOtJ jyavfmivov fttri row iTrofjfvov <i>s if os wpo* auTo
TO trontfav.

The tomfosition of a ratio is to be distinguished from the cetnpounding of
ratios and compounded ratio (<rvyttiijifvo  Aoyoi) as explained above in the note
on Deff. 9, ro. The fact is that oTPrntftj/it and what serves for the passive of
it (ffvymtfww) are used for adding as well as compounding in the sense of
compounding ratios. In order to distinguish the two senses, I have always
used the word wmpomndo where the sense is that of this definition, though
this requires a slight departure from the literal rendering of some passages.
Thus the enunciation of v. 17 says, literally, ``if magnitudes compounded be
in proportion they will also be in proportion separated'' (Wv crvyitttfMfa
/jwytAj ixiktrjov , Kol Sto(p«6fl/To omAoyov fo-rai). This practically means
that, if j4 + 5 is to .S as C + Z) is to A then v4 is to J as C is to D.
I have accordingly translated as follows : `` if magnitudes be proportional
eompomnde, they will also be proportional separando'' (It will be observed
that stparattdo, a term explained in the next note, is here used, not relatively
to the proportion /(! is to .ff as C is to D, but relatively to the proportion
compenmdo, viz. A - B \ xo B as, C + Z* is to 2).) The corresponding
term for eomponendo in the Greek mathematicians is avvBivn, literally ``to one
who has compounded,'' i.e. `` if we compound.'' (For this absolute use of the
dative of the participle cf Nicomachus i. 8, 9 otto /ioi'<iSijt...™ia tov SurXaa-ioc
Xdyoi' jrpojfiupouvTi ni)(pK Aimpow, hoi ictu v yiviiivTat, ovrot Travrn pTWiKK
apTioi tiinv. A very good instance from Aristotle is £(A. Ni(. 1. 5, 1097 b la
ittttTivovTK yap 4 roii yokcis koX tous diroyovovi koX rw fikattf roix lXau
tK arfifiov wfxifunv.) A variation for mivSivri, found in Archimedes is nara
<rvi/nTLv, Perhaps the more exclusive use of the form <rur0ivTi by geometers
later than Euclid to denote the composition of a ratio, as compared with
Euclid's more general use of mvStaxa and other parts of the verb truirtftjfu
or ovymifuu, may point to a desire to get rid of ambiguity of terms and to
make the terminology of geometry more exact.

\section*{Definition 15}

Ataip«r(s Xoyov itm. Xifii rv inrti,  vvtpi-L to iyoucpoi tov
kicovvfOy TTpb (XvTO rh hrofAivov,

As composition of a ratio means the transformation, e.g., of the ratio of
A Xa B into the ratio oi A + B to S, so the uparcUion of a ratio indicates
the transformation of it into the ratio oi A~ B to B. Thus, as the new
antecedent is in one case got by adding the original antecedent to the original
consequent, so the antecedent in the other case is obtained by subtracting the
original consequent from the original antecedent (it being assumed that the
latter is greater than the former). Hence the literal translations of tiaifnirn
Xoyou, ``division of a ratio,'' and of htXovti (the corresponding term to
<rvr6ivri) as dividtndo, scarcely give a sufficiently obvious explanation of the
meaning. Heiberg accordingly translates by ``subtractio lationis,'' which
again may be thought to depart too far from the Greek. Perhaps ``separation''
and separando may serve as a compromise.

  *.ir -»*r!\ jnj

\section*{Definition 16}

``AKotTTpo Ao'yoii /<rTt AiJis roG ijyoufiA'Oii nrpos njc vrtpoxijV,   Airtpiyn

Conversion of a ratio means taking, e.g., instead of the ratio of A to B,
the ratio oi A to A —JB (A being again supposed greater than B). As
iwwTpoi7 is used for conversion, so ivaxTtpiiftavri is used for conver tends
(corresponding to the terms aviiivtt, and SwAdvrt).

\section*{Definition 17}

trivia XafiavQfiivttfr nal iv T(3 avrcp Aoyitij orav  ii iy rol wp-rotv fuytOttrt to
vpwTOr itpm TO itr)(aToy, outuie ivTott Sivripoit fiTytSiiTno irfKuTOk' *poi to JtrjjaToi'
 oAXuT' >frK T)a¥ axpuiv koB' vw4fyitpt<n.r Tmv jiiam>.

Si tcrov, tx atquali, must apparently mean ex atquali dislaniia, at an equal
distance or interval, i.e. after an equal number of intervening terms. The
wording of the definition suggests that it is rather a proportion ex atquali
than a ratio ex aet/uali which is being defined (cf. Def. 12). The meaning is
clear enough. If a, 6,c,d...ht one set of magnitudes, and A, B, C, D...
another set of magnitudes, such that

(I is to * as vi is to B,

 is to f as .d is to C,
and so on, the last proportion being, e.g.,

/' is to /, as A* is to Z,
then the inference ex aeqtmli is that

a is to / as /4 is to i. '

The/orf that this is so, or the truth of the inference from the hypothesis,
is not proved until v. 22. The definition is therefore merely verbal; it gives
a convenient name to a certain inference which is of constant application in
mathematics. But ex aequali could not be intelligibly defined except with
reference to two sets of ratios respectively equal. ,

\section*{Definition 18}

TtTopayjiivTi Si A.vaXoyia ttniv orav tloiv VTtffV fMytS ircu aXAai»' aurotc
r<r«i' TO Trkftov yiVip-ai ut /xiv iv to« jfpcoToit fHyiSiirtvoviityoir irpos Irofitvov,
ovT4i>c iy Tots otvTtpot. fiiyitiTtv tivfitvtty wpo Iwofi-tvov, mi Si iv rots Tr/xJroi
lktyiBi<rKV inopfvoy irpos aXka Tc, ourcii? iy this £<UTCpot; aXXo Tt vphi yovfttvov

Though the words Bi' 'urav, ex atqiitUt, are not in this definition, it gives a
description of a case in which the inference ex aeguali is still true, as will be
hereafter proved in v. 23, A perturbed proportion is an expression for the
case when, therp being three magnitudes a, b, c and three others A, B, C,

a is to  as . is to C,

and  is to f as j4 is to if.

Another description of this case is found in Archimedes, ``the ratios being
dissimilarly ordered `` (nVooitut TtTayji-iyiay ruy koymy). The full description of
the in/ereiue in this case (as proved in v. 23), namely that

a is to f as .f4 is to C,

is ex aequali in perturbed proportion (8c' lO-ou iv Tviapayp,ivj) dvaXoyl),
Archimedes sometimes omits the Eo-ou, first giving the two profwrtions and
proceeding thus: ``therefore, the proportions being dissimilarly ordered, a has
to e the same ratio as A has to C.''

The fact that Def. 18 describes a particular case in which the inference
St' to-oii will be proved true seems to have suggested to some one after
Theon's time the interpolation of another definition between 17 and 18 eo
describe the ordinary case where the argument ex atquaii holds good. The
interpolated definLtion runs thus ; ``an ordered proportion (T<ray/it»T; aVoXoyw)
arises when, as antecedent is to consequent, so is antecedent to consequent,
and, as consequent is to something else, so is consequent to something else.''
This case needed no description after Def, 17 itself j and the supposed
definition is never used.

After the definitions of Book v. Simson supplies the following axioms.

I. Equimultiples of the same or of equal magnitudes are equal to one
another.

3. Those magnitudes of which the same or equal magnitudes are
equimultiples are equal to one another.

3. A multiple of a greater magnitude is greater than the same multiple
of a less.

4. That magnitude of which a multiple is greater than the same multiple
of another is greater than that other magnitude.

.I''. . r.'' .1:. T

i. . ..;

jL»u'..''fi 1'

l'' - t.-.rli ,>, i

I .1

I i,

I- • <>•

\part{Book V. Propositions}

\begin{proposition}
\label{prop:V_1}

\begin{statement}
If there be any number of magnitudes  whatever which are,
respectively, equimultiples of any magnitudes equal in multitude,
then, whatever multiple one of the magnitudes is of one, that
multiple also will all be of all.
\end{statement}

\begin{proof}

Let any number of magnitudes whatever AB, CD be
respectively equimultiples of any magnitudes E, F equal in
multitude ;

I say that, whatever multiple AB is of E, that multiple will
AB, CD also be of E, F.

For, since  is the same multiple of E that CD is of F,
as many magnitudes as there are in AB equal to E, so many
also are there in CD equal to F.

Let AB be divided into the magnitudes AG, GB equal
to E,

and CD into CH, HD equal to F ;

then the multitude of the magnitudes AG, GB will be equal

to the multitude of the magnitudes CH, HD.

Now, since 6 is equal to E, and CH to F,
therefore AG is equal to E, and AG, CH to E, F.

For the same reason

GB is equal to E, and GB, HD to E, E;

therefore, as many magnitudes as there are in AB equal to £,
so many also are there in AB, CD equal to E, F;
therefore, whatever multiple AB Is of E, that multiple will
AB, CD also be of E, F.
Therefore etc.

Q. E, n.
\end{proof}

\begin{notes}

De Morgan remarks of v, i — 6 that they are ``simple propositions of
concrete arithmetic, covered in language which makes them iminte)!igible to
modern ears. The lirst, for instance, states no more than that len acres and
Un roods make ten times as much as one acre and one rood.'' One aim
therefore of notes on these as well as the other propcitions of Book v.
should be to make their purport clearer to the learner by setting them side by
side with the same truths expressed in the much shorter and more familiar
modern (algebraical) notation. In doing so, we shall express magnitudes by
the first letters of the alphabet, a, b c etc., adopting small instead of capital
letters so as to avoid confusion with Euclid's lettering ; and we shall use the
small letters ot, n,p etc to represent integral numbers. Thus ma will always
mean m times a or the m''' multiple of a (counting i . a as the first, i . a as the
second multiple, and so on).

Prop. I then asserts that, if ma, mb, mc etc. be any equimultiples of a, b, i
etc, then

ma*mb-mt+ ...=m (a + i + c+...).

\end{notes}

\end{proposition}

\begin{proposition}
\label{prop:V_2}

\begin{statement}
J/ a first magnitude be the same multiple of a second
that a third is of a fourth, and a fifth also be the same multiple
of the second thai a sixth is of the fourth, the sum of the first
and fifth will also be the same multiple of the second that the
sum of the third and sixth is of the fourth.
\end{statement}

\begin{proof}

Let a first magnitude, AB, be the same multiple of a
second, C, that a third, DE,
is of a fourth, F, and let a  , , b g

fifth, EG, also be the same

multiple of the second, C, that °

a sixth, EH, is of the fourth °'' — — '

F; F

I say that the sum of the

first and fifth, AG, will be the same multiple of the second, C,

that the sum of the third and sixth, DN, is of the fourth, F,

For, since AB is the same multiple of C that DE is of F,
therefore, as many magnitudes as there are in AB equal to C,
so many also are there in D£ equal to F.

For the same reason also,
as many as there are in BG equal to C, so many are there
also in £// equal to F;

therefore, as maay as there are in the whole AG equal to C,
so many also are there in the whole DN equal to F.

Therefore, whatever multiple AG is of C, that multiple
also is DM of F.

Therefore the sum of the first and fifth, AG, is the same
multiple of the second, C, that the sum of the third and sixth,
DH, is of the fourth, F. . ,

Therefore etc.
\end{proof}

\begin{notes}

To find the corresponding formula for the result of this proposition, we
may suppose rt to be the `` second `` magnitude and b the `` fourth.'' If now
the `` first `` magnitude is ma, the `` third `` is, by hypothesis, mb ; and, if the
``fifth `` magnitude is na, the ``si)cth'' is nf>. The proposition then asserts that
ma + na is the same multiple of a that mh- nb's of *.

More generally, if /a, ya... and j>i, gfi... be any further equimultiples of
a, b respectively, ma + na--pa-yqa-     is the same multiple of a that mb-y
fib+pb -yqb -y- ,., ts of b. This extension is stated in Simson's corollary to
V. a thus ;

`` From this it is plain that, if any number of magnitudes AB, BG., GH
be multiples of another C; and as many DE, EK, KL be the same
multiples of F, each of each ; the whole of the first, viz. AH, is the same
multiple of C that the whole of the last, viz. DL, is of F''

The course of the proof, which separates m into its units and also n into
its units, practically tells us that the multiple of a arrived at by adding the
two multiples is the (« + n)th multiple ; or practically we are shown that

i»«n- «(j = (w 4 n) a,
or, more generally, that

ma + /la +pa + ... =(»i + 11 +/ + , , ,) o.

\end{notes}

\end{proposition}

\begin{proposition}
\label{prop:V_3,   ..»>•< }

\begin{statement}
J/ a first magnitude be the same multiple of a second
that a third is of a fourth, and if equimultiples be taken
of the first and third, then also ex aequali the magnitudes
taken will be equimultiples respectively, the one of the second
and the other of the fourth.
\end{statement}

\begin{proof}

Let a first magnitude A be the same multiple of a second
B that a third C is of a fourth D, and let equimultiples MF,
GH be taken oi A, C;
I say that £F is the same multiple of S that G/f is of D.

For, since £F is the same multiple of A that G// is of C,
therefore, as many magnitudes as there are in £F equal to
A, so many also are there in G// equal to C.

Let £F be divided into the magnitudes £JC, KF equal
to A, and GH mXo the magnitudes GL, LH equal to C

then the multitude of the magnitudes EK, /will be equal
to the multitude of the magnitudes GL, LH.

A-
B-

E-
C-
D -
O-

And, since A is the same multiple of B that C is of /),

while EK is equal to A, and GL to C

therefore EK is the same multiple of B that GZ, is of D.
For the same reason ,'

KF is the same multiple of B that LH is of Z?.

Since, then, a first magnitude EK is the same multiple
of a second B that a third GL is of a fourth D,
and a fifth KF is also the same multiple of the second B that
a sixth LH is of the fourth D,

therefore the sum of the first and fifth, EF, is also the same
multiple of the second B that the sum of the third and sixth,
GH, is of the fourth Z>. [v. 2]

Therefore etc,

Q, E. D,
\end{proof}

\begin{notes}

Heiberg remarks of the use of ex aeqiiali in the enunciation of this projK)-
sition that, strictly speaking, it has no reference to the definition (17) of a
ratio fx atquaU. But the uses of the expression here and in the definition
are, I think, sufficiently parallel, as may be seen thus. The proposition
asserts that, if

na, nb are equimultiples of a, b,
and if m .na, m . nh are equimultiples of na, nb,

then M , na is the same multiple of a that m .nils of *. Clearly the proposi-
tion can be extended by taking further equimultiples of the last equimultiples
and so on ; and we can prove that

p .f...t!t.nais the same multiple o( a that/ ,q...m.nb is of ,
where the series of numbers p .q...m .n is exactly the same in both
expression ;

and tx atquali (' laov) expresses the fact that the equimultiples are at the
same dhtanee from 3,  in the series na, m .na... and nb, m.nb... respectively.

Here again the proof breaks m into its units, and then breaks n into its
units ; and we are ptacticalty shown that the multiple of a arrived at, viz.
m . rta, is the multiptt: denoted by the product of the numbers m, h, Le. the
(m«)th multiple, or in other words that

\end{notes}

\end{proposition}

\begin{proposition}
\label{prop:V_4}

\begin{statement}
// a first magnitude have to a second the same ratio as a
third to a fourth, any equimultiples whatever of the first and
third will also have the sanu ratio to any equimultiples
whatever of the second and fourth respectively, taken in
corresponding order.
\end{statement}

\begin{proof}

For let a first magnitude A have to a second B the same
ratio as a third C to a fourth D \ and let equimultiples E, F
be taken of A, C, and G, H other, chance, equimultiples of
B,D
I say that, as E is to G, so is Flo H. ,

A      •'

B

E '

• Q .   I

K 1

M- 1 1

C— — -
D-

F 1

  . ... L 1

N . 1

For let equimultiples A', Z. be taken of E, F, and other,
chance, equimultiples M, JV of G, H.

Since E is the same multiple of A that F is of C,

and equimultiples K, L oi E, ./have been taken,

therefore K is the same multiple of A that L is of C, [v. 3]

For the same reason

J/ is the same multiple of B that A'' is of Z?,

`` And, since, as A is to B,so  C to D, • • - • '
and oi A, C equimultiples K, L have been taken,
and of , D other, chance, equimultiples M, N,
therefore, if K is in excess of M, L also is in excess of N
if it is equa], equal, and if less, less, [v. Def, 5]

And K, L are equimultiples of E, F,
and My TV other, chance, equimultiples of G, H
therefore, as £'' is to G, so is F to H. [v. Def. 5]

Therefore etc.
\end{proof}

\begin{notes}

This proposition shows that, if a, b, (, d are proportionals, then
«ki is to n as Mf is to nd; w. -t :

and the proof is as follows :

Take pma, pmc any equimultiples of ma, mc, and qnb, qnd any equimulti-
ples of «*, nd.

Since a : i= : rf, it follows [v. Def. 5] that,

according as pma  > = < gnb, pmc-> = < qnd.
But the and -equimultiples are any equimultiples; ,

therefore [v. Def. 5] . „ , -

ma : nd = me : nd.

It will be observed that Euclid's phrase for taking any equimultiples of
A, C and any other equimultiples of .5, Z> is `` let there be taken equimulti-
ples E, F o( A, C, and G, H other, chance, equimultiples of B, D,'' E, F
being called aaXK woUaTAoirui simply, and G, H oAa, a v•)tv, EiriutK
iroAAairAao-ia. And similarly, when any equimultiples (/T, L) of E, F
come to be taken, and any other equimultiples (M, N) of G, H. But
later on Euclid uses the same phrases about the nrtu equimultiples with
reference to the original magnitudes, reciting that `` there have been taken, of
A C, equimultiples K, L and of B, D, other, chamt, equimultiples M, JV'' ;
whereas M, JV are not any equimultiples whatever of B, D, but are any
equimultiples o( the parlicu/ar multiples (G, //) which have been taken of £,
D respectively, though these tatter have been taken at random. Simson would,
in the first place, add 5 trvytv in the passages where any equimultiples E, F
are taken of A, C and any equimultiples A', £ are taken of E, F, because the
words are ``wholly necessary'' and, in the second place, would leave them
out where M, iVare called oAAn, a inxty, liraKii TrokJuurKaria of B, D, because
it is not true that of B D have been taken ``any equimultiples whatever («
hvyt), M, N.'' Simson adds: ``And it is strange that neither Mr Bris, who
did right to leave out these words in one place of Prop. 13 of this book, nor
Dr Gregory, who changed them into the word some in three places, and
left them out in a fourth of that same Prop, 13, did not also leave them out
in this place of Prop. 4 and in the second of the two places where they occur
in Prop. 1 7 of this book, in neither of which they can stand consistent with
truth : And in none of all these places, even in those which they corrected in
their Latin translation, have they cancelled the words « hvx' '`` 'he Greek
text, as they ought to have done. The same words S. iToj(i are found in
four places of Prop. 1 1 or this book, in the first and last of which they are
necessary, but in the second and third, though they are true, they are quite
superfluous ; as they likewise are in the second of the two places in which
they are found in the 12th prop, and in the like places of Prop. 21, 33 of this
book; but are wanting in the last place of Prop. 13, as also in Prop, 25,
Book XI,''

As will be seen, Sirason's emendations amount to alterations of the text
so considerable as to suggest doubt whether we should be justifled in making
them in the absence of MS, authority. The phrase `` equimultiples of A, C
and other, chance, equimultiples of £, D `` recurs so constantly as to suggest
that it was for Euclid a quasi-stereotyped phrase, and that it is equally genuine
wherever it occurs. Is it then absolutely necessary to insert i trvxt in places
where it does not occur, and to leave it out in the places where Simson holds
it to be wrong ? I think the text can be defended as it stands. In the first
place to say ``take equimultiples of A, C'' is 3. fair enough way of saying
take any equimultiples whatevtr of A, C. The other difliculty is greater, but
may, I think, be only due to the adoption of any whatever as the translation
of a, Xrv-jif. As a matter of fact, the words only mean chance equimultiples,
equimultiples which are the result of random selection. Is it not justifiable
to describie the product of two chance numbers, numbers selected at random,
as being a `` ekance number,'' since it is the result of two random selections ?
1 think so, and I have translated <i cruxc accordingly as implying, in the case
in question, `` other equimultiples whatever they may happen to be,''

To this proposition Theon added the following :

`` Since then it was proved that, if K is in excess of M, L is also in excess
of N, if it is equal, (the other is) equal, and if less, less,
it is dear also that,

if  is in excess of A!*, A'' is also in excess of Z, if it is equal, (the other is)
equal, and if less, less ;
and foi this reason,

as C is to £, so also is HXa F.

PosiSM. From this it is manifest that, if four magnitudes be proportional,
they will also be proportional inversely.''

Simson rightly pointed out that the demonstration of what Theon intended
to prove, viz. that, if E, G, F, H be proportionals, they are proportional
inversely, i.e. (7 is to .£ as /T is to / does not in Che least depend upon this
4th proposition or the proof of it ; for, when it is said that, `` if A'' exceeds M,
Z also exceeds N etc.,'' this is not proved from the fact that E, G, F, H are
proportionals (which is the conclusion of Prop. 4), but from the fact that
A, B, C, D are proportionals.

The proposition that, if A, B, C, D are proportionals, they are also
proportionals inversely is not given by Euclid, but Simson supplies the proof
in his Prop. B. The fact is really obvious at once from the 5th definition
of Book V. (cf, p. 127 above), and Euclid probably omitted the proposition
as unnecessary.

Simson added, in place of Theon's corollary, the following :

`` Likewise, if the first has the same ratio to the second which the third
has to the fourth, then also any equimultiples whatever of the first and third
have the same ratio to the second and fourth : And, in like manner, the first
and the third have the same ratio to any equimultiples whatever of the second
and fourth,''
The proof, of course, Tollows exactly the method of Euclid's proposition
itself, with the only difference that, instead of one of the two pairs of equi-
multiples, the magnitudes themselves are taken. In other words, the conclu-
sion that

MM is to n as KM' is to m i •  •i

is equally true *hen either « or « is equal to unity.

As De Morgan says, Simson's corollary is only necessary to those who will
not admit jl/'into the list M, lAf, 3 J/' etc.; the exclusion is grammatical and
nothing else. The same may be said of Simson's Prop. A to the effect that,
`` If the first of four magnitudes has to the secotid the same ratio which the
third has to the fourth : then, if the first be greater than the second, the third
is also greater than the fourth ; and if equal, equal ; if less, less.'' This is
needless to those who believe ona A to be a proper component of the list of
multiples, in spite of mullut signifying many.

\end{notes}

\end{proposition}

\begin{proposition}
\label{prop:V_5}

\begin{statement}
1/ a magnitude be the same multiple of a magnitude that
a part subtracted is of a part subtracted, the remainder will
also be the same multiple of the remainder that the whole is of
the whole.
\end{statement}

\begin{proof}

S For let the magnitude AB ht. the same multiple of the
magnitude CD that the part AE subtracted is of the part CF
subtracted ;

I say that the remainder EB is also the same multiple of the
remainder ED that the whole AB is of the whole CD.

: . , . — 1 1 ? 1 1 ?

10 For, whatever multiple AE is uf CF, let EB be made
that multiple of CG.

Then, since AE is the same multiple of CF that EB ts
of GC,
therefore AE is the same multiple of CF that AB is of GF,

[V. .]

IS But, by the assumption, AE is the same multiple of CF
that AB is of CD.

Therefore AB is the same multiple of each of the magni-
tudes GF, CD ;

therefore GF is equal to CD.
» Let CF be subtracted from each ;
therefore the remainder GC is equal to the remainder FD.

And, since AE is the same multiple of CF that EB is of
GC,

and GC is equal to DF,
as therefore AE is the same multiple of CF that 5 is of FD.
But, by hypothesis,

AE is the same multiple of CF that -(4 is of CD ;
therefore EB is the same multiple of FD that - is of CZ?,
That is, the remainder EB will be the same multiple of
30 the remainder FD that the whole AB is of the whole CD.
Therefore etc.

Q, E. D,
\end{proof}

\begin{annotations}

10. let EB be made that muJtiple of CO, ntavn-rKinaf ytYmirw rol to EB tw
rB. From this way of stating; the construction one mit suppose that CG was given and
EB had to be found equal to a certain multiple of it. But in fact EB ia what is given and
CG has to be found, i.e. CG has to be constructed as a certain juuttiplc of EB*

\end{annotations}

\begin{notes}

This proposition correspotids to V. i, with subtraction taking the place of
addition. It proves the foimula

ma~mb = M(a—b). '" '

Euclid's construction assumes that, ii AE  any multiple of CF and EB
is any other magnitude, a fourth straight line can be found such that EB is
the same multiple of it that AE is of CF, or in other words that, given any
magnitude, we can divide it into any number of equal parts. This is however
not proved, even of straight lines, much less other magnitudes, until vi. 9.
Peletarius had already seen this objection to the construction. The difficulty
is not got over by regarding it merely as a hyfothetkal construction ; for
hypothetical constructions are not in Euclid's manner. The remedy is to
substitute the alternative construction given by Simson, after Peletarius and
Campanus' translation from the Arabic, which only requires us to add a
magnitude to itself a certain number of times. The demonstration follows
Euclid's line exactly.

``Take AG the same multiple of FD that AE is of CF;

therefore AE is the same multiple of Cthat EG is of CD.

But AE, by hypothesis, is the same multiple of CF that
. is of CD ; therefore EG is the same multtple of CD that
A£ is of CD;

wherefore C is equal to AJ3.

Take from them the common magnitude AE ; the remainder
AG is equal to the remainder EB.

Wherefore, since AE is the same multiple of CFthu AG is
of FD, and since AG is eqaa.1 to EB,
therefore AE is the same multiple of CFtitat EB is o FD,

But AE is the same multiple of Cthat AB is of CD;
therefore EB is the same multiple of FD that AB is of CD.''

Euclid's proof amounts to thb.

Suppose a magnitude x taken such that  •  '

ma — mimx, say.
Add TTii to each side, whence (by v. i)

Therefore a=jf + , or *=a — *, *' '

so that   mi-tni> = m(a-b). '

Simson's proof, on the Other hand, argues thus.

Take x = m(a~ i), the same multiple of (a — i) that md is ol d.

Then, by addition of mfi to both sides, we have [v. i]
x + m6 = ma,
or at = ma — mb. • •

That ia, ma~mi = m(a — i).

\end{notes}

\end{proposition}

\begin{proposition}
\label{prop:V_6}

\begin{statement}
If (wo magnitudes be equimultiples 0/ two magnitudes, and
any magnitudes subtracted from them 6e equimultiples of the
sam4, the remainders also are either equal to the same or equi-
multiples of them.
\end{statement}

\begin{proof}

For let two magnitudes AB, CD be equimultiples of two
magnitudes E, F, and let AG, CH

subtracted from them be equi- a q 8

multiples of the same two E, F; '

I say that the remainders also, GB,  o h

HD, are either equal to E, F or — *  — 1 — >- — 1 —
equimultiples of them. F —

For, first, let GB be equal to  ;
I say that HD is also equal to F.

For let CK be made equal to F.

Since AG  the same multiple of E that CH is of F,
while GB is equal to E and KC to F,
therefore AB is the same multiple of E that KH is of F.

[v. 2]

But, by hypothesis, AB is the same multiple of E that
CDvQiF;
therefore KH is the same multiple of F that CD is of F.

Since then each of the magnitudes KH, CD is the same
multiple of F,

therefore KH ts equal to CD,

Let C// be subtracted from each ;
therefore the remainder A'C is equal to the remainder //D.

But F is equal to JCC ;
therefore //D is also equal to J.

Hence, if GB is equal to £, HD is also equal to F.

Similarly we can prove that, even if GB be a multiple
of B, HD is also the same multiple of F.

Therefore etc
\end{proof}

\begin{notes}

This proposition corresponds to v. 3, with subtraction taking the place of
addition. It asserts namel)' that, if n \ less than m, ma — na is the same
multiple of a that mb-nb'  of b. The enunciation distinguishes the cases in
which m-» is equal to i and greater than i respectively.

Simson observes that, while only the first case the simpler one) is proived
in the Greek, both are given in the Latin translation from the Arabic ; and
he supplies accordingly the proof of the second case, which Euclid leaves to
the reader. The fact is that it is exactly the same as the other except that, in
the construction, CK is made the same multiple of /''that GB is of E, and
at the end, when it has been proved that KC is equal to HD, instead of
concluding that'/fZ> is equal to F, we have to say `` Because GB is the same
multiple of E that KC is of F, and KC is equal to HD, therefore HD is
the same multiple of ./that GB is of E.''

\end{notes}

\end{proposition}

\begin{proposition}
\label{prop:V_7}

\begin{statement}
Equal magnitudes have to (he same the same ratw, as also
has the same io equal magnitudes.
\end{statement}

\begin{proof}

Let A, B be equal magnitudes and C any other, chance,
magnitude ;

I say that each of the magnitudes A, B has the same ratio
to C, and C has the same ratio to each of the magnitudes
A,B.

A D

B Er.

Ci < f-

For let equimultiples D, E o A, B be taken, and of C
another, chance, multiple F.

Then, since D is the same multiple of A that E is of B,
while A is equal to ,

therefore D is equal to E.

But F is another, chance, magnitude.

V. 7. 8] PROPOSITIONS 6—8 149

If therefore D is in excess of F, E is also in excess of F,
if equal to it, equal ; and, if Jess, less.

And D, E are equimultiples oi A, B,
while F'ls another, chance, multiple of C;

therefore, as A is to C, so is B to C, [v. Def. 5]

I say next that C also has the same ratio to each of the
magnitudes A, B.

For, with the same construction, we can prove similarly
that D is equal X,o E

and F is some other magnitude.

If therefore F n excess of D, it is also in excess of E,
if equal, equal ; and, if less, less.

And / is a multiple of C, while D, E are other, chance,
equimultiples of , B ;

therefore, as C is to -r , so is C to B. [v. Def. 5]

Therefore etc.

PoRiSM, From this it is manifest that, if any magnitudes
are proportional, they will also be proportional inversely.
\end{proof}

\begin{notes}

In this proposition there is a similar use of t irvxy to that which has
been discussed under Prop. 4. Any multiple F <A C is taken and then,
four lines lower down, we are told that `` F is another, chance, magnitude.''
It is of course not any magnitude whatever, and Simson leaves out the
sentence, but this time without calling attention to it.

Of the Porism to this proposition Heiberg says that it is properly put here
in the best ms.j for, as August had already observed, if it was in its right
place where Theon put it (at the end of v. 4), the second part of the proof of
this proposition would be unnecessary. But the truth is that the Porism is no
more in place here. The most that the proposition proves is that, if A, B
are equal, and Cany other magnitude, then two conclusions are simultaneously
established, (1) that A is to C s.  B is to Cand (2) that C io A . C is to
B. The second conclusion is not established from the first conclusion (as
it ought to be in order to justify the inference in the Porism), but from a
hypothesis on which the first conclusion itself depends ; and moreover it is
not a proportion in its genera! form, i.e. between four magnitudes, that is in
question, but only the particular case in which the consequents are equal.

Aristotle tacitly assumes inversion (combined with the solution of the
problem of Eucl.\ vi. 11) in Meteoroiogiea ni. 5, 37G a 14 — 16.

\end{notes}

\end{proposition}

\begin{proposition}
\label{prop:V_8}

\begin{statement}
Of unequal magnitudes, the greater has to the same a
greater ratio than the /ess has ; and the same has to the less
a greater ratio than it has to the greater.
\end{statement}

\begin{proof}

Let AB, C be unequal magnitudes, and let AB be greater ;

let D be another, chance, - n

magnitude ;

r say that AB has to Z? a

greater ratio than C has to

£), and /? has to C a greater

ratio than it has to AB.

For, since  is greater
than C, let BB be made equal
toC;

then the less of the magni-
tudes A£, JSB, if multiphed,
will sometime be greater than £f. [v, Def. 4]

ICase I.]

First, let A£ be less than BB ;

let AB he multiplied, and let BG be a multiple of it which is
greater than D ;

then, whatever multiple BG is of AB, let G/f be made the
same multiple of BB and  of C ;

and let Z be taken double of D, M triple of it, and successive
multiples increasing by one, until what is taken is a multiple
of D and the first that is greater than K, Let it be taken,
and let it be N which is quadruple of D and the first
multiple of it that is greater than a.

Then, since K is less than N first,

therefore K is not less than M.

And, since BG is the same multiple of AB that G/f is of
BB,

therefore BG is the same multiple of AB that B/f is of AB.

[V. ,]

But BG is the same multiple of AB that A' is of C ;

therefore F/f is the same multiple of AB that A' is of C ;

therefore B//, K are equimultiples of AB, C.

Again, since GH is the same multiple of EB that K is
of C

and EB is equal to C,

therefore GH is equal to A'.

But K is not less than M;   - s

therefore neither is (7/ less than j)/. .1 .• • • 1

And FG is greater than Z? ;
therefore the whole Jff is greater than Z>, M together.

But D, M together are equal to A', inasmuch as M is
triple of D, and M, D together are quadruple of D, while
A'' is also quadruple of D ; whence M, D together are equal
o N.

But FH is greater than M, D ;

therefore / is in excess of A,   ``-•-»''•

while K is not in excess of N. '

And FH, K are equimultiples of AB, C, while N is
another, chance, multiple of D ;

therefore AB has to D a. greater ratio than C has to D.

[v. Def. 7]

I say next, that D also has to C a greater ratio than D
has to AB.

For, with the same construction, we can prove similarly
that TV is in excess of K, while N is not in excess of FH,

And A is a multiple of/?,
while FH, K are other, chance, equimultiples of AB, C ;

therefore D has to C a greater ratio than D has to AB.

[v, Def. 7]

Case 2.]

Again, let AE be greater than EB.

Then the less, EB, if multiplied, will sometime be greater
than D, y. Def. 4]

Let it be multiplied, and e q

let GH be a multiple of EB * '

and greater than D ;  '

and, whatever multiple GH is '

of EB, let FG be made the ``  — '

same multiple of AE, and K  • •

of C L    ..

Then we can prove simi- `` '

larly that FH, K are equi- `` •

multiples of AB, C

and, similarly, let N be taken a multiple of D but the first

that is greater than FG,

so that FG is again not less than M.

But GH is greater than D ;
therefore the whole FH is in excess of D, M, that is, of N.

Now A' is not in excess of N', inasmuch as FG also, which
is greater than GH, that is, than K, is not in excess of N.

And in the same manner, by following the above argu-
ment, we complete the demonstration.

Therefore etc.

Q. E. I).
\end{proof}

\begin{notes}

The two separate cases found in the Greek text of the demonstration can
practically be compressed into one. Also the expositor of the two cases
makes them differ more than they need. It is necessary in each case to
select the smaller of the two segments AE, EB of AB with a view to taking
a multiple of it which is greater than D ; in the first case therefore A£ is
taken, in the second EB. But, while in the first case successive multiples of
D are taken in order to find the first multiple that is greater than (7 (or A''),
In the second case the multiple is taken which is the first that is greater than
EG. This difference is not necessary; the first multiple of Z) that is greater
than G/f would equally serve in the second case. Lastly, the use of the
magnitude /C might have been dispensed with in both cases ; it is of no
practical use and only lengthens the proofs. For these reasons Simson
considers that Theon, or some other unskilful editor, has vitiated the
proposition. This however seems an unsafe assumption ; for, while it was
not the habit of the great C J reek geometers to discuss separately a number of
different cases (eg, in i, 7 and t. 35 Euclid proves one case and leaves the
others to the reader), there are many exceptions to prove the rule, e.g. Eucl,
III. 15 and 33 ; and we know that many fundamental propositions, after-
wards proved generally, were first discovered in relation to particular cases
and then generalised, so that Book v., presenting a comparatively new
theory, might fairly be expected to exhibit more instances than the earlier
books do of unnecessary subdivision. The use of the JC is no more con-
clusive against the genuineness of the proofs.

Nevertheless Simson 's version of the proof Is certaimy snorter, and more-
over it takes account of the case in which AE is efua/ to £B, and of the case
in which AE, EB are both greater than D (though these cases are scarcely
worth separate mention).

`` If the magnitude which is not the greater of the two AE, EB be (i)
not less than D, take FG, G/f the doubles of AE, EB.

But if that which is not the greater of the two AE, EB be (2) less than
£>, this magnitude can be multiplied so as to become greater than £> whether
it he AE or EB,

Let it be multiplied until it becomes greater than D, and let the other be
multiplied as often ; let EG be the multiple thus taken of AE and GJ/ the
same multiple of EB ,
therefore EG and G/f are each of them greater than D,

And, in every one of the cases, take Z the double of D, M its triple and
so on, till the multiple of Z) be that which first becomes greater than GH.

Let N be that multiple of D which is first greater than ff/ and jthe
multiple of D which is next less than N.

Then, because iV is the multiple of J) which is the first that becomes
greater than GJf,      :. .

the next preceding tnultipU is not greater than G/f;
that is, Gff is not less than M.

And, since FG is the same multiple of AE that G/f is of EB,
GH'is ttie same multiple of EB that FH moi AB\ [v. 1]

wherefore FH, C/f are equimultiples of AB, EB.

And it was shown that Glfas not less than Af;

and, by the construction, FG is greater than D
therefore the whole FHs greater than M, D together. ,

But M, D together are etjual to N ;
therefore FH  greater than N. ,

But Gff'ss not greater than N;
and FH, GHa-tn equimultiples of AB, BE,

and jVis a multiple of D
therefore AB has to /? a greater ratio than BE (or C) has to D. [v. Def, 7]

Also D has to BE a greater ratio than it has to A/i.

For, having made the same construction, it may be shown, in like manner,
that N is greater than GH but that it is not greater than FH;
and TV is a multiple of D, 1 . ; .

and GH, FH zxa equimultiples of EB, AB;

Therefore Dhas,to EB a greater ratio than it has to AB.'' [v, Def. 7]

The proof may perhaps be more readily grasped in the more symbolical
form thus.

Take the w;th equimultiples of C, and of the excess of AB over C (that is,
oi AE), such that each is greater than D

and, of the multiples of/?, let i? be the first that is greater than mC, and nD
the ne<;t less multiple of D.

Then, since wC is not less than nD, • • * '

and, by the construction, m(AE) is greater than D,

the sum of wCand rii(AE) is greater than the sum of aD and D.

That is, m(AB) is greater than/Z>. ...

And, by the construction, mC is less than pD.

Therefore [v. Def. j] AB has to D a. greater ratio than C has to D.

Again, since //J is less than m(AB),

i.nApD is greater than mC, <'-   < •' ``> '

D has to C a greater ratio than D has to AB. ''    '

• *' '.

\end{notes}

\end{proposition}

\begin{proposition}
\label{prop:V_9}

\begin{statement}
Magnitudes which have the same ratio to the same are
equal to one another ,  and magnitudes to which the same has
the same ratio are equal.
\end{statement}

\begin{proof}

For let each of the magnitudes , B have the same
ratio to C ;
I say that A is equal to B.

For, otherwise, each of the
magnitudes A, B would not °

have had the same ratio to C\ [v. 8]

but it has ;

therefore A is equal to B.

Again, let C have the same ratio to each of the magni-
tudes A, B ;
I say that A is equal to B.

F'or, otherwise, C would not have had the same ratio to
each of the magnitudes A, B \ y-

but it has ;

therefore A is equal to B.

Therefore etc.
\end{proof}

\begin{notes}

If ii is to C as B is to C,
or if C is to V* as C is to B, then A is equal to B.

Simson gives a more expticit proof of this proposition which has the
advantage of referring back to the fundamental sth and 7th definitions,
instead of quoting the results of previous projxjsitions, which, as will be seen
from the next note, may be, in the circumstances, unsafe.

`` Let A, B have each of them the same ratio to C

A is equal to B.

For, if they are not equal, one of them is greater than the other ;
let A be the greater.

Then, by what was shown in the preceding proposition, there are some
equimultiples of A and B, and some multiple of C, such that the multiple of
A is greater than the multiple of C, but the multiple of B is not greater than
that of C.

Let such multiples be taken, and let ZJ, j£ be the equimultiples of A, B,
and F the multiple of C, so that D may be greater than F, and E not greater
than F.

But, because vf is to C as  is to C,
and of A, B are taken equimultiples D, £, and of C is taken a multiple F,
and I> is greater than F,

E must also be greater than F. [v. Def. s]

But £ is not greater than F: which is impossible.

Next, let C have the same ratio to each of the magnitudes A and S ;
A is equal to B,

For, if not, one of them is greater than the other ;
let A be the greater.

Therefore, as was shown in Prop. 8, there is some multiple F of C, and
some equimultiples E and D al B and A, such that F is greater than E and
not greater than D.

But, because C is to - as C is to A,
and /''the multiple of the first is greater than E the multiple of the second,

the multiple of the third is greater than D the multiple of the fourth.

[v, Def, 5]

But  is not greater than D : which is impossible.

Therefore A is equal to B.''

\end{notes}

\end{proposition}

\begin{proposition}
\label{prop:V_id. ,.. . ,}

\begin{statement}
0/ magnitudes which have a ratio to (fie same, that
which has a greater ratio is greater ; and that to which the
same has a greater ratio is less.
\end{statement}

\begin{proof}

For let A have to C a greater ratio than B has to C ;
I say that A is greater than B.

For, if not, A is either equal to B or less,

Now /4 is not equal to B
for in that case each of the magnitudes A, B would have
had the same ratio to C ; [v. 7]

but they have not ;

therefore A is not equal to B.

Nor again is A less than B ;
for in that case A would have had to C a less ratio than B
has to C ; [v. 8]

but it has not ;

therefore A is not less than B.

But it was proved not to be equal either ;
therefore A is greater than B.

Again, let C have to  a greater ratio than C has to A ;
I say that B is less than A.

For, if not, it is either equal or greater. '

Now B is not equal to A ;
for in that case C would have had the same ratio to each of
the magnitudes A, B ; [v. 7]

but it has not ;

therefore A is not equal to B.

tSU BOOK V [v. lo

Nor again is  greater than W ; *  • -> >

for in that case C would have had to  a less ratio than it
has to A ; [v. 8]

but it has not ; •' •'=

therefore B is not greater than A.
But it was proved that it is not equal either ;

therefore B is less than A.
Therefore etc. Q.E.D.-
\end{proof}

\begin{notes}

No better example can, I think, be found of the acuteness which Simson
brought to bear in his critical examination of the £mcnfs, and of his great
services to the study of Euclid, than is furnished by the admirable note on
this proposition where he points out a serious flaw in the proof as given in
the text.

For the Rrst time Euclid is arguing about greater and ieis ratios, and it
will be found by an examination of the steps of the proof that he assumes
more with regard to the meaning of the terms than he is entitled to assume,
having regard to the fact that the definition of greater ratio (Def, 7) is all
that, as yet, he has to go upon. That we cannot argue, at present, about
greaifr and less as applied to mtwi m the same way as about the same terms
in relation to nrngniludes is indeed sufficiently indicated by the fact that Euclid
does not assume for ratios what is in Book i. an axiom, viz. th;it things which
are equal to the same thing are equal to one another ; on the contrary, he
proves, in Prop. 11, that ratios which are the same with the same ratio are the
same with one another.

Let us now examine the steps of the proof in the text. First we are told
that

``j4 is greater than B.

For, if not, it is either equal to B or less than it.

Now jJ is not equal to B ;

for in that case each of the two magnitudes A, B would have had the
same ratio to C: [v. 7]

but they have not :

therefore A is not equal to B''

As Simson remarks, the force of this reasoning is as follows.

If A has to C the same ratio as B has to C,
then — supposing any equimultiples of A, B to be taken and any multiple
of C—

by Def. 5, if the multiple of /i be greater than the multiple of C, the multiple
of B is also greater than that of C.

But it follows from the hypothesis (that  has a greater ratio to C than B
has to C) that,

by Def. 7, there must be some equimultiples of A, B anA somt multiple of
C such that the multiple of  is greater than the multiple of C, but the
multiple of B is not greater than the same multiple of C.

And this directly contradicts the preceding deduction from the supposition
that A has to C the same ratio as B has to C ;

therefore that supposition is impossible.

The proof now goes on thus :

*' Nor again is A less than B ;
for, in (hat case, A would have had to C a less ratio than B has to C

' but tt has not ;

therefore A is not less than B.''

It is here that the difficulty arises. As before, we must use Def. 7. ``A
would have had to C a less ratio than B has to C,'' or the equivalent state-
ment that B would have had to C a greater ratio than A has to C, means
that there would have been same equimultiples of B, A and some multiple of
C such that

(i) the multiple of B k greater than the multiple of C, but

(z) the multiple of .1 is nat greater than the multiple of C,
and it ought to have been proved that this can never happen if the hypothesis
of the proposition is true, vh. that A has to C a greater ratio than B has to
C: that is, it should have been proved that, in the latter case, the multiple of
A is always greater than the multiple of C whenever the multiple of B is
greater than the multiple of C (for, when this is demonstrated, it will be
evident that B cannot have a greater ratio to C than A has to C). But this
is not proved (cf. the remark of De Morgan quoted in the note on v, Def 7,
p. 130), and hence it is not proved that the above inference from the supi>osi-
tion that A is less than B is inconsistent with the hypothesis in the enunciation.
The proof therefore fails.

Simson suggests that the proof is not Euclid's, but the work of some one
who apparently ``has been deceived in applying what is manifest, when
understood of magnitudes, unto ratios, viz. that a magnitude cannot be both
greater and less than another,''

The proof substituted by Simson is satisfactory and simple. . , i , r

``Let A have to Ca greater ratio than B has to C;
A is greater than B.

For, because A has a greater ratio to C than B has to C, there are some
equimultiples of A, B and some multiple of C such that

the multiple of A is greater than the multiple of C, but the multiple of B
is not greater than it. [v. Def. 7]

Let them be taken, and let D, E y equimultiples of A, B, and F a
multiple of C, such that

 , D  greater than F

bilt £ is not greater than F. «

Therefore D is greater than E.

And, because D and E are equimultiples of A and B, and D is greater
than E,

therefore A is greater than B. [Simson's 4th Ax.]

Next, let C have a greater ratio to B than it has a A
B is less than A.

For there is some multiple F of C and some equimultiples E and D ai B
and A such that

is greater than E but not greater than D. [v. Def. 7]

Therefore E is less than D ;
and, because E and D are equimultiples of B and A, :  -i -m '

therefore B is less than A.''

\end{notes}

\end{proposition}

\begin{proposition}
\label{prop:V_11}

\begin{statement}
Ratios which are the same with the same ratio are also
the same with one another.
\end{statement}

\begin{proof}

For, as W is to B, so let C be to jD,
and, as C is to D, so let  be to A; '

I say that, as A is to B, so   E Xo F. , .'i

For of Ay C, E let equimultiples G, H, K be taken, and
oi B, D, Mother, chance, equimultiples L, M, N.

Then since, as 4 is to B, so is C to D,
and of y4, C equimultiples G, /Thave been taken,
and of B, D other, chance, equimultiples L, M,
therefore, if C is in excess of Z, H s also in excess of M,
if equal, equal,
and if less, less.

Again, since, as C is to D, so is E to E,
and of C, E equimultiples H, K have been taken,
and of D, F other, chance, equimultiples M, N,
therefore, if H is in excess of M, K is also in excess of N,
if equal, equal,
and if less, less. •.     -i

But we saw that, if H was in excess of M, G was also
in excess of Z- ; if equal, equal ; and if less, less ;
so that, in addition, if G is in excess of Z, K is also in excess
kAN,

if equal, equal, „ , , ,. ,., „ ,, . ,,  .

and if less, less.

And G, K are equimultiples oi A, £,
while L, N are other, chance, equimultiples of B, F;

therefore, as A is to B, so is E to E. -i

Therefore etc, '
\end{proof}

\begin{notes}

Algebraically, if - a -.b-c: d, .«

atid cd=tf,

then a:b = ef.

The idiomatic use of the imperfect in quoting a result previously obtained
is noteworthy. Instead of saying `` But it was proved that, if H is in excess
of M, G is also in excess of L,'' the Greek text has ``But if H was in excess
of M, G was also in excess of L,'' oXAa tl vittfulyft to © rov M, hvipaxt noi

TO H ToZ A.

This proposition is tacitly used in combination with V. i6 and v. 14 in the
geometrical passage in Aristotle, Miteorohgica 111. 5, 376 a a 2 — *6, j

\end{notes}

\end{proposition}

\begin{proposition}
\label{prop:V_12}

\begin{statement}
If any number of magnitudes be proportional, as one of
\end{statement}

\begin{proof}

the antecedents is to one of the consequents, so will all the
antecedents be to all the consequents.

Let any number of magnitudes A, B, C, D, E, F be
proportional, so that, as A is to B, so is C xo D and E
to F
I say that, as A is to B, so are A, C, E to B, D, F.

fi.

For o( A, C, E let equimultiples G, N, K be taken,
and q( B, D, i other, chance, equimultiples L, M, N.

Then since, as A is to B, so is C to D, and E to F,
and of A, C, E equimultiples G, H, K have been taken,
and of B, D, F other, chance, equimultiples L, M, N,
therefore, if G is in excess of Z, / is also in excess of M,
andofiV,
if equal, equal,
and if less, less ;
so that, in addition,

if G is in excess of L, then G H, K are in excess of Z,, M, N,
if equal, equal,
and if less, less, • i -

Now G and G, H, K are equimultiples of A and A, C, E,
since, if any number of magnitudes whatever are respec-
tively equimultiples of any magnitudes equal in multitude,
whatever multiple one of the magnitudes is of one, that
multiple also will all be of all. [v. i]

For the same reason

L and L, M, N are also equimultiples of B and B, D, F

therefore, as A is to B, so are A, C, E lo B, D, P.

[v. Def. s]
Therefore etc.
\end{proof}

\begin{notes}

Algebraically, i a : a' = b : b' = e : / eic, each ratio is equal lo the ratio
(a + 6 + e+ ...) : (a' +'4-+ ...).

This theorem is quoted hy Aristotle, El A, Nk. v. 7, 1 131 b 14, in the
shortened form ``the whole is to the whole what each part is to each part
(respectively).''

\end{notes}

\end{proposition}

\begin{proposition}
\label{prop:V_13}

\begin{statement}
If a first magnitude have to a second the same ratio as a
third to a fourth, and the third have to the fourik a greater
ratio than a fifth has to a sixth, the first will also have to the
second a greater ratio than the fifth to (he sixth.
\end{statement}

\begin{proof}

For let a first magnitude A have to a second B the
same ratio as a third C has to a fourth D,

and let the third C have to the fourth D a greater ratio than
a fifth E has to a sixth E;

I say that the first A will also have to the second B a greater
ratio than the fifth E to the sixth E.

f, o   M-

B — D N

E-
F-

L-

For, since there are some equimultiples of C, E,

and of D, E other, chance, equimultiples, such that the
multiple of C is in excess of the multiple of D,

V. 13) PROPOSITIONS ii, 13 xBi

while the multiple of £ is not in excess of the multiple of P,

[v. Def.. 7]
let them be taken,

and let G, Hhe. equimultiples of C, E,

and K, L other, chance, equimultiples of D, F,

so that G is in excess of K, but H is not in excess of L ;

and, whatever multiple G is of C, let M be also that multiple

of ,

and, whatever multiple K is of D, let N be also that multiple

of.

Now, since, as  is to B, so is C to D,
and Qi A, C equimultiples M, G have been taken,
and of B, D other, chance, equimultiples A, K,
therefore, if M is in excess of N, G is also in excess of K,
if equal, equal,
and if less, less. , .. , . ;, li' [v. Def. 5]

But 6 is in excess of AT ; .' 1. c

therefore M is also in excess of A. '

But H is not in excess of Z ; •

and M, /are equimultiples of , E, .,

and jV, L other, chance, equimultiples of , F

therefore A has to 5 a greater ratio than E has to F.

[v. Def. j]
Therefore etc.
\end{proof}

\begin{notes}

Algebraically, if ah = tdt ~*  •

and e : d->e :/ , ,

then a \ b-rtf.

After the words `` for, since ``in the first line of the proof, 'Hieon added
`` C has to i> a greater ratio than E has to F'' so that `` there are some
equimultiples'' ban, with him, the principal sentence.

The Greek text has after `` of D, F other, chance, equimultiples,'' `` and
the multiple of C is in excess of the multiple of D....'' The meaning being
`` such that,'' I have substituted this for `` and,'' after Simson.

The following will show the method of Euclid's proof.

Since e:d->ef,

there will be some equirnultiples me, nu of t, e, and some equimultiples nd,
oftf,/, auch that

momt, while me'nf.

But, since a:i = c;d, (. ,:/v * •)

therefore, according as ma > = <ni, mc> — < nd.

And mi>»d;
therefore ma -> nb, while (from above) melnf.
Therefore ab>tf.

Simson adds as a corollary the following :

`` If the first Kave a greater ratio to the second than the third has to the
fourth, but the third the same ratio to the fourth which the fifth has to the
sixth, it may be demonstrated in like manner that the first has a greater ratio
to the second than the fifth has to the sixth.''

This however scarcely seems to be worth separate statement, since it only
amounts to changing the order of the two parts of the hypothesis.

\end{notes}

\end{proposition}

\begin{proposition}
\label{prop:V_14}

\begin{statement}
If a first magnitttde have to a second the same ratio as a
third has to a fourth, and the first be greater than the third,
the second will also he greater than the fourth; if equal, equal;
and if less, less.
\end{statement}

\begin{proof}

For let a first magnitude A have the same ratio to a
second . as a third C has to a fourth D\ and let A be
greater than C ;
I say that B Is also greater than D.

A c

8 D

For, since A is greater than C,
and B is another, chance, magnitude,
therefore A has to .5 a greater ratio than C has to B. [v, 8]

But, as  is to B, so is C to Z* ;
therefore C has also to /? a greater ratio than C has to B.

[v. Jl]

But that to which the same has a greater ratio is less ;

[v. 10]
therefore D is less than B ;
so that B is greater than D.

Similarly we can prove that, if .4 be equal to C, B will
also be equal to D ;

and, if A be less than C, B will also be less than D.
Therefore etc.
\end{proof}

\begin{notes}

Algebraically, if a : i = t : d,

then, according aso> = <f, fi> = <<f,
Simson adds the specific proof of the second and third parts of this
proposition, which Euclid dismisses with ``Similarly we can prove....''

`` Secondly, if v be equal to C, B is equal to D\ for /4 is to 5 is C, that
is A, is to Z> ;

therefore B is equal to D. [v. 9]

Thirdly, if 4 be less than C, B shall be less than D,
For C is greater than A ;
and, because C'iXo D »s A is to B,

D is greater than B, by the first case. >, , .

Wherefore B is less than D.''

Aristotle, Mtteorol. iii. 5, 376 a ti— i4i quotes the equivalent proposition
that, if a>, e->d.

\end{notes}

\end{proposition}

\begin{proposition}
\label{prop:V_15}

\begin{statement}
Parts have the same ratio as the same multiples of them
taken in corresponding order.
\end{statement}

\begin{proof}

For let AB be the same multiple of C that DE is of /'';
I say that, as C is to F, sos AB to DE.

Af 1 1 tB Ct-

Oi   'E f'-

For, since AB is the same multiple of C that DE is of E,
as many magnitudes as there are in AB equal to C, so many
are there also in DE equal to E.

Let AB be divided into the magnitudes AG, GH, HB
equal to C,

and DE into the magnitudes DK, KL, LE equal to E
then the multitude of the magnitudes AG, GH, HBW\ be
equal to the multitude of the magnitudes DK, KL, LE.

And, since AG GH, HB are equal to one another,
and DK, KL, LE are also equal to one another,
therefore, as AG is to DK, so is GH to KL, and HB to LE.

(- 7]

Therefore, as one of the antecedents is to one of the
consequents, so will all the antecedents be to all the
consequents ; [v. la]

therefore, as G is to DK, so is AB to DE.

But AG is equal to C and DK to F;

therefore, as C is to i so is AS to IS,
Therefore etc. v,, ,
\end{proof}

\begin{notes}

Algebraically, a : b~ma : mi.

\end{notes}

\end{proposition}

\begin{proposition}
\label{prop:V_i6}

\begin{statement}
If four magnitudes be proportional, they will also be
proportional alternately.
\end{statement}

\begin{proof}

Let A, B, C, D )x. four proportional magnitudes,
so that, as A is to , so is C to Z? ;

I say that they will also be so alternately, that is, as W is
to C, so is B to D.

A c

o-

E< 1 1 1 Qi 1

Fi 1 1 1 Hi '-—I

For o( A, B let equimultiples E, F be taken, ,, ,i .
and of C, D other, chance, equimultiples G, H.

Then, since E is the same multiple of A that F is of B,
and parts have the same ratio as the same multiples of
them, [v. is]

therefore, as  is to , so is £* to F.

But as  is to 5, so is C to /? ;
therefore also, as C is to D, so is E to F. [v. ii]

Agin, since G, H are equimultiples of C, D,
therefore, as C is to /?, so is C to H. [v. ij]

But, as C is to Z?, so is  to F
therefore also, as  is to F, so is G to H. [v. n]

But, if four magnitudes be proportional, and the first be
greater than the third,

the second will also be greater than the fourth ;
if equal, equal ;
and if less, less. •. ,-v-y,- [v- m]

Therefore, if E Is in excess of G, F is also in excess of H,
if equal, equal,
and if less, less.

Now E, 7 are equimultiples oi A, B,
and G, H other, chance, equimultiples o( C, D

therefore, as A is to C, so is B to D. [v. Def. s]

Therefore etc.
\end{proof}

\begin{annotations}

3, `` Let A, B, C, D be four proportignsl magnitudes, so that, as A Is to B, so fs
C to D.'' In a number of expressions like this it is absolutely necefisffty, wheit translatitig
inta Endbh, to interpolate words which are not in the Greek. Thus the Greek here is ;
litfTup Tiaaapa liiPTi dXtyyoi' r A, U, T, A, it rh A rp6  TO B, oih-iifj ti r rpii tA A,
literally *' Let At B C, D he four proportional magnitudes, as  to , 60 C to Z>'' The
same remark applies to the eotresponding expressions in the neil proposiiions, v. 17, 18,
and to other forms of expression in V. to — 13 and later propositions : eg. in v, 10 we have
a phrase meaning literally '*It there be mngnitudes... which taken two and two are in the
same ratio, as if to , so Z> to £,'' etc.: in v. it `` (magnitudes)... which taken two and
two are in the same rtio, atvd ht the proportion of them be perturbed, as  to , so
£ to /'','' etc. In all such cases (where the Greek is so terse as to be almost ungrammatical)
I shall insert the words necessary in English, without further remark.

\end{annotations}

\begin{notes}

Algebraically, if a : bc : d,

then a; ( = b : d.

Taking equimultiples /fta, mb of a, b, and equimultiples m, nd of (,d, we
have, by v. 15,

a : b = Ma : mb,

c ; d=ne : nd.

And, since a : b = e ; d,

we have [v. 11] ma : mb = n£ md.

Therefore [v. 14], according as wa > = < «<r, mb> = <ii4,'

so that a : e = b : d.

Aristotle tacitly uses the theorem in MetetfrologUa ill. 5, 376 a xz — 34.
The four magnitudes in this proposition mtist all be  the same kind, and
Simson inserts `` of the same kind `` in the enunciation.

This is the first of the propositions of Eucl.\ v. which Smith and Bryant
(Euclid's Ekmtnti of Geomttry, tgoi, pp. 298 sqq.) prove by means of vi, i
so far as the only geometrical magnitudes in question are straight lines or
rectilineal areas \ and certainly the proofs are more easy to follow than
Euclid's. The proof of this proposition is as follows.

To prove that, If Jour magnitudes of the same kind [straight lines or
rectilineal areas] be proportionals, they will be proportionals -when taken
alttmately.

Let F, Q, /(, Sbe the four magnitudes of the same kind such that

P:Q = R:S;

then it is required to prove that '.   . 1 < ;

P-.RQ-.S. \ -'

First, let all the magnitudes be areas.

Construct a rectangle abed equal to the area P, and to be apply the
rectangle beef equal to Q,

Also to cd>, bf apply rectangles ag, bk equal to JF, S respectively. '

Then, since the rectangles ac, be have the same height, they are to one
another as their bases, [vl. i]

Hence P:Q = ah:bf.

But P:Q = R:S.

Therefore R •S = ab:b/y [v. 1 1]

i.e. rect. ag : rect. At = ab : bf.

Hence (by the converse of vi. i) the rect-
angles ag, bk have the same height, so that k
is on the line kg.

Hence the rectangles ae, ag have the same
height, namely ab ; also , bk have the same
height, namely h/.

Therefore rect. ac • rect ag=bc

and rei;t be : recL bk = bc ; bg.

Therefore rect, ac : rect, ag - recL be ; rect, bk.

That is, P:Ji=Q:S.

Se(ondl)\ let the magnitudes be straight lines AB, BC, CD, DE.
Construct the rectangles Ab, Be, Cd, Dt with the same height.

i t t

a

f

b

h 3

k

bg.

[VI-
[v. Il]

a

be d e

A f

i i

J 1

3 E

Then AbBc = ABBC,

aiid Cd : De= CD : DE.

But AB:BC=CD:DE.

Therefore Ab .Bc=Cd: Dt.

Hence, by the first case,

Ab: Cd=Be-De,
and, since these rectangles have the same height,
AB: CD = BC : DE.

[VI. i]
[V. „]

\end{notes}

\end{proposition}

\begin{proposition}
\label{prop:V_17}

\begin{statement}
1/ magnitudes be proportional componendo, they will also
be proportional separando.
\end{statement}

\begin{proof}

Let AB, BE, CD, DF be magnitudes proportional com-
ponendo, so that, as AB is to BE, so is CD to DF
I say that they will also be proportional separando, that is,
as AE is to EB, so is CF to DF.

For of AE, EB, CF, FD let equimultiples GH, HK,
LM, MN be taken,
and of EB, FD other, chance, equimultiples, KO, NP,

Then, since GH is the same multiple of AE that HK is
oiEB,

therefore GH is the same multiple of AE that GK is of AB.

[V. ,]

But GH is the same multiple of AE that LM is of CF
therefore GK is the same multiple of AB that LM is of C/''.

``E — B e — r~B

H K O

Again, since LM is the same multiple of CF that MN
is of FD,

therefore LM is the same multiple of CF that LN is of CD.

Iv. l]

But LM was the same multiple of CF that CA' is of AB

therefore GK is the same multiple of AB that LN is of CD.

Therefore GK LN 3.r equimultiples oi AB, CD.

Again, since HK is the same multiple of EB that MN is
of/Z?,

and KO is also the same multiple of EB that NP is of /''/>,
therefore the sum HO is also the same multiple of EB that
MP is of /£'. [v. i]

And, since, as AB is to , so is CD to /?/

and oi AB, CD equimultiples GK, LN have been taken,

and of EB, FD equimultiples HO, MP,

therefore, if GK is in excess of HO, LN is also in excess of
MP,

if equal, equal, , . i,

and if less, less.

Let GK be in excess of HO ;

then, if HK be subtracted from each, '

GH is also in excess of KO.

But we saw that, if GK was in excess of HO, LN was
also in excess of MP ;

therefore LN is also in excess of MP, i

and, if MN be subtracted from each,

LM is also in excess of NP ;
so that, if GH is in excess of KO, LM is also in excess of
NP.

Similarly we can prove that,
if GH be equal to KO, LM will also be equal to NP
and if less, less.

And GH, LM are equimultiples of AE, CF,

while KO, NP are other, chance, equimultiples of EB, FD ;

therefore, as AE is to EB, so is CF to FD.

Therefore etc.
\end{proof}

\begin{notes}

Algebraically, if a b = c ; d,

then (a-b):b = (c-d)d.

I have already noted the somewhat strange use of the participles of
(TuyKttcrcu and Stai/xurat to convey the sense of the technical ivBa and
WiptiTK Xoyou, or what we denote by (ompontndo and sefarafido. lax
<rvyi«iV«'a fiiyiOr) dirdkoyov J, itai StaifitOivra draXjryov IcrtOi is, literally, ``if
magnitudes compounded be proportional, they will also be proportional
separated,'' by which is meant ``if one magnitude made up of two parts is to
one of its parts as another magnitude made up of two parts is to one of its
parts, the remainder of the first whole is to the part of it first taken as the
remainder of the second whole is to the part of it first taken.'' In the
algebraical formula above a, c are the wholes and b,a-b and d, c-are the
parts and remainders respectively. The formula might also be stated thus ;

If a-b •.b=-c d vd,

then a : b = c : d,

in which case a + , c + d are the wholes and 0, a and d, i the parts and
remainders respectively. Looking at the last formula, we observe that
``separated,'' Siatp«fl(fTa, is used with reference not to the magnitudes a, i, c, d
but to the comfamndid magnitudes a + b, b, c + d, d.

As the proof is somewhat long, it will be useful to give a conspectus of it
in the more symbolical form. To avoid minuses, we will takf for the
hypothesis the form

a + bs xa b zs c + dsUi d.

Take any equimultiples of the four magnitudes a, b, t, d, viz.

ma, mb, ntt, md,
and any other equimultiples of the consequents, viz.

nb and nd.
Then, by v. i, m(a + b), ni (c + d) are equimultiples of a + i, c-¥d,
and, by v. 2, (m + «) b, (tn + n)d are equimultiples of b, d.
Therefore, by Def, s, since a + b is to 6 its c+d is to d,

according as (w (« + i) > = <(« + /«) , ot (f + <()>-< (m + «) rf.

Subtract from m (a + i), (m + n)i the common part mi, and from
m(c + d), (m + »)d the common part md; and we Itave,

according as ma> = <n6, mc> = <nd.
But ma, mc are any equimultiples of «, e, and nb, nd any equimultiples of
kd,

therefore, by v. Def. 5,

a is to  as r is to d.

Smith and Bryant's proof follows, mutatis mutandis, their alternative proof
of the next proposition (see pp. 173 — 4 below).

\end{notes}

\end{proposition}

\begin{proposition}
\label{prop:V_18}

\begin{statement}
If magnitudes be proportional separando, they will also be
proportional com pone ndo.
\end{statement}

\begin{proof}

Let AE, EB, CF, FD be magnitudes proportional
separando, so that, as AE is
to EB, so is CF to FD ; a e b

I say that they will also be '

proportional componendo, that a

is, as AB is to BE, so is .1

CD to FD.

For, if CD be not to DF as AB to BE,

then, as AB is to .5.5', so will CD be either to some
magnitude less than DF or to a greater.

First, let it be in that ratio to a less magnitude DG,

Then, since, as AB is to BE, so is CD to DG,

they are magnitudes proportional componendo;

so that they will also be proportional separando. [v. 17]

Therefore, as AE is to EB, so is CG to GD.

But also, by hypothesis,

as AE is to EB, so is CF to FD.

Therefore also, as CG is to GD, so is CF to FD. [v, it]

But the first CG is greater than the third CF;

therefore the second GD is also greater than the fourth
FD. [v. 14]

But it is also less : which is impossible.
Therefore, as AB is to BE, so is not CD to a less
mnitude than FD.

Similarly we can prove that neither is it in that ratio to
a greater;

it is therefore in that ratio to FD itself.
Therefore etc.
\end{proof}

\begin{notes}

Algebraically, if a b = e : d

then (ab) \ b(f-kd) ; d.

In the enunciation oC this proposition there is the same special use of
Siijp>jli.(ya and crvtriiVra as there was of cruyKtiftira and SuupiSiyra in the
last enunciation. Practically, as the algebraical form shows, Stgpriiiira, might
have been left out.

The following ts the method of proof employed by Euclid,

Given tnat a:b = e:d,

suppose, if possible, that

(o + jS)  .b = (c + d):(d±x).
Therefore, J<ir«i«<fo [v. 17],

a : b = (c + x) •.(d±x),
whence, by v. 1 1, (cT x) ; (d ±x) = c : d.

But (e—x)< c, while (d + x)> d,

and (c--x)>e, while (d-x)<d,

which relations respectively contradict v. 14.

Simson pointed out (as Saccheri before him .saw) that Euclid's demonstra-
tion is not legitimate, because it assumes without proof that to any three
magnitudes, two of which, at ieasi, are 0/ the same kind, there exists a fourth
prporiionai. Clavius and, according to him, other editors made this an
a;<iom. But it is far from axiomatic ; it is not till vi. 1 2 that Euclid shows,
by construction, that it is true even in the particular case where the three
given magnitudes are alt straight lines.

In order to remove the defect it is necessary either (r) to prove beforehand
the proposition thus assumed by Euclid or (2) to prove v. tS independently
of it.

Saccheri ingeniously proposed that the assumed proposition should be
proved, for areas and straight Hues, by means of Euclid vi. i, 2 and 12. As
he says, there was nothing to prevent Euclid from interposing these proposi-
tions immediately after v. 17 and then proving v. 18 by means of them.
VI. 12 enables us to construct the fourth proportional when the three given
magnitudes are straight lines ; and vi. 1 2 depends only on vi. i and 2.
`` Now,'' says Saccheri, `` when we have once found the means of constructing
a straight line which is a fourth proportional to three given straight lines, we
obviously have the solution of the general problem To construct a straight
line which shall have to a given straight line the same ratio which two polygons
have (to one another).'`` For it is sufficient to transform the polygons into
two triangles of equal height and then to construct a straight line which shall
be a fourth proportional to the bases of the triangles and the given straight
line.

The method of Saccheri is, as will be seen, similar to that adopted by
Smith and Bryant (/i. at.) in proving the theorems of Euclid v. i6, 17, 18, 21,
so far as straight lines and rectilineal areas are concerned, by means of vi. i.

De Morgan gives a sketch of a general proof of the assumed proposition
that, B being any magnitude, and P and Q two magnitudes of the same kind,
there does exist a magnitude A which is to  in the same ratio as /'' to Q.

`` The right to reason upon any aliquot part of any magnitude is assumed ;
though, in truth, aliquot parts obtained by continual bisection would suffice :
and It is taken as previously proved that the tests of greater and of less ratio
are never both presented in any one scale of relation as compared with
another'' (see note on v. Def, 7 ad n,).

``(i) If be to  in a greater ratio than Pto Q, so is every magnitude
greater than AT, and so are sme leu magnitudes ; and if jW be to  in
a less ratio than P to Q, so is every magnitude less than M, and so are
some greater magnitudts. Part of this is in every system : the rest is proved
thus. If j be to J in a greater ratio than P to Q, say, for instance, we find
that isjI/ lies between »i2? and 23i?, while 15/'' lies before J2Q. Let M
exceed 22? by Z; then, if iV be less than M by anything less than the 15th
part of Z, iN is between ziB and 23; or JVJ less than M, is in a greater
ratio to B than P to Q. And similarly for the other case,

(2) ,can certainly be taken so small as to be in a less ratio to B than
P to Q, and so large as to be in a greater ; and since we can never pass from
the greater ratio back again to the smaller by increasing M, it follows that,
while we pass from the first designated value to the second, we come upon an
intermediate magnitude A such that every smaller is in a less ratio to B than
P to Q, and every greater in a greater ratio. Now A cannot be in a less ratio
to B than P to Q, for then some greater magnitudes would also be in a less
ratio ; nor in a greater ratio, for then some less magnitudes would be in a
greater ratio; therefore A is in the same ratio to 2? as  to Q. The previously
proved proposition above mentioned shows the three alternatives to be the
only ones.''

Alternative proofs of V. 18.

Simson bases his alternative on v, 5, 6, As the 18th proposition is the
converse of the 17th, and the latter is proved by means of v. i and j, of
which V. 5 and 6 are converses, the proof of v. 18 by v. 5 and 6 would be
natural; and Simson holds that Euclid must have proved v, i3 in this way
because ``the sth and 6th do not enter into the demonstration of any
proposition in this book as we have it, not can they be of any use in any
proposition of the Elements,'' and ``the sth and 6th have undoubtedly been
put into the 5th book for the sake of some propositions in it, as all the other
propositions about equimultiples have been.''

Simson's proof is however, as it seems to me, intolerably long and difficult
to follow unless it be put in the symbolical form as follows.

Suppose that a is to i as ( is to rf; , ..'',.'`` ;

it is required to prove that a 4- j is a b »c-¥dX.o i.

Take any equimultiples of the last four magnitudes, say

w(a + i), iiib, m(c + d), md,

and any equimultiples of i, d, as

nb, nd.

Oearly, if «d is greater than mi, ...

lid is greater than md;
if equal, equal ; and if less, less.
I Suppose nd not greater than mi, so that nd is also not greater than W.

Now m(a4-i') is greater than mi ;

therefore m[a + i) is greater than »6.

Similarly m ( + rf) is greater than nd.

II. Suppose ni greater than mi.

Since « (a + i), md, m(e + d), md are equimultiples of (o + i), i, (f + rf)i <>
ma is the same multiple of a that m(a-y i) is of (a + i),
and mc is the same multiple of c that m (f + d) is of (c + d),

so that ffio, mc are equimultiples of a, c. [v. 5]

Again n, nd are equimultiples of J, d,
and so are m, md
therefore (n-m)i, (H-m)d are equimultiples of b, d and, whether n-m
is equal to unity or to any other integer [v. 6 it follows, by Def. S, that,
since a, b, r, d are profxsrtionals,
if ma is greater than (n-m)i,

then mc is greater than (n-m)d;

if equal, equal ; and if less, less.

(i) If now m(a--i) is greater than ni, subtracting mi from each, we have
ma is greater than (n-m)i;
therefore mc is greater than (n - m)d,

and, if we add md to each,

m(c-¥d) is greater than nd.

(3) Similarly it may be proved that,
if w (a + ) is equal to ni,

then j» (f + rf) is equal to nd,

and (3) that, if m(aA- i) is less than tti,

then mic + d) is less than nd.

But (under I. above) jt was proved that, in the case where ni is not
greater than mb,

m(a +i) a always greater than ni,

and m(c + d) is always greater than nd.

Hence, whatever be the values of m and n, m (c + d) is always greater than,
equal to, or less than nd according as m(a + i) is greater than, equal to, or
less than nb.

Therefore, by Def. 5,

a+i is U>baac + d is tod,

Todhunter gives the following short demonstration from Austin (Exami-
nation 0/ the fin t six books of Euclid's Elements).

``Let AE be to J?J? as CFis to FD:

AB shall be to BE as CD is to DF.

For, because AE is to EB as CF'xs to FD,
therefore alternately,

AE is to CFtts EB is to FD. [v. i6]

And, as one of the antecedents is to its consequent, so is the sum of the
antecedents to the sum of the consequents: [v. 12]

therefore, as EB is to FD, so are AE, EB together to CF,
FD together ;

that is, AB is to CD as EB is to FD.

Therefore, alternately,

AB is to BE as CD is to FD.''

The objection to this proof is that it is only valid in the case
where the proposition v. t6 used in it is valid, i.e. where all four
magnitudes aie of the same kind.

Smith and Bryant's proof avails where all four magnitudes
are straight lines, where all four magnitudes are rectilineal areas,
or where one antecedent and its consequent are straight lines and the other
antecedent and its consequent rectilineal areas. .. ,

Suppose that A : B= C : D.

First, let all the magnitudes be areas.

Construct a rectangle abed equal to A, and to be apply the rectangle baf
equal to B.

Also to ab, bj apply the rectangles ag, bk
equal to C, D respectively.

Then, since the rectangles m, be have equal
heights be, they are to one another as their
base [vi. i]

Hence ab:bf= recL tu : rect bt

= C:D

= rect, ag ; rect bk.

Therefore [vi. i, converse] the rectangles ag, bk have the same height, so
that i is on the strght line hg.

Hence A - BB- recL ae ; rect. bt

= af:bf

= rect ak : rect bJi
= C+D:D.

SMondiy, let the magnitudes A, B' straight lines and the magnitudes
C, i> areas.

Let ab, if x equal to the straight lines A, B, and to ab, bf apply the
rectangles ag, bk equal to C, D respectively.

Then, as before, the rectangles ag, bk have the same height.

Now A + B:Ba/.bf

= rect. aJk : rect. bk

= C + D:D. i , , ,.

Thirdly, let all the magnitudes be straight lines.

Apply to the straight lines C, D rectangles F, Q having the same height.

<t <

b

J

k t

1 k

4

Then

P:Q=C:D.

[v.. l]

Hence, by

the second case,

Also

P+Q:Q=C+D.D.

Therefore

A + B:B = C+I):£>.

\end{notes}

\end{proposition}

\begin{proposition}
\label{prop:V_19}

\begin{statement}
If f as a whoU is to a whole, so is a part subtracted to a
part subtracted, the remainder will also be to the remainder
as whole to whole.
\end{statement}

\begin{proof}

For, as the whole AB is to the whole CD, so let the
part AE subtracted be to the part CF
subtracted ;

I say that the remainder EB will also be   ?

to the remainder FD as the whole AB to c  d
the whole CD.

J

For since, as AB is to CD, so is AE
to CF,
alternately also, as BA is to AE, so is DC to CF. [v. 16]

And, since the magnitudes are proportional componendo,
they will also be proportional separando, [v, 17]

that is, as BE is to BA so is DF to CF
and, alternately,

as BE is to DF, so is EA to FC. [v, ifi]

But, as AB is to CF, so by hypothesis is the whole AB
to the whole CD.

Therefore also the remainder EB will be to the remainder
FD as the whole AB is to the whole CD. [v. 1 1]

Therefore etc.

[PoRisM. From this it is manifest that, if magnitudes be
proportional componendo, they will also be proportional
eonvertendo

p. E. D.
\end{proof}

\begin{notes}

Algebraically, l a:b = cd (where e<.a and d < fi), then
(a-<;):(i-d) = a:i.

The `` Porism `` at the end of this projwsition is led up to by a few lines
which Heiberg brackets because it is not Euclid's habit to explain a
Porism, and indeed a Porism, from its very iwture, should not need any

explanation, being a sort of by-product appearing without effort or trouble,
air/nYfioTtvruf (Proclus, jpt 303, 6). But Heiberg thinks that Simson does
wrong in finding fault with the argument leading to the ``Porism,'' and that
it does contain the true demonstration of conversion of a ratio. In this it
appears to me that Heiberg is clearly mistaken, the supposed proof ott the
basis of Prop. 19 being no more correct than the similar attempt to prove the
inversion of a ratio from Prop. 4. Thu words are : `` And since it was
proved that, as A£ is to CB, so is EB to FJD,

alternately also, as AB is to BE, so is CD to FD :

therefore magnitudes when compounded are proportional. ,.,

But it was proved that, as £A is to AE, so is DC to CE and this is
foniferiendo.''

It will be seen that this amounts to proving /ram the hypothesis a:b = €d
that the following transformations are simultaneously true, viz. :

a~e-=b:h~d,

and a'.c-b:d.

The former is not proved from the latter as it ought to be if it were intended
to prove conversion.

The inevitable conclusion is that both the ``Porism'' and the argument
leading up to it are interpolations, though no doubt made, as Heiberg says,
before Theon's time.

The conversion of ratios does not depend upon v. rg at all but, as Simson
shows in his Proposition E (containing a proof already given by Clavius), on
Props. 17 and 18. Prop. E is as follows.

If four magnitudes U proportionals they are also proportionals by conversion,
that is, tht first is to its excess above the second as the third is to
its excess above tht fourth. '

Let .J be to BE as CD to DF:
then BA hloAE?ts DC to CE  °

Because AB is to BE as CD to DF, F

by division [sarando],

AE'is to EB as CEto FD, [v. ij]

and, by inversion,

BE is to EA as DE to EC.

[Simson's Prop, B directly obtained from v, Def. 5]
Wherefore, by composition [comnendo],

BA is to AE as DC to CE [v. 18]

\end{notes}

\end{proposition}

\begin{proposition}
\label{prop:V_20}

\begin{statement}
If there be three magnitudes, and others equal to them in
multitttde, which taken two and two are in the same ratio, and
if x aequali the first be greater than the third, the fourth will
also be greater than the sixth; if equal, equal; and, if less, less.
\end{statement}

\begin{proof}

Let there be three magnitudes A, B, C, and others
D, £, F equal to them in multitude, which taken two and
two are in the same ratio, so that,

  as / is to J3t so is D to E, •
and as B is to C, so is  to i;

and let y4 be greater than C ex aequali ;
I say that D will also be greater than F\ i A is equal to C,
equal ; and, if less, less.

A o-

B E-

c— F-

For, since A is greater than C,
and B is some other magnitude,

and the greater has to the same a greater ratio than the less
has, [v. 8]

therefore A has to  a greater ratio than C has to B.

But, as  is to B, so is D to E,
and, as C is to B, inversely, so is / to  ;
therefore Z? has also to  a greater ratio than 7 has to , [v. 13]

But, of magnitudes which have a ratio to the same, that
which has a greater ratio is greater ; [v. 10]

therefore D is greater than F.

Similarly we can prove that, if j4 be equal to C, D will
also be equal to F ; and if less, less.

Therefore etc.
\end{proof}

\begin{notes}

Though, as already remarked, Euclid ha£ not yet given us any definition
of cempoanded ratios. Props, 20 — 23 contain an important part of the theory
of such ratios. The term ``compounded ratio'' is not used, but the propositions
connect themselves with the definitions of ex atguali in its two forms, the
ordinary form defined in Def. 1 7 and that called ptrturbid proportion in
Def. 18. The compounded ratios dealt with in these propositions are those
compounded of successive ratios in which the consequent of one is the
antecedent of the next, or the antecedent of one is the consequent of
the next.

Prop. Z2 states the fundamental proposition about the ratio tx aequali in
its ordinary form, to the effect that,

if a is to  as  is to f,

and  is to If as « is to/,

then a is to  as  is to/

with the extension to any number of such ratios ; Prop, 23 gives the
corresponding theorem for the case al perturbed proportion, namely thati

if a is to  as « is to  .,i

and j is to  as </ is to e,

then a IS to f as rf is to/

Each depends on a preliminary proposition. Prop, a a on Prop. 20 and
Prop. 33 on Prop, a i. The course of the proof will be made most clear by
using the algebraic notation.

The preliminary Prop, 20 asserts that,
if a  .b = d:e,

and t:c-t:/,

then, according as «> = << , ds- = <./.
For, according as a is greater than, equal to, or less than c,
the ratio o ; * is greater than, equal to, or less than the ratio ( : b, [v, 8 or v. 7]
or (since d:e = a:b,

and £:b=/:e)

the ratio d:es greater than, equal to, or less than the ratio/; t,

[by aid of V. 13 and v. 11]
and therefore d is greater than, equal to, or less than/ [v. 10 or v. 9]
It is next proved in Prop. 22 that, by v. 4, the given proportions can be
transformed into

ma : nb = md : tie,
and nb : pc = ne : pf,

whence, by v. 20,

according as ww is greater than, equal to, or less than, pc,
md is greater than, equal to, or less than
80 that, by Def. 5, •     - . '

a:e = d:/.

Prop, 23 depends on Prop. 21 in the same way as Prop. 22 on Prop, ao,
but the transformation of the ratios in Prop. 23 is to the following :
(i) ma : mb = ne : ttf

(by a double application of v. 1 5 and by v. 11),
(a) mb '.nc ~mdnt

(by V. 4, or equivalent steps),
and Prop, a I is then used. '`` . r . ..

Simson makes the proof of Prop, 20 slightly more explicit, but the main
difference from the text is in the addition of the two other cases which Euclid
dismisses with `` Similarly we can prove.'' These cases are ;

``Secondly, let A be equal to C; then shall D be equal to F.
Because A and C are equal to one another,

j4 is to .5 as C is to jB. [v, 7]

But   A   ifi B a D vo E,  '.'.,

and C is to .f as .F is to E,

wherefore i? is to £  as 7 to E ; . 1 -i [v. 11]

and therefore Z> is equal to . 1 • .• [v. g]

•..1.1

Next, let A be less than C; then shall JJ be less than J.
For C is greater than A,   -

and, as w shown in the lirst case,

Cis to Jas ito,
and, in like manner,

J is to vSt as  to i? ;

therefore F is greater than D, by the first case ; and therefore D is less
than JK''

\end{notes}

\end{proposition}

\begin{proposition}
\label{prop:V_21}

\begin{statement}
// there be three magnitudes, and others equal to them in
multitude, which taken two and two together are in the same
ratio, and the proportion of them be perturbed, then, if ex
aequali the first magnitude is greater than ths third, the
fourth will also be greater than the sixth ; if equal, equal;
and if less, less.
\end{statement}

\begin{proof}

Let there be three magnitudes A, B, C, and others D, E, F
equal to them in multitude, which taken two and two are In
the same ratio, and let the proportion of them be perturbed,
so that,

as A is to i?, so is  to F, ,

and, as  is to C, so is /? to E,

and let A be greater than C ex aequali ;

I say that D will also be greater than F\ if A is equal to

C, equal ; and if less, less.

A D-

B — E-

o F-

For, since A is greater than C,
and B is some other magnitude,
therefore A has to  a greater ratio than C has to B. [v. 8]

But, as A is to B, so is E to F,
and, as C is to B, inversely, so is A to D.
Therefore also E has to a greater ratio than E has to JD.

[v-3]

But that to which the same has a greater ratio is less ;

[v. 10]

therefore F is less than D ;

therefore Z? is greater than F,

Similarly we can prove that, f

if  be equal to C, D will also be equal to F
and if less, less.

Therefore etc. Q.E.D.

Algebraically, if a:b = e:f,

and b:c=d€,

then, according asa> = <f, ;/> = <f.
Simson's alterations correspond to those which he makes in Prop, a a. After
the first case he proceeds thus.

``Secondly, let A be equal to C; then shall D be equal to F.
Because A and Care equal,

H,i is to  as C is to . [v. 7]

But /4 is to  as £ is to .

and C is to . as £ is to  :

wherefore E is to /''as E to D, [v. 11]

and therefore D is equal to F. [v. 9]

Next, let A be less than C\ then shall D be less than F.
For C is greater than A
and, as was shown,*

C is to .5 as .E to 2?,
and, in like manner,

.ff istoas F.aE

therefore iis greater than D, by the first case,
and therefore D is less than F'
The proof may be shown thus.

According as (J > = < f, a;h> = <(;h. '
But a:b = e:fy and, by inversion, c:b = t:d.

Therefore, according as «> = <<:, e:/> = <e:d,
and therefore d> = </.

Proposition 22.
1/ there be any number 0/ magniiudes whatever, arui others
equal to them in multitude, which taken two and two together
are in the same ratio, they will also be in the same ratio ex
aequali.

Let there be any number of magnitudes A, B, C, and
others Z>, £, F equal to them in multitude, which taken two
and two tcether are in the same ratio, so that,

as j4 is to .5, so is Z* to E
and, as .5 is to C, so is £'' to ;

I say that they will also be in the same ratio ex aequali,
< that is, as . is to C, so is D to F> .

For of A, D let equimultiples G, H be taken,
and of B, E other, chance, equimultiples A', L ;
and, further, of C, F other, chance, equimultiples J/, N.

A B c-

D E- — F-

— I K 1

Then, since, as A is to B, so is Z? to £'',
and of A, D equimultiples G, H have been taken,
and of B, E other, chance, equimultiples K, L,

therefore, as 6 is to K, so is H to L. [v. 4]

For the same reason also,

as A' is to M, so is L to N.

Since, then, there are three magnitudes G, K, M, and
others H, L, N equal to them in multitude, which taken two
and two together are in the same ratiQ,

therefore, ex aeguali, if G is in excess of M,Hs also in excess
oiN;

if equal, equal; and if less, less.   - .i [v. ao]

And G, H are equimultiples o( A, D,

and M, N other, chance, equimultiples of C, F.

Therefore, as  is to C, so is D to F. [v. Def. 5]

Therefore etc,
\end{proof}

\begin{notes}

Euclid enunciates this proposition as true of any number of magnilvdti
whatetier forming two sets connected in the manner described, but bis proof is
confined to the case where each set consists of three magnitudes only. The
extension to any number of magnitudes is, however, easy, as shown by
Simson.

``Next let there be four magnitudes A,B,C, D, and other four E, F, G, Jf,
which two and two have the same ratio, viz. :
as /4 is to , so is -£ to ./

A B C D
E F O H

and as £ is to C, so is .to G,

and as C is to A so is C to .ff ;
A shall he to D as £ to IT.
Because A, S, C are three magnitudes, and E, .f, G other three, which
taken two and two have the same ratio,
by the foregoing case,

v is to C as £ to 6.

But C is to Z) as C is to If;
wherefore again, by the fiist case,

 is to Z> as £ to /
And so on, whatever be the number of magnitudes.''

\end{notes}

\end{proposition}

\begin{proposition}
\label{prop:V_23}

\begin{statement}
// there be three magnitudes, and others equal to them in
mtdtilude, which taken two and two together are in the same
ratio, and the proportion of them be perturbed, they will also
be in the same ratio ex aequali.
\end{statement}

\begin{proof}

Let there be three magnitudes A, B, C, and others equal
to them in multitude, which, taken two and two together, are
in the same proportion, namely D, E, F\ and let the propor-
tion of them be perturbed, so that,

as >4 is to B, so is £'' to F,
and, as j9 is to C, so is Z? to  ; '

1 say that, as  is to C, so is Z? to .F.

A B — c

D E F- — -

O 1 1 H 1 1 L

K 1 1 M 1 N' 1

Of W, B, D let equimultiples G, H, Kh taken,
and of C, E, Z other, chance, equimultiples L, M, N.

Then, since G, //are equimultiples oi A, B,
and parts have the same ratio as the same multiples of
them, [v. is]

therefore, as A is to .5, so is G to H.
For the same reason also,

as .£  is to /  so is i?/ to A.
And, as . is to B, so is E to E

therefore also, as G is to H, so is M to N. [v. 11]

Next, since, as  is to C, so is D to E,
alternately, also, as B is to D, so is C to E. [v. i«]

And, since H, K are equimultiples of B, D,
and parts have the same ratio as their equimultiples,

therefore, as .f is to Z?, so is Z to K. [v. i s]

tti tA • BOOK V [v. 23

But, as  is to Z7, so is C to £' ;
therefore also, as // is to A', so is C to B, [v. n]

Again, since L, M are equimultiples of C, E,

therefore, as C is to E, so is L to jIT. [v. 15]

But, as C is to E, sovsHtoK;

therefore also, as H is to K, so is L to M, (v. 1 1]

and, alternately, as // is to Z, so is A' to M. [v. 16]

But it was also proved that,

s G s to H, so is M to A''.

Since, then, there are three magnitudes G, H, L, and
others equal to them in multitude K, M, N, which taken two
and two together are in the same ratio,
and the proportion of them is perturbed,
therefore, ex aequalif if G is in excess of L, K is also in excess
of A''; , ., , .,

if equal, equal; and if less, less. [v. n]

And G, K are equimultiples of A, D,
and L, N oi C, F.

Therefore, as A is to C, so is Z? to /

Therefore etc.
\end{proof}

\begin{notes}

There is an important difference between the version given by Simson of
one part of the proof of this proposition and that found in the Greek text of
Heiberg. Peyrard's ms. has the version given by Heiberg, but Simson's
version has the authority of other mss. The Basel editw prinaps gives both
versions (Simson 's being the first). After it has been proved by mean? of
V. 1 5 and V. 1 1 that,

as G is to jff, so is jV to A
or, with the notation used in the note on Prop, *o,

ma \ mb = ne ; nf,
it has to be proved further that,

• as .ff is to Z, so is /T to M,
or mb •.nc = md : ne,

and it is clear that the latter result may be directly inferred from v. 4, The
reading translated by Simson makes this inference :

`` And because, as £ is to C, so is Z? to £,
and H, K sxt equimultiples of S, /?,
and L, Mot C, E,

therefore, as H is to Z, so is K to M'' [v. 4]

The version in Hei berg's text is not only much longer (it adopts the

roundabout method of using each of three Propositions v. 11, 15, 16 twice
over), but it is open to the objection that it uses v, 1 6 which is only applicable
if the four magnitudes are of the same kind; whereas v. 33, the proposition
now in question, is not subject to this restriction.

Simson rightly observes that in the last step of the proof it should be
stated that `` G, K are any equimultiples whatever of A, D, a.nA L, N any
whatever of C, F.''

He also gives the extension of the proposition to any number of magnitudes,
enunciating it thus 1

`` If there be any number of magnitudes, and as many others, which, taken
two and two, in a cross order, have the same ratio ; the first shall have to the
last of the first magnitudes the same ratio which the first of the others has to
the last `` ; -,.••,,'.

and adding to the proof as follows : \ . t < 1

``Next, let there be four magnitudes A, B, C, D, and other four E, F, G, If,
which, taken two and two in a cross order, have the same ratio, viz. :
Am BasGto H,

Bto Cas FioG, A B C

-W6A CtoDasM toF; | E F Q H

then A is to J) a £ to M.

Because A, B, C are three magnitudes, and F, G, H other three which,
taken two and two in a cross order, have the same ratio,

by the first case, . is to C as to H. Ii  <•

But C is to .0 as £ is to F

wherefore again, by the first case, . ,

As,oDiEoH.
And so on, whatever be the number of magnitudes.''

\end{notes}

\end{proposition}

\begin{proposition}
\label{prop:V_24}

\begin{statement}
If a first magnitude have to a second the same ratio as a
third has to a fourth, and also a fifth have to the second the
same ratio as a sixth to the fourth, the first and fifth added
together milt have to the second the same ratio as the third and
sixth have to the fourth.
\end{statement}

\begin{proof}

Let a first magnitude AB have to a second C the same
ratio as a third DE has to a

fourth F; f — : g q

and let also a fifth BG have to o

the second C the same ratio as d 1 H

a sixth EJ/ has to the fourth f

E;

1 say that the first and fifth added together, AG, will have
to the second C the same ratio as the third and sixth, I? J/,
has to the fourth E.

1*4 BOOK V [v. 34

For since, as BG is to C, so is B// to F,
inversely, as C is to BG, so is  to BH.

Since, then, as AB is to C, so is DB to B,
and, as C is to BG, so is F to B//,
therefore, ex aequali, as AB is to G, so is DB to /. [v. n]

And, since the magnitudes are proportional separando, they
will also be proportional componendo ; [v. 18]

therefore, as AG is to GB, so is Z?/ to HB.

But also, as BG is to C, so is BH to A ;
therefore, ex aegtta/i, as 4 6'' is to C, so is DH to F. [v. aa]

Therefore etc.
\end{proof}

\begin{notes}

Algebraically, if a -.c = d:f,

and b(~e:f,

then  • (a--b):c=(d-¥i):J.

This profwsition is of the same character as those which precede the
propositions relating to compounded ratios   but it could not be placed earlier
than it ts because v. 22 is used in the proof of it.

Inverting the second proportion to

cb-f:t,
it follows, by v, 23, that a;6 = d:t,

whence, by v. 18, (a + b):i ~(d-i-e) : e,

and from this and the second of the two given proportions we obtain, by a
fresh application of v. 22,

(a-l,):c=(d*e):/.

The first use of v. as is important as showing that the opposite process to
compounding ratios, or what we should now call division of one ratio by
another, does not require any new and separate propositions.

Aristotle tacitly uses v. 24 in combination with v. 1 1 and v, 16, Meleorologica
'``- S> 37a 22 — 26.

Simson adds two corollaries, one of which (Cor. 3) notes the extension to
any number of magnitudes.

`` The proposition holds true of two ranks of magnitudes whatever be their
number, of which each of the first rank has to the second magnitude the same
ratio that the corresponding one of the second rank has to a fourth magnitude ;
as is manifest''

Simson's Cor. i states the corresponding proposition to the above with
separando taking the place of compomnds, viz., that corresponding to the
algebraical form

(a-b);c(fi-e):f.

``Cor. I. If the same hypothesis be made as in the proposition, the
excess of the flrst and fifth shall be to the second as the excess of the third
and sixth to the fourth. The demonstration of this is the same with that of
the proposition if division be used instead of composition.'' That is, we use
V. 17 instead of v. 18, and conclude that

(a~b):b = (d-t):t.

\end{notes}

\end{proposition}

\begin{proposition}
\label{prop:V_25}

\begin{statement}
If four magnitudes be proportional, the greatest and the
least are greater than the remainittg two.
\end{statement}

\begin{proof}

Let the four magnitudes A£, CD, E, F be proportional

so that, as AB is to CD, so is E to

F, and let AB be the greatest of them

and F the least ;  Q b

I say that AB, F are greater than c

CD, E. H P

c 1 —

For let AG be made equal to E, ,

and CH equal to F,

Since, as  is to CD, so is
to F,

and E is equal to AG, and / to CM,

therefore, as AB is to CD, soh AG to C/T.

And since, as the whole AB is to the whole CD, so is
the part AG subtracted to the part C// subtracted,

the remainder GB will also be to the remainder HD as
the whole AB is to the whole CD. [v. 19]

But AB is greater than CD ;
therefore GB is also greater than HD.

And, since AG\ equal to E, and CH to F,
therefore AG, F a.re equal to CH, E.

And if, GB, HD being unequal, and GB greater, AG, F
be added to GB and C,  be added to HD,

it follows that AB, F are greater than CD, E.

Therefore etc.
\end{proof}

\begin{notes}

Algebraically, if a-.bcid,

and a is the greatest of the four magnitudes and d the least,

a-¥ d> 6 + c.

Simaon is right in inserting a word in the setting-out, ``let AB be the
greatest of Ihem and <censequenily> J' the least.'' This follows from the
particular case, really included in Def. 5, which Sinison makes the subject of
his proposition A, the case namely where the equimultiples taken are ante the
several magnitudes.

The proof is as follows.

Since a:b = £:d,

a — (h — d=a:b, [v. 19]

But «>*; therefore (tf-)>(*-i/). • [v. 16 and 14]

Add to each (c+d);
therefore (o + rf) > (i + c).

There is an important particular case of this proposition, which is,
however, not mentioned here, vh. the case where * = c. The result shows, in
this case, that tAe arithmttic PKan between two magnitudes is greater than
ffielr geometric mean. The truth of this is proved for straight lines in vi.- 27
by ``geometrical algebra,'' and the theorem forms the Siopur/ioj for equations
of the second degree.

Simson adds at the end of Book y, four propositions, F, G, H, K, which,
however, do not seem to be of sufficient practical use to justify their inclusion
here. But he adds at the end of his notes to the Book the following
paragraph which deserves quotation word for word.

``The 5th book being thus corrected, I most le.idily agree to what the
learned Dr Barrow says, 'that there is nothing in ihe whole boily of the
elements of a more subtile invention, nothing more solidly established, and
more accurately handled than the doctrine of proportionals.' And there is
some ground to hope that geometers will think that this could not have been
said with as good reason, since Theon's time till the present.''

Simson's claim herein will readily be admitted by all readers who are
competent to form a judgment upon his criticisms and elucidations of
Book~\book{V}.

\end{notes}

\end{proposition}

\part{Book VI}

\chapter*{Introductory Note}

The theory of proportions has been established in Book~\book{v} in a
perfectly general form applicable to all kinds of magnitudes (although
the representation of magnitudes by straight lines gives it a jeo
metrical appearance) ; it is now necessary to apply the theory to the
particular case oi geometrical investigation.  The only thing still
required in order that this may be done is a proof of the existence of
such a magnitude as bears to any given finite magnitude any given
finite ratio ; and this proof is supplied, so far as regards the
subject matter of geometry, by vi. n which shows how to construct a
fourth pro- portional to three given straight lines,

A few remarks on the enormous usefulness of the theory of proportions
to geometry will not be out of place. We have already in Books i. and ii.
made acquaintance with one important part of what has been well called
geometrical algebra, the method, namely, of application of areas. We have
seen that this method, working by the representation of products of two
quantities as rectangles, enables us to solve some particular quadratic equations.
But the limitations of such a method are obvious. So long as general
quantities are represented by straight lines only, we cannot, if our geometry
is plane, deal with products of more than two such quantities ; and, even
by the use of three dimensions, we cannot work with products of more
than three quantities, since no geometrical meaning could be attached to
such a product. This limitation disappears so soon as we can represent any
general quantity, corresponding to what we denote by a letter in algebra, by
a ratio; and this we can do because, on the general theory of proportion
established in Book v., a ratio may be a ratio of two incommensurable
quantities as well as of com mensu rabies. Ratios can be compoundeti ad
infinitum, and the division of one ratio by another is equally easy, since it is
the same thing as compounding the first ratio with the inverse of the second.
Thus e,g. it is seen at once that the coefficients in a quadratic of the most
general form can be represented by ratios between straight lines, and the
solution by means of Books i. and n, of problems corresponding to quadratic
equations with particular coefficients can now be extended to cover any
quadratic with real roots. As indicated, we can perform, by composition of
ratios, the operation corresponding to multiplying algebraical quantities, and
this to any extent. We can divide quantities by compounding a ratio with
the inverse of the ratio representing the divisor. For the addition and
subtraction of quantities we have only to use the geometrical equivalent of
bringing to a common denominator, which is effected by means of the fourth
proportional. ..  . ,

\chapter*{Definitions}

\begin{enumerate}

\item\label{def:VI_1} Similar rectilineal figures are such as have
  their angles severally equal and the sides about the equal angles
  proportional.

\item\label{def:VI_2} [Reciprocally related figures. See noie.]

\item\label{def:VI_3} A straight line is said to have been cut in
  extreme and mean ratio when, as the whole line is to the greater
  segment, so is the greater to the less.

\item\label{def:VI_4} The height of any figure is the perpendicular
  drawn from the vertex to the base.

\end{enumerate}

\section*{Definition 1}

QfiGija. (Tjara tvvypofifid fiTTtv, wra rav re ymvia ltK iVr Kara fjttav ital
Tf TTtpl Tat tirat 'ott'taf frXpa (ikoAdOK

T'lis definition is quoted by Aristotle, Ana/, post. 11. 17, 99 a 13, where
he says that simUantjf (to o;u.Diof) in the case of figures ``consists, let us say
(htuk), in their having their sides proportional and their angles equal.'' The
use of the word laat may suggest that, in Aristotle's time, this definition had
not quite established itself in the text-books (Heibeig, MathemaiUches zu
Arisiolties, p. g).

It was pointed out in Van Swinden's Eknunts of Geometry (Jacobi's
edition, 1834, pp. 1 14 — 5) that Eiuclid omits to stale an essential part of the
definition, namely that ``the corresponding sides must be opposite to equal
angles,'' which is necessary in order that the corresponding sides may follow
in the same order in both figures.

At the same time the definition states more than is absolutely necessary,
for it is true to say that iwo polygons are similar when, if the iides and angles
are taken in the same order, the angles are equal and the sides about the equal
angles are proportional, omitting

(i) three consecutive angles,

or (2) two consecutive angles and the side common to them,

or (3) two consecutive sides and the angle included by them,

and making no assumption with regard to the omitted sides and angles.

Austin objected to this definition on the ground that it is not obvious that
the properties (i) of having their angles respectively equal and (2) of having
the sides about the equal angles proportional can coexist in two figures ; but,
a definition not being concerned to prove the existence of the thing defined,
the objection falis to the ground. We are property left to satisfy ourselves as
to the existence of similar figures in the course of the exposition in Book vi.,
where we learn how to construct on any given straight line a rectilineal figure
similar to a given one (vi. i8j.

\section*{Definition 2}

The Greek text gives here a. definition of riciprocally related fibres
(dtTTnrtTTOVd™ ayT<i. ``[Two] figures are redprocally relaUd when there
are in each of the two figures antecedent and consequent ratios'' ('An-HrnrokSora
Si o-j()j/iaTa i<mv, oray tv iKorifxf Tiui' (j)(T)iiaTiati ijyiivfio'oi t€ nai Iro/uTOt Aoyoi
wiTtv). No intelligible meaning can be attached to ``antecedent and con-
sequent ratios `` here ; the sense would require rather `` an antecedent and a
consequent of (two equal) ratios in each figure.'' Hence Candalla and
Peyrard read Xoywc Spot (``terms of ratios'') instead of Ao'yiii. Camerer reads
Xayar without upoi. But the objection to the definition lies deeper. It is
never used; when we come, in vi. 14, 15, xi. 34 etc. to jjaiallelograms,
triangles etc. having the property indicated, they are not called `` reciprocal ``
parallelograms etc., but parallelograms etc. ``/A sides ofwhUh are reciprocally
proportional,'' w/ ot-r(irtiroi'Sa<ric at irAtupoi, Hence Simson appears to be
right in condemning the definition; it may have been interpolated from Heron,
who has it.

Simson proposes in his note to substitute the following definition. ``Two
magnitudes are said to be reciprocally proportional to two others when one
of the first is to one of the other magnitudes as the remaining one of the last
two is to the remaining one of the first.'' This definition requires that the
magnitudes shall be all of the same kind.

\section*{Definition 3}

'Afipav nat, fiiirov Aoyoi' tiStia Tir/iigcrAu Xiytmt, Srav  wt 7 Skii -rp rd
futotf T/jiTJ/juij oi/ru? T fArtiov irpof ra IXaTTCv*

Definition 4.

'Ayofimj

The definition of `` height `` is not found in Campanus and is perhaps
rightly suspected, since it does not apply in terms to parallelograms, parallele-
pipeds, cylinders and prisms, though it is used in the Elements with reference
to these latter figures. Aristotle does not appear to know altitude (vot) in
the mathematical sense; he uses naStTiK of triangles (Meieareloea tn. 3,
373 a 11). The term is however readily understood, and scarcely requires
definition.

Aoyoc Ik koywv (TvyKturBai XcycTat, crap at twv koytav injAuroTi/rc; j0* lavrac
woXXankatTiauSturat vtumtrl riva.

``A ratio is said to be compounded of ratios when the sizes (jnjAiKortjrt?) of
the ratios multiplied together make some (? ratio, or size),'']

As already remarked (pp. 116, 132), it is beyond doubt that this definition
of ratio is interpolated. It has little MS. authority. The best MS. (P) only has
it in the margin; it is omitted altogether in Campanus' translation from the

Arabic ; and the other mss. which contain it do not agree in the position
which they give to it. There is no reference to the definition in the place
where compound ratio is mentioned for the first time (vi. t), nor anywhere
else in Euclid; neither is it ever referred to by the other great geometers,
Archimedes, Apollonius and the rest. It appears to be only twice mentioned
at all, (:) in the passage of Eutocius referred to above (p. ii6) and (») by
Theon in his commentary on Ptolemy's ffun-ou. Moreover the content of
the definition is in itself suspicious. It speaks of the `` sizes of ratios being
multiplied together (literally, into themselves),'' an operation unknown to
geometry. There is no wonder that Eutocius, and apparently Theon also, in
their efforts to explain it, had to give the word jrrfAiiHjnjt a meaning which has
no application except in the case of such ratios as can be expressed by
numbers (Eutocius e.g. making it the ``number by which the ratio is called'').
Nor is it surprising that Wallis should have found it necessary to substitute
for the `` quantitas `` of Commandinus a different translation, `` quantuplicity,''
which he said was represented by the ``expeneni af the ratio'' ( ratio nis ex-
ponens), what Peletarius had described as ``denominatio ipsae pro portion! s''
and Clavius as ``denominator.'' The fact is that the definition is ungeometrical
and useless, as was already seen by Savile, in whose view it was one of the
two blemishes in the body of geometry (the other being of course Postulate 5).
It is right to add that Hultsch (art. ``Eukleides'' in Pauly-Wissowa's Real-
EtuytlopddU dtr danischen Allertumswissenschaft) thought the definition
genuine. His grounds are (i) that it stood in the iroAaui lnooi repre-
sented by P (though P has it in the margin only) and (a) that some ex-
planation on the subject must have been given by way of preparation for
VI. 25, while there is nothing in the definition which is incomisteni with the
mode of statement of vi. 23. If the definition is after all genuine, I should
be inclined to regard it as a mere survival from earlier textbooks, like the first
of the two alternative definitions of a solid angle (xt, Def 11); for its form
seems to suit the old theory of proportion, applicable to commensurable
magnitudes only, better than the generalised theory of Eudoxus,

\part*{Book VI. Propositions}

\begin{proposition}
\label{prop:VI_1}

\begin{statement}
Triangles and parallelograms which are under the same
height are to one another as their bases.
\end{statement}

\begin{proof}

Let ABC, A CD be triangles and EC, C/ parallelograms
under the same height ;

j I say that, as the base BC is to the base CZ7, so is the
triangle ABC to the triangle A CD, and the parallelogram
£C to the parallelogram CF.

For let BD be produced in both directions to the points
ff, L and let [any number of straight lines] BG, GH be
lo made equal to the base BC, and any number of straight lines
DK, KL equal to the base CD ;

let AG, AH, AK, AL be joined.

Then, since CB, BG, GH are equal to one another,
the triangles ABC, AGB, AHG are also equal to one
IS another. [i. 38]

Therefore, whatever multiple the base HC is of the base

BC, that multiple also is the triangle AHC of the triangle

ABC.

For the same reason,
10 whatever multiple the base ZC is of the base CD, that
multiple also is the triangle ALC of the triangle ACD ;
and, if the base HC is equal to the base CL, the triangle
AHC is also equal to the triangle ACL, [i. 38]

if the base //C is in excess of the base CZ., the triangle AHC
as is also in excess of the triangle A CL,
and, if less, less.

Thus, there being four magnitudes, two bases BC, CD
and two triangles ABC, ACD,

equimultiples have been taken of the base BC and the
30 triangle ABC, namely the base HC and the triangle AHC,
and of the base CD and the triangle Z?C other, chance, equi-
multiples, namely the base LC and the triangle ALC

and it has been proved that,
if the base HC is in excess of the base CL, the triangle AHC
3S is also in excess of the triangle ALC ;
if equal, equal ; and, if less, less.

Therefore, as the base BC is to the base CD, so is the

triangle ABC to the triangle ACD. [v. Def. 5]

Next, since the parallelogram EC is double of the triangle

AoABC, [i. 4>]

and the parallelogram FC is double of the triangle ACD,

while parts have the same ratio as the same multiples of

them, [v. 15]

therefore, as the triangle ABC is to the triangle ACD, so is

45 the parallelogram £C to the parallelogram FC.

Since, then, it was proved that, as the base BC is to CD,
so is the triangle ABC to the triangle A CD,
and, as the triangle ABC is to the triangle ACD, so is the
parallelogram EC to the parallelogram CF,
50 therefore also, as the base BC is to the base CD, so is the
parallelogram EC to the parallelogram FC. [v. n]

Therefore etc.
\end{proof}

\begin{annotations}

4. Under the same height. The Greek text has ``under Ihe iame height AC,'' with
a figure in which the side C commun to the two triangles is perpendicular to the base and
is therefore iudf the ``height.'' But, even if tlie two triangles are placed contiguously so as
to have a commbn side AC, it is quite gratuiicnis to require it to be perpendicular to the base.
Theon, on this occasion making an improvement, altered to `` which are [Ato) under the
same height, (namely! 'he perpendicular drawn from A to BD,'' I iiave vetitured lo alter so
far as lo omit ``AC'' and to draw the figure in the usual way.

14. ABC.AGBiAHG. Euclid,indiaferenttoeitactorder,writes'' AffG, AGB,ABC.''
46. Since then it was proved that, as the base BC i( to CD, *o It the triangle
ABC lo the triangle ACD. Here again words have to be supplied in translatin|> the
eitremely terse Greek irtl a Htlxfj ``• 1''  fii'`` Br rpit Te Ti, otrut ri ABr
Tf/iyuror ri ri ATA Tplywrwr, literSly `` since was proned, as the base BC to CO, to the
truaigle ABC lo Ihe triangle ACJi.'' Cf. note on v. 16, p. i6j.

\end{annotations}

\begin{notes}

The proof assumes — what is however an obvious deduction from 1. 38 —
that, of triangles or parallelograms on unequal bases and between the same
parallels, the greater is that which has the greater base.

It is of course not necessary that the two given triangles should have a
common side, as in the figure ; the proof is just as easy if they have not.
The proposition being equally trtie of triangles and parallelograms of eqital
heights, Simson states this fact in a corollary thus:

`` From this it is plain that triangles and parallelograms that have equal
altitudes are to one another as their bases.

Let the figures be so placed as to have their bases in the same straight
line ; and, if we draw perpendiculars from the vertices of the triangles to the
basw, the straight line which joins the vertices is parallel to that in which
their bases are, because the perpendiculars are both equal and parallel to one
another [i, 33]. Then, if the same construction be made as in the proposition,
the demonstration will be the same.''

The object of placing the bases in one straight line is to get the triangles
and parallelograms within (hi same parallels. Cf. Proclus' remark on i. 38
(p. 405, 17) that having the same height is the same thing as being in the
same parallels.

Rectangles, or right-angled triangles, which have one of the sides about
the right angle of the same length can be placed so that the equal sides
coincide and the others are in a straight line. If then we call the common
side the base, the rectangles or the right-angled triangles are to one another
as their heights, by vi. i. Now, instead of each right-angled triangle or
rectangle, we can take any other triangle or parallelogram respectively with an
equal base and between the same parallels. Thus

Triangles and paralklograms having eguai bases art to one another as their
heights.

Legendie and those authors of tnodem text-books who follow him in
basing their treatment of proportion on the algebraical definition are obliged
to divide their proofs of propositions like this into two parts, the first of
which proves the particular theorem in the case where the magnitudes are
commensurable, and the second extends it to the case where they are
incommensurable.

Lendre (Aliments dt G'eometrie, . 3) uses for this extension a rigorous
method by reductie ad absurdum similar to that
used by Archimedes in his treatise On the
equilibrium 0/ planes 1. 7. The following is
Legendre's proof of the extension of vi, i to in-
commensurable parallelograms and bases.

The proposition having been proved for
commensurable bases, let there be two rectangles
AS CD, AEFD as in the figure, on bases AB,
.£which are incommensurable with one another.

To prove that recL A BCD: recL AEFD =AB: AE.

For, if not, let red. A3CD -rect. AEFD = AB : AO, (ij

where AO h (for instance) greater than AE.

Divide AS into equal parts each of which is less than EO, and mark off
on AO lengths equal to one of the parts; then there will be at least one point
of division between E and O. -r

Let it be /, and draw /AT parallel to EF. *

Then ihe rectangles A BCD, AIKD are in the ratio of the bases AB, A I,
since the latter are commensurable.
Therefore, inverting the proportion,

rect. AIKD:<ixx. ABCD'AI.AB (i).

From this and (i), « atquati,

rect, AIKD : rect. AEFD = AI.AO. ,

But A0> A/; therefore rect. AEBD>Ttci. AIKD.
But this is impossible, for the rectangle AEFD is less than the rectangle
AIKD.

Similarly an impossibility can be proved i AO < AE. >

Therefore lecL ABCD : rect. AEFD = AB : AE.

Some modern American and German text-books adopt the less rigorous
method of appealing to the theory of iimiis.

\end{notes}

\end{proposition}

\begin{proposition}
\label{prop:VI_2}

\begin{statement}
If a straight line be drawn parallel to one of the sides of a
triangle, it will cut the sides of the triangle proportionally ;
and, if Ike sides of the triangle be cut proportionally, the line
joining the points of section will be parallel to the remaining
side of the triangle.
\end{statement}

\begin{proof}

For let DE be drawn parallel to BC, one of the sides of
the triangle ABC;

I say that, as BD is to DA, so is CE to
EA.

For let BE, CD be joined.

Therefore the triangle BD E is equal to
the triangle CDE ;

for they are on the same base DE and in
the same parallels DE, BC. [1. 38]

And the triangle ADE is another area.

But equals have the same ratio to the same ; [v. 7]

therefore, as the triangle BDE is to the triangle ADE so
is the triangle CDE to the triangle ADE,

But, as the triangle BDE is to ADE, so is BD to DA ;

for, being under the same height, the perpendicular drawn
from E to AB, they are to one another as their bases, [vi. i]

For the same reason also,
as the triangle CDE is to ADE, so is CE to EA. '-

Therefore also, as BD is to DA, so is CE to EA. [v. n]

VI. a, 3] PROPOSITIONS 1—3 195

Again, let the Sides AB, AC oi the triangle ABC be cut

proportionally, so that, as BD is to DA, so is C£ to EA ;

and let DE be joined.

I say that DB is parallel to BC - f •'' •

For, with the same construction, i   •''> - •

since, as BD is to DA, so is C£ to £A,

but, as BD is to DA, so is the triangle BDE to the triangle

ADE,

and, as CE is to EA, so is the triangle CDE to the triangle
ADE, [v.. ,]

therefore also,

as the triangle BDE is to the triangle ADE, so is the
triangle CDE to the triangle ADE. [v. n]

Therefore each of the triangles BDE, CDE has the same
ratio to ADE.

Therefore the triangle BDE is equal to the triangle CDE

[V.9]
and they are on the same base DE.

But equal triangles which are on the same base are also
in the same parallels. [i. 39]

Therefore DE is parallel to BC.
Therefore etc.
\end{proof}

\begin{notes}

Euclid evidently did not think it worth while to distinguish in the
enunciation, or in the figure, the cases in which the parallel to the base cuts
the othei two sides produced (a) beyond the point in which they intersect,
(i) m the other direction. Simson gives the three figures and inserts words
in the enunciation, reading ``it shall cut the other sides, or those lidts produced,
proportionally'' and ``if the sides, or the sidts produced, be cut proportionally.''

Todhunter observes that the second part of the enunciation ought to
make it clear which segments in the proportion correspond to which. Thus
e.g., if AD were double of DB, and CE double of EA, the sides would be
cut proportionally, but DE would not be parallel to BC. The omission
could be supplied by saying ``and if the sides of the triangle be cut
proportionally io that tht segments adjacent to the third side «ri corresponding
terms in the proportion.''

\end{notes}

\end{proposition}

\begin{proposition}
\label{prop:VI_3}

\begin{statement}
If an angle of a triangle be bisected and the straight line
cutting the angle cut the base also, the segments of the base
will have the same ratio as the remaining sides of the triangle;
and, if ike segments of the base have the same ratio as the
remaining sides of the triangle, the straight line joined from
the vertex to the point of section will bisect the angle of the
triangle.
\end{statement}

\begin{proof}

Let ABC be a triarle, and let the angle BA C be bisected
by the straight line AD

I say that, as BD is to CD, so
is BA to AC.

For letC be drawn through
C parallel to DA, and let BA
be carried through and meet it

Then, since the straight line
A C falls upon the parallels AD,
EC,

the angle ACE is equal to the angle CAD. [i, 39]

But the angle CAD is by hypothesis equal to the angle
BAD;

therefore the angle BAD is also equal to the angle ACE.

Again, since the straight line BAE falls upon the parallels
AD, EC,

the exterior angle BAD is equal to the interior angle
A EC. [i. 39]

But the angle ACE was also proved equal to the angle
BAD;

therefore the angle A CE is also equal to the angle A EC,

so that the side AE is also equal to the side AC. [i. 6]

And, since AD has been drawn parallel to EC, one of
the sideij of the triangle BCE,

therefore, proportionally, as BD is to DC, so is BA to AE.

But AE is equal to AC; t'- ']

therefore, as BD is to DC, so is BA to AC.

Again, let BA be to Cas BD to DC, and let AD be
joined ;

I say that the angle BAC has been bisected by the straight
line AD.

For, with the same construction,
since, as BD is to DC, so is BA to AC, •'  - i' • <y-

n.3] PROPOSITION 3 197

and also, as BD is to DC, so is BA to AE\ for AD has
been drawn parallel to EC, one of the sides of the triangle
BCE : [VI. 2]

therefore also, as BA is to A C, so is BA to AE. [v. i r]

Therefore AC is equal to AE, [v. 9]

so that the angle A EC is also equal to the angle ACE. [i. 5]

But the angle A EC is equal to the exterior angle BAD,

[I. jg]
and the angle ACE is equal to the alternate angle CAD; ['d.]

therefore the angle BAD is also equal to the angle CAD.

Therefore the angle C has been bisected by the straight
line AD.

Therefore etc. •' •
\end{proof}

\begin{notes}

The demonstration assumes that C£ will meet BA produced in some
point £. This is proved in the same way as it is proved in vi. 4 that BA, ED
will meet if produced. The angles ABD, SDA in the figure of vc, 3 are
together less than two right angles, and the angle BDA is equal to the angle
BCE, since DA, CE are parallel. Therefore the angles ABC, BCE are
together less than two right angles ; and BA, CE must meet, by 1. Post. 5.

The corresponding proposition about the segments into which C is
divided externally by the bisector of the external angU at A when that
bisector meets BC produced (i.e. when the sides AB, AC ak not equal) is
important. Simson gives it as a separate proposition, A, noting the fact that
Pappus assumes the result without proof (Pappus, vii. p, 730, 24).

The best plan ts however, as De Morgan says, to combine Props. 3 and A
in one proposition, which may be enunciated thus : If an angle of a triangle
be bisected internally or externally by a straight line which cuts the opposite side
or the opposite side produced, the segments of that side will have the same ratio
as the other sides of the triangle; and, if a side of a triangle be divided internally
or externally so that its segments have the same ratio as the other sides of the
triangle, the straight line drawn from the point of section to the angular point
which is opposite to the first ?nentiomd side will bisect the interior or exterior angle
at that angular point.

Let ..4 C be the smaller of the two sides AB, AC, so that the bisector AD
of the exterior angle at A may meet BC produced beyond C. Draw CE
through C parallel to DA, meeting BA in E.

Then, if EA C is the exterior angle bisected by AD in the case of external
bisection, and if a point Eis taken on AB io the figure of vi. 3, the proof of

VL 3 can be used almost word for word for the other ease. We have only to
spak of the angle ``/C'' for the angle `` BAC,'' and of the angle ``FAD''
for the angle `` BAD `` wherever they occur, to say ``let SA or BA produced,
meet CE in E,'' and to substitute `` BA or BA produced'' for ``BAE''
lower down.

. h

If AD, AE be the internal and external bisectors of the angle A in a.
triangle of which the sides AB, AC are unequal, AC being the smaller, and
if AD, AE meet BC and BC produced n D, E respectively,

the ratios of BD to i>Cand of BE to EC are alike equal to the ratio of
BA to AC.

Therefore BE is to ECtis BD to DC,

that is, BE is to EC as the difference between BE and ED is to the
difference between ED and EC,

whence BE, ED, EC are in karmonit prograsion, or DE is a harmonic mtan
between BE and EC, or again B, D, C, £ is a harmnu range.

Since the angle DAC is half of the angle BAC,

and the angle CAE half of the angle CAF,
while the angles BAC, CAF are equal to two right angles,
the angle DAE is a right angle.

Hence the circle described on DE as diameter passes through A.

Now, if the ratio of BA to Cis given, and if BC is given, the points
D, E on BC and BC produced are given, and therefore so is the circle on
D, E as diameter. Hence M« /acus of a point sueh that its dittanas from two
given points are in a given ratio (net being a ratio of equality) is a rirck.

This locus was discussed by ApoUonius in his Plane Loci, Book ii., as we
know fr«m Pappus (vii, p. 666), who says that the book contained the
theorem that, if from two given jwjnts straight lines inflected to another
point are in a given ratio, the point in which they meet will lie on either a
straight line or a circumference of a circle. The straight line is of course the
locus when the ratio is one of equality. The other case is quoted in the
following form by Eutocius (ApoUonius, ed. Heiberg, ti. pp. \ 80—4).

Given two points in a plane and a proportion between unequal strwght lines,
it is possible to describe a circle in the plane so that the straight lines inflected
from the given points to the dreumference of the circle shall have a ratio the
tame as the given one.

ApoUonius' construction, as given by Eutocius, is remarkable because he
makes no use of either of the points D, E. He finds 0, the centre of the
required circle, and the length of its radius directly from the data BC and the
given ratio which we will call h : k. But the construction was not discovered
by ApoUonius j it belongs to a much earlier date, since it appears in exactly

the same rorm in Aristotle, Mekorolegica in. 5, J76 a 3 sqq. The
analysis leading up to the construction is, as usual, not given either by
Aristotle or Eutocius. We are told to take three straight lines x, CO (a
length measured along BC produced beyond C, where J, C are the points at
which the greater and smaller of the inflected lines respectively terminate),
and r, such that, if hkhi the given ratio and h>k,

k:h = hik + x, (a)

  '' x:SC=k:CO = h:r 08)

This determines the position of O, and the length of r, the radius of the
required circle. The circle is then drawn, any point P is taken on it and
joined to B, C respectively, and it is proved that

FB'.PCh.k.

We may conjecture that the analysis proceeded somewhat as follows.

Ft would be seen that £, C are ``conjugate points'' with reference to the
circle on DE as diameter. (Cf. ApoUonius, Cmks, i. 36, where it is proved,
in terms, for a circle as well as for an ellipse and a hyperbola, that, if the
polar of j5 meets the diameter DE in C, then EC: CD = EB : BD.)

If O be the middle point of DE, and therefore the centre of the circle,
D, E may be eliminated, as in the Conies, i. 37, thus.

Since EC : CD = EB : BD,

it follows that EC+CD: EC~ CD = EB  ¥ BD -.EB- BD,

or iOD : 20C= tOB : tOD, '«  '

that is, BO.OC= OD' = r', say.

If therefore B be any point or the circle with centre O and radius r,

BO: OP=OP:OC,

so that 50/'', PO Care similar triangles. ., .,, , ...

a.3A6:*:on,h:k-BD:DC = BE-EC •• 11 • •

BD + B£:D£ = BO:r. •• •'•  -'``'

Hence we require that

BO:r = r:OC=BP:PC=h.k (8)

Therefore, alternately,

k:COh:r,

which is the second relation in () above. '• '

Now assume a length x such that each of the last ratios is equal axBC,
as in ().

Then • . x:BC-k:CO = h:r.

Therefore .r + A : BO -h:r,

and, alternately, x + k:h = BO : r

I =hik, from (S) above ;

and this is the relation (a) which remained to be found.

ApoUonius' proof of the construction is given by Eutocius, who begins by
saying that it is manifest that r is a mean proportional between BO and OC.
This IS seen as follows .
From (j8) we derive

xBC=k: CO = A:r = (k-yx):BO,
whence BO  .r=(k + x):h

m:A:A, by <a),
= r:CO, by(),
and therefore r* = BO . CO.

But the triangles BOP, POC have the angle at common, and, since
BO: OP = OF: OC, the triangles are similar and the angles OPC, OBP
are equal

[Up to this point Aristotle's proof is exactly the same ; from this point it
diverges slightly.]

If now CL be drawn parallel to BP meeting OP in L, the angles BPC
ZCP are equal also.

Therefore the triangles BPC, PCI. are similar, and

BP:PC=PC:CL,

whence BP.PCBP: CL

<= BO : OC, by parallels,

= BO : OP* (sincere ; OP= OP: OC).

Therefore BF:PCBO:OP

= A;*(for OP=-r).

[Aristotle infers this more directly from the similar triangles P03, COP.
Since these triangles are similar,

OP: CP=OB:BP,

whence BP: PC'' BO :0P ``•

= h:k.'

ApoUonius proves lastly, by reductio ad ahsurdum, that the last equation
cannot be true with reference to any point P which is not on the circle so
described.

\end{notes}

\end{proposition}

\begin{proposition}
\label{prop:VI_4}

\begin{statement}
In equiangular triangles the sides about (he equal angles
are proportional, and those are corresponding sides which
subtend the equal angles.
\end{statement}

\begin{proof}

Let ABC, DCE be equiangular triangles having the
angle ABC equal to the angle
DCE, the angle BAC to the
angle CDE, and further the angle
ACB to the angle CED ;

I say that in the triangles ABC,
DCE the sides about the equal
angles are proportional, and those
are corresponding sides which
subtend the equal angles.

For let BC be placed in a
straight line with CE.

Then, since the angles ABC, ACB are less than two right
angles, [i, 17]

and the angle ACB is equal to the angle DEC,

therefore the angles ABC, DEC are less than two right

angles ;

therefore BA, ED, when produced, will meet. [i. Post. 5]

Let them be produced and meet at E,

Now, since the angle DCE is equal to the angle ABC,

BF is parallel to CD. [1. *8]

Again, since the angle ACB is equal to the angle DEC,

AC is parallel to FE. [i. 38]

Therefore FACD is a parallelogram ;

therefore FA is equal to DC, and AC to FD. fi. 34]

And, since AC has been drawn parallel to FE, one side
of the triangle FBE,

therefore, as BA is to AF, so is BC to CE. [vi. 1]

But AF is equal to CD ;

therefore, as BA is to CD, so is BC to CE,

and alternately, as AB is to BC, so is DC to CE. [v. 16]

'1 Again, since CD is parallel to BF, '

therefore, as BC is to CE, so is FD to DE. [vi. »]

But FD is equal to AC;

therefore, as BC is to CE o\ AC to DE,

and alternately, as BC is to CA, so is CE to ED. [v. r6]

 M3 BOOK VI [VI. 4, s

Since then it was proved that, '

as AB is to BC, so is DC to CE,
and, as BC is to CA, so is CE to ED ;

therefore, ex aeguali, as BA is to C, so is CD to DE. [v. sa]

Therefore etc.

Q, E. D,
\end{proof}

\begin{notes}

Todhunter remarks that `` the manner in which the two triangles are to be
placed is very imperfectly described; their bases are to be in the same straight
line and contiguous, their vertices are to be on the same side of the base, and
each of the two angles which have a common vertex is to be equal to the
remote angle of the other triangle,'' But surely Euclid's description is
sufficient, except for not saying that B and D must be on the same side
of BCE.

VI. 4 can be immediately deduced from vi. a if we superpose one triangle
on the other three times in succession, so that each angle successively
coincides with its equal, the triangles being similarly situated, e.g. if (A, B, C
and D, E, F being the equal angles respectively) we apply the angle DEFio
the angle ABC so that D lies on AB (produced if necessary) and Jon BC
(produced if necessary). De Morgan prefers this method. `` Abandon,'' he
says, `` the peculiar mode of construction by which Euclid proves two cases at
once; make an angle coincide with its equal, and suppose this process repeated
three times, one for each angle.''

\end{notes}

\end{proposition}

\begin{proposition}
\label{prop:VI_5}

\begin{statement}
If two triangles have their sides proportional, the triangles
will be equiangular and will have those angles equal which the
corresponding sides subtend.
\end{statement}

\begin{proof}

Let ABC DEF be two triangles having their sides
proportional, so that,

as AB is to BC, so is DE to EF,

sis BC is to C A, so is EF to FD, • '

and further, as BA is to AC, so is ED to DF;

I say that the triangle ABC is equiangular with the triangle
DEE, and they will have those angles equal which the corre-
sponding sides subtend, namely the angle ABC to the angle
DEE, the angle BCA to the angle EFD, and further the
angle BAC to the angle EDF.

For on the straight line EF, and at the points E, F on
it, let there be constructed the angle FEG equal to the angle
ABC, and the angle EFG equal to the angle A CB ; [i. aj]

therefore the remaining angle at A is equal to the remaining
angle at G. [t- 3*]

Therefore the triangle ABC is equiangular with the
triangle G£F.

Therefore in the triangles ABC, GEF the sides about
the equal angles are proportional, and those are corresponding
sides which subtend the equal angles ; [vi. 4]

therefore, as AB is to BC, so is GE to EF.

But, as AB is to BC, so by hypothesis is DE to EF

therefore, as DE is to EF, so is GE to EF. [v. n]

Therefore each of the straight lines DE, GE has the
same ratio to EF;

therefore DE is equal to GE. [v. g]

For the same reason

DF is also equal to GF.

Since then DE is equal to EG,

and EF is common,

the two sides DE, EF are equal to the two sides GE, EF;

and the base DF is equal to the base EG ;

therefore the angle DEF is equal to the angle GEF, [1. 8]

and the triangle DEF is equal to the triangle GEF,

and the remaining angles are equal to the remaining angles,
namely those which the equal sides subtend. [1. 4]

Therefore the angle DEE is also equal to the angle GFE,

and the angle EDF to the angle EGF.

And, since the angle FED is equal to the angle GEF,
while the angle GEF is equal to the angle ABC,
therefore the angle ABC is also equal 10 the angle DEF.

For the same reason

the angle ACB is also equal to the angle DFE ``
and further, the angle at A to the angle at D ;

therefore the triangle ABC is equiangular with the triangle
DBF.

Therefore etc.
\end{proof}

\begin{notes}

This proposition is the complete converse, vi. 6 a partial converse, of vi. 4.
Todhuntcr, after Walker, remarks that the enunciation should make it
clear that the sides of the triangles laken in order are proportional. It is quite
f»osible that there should be two triangles ABC, Z'£/'such that

AB is to .SCas DE to EF,
and .SCisto Cas DF'xs, to EJ> (instead of ito/Z)),

so that A3 is to AC as J?Fto EF

(fx aequali n ptrlurbed prepartion)

in this case the sides of the triangles are proportional, but not in the same
order, and the triangles are not necessarily equiangular to one another. For a
numerical illustration we may suppose the sides of one triangle to be 3, 4 and
5 feet respectively, and those of another to be 11, 15 and 10 feet respectively.
In VI. 5 there is the same apparent avoidance of indirect demonstration
which has been noticed on t. 48. i • . .m u 1   t '

\end{notes}

\end{proposition}

\begin{proposition}
\label{prop:VI_6}

\begin{statement}
If two triangles have one angle equal to one angle and the
sides about the equal angles proportional, the triangles will be
equiangular and will have those angles equal which the corre-
sponding sides subtend.
\end{statement}

\begin{proof}

Let ABC, DEF be two triangles having one angle BAC
equal to one angle EDF and the sides about the equal angles
proportional, so that,

as BA is to AC, so is ED to DF
I say that the triangle ABC is equiangular with the triangle
DEF, and will have the angle ABC equal to the angle DEF,
and the angle ACB to the angle DFE.

For on the straight line DF, and at the points D, Fon it,
let there be constructed the angle FDG equal to either of the
angles BAC, EDF, and the angle DFG equal to the angle
ACB; [1.23]

therefore the remaining angle at B is equal to the remaining
angle at G, -. ... . , » .. [•• 3']

Therefore the triangle ABC is equiangular with the
triangle DGF.

Therefore, proportionally, as BA is to AC, so is GD to

DF. [vi. 4]

But, by hypothesis, a.sBA is to AC, so also is £/? to I)F;

therefore also, as FD is to /)F, so Is GD to DF. [v. n]

Therefore £D is equal to BG ; • • [- j

and Z?/ is common ; 1 •

therefore the two sides £D, DFare equal to the two sides
GD, DF; and the angle EDF is equal to the angle GDF;

therefore the base EF is equal to the base GF, '

and the triangle DEF is equal to the triangle DGF,

and the remaining angles will be equal to the remaining angles,
namely those which the equal sides subtend. [i. 4]

Therefore the angle DFG is equal to the angle DFE,
and the angle DGF to the angle DEF.
But the angle DFG is equal to the angle ACB; ; 1 li--
therefore the angle ACB is also equal to the angle DFE.

And, by hypothesis, the angle BAC is also equal to the
angle EDF;

therefore the remaining angle at B is also equal to the
remaining angle at E\    y .1 ['-3*]

therefore the triangle ABC is equiangular with the triangle
DEF.

Therefore etc. • < cj.   .  '

'   -' .  1   *• • ~ ; >
\end{proof}

\end{proposition}

\begin{proposition}
\label{prop:VI_7. '•' •'•' '.f' }

\begin{statement}
1/ two triangles have one angle equal to one angle, the
sides about other angles proportional, and th-e remaining angles
either both less or both not less than a right angle, the triangles
will be equiangular and will have those angles equal, the stdes
about which are proportional.
\end{statement}

\begin{proof}

Let ABC, DEFhe two triangles having one angle equal
to one angle, the angle BAC to
the angle EJDF, the sides about
other angles ABC, DEF propor-
tional, so that, as AB ts to BC,
so is DE to EF, and, first, each
of the remaining angles at C, F
less than a right angle ;

I say that the triangle ABC is

equiangular with the triangle

DEF, the angle ABC will be

equal to the angle DEF, and the remaining angle, namely

the angle at C, equal to the remaining angle, the angle

at F.

For, if the angle ABC is unequal to the angle DEF, one
of them is greater.

Let the angle ABC be greater ;

and on the straight line AB, and at the point B on it, let the
angle ABG be constructed equal to the angle DEF. [i. 33]

Then, since the angle A is -equal to D, a .:. ui
and the angle ABG to the angle DEF,

therefore the remaining angle A GB is equal to the remaining
angle DFE. [i. jj]

Therefore the triangle ABG is equiangular with the
triangle DEF.

Therefore, as AB is to BG, so is DE to EF [vi. 4]

But, as DE is to EF, so by hypothesis is AB to BC

therefore AB has the same ratio to each of the straight
lines BC, BG ; [v. 11]

therefore BC is equal to BG, [v. 9]

so that the angle at C is also equal to the angle BGC. [i. 5]

But, by hypothesis, the angle at C is less than a right

angle ;

therefore the angle BGC is also less than a right angle ;

so that the angle A GB adjacent to it is greater than a right
angle. [1. 13]

And it was proved equal to the angle at F
therefore the angle at F'  also greater than a right angle.

But it is by hypothesis less than a right angle : which is
absurd.

Therefore the angle ABC is not unequal to the angle
DEF

therefore it is equal to it.

But the angle at A is also equal to the angle at D ;

therefore the remaining angle at C is equal to the remaining
angle at F, [i. 3a]

Therefore the triangle ABC is equiangularwith the triangle
DEF.

But, again, let each of the angles at C, F be supposed not
less than a right angle ;
1 say again that, in this case too, the
triangle ABC is equiangular with the
triangle DEF.

For, with the same construction,
we can prove similarly that

BC is equal to BG

so that the angle at C is also equal to
the angle BGC. y s]

But the angle at C is not less than a right angle ;
therefore neither is the angle BGC less than a right angle.

Thus in the triangle BGC the two angles are not less
than two right angles : which is impossible, [i- 17]

Therefore, once more, the angle ABC is not unequal to
the angle DEF;

therefore it is equal to it.

But the angle at A is also equal to the angle at D ;

therefore the remaining angle at C is equal to the remaining
angle at F, [i, 33)

Therefore the triangle ABC is equiangular with the triangle
DEF.

Therefore etc.  .   ;-
\end{proof}

\begin{notes}

Todhunter points out, after Walker, that some more words are necessary
to make the enunciation precise : ``If two triangles have one angle equal to one
angle, the sides about other angles proportional <so that tht sides sttbttnding
ihe equal anglts are homologous. ...''

This proposition is the extension to similar triangles of the ambiguous ase
already mentioned as omitted by Euclid in relation to equality of triangles in
all respects (cf. note following i. 26, Vol, 1. p. 306). The enunciation of vi. 7
has suggested the ordinary method of enunciating the ambiguous (ase where
equality and not similarity is in question. Cf. Todhunter's note on 1. 26,

Another possible way of presenting this proposition is given by Todhunter,
The essential theorem to prove is :

ff two triangles have two sides of the one proportional to two sides of the
other, and the angles opposite to one pair of corresponding sides equal, the angles
which are opposite to the other pair of €orresponding sides shell either e equal or
lie together equal to two right angles.

For the angles included by the proportional sides must be either equal or
uneqtiaL

If they are equal, then, since the triangles have two angles of the one
equal to two angles of the other, respectively, they are equiangular to one
another.

We have therefore only to consider the case in which the angles included
by the proportional sides are unequal.

The proof is, except at the end, like that of vi. 7.

Let the triangles ABC, DEF have the angle at A equal to the angle at D ;
let AB be to BC as DE to EF,
but let the angle ABCht not equal to the angle DEF.

The angles ACB DFE shall be together equal to two right angles.

For one of the angles ABC, DEF must be the greater.

Let ABC\ the greater; and make the angle ABG equal to the angle
DEF

Then we prove, as in vi. 7, that the triangles ABG, DEF are equiangular,
whence

AB is to BG as DE is to EF.

But AS is to BC as DE is to EF, by hypothesis.

Therefore BG is equal to BC,

and the angle BGC is equal to the angle BCA.

VI. 7. 8]

Now, since the triangles ABG, DMF'wfi equiangular,

the angle SGA is equal to the angle EFD,
Add to them respectively the equal angles BGC, EC A; therefore the
angles BCA, EFD are together equal to the angles BGA, BGC, i.e. to two

right angles.

It follows therefore that the angles BCA, EFD must be either equal or

supplementary.

But (i), if each of them is less than a right angle, they cannot be
supplementary, and they must therefore be equal ;

(2) if each of them is greater than a right angle, Utey cannot be
supplementary and must therefore be equal;

(3) if one of them is a right angle, they are supplementary and also equal.

Simson distinguishes the last case (3) in his enunciation : ``then, if each of
the remaining angles be either less or not less than a right angle, or if ont 0/
ihtm be a rigAt angle,., ,''

The change is right, on the principle of lestricring the conditions to the
minimum necessary to enable the conclusion to be inferred. Simson adds a
separate proof of the case in which one of the remaining angles is a right
angle.

`` Lastly, let one of the angles at C, F, viz. the angle at C, be a right angle;
in this case likewise the triangle ABC
IS equiangular to the triangle DEF,

For, if they be not equiangular,
make, at the point B of the straight
line AB, the aile ABG equal to the
angle DEF\ then it may be proved,
as in the first case, that BG vi equal
vaBC.

But the angle BCG is a right
angle;

therefore the angle BGC is also a
right angle;

whence two of the angles of the tri-
angle BGC .Tc together not less than
two right angles : which is impossible.
Therefore the triangle ABC is equiangular to the triangle DEF.''

\end{notes}

\end{proposition}

\begin{proposition}
\label{prop:VI_8}

\begin{statement}
1/ in a rigkt-anghd triangle a perpendicular be drawn
from ike right angle to the base, ike triangles adjoining ike
perpendicular are similar both to the whole and to one another.
\end{statement}

\begin{proof}

Let ABC be a right-angled triangle having the angle
BAC right, and let AD be drawn from A perpendicular
toC;

I say that each of the triangles ABD, ADC is similar to
the whole ABC and, further, they are similar to one another.

For, since the angle BAC is equal to the angle ADB,
for each is right,

and the angle at B is common to the
two triangles ABC and ABD,
therefore the remaining angle ACB
is equal to the remaining angle
BAD ; [.. 3]

therefore the triangle ABC is equi-
angular with the triangle ABD.

Therefore, as BC which subtends the right angle in the
triangle ABC is to BA which subtends the right angle in
the triangle ABD, o  AB itself which subtends the angle
at C in the triangle ABC to BD which subtends the equal
angle BAD in the triangle ABD, and so also  AC to AD
which subtends the angle at B common to the two triangles.

[v,.4]

Therefore the triangle ABC is both equiangular to the
triangle ABD and has the sides about the equal angles
proportional.

Therefore the triangle ABC is similar to the triangle
ABD. [VI. Def. i]

Similarly we can prove that
the triangle ABC is also similar to the triangle ADC ;
therefore each of the triangles ABD, ADC is similar to the
whole ABC.

I say next that the triangles ABD, ADC are also similar
to one another.

For, since the right angle BDA is equal to the right angle
ADC,

and moreover the angle BAD was also proved equal to the
angle at C,

therefore the remaining angle at B is also equal to the
remaining angle DAC; [»• 32]

therefore the triangle ABD is equiangular with the triangle
ADC.

Therefore, as BD which subtends the angle BAD in the
triangle ABD is to DA which subtends the angle at C in the
triangle ADC equal to the angle BAD, so is AD itself
which subtends the angle at B in the triangle ABD to DC
which subtends the angle DAC in the triangle ADC equal

to the angle at B, and so also is BA to AC, these sides
subtending the right angles ; [vi. 4]

therefore the triangle ABD is similar to the triangle ADC.

[vi. Def. i]

Therefore etc.

PoRiSM. From this it is clear that, if in a right-angled
triangle a perpendicular be drawn from the right angle to the
base, the straight Hne so drawn is a mean proportional
between the segments of the base.
\end{proof}

\begin{notes}

Simson remarks on this proposition : ``It seems plain that some editor
has changed the demonstration that Euclid gave of this proposition : For,
after he has demonstrated that the triangles are equiangular to one another,
he particularly shows that their sides about the equal angles are proportionals,
as if this had not been done in the demonstration of prop, 4 of this book :
this superfluous part is not found in the translation from the Arabic, and is
now left out.''

This seems a little hypercritical, for the ``particular showing'' that the
sides about the e<jual angles are proportionals is really nothing more than
a somewhat full citation of vi. 4. Moreover to shorten his proof still
morci Simson says, after proving that each of the triangles ABD, ADC is
similar to the whole triangle ABC, ``And the triangles ABD, ADC being
both equiangular and similar to ABC are equiangular and similar to one
another,'' thus assuming a particular case of vi. zi, which might well be
proved here, as EucSid proves it, with somewhat more detail.

We observe that, here as generally, Euclid seems to disdain to give the
reader such small help as might be afforded by arranging the letters used to
denote the triangles so as to show the corresponding angular points in the
same order for each pair of triangles ; A is the first letter throughout, and the
other two for each triangle are in the order of the figure from left Co righL It
may be in compensation for this that he states at such length which side
corresponds to which when he comes to the proportions.

In the Greek texts there is an addition to the Porim inserted after
``(Being) what it iras required to prove,'' viz. ``and further that between the
base and any one of the segments the side adjacent to the sment is a mean
proportional'' Heiberg concludes that these words are an interpolation
(i) because they come after the words wrcp iSh Stifm which as a rule follow the
Porism, (2) they are absent from the best Theonine MSS., though P and
Campanus have them without the wip (£« Stufiu. Heiberg's view seetns to
be confirmed by the fact noted by Austin, that, whereas the first part of the
Porism is quoted later in vi. 15, in the lemma before x. 33 and in the lemma
after xin. 13, the second part \ prmted vci the former lemma, and elsewhere,
as also in Pappus (in. p. 72, 9— »3).

\end{notes}

\end{proposition}

\begin{proposition}
\label{prop:VI_9}

\begin{statement}
From a given straight line to cut off a prescribed part.
Let AB be the given straight line ;
thus it is required to cut off from AB a prescribed part
\end{statement}

\begin{proof}

Let the third part be that prescribed.
S Let a straight line AC he drawn through from A con-
taining with AS any angle ;

let a point /? be taken at random on
AC, and let DB, EC be made equal
to AD. [1- 3]

10 Let Cbe joined, and through D
let DF be drawn parallel to it. [i- 31]
Then, since FD has been drawn
parallel to BC, one of the sides of the triangle ABC,

therefore, proportionally, as CD is to DA, so is BFto FA.

1 • [n. a]

•S But CD is double of DA ;

therefore BF is also double of FA ; -

therefore BA is triple of AF.

Therefore from the given straight line AB the prescribed
third part AF has been cut off.

Q.E.F.
\end{proof}

\begin{annotations}

6. any angle. The exptessian here and in the two foltowing propositions is ruxoura
yurtu, corresponding exactly to Tirjir (Ttjmiw which I have Iransfated u ``« point (taken)
at randffm'' but *'an angle (talten) at random'' would not be so appropriate where it is a
question, not of taking any angle at all, but of drawing a straight line casuAlly so as to make
any angle with another straight line.

\end{annotations}

\begin{notes}

Simson observes that `` this b demonstrated in a parti cuUt case, viz. that
in which the third part of a straight line is required to be cut off; which b
not at all like Euclid's inanner. Besides, the author of that demonstration,
from four magnitudes being proportionals, concludes that the third of them is
the same multiple of the fourth which the first is of the second ; now this is
nowhere demonstrated in the sth book, as we now have it ; but the editor
assumes it from the confused notion which the vulgar have of proportionals.''

The truth of the assumption referred to is proved by Simson in hb
proposition D given above (p, laS); hence he is
able to supply a general and legitimate proof
of the present proposition. A

`` Iet AB be the given straight line ; it b
required to cut oW any part from it.

From the point A draw a straight line AC
making any angle with AB; in AC take any
point 2>, and take 4 C the same multiple of AD
that AB is of the part which is to be cut off
from it ;

join SC, and draw jD£ parallel to it ;

then A Eh the part required to be cut off.

Because ED is parallel to one of the sides of the triangle ABC, 'a, to BC,
as CD is to DA, so is BE to EA, [vi. 2]

and, eomponaido,

CA is to AD, as S A to AE. [v. 1 8]

But CA is a multiple of AD ;
therefore BA is the same multiple of AE. [Prop. D]

Whatever part therefore AD is of AC, AE is the same part of AB ;
wherefore from the straight line AB the part required is cut off.''

The use of Simson's Prop. D can be avoided, as noted by Camerer after
Baermann, in the following way. We first prove, as above, that
CA is to AD as BA is to AE.
Then we infer that, alternately,

CA is to BA as AD to AE. [v. 16]

But AD is to AE as n . AD to n . AE

(where n is the number of times that AD  contained m AC); [v, 15]

whence ACkIo AB as n . AD is to « . AE. [v, 1 1]

In this proportion the first term is equal to the third ; therefore [v. 14]

the second is equal to the fourth,

so that AB is equal to n times AE.
Prop. 9 is of course only a particular case of Prop 10. ``

\end{notes}

\end{proposition}

\begin{proposition}
\label{prop:VI_10}

\begin{statement}
To cut a given uncut straight line similarly to a given cut
straight line.
\end{statement}

\begin{proof}

Let AB be the given uncut straight line, and AC the
straight line cut at the points D,
E ; and let them be so placed as
to contain any angle ;
let CB be joined, and through D,
E let DF, EG be drawn parallel
to BC, and through D let DHK
be drawn parallel to AB. [1, 31]

Therefore each of the figures
FH, HB is a parallelogram ;
therefore DH is equal to FG and HK to GB. [i. 34]

Now, since the straight line HE has been drawn parallel
to KC, one of the sides of the triangle DKC,

therefore, proportionally, as CE is to ED, so is KH to HD.

[vi. j]

 But KH is equal to BG, and HD to GF;   '

therefore, as C£ is to ED, so is 3G to GF,

Again, since FD has been drawn parallel to GF, one ot
the sides of the triangle j4 GF,
therefore, proportionally, as FD is to DA, so is GF to FA.

[vt. j]

But it was also proved that,

as CF is to FD, so is FG to GF;
therefore, as CF is to FD, so is BG to GF,

and, as FD is to /?, so is GF to /.
Therefore the given uncut straight line AB has been cut
similarly to the given cut straight line AC.

Q.E.F.
\end{proof}

\end{proposition}

\begin{proposition}
\label{prop:VI_11}

\begin{statement}
' - To two given straight lines to find a third proportional.
\end{statement}

\begin{proof}

Let BA, AC be the two given straight lines, and let
them be placed so as to contain any
angle ;

thus it is required to find a third pro-
portional to BA, AC.

For let them be produced to the
points D, E, and let BD be made equal
toC; [t. 3]

let BC be joined, and through D let DF
be drawn parallel to it. [i. 31]

Since, then, BC has been drawn
parallel to DF, one of the sides of the triangle ADE,
proportionally, as AB is to BD, so is C to CF. [vi. aj

But BD is equal to AC;
therefore, as AB is to AC, so is AC tc CE.

Therefore to two given straight lines AB, AC 3. third

proportional to them, CF, has been found,

Q.E.F.
\end{proof}

\begin{annotations}

I. to And. The Greek word, bat and in tbe Bext two piopoHttom, is ir pnirivptir,
liter*] I7 ``to find in addilien.''

\end{annotations}

\begin{notes}

This proposition is again a particular case of the succeeding Prop, i »,
Given a ratio between straight lines, VI, ii enables us to find the ratio
which i its duplicate;.

\end{notes}

\end{proposition}

\begin{proposition}
\label{prop:VI_12}

\begin{statement}
To three given straight lines to find a fourth proportional.
Let A, B, C be the three given straight lines ;
thus it is required to find a fourth proportional to A, B, C.
\end{statement}

\begin{proof}

Let two straight lines DE, DF be set out containing any
angle EDF ;

let DG be made equal to A, GE equal to B, and further DH
equal to C

let GH be joined, and let EF be drawn through E parallel
to it. [1. 31]

Since, then, GH has been drawn parallel to EF, one of
the sides of the triangle DEF,
therefore, as DG is to GE, so is DH to HF. [vi. a]

But DG is equal to A, GE to B, and DH to C ;
therefore, as A is to B, so is C to HF.

Therefore to the three given straight lines A,B,Ca. fourth
proportional HF has been found.

Q.E.F.
\end{proof}

\begin{notes}

We have here the geometrical equivalent of the `` rule of three.''

It is of course immaterial whether, as iti Euclid's proof, the first and
second straight lines are measured on one of the lines forming the angle and
the third on the other, or the first and third are measured on one and the
second on the other.

If it should be desired that the first and the required fourth be measured
on one of the lines, and the second and third on
the other, we can use the following construction.
Measure -DE on one straight line equal to A, and
on any other straight line making an angle with
the first at the point D measure I>F equal to £,
and DG equal to C, Join B.F, and through G
draw GJf an/i-fiamiltl to EF, le. make the angle
DGH equal to the angle DEF; let GH meet
DE (produced if necessary) in H.

DHis then the fourth proportional.

For the triangles EDF, GDH are similar, and the sides about the equal
angles are proportional, so that

DE is to DFa DG to DH,

OT  is to  as C to DJf,

\end{notes}

\end{proposition}

\begin{proposition}
\label{prop:VI_13}
To two given strai
\begin{statement}ght lines to find a mean proportional.

Let AB, BC be the two given straight lines ;
thus it is required to find a mean
proportional to AB, BC.
\end{statement}

\begin{proof}

Let them be placed in a straight
line, and let the semicircle ADC be
described or\ AC

let BD be drawn from the point B at
right angles to the straight line AC, ,

and let AD, DC be joined.

Since the angle ADC is an angle in a semicircle, it is
right. [ill. 31]

And, since, in the right-angled triangle ADC, DB has
been drawn from the right angle perpendicular to the base,
therefore DB is a mean proportional between the segments of
the base, AB, BC. [vi. 8, For.]

Therefore to the two given straight lines AB BC a mean
proportional DB has been found,

Q.E.F.
\end{proof}

\begin{notes}

This proposition, the Book vi. version of ii. 14, is equivalent to the
extraction of the square root. It further enables us, given a ratio between
straight lines, to find the ratio which is its sub-dupiieate, or the ratio of which
it is duplicate. , -

\end{notes}

\end{proposition}

\begin{proposition}
\label{prop:VI_14}

\begin{statement}
In equal and equiangular parallelograms the sides about
the equal angles are reciprocally proportional ; and equiangular
parallelograms in which the sides about (he equal angles are
reciprocally proportional are equal
\end{statement}

\begin{proof}

 • Let AB, BC be equal and equiangular parallelograms
having the angles at B equal, and
let DB, BE be placed in a straight
line ;

therefore FB, BG are also in
a straight line. [i. m]

I say that, in AB, BC, the
sides about the equal angles are
reciprocally proportional, that is to
say, that, as DB is to BE, so is
GB to BF.

For let the parallelogram FE be completed.

Since, then, the parallelogram AB is equal to the parallelo-
gram BC-,

and FE is another area,
therefore, as AB is to FE, so is BC to FE. [v. 7]

But, as W is to FE, so is DB to BE, [vi, i]

and, as BC is to FE, so is GB to BF, id

therefore also, as DB is to BE, so is GB to BF. [v. n]

Therefore in the parallelograms AB, BC the sides about
the equal angles are reciprocally proportional.

,i Next, let GB be to BF as DB to BE-, .n

I say that the parallelogram AB is equal to the parallelogram
BC.

For since, as DB Is to BE, so is GB to BF,
while, as DB is to BE, so is the parallelogram AB to the
parallelogram FE, [vi. i]

and, as GB is to BF, so is the parallelogram BC to the
parallelogram FE, [vi. i]

therefore also, as AB is to FE, so is C to FE ; [v. n]

therefore the parallelogram AB is equal to the parallelogram
BC ...  ,. . . [V.9]

Therefore etc.
\end{proof}

\begin{notes}

De Morgan says upon this proposition : `` Owing to the disjointed manner
in which Euctid treats compound ratio, this prrvpositton is strangely out of
place. It is a particular case of vi, 23, being that in which the ratio of the
sides, compounded, gives a ratio of equality. The proper definition of four
mapiitudes being reciprocally proportional is that the ratio compounded of
thetr ratios is that of equality.''

It IS true that vi. 14 is a particular /case of vi. 23, but, if either is out of
platt, it is rather the latter that should be placed before vi. 14, since most of
the propositions between vi, 15 and vi. 23 depend upon vi. 14 and 15. But
is perfectly consistent with Euclid's manner to give a particular case first
and its extension later, and such an arrangement often has great advantages
in that it enables the more difficult parts of a subject to be led up to more
easily and gradually. Now, if De Morgan's view were here followed, we
should, as it seems to me, be committing the mistake of explaining what is
relatively easy to understand, viz, two ratios of which one is the inverse of
the other, by a more complicated conception, that of compound ratio. In
other words, it is easier for a learner to realise the relation indicated by the
statement that the sides of equal and equiangular parallelograms are ``recipro-
cally proportioral `` than to form a conception of parallelograms such that
`` the ratio compounded of the ratio of their sides is one of equality.'' For
this' reason I would adhere to Euclid's arrangement.

The conclusion that, since I>B, BE are placed in a straight line, ES, BG
are also in a straight line is referred to t. 14. The deduction is made clearer
by the following steps.

The angle DBF'm equal to the angle GBE; ``

add to each the angle FB£ ;

therefore the angles DBF, FBE are together equal to the angles GBE, FBE.

(C. N. i]

But the angles DBF, FBE are together equal to two right angles, [i. 13]

therefore the angles GBE, FBE are together equal to two right angles,

[C.N.t]

and hence FB, BG are in one straight line. [i. 14]

The result is also obvious from the converse of 1. 15 given by Proclus
(see note on i. 15).

The proposition vi. 14 contains a theorem and one partial converse of it;
so also does vi. 1 5. To each proposition may be added the other partial
converse, which may be enunciated as follows, the words in square brackets
applying to the case of triangles (vi. 15).

Equal paralhlogrami triattgUs\ which havt the sides absut one angle in
each redprocally proportional art equiansular have the angles included by those
sides either equal or supplementary

Let ABf BC be equal parallelograms, or let FBD, EBG be equal

 n. 14, isl PROPOSITIONS 14, 15 119

triangles, such that the sides about the angles at B are reciprocally propor-
tional, i.e. such that

DB : BE = GB : BF.

We shall prove that the angles FED, EBG are either equal or supple-
mentary.

Place the figures so that DB. BE are in one straight line.

Then FB, BG are either in a straight line, or not in a straight line.

(i) If FB, BG are in a straight line, the figure of the proposition
(with the diagonals FD, EG drawn) represents the facts, and

the angle FBD is equal to the angle EBG. ['. 15]

(2) If J'B, BG are not in a straight line,
produce FB to H so that BH may be equal to BG.

Join EJf, and complete the parallelogram EBHK.

Now, since DB : BE  GB \ BF

xbA. GB = HB, ~

DB : BE = HB . BF,
and therefore, by vi. r4 or 15,
the parallelograms j4B, BK 3.x equal, or the triangles FBD, EB/fare equal.

But the parallelograms AB, BCart eciual, and the triangles FBD, EBG
are equal ;

therefore the parallelograms BC, BK are equal, and the triangles EBH,
EBG are equal.

Therefore these parallelograms or triangles are within the same parallels :
that is, G, C, H, K are in a straight line which is parallel to DE. [1, 39]

Now, since BG, BHare. equal,
the angles BGH, BHG are equal.

By parallels, it follows that

the angle EBG is equa' to the angle DBH,
whence the angle EBG is supplementary to the angle FBD.

\end{notes}

\end{proposition}

\begin{proposition}
\label{prop:VI_15}

\begin{statement}
In equal triangles which have one angle equal to one angle
the sides about the equal angles are reciprocally proportional ;
and those triangles which have otie angle equal to one angle,
and in which the sides about the equal angles are reciprocally
proportional, are equal
\end{statement}

\begin{proof}

Let ABC, ADE be equal triangles having one angle
equal to one angle, namely the angle BAC to the angle
DAE,

I say that in the triangles ABC, ADE the sides about the

equal angles are reciprocally proportional, that is to say, that,

as CA is to AD, so is EA to AB,

aao BOOK VI [vi. 15

For let them be placed so that CA is in a straight
line with AD;

therefore EA is also in a straight line with
AB. [i. 14]

Let BD be joined.

Since then the triangle ABC is equal to
the triangle ADE, and BAD is another
area,

therefore, as the triangle CAB is to the
triangle BAD, so is the triangle EAD to
the triangle BAD. [v. 7]

But, as CAB is to BAD, so is CA to AD, [vi. i]

and, as EAD is to BAD, so is EA to AB. id:

Therefore also, as CA is to AD, so is EA to AB. (v. n]
Therefore in the triangles ABC, ADE the sides about
the equal angles are reciprocally proportional.

Next, let the sides of the triangles ABC, ADE be reci-
procally proportional, that is to say, let EA be to AB as CA
to AD ;

I say that the triangle ABC is equal to the triangle ADE.

For, if BD be again joined, 1 ,

since, as CA is to AD, so is EA to AB,

while, as CA is to AD, so is the triangle ABC to the triangle
BAD,

and, as EA is to AB, so is the triangle EAD to the triangle
BAD, [vt. ,]

therefore, as the triangle ABC is to the triangle BAD, so is
the triangle EAD to the triangle BAD. [v. 11]

Therefore each of the triangles ABC, EAD has the same
ratio to BAD.

Therefore the triangle ABC is equal to the triangle EAD.

[v. 9]
Therefore etc.   -> ;  r
\end{proof}

\begin{notes}

As indicated in the partial converse given in the last note, this proposition
is equally true if the angle included by the two sides in one triangle h
supplementary, instead of being equal, to the angle included by the two sides
in the other.

Let ABC, ADE be two tria.ngles such that the angles BA C, DAE are
supplementary, and also .

CA:AD = EA: AB.

In this case we can place the triangles so that
CA is in a straight line with AD, and AB lies
along AE (since the angle EAC, being supple-
mentary to the angle EAD, is equal to the anglo
BAC).

If we join BD, the proof given by Euclid
applies to this case also.

It is true that vi, 15 can be immediately inferred from vi. 14, since a
triangle is half of a parallelogram vrith the same base and height. But,
Euclid's object being to give the student a grasp of mttkods rather than
results, there seems to be no advantage in deducing one proposition from the
other instead of using the same method on each.

\end{notes}

\end{proposition}

\begin{proposition}
\label{prop:VI_16}

\begin{statement}
If four straight lines be proportional, the rectangle con-
tained by the extremes is equal to the rectangle contained by
the means ; and, if the rectangle contained by the extremes be
equal to the rectangle contained by the means, the four straight
lines will be proportional.
\end{statement}

\begin{proof}

Let the four straight lines AB, CD, E, F be propoitional,
so that, as AB is to CD, so is B to F;

I say that the rectangle contained by AB, F is equal to the
rectangle contained by CD, E.

Let AG, CH be drawn from the points A, C t right
angles to the straight lines AB, CD, and let AG he made
equal to F, and Cff equal to E.

Let the parallelograms BG, DH be completed.

Then since, as AB Is to CD, so is .£  to F,
while E is equal to CH, and f to AG,
therefore, as AB is to CD, so is CH to AG.

Therefore in the parallelograms BG, DH the sides about
the equal angles are reciprocally proportional.

But those equiangular parallelograms in which the sides
about the equal angles are reciprocally proportional are equal ;

[VI. 14]
therefore the parallelogram BG is equal to the parallelogram

And BG is the rectangle AB, F, for AG is equal to F;
and DH is the rectangle CD, E, for  is equal to CH
therefore the rectangle contained by AB, F is equal to the
rectangle contained by CD, E.

Next, let the rectangle contained by AB, F be equal to
the rectangle contained by CD, E ;

I say that the four straight lines will be proportional, so that,
as AB is to CD, so is E to F.

For, with the same construction,
since the rectangle AB, / is equal to the rectangle CD, E,
and the rectangle AB, F is BG, for AG is equal to F,
and the rectangle CD, £ is D/f, for C// is equal to £,
therefore BG is equal to D/f.

And they are equiangular , •, ••   .

But in equal and equiangular parallelograms the sides about
the equal angles are reciprocally proportional. [vi. 14]

Therefore, as AB is to CD, so is Cff to A G.

But CH is equal to E, and AG to F;
therefore, as AB is to CD, so is £'' to F.

Therefore etc.
\end{proof}

\begin{notes}

This proposition is a, particular case of vi. 14, but one which is on all
accounts worth separate statement. It may also be enuticiated in the follow-
ing form :

Jitetartgki which have thtir bases rtdprocaUy proporiumal to ihcir Atighis-
art equcd in arta; and egml rectangles have their bases redproealfy proportional
to thdr heights.

Since any fkarallelogTam is equal to a rectangle of the same height and
Dn the same base, and any triangle with the same height and on the same
base is equal to half the parallelcram or rectangle, it follows that Equal
parallelograms or triangles have their bases reciprocally proportional to their
heights and vice vena.

The present place is suitable for giving certain important propositions,
including those which Simson adds to Book vi. as Props, B, C and D, which
are proved directly by means of vi, 16.

I. Proposition B is a particular case of the following theorem.
J/ a circle be circumscribed about a triangle ABC and there be drawn through
A any two straight lines either both within or both without the an BAG, pm.

AD mteting BC (produad if Mcttsary) in D and AE mealing the circle again
in E, suth that the angles DAB, EAC are equal, then the rectangle AD, AE is
equal to the rectangle BA, AC.

Join CE.

The angles BAD, £AC 3,k equal, by hypothesis ;

and the angles ABZ>, AEC are equal, [in. n, as]

Therefore the triangles ABD, AEC are equiangular.

Hence BA is to AD as EA is to AC,

and therefore the rectangle BA, AC is equal to the rectangle AD, AE,

[VI. 1 6]
There are now two particular cases to be considered.

(a) Suppose that AD, AE coincide ;

ADE will then bisect the angle BAC. •

(b) Suppose that AD, v4£ are in one straight line but that D, E ate on
opposite sides of A ;

AD will then bisect the external angle at A.

e

In the fiist case (a) we have a; >•' - <

the rectangle BA, iC equal to the rectangle EA, AD;

and the rectangle EA, AD is equal to the rectangle ED, DA together with
the square on AD, [i. 3]

i.e. to the rectangle BD, DC together with the square on AD. [11 1. 35]

Therefore the rectangle BA, AC b, equal to the rectangle BD, DC
tfetber with the square on AD. [This is Simson's Prop B]

In case () the rectangle EA, AD is equal to the excess of the rectangle
ED, DA over the square on AD ;

therefore the rectangle BA, AC k equal to the excess of the rectane BD,
DC over the square on AD.

The following converse of Simson's Prop. B may be given :  a straht
Uiu AD  drawn from the virtex K of a Iriatlt to meH the base, S9 that the
square on AD together with the reciane BD, DC is tqual to the rectangU BA,
AC, the line AD will biieet the angle BAG exapt when the sides AB, AC are
tqual, in which atse every line drawn to the base will have the property men-
timed,

Let the circumscribed circle be drawn, and let AD produced meet it in
E; join CE.

The rectangle S£>, DC is equal to the rectangle ED, DA. [iii. 35]

Add to each the square on AD ;
therefore the rectangle BA, ACs equal to the rectangle EA, AD.

[hyp. and Ii. 3]

Hence AS is to AD as AE to AC. [vi, 16]

But the angle ABD is equal to the angle AEC. [111. 11]

Therefore the angles BDA, EC A are either equal or supplementary.

[vi. 7 and note]

(o) If they are equal, the angles BAD, SAC
are also equal, and AD bisects the angle BAC.

(b) If they are supplementary, the angle ADC
must be equal to the angle ACE.

Therefore the angles BAD, ABD are together
equal to the angles ACB, BCE, i.e. to the angles
A CD, BAD.

Take away the common angle BAD, and
the angles ABD, ACD are equal, or
jJJ is equal to AC.

Euclid himself assumes, in Prop. 67 of the Data, the result of so much of
this proposition as relates to the case where BA = AC. He assumes namely,
without proof, that, if BA -AC, and if ZJ be any point on BC, the rectangle
BD, Z>C together with the square on AD\ equal to the square on AB.

Proposition C.

Jffrom any angle of a triangle a straight line be drawn perpendicular to the
opposite side, the rectangle contained by the other two sides of the triangle is equal
to the rectangle contained by the perpendicular and the diameter of the circle
(iraimscriied about the triangle.

Let .i4C be a triangle and AD the perpendicular on AB. Draw the
diameter A£ of the circle circumscribed about the triangle ABC.

Then shall the rectangle BA, ACh equal to the rectangle EA, AD.
Join EC.

Since the right angle BDA is equal to the right angle ECA in a semi-
circle, [ill. 31]
and the angles ABD, AEC in the same spnent are equal, [in. 21]
the triangles ABD, AEC are equiangular.

Therefore, as 4 is to AD, so is EA to AC, [vj. 4]

whence the rectangle BA, AC is equal to the rectangle EA, AD. [vi, 16]

This result corresponds to the trigonometrical formula, for JP, the radius of
the circumscribed circle,

E =

4

Proposition D.

This is the highly important lemma given by Ptolemy (ed, Heiberg, Vol. i,
pp. 36 — 7) which is the basis of his calculation of the table of chords in the
section of Book i. of (he /xcyoAi; trvrraii  entitled `` concerning the siite of the
straight lines [i.e. chords] in the circle `` (rtpl i-ijt TnjKiKonfTot too' v riji KixXif

The theorem may be enunciated thus.

7 e ruiangk wfitaimd by iht diagsnah of any quadrilateral inscribtdin a
drclt is equal to the sum afthe nclangks contained by the pairs of opposite sides.

I shall give the proof in Ptolemy's words, with the addition only, in
brackets, of two words applying to a second figure not giver by Ptolemy.

`` Let there be a circle with any quadrilateral ABCD inscribed in it, and
exAC, BD be joined

It is to be proved that the rectangle contained by ACznA BD is equal
to the sum of the rectangles AB, DC and AD, BC.

For let the angle ABE be made equal to the angle contained by DB, BC.

-i' (jMl

If then we add [or subtract] the angle EBD, ~ <

the angle ABD will also be equal to the angle ESC.

But the angle BDA is also equal to the angle BCE, [m. ai]

for they subtend the same segment ;
therefore the triangle ABD is equiangular with the triangle EBC.

Hence, proportionally,

as BC is to CE, so is BD to DA. [vi. 4]

Therefore the rectangle BC,ADk equal to the rectangle BD, CE.

[VI. 16]
Again, since the angle ABE is equal to the angle DBC,
and the angle BAE is also equal to the angle BDC, [m. ai)

the triangle ABE is equiangular with the triangle DBC.

Therefore, proportionallj',

as £A is to A£, so is SJD to DC;
therefore the rectangle BA, DC is equal to the rectangle BD, AE.

But it was also proved that

the rectangle BC, AD is equal to the rectangle BD CE
therefore the rectangle AC, BD as a whole is equal to the sum of the
rectangles AB, i»Cand AD, BC:

(being) what it was required to prove.''

Another proof of this proposition, and of its converse, is indicated yj
Dr Lachlan (Elements of Euclid, pp. 273 — 4). It depends on two preliminary
propositions.

(1) If two tirdes bi divided, by a chord in each, into segments which are
similar respeettvcfy, the chords are proportional to the corresponding diameters.

The proof is instantaneous if we join the ends of each chord to the centre
of the circle which it divides, when we obtain two similar triangles.

(1) IfYibe any point on the circle circumscribed about a triangle ABC, and
DX, DY, DZ be perpendicular to the sides BC, CA, AB of the triangle
respectively, then X, Y, Z lie in one straight line ; and, conversely, if the feet of
the perpendiculars from any point D on the sides of a triangle lie in one straight
line, D ties on the circle circumscribed about the triangle.

The proof depending on ill. 21, 22 is well known.

Now suppose that D is any point in the plane of a triangle ABC, and
that DX, D y, DZ are perpendicular to the sides
BC, CA, AB respectively.

Join YZ, DA.

Then, since the angles at ] Z are right,
A, Y, D, Z lie on a circle of which DA is the
diameter.

And YZ divides this circle into segments which
are similar respectively to the segments into which
BC divides the circle circumscribing ABC, since
the angles ZAY, BAC coincide, and their supple-
ments are equal.

Therefore, if i be the diameter of the circle
circumscribing ABC,

BChlodas YZistoDA;
and therefore the rectangle AD, BC is equal to the rectangle d, YZ.

Similariy the rettangle BD, CA is equal to the rectangle d, ZX, and the
rectangle CD, AB\ equal to the rectangle d, XY.

Hence, in a quadrilateral in general, the rectangle
contained by the diagonals is less than the sum of the
rectangles contained by the pairs of opposite sides.

Next, suppose that D lies on the circle circum-
scribed about ABC, but so that A, B, C, D follow
each other on the circle in this order, as in the figure
annexed.

Let DX, DY, DZ be perpendicular to BC, CA,
AB respectively, so that X, Y, Zare in a straight line.

Then, since the rectangles AD, BC; BD, CA; CD, AB are equal to the
rectangles d, YZ; d, ZX; d, .VF respectively, and XZi  equal to the sum of

Xy, YZ, so that the rectangle 4 XZ is equal to the sum of the rectangles
4 JfKand d, YZ, It follows that

the rectangle AC, BD is equal to the sum of the rectangles AD, BCand
AB, CD.

Cenvirsefy, if the latter statement is true, while we are supposed to know
nothing about the position of D, it follows that

XZ must be equal to the sum of XY, YZ,
so that X, Y, Z must be in a straight line.

Hence, from the theorem (2) above, it follows that D must lie on the
circle circumscribed about ABC, i.e. that A BCD is a quadrilateral about
which a circle can be described.

All the above propositions can be proved on the basis of Book iii. and
without using Book vi., since it is possible by the aid of 111.21 and 35 alone
to prove that in equiaitgular triangles the rectangles contained by the iwh-
correspsnding sides about equal angles are equal to one another (a result arrived
at by combining vi, 4 and v[. 16). This is the method adopted by Casey,
H. M. Taylor, and Lachlan ; but I fail to see any particular advantage in it

Lastly, the following proposition may be given which Playfair added as
VI. E. It appears in the Data of Euclid, Prop. 93, and may be thus
enunciated.

If the angle BAG of a triangle ABC be bisected by the straight line AD
meeting (he tirde circumscribed about the triangle in D, and if BD be joined,
then

the sum of BA, AC /> to AD as BC is to BD,

Join CD. Then, since AD bisects the angle BAC, the subtended arcs
SD, DC, and therefore the chords BD, DC, are
equal.

(i) The result can now be easily deduced from
Ptolemy's theorem.

For the rectangle AD, BC is equal to the sum of
the rectangles AB, DC and .4C, BD, i.e. (since
BD, CD are equal) to the ret tangle contained by
BA + ACmdBD.

Therefore the sum of BA, AC is to AD as BC
is to BD. [vi. 16]

(2) Euclid proves it differently in Data, Prop. 93,

Let AD meet BC in JB.

Then, since A£ bisects the angle BAC,

BA i to AC as BE to £C, [vi. 3]

or, alternately,

AB is to BE as AC to C£. [v. r6]

Therefore also

BA + AC is to BC BS AC to CE. [v, la]

Again, since the angles BAD, EA C are equal, and the angles ADB, ACE
are also equal, [in. 2i]

the triangles ABD, A EC are equiangular.

Therefore C is to CE as AD to BD. [vi, 4]

ail BOOK VI [vi. 16, 17

Hence BAACisXaBCiBADloBD, [v. 11]

and, alternately,

BA + AC is to AD as BC is to BD. (v. 16]

Euclid concludes that, if the circle ABC is pven in magnitude, and the
chord BC cuts off a segment of it containing a given angle (so that, by Data
Prop. 87, 2fCand also BD are given in magnitude), , ,, ,

the ratio of BA + AC to AD is given,

and further that (since, by similar triangles, BD is to D£ as 4 C is to C£,
vhit BA + AC is to BCsiACis to C£),

the rectangle (BA + AC), D£, being equal to the rectangle BC, BD, is
also given.

\end{notes}

\end{proposition}

\begin{proposition}
\label{prop:VI_17}

\begin{statement}
If three straight lines be proportional, the rectangle con-
tained by the extremes is equal to the square on the mean;
and, if the rectangle contained by the extremes be equal to the
square on the mean, the three straight lines will be proportional.
\end{statement}

\begin{proof}

Let the three straight lines A, B, C be proportional, so
that, as A is to B, so is £ to C ;

I say that the rectangle contained by A, C is equal to the
square on B.

Let D be made equal to B.

Then, since, as A is to B, so is  to C,

and B is equal to D,

therefore, as A is to B, so is Z* to C,

But, if four straight lines be proportional, the rectangle
contained by the extremes is equal to the rectangle contained
by the means. [vi, i6'

Therefore the rectangle A, C h equal to the rectangle
B,D.

But the rectangle S, D is the square on B, for B is
equal to D ;

therefore the rectangle contained by A, C is equal to the
square on B.

Next, let the rectangle A, Che equal to me square on B
I say that, as  is to B, so is B to C.

VI. 17, i8] PROPOSITIONS 16—18 aaff

For, with the same construction,
since the rectangle A, C is equal to the square on B,
while the square on B is the rectangle B, D, for B is equal
to A
therefore the rectangle A, C 's equal to the rectangle B, D.

But, if the rectangle contained by the extremes be equal
to that contained by the means, the four straight lines are
proportional. [vi. 16]

Therefore, as A is to B, so is D to C

But 5 is equal to /?; '•''

therefore, as A is to B, so is B to C.

Therefore etc.
\end{proof}

\begin{notes}

VI. 17 is, of course, a particular case of vi. 16.

\end{notes}

\end{proposition}

\begin{proposition}
\label{prop:VI_18}

\begin{statement}
On a given straight line to describe a rectilineal figure
similar and similarly situated to a given rectilineal figure.
\end{statement}

\begin{proof}

Let AB be the given straight line and CE the given
rectilineal figure ;

thus it is required to describe on the straight line AB a
rectilineal figure similar and similarly situated to the recti-
lineal figure CE.

Let DF be joined, and on the straight line AB, and at
the points A, B on it, let the angle GAB be constructed
equal to the angle at C, and the angle ABG equal to the
angle CDF. [1. 13]

Therefore the remaining angle CFD is equal to the angle
AGB ; [I. 32]

therefore the triangle FCD is equiangular with the triangle
GAB.

Therefore, proportionally, as FD is to GB, so is FC to
GA, and CD to AB.

Again, on the straight line BG, and at the points B, G on
it, let the angle BGH be constructed equal to the angle DFE,
and the angle GBH equal to the angle FDE. [i. »3]

Therefore the remaining angle at E is equal to the re-
maining angle at H ; [i. 3]

therefore the triangle FDE is equiangular with the triangle
GBH

therefore, proportionally, as FD is to GB, so is FE to
GH, and ED to HB. [vi. 4]

But it was also proved that, as FD is to GB so is FC to
GA, and CD to AB ;

therefore also, as FC is to AG, so is CD to AB, and /£'
to GIf, and further £''/? to HB.

And, since the angle CFD is equal to the angle AGB,

and the angle DFE to the angle GIf,

therefore the whole angle CFE is equal to the whole angle
AGH.

For the same reason

the angle CDE is also equal to the angle ABH.
And the angle at C is also equal to the angle at A,

and the angle at E to the angle at H.
Therefore AH is equiangular with CE ;
and they have the sides about their equal angles proportional ;

therefore the rectilineal figure AH is similar to the

rectilineal figure CE. [vi. Def. i]

Therefore on the given straight line AB the rectilineal
figure AH has been described similar and similarly situated
to the given rectilineal figure CE.

Q.E.F.
\end{proof}

\begin{notes}

Simson thinks the proof of this proposition has been vitiated, his grounds
for this view being (1) that it is demonstrated only with reference to
quadrilaterals, and does not show how tt may be extended to figures of five or
more sides, (i) that Euclid infers, from the fact of two triangles being
equiangular, that a side of the one is to the corresponding side of the other as
another side of the first is to the side corresponding to it in the other, i.e. he
permutes, without mentioning the fact that he does so, the proportions
obtained in vi. 4, whereas the proof of the very next proposition gives, in a
similar case, the intermediate step of permutation. I think this is hyper-
criticism. As regards (a) it should be noted that the permuted form of the
proportion is arrived at first in the proof of vi, 4 ; and the omission of the
intermediate step of allertiandOy whether accidental or not, is of no importance.
On the other hand, the use of this form of the proportion certainly simplifies
the proof of the proposition, since it makes unnecessary the subsequent
IX aequali steps of Simson's proof, their place being taken by the inference
[v. 1 1] that ratios which are the same with a third ratio are the same with one
another.

Nor is the first objection of any importance. We have only to take as the

given polygon a polygon of five sides at least, as CDEFG, pin one extremity
of CD, say D, to each of the angular points other than C and E, and then
use the same mode of construction as Euclid's for any number of successive
triangles as ABL, LBK, etc, that may have to be made. Euclid's con-
struction and proof for a quadrilateral are quite sufficient to show how to deal
with the case of a figure of five or any greater number of sides.

Clavius has a construction which, given the power of moving a figure

bodily from one position to any other, is easier. CDEFG being the given
polygon, join CE, CF. Place AB on CD so that A falls on C, and let B
fall on ly, which may either lie on CD or on CD produced.

Now draw DE parallel to DE, meeting CE, produced if necessary, in E,
EF' parallel to EF, meeting CF, produced if necessary, in E'', and so on.

Let the parallel to the last side but one, FG meet CG, produced if
necessary, in C.

Then CUE FG is similar and similarly situated to CDEFG, and it is
constructed on CD, a straight Hne equal to AB.

The proof of this is obvious.

A more general construction is indicated in the subjoined figure. If
CDEFG be the given polygon, suppose its angular points all joined to any
point O and the connecting straight lines produced both ways. Then, if CD,
a straight line equal io AB, be placed so that it is parallel to CD, and C, D
lie respectively on OC, OD (this can of course be done by finding fourth
proportionals), we have only to draw UE, EF, etc., parallel to the
corresponding sides of the original polygon in the manner shown.

De Morgan would rearrange Props. 18 and 20 in the following manner.
He would combine Prop. 18 and the first part of Prop, ao into one, with the
enunciation:

Pairs of similar triangits, similarly put tcgeihtr, give similar figures ; and
every pair of similar figures is composed of pairs of similar triangles similarly
put together.

He would then make ihs problem of vi. 18 an application of the first part
In form this would certainly appear to be an improvement; but, provided that
the relation of the propositiotis is understood, the matter of form is perhaps
not of great importance.

\end{notes}

\end{proposition}

\begin{proposition}
\label{prop:VI_19}

\begin{statement}
Similar triangles are to one another in the duplicate ratio
of ike corresponding sides.
\end{statement}

\begin{proof}

Let ABC, DEFhe. similar triangles having the angle at
B equal to the angle at £, and such that, as AB is to BC, so
s is DE to EF, so that BC corresponds to EF; [v, Def. n]

I say that the triangle ABC has to the triangle DEF a ratio
duplicate of that which BC has to EF.

For let a third proportional BG be taken to BC, EF, so
that, as BC is to EF, so is EF to BG ; [vi. 1 1]

10 and let AG be joined.

Since then, as AB is to BC, so is DE to EF,
therefore, alternately, as AB is to DE, so is BC to EF. [v. 16]

But, as BC is to EF, so is EF to BG ;  ``
therefore also, as AB is to DE, so is EF to BG. [v. n]

IS Therefore in the triangles ABG, DBF the sides about
the equal angles are reciprocally proportional.

But those triangles which have one angle equal to one
angle, and in which the sides about the equal angles are
reciprocally proportional, are equal; [vi. 15]

30 therefore the triangle ABG is equal to the triangle DEF.
Now since, as BC is to EF, so is EF to BG,

and, if three straight lines be proportional, the first has to
the third a ratio duplicate of that which it has to the second,

[v, Def. 9]
therefore BC has to BG a ratio duplicate of that which CB
as has to EF.

But, as CB is to BG, so is the triangle ABC to the
triangle ABG ; [vi. i)

therefore the triangle ABC also has to the triangle ABG a
ratio duplicate of that which BC has to EF.

30 But the triangle ABG is equal to the triangle DEF;

therefore the triangle ABC also has to the triangle DEF a
ratio duplicate of that which BC has to EF. ...... ,. ,. ,

Therefore etc.

\begin{porism*}
From this it is manifest that, if three straight
35 lines be proportional, then, as the first is to the third, so is
the figure described on the first to that which is similar and
similarly described on the second,
\end{porism*}
\end{proof}

\begin{annotations}

4. uid such that, as AB is Q BC, so Is DE to BP, Utcrally ``(tiiangls) having
the anjfle at B equal to the anj;1e at f , and (AavtHg). as AB to BC so DK to EF

\end{annotations}

\begin{notes}

Having combined Prop. [8 and the first part of Prop. 20 as just indicated,
De Morgan would tack on to Prop. 19 the second part of Prop. 20, which
asserts that, if similar polygons be divided into the same number of similar
triangles, the triangles are `` hemokgoui to the wholes `` (in the sense that the
polygons have the same ratio as the corresponding triangles have), and that
the polygons are to one another in the duplicate ratio of corresponding sides.
This again, though no doubt an improvement of form, would necessitate the
drawing over again of the figure of the altered Proposition 18 and a certain
amount of repetition.

Agreeably to his suestion that Prop, 23 should come before Prop. 14
which is a particular case of it, De Morgan would prove Prop. 19 for
parctUtlsgrams by means of Prop, 13, and thence infer the truth of it for
triangles or the halves of the parallelograms. He adds ; `` The method of
Euclid is an elegant application of the operation requisite to compound equal
ratios, by Which the conception of the process is lost sight of.'' For the
general reason given in the note on vi, 14 above, I think that Euclid showed
the sounder discretion in the arrangement which he adopted. Moreover it is
not easy to see how performing the actual operation of compoutiding two
equal ratios can obscure the process, or the fact that two equal ratios are
being comfXJunded. On the definition of compounded ratios and duplicate
ratio, De Morgan has himself acutely pointed out that ``composition'' is here
used for the process of detecting the single alteration which produces the
effect of two or more, the duplicate ratio being the result of compounding two
equal ratios. The proof of vi. 19 does in fact exhibit the single alteration
which produces the effect of two. And the operation was of the essence of
the Cireek geometry, because it was the manipulation of ratios in this manner,
by simplification and transformation, that gave it so much power, as every one
knows who has read, say, Archimedes or ApoUonius. Hence the introduction
of the necessary operation, as well as the theoretical proof, in this proposition
seems to me to have been distinctly worth while, and, as it is somewhat
simpler in this case than in the more general case of vj. 23, it was in
accordance with the plan of enabling the difficulties of Book vi. to be more
easily and gradually surmounted to give the simpler case first.

That Euclid wished to emphasise the importance of the method adopted,
as well as of the result obtained, in vi. 19 seems to me clearly indicated by
the Porism which follows the proposition. It is as if he should say : ``I have
shown you that similar triangles are to one another in the duplicate ratio of
corresponding sides; but I have also shown you incidentally how it is possible
to work conveniently with duplicate ratios, viz. by transforming them into
simple ratios between straight lines. I shall have occasion to illustrate the
use of this method in the proof of vi. 22.''

Tlie Porism to VI. 19 presents one difficulty. It will be observed that it
speaks of  figure ((Bos) described on the first straight line and of that which
is similar and similarly described on the second. If `` figure `` could be
regarded as loosely used for the figure of the proposition, i.e. for a triangle,
there would be no difficuity. If on the other hand `` the figure `` means any
rectilineal figure, i.e. any p>olygon, the Porism is not really established until
the next proposition, vi, 20, has been proved, and therefore it is out of place
here. Yet the correction Tpiynivoi', triangle, for Ahot, figure, is due to Theon
alone ; P and Campanus have `` figure,'' and the reading of Philoponus and
Psellus, TtTpayiiivov, square, partly supports (TS<n, since it can be reconciled with
t'Sot but not with Tpiyuvoi'. Again the second Porism to vj. jo, in which this
Porism is reasserted for any rectilineal figure and which is omitted by
Campanus and only given by P in the margin, was probably interpolated by
Theon. Heiberg concludes that Euclid wrote ``figure'' (<ISot), and Theon,
seeing the difficulty, changed the word into `` triangle `` here and added For. 2
to VI. «o in order to make the matter clear. If one may hazard a guess as to
how Euclid made the slip, may it be that he first put it after vi 20 and then,
observing that the expression of the duplicate ratio by a single ratio between
two straight lines does not come in vi. so but in vi. rg, moved the Porism to
the end of vi. 19 in order to make the connexion clearer, without noticing
that, if this were done, tTSo! would need correction ?

The following explanation at the end of the Porism is bracketed by
Heiberg, viz. ``Since it was proved that, as CB is to BG, so is the triangle
ABCio the triangle ABG, that is DEF.'' Such explanations in Porisms are
not in Euclid's manner, and the words are tiot in Campanus, though they date
from a time earlier than Theon.

\end{notes}

\end{proposition}

\begin{proposition}
\label{prop:VI_20}

\begin{statement}
Similar polygons are divided into similar triangles, and
into triangles equal in multitude and in the same ratio as
the wholes, and the polygon has to the polygon a ratio duplicate
of that which the corresponding side has to the corresponding
side.
\end{statement}

\begin{proof}

Let ABCDE, FGHKL be similar polygons, and let AB
correspond to FG ;

I say that the polygons ABCDE, FGHKL are divided into
similar triangles, and into triangles equal in multitude and in
10 the same ratio as the wholes, and the polygon ABCDE has
to the polygon FGHKL a ratio duplicate of that which AB
has to FG.

Let BE, EC, GL, Z be joined.

f.h

Now, since the polygon ABCDE is similar to the polygon
,5 FGHKL,

the angle BAE is equal to the angle GFL ;

and, as BA is to AE, so is GF to FL, [vi. Def. i]

Since then ABE, FGL are two triangles having one
angle equal to one angle and the sides about the equal angles
K proportional,

therefore the triangle ABE is equiangular with the triangle

FGL ; [v.. 6]

so that it is also similar ; [vi. 4 and Def. i]

therefore the angle ABE is equal to the angle FGL.

136   •- BOOK VI :. [vt. 20

»S But the whole angle ABC is also equal to the whole angle
FGH because of the similarity of the polygons ;

therefore the remaining angle EBC is equal to the angle
L,GH.

, And, since, because of the similarity of the triangles ABE,
3PFGL,

as EB is to BA, so is LG to GF,
and moreover also, because of the similarity of the polygons,

as AB is to BC, so is FG to GH,
therefore, ex aequali, as EB is to BC, so is Z G to GH ; [v. aa]

35 that is, the sides about the equal angles EBC, LGH are
proportional ;

therefore the triangle EBC is equiangular with the triangle
LGH, [vi. 6]

so that the triangle EBC is also similar to the triangle
¥>LGH. [vi. 4 and Def. i]

For the same reason
the triangle ECD is also similar to the triangle LHK.

Therefore the similar polygons ABCDE, FGHKL have
been divided into similar triangles, and into triangles equal in
45 multitude.

I say that they are also in the same ratio as the wholes,
that is, in such manner that the triangles are proportional,
and ABE, EBC, ECD are antecedents, while FGL, LGH,
LHK are their consequents, and that the polygon ABCDE
so has to the polygon FGHKL a ratio duplicate of that which
the corresponding side has to the corresponding side, that is
AB to FG.

For let AC, FH be joined.

Then since, because of the similarity of the polygons,

iS the angle ABC is equal to the angle FGH,
and, as AB is to BC, so is FG to GH,

the triangle ABC is equiangular with the triangle FGH ;

[VI. 6]

therefore the angle BAC is equal to the angle GFH,
and the angle BCA to the angle GHF.
60 And, since the angle BAM is equal to the angle GFN,
and the angle ABM is also equal to the angle FGN,

VI. jo] proposition lo 83J

therefore the remaining angle A MB is also equal to the
remaining angle FNG ; [i. 3a]

therefore the triangle ABM is equiangular with the triangle
6s FGN.

Similarly we can prove that
the triangle BMC is also equiangular with the triangle GNH,
Therefore, proportionally, as AM is to MB, so is FN to
NG,
70 and, as BM is to MC, so is GN to NH ;

so that, in addition, ex aequali,

as AM is to MC, so is FN to NH.

But, as AM is to MC, so is the triangle ABM to MBC,

and AME to EMC; for they are to one another as their

7S bases. [vi. i]

Therefore also, as one of the antecedents is to one of the

consequents, so are all the antecedents to all the consequents ;

[v. is]

therefore, as the triangle A MB is to BMC, so is ABE to

CBE.
80 But, as AMB is to BMC, so is AM to MC ;

therefore also, as AM is to MC, so is the triangle ABE to

the triangle EBC

For the same reason also,

as FN is to NH, so is the triangle FGL to the triangle
8s GLH.

And, as AM is to MC, so is FN to NH;

therefore also, as the triangle ABE is to the triangle BEC,

so is the triangle FGL to the triangle GLH ;

and, alternately, as the triangle ABE is to the triangle FGL,
90 so is the triangle BEC to the triangle GLH.

Similarly we can prove, if BD, GK be joined, that, as the

triangle BEC is to the triangle LGH, so also is the triangle

ECD to the triangle LHK.

And since, as the triangle ABE is to the triangle FGL,
9S so is EBC to LGH, and further ECD to LHK,

therefore also, as one of the antecedents is to one of the

consequents, so are all the antecedents to all the consequents ;

[v. 13

therefore, as the triangle ABE is to the triangle FGL,
so is the polygon ABCDE to the polygon FGHKL.

too But the triangle ABE has to the triangle FGL a ratio

duplicate of that which the corresponding side AB has to the

corresponding side FG\ for similar triangles are In the

duplicate ratio of the corresponding sides. [vi. 19]

Therefore the polygon ABCDE also has to the polygon

 OS FGHKL a ratio dupUcate of that which the corresponding

side AB has to the corresponding side FG.

Therefore etc.

\begin{porism*}
Similarly also it can be proved in the case of
quadrilaterals that they are in the duplicate ratio of the
110 corresponding sides. And it was also proved in the case of
triangles; therefore also, generally, similar rectilineal figures
are to one another in the duplicate ratio of the corresponding
sides.
\end{porism*}
\end{proof}

\begin{annotations}

1. in the same ratio as the wholes. The Mune word J/iiXnTmi is used which 1 have
generally translated by * corresponding.' But here it is followed by a dative. 5;4£X(a r«r
fiXeif *' AcmoioMi viih the wholes ,'' instead of being used abiotutely. The meaning can
therefore here be nothing else but 'Mn the same ratio with'' or ``proportional to the
wholes'' and Euclid seems to recognise that he is making a special use of the word,
because he eitplains it lower down (1. 46) : ``the triangles are homolcous to the wholes, thai
is, in such manner that the triangles are proportional, and AB£j EBC, ECD are ante-
cedents, while FGL, LGH, I. UK are their consequents,''

49. iv6fu¥a rr0Twf, `` iAtr consequents,'' is a little awkward, but may be supposed to
indicate which triangles curtespond to which ai consctjuent to antecedent. . ,., , .

\end{annotations}

\begin{notes}

An alternative proof of the second part of this proposition given after the
Porisms is relegatetj by August and Heiberg to an Appendix as an interpolation.
It is shorter than the proof in the text, anti is the only one given by many
editors, including Clavius, Billingsley, Barrow and Simson. It runs as follows:

`` We will now also prove that the triangles are homolcous in another and
an easier manner.

Again, let the polygons ABCDE, FGHKL be set out, and let BE, EC,
GL, ZIf be joined

I say that, as the triangle A BE is to FGL, so is EBC to LGH and ODE
to HKL.

For, since the triangle ABE is similar to the triangle FGL, the triangle
ABE has to. the triangle FGL a ratio duplicate of that which BE has to GL,

For the same reason also
the trkngte BEC has to the triangle GLIf a ratio duplicate of that iich
£E has to GL.

Therefore, as the triangle ABE is to the triangle EG£, so is BEC
10 GZff.

Again, since the triangle EBC s similar to the triangle LGH,

EBC has to LGH a ratio duplicate of that which the straight line CE has
to HL.

For the same reason aiso

the triangle ECD has to the triangle LHK a ratio duplicate of that which
CE has to HL.

Therefore, as the triangle EBC is to LGH, so is ECD to LHK.
But it was proved that,

as EBC is to LGH, so also is ABE to FGL.

Therefore also, as ABE is to FGL, so is BEC to GLH and ECD to
LHK. • .

••   It

Q.E.D.

Now Euclid cannot fail to have noticed that the second part of his
proposition could be proved in this way. It seems therefore that, in giving
the other and longer method, he deliberately wished to avoid using the result
of VI. 19, preferring to prove the first two parts of the theorem, as they can be
proved, independently o( any relation between the areas of similar triangles.

The first part of the Porism, stating that the theorem is true oi quadriiaitrals,
would be superfluous but for the fact that technically, according to Book i.
Def, 19, the term ``polygon ``(or figure of many sides, iroXuTrXtupoi') used in the
enunciation of the proposition is confined to rectilineal figures of more than
four sides, so that a quadrilateral might seem to be excluded. The mention
of the triangle in addition fills up the tale of `` similar rectilineal figures.''

The second Porism, Theon's interpolation, given in the text by the editors,
but bracketed by Heiberg, is as follows :

``And, if we talce O a third proportional to AB, EG, then BA has to O a
rtUio duplicate of that which AB has to EC.

But the polygon has also to the polygon, or the quadrilateral to the
quadrilateral, a ratio duplicate of that which the corresponding side has to
the corresponding side, that is AB to EG;
and this was proved in the case of triangles aiso ;

so that it is also manifest generally that, if three straight lines be proportional,
as the first is to the third, so will the figure described on the first be to the
similar and similarly described figure on the second.''

\end{notes}

\end{proposition}

\begin{proposition}
\label{prop:VI_21}

\begin{statement}
Figures which are similar to the same rectilineal figure
are also similar to one another.
\end{statement}

\begin{proof}

For let each of the rectilineal figures A,B]x. similar to C
1 say that A is also similar to .

For, since A is similar to C *

it is equiangular with it and has the sides about the equal

angles proportional. [vi. Def. i]

tn

Again, since B is similar to C,

it is equiangular with it and has the sides about the equal
angles proportional,

Therefore each of the figures A, B is equiangular with C
and with C has the sides about the equal angles proportional;

therefore A is similar to B,
\end{proof}

\begin{notes}

It will be observed that the text above omits a. step which the editions
generally have before the final inference `` Therefore A is similar to B.'' The
words omitted are ``so that A is also equiangular with B and [with B] has the
sides about the equal angles proportional.'' Heiberg follows P in leaving
them out, conjecturing that they may be an addition of Th eon's.

\end{notes}

\end{proposition}

\begin{proposition}
\label{prop:VI_22}

\begin{statement}
J/ /our straight lines be proportional, the rectilineal figures
\end{statement}

\begin{proof}

similar and similarly described upon them -will also be pro-
portional ; and, if the rectilineal figures similar and similarly
described upon them be proportional, the straight lines will
themselves also be proportional.

Let the four straight lines AB, CD, EF, GH be pro-
portional,

so that, as .ff is to CD, so is EFxo GH,
and let there be described on AB, CD the similar and similarly
situated rectilineal figures KAB, LCD,
and on EF, GH the similar and similarly situated rectilineal
figures J/A, NH;
I say that, as KAB is to LCD, so is MF to NH.

For let there be taken a third proportional O to AB, CD,
and a third proportional fi to EF, GH. [vi. n]

Then since, as AB is to CD, so is EF to GH,
and, as CZ? is to O, so is GH to /*,
therefore. Mr aequali, as Wi? is to O, so is EF to P. [v. u]

But, as -f4 is to t>, so is AVJ to ZCZ>,

and, as EF Is to /'', so is MF to A; '''' ``'' **''

therefore also, as KAB is to Z,CA so is MF to A''If, [v. n]

/7

Next. let MF be to AIf as KAB is to ZLCZ? ;
I say also that, as 4 5 is to CD, so is EFto GH.

For, if EF is not to GH as  to CZ?,

let j£''/be to jjj? as AB to CA [vi. u]

and on QR let the rectilineal figure SR be described similar
and similarly situated to either of the two MF, NH. [vi. i8]

Since then, as AB is to CD, so is EF to QR,
and there have been described on AB, CD the similar and
similarly situated figures KAB, LCD, '-

and on EF, QR the similar and similarly situated figures
MF, SR,

therefore, as KAB is to LCD, so is MF to SR.
But also, by hypothesis,

as KAB is to L CD, so is MF to NH ;
'   therefore also, as MF is to SR, so is MF to A''If [v. 1 1]

Therefore MF has the same ratio to each of the figures
NH,SR;

therefore NH is equal to SR. [v. 9]

But it is also similar and similarly situated to it ;
therefore GH is equal to QR.

And, since, as AB is to CD, so is £F to QJ?,

while QH is equal to GIf,

therefore, as AB is to CD, so is EFto GH. `` '``

Therefore etc.
\end{proof}

\begin{notes}

The second assumption in the first step of the first part of the proof, vi.
that, as CD is to O, so GM to J'', should perhaps be explained. It is a
deduction [by v. ii] from the facts that

-. AMis to CJ? as CD to O,

£I!'is to Gffas GUto P,

Mid •- AB is to CD as EFta GH.

The defect in the proof of this proposition is well known, namely the
assumption, without proof, that, because the figures NH, SR are equal,
besides being similar and similarly situated, their corresponding sides GH, QR
are equai. Hence the minimum addition necessary to make the proof
complete is a proof of a lemma to the effect that, iffim similar figurn are also
equal, any pair of corresponding sides are equal.

To supply this lemma is one alternative ; another is to prove, as a
preliminary proposition, a much more general theorem, viz. that, tf the
duplicate ratios of two ratios an equal, the tnio ratios are themselves equal.
When this is proved, the second part of vi, 12 is an immediate infeience from
it, and the effect is, of course, to substitute a new proof instead of
supplementing Euclid's.

I. It is to be noticed that the lemma requitt:d as a mini mum is very like
what is needed to supplement vi. 28 and 19, in the proofs of which Euclid
assumes that, if tivo similar parallelograms are unequal, any side in the greater
is greater than the corresponding side in the smaller. Therefore, on the whole, it
seems preferable to adopt the alternative of proving the simpler lemma which
will serve to supplement all three proofs, vii, that, if of two similar rectilineal
figures the first is greater than, equal to, or less thaft, the second, any side of the
first is greater than, equal to, or less than, the corresponding side of the second
respectively.

The case of equality of the figures is the case required for vi. t2 ; and the
proof of it is given in the Greek text after the proposition, Since to give such
a `` lemma `` after the proposition in which it is required is contrary to Euclid's
manner, Heiberg concludes that it is an interpolation, though it is earlier than
TheotL The lemma runs thus ;

``But that, if rectilineal figures be equal and similar, their corresponding
sides are equal to one another we will prove thus.

Let Nff, SR be equal and similar rectilineal figures, and suppose that,
as HG is to GN, so is RQ to QS',
I say that RQ is equal to HG.

For, if they are unequal, one of them is greater;
let Q be greater than HG.

Then, since, as 0 is to QS, so is I/G to GNi
alternately also, as Q is to IfG, so is QS to GJVi
and Q£ is greater than J/G ;

therefore QS is also greater than GN;

so that /IS is also greater than HN*.

But it is also equal : which is impossible; `` '

Therefore QR is not unequal to GH\ •. vk D (•'

therefore it is equal to it,'' «f. ,>

[The step marked • is easy to see if it is remembered that it is only
necessary to prove its truth in the case of triangles (since similar polygons are
divisible into the same number of similar and similarly situated triangles
having the same ratio to each other respectively as the polygons have). If the
triangles be applied to each other so that the two corresponding sides of each,
which are used in the question, and the angles included by them coincide,
the truth of the inference is obvious.]

The lemma might also be arrived at by proving that, « ratio is greater than
a ratio of equality, tht ratio which «> its dupHeate is also greater than a ratio of
equality ; and if ike ratio lohich is duplicate of anotfur ratio is greater than a
ratio of equally, the ratio of which it is the duplicate is also greater than a ratio
of equality. It is not difficult to prove this from the particular case of v. 25 in
which the second magnitude is equal to the third, i.e. from the fact that in
this case the sum of the extreme terms is greater than double the middle term.

II. We now come to the alternative which substitutes a new proof for the
second part of the proposition, making the whole proposition an immediate
inference from one to which it is practically equivalent, viz. that

(i) If duo ratios be equal, their duplicate ratios are equal, and (2) cott'
versely, if the duplicate ratios of two ratios be equal, the ratios are equal

The proof of part (i) is after the manner of Euclid's own proof of the first
part of VI, a*.

Let j1 be to .5 as C to A
and let Jf be a third proportional to A, B, and Fa third proportional to C, D,
so that

A  !0 B » B to X,

and C is to Z) as 2J to Y

whence A' a X in the duplicate ratio of A to B,

and C is to Y in the duplicate ratio of C to D.

Since j4 13 to i? as C is to A

and  is to Jf as /i is to .5,

i.e. as C is to Z*, , ,

IV. Ill

i.e as /J is to K, .

therefore, ex aequali, /f is to Jf as C is to K -•

Part (2) is much more difficult and is the crux of the whole thing.

Most of the proofs' depend on the assumption that, B being any magnitude
and P and Q two magnitudes of the same kind, there does exist a magnitude
A which is to S in the same ratio as /' to Q, It is this same assumption
which makes Euclid's proof of v. i8 illegitimate, since it is nowhere proved
in Book v. Hence any proof of the proptosition now in question which
involves this assumption even in the case where B, P, Q are all straight lines
should not properly be given as an addition to Book v. ; it should at least be
postponed until we have leamt, by means of vi. iz, giving the actual
construction of a fourth proportional, that such a fourth proportional exists.

Two proofs which are given of the proposition depend upon the following
lemma.

A, B, C  tAru magnitudes of ofte kind, and D, E, F three Mitgnitudes
of one kind, then, if

Ike ratio of A fa B is greater than that of D toE, i '

and tht ratio of `` to Q, greater than that of 'R to'F, '

ex aequali, the ratio of A to C is greater than thai of V to F.

One proof of this does not depend upon (he assumption referred to, and
therefore, if this proof is used, the theorem can be added to Book v. The
proof is that of Hauber (Camerer's Euclid, p. -358 of Vol. u.) and is reproduced
by Mr H. M. Taylor. For brevity we will use symbols.

Take equimultiples m, mD of A, D and nB, nE of B, E such that

mA>nB, but mD'nE.

Also let pB, pE be equimultiples of B. E and qC, qP equimultiples of
C, .f such that

pB>qC, hai pElf-gF.

Therefore:, multiplying the first line by and the second by n, we have
pmA>pnB,pmD1(-pnE, i.

and npS>nqC,npE1f-nqF,

whence pmA>nqC, pmDInqF. '

Now P*nA, pmD are equimultiples of mA, mU,

and nqC, ffl?/ equimultiples of qC, qF.

Therefore [v. 3] they are respectively equimultiples of A, D and of C, F.

Hence [v. Def. 7] A : C>D : F.

Another proof given by Claviua, though depending on the assumption
referred to, is neat
Take G such that

GiC = E'.F. •  -

A

B
A

Therefore
and

Therefore

«i .

 .C>G.C,
B>G.
 .G>AiB.

[v. 13]
[v. 10]
[v. 8]

But

A ;

3>DiE.

Therefore, \emph{a fortiori}.

A :

G>D:E.

Suppose  taken such that

H:

G= D:B.

Therefore

A>H.

Hence

A :

: C>H'. C.

But

H.

.G = D:£,

G .

CE:F.

Therefore, ex

aequali

J5f

; C=£).K

Hence

A :

C>D.R

[V.

'3.

.0]

[V

.8)

[V.

2.]

[V-

 3]

Now we can prove that
Ratios of whUk equai ratios are duplitate are equal.

Suppose that

A :

.B = B:

C,

and

D

:£ = £

:iS

and further that

A :

; €=£>:

E

it is required '0 prove that

A BD-.E.

For, if not, one of the ratios must be greater than the othur.
I>et A : Bxt greater.

Then, since A : B = B : C

and £>:E = E:E,

while A: B>D: E,

it follows that B: C>EF. [v- 13]

Hence, by the lemma, ex aequali,

A : C>D -.F,

which contradicts the hypothesis.

Thus the ratios A : B and D : E cannot be unequal j that is, they are equal.

Another proof, given by Dr Lachlan, also assumes the existence of a
fourth proportional, but dei:)ends upon a simpler lemma to the effect that

If is impussible that two different ratios can have the same duplicate ratio.

For, if possible, let the ratio  :  be duplicate both ol A -.X and A : V,
so that

[V.8]

[v. II, 13]

[V. lo]

A

 .X =

X

 B,

and

A

: y=

: V

 B.

Let X be greater

than

y.

Then

A:

x<

A :

y-.

that is,

X .

B<

Y:

-e,

or

x<

Y.

But X is greater

than

Y:

which IS a

Lbsurd, etc.

Hence

x=

y.

 t6

Now suppose that

A : BB: C,

JDiE = E .F,

and

A : C = D:F.

To prove that

AB = D:E.

If this is not so, suppose

that

A:B = D:Z.

Since

A.CD.F,

therefore, inversely,

C:A=F:D.

Therefore, «c aequali.

C: JS = F:2,

or, inversely.

B: C = Z: F.

Therefore

A.B = Z:F.

But

A : B  D : Zyyj hypothesis.

Therefore

D:Z=Z. F.

Also, by hypothesis,

D:E=E:F;

whence, by the lemma,

E = Z.

Therefore

A-.B-D-.E.

[VI.

[v. J a]
[v. m]
[V. ..J

De Morgan remarks that the best way of remedying the defect in Euclid
is to insert the proposition (the lemma to the last proof) that If is imposiihU
thai two differml ratios can have the same duflitate ratio, ``which,'' he says,
``immediately proves the second (or defective) case of the theorem.'' But this
seems to be either too much or too little : too much, if we choose to make
Che minimum addition to Euclid (for that addition is a lemma which shall prove
that, if a duplicate ratio is a ratio of equality, the ratio of which it is duplicate
is also one of equality), and too little if the proof is to be altered in the more
fundamental manner explained above,

I think that, if Euclid's attention had been drawn to the defect in his
proof of VI. 2 2 and he had been asked to remedy it, he would have done so
by supplying what I have called the minimum lemma and not by making the
more fundamental alteration. This I infer from Prop, 24 of the Data, where
he gives a theorem corresponding to the proposition that ratios of which equal
ratios, are duplicate are equal. The proposition in the Data is enunciated
thus : If three straight lines be proportional, and the first have to the third a
given ratio, it wilt also have to the second a given ratio.

A, B, C being the three straight lines, so that

A .B = B.C,

and A : C being a given ratio, it is required to prove that A : B vi also a

given ratio.

Euclid takes any straight line D, and first finds another, F, such that
D .F=A -.C,
whence D : F must be a given ratio, and, as  is given, F is therefore given.

Then he takes E a mean proportional between D, P, so that

D .£ = E:F.

It follows [vi. 17] that
the rectangle D, Ph equal to the square on E.
But D, Fzxe both given ;

therefore the square on E is given, so that E is also given.
[Observe that De Morgan's lemma is here assumed without proof. It
may be proved (ij as it is by I>e Morgan, whose proof is that given above,
p. 345, (i) in the manner of the ``minimum Semma,'' pp. 342 — 3 above:, or
(3) as it is by Proclus on i. 46 (see note on that proposition).]
Hence the ratio D \ Es given.
Now, since A : C= D : J<,

and A : C= (square an A): (rect. A, C),

while D : F~ (square on D') : (tect. D, F), [vi. i]

therefore (square on A) ; (rect. A, C)~ (square on D) • (rect. D, F). [v. 11]
But, since A -. B = B : C, (rect. A, C) - (sq. an S); [vi. 17]

and (rect. £>, F) = (sq. on E), from above ;

therefore (square on A) : (square on ) = (sq, on i?) : (sq. on £).   '

Therefore says Euclid, • 1 w

A;B = D:E,
that is, ht assumes the truth of •i. a fat squarts.

Thus he deduces his proposition from vi, 21, instead of proving vi. 33 by
means of it (or the corresponding proposition used by Mr Taylor and
Or Ichlan).

\end{notes}

\end{proposition}

\begin{proposition}
\label{prop:VI_23}

\begin{statement}
Equiangular parallelograms have to one another the ratio
compounded of the ratios of their sides.
\end{statement}

\begin{proof}

Let AC, CF be equiangular parallelograms having the
angle BCD equal to the angle ECG ;
jl say that the parallelogram AC has to the parallelogram
CF the ratio (;ompounded of the ratios of the sides.

For let them be placed so that BC is in a straight line
with CG ;

therefore DC is also in a straight line with CE.
I Let the parallelngram DG be completed ;
let a straight line K be set out, and let it be contrived that,

as BC is to CG, so is i to Z. ,
and, as DC is to CE, so is L to M. [vi. 12]

348 BOOK VI [vi. »3

Then the ratios of K to L and oi L, Xo M are the same
ij as the ratios of the sides, name]y of BC to CG and of DC
to CE.

But the ratio of A' to  is compounded of the ratio of K
to L an<i of that of Z to M;

so that JC has also to M the ratio compounded of the ratios
ao of the sides.

Now since, as BC is to CG, so is the parallelogram AC
to the parallelogram CIf, [vi. i]

while, as BC is to CG, so is A'' to Z,

therefore also, as K is to L, so is AC to CIf. [v. n]

as Again, since, as Z?C is to C£, so is the parallelogram CIf
to CjS [vi. r]

while, as DC is to C£, so is L to ,

therefore also, as Z. is to M, so is the parallelogram C/f to
the parallelogram CK [v, 1 1]

30 Since then it was proved that, as /f is to Z, so is the
parallelogram AC to the parallelogram CIf,

and, as Z is to , so is the parallelogram CH to the
parallelogram CB'',

therefore, mt aeuali, as A' is to M, so is AC to the paralle lo-
ss gram CF.

But A'' has to M the ratio compounded of the ratios of

the sides ;

therefore A C also has to CF the ratio compounded of the

ratios of the sides.

40 Therefore etc,
\end{proof}

\begin{annotations}

1,6, 19, 36. the ratio compounded of the ratios of the sides, Mw riy rirrmifunr
4t T«>' TrXtvpi which, nieining literally `` the ratio coropounded 0/ (it lijes,'' is negligently
writlen here and commonly for Xiryav rir rvyKtiftrov in t«v tup i[xvpi4» (sc. Xyuv)*

11. let it be contrived that, as BC is to CO, so is K to L, The Greek phraiie in
of the usual tere kind, untranslatable literally : xol yeyofirw ajt ftir  EF Tpit rip TE,
offrwi 17 K irpij rf A, the words meaning ``and let (there) be made, u BC to CG, sa Kio
Z,'' where L is the straight line which has to be constructed.

\end{annotations}

\begin{notes}

The second definition of the Daia says that A ratio is said to fe
given if we can find (Tropi'o-ao-flot) [another ratio that />] the
same with it. Accordingly VI, 13 not only proves that equiangular
parallelograms have to one another a ratio which is compounded of two
others, but shows that that ratio is ``given'' when its component
ratios are given, or that it can be represented as a simple ratio
between straight lines.

Just as vt. 13 exhibits the o[>eration necessary for eomfounding two
ifttios, a proposition (8) of the Data indicates the operation by which we may
divide one ratio hy another. The proposition proves that Thingt which
have a given raiio to the same thing have also a grven ratio to one another.
Euclid's procedure is of course to comf>ound one ratio vrith the inverse of the
other ; but, when this is once done and the result of Prop. 8 obtained, he
uses the result in the later propositions as a substitute for the method of
composition. Thus he uses the division of itios, instead of composition,
in the propositions of the Data which deal with the same subject-matter as
VL 13. The effect is to represent the ratio of two equiangular parallelcrams
as a ratio between straight lines one of which is one side of one of the
paraiMograms, Prop. 56 of the Data shows us that, if we want to express
the ratio of the parallelogram AC Xx> the paraDelogram CF in the figure

/

ov. ur.'

'.I (

..I   '

....    , A''

of VI. 23 in the form of a ratio in which, for exampie, the side flC is the
antecedent term, the required i-atio of the parallelograms is BC ; X-, where

DC:CE=CG: X,
or A'' is a fourth proportional to DCarA the two sides of the parallelogram CF.
Measure CK along CB, produced if necessary, so that
DC .CE=CG . CK
(whence CK is equal to X). *

[This may be simply done by joining DG and then drawing EK parallel
to it meeting CB in /T.]

Complete the parallelogram AK.
Then, since DC : CE  CG : CK,

the parallelcrams DK, Care equal. [vi. 14]

Therefore (AC).(CF) = (AC).(DK) >   [v. 7]

= BC .CK [VI. i]

BC.X.

Prop. 68 of the Data uses the same construction to prove that. If two
equiangular parallelograms havt to one another a given ratio, and one side have
to one side a given ratio, the remaining side will also have to the remaining side
a given ratio.

I do not use the figure of the Data but, for convenience' sake, I adhere
to the figure given above. Suppose that the ratio of the parallelograms is
given, and also that of CD to CE.

Apply to CD the parallel<ram DK equal to CE and such that CK, CB
coincide in direction. [i. 45]

TTien the ratio of AC to KD is given, being equal to that of ACio CF.
And (AC) : (KB) = CB : CK;

therefore the ratio of CB to CK is given.

But. since KD= CF,

Cn-.CECG: CK. [vi. 14]

Hence CG : CK'xs a given ratio.

And CB : CK was proved to be a given ratio.

Therefore the ratio of CB to CG is given. Data, Prop. 8]

lastly we may refer to Prop. 70 of the Data, thL- first part of which proves
what corresponds exactly to vi. 33, namely that, If in ttvo fquiangular paral-
lelograms Ihe sides containing tht equal angles ham a given ratio to one atmlher
[i.e. one side in one to one side in the other], the paralUlograms themselves will
also have agiivn rat is to one another. [Here the ratios of BC to CG and of
CD to CE are given.]

The construction ts the same as iti the last case, and we have KD equal
to CF so that

CD . CE=CG : CK. [vi. 14]

But the ratio of CD to CE is given ;
therefore the ratio of CG to Cis given.

And, by hypothesis, the ratio of CG to CB is given.

Tlierefore, by dividing the ratios [ZJd/ir, Prop. 8], we see that the ratio of
CB to CK and therefore [vj, i] the ratio of .C to DK, or of AC to CF,
is given.

Euchd extends these propositions to the case of two parallelograms which
have givfn but not equal angles.

Pappus (v[i. p. 928) exhibits the result of vi. 13 in a different way,
which throws new light on compounded ratios. He proves, namely, that a
paratUlograin is to an equiangnlar parallelogram as the rectangle amtained hy
tlu adjacent sides of the first is to the rectangle contained by the adjacent sides
oj the second,

A

o

C E

t-et A C, OF he equiangular parallelograms 01 r the bases BC, EF, and let
the angles at .5, £ he equal.

Draw perjjendiculars AG, DH .o BC, £/'' respectively.
Since the angles at B, G are equal to those at E, H,

the triangles ABG, DEHsliu equiangular.
' Therefore BA : AG ED: DM. [vi. 4]

But BA 'AG = (rect. BA, BC) : (rect. AG, BC),

and ED : DH  (rect, ED, EF) : (rect DH, EF). [vi, i]

Therefore [v. [ 1 and v. 16]

(TticX.. AB, BC) : (rect. DE, EF) = (rect. AG, BC) : (rect DH, EF)

= (AC).(DF).

Thus it is proved that the ratio compounded of the ratios A3 : DE and
BC : EF is equal to the ratio of the rectangle AB, BC to the rectangle
DE, EF.

Since each parallelogram in the figure of the proposition can be divided
into pairs of equal triangles, and all the triangles which are the halves of either
parallelogram have two sides respectively equal and the angles included by
them equal or supplementary, it can be at once deduced from vt. 23 (or it
can be independently proved by the same method) that triangles which have
ont angle of the one equal or supplementary to one angle of the other are in (he
ratio (ompoa tided of the ratios of the sides about the eqval or supplementary
angles. Cf- Pappus vii. pp. 894—6.

VI. 23 also shows that rectangles, and therefore parallelograms or triangles,
are to one another in the ratio compounded of the ratios of theit bases and
heights.

The converse of vi. 23 is also true, as is easily proved by reductio ad
absurdum. More generally, if two parallelograms or triangles are in the ratio
compounded of the ratios of two adjacent sides the angles included by those sides
are either equal or supplementary.

\end{notes}

\end{proposition}

\begin{proposition}
\label{prop:VI_24}

\begin{statement}
In any parallelogram the parallelograms about the diameter
are similar both to the whole and to one another.
\end{statement}

\begin{proof}

Let ABCD be a parallelogram, and AC its diameter,
and let EG, HK be parallelograms
about AC

I say that each of the parallelograms
EG, HK is similar both to the whole
ABCD and to the other.

For, since EF has been drawn
parallel to BC, one of the sides of the
triangle ABC,

proportionally, as BE is to EA, so is CF to FA. [vi. 2]

Again, since FG has been drawn parallel to CD, one of
the sides of the triangle ACD,

proportionally, as CF\ to FA, so is DG to GA. [vi. 2],

But it was proved that,

as CF is to FA, so also is BE to EA ;

therefore al.so, as BE is to EA, so is DG to GA,
and therefore, componendo,

as BA is to AE, so is DA to AG, y. 18]

and, alternately,

as BA is to AD, so is EA to AG. [v. 16]

Therefore in the parallelograms ABCD, EG, the sides
about the common angle BAD are proportional.

And, since GF is parallel to DC,

the angle AFG is equal to the angle DC A ;
and the angle DAC is common to the two triangles ADC,
AGF,

therefore the triangle ADC is equiangular with the triangle
AGF.

For the same reason

the triangle ACB is also equiangular with the triangle
AFE.

and the whole parallelogram ABCD is equiangular with the
parallelogram EG.

Therefore, proportionally,

as AD is to DC, so is v4G to GF,

as DC is to CA, so is GF to FA.,

as C is to CB, so is AF to FE,

and further, as CB is to BA, so is FE to EA.

And, since it was proved that,
as DC is to CA, so is GF to /v?,
and, as AC is to C, so is AF to /£,
therefore, ex aequali, as Z?C is to CB, so is GF to 7/5'. [v. jJ

Therefore in the parallelograms ABCD, EG the sides
about the equal angles are proportional ;

therefore the parallelogram ABCD is similar to the parallelo-
gram EG. [vi. Def. i]

For the same reason
the parallelogram ABCD is also similar to the parallelogram
KH',

therefore each of the paralleJograms EG, HK is simitar to
ABCD.

But figures simlliar to the same rectilineal figure are also
similar to one another ; [vi. ai]

therefore the parallelogram EG is also similar to the parallelo-
gram HK.

Therefore etc.
\end{proof}

\begin{notes}

Simson was of opinion that this proof was made up by some unskilful
editor out of two others, the first of which proved by parallels (vi. 2) that
the sides about the common angle in the parallelograms are proportional,
while the other used the similarity of triangles (vi. 4.). It is of course true
that, when we have proved by vi. 2 the fact that the sides about the common
angle are proportional, we can infer the proportionality of the other sides
directly from 1, 34 combined with v. 7. But it does not seem to me unnatural
that Euclid should (i) deliberately refrain from making any use of t. 34 and
(3) determine beforehand that he would prove the sides proportional in a
difinite ordtr beginning with the sides EA, AG and BA, AD about the
common angle and then taking the remaining sides in the order indicated
by the order of the letters A, G, F, E. Given that Euclid started the proof
with such a fixed intention in his mind, the course taken presents no difficulty,
nor is the proof unsystematic or unduly drawn out. And its genuineness
seems to me supported by the fact that the proof, when once the first two
sides about the common angle have been disposed of, follows closely the
order and method of vi. 18. Moreover, it could readily be adapted to the
more general case of two p>olygons having a common angle and the other
corresponding sides respectively parallel.

The parallelograms in the proposition are of course similarly situated as
well as similar; and those ``about the diameter'' may be ``about'' the
diameter produced as well as about the diameter itself

From the first part of the proof it follows that parallelograms which have
one ante equal to one angle and the sides about those angles proportional
are similar.

Prop. 26 is the converse of Prop. 24, and there seems to be no reason
why they should be separated as they are in the text by the interposition of
VI.  i. Campanus has vi. 24 and 26 as vi. 22 and 23 respectively, vi. 23 as
VI, 14, and VI. 35 as we have it.

\end{notes}

\end{proposition}

\begin{proposition}
\label{prop:VI_25}

\begin{statement}
To construct one and the same figure similar to a given
rectilineal figure and equal to another given rectilineal figure.
\end{statement}

\begin{proof}

Let ABC be the given rectilineal figure to which the
figure to be constructed must be similar, and D that to which
it must be equal ;

thus it is required to construct one and the same figure similar
to ABC and equal to D.

Let there be applied to BC the parallelogram BE equal
to the triangle ABC [1. 44], and to CE the parallelogram CM
equal to D in the angle FCE which is equal to the angle
CBL. [1. 45]

 Therefore BC is in a straight line with CF, and LE with
EM.

Now let GH be taken a mean proportional to BC, CF
[vL 13J, and on GHlel KGHh described similar and similarly
situated to ABC. y. 18]

Then, since, as BC is to GH, so is GH to CF,
and, if three straight lines be proportional, as the first is to
the third, so is the figure on the first to the similar and
similarly situated figure described on the second, [vi. 19, For,]
therefore, as BC is to CF, so is the triangle ABC to the
triangle KGH.

But, as BC is to CF, so also is the parallelogram BE to
the parallelogram EF. [vi. i]

Therefore also, as the triangle ABC is to the triangle
KGH, so is the parallelogram BE to the parallelogram EF
therefore, alternately, as the triangle ABC is to the parallelo-
gram BE, so is the triangle KGH to the parallelogram EF.

[v. 16]

But the triangle ABC is equal to the parallelogram BE ;
therefore the triangle jf 6'If' is also equal to the parallelogram
EF.

But the parallelogram EF is equal to D
therefore KGH is also equal to D,

And KGH is also similar to ABC.

Therefore one and the same figure KGH has been con-
structed similar to the given rectilineal figure ABC and equal
to the other given figure D.
\end{proof}

\begin{annotations}

3. to which the flgiire to be const ructed muBt be similar, literally `` lo which it
Is requiied to construct (one) Eimiisr,''  itiinoaio (rwrriiffMSm.

\end{annotations}

\begin{notes}

This is the highly important problem which Pythagoras is credited with
having solved. Compare the passage from Plutarch (Symp. vtu. 2, 4) quoted
 n the note on i. 44 above, Vol, 1. pp. 343 — 4.

We are bidden to construct a rectilineal figure which shall have the form of
one and the siu of another rectilineal figure. The corresponding proposition
of the Data, Prop. 55, asserts that, ``if an area (xtapiav) be given in form
((iSii) and in magnitude, its sides will also be given in magnitude.''

Simson sees signs of corruption in the text of this proposition also. In
the first plac the proof speaks of the triangle ABC, though, according to the
enunciation, the figtire for which ABC is taken may be any rectilineal figure,
tutfuypafifiop ``rectilineal figure'' would be more correct, or <ISo9, ``figure''; the
mistsjie, however, of using rpiyavoy is not one of great importance, being no
doubt due to the accident by which the tigure was drawn as a. triangle in the
diagram.

The other observation is more important. After Euclid has proved that

(fig. ABC) ; (fig. JCG/T) = (B£) : (Ef),

he might have inferred directly from v. 14 that, since ABC is equal to BE,
KGH is e<jual to EF. For v. 14 includes the proof of the fact that, if A is
to . as C IS to D, and A is etjual to C, then B is equal to D, or that of four
proportional magnitudes, if the first is equal to the third, the second is equal
to the fourth. Instead of proceeding in this way, Euclid first permutes the
proportion by v. 16 into

(fig. ABC) : (BE) = (fig. KGH) : (EF),

arid then infers, as if the inference were easier in this form, that, since the
Jlftt is equal to the secend, the third is equal to the fourth. Yet there is no
proposition to this effect in Euclid. The same unnecessary step of permutation
is also found in the Greek text of xi. 23 and xii. 2, 5, ir, 12 and 18. In
reproducing the proofs we may simply leave out the steps and refer to v. 14.

\end{notes}

\end{proposition}

\begin{proposition}
\label{prop:VI_26}

\begin{statement}
If frotn a parallelogram there be taken away a parallelo-
gram similar and mnilarly situated to the whole and having
a common angle with it, it is about the same diameter with the
whole
\end{statement}

\begin{proof}

For from the parallelogram A BCD let there be taken
away the parallelogram AF similar and
similarly situated to ABCD, and having
the angle DAB common with it ;

I say that ABCD is about the same
diameter with AF.

For suppose it is not, but, if possible,
let AHCh the diameter < oiABCD > ,
let GF be produced and carried through
to ff, and let /fK be drawn through If
parallel to either of the straight lines AD, BC. [i. 31]

Since, then, ABCD is about the same diameter with KG,
therefore, as DA is to AB, so is GA to AK. [vi. 24]

But also, because of the similarity of ABCD, EG,

as DA is to AB, so is GA to AE ;

therefore also, as GA is to AK, so is GA to AE. [v. 1 1]

Therefore GA has the same ratio to each of the straight
lines AK, AE.

Therefore A£ is equal to A' [v. 9], the less to the
greater: which is impossible.

Therefore A BCD cannot but be about the same diameter
with AF

therefore the parallelogram A BCD is about the same diameter
with the parallelogram AF.

Therefore etc.
\end{proof}

\begin{notes}

`` For suppose it is not, but, if possible, let AHC be the diameter.'' WKat
is meant is `` For, if AFC is not the diameter of the paralleiogram AC, let
AlfC be its diameter.'' The Greelt text has fo-Tut amav Sta/wrpiK ij AOT;
but clearly avrwv is wrong, as we cannot assume that one straight line is the
diameter of both parallelograms, which is just what we have to prove. F and
V omit the auTiui-, and Heiberg prefers this correction to substituting aurou
after Peyrard. I have inserted `` < of A BCD > `` to make the meaning clear.

If the straight line AHC does not pass through F, it must meet either
GF (X £?/' produced in some point H. The reading in the text ``and let
GF be produced and carried through io H `` (uat imX-ifivxta,  HZ 7xSm Jjrt
TO ®) corresponds to the supposition that H is on GF produced. The words
were left out by Theon, evidently because in the figure of the mss. the letters
E, Z and K, were interchanged, Heiberg therefore, following August, has
preferred to reKiin the words and to correct the figure, as well as the passage in
the text where AE, AK were interchanged to be in accord with the MS. figure.

It is of course possible to prove the proposition directly, as is done by
Dr Lachlan. Let AF, ACh the diagonals, and let us matte no assumption
as to how they fall.

Then, since EFs parallel to AG and therefore to BC,

the angles AEF, ABC are equal. c

And, since the paraRetograms are similar,

AE : EF=AB : BC. [vi. Def. i]

Hence the triangles AEF, ABC are similar, [vi. 6]

and therefore the angle FAE is equal to the angle CAB.

Therefore AFfalh on AC.

The proposition is equally true if the parallelogram which is similar and
similarly situated to the given parallelogram is not `` taken
away'' from it, but is so placed that it is entirely outside the
other, while two sides form an angle vertically opposite to
an angle of the other. In this case the diameters are not
``the same,'' in the words of the enimciation, but are in
a straight line with one another. This extension of the
proposition is, as will be seen, necessary for obtaining,
according to the method adopted by Euclid in his solu-
tion of the problem in vi. s8, the second solution of that
problem.

\end{notes}

\end{proposition}

\begin{proposition}
\label{prop:VI_27}

\begin{statement}
0/ all ike parallelograms applied to the same straight line
and deficient by parallelogrammic figures similar and similarly
situated to that described on the half of the straight line, that
parallelogram is greatest which is applied to the half of the
straight line and is similar to the defect.
\end{statement}

\begin{proof}

Let AB be a straight line and let it be bisected at C;
let there be applied to the straight
line AB the parallelogram AD
deficient by the parallelogrammic
figure DB described on the half of
AB, that is, CB;

I say that, of all the parallelograms
applied to AB and deficient by
parallelogrammic figures similar and
similarly situated to T)B, A J? is greatest.

For let there be applied to the straight line AB the
parallelogram AF deficient by the parallelogrammic figure
FB similar and similarly situated to DB;
I say that AD is greater than AF.

P'or, since the parallelogram DB is similar to the parallelo-
gram FB,

they are about the same diameter. [vi. 36]

Let their diameter DB be drawn, and let the figure be
described.

Then, since CF is equal to F£,  [i. 43]

and FB is common,

therefore the whole CIf is equal to the whole /CF.

But CIf is equal to CG, since AC is also equal to CB,

[I. 36]
Therefore GC is also equal to F/l.
Let CF be added to each ;

therefore the whole AFis equal to the gnomon LMN

so that the parallelogram DB, that is, AD, is greater than
the parallelogram AF.

Therefore etc.
\end{proof}

\begin{notes}

We have already (note on i. 44) seen the significance, in Greek geometry,
of the theory of `` the application of areas, their exceeding and their falling-
short.'' In I. 44 it was a question of `` applying to a given straight line
(exactly, without 'excess' of 'defect') a parallelogram equal to a given
rectilineal figure, in a given angle,'' Here, in vi. 27 — 29, it is a question
of parallelograms applied to a straight line but ``deficient (or exceeding) by
paralUiograms similar aiid similarly
situated to a given parallelogram.''

Apart from size, it is easy to construct  7

any number of parallelograms `` de- r / 7''

ficient'' or ``exceeding'' in the manner / / / A l

described. Given the straight line /' e F (™......

AB to which the parallelogram has to L . ('....s/

be applied, we describe on the base   '

CB, where C is on AB, or on BA

produced beyond A, any parallelogram `` similarly situated `` and either equal
or similar to the given parallelogram (Euclid takes the similar and similarly
situated parallelogram on half the line), draw the diagonal BD, take on it
(produced if necessary) an> points as E, K, draw EF, or KL, parallel to CI)
to ineet AB ot AB produced and complete the parallelograms, as AH, ML,

If the point E is taken on BD or BD produced beyond Z>, it must be so
taken that EF meets AB between A and B. Otherwise the parallelogram
AE would not be applied to AB itself, as it is required to be.

The parallelograms BD, BE, being about the same diameter, are similar
[vi, 24], and BE is the defect of the parallelogram AE relatively to AB,
AE is then a parallelogram applied ta AB but deficient by a parallelogram
similar and similarly situated to BD,

If A* is on D3 produced, the parallelogram BK is similar to BD, but it
is the excess of the parallelogram AK relatively to the base AB. AK h a.
parallelcram applied to AB but exceeding by a parallelogram similar and
similarly situated to BD.

Thus it is seen that BD produced both ways is the locus of points, such
9S E OT K, which determine, with the direction of CD, the position of A, and
the direction of AB, parallelograms applied to AB and deficient or exceeding
by parallelograms similar and similarly situated to the given parallelogram.

The importance of vi. 27 — 29 from a historical point of view cannot be
overrated. They give the geometrical equivalent of the algebraical solution
of the most general form of quadratic equation when that equation has a real
and positive root. It will also enable us to find a real negative root of a
quadratic equation ; for such an equation can, by altering the sign of x, be
turned into another with a real positive root, when the geometrical method
again becomes applicable. It will also, as we shall see, enable us to represent
both roots when both are real and positive, and therefore to represent both
roots when both are real but either positive or negative.

The method of these propositions was constantly used by the Greek
geometers in the solution of problems, and they constitute the foundation of
Book X. of the Elements and of ApoUonius' treatment of the conic sections.
Simsons observation on the subject is entirely justified. He says namely on
VI. 28, 29: ``These two problems, to the first of which the 27th Prop, is
necessary, are the most general and useful of all in the Elements, and are
most frequently made use of by the ancient geometers in the solution of
other problems ; and tbereft)re are very ignorantly left out by Tacquet and

Dechales in their editions of the Elements, who pretend that they are scarce
of any use.''

It is strange that, with this observation before him, even Todhunter should
have written as follows. `` We have omitted in the sixth Book Propositions
37, z8, 39 and the first solution which Euclid gives of Proposition 30, as they
appear now to be never required, and have been condemned as useless t
various modern commentators ; see Austin, Walker and Lardner.''

VI. 27 contains the Stapto-ftat, the condition for a real solution, of the
problem contained in the proposition following it The maximum of a!l the
parallelograms having the given property which can be applied to a given
straight line is that which is described upon half the line (to airo ts (luntat
ayaypa/urmr). This corresponds to the condition that an equation of the
form

u' -/jc* = A

may have a real root. The correctness of the result may be seen by taking
the case in which the parallelograms are
rectangles, which enables us to leave out
of account the sint of the angle of the
parallelograms without any real loss of
generality. Suppose the sides of the rect-
angle to which the dtfeci is to be similar
to be as £ to f,  corresponding to the
side of the defect which lies along AB.
Suppose that AKFG is any parallelogram

applied to AB having the given property, that AB=a, and that JFK=x.
Then

KB = -x, and therefore AK=a- - x.

c (

Hence [a-- xx=-S, where S is the area of the rectangle AKFG.
Thus, given the equation

F

X

C K

c

where 5 is undetermined, vi. J7 tells us that, if x is to have a real value, S
cannot be greater than the rectangle CE,

Now CB = -, and therefore CD = -..-;
a' b 2'

whence 5 It i . — ,

which is just the same result as we obtain by the algebraical method.

In the particular case where the defect of the parallelogram is to be t
square, (he condition becomes the statement of the fact that, if a straight lint
be divided into two parts, the rectangle contained by the parts cannot exceed the
squart on half the line.

Now suppose that, instead of taking F on BD as in the 6gure of the
proposition, we take F on BD produced beyond D but so that DF is less
than BD.

Complete the figure, as shown, after the manner of the construction in
the proposition.

Then the pamllelogram FKBH is similar to the given paratlelogram to
which the defect is to be simikr. Hence the parallelogram GAKF is also a
paralletograin applied to AB and satisfying
the given condition.

We can now prove that GAKF is less
than CE or AD.

Let ED produced meet AG in O.

Now, since BF is the diagonal of the
parallelogram KH, the complements KD,
DH are equal.

But

DH= DG, and DG is greater than OF.
Therefore KD > OF.

Add OK to each ;
and AD, or CE,>AF,   `` '

This other " case `` of the proposition is found in all the MSS,, but Heiberg
relegates it to the Appendix as being very obviously interpolated. The
reasons for this course are that it is not in Euclid's manner to give a separate
demonstration of such a `` case ``; it is rather his habit to give one case only
and to leave the student to satisfy himself about any others (cf. i, 7). Internal
evidence is also against the genuineness of the separate proof. It is put after
the conclusion of the proposition instead of before it, and, if Euclid had intended
to discuss two cases, ne would have distinguished them at the beginning of
the proposition, as it was his invariable practice to do. Moreover the second
``case'' is the less worth giving because it can be so easily reduced to the
first. For suppose F' to be taken on BD so that FD - F D. Produce BF
to meet AG produced in P. Complete the parallelogram BAPQ, and draw
through F'' straight lines parallel to and meeting its opposite sides.

Then the complement F'Q is equal to the complement AF'.

And it is at once seen that AF, F'Q are equal and similar. Hence the
solution of the problem represented by AF or F'Q gives a parallelogram of
the same size as AF'' arrived at as in the first `` case.''

It is worth noting that the actual difference between the parallelogram
AF and the maximum area AD that it can possibly have is represented in
the figure. The difference is the small parallelcram DF.

\end{notes}

\end{proposition}

\begin{proposition}
\label{prop:VI_28}

\begin{statement}
To a given straight line to apply  parallelogram equal to
a given rectilineal figure and deficient by a parallelogrammic
figure similar to a. given one : thus the given rectilineal figure
must not be greater than the parallelogram, described on the
half of the straight line and similar to the defect.
\end{statement}

\begin{proof}

Let AB be the given straight line, C the given rectilineal
figure to which the figure to be applied to AB is required to
be equal, not being greater than the parallelogram described
on the half of AB and similar to the defect, and D the
parallelogram to which the defect is rt;quired to be similar ;

thus it is required to apply to the given straig line AB a
parallelogram equal to the given rectilineal figure C and
deficient by a parallelogrammic figure which is similar to D,

Let ABhe bisected at the point E, and on EB let EBFG
be described similar and similarly situated to D ; [vi. 18]

let the parallelogram AG he completed.

If then AG IS equal to C, that which was enjoined will
have been done ;

for there has been applied to the given straight line AB
the parallelogram AG equal to the given rectilineal figure C
and deficient by a parallelogrammic figure GB which is similar
to/?.

But, if not, let IfE be greater than C,
Now HE is equal to GB ;

therefore GB is also greater than C,

Let KLMN be constructed at once equal to the excess
by which GB is greater than C and similar and similarly
situated to D. [vi, 25)

But D is similar to GB
therefore KM s also similar to GB. [vi. zi'']

Let, then, KL correspond to GE, and LM to GF.
Now, since GB is equal to C, KM,

therefore GB is greater than KM ;
therefore also GE is greater than KL, and GF than LM.

Let GO be made equal to KL, and GP equal to LM;
and let the parallelogram OCPQ be completed ;
therefore it is equal and similar to KM.

Therefore GQ is also similar to GB; y. i

therefore GQ is about the same diameter with GB. [vi. a6]

afo r BOOK VI .;t't [vi. aS

,. Let C55 be their diameter, and let the figure be described.
Then, since BG is equal to C, KM,

and in them GQ is equal to KM,

therefore the remainder, the gnomon UWV, is equal to tht

remainder C.

And, since PR is equal to OS, ''

let QB L e added to each ;  * '
therefore the whole PB is equal to the whole OB.

But OB is equal to TE, since the side AE is also equal
to the side EB ; [1. 3]

therefore TE is also equal to PB.

Let 05 be added to each ;

therefore the whole TS is equal to the whole, the gnomon
VWU.

But the gnomon VWU was proved equal to C ;
therefore TS is also equal to C.

Therefore to the given straight line AB there has been
applied the parallelogram ST equal to the given rectilineal
figure Cand deficient by a parallelogrammic figure  which
is similar to D.

Q.E.F.
\end{proof}

\begin{notes}

The second part of the enunciation of this proposition which states the
SiapuT/uH appears to have been considerably amplified, but not improved in
the process, by Theon. His version would read as follows. `` But the given
rectilineal figure, that namely to which the applied parallelogmm must be
equal (tf S«t lo-ov impaaXtiv), must not be greater than that applied to the half
(irapa/3iiAAo(Miv instead of avaypa/inov), the defects being similar, (namely)
that (of the parallelogram applied) to the half and that (of the required
parallelogram) which must have a similar defect'' (onoimr otrui' tuv IWtt/i-
fidrwv roD rt air rij fjiurtvK koI cp Sft uftoiov iXKtiTrnv). The first ampHflCfttion
``that to which the applied parallelogram must be equal'' is quite unnecessary,
since `` the given rectilineal figure `` could mean nothing else. The above
attempt at a translation will show how difficult it is to make sense of the
words at the end , they speak of two defects apparently and, while one may
well be the `` defect on the half,'' the other can hardly be tfu pvm paralUiaam
`` to which the defect (of the required [rallelogram) must be similar.'' Clearly
the reading given above (from P) is by far the better.

In this proposition and the next there occurs the tacit assumption (already
alluded to in the note on vi. 31) that if, of two similar paraiitlogramz, one is
gnatfr than the of her, ttther side of the greater is greater tkan the corresponding
side of the kit.

As already remarked, vi. 28 is the geometrical equivalent of the solution
of the quadratic equation

tf jc - - a*

c

S.

subject to the condition necessary to admit of a real solution, namely that

* 4

The corresponding proposition in the Data is (Prop. 58), If a given (ana)
be applied (i.e. in the fonn of a parallelogram) to a given straight line and be
defieitnt by a figure (i.e. a parallelogram) f<Wfl in species, the breadths of tht
defeet are given.

To exhibit the exact correspondence between Euclid's geometrical and
the ordinary algebraical method of solving the equation we will, as before
(in order to avoid bringing in a constant dependent on the sine of the angle
of the parallelograms), suppose the parallelograms to be rectangles. To solve
the equation algebraically we change the signs and write it

-x'-ax = -S.

e a*

We may now complete the square by adding

*'4'

Thus

 <Mf + T. — -

b 4

e a „

and, extracting the square root, we have

Now let us observe Euclid's method. -

o

S

X

s a

 0

He first describes GEBF on EB (half of AB'') similar to the given
parallelogram D.

He then places in one angle FGE of GEBF a similar and similarly
situated parallelogram GQ, equal to the difference between the parallelogram
GB and the «irea C.

With our notation,
whence

GO: OQ = e:b,

»

OQ = GO.'

Similarly - = £B=GE.-,

so that GE=i.-. '

a

Therefore the parallelogram GQ GO* . - ,

( a'
and the parallelogram GB = 1  —

Thus, in taking the parallelogram GQ equal to (GB - S), Euclid really
finds GO from the equation

GO'.''-i.''--S.

f * 4

The value which he finds is

and he finds QS (or x) by subiracting GO from GE ; whence
t a f'cfe a«

It will be observed that Euclid only gives one solution, that corresponding
to the negative sign before the radical. But the reason must be the same as that
for which he only gives one ``case'' in vi. s 7. He cannot have failed to see how
to adti GO to GE would give another solution. As shown under the last
proposition, the other solution can be arrived at

(i) by placing the parallelogram GOQ/' in B' A'

the angle vertically opposite to FGE so that '. ''--.Q'

G( lies along BG produced. The parallelo- \ Yx ;
gram A Q then gives the second solution. The \ I \ \ \

side of this parallelogram lying along AB   \ p' x\ '. '

equal to S3. The other side is what we have \ T gVv p\
called Xy and in this case \ \ \ \ ',

xEG+GO J \ o\ iQ

' *   2  V i (i  4 -j

S B

(z) A parallelogram similar and equal to A< can also be obtained by
producing BG till it meets A T produced and completing the parallelogram
JfABA', whence it is seen that the complement QA' is equal to the comple-
ment j4, besides being equal and sin'ilar and similarly situated to AQ.

A particular case of this proposition, indicated in Prop. 85 of the Data, is
that in which the sides of the defect are equal, so that the defect is a rhombus
with a given angle. Prop. 85 proves that, 1/ two straight lints (oniain a
givtn area in a given angle, and the sum

of the straight lines be given, each of them E *

will be gioen also. AB, BC being the /

given straight lines ``containing a given /

area AC in  given angle ABC,'' one /

side CB is produced to J) so that BD

is equal to AB and the parallelograms are

completed. Then, by hypothesis, CD is of given length, and .C is a parallelogram applied to CZ> falling short by a rhombus (AD) with a given angle
EDB, The case is thus a particular case of Prop. 58 of the Data quoted
above (p. 263) as corresponding to vi. 28.

A particular case of the last, that namely in which the defect \ a squart,
corresponding to the equation

is importai.t. This is the problem of applying to a gwen straight lint a
tedangli tqual to a given area and falling short by a square ; and it can be
solved, without the aid of Book vi., as shown above under 11. 5 (Vol. 1.
PP- 383—4).

\end{notes}

\end{proposition}

\begin{proposition}
\label{prop:VI_29}

\begin{statement}
To a given siraigkt line to apply a parallelogram equal to
a given rectilineal figure and exceeding by a parallelogrammic
figure similar to a given one.
\end{statement}

\begin{proof}

Lei j4B be the given straight line, C the given rectilineal
figure to whtcli the figure to be applied to A J) is required to
be equal, and D that to which the excess is required to be

similar ;

thus it is required to apply to the straight line AB a parallelo-
gram equal to the rectilineal figure C and exceeding by a
parallelogrammic figure similar to D.

Let AB be bisected at  ; ,

let there be described on BB the parallelogram B/ similar
and similarly situated to D ;

and let GIf be constructed at once equal to the sum of BB'',
C and similar and similarly situated to £f. [vi. 35]

Let JCIf correspond to BL and /CG to BB.
Now, since GIf is greater than BB,

therefore JIf is also greater than BL, and JCG than BB.

SH BOOK VI [vi. 39

Let PL, FE be produced,
let FLM be equal to KH, and FEN to KG,
and let MN be completed ;

therefore MN is both equal and similar to GH,
But GH is similar to EL ;
therefore A?'' is also similar to EL ; [vi, a i]

therefore J?/, is about the same diameter with MN. [vi. 16]
Let their diameter FO be drawn, and let the figure be
described.

Since GH's equal to EL, C,
while GH is equal to MN,
therefore MN is also equal to EL, C.
Let EL be subtracted from each ;
therefore the remainder, the gnomon XIVV, is equal to C.
Now, since AE is equal to EB,
AN is also equal to NB [1. 36], that is, to LP [i. 43]-
Let EO be added to each ;

therefore the whole AO is equal to the gnomon VIVX,
But the gnomon VJVX is equal to C

therefore AO is also equal to C.
Therefore to the given straight line AB there has been
applied the parallelogram AO equal to the given rectilineal
figure C and exceeding by a parallelogrammic figure QP
which is similar to D, since PQ is also similar to EL [vi. n].

Q.E.F.
\end{proof}

\begin{notes}

The corresponding proposition in the Data is (Prop. 59), If a given (area)
Ar applied (i.e. in the form of a parallelogram) tc a given straight lint exteeding
by a fire gii'en in species, the breadths of the excess are givi».

The problem of vi, 29 corresponds of course to the solution of the
quadratic equation

ax -y- - 3? = S,
c

The algebraical solution of this equation gives

The exact correspondence of Euclid's method to the aigebraical solution
may be seen, as in the case of vi. 28, by supposing the parallelograms to be
rectangles. In this case Euclid's construction on EB of the parallelogram
EL similar to D is equivalent to finding that

FE = .-, and EL = .'.
b z * 4

His determination of the similar parailelogram />/jV equal to the sum of £L
and S corresponds to proving that

f d 4

or

whence x is found as

a

Euclid takes, in this case, the solution corresponding to the positive sign
before the radical because, from his point of view, that would be the only
solution.

No iiofiitriiai is necessary because a real geometrical solution is always
possible whatever be the size of S.

Again the £>afa has a proposition indicating the particular case in which
the excess is a rhombus with a given angle. Prop. 84 proves that, 1/ itvo
straight lines contain a givtn area in a ven angle, and one of thf straight lines
is greater than the other by a given straight line, each of the two straight lines is
given also. The proof reduces the proposition to a particular case of Data,
Prop. 59, quoted above aa corresponding to vi. 29.

Again there is an important particular case which can be solved by means
of Book II. only, as shown under 11. 6 above (Vol. 1. pp. 386 — 8), the case namely
in which the excess is a square, corresponding to the solution of the equation

This is the problem of applying to a given straight line a rectangle equal to a
ffven area and exceeding by a square.

\end{notes}

\end{proposition}

\begin{proposition}
\label{prop:VI_30}

\begin{statement}
To cut a given finite straight line in extreme and mean
ratio.
\end{statement}

\begin{proof}

Let AB be the given finite straight line ;

thus it is required to cut AB in extreme and mean ratio.

On AB let the square BC be described ;
and let there be applied to AC the parallelo-
gram CD equal to BC and exceeding by
the figure AD similar to BC. [vi. 29]

Now BC is a square ;
therefore AD is also a square.

And, since BC is equal to CD,

let CE be subtracted from each ;

therefore the remainder BF is equal to
the remainder AD.

88 cf , BOOK VI [vi. 30, 31

But it is also equiangular with it ;
therefore in BF, AD the sides about the equal angles are
reciprocally proportional ; [vi- h]

therefore, as FE is to ED, so is AE to EB,
But FE is equal to AB, and ED to AE,
Therefore, as BA is to AE, so is AE to EB.

And AB is greater than AE ;

therefore AE is also greater than EB.
Therefore the straight line AB has been cut in extreme
and mean ratio at E, and the greater segment of it is AE.

Q.E.F.
\end{proof}

\begin{notes}

It will be observed that the construction in the tent is a direct application
of the preceding Prop. 29 in the particular case where the txass of the
parallel ogram which is applied is a squar*. This fact coupled with the
position of V(, 30 is a sufficient indication that the construction is Euclid's.

In one place Theon appears to have amplified the argument. The text
above says ``But fE is equal to AB'' while the mss. B, F, V and p have
`` But /ff is equal to  C, that is, to .''

The MSS. give after vtp Ihu iroujiriH an alternative construction which
Heibei relegates to the Appendix. Thf text-books give this construction
alone and leave out the other. It will be remembered that the alternative
proof does no more than refer to the equivalent construction in ii. 11.

``Let AB ie. cut at C so that the rectangle AB, BC is equal to the
square on CA. [ii. ( jj

Since then the rectangle AB, BC is equal to the stjuare on CA,
therefore, as BA is to .rfC, so is /4Cto CB. [vi. 17]

Therefore AB has been cut in extreme and mean ratio at tV

It is intrinsically improbable that this alternative construction was added
to the other by Euclid himself. It is however just the kind of interpolation
that might be expected from an editor. If Euclid had preferred the alternative
construction, he would have been more likely to give it alone.

\end{notes}

\end{proposition}

\begin{proposition}
\label{prop:VI_31}

\begin{statement}
In right-angled triangles the figure on Ike side suit ending
the right angle is equal to the similar and similarly described
Jtgures on the sides containing the right angle.
\end{statement}

\begin{proof}

Lt ABC be a right-angled triangle having the angle BAC
right ;

I say that the figure on BC is equal to the similar and
similarly described figures on BA, AC.

Let AD be drawn perpendicular.

Then since, in the right-angled triangle ABC, AD has

Vf. 3']

been drawn from the right angle at A perpendicular to the
base BC,

the triangles ABD, v4/?C adjoin-
ing the perpendicular are similar
both to the whole ABC and to
one another, [vi, 8]

And, since ABC is similar to
ABD,

therefore, as CB is to BA so is
AB to BD. [vi. I>er. 1]

And, since three straight lines
are proportional,

as the first is to the third, so is the figure on the first to the
similar and similarly described figure on the second, [vi. 19, For.]

Ihercfore, as CB is to BD, so is the figure on CB to the
similar and similarly described figure on BA. .-     'l '.

For the same reason also, '

as BC is to CD, so is the figure on BC to that on CA ;
so that, in addition,

as BC is to BD, DC, so is the figure on BC to the similar
and similarly described figures on BA, AC.

But BC is equal to BD, DC ;
therefore the figure on BC is also equal to the similar and
similarly described figures on BA, AC.

Therefore etc.
\end{proof}

\begin{notes}

As we have seen (note on i. 47), this extension of [. 4; is credited by
Proclus to Euclid persorally.

There is one inference in the proof which requires examination. Euchd
proves that

CB : .ff/> = (figure on CB) -. (figure on BA),

and that BC : CZ> = (figure on BC) : (figure on CA),

and then infers directly that

BC : (BD+ CZ))- (fig. on BC) : (sum of figs, on BA and AC).

Apparently v. 24 must be relied on ss justifying this inference. But it is not
directly applicable ; for what it proves is that, if

a:b = c:d, .,-.. A- >.V-.

and   f:b=f:d,  .-'MV

then (a + (t) : b = (f f) : d.

Thus we should itwtrt the first two proportions given above (by Simson's

»jo «t . BOOK VI [vi. 31, 31

Prop. B which, as we have seen, is a direct consequence of the definition of
proportion), and thence infer by v. 34 that

(BD+CD) : J3C= (sum of figs, on £j4, AC) : (fig. on BC).
But BD 4 CD is equal to BC;

therefore (by Simson's Prop. A, which again is an immediate consequence of
the definition of proportion) the sum of the figures on BA, AC is equal to
the figure on BC.

The Mss. agin give an alternative proof which Heibetg places in the
Appendix. It first shows that the simitar figures on the three sides have the
same ratios to one another as the squara on the sides respectively. Whence,
by using I. 47 and the same argument based on v. 24 as that explained above,
the result is obtained.

If it is considered essential to have a proof which does not use Simson's
Props, fi and A or any proposition but those actually given by Euclid, no
method occurs to me except the following.

Eucl.\ V. 12 proves that, if a, J, c are three magnitudes, and d, e, f three
Others, such that

a : b=:d : t,
 ..-:- :.:   bie'-f.f,  ., :

then, ex aeptali, a : c = d :/, , i '

If now in addition a .b=b -.Cf

so that, also, d i e = e :/,

the ratio a -. ( ii duplicate of the ratio a : i, and the ratio d :/ duplicate of
the ratio d : t, whence the ratios which ar< duplicate of equal ratios are equal.
Now (fig. on AC)\ (fig. on AB) = the ratio duplicate of AC : AB

= the ratio duplicate of CD : DA
= CD : BD.
H»ice (sum of figs, on AC, AB) : (fig. on AB) = BC -. BD. [v. iS]
But (fig. on BC) : (fig. on AB) = BC : BD

(as in Euclid's proof).
Therefore the sum of the figures on AC, AB has to the figure on AB the
same ratio as the figure on BC has to the figure on AB, whence

the figures on AC, AB are together equal to the figure on BC- [v. 9]

\end{notes}

\end{proposition}

\begin{proposition}
\label{prop:VI_32}

\begin{statement}
If two triangles having two sides proportional to two sides
be placed together at one angle so that their corresponding sides
are also parallel, the remaining sides of the triangles will be
in a straight line.
\end{statement}

\begin{proof}

Let ABC, DCE be two triangles having the two sides
BA, AC proportional to the two sides DC, DE, so that, as
AB is to A C, so is DC to D£, and AB parallel to DC, and
AC to D£;
I say that BC is in a straight line with C£.

For, since AB is parallel to DC,

and the straight line AC has fallen upon them,

the alternate angles BAC, ACD
are equal to one another, [r. 19]

For the same reason

the angle CDE is also
equal to the angle A CD ;
so that the angle BAC is equal
to the angle CDE.

And, since ABC, DCE are
two triangles having one angle, the angle at A, equal to one
angle, the angle at D,

and the sides about the equal angles proportional,
so that, as BA  ja AC, so is CD to DE,

therefore the triangle ABC is equiangular with the
triangle DCE ; [vi. 6]

therefore the angle ABC is equal to the angle DCE.
But the angle CZ? was also proved equal to the angle
BAC

therefore the whole angle A CE is equal to the two angles
ABC, BAC.

Let the angle ACB be added to each ;

therefore the angles ACE, ACB are equal to the angles BAC,
ACB, CBA.

But the angles BA C, ABC, A CB are equal to two right
angles ; [1. 3a]

therefore the angles ACE, ACB are also equal to two
right angles.

Therefore with a straight line AC, and at the point C on
it, the two straight lines BC, CE not lying on the same side
make the adjacent angles ACE, ACB equal to two right
angles ;

therefore BC is in a straight line with CE. [1. 14]

Therefore etc.
\end{proof}

\begin{notes}

It has often been pointed out (e.g. by Clavm, Lardner and Todhunter)
that the enunciation of this proposition is not precise enough. Suppose that
ABC is a triangle. From C draw CD parallel to BA and of any length.
From D draw D£ parallel to CA and of such length that

CD:DE = BA : AC.
Then the triangles ABC, BCD, which have the angular jxiint C coniRion
literally satisfy Euclid's enunciation ; but by no possi-
bility can CE be iti a straight line with CB if, as
in the case supposed, the angles included by the
corresponding sides are supplementary (unless both are
right angles). Hence the included angles must be
equal, so that the triangles must be similar. That
being so, if they are £0 have nothing more than one
angular point common, and two pairs of corresponding
sides are to he/aralMas distinguished froni one or both being in the same
slraigiU line, the triangles can only be placed so that the corresponding sides
in both are on the same side of the third side of either, and the sides (other
than the third sides) which meet at the common angular point are not corre-
sponding sides.

Todhunter remarks that the proposition seems of no use. Presumably he
did not know that it is used by Euclid himself in xiii. 17. This is so
however, and therefore it was not necessary, as several writers have thought, to
do away with the proposition and Irnd a substitute which should be more useful.

1. De Morgan proposes this theorem : ``If two similar triangles be placed
with their bases parallel, and the equal angles at the bases towards the same
parts, the other sides are parallel, each to each ; or one pair of sides are in
the same straight line and the other pair are parallel,''

2. Ur Lachlan substitutes the somewhat similar theorem, ``If two similar
triangles be placed so that two sides of
the one are parallel to the corresponding q
sides of the other, the third sides are a /
parallel,'' /\ /

But it is to be observed that these qZ.....,.. Z. a
propositions can be proved without /
using Book vi. at all ; they can be /
proved from Book i,, and the triangles = c
may as well be called ``equiangular''
simply. It is true that Book vi. is no more than formally neceary to
Euchd's proposition. He merely uses vt, 6 because his enunciation does not
say that the triangles are similar ; and he only proves them to be similar in
order to conclude that they are equiangular. From this point of view
Mr Taylor's substitute seems the best, viz,

3. ``If two triangles have sides parallel in pairs, the straight lines joining
the corresponding vertices meet in a point,
or are parallel.''

Simson has a theory (unnecessary in
the circumstances) as to the possible
object of VI. 32 as it stands. He points
out that the enunciation of vi, 26 might
be more general so as to cover the case
of similar and similarly situated parallelo-
grams with equal angles not coincident
but vertically opposite. It can then be proved that the diagonals drawn
through the common angular point are In one straight line. If ABCF, CDEG
be similar and simikrl)' situated parallelograms,
so that BCGf DCF are straight lines, and if
the diagonals AC CE be drawn, the triangles
ABC, CDR are similar and are plated exactly
as deseriied in vi, 32, so that AQ CE are in a
straight line. Hence Simson suggests that
there may have been, in addition to the in-
direct demonstration in vi. 26, a direct proof
covering the case just given which may have
used the result of vi, 31. I think however
that the place given to the latter proposition in Book vi. is against this view.

\end{notes}

\end{proposition}

\begin{proposition}
\label{prop:VI_33}

\begin{statement}
In equal circles angles have the same ratio as the circum-
ferences on which they stand, whether they stand at the centres
or at the circumferences.
\end{statement}

\begin{proof}

Let ABC, DEF be equal circles, and let the angles BGC,
EHFhG. angles at their centres G, H, and the angles BAC,
£'/?/ angles at the circumferences ;

I say that, as the circumference BC is to the circumference
£F, so is the angle BGC to the angle EHF, and the angle
C to the angle jp/?/.

For let any number of consecutive circumferences CK,
KL be made equal to the circumference BC,
and any number of consecutive circumferences FM, MN equal
to the circumference EF',
and let GK, GL, HM, HN be joined.

Then, since the circumferences BC, CK, KL are equal
to one another,

the angles BGC, CGK, KGL are also equal to one another ;

[in. 27]

therefore, whatever multiple the circumference BL is of BC,
that multiple also is the angle BGL of the angle BGC.

For the same reason also,
whatever multiple the circumference NE is of EF, that
multiple also is the angle NHE of the angle EHF.

If then the circumference BL is equal to the circumference
A, the angle BGL is also equal to the ugEHN; [in. ij]
if the circumference BL is greater than the circumference
EN, the angle BGL is also greater than the angle EHN ;
and, if less, less.

There being then four magnitudes, two circumferences
BC, EF, and two angles BGC, EHF,

there have been taken, of the circumference BC and the angle
BGC equimultiples, namely the circumference BL and the
angle BGL,

and of the circumference EF and the angle EHF equi-
multiples, namely the circumference EN and the angle EHN.

And it has been proved that,
if the circumference BL is in excess of the circumference EN,
the angle BGL is also in excess of the angle EHN ;
if equal, equal ;
and if less, less.

Therefore, as the circumference BC is to EF, so is the
angle BGC to the angle EHF. [v. Def. 5]

But, as the angle BGC is to the angle EHF, so is the
angle BAC to the angle EDF; for they are doubles respec-
tively.

Therefore also, as the circumference BC is to the circum-
ference EF, so is the angle BGC to the angle EHF, and
the angle BA C to the angle EDF.

Therefore etc,
\end{proof}

\begin{notes}

This proposition as generally given includes a second part relating to sectors
of circles, corresponding to the following words addt to the enunciation :
`` and further the sectors, as constructed at the centres `` (hi it not al to/«« art
[or oiTt] irf)o« rots «rKTpo« ot)hotq'(voi). There is of course a corresponding
addition to the ``definition'' or ``particular statement,'' ``and further the sector
GBOC to the sector HEQF'' These additions are clearly due to Theon, as
may be gathered from his own statenient in his commentary on the SijtijoJ
<runai<i of PtoIemy, `` But that sectors in equal circles are to one another as
the angles on which they stand, has been proved by me in my edition of the
Elements at the end of the sixth book.'' Campanus omits them, and P has them
only in a later hand in the margin or between the lines. Theon's proof scarcely
needs to be given here in full, as it can easily be supplied. From the equality
of the arcs BC, CK he infers [in. 29] the equality of the chords BC, CK,
Hence, the radii being equal, the triangles GBC, GCK are equal in all
respects [i. 8, 4]. Next, since the arcs BC, CK are equal, so are Xhz arcs
BAC, CAK. Therefore the angles at the circumference subtended by the
latter, i.e. the angles in the segments BOC, CPK, are equal [iii. v\ and the
segments are therefore similar [in. Def.. iil and equal [in. 24].

Adding to the equal segments the equal tnangles GBC CCA'' respectively,
we see that

the sectors GBC, GCK are equal.

Thus, in equal circles, sectors standing on equal arcs are equal ; and the rest
of the proof proceeds as in Euclid's proposition.

As regards Euclid's proposition itself, it will be noted that (i), besides
quoting the theorem in hi. 27 that in equal circles angles which stand on
equal arcs are equal, the proof assumes that the angle standing on a greater
arc is greater and that standing on a less arc is less. This is indeed a suffi-
ciently obvious deduction from in. 27.

(2) Any equimultiples whaievtr are taken of the angle BGC and the arc
BC, and any equimultiples whatever of the angle EHF and the arc EF.
(Accordingly the words ``any (quimuUipUs whaievtr'' should have been used in
the step immediately preceding the inference that the angles are proportional
to the arcs, where the text merely states that there have been taken of the
circumference BC and the angle BGC equimultiples BL and BGL.) But, if
any multipk of an angle is regarded as being itself an angle, it follows that the
restriction in 1. Deff, 8, 10, ii, 12 of the term angltio an angle less than two
right angles is implicitly given up ; as De Morgan says, ``the angle breaks
prison,'' Mr Dodgson (Euclid and his Modern Rivals, p, 193) argues that
Euclid conceived of the multiple of an angle as so many separate angles not
added together into one, and that, when it is inferred that, where two such
multiples of an angle are equal, the arcs subtended are also equal, the argu-
ment is that the sum total of the first set of angles is equal to the sum total
of the second set, and hence the second set can be broken up and put
together again in such amounts as to make a set equal, each to each, to the
first set, and then the sum total of the arcs will evidently be equal also. If
on the other hand the multiples of the angles are regarded as single angular
magnitudes, the equality of the subtending arcs is not inferrible directly from
Euclid, because his proof of ni. 26 only applies to cases where the angle is
less than the sum of two right angles, (.'s a matter of fact, it is a question of
inferring equality of angles or multiples of angles from equality of arcs, and
not the converse, so that the reference should have been to in. 27, but this
does not affect the question at issue.) Of course it is against this view of
Mr Dodgson that Euclid speaks throughout of `` the angle BGL `` and `` the
ane EHN `` (ij imo BHA  yiui'i'a, jj ujro E0N yoivia). I think the probable
explanation is that here, as in in. 20, 21, 26 and 27, Euclid deliberately took
no cognisance of the case in which the multiples of the angles in question
would be greater than two right angles. If his attention had been called to
the fact that in. 20 takes no account of the case where the segment is less
than a semicircle, so that the angle in the segment is obtuse, and therefore the
`` angle at the centre `` in that case (if the term were still applicable) would be

greater than two right angles, Euclid would no doubt have refused to regard
the latter as an angle, and would have represented it otherwise, e.g. as the
sum of two angles or e what is left when an angU in the true sense is sub-
tracted from four right angles. Here then, if Euclid had been asked what
course he would take if the multiples of the angles in question should be
greater than two right angles, he would probably have represented them, I
think, as being equal to so many right anghs plus an angU less than a right
angle, or so many limes two right angles plus an angle, acute or obtuse. Then
the equality of the arcs would be the equality of the sums of so many circum-
ferences, semi-circumferences or quadrants plus arcs less than a semicircle or
a quadrant. Hence I agree with Mr Uodgson that vi. 33 affords no evidence
of a recognition by Euclid of `` angles `` greater than two right angles

Theon adds to his theorem about sectors the Porism that. As the sector is
to the sector, so also is the angle to the angle. This corollary was used by
Zenodorus in his tract -n-fpt uro/ifrpcuv axp-ituiv preserved by Theon in his
commentar)' on Ptolemy's oi/n-afu, unless indeed 'i'heon himself interpolated
the words (s S'' tomut tt/m Ttv rofiia, tJ vro E0A yiui'Ja Trp tv viro
M@A),

\end{notes}

\end{proposition}

\part{Book VII}

\chapter*{Definitions}

\begin{enumerate}

\item\label{def:VII_1} An unit is that by virtue of which each of the
  things that exist is called one.

\item\label{def:VII_2} A number is a multitude composed of units.

\item\label{def:VII_3} A number is a part of a number, the less of the
greater, when it measures the greater ;

\item\label{def:VII_4} but parts when it does not measure it.

\item\label{def:VII_5} Tue greater number is a multiple of the less when
it is measured by the less.

\item\label{def:VII_6} An even number is that which is divisible into two
equal parts,

\item\label{def:VII_7} An odd number is that which is not divisible into
two equal parts, or that which differs by an unit from an
even number,   . ...

\item\label{def:VII_8} An even-times even number is that which is
measured by an even number according to an even number.

\item\label{def:VII_9} An even-times odd number is that which is
measured by an even number according to an odd number,

\item\label{def:VII_10} An odd-times odd number is that which is
measured by an odd number according to an odd number.

\item\label{def:VII_11} A prime number is that which is measured by an
unit alone, ..  .

\item\label{def:VII_12} Numbers prime to one another are those which
are measured by an unit alone as a common measure.

\item\label{def:VII_13} A composite number is that which is measured
by some number.

\item\label{def:VII_14} Numbers composite to one another are those
which are measured by some number as a common measure.

\item\label{def:VII_15} A number is said to multiply a number when that
which is multiplied is added to itself as many times as there
are units in the other, and thus some number is produced.

\item\label{def:VII_16} And, when two numbers having multiplied one
another make some number, the number so produced is
called plane, and its sides are the numbers which have
multiplied one another.

\item\label{def:VII_17} And, when three numbers having multiplied one
another make some number, the number so produced is
solid, and its sides are the numbers which have multiplied
one another.

\item\label{def:VII_18} A square number is equal multiplied by equal, or
a number which is contained by two equal numbers.

\item\label{def:VII_19} And a cube is equal multiplied by equal and again
by equal, or a number which is contained by three equal
numbers.

\item\label{def:VII_20} Numbers are proportional when the first is the
same multiple, or the same part, or the same parts, of the
second that the third is of the fourth.

\item\label{def:VII_21} Similar plane and solid numbers are those which
have their sides proportional.

\item\label{def:VII_22} A perfect number is that which is equal to its own
parts.

\end{enumerate}

\section*{Definition 1}

Movas teric, Kaff 4jv inatrrov tav 6vTtov ty Xryerai.

lamblichus (fl. ana 300 a.d.) ceKs us (Comm. on Nieemaehus, ed. Pistelli,
p. 1 1, 5) that the Euclidean definition of an unii or a monad was the definition
given by `` more recent `` writers (ol vta/rtpoi.), and that it lacked the words
``even though it be collective'' (ni-v (twmiiijiTuthi  ). He also gives (ibid.
p. 11) a number of other definitions, (t) According to ``some of the Pytha-
goreans,'' `` an unit is the boundary between number and parts `` (omt brrar
ApSfuiv KQi noftiuai iitSopuir), `` because from it, as from a seed and eternal
root, ratios increase reciprocally on either side,'' i.e. on one side we have
multiple ratios continually increasing and on the other (if the unit be sub-
divided) submultiple ratios with denominators continually increasing. (2) A
somewhat similar definition is that of Thyniaridas, an ancient Pythorean,
who defined a monad as `` limiting quantity `` (ir«patVow<ra iroowij!), the
beginning and the end of a thing being equally an extremity (vipa. ). Perhaps
the words together with their explanation may hest be expressed by `` limit of
fewness.'' Theon of Smyrna (p. 18, 6, ed. Hill'er) adds the explanation that
the monad is `` that which, when the multitude ts diminished by way of
continued subtraction, is deprived of all number and takes an abiding position
(liBv) and rest.'' If, after arriving at an unit in this way, we proceed to divide
the unit itself into parts, we straightway have multitude again. (3) Some, ac-
cording to lamblichus (p. r r, 16), defined it as the ``form of forms'' (dStuK <I8«)
because it potentially comprehends all forms of number, e.g. it is a polygonal
number of any number of sides from three upwards, a solid number in all
forms, and so on, (We are forcibly reminded of the latest theories of number
as a ``Gattung'' of ``Mengen'' or as a ``class of classes.'') (4) Again an
unit, says lamblichus, is the first, or smallest, in the category of how many
(rixrov), the common part or beginning of Aow many. Aristotle defines it as
`` the indivisible in the (category of) quantity,'' to xari to voaov dSuupfrov
(Mdaph. 10S9 b 35), ttottoy including in Aristotle continuous as well as
discrete quantity ; hence it is distinguished from a point by the fact that it
has not position : ``Of the indivisible in the category of, and quA quantity,
that which is every way (indivisible) and destitute of position is called an
unii, and that which is every way indivisible and has position is a point''
(Meiaph. io[6b2S). (5) In accordance with the last distinction, Aristotle
calls the unit `` a point without position,'' ariyiii) afftros (Mtlaph. 1084 b 26),
(6) Lastly, lamblichus says that the school of Chrysippus defined it in a con-
fused manner (<niyKr)(V)Uviiii) as `` multitude one (ttkOm tv),'' whereas it is
alone contrasted with multitude. On a comparison of these definitions, it
would seem that Euclid intended his to be a more popular one than those
of his predecessors, S);/ui! c, as Nicomachus called Euclid's definition of an
iven number.

The etymological signification of the word floras is supposed by Theon of
Smyrna (p. 19, 7 — 13) to be either (i) that it remains unaltered if it be
multiplied by itself any number of times, or (») that it is separated and isolattd
(MjaofiHaat) from the rest of the multitude of numbers. Nicomachus also
observes (1. 8, a) that, while any number is half the sum (i) of the adjacent
numbers on each side, (2) of numbers equidistant on each side, the unit is
moi solitary (/iwdtran;) in that it has not a number on each side but only on
one side, and it is half of the latter done, i.e. of 2.

\section*{Definition 2}

The definition of a numbir is again only one out ol many that are on
record. Nicomachus (i. 7, i) combines several into one, saying that it is
`` a defined multitude (ttXijo? (ijptcr«Vor), or a collection of unit.s (afa£ti>v
o-uo-nita), or a flow of quantity made Up of units `` (ttoitotiitik xfo *'' liovaaiv
avyKtljufvov). Theon, in words almost identical with those attributed by
Stobaeus (Edogae, 1. i, 8) to Moderatus, a Pythagorean, says (p. 18, 3—5):
`` A number is a collection of units, or a progression (jrpcHrotrfio'v) of mul-
titude beginning from an unit and a retrogression (dva7ro8«rfio'«) ceasing at an
unit.'' According to lamblichus (p. 10) the description ``collection of units''
(ovaSujv o-u'crnj/ui) was applied to the how many, i.e. to number, by Thales,
following the Egyptian view (kq™ to .yvimaKov aftioKov), while it was
Eudoxus the Pythagorean who said that a number was ``a defined multitude''
(n-Xijfio! Kifn.tr i.ivav). Aristotle has a number of definitions which come to the
same thing: ``limited multitude'' (irXflos- to TrfTrfpao-tfof, Metapk. lOio a
13), ``multitude'' (or ``combination'') ``of units'' or ``multitude of indivi-
sibles'' (ibid. 1 05 J a 30, 1039 a 12, 1085 b 22), ``several outi'' iva. TrXtim,
Phys. HI. 7, 207 b 7), ``multitude measurable by one'' (Meiaph. 1057 a 3)
and `` multitude measured and multitude of measures,'' the `` measure `` being
unity, TO tv (ibid. 1088 a 5).

\section*{Definition 3}

By a pari Euclid means a submultiple, as he does in v. Def. i, with which
definition this one is identical except for the substitution of number (aptSyaojJ
for magnitude (/tiyfffo) ; cf note on v. Def. 1 , Nicomachus uses the word
``submultiple'' (vwmroXKaTrXairwi) also. He defines it in a way corre.'jponding
to his definition of multiple (see note on Def 5 below) as follows (1, 18, 2):
`` The submultiple, which is by nature first in the division of inequality
(called) less, is the number which, when compared with a greater, can
measure it more times than once so as to fill it exactly (jrAjpoufru).'' Simi-
larly sub-double (iIiroStTrXao-io;) is found in Nicomachus meaning half, and
so on.

\section*{Definition 4}

Mept) S(, orav /i KaTOtTp.

By the expression parts (fiipij, the plural of ftipirt) Euclid denotes what we
should call A proper fraition. That is, a pari being a submultiple, the rather
inconvenient term parts means any numbtrr of such submultiples making up
a fraction less than unity. I have not, found the word used in this special
sense elsewhere, e.g. in Nicomachus, Theon of Srnyrna or lamblichus, except
in one place of Theon (p. 79, 26) where it is used of a proper fraction, of
which 1 is an illustration.

\section*{Definition 5}

The definition of a multiple is identical with that in v. Def, 2, except that
the masculine of the adjectives is used agreeing with npipi's understood
instead of the neuter agreeing with fitytOo understood. Nicomachus (i. 18,
i) defines a multiple as being ``a species of the greater which is naturally-
first in order and origin, being the number which, when considered in com-
parison with another, contains it in itself completely more than once.''

\section*{Definitions 6, 7}

6. ApniK apt6)iot ivTiv 6 Sijfa Siaipovittro<,.

7. Tlfpicro'os oc fLrj oiatfjofi.H'Oi t)a.  [o] ftovdi Siapbrv Aprlov pifLOv.

Nicomachus (1. 7, a) somewhat amplifies these definitions of eve/t and oild
numbers thus, ``That is evtri which is capable of being divided into two
equal parts without an unit falling in the middle, and that is odd which cannot
be divided into two equal parts because of the aforesaid intervention (/i«ri-
niay) of the unit.'' He adds that this definition is derived `` from the popular
conception `` (Ik tij! Sij/iuBout uToAijfon). In contrast to this, he gives (t. 7, 3)
the Pythagorean definition, which is, as usual, interesting. ``An av» number
is that which admits of being divided, by one and the same operation, into the
greatest and the least (parts), greatest in size (infKtKo-njTi) hut least in quantity
(irmroDTTt). ..while an odd number is that which cannot be so treated, but is
divided into two unequal parts.'' That is, as lamblichus says (p. 12, 2—9), an
even number is divided into parts which are the p-e<iUst possible ``parts,'' namely
halves, and into the ftwtst possible, namely two, two being the first `` num-
ber'' or ``collection of units.'' According to another ancient definition quoted
by Nicomachus (i. 7, 4), an even number is that which can be divided both
into two equal parts and into two unequal parts (except the first one, the
number 2, which is only susceptible of division into eiiuals), but, however it
is divided, must have its two parts of the same kind, i.e. both even or both
odd ; while an odd number is that which can only l)e divided into two
unequal parts, and those parts always of differait kinds, i.e. one odd and
one even. Lastly, the definition of odd and even ``by means of each other''
says that an odd number is that which differs by an unit from an even
number on both sides of it, and an even number that which differs by an
unit from an odd number on each side. This alternative definition of an
odd number is the same thing as the second half of Euclid's definition, `` the
number which differs by an unit from an even number.'' This evidently
pre-Euclidean definition is condemned by Aristotle as unscientific, because
odd and even are coordinate, both being differentiae of number, so that one
should not be defined by means of the other (Topics vi. 4, 142 b 7 — 10).

\section*{Definition 8}

Kpnaxi aprio aptVfjLO temv o vtto ofyrlov dptBfAOv fitrpovfitvo Kara apTiOV
iipifi6v.

Euclid's definition of an ei'en-times even number dift'ers from that given by
the later writers, Nicomachus, Theon of Smyrna and lamblichus ; and the
inconvenience of it is shown when we come to ix. 34, where tt is proved
ihat A. certain sort of number is Aa/A ``even-times even ``and ``even-times odd.''
According to the more precise classification of the three other authorities, the
`` even-times even `` and the `` even-times odd `` are mutually exclusive and are
two of three subdivisions into which even numbers fall. Of these three sub-
divisions the ``even-times evt;ii `` and the ``tven-timcs odd'' form the extremes,
and the ``odd-times even'' is as it were intermediate, showing the character
of both extremes (cf, note on the following definition). The even-times emn is
then the number which has its halves even, the halves of the halves even, and
so on, until unity is reached. In short the evtn-times evin number is always
of the form z''. Hence lamblichus (pp. lo, zi) says Euclid's definition of it
as that which is measured by an even number an even number of times is
erroneous. In support of this he quotes the numljer 24 which is four times 6,
or six times 4, but yet is not `` even times even `` according to Euclid himself
(oiSi na/ auToi'), by which he musjl apparently mean that 24 is also 8 times 3,
which does not satisfy Euclid's definition, '['here can however be no doubt that
Euclid meant what he said in his definition as wt have it ; otherwise ix. 32,
which proves that a number of the form 2'' is even-limes even only, would be quite
superfluous and a mere repetition of the definition, while, as already stated,
IX. 34 clearly indicates Euclid's view- that a numlier might at the same time
be both even. times even and even-times odd. Hence the ftdt'ias which some
editor of the commentary of I'hiloponus on Nicomachus found in some
copies, making the definition say that the even-times even number is only
measured by even numbers an even number of times, is evidently an interpo-
lation by some one who wished to reconcile Euclid's definition with the
Pythagorean (cf Heiberg, Eiiklid-siudien, p. a 00).

A consequential characteristic of the series of even-times even numbers
noted by Nicomachus brings in a curious use of the word Suvqi (generally
power in the sense of square, or square root). He says (). 8, 6 — 7) that any
part, i.e. any submultiple, of an even -times even number is called by an even-
times even designation, while it also has an even-times even value (it is
apTtciitts dpTmSurafioi') when expressed as so many actual units. That is, the

-,th part of 2'' (where m is less than «) is called after the even-times even

number z™, while its actual value (Sufa/ut) in units is 2''-''*, which is also an
even-times even number. Thus all the parts, or submultiples, of even-times
even numbers, as well as the even-times even numbers themselves, are con-
nected with one kind of number only, the even.

\section*{Definition 9}

'ApTiaKit S( TTtpunrd; hrtw h viro iptiov ifuBoxt fLcrpovjucvos Kara npurvvi'

Euclid uses the term even- times odd (dpTniic« vipvaao), whereas Nicomachus
and the others make it one word, even-odd (aprtairifiirro). According to the
stricter definition given by the latter (i, 9, i), the even-odd number is related to
the even-limes even as the other extreme. It is such a number as, when once
halved, leaves as quotient an odd number; that is, it is of the form j(2«+ i).
Nicomachus sets the even-odd numbers out as follows,

6, 10, 14, 18, zz, 26, 30, etc.
In this case, as Nicomachus observes, any part, or submultipie, is called by a
name not corresponding in kind to its actual value (Sura/iw) in units. Thus,
in the case of i3, the  part is calied a.rter the even number 2, but its va/ue is
the odd number 9, and the Jrd part is called after the odd number 3, while its
value is the even number 6, and so on.

The third class of even numbers according to the strict subdivision is the
odd-even (TrtpunrdpTioi). Numbers are of this class when they can be halved
twice or more times successively, but the quotient left when they can no
longer be halved is an odd number and not unity. They are therefore of
the form 2''(7« + 1), where », nt are integers. They are, so to say, inter-
mediate between, or a mixture of, the extreme classes eiien-times tvin and even-
odd, for the following reasons, (r) Their subdivision by 3 proceeds for some
way like that of the even-times even, but ends in the way that the division of
the even-odd by i ends. (2) The numbers after which submultiples are
called and their value (Sv'raj««) in units may be both of one kind, i.e. both odd
or both even (as in the case of the even-times even), or again may be one odd
and one even as in the case of the even-odd. For example »4 is an odd-even
number; the  ih, tV''i ir'h or  parts of it are even, but the Jrd part of it,
or 8, is even, and the Jth part of it, or 3, is odd. (3) ``Nicomachus shows
(i. 10, 6 — 9) how to form all the numbers of the odd-even class. Set out two
lines (a) of odd numbers beginning with 3, (fi) of even-times even numbers
beginning with 4, thus :

(a) 3. S> 7i 9i r»> »3. >S etc.

(b) 4, 8, 16, 3*, 64, 128, 156 etc.

Now multiply each of the first numbers into each df the second row. Let
the products of one of the first into all the second set make horizontal rows ;
we then get the rows

12,24, 48, 96,192, 384, 768 etc.

20, 40, 80, 160, 320, 640, 1280 etc,

28, 56, 1 1 J, i24, 448, 896, 179J etc. ,-

36, 72, 144, 288, 576, 1152, 2304 etc,
and so on.

Now, says Nicomachus, you will be surprised to see ((Tfcrtrat trm ftiupMr-
rwi) that (a) the vertical rows have the property of the ezfen-odd series, 6, 10,
14, 18, 22 etc., viz, that, if an odd number of successive numbers be taken,
the middle number is half the sum of the extremes, and if an even number,
the two middle numbers together are equal to the sum of the extremes,
(t) the horizontal rows have the property of the even-times even series 4, 8, 16
etc., viz. that the product of the extremes of any number of successive terms
is equal, if their number be odd, to the square of the middle term, or, if their
number be even, to the product of the two middle terms.

Let us now return to Euclid. His 9th definition states that an even-timef
odd number is a number which, when divided by an even number, gives an
odd number as quotient. Following this definition in our text comes a loth
definition which defines an odd-times even number ; this is stated to be a
number which, when divided by an odd number, gives an even number as
quotient. According to these definitions any even-times odd number would
also be odd-times even, anti, from the fact that lamblichus notes this, we may
fairly conclude that he found Def, 10 as well as Def. 9 in the text of Euclid
which he used. But, if both definitions are genuine, the erjunciations of ix. 33
and IX. 34 as we have them present difficulties, ix. 33 says that `` If a num-
ber have its half odd, it is even-times odd only `` ; but, on the assumption that
both definitions are genuine, Ihis would not be true, for the number would be
odd-timts even as well. ix. 34 says that `` If a number neither be one of those
which are continually doubled from 2, nor have its half odd, it is both even-
times even and even- times odd.'' The term odd-timts even (irtpuraaKK aprtot)
not occurring in these propositions, nor anywhere else after the definition, that
definition liecomes superfluous. Iambi ichus however (p. 24, 7 — m) quotes
these enunciations differently. In the first he has instead of `` even-times odd
only `` the words `` both tven-timei odd and odd-times even `` ; and, in the second,
for `` both even-times even and even-times odd `` he has `` is both even-times
even and at the same time even-times odd and odd-times even.'' In both
cases therefore `` odd-times even `` is added to the enunciation as lamblichus
had it the words catrnot have been added by lamblichus himself because
he himself does not use the term odd-timts even, but the one word odd-even
(rtfiuTirafyno). In Order to get over the difficulties involved by Def, 10 and
these differences of reading we have practically to choose between (i) accept-
ing lamblichus' reading in all three places and (2) adhering to the reading of
our Mss. in ix. 33, 34 and rejecting Def. 10 altogether as an interpolation.
Now the readings of our text of ix. 33, 34 are those of the Vatican MS.
and the Theonine mss. as well ; hence they must go back to a time before
Theon, and must therefore be almost as old as those of lamblichus.
Heiberg considers it improbable that Euclid would wish to maintain a point-
less distinction between even-times odd and odd-times even, and on the whole
concludes that IJef. [O was first interpolated by some ignorant person who
did not notice the difference between the Euclidean and Pythagorean clissi-
fication, but merely noticed the absence of a definition of odd-times even
and fabricated one as a companion to the other. When this was done, it
would be easy to see that the statement in )X. 33 that the number referred
to is `` even-times odd only `` was not strictly true, and that the addition of
the words ``and odd-times even'' was necessary in ix. 33 and tx. 34 as
well.

\section*{Definition 10}

n*p«r<raKW Be letpuruoi apSjtan iarw viro irtpurami dpiff/iov fierpois/it™?
Kara Trtpt(r<rov dpiOfxov*

The Olid-times add number is not defined as such by Nicomachus and
lamblichus ; for them these numbers would apparently belong to the «»«-
osite subdivision of odd numbers. Theon of Smyrna on the other hand
says (p. 23, 21) that odd-times odd was one of the names applied to prime
numbers (excluding 2), for these have two odd factors, namely i and the
number itself. This is certainly a curious use of the term.

\section*{Definition 11}

IXpitfTiK d,pSiia  ttTTiv b /tovaSi /lovij ftfrpttijitvov.

A prijne number (vfiS/ro iptSjtios) is called by Nicomachus, Theon, and
lamblichus a `` prime irnrf inwm/oj(V (iovktrtre) number.'' Theon (p. 23, 9)
defines it practically as Euclid does, viz. as a number ``measured by no number,
but by an unit only.'' Aristotle too says that a prime number is not measured by
any number (Ana/, post. ti. 13, 96 a 36), an unit not being a number [Metaph,
1088 a 5), but only the beginning of number (Theon of Smyrna says the same
thii, p. t, 13). According to Nicomachus (1. 11, a) the prime number is a
subdivision, not of numbers, but of odd numbers; it is ``an odd number
which admits of no other part except that which is called after its own name
(n-apiiJniftov <avri2).'' The prime numbers art; 3, 5, 7 etc., and (here is no
submuUiple of 3 except rd, no subtnultipiL' of 1 1 except y j th, and so on. hi
all these cases the only submultiple is an unit. According to Nicomachus 3
is the first prime number, whereas Aristotle (Topics viii. 2, 157 a 39) regards
a as a prime number ; ``as the dyad is the only even number which is prime,''
showing that this divergence from the lythagortan doctrine was earlier than
Euclid. The number 2 also satisfies Euchd's definition of a prime number,
lamblichus (p. 30, 27 sqq.) makes this the ground of another attack upon Euclid.
His argument (the text of which, however, leavijs much to be desired) appears
to be that i is the only even number which has no other part except an
unit, while the subdivisions of the even, as previously explained by him (the
ti-en-timts even, the even-odd, and odd-even), all exclude primeness, and he has
previously explained that 2 is pottniially evenxld, being obtained by
multiplying by 2 i\ potentiaiiy odd, i.e. the unit; hence 2 is regarded by him
as bound up with the subdivisions of even, which exclude primeness. 'I'heon
seems to hold the same view as regards i, but supports it by an ap|)arent
circle. A prime number, he says (p. 23, 14 — 23), is also called odd-times odd;
therefore only odd numbers are prime and in composite. Even numbers are
not measured by the unit alone, except i, wliich therefore (p. 24, 7) is odd-Aiti'
(itipariratirfi) without being prime.

A variety of other names were applied to prime numbers. We have
already noted the curious designation of them as add-titnts odd. According to
lamblichus (p. 27, 3 — 5) some called them evthymtiric ((uSufTpwds), and
Thymaridas rectilinear (<iSv7pa/tfuitOT), the ground being that they can only be
set out in one dimension with no breadth (iirXars yap iv i-g Mtati iiji' iv
liovot iiuTTafitun). The same aspect of a prime number is also expressed by
Aristotle, who (Metaph. 1020 b 3) contrasts the composite number with that
which is only in one dimension (fioi'oi' i< tv cuv). Theon of Smyrna (p. 23, 1 2)
gives  )>pa/iutdt (linear) as the alternative name instead of cvuypooidt. In
either ease, to make the word a proper description of a prime nun>ber we have
to understand the word only ; a prime number is that which is linear, or
rectilinear, only. For Nicomachus, who uses the form linear, expressly says
(11. 13, 6) that all numbers are so, i.e. all can be represented as linear by dots
to the required amount placed in a line.

A prime number was called prime or first, according to Nicomachus
(1. II, 3), because it can only be arrived at by putting together a certain
number of units, and the unit is the beginning of number (cf. Aristotle's
second sense of irpiuTos ``as not being composed of numbers'' wi /jltj o-uymurSat
i( dpiSfiMv, Anal. Post. . 13, 96 a 37), and also, according to lamblichus,
because there is no number before it, being a collection of units (ofaSur
<riim]yut), of which it is a multiple, and it appears firsl as a basis for other
numbers to be multiples of.

\section*{Definition 12}

EtpioT'Ol Jrpot aAijXou! apSpai vaw oi /iom judv]) fttTpoi,! -  jt koh-iJ liirpi.

By way of further emphasising the distinction between ''prime'' and
``prime to one another,'' Theon of Smyrna (p. 23, 6—8) calls the former
`` prime aholtttefy `` (dwKm), and the latter `` prime to one another and not
absolttfdy'' or *noi in themseives'' (oi xaff aarmi). The latter (p. 44, 3 — lo)
are `` measured by the unit [sc. only] as common measure, even though, taken
by themselves (w jrpo? Javrm), they be measured by some other numbere.''
From Theon's illustrations it is clear that with him as with Euclid
a. number prime to another may be even as well as odd. In Nicomachus
(i. 1 1, i) and lambllchus (p. id, 19), on the other hand, the number which is
`` in itself secondary (ScvreMt) and composite (tr!ni6tTo ), but in relation to
another prime and incomposite,'' is a subdivision of odd. I shall call more
particular attention to this difference of classification when we have reached
the definitions of `` composite `` and `` composite to one another `` ; for the
present it is to be noted that Nicomachus (1. 13, i) defines a number prime to
another after the same manner as the absolutely prime ; it is a number which
`` is measured not only by the unit as the common measure but also by some
other measure, and for this reason can also admit of a part or parts called by
a difTerent name besides that called hy the same name (as itself), but, when
examined in comparison with another number of similar character, is found
not to be capable of being measured by a common measure in relation to the
other, nor to have the same part, called by the same name as (any of) those
simply (airc«) contained in the other; e.g. 9 in relation to 25, for each of
these is in itself secondary and composite, but, in comparison with one
another, they have an unit alone as a common measure and no part is called
by the same name in both, but the third in one is not in the other, nor is the
fifth in the other found in the first.''

\section{Definition 13}

SwrftTOt optC/«k <rr(v o afiSfu tiki /iMrpoufitvot.

Euclid's definition of compositt is again the same as Theon's definition
of numbers ``composite in relation to themselves,'' which (p. 24, 16) are
`` numbers measured by any less number,'' the unit being, as usual, not
regarded as a number. Theon proceeds to say that `` of composite numbers
they call those which are contained by two numbers plane, as being
investigated in two dimensions and, as it were, contained by a length and a
breadth, while (they call) those (which are contained) by three (numbers)
iolid, as having the third dimension added to them,'' To a similar effect is
the remark of Aristotle (Mtlaph. loio b 3) that certain numbers are
`` composite and are not only in one dimension but such as the plane and the
solid (figure) are representations of (iiijjiia), these numbers being so many
times so many (irocraKK iroa-oi), or so many times so many times so many
(iroo-am! wtwaitt* voaai) respectively.'' These subdivisions of composite
numbers are, of course, the subject of Euclid's definitions 17, 18 respectively.
Euclid's composite numbers may be either even or odd, like those of Theon,
who gives 6 as an instance, 6 being measured by both » and 3.

\section*{Definition 14}

liirpit.

Theon (p. 44, 18), like Euclid, defines numbers eomposite to one another as
*' those which are measured by any common measure whatever `` (excluding
unity, as usual). Theon instances 8 and ti, with i as common measure, and
6 and 9, with 3 as common measure.

As hinted above, there is a great difference between Euclid's classification
of prime and composite numbers, and of numbers prime and comp>osite
to one another, and the classification found in Nicomachus (1. 11 — 13) and
lamblichus. According to the latter, all these kinds of numbers are sub-
divisions of the class of odd numbers only. As the class of even numbers is
divided into three kinds, (i) the even-times even, (2) the even-odd, which
form the extremes, and (3) the odd-even, which is, as it were, intermediate to
the other two, so the class of odd numbers is divided into thtee, of which the
third is again a mean between two extremes. The three are :

(i) tt primt and ineomposite, which is like Euclid's prime number except
that it excludes 2 ;

(i) the ieconiary and composite, which is ``odd because it is a distinct
part of one and the same genus (Sia tu ii jvot kcu tov avroii yirous Stciit<Kpur0(u)
but has in it nothing of the nature of a first principle (ap;(«iSft) ; for it arises
from adding some other number (to itself), so that, besides having a part
called by the same name as itself, it possesses a part or parts called by another
name.'' Nicomachus cites 9, 15, 21, 25, 27, 33, 35, 39. It is made clear that
not only must the factors be both odd, but they must all be prime numbers.
This is obviously a very inconvenient restriction of the use of the word
composite, a word of general signification.

(3) is that which is ``secondary and composite in itself but prime and
ineomposite to another'' The actual words in which this is defined have been
given above in the note on Def. 12. Here again all the factors must be odd
and prime.

Besides the inconvenience of restricting the term composite to odd numbers
which are composite, there is in this classification the further serious defect,
pointed out by Nesselmann (Die Algebra der Griechen, 1842, p. 194), that
subdivisions (2) and (3) overlap, subdivision (2) including the whole of
subdivision (3). The origin of this confusion is no doubt to be found in
Nicomachus' perverse anxiety to be symmetrical ; by hook or by crtxik he
must divide odd numbers into three kinds as he had divided the even.
lamblichus (p. 28, 13) carries his desire to be Iccal so far as to point out
why there cannot be a fourth kind of number contrary in character to (3),
ramely a number which should be ``prime and ineomposite in itself, but
secondary and composite to another `` !

Definition 15.

'ApiB/iAi dpSfthv iroXAaTrAxuruLcif Xiynat, oraf, 3crot »uriv h earr )tovK,
T«ravrax(! inivTi o iraXXaiTAairiafDci'Ov, ««! yiyijrrai tk.

This is the well known primary definition of multiplication as an
abbreviation of addition.

\section*{Definition 16}

iviirtoo KaktiTat irAcvpat Bi airroZ ol vokairaeTaurayT€ dXXijKom dpiBfioL

The words plane and solid applied to numbers are of course adapted from
Iheir use with reference to geometrical figures. A number is therefore called
linear (ypaiifuKVi) when it is regarded as in one dimension, as being a lengtli
(lAiJKiys), When it takes another dimension in addition, namely breadth
(TrAaros), it is in two dimensions and becomes plane (in-iVtScn). The
distinction bet wet! n a plane and a plane number is marked by the use of the
neuter in the former case, and the masculine, agreeing with aptS/io!, in the
latter case. So witli a square and a square number, and so on. Tlie most
obvious form of a plane number is clearly that corresponding to a rectangle in
geometry ; the number is the product of two linear numbers regarded as sides
(TrKfvpai) forming the length and breadth respectively. Such a number is, as
Aristotle says, ``so many times so many,'' and a plane is its counterpart
(/ii/iTj/ia). So I'lato, in the Thcaeieiiis (147 E — 148 b), says : ``We divided all
numbers into two kinds, ( i ) that which can be expressed as equal multiplied
by equal (tov ivvi.i.(vav Xaov Xaixtx yiyno-Sai), and which, likening its form to
the square, we called square and equilateral ; («) that which is intermediate,
and includes 3 and 5 and every number which cannot be expressed as equal
multiplied by e<|uat, but is cither less times more or more times less, being
always ``contained by a greater and a less side, which number we likened to
the oblong figure (ttpo/ijjkh axnt'``) and calleo an obhng number.... Such
lints therefore as square tlie equilateral and plane nujnber fi.e, which can
form a plane number with equal sides, or a square] we defined as length
iltijxo'i) ; but such as square the oblong (here fTfpo/iijKJ)) [i.e, the square of
which is equal to the oblong] we called roots (vulitai) as not being com-
mensurable with the others in length, but only in the plane areas (irwiBow),
to which the squares on them are equal (a Suyavrai).'' This passage seems
to make it clear that Plato would have represented numbers as Euclid does,
by straight lines proportional in length to the numbers they represent (so far
as practicable) ; for, since 3 and S are with Plato oblong numbers, and lines
with him represent the sides of oblong numbers (since a line represents the
`` root,'' the square on which is equal to the oblong), it follows that the unit
representing the smaller side must have been represented as a line, and 3, the
larger side, as a line of three times the length. But there is another possible way
of representing numbers, not by lines of a certain length, but hypain/s disposed
in various ways, in straight lines or otherwise. lamblichus tells us (p. 56, 27)
that `` in old days they represented the ijuantuplicities of number in a more
natural way (ucriKuir (;»>') by splitting them up into units, and not, as in our
day, by symbols'' (o-«ftj3oXuiiu?). Aristotle too (Metaph. 1092 b 10) mentions
one Eurytus as having settled what number belonged to what, such a number
to a man, such a number to a horse, and so on, ``copying their shapes''
(reading rovruii', with Zeller) with pebbles (rms mi), just as those da who
arrange numbers in the forms of triangles or squares.'' We accordingly find
numbers represented in Nicomachus and Theon of Smyrna by a number of
a's ranged like points according to geometrical figures. According to this
system, any number could be represented by points in a straight line, in which
case, says lamblichus (p. 56, 26), we shall call it rectilinear because it is
without breadth and only advances in length (oirXaTttfi inl isavav to 7k«
vpotunv). The prime number was called by Thymaridas rectilinear par
excellence, because it was without breadth and in one dimension only (iift fc
vov SiuTTOfitvin). By this must hi meant the impossibility of representing,
say, 3 as a plane number, in Plato's sense, i.e. as a product of two numbers
corresponding to a rectangle in geometry ; and this view would appear to rest
simply upon the representation of a number by points, as (iistinct from lines.
Three dots in a straight line would have no breadth ; and if breadth were
introduced in the sense of producing a rectangle, i.e. by placing the same
number of dots in a second line below the first line, the first f/ane number
would be 4, and 3 would not be a plane number at ali, as Plato says it is. It
seems therefore to have been the alternative representation of a number by
points, and not lines, which gave rise to the different view of a plane number
which we find tn Nicomachus and the rest. By means of separate points we
can represent numbers in geometrical forms other than rectangles and squares.
One dot with two others symmetrically arranged below it shows a triangle,
which is a figure in two dimensions as much as a rectangle or parallelogram is.
Similarly we can arrange certain numbers in the form of regular ptnlagons or
other polygons. According therefore to this mode of representation, 3 is the
first plane number, being a triangular number. The method of formation of
triangular, square, pentagonal and other polygonal numbers is minutely
described in Nicomachus (11. 8 — 11), who distinguishes the separate series of
gnomons belonging to each, i.e. gives the law determining the number which
has to be added to a polygonal number with n in a side, in order to make it
into a number of the same form but with n + i in a side (the addend being of
course the gnomon). Thus the gnomon ic series for triangular numbers is
'> 2i 3) At 5''- 't that for squares i, 3, 5, 7... ; that for pentagonal numbers
I, 4, 7, 10,,., and so on. The subject need not detain us longer here, as we
ate at present only concerned with the different views of what constitutes a
plane number.

Of plane numbers in the Platonic and Euclidean sense we have seen that
Plato recognises two kinds, the square and the oblong (vpoii-icf)<s or htpoinjieifs).
Here again Euclid's successors, at all events, subdivided the class more
elaborately. Nicomachus, Theon of Smyrna, and lamblichus divide plane
numbers with unequal sides into (i) Irtpofi-iJKtK, the nearest thing to squares,
viz. numbers in which the greater side exceeds the less side by i only, or
numbers of the form n(n+ i), e.g. i . J, a - 3, 3 . 4, etc. (according to Nico-
machus), and (2) wpo(tTKfi<s, or those whose sides differ by z or more, i.e. are of
the form n(n + m), where m is not less than z (Nicomachus illustrates by 2 . 4,
3 . 6, etc.). Theon of Smyrna (p. 30, 8 — 14) makes wpofujuii! include Irf/jo/iijMtt,
saying that their sides may differ by i or more; he also speaks of parallelogram-
numbers as those which have one side different from the other by 3 or more ;
I do not find this latter term in Nicomachus or lamblichus, and indeed it
seems sufterfluous, as parallelogram is here only another name for oblong,
lamblichus (p. 74, z 3 sqq,), always critical of Euclid, attacks him again here
for confusing the subject by supposing that the htpofiinp number is the pro-
duct of any two different numbers multiplied together, and by not distinguishing
the oblong (irpoifiti;;) from it : `` for his definition declares the same number
to be square and also htpoitxtft, as for example 36, 16 and many others :
which would be equivalent to the odd number being the same thing as the
even.'' No importance need be attached to this exaggerated statement ; it is
in any case merely a matter of words, and it is curious that Euclid does not in
feet use the word fr«po/iijir7t of numiers at all, but only of geometrical oblong
figures as opposed to squares, so that lamblichus can apparendy only have
inferred that he used it in an unorthodox manner from the geometrical use of
the term in the definitions of Book i. and from (he fact that he does not give
the two subdivisions of plane numbers which are not square, but seems only
to divide plane numiers into square and not-square. The aigument that
IrcpoijKttf numbers are a natural and therefore essential, subdivision
lamblichus appears to fotmd on the method of successive addition by which
they can be evolved ; as square numbers are obtained by successively adding
odd numbers as gnomons, so ir<pci/ij)Kf« are obtained by adding even numbers
as gnomons. Thus i.z = 2, 2.3 = 2 + 4, 3,4–2 + 4 + 6, and so on.

\section*{Definition 17}

(TTfpm i(mv, vtvpal Sc airoO oi jroAAairXacricio-aiTf! dAAiAou; apSfioi.

What has been said of the two apparently different ways of regarding a
plane number seerns to apply equally, mutatis mutandis, to the definitions of a
solid number. Aristotle regards it as a number which is so many times so
many times so many (jrpcrojtu irt«rai«! itoaoi). Plato finishes the passage about
lines which represent the sides of square numbers and lines which are roots
ivva/itK), i.e. the squares on which are equal to the rectangle representing a
number which is oblong and not square, by adding the words, `` And another
similar property belongs to solids `` (kqI ircpi to o-rtpta aAAo toioZtov). That is,
apparently, there would be a corresponding term to root (ftJca/in) — practically
representing a surd— to denote the side of a cube equal to a parallelepiped
representing a solid number which is the product of three factors but
not a cube. Such is a solid number when numbers are represented by
straight lines : it corresponds in general to a parallelepiped and, when all
the factors are equal, to a cube.

But again, if numbers be represented by points, we may have solid numbers
(i.e. numbers in three dimensions) in the form of pyramids as well. The first
number of this kind is 4, since we may have three points fonTiing an
equilateral triangle in one plane and a fourth point placed in another plane.
The length of the sides can be increased by i successively ; and we can have
a series of pyramidal numbers, with triangles, squares or polygons as bases,
made up of layers of triangles, squares or similar polygons respectively, each
of which layers has one less in the side than the layer below it, until the top
of the pyramid is reached, which of course, is one point representing unity,
Nicomachus (11, 13 — 16), Theon of Smyrna (p. 41 — 2), and lamblichus
(P- 9S> '5 sqq.), all give the different kinds c>( pyramidal solid numbers in
addition to the other kinds.

These three writers make the following further distinctions between solid
numbers which are the product of three factors.

1. First there is the equal by equal by equal (Icrant to-aittt urof), which is,
of course, the cube.

2. The other extreme is the unequal by unequal by unequal (aVto-ciKis
o.vuia.Ki.  ai'io-ot), or that in which all the dimensions are different, e.g. the
product of 2, 3, 4 or 2, 4, 8 or 3, 5, la. These were, according to Nicomachus
(11, 16), called scalene, while some called them atjvlanoi (wedge-shaped), others
tri/KUTKot (from trij'f, a wasp), and others mitlantot (altar-shaped). Theon
appears to use the last term only, while lamblichus of course gives all three
names.

3. Intermediate to these, as it were, come the numbers `` whose planes
form frepo/nJ«(i numbers'' (i.e. numbers of the form*«(« + i)). These, says
Nicomachus, are QaXXei parallelepipedal.

Lastly come two classes of such numbers each of which has two equal
dimensions but not more.

4. If the third dimension is less than the others, the number b efual ly
tquai iy less (uraKtt itroi i'Aairoi'aicw) and is called a plinth (nrXicSw), e.g.
8.8.3.

5- If the third dinnensioti is greatei than the others, the number is equal
by equal by greater (ktb'ms ujik fniioraitw) and is called a beam (okk), e.g,
3.3.7. Another name for this latter kind of number (according to
lambltchus) was ernjXU (diminutive of onjXij).

Lastly, in connexion with pyramidal numbers, Nifcomachus (11. 14, 5) dis-
tinguishes numbers corresponding Xi frusta of pyramids. These are truncated
(miupoi), twUe-truntated (BtKoXoupot), thrict-fruncated (rpiKoAoupm) pyramids,
and so on, the term being used mostly in theoretic treatises («v mr/ypdiiftairi
fiakuTTa To« StiofnifumKoU). The truncated pyramid was formed by cutting
off the point forming the vertex. The twice-truncaied was that which lacked
the vertex and the next plane, and so on. Theon of Smyrna (p. 42, 4) only
mentions the truncated pyramid as `` that with its vertex cut off'' (ij ttjc
Kofsiiijv diroTiTixsjiLivrj), saying that some also called it a trapezium, after the
similitude of a plane trapezium formed by cutting the top off a triangle
by a straight line parallel to the base.

\section*{Definition 18}

Tcrpaywvo; .pSlx6  i<rry h (cranf uro(  [6j vtro Svo wrtav piftiZv ntpi-

A particular kind of square distinguished by Nicomachus and the rest was
the square number which ended (in the decimal notattor) with the same
number as its side, e.g. i, 25, 36, which are the squares of r, 5 and 6. These
square numbers were called cyclic (kukXmw) on the analogy of circles in
geometry which return again to the point from which they started.

\section*{Definition 19}

KiJjSps Si 6 ttraKit ta-ov liraicK  [i] inro rpiuv t(Fii>r Jpifl/uuv wipiixontum.

Similarly cube numbers which ended with the same number as their sides,
and the squares of thosr sides aisa, were called spherical (o-ftpiKof) or reatrrtnt
(a.iraiaaassta.fma). One might have expected that the term spherical would be
applicable also to the cubes of numbers which ended with the same digit as the
side but not necessarily with the same digit as the square of the side also.
E.g, the cube of 4, i.e. 64, ends with the same digit as 4, but not with the
same digit as 1 6, But apparently 64 was not called a spherical number, the
only instances given by Nicomachus and the rest being those cubed from
numbers ending with 5 or 6, which end with the same digit if squared. A
spherical number is in fact derived from a circular number only, and that by
adding another equal dimension. Obviously, as Nesselmann says, the names
cyclic and spherical applied to numbers appeal to an entirely different principle
from that on which the figured numbers so far dealt with were formed.

\section*{Definition 20}

£(7ttic( ]j iroXAairXdmof i to awo fiipo  ra aura ftc ONrtv,

Euclid does not give in this Book any definition of latio, doubtless because
it could only be the same as that given at the beginning of Book v., with
numbers substituted for ``homogeneous magnitudes `` and ``in respect of size''
(njXiKonp-a) omitted or altered. We do not find that Nicomachus and the
rest give any substantially diflerent definition of a ra/io between numbers.
Theon of Smyrna says, in fact (p, 73, 16), that `` ratio in the sense of
proportion (K6yoi 4 itar' ara'Xoyok) is a sort of relation of two homogeneous
terms to one another, as for example, double, triple.'' Similarly Nicomachus
says (11. II, 3) that ``a ratio is a relation of two terms to one another,'' the word
for `` relation `` being in both cases the same as Euclid's (cr)(i<nt). Theon of
Smyrna goes on to classify ratios as greater, less, or equal, i.e. as ratios of greater
inequality, less inequality, or equality, and then to specify certain arithmetical
ratios which had special names, for which he quotes the authority of Adrastus.
The names were iroXXaTrXoo'tof, irtfioptoif ivifitp, iroAAairAofrtcntopiof)
iroXXariiurt*rtfAM(«j<! (the first of which is, of course, a multiple, while the rest
are the equivalent of certain types of improper fractions as we should call
them), and the reciprocals of each of these described by prefixing vini or fui.
After describing these particular classes of arithmetical ratios, Theon goes on
to say that numbers still have ratios to one another even if they are different
from all those previously described. We need not therefore concern ourselves
with the various types ; it is sufficient to observe that any ratio between
numbers can be expressed in the manner indicated in Euclid's definition of
arithmetical proportion, for the greater is, in relation to the less, either one or
a combination of more than one of the three things, (i) a multiple, (3) a
submultiple, (3) a proper fraction.

It is when we come to the definition of proportion that we begin to find
differences between Euclid, Nicomach us, Theon and lamblichus, `` Proportion,''
says Theon (p. 81, 6), `` is similarity or sameness of more ratios than one,''
which is of course unobjectionable if it is previously understood what a ratio
is ; but confusion was brought in by those (like I'hiasyllus) who said that
there were tkret proporiitms (aVoXoiai), the arithmetic, geometric, and
harmonic, where of course the reference is to arithmetic, geometric and
harmonic means (litcronfut). Hence it was necessary to explain, as Adrastus
did (Theon, p. 106, 15), that of the several mtatts ``the geometric was called
both proportion /ar extelience and primary... though the other means were
also commonly called proportions by some writers.'' Accordingly we have
Nicomachus trying to extend the term `` proportion `` to cover the various
meam as well as a proportion in three or four terms in the ordinary sense. He
says (it. »i, 3): `` Proportion, /ar «ciy/An« (kv/jhus), is the bringing together
(<ni>VXt)i;) to the same (point) of two or more ratios \ or, more generally, (the
bringing tcether) of two or more relations (trifn, even though they be
subjected not to the same ratio but to a difference or some other (law).''
lamblichus keeps the senses of the word more distinct. He says, like Theon,
that `` proportion is similarity or sameness of several ratios `` (p. 98, 14), and
that `` it is to be premised that it was the geometrical (proportion) which the
ancients called proportion par excelknct though it is now common to apply
the name genemlly to all the remaining means as well `` (p. 100, 15). Pappus
remarks (in. p. 70, 17), ``A mean differs from a proportion in this respect tha if
anything is a proportion, it is also a mean, but not conversely. For there are
three means, of which one is arithmetic, one geometric and one harmonic.''
The last remark implies plainly enough that there is only one proportion
(d™Aoy«i) in the proper sense. So, too, says lamblichus in another place
(p. 104, 19): ``the second, the geometric, mean has been called proportion
par excclletue because the terms contain the same ratio, being separated
according to the same proportion (aVcl tov airov Xoyov Swaron-t?).'' The
natural conclusion is that of Nesselmann, that originally the geometric
proportion was called ivakoyia, the others, the arithmetic, the harmonic, etc,
tneans ; but later usage had obliterated the distinction.

Of proportions in the ancient and Euclidean sense Theon fp, 82, 10)
distinguished the continuous (trui-tjf!) and the separated (Zijjpijfi.iv'ri), using the
same terms as Aristotle (Eth. It'c. 1131 a 32). The meaning is of course
clear : in the continuous proportion the consequent of one ratio is the ante-
cedent of the next ; in the separated proportion this is not so. Nicomachus
(11. 21, 5 — 6) uses the words (onnected (<rwrijLit.ivi)) and disjoined (Snitvyfiivi))
respectively. Euclid rularly speaks of numbers in continuous proportion as
`` proportional in order, or successively `` (ifij« liraXoyoi').

\section*{Definition 21}

'QfioiOi lirtBoi KoX <rr€pto\ afiiftOL ihnv al drvdXoyov corrc; ras rXcupa.

Theon of Smyrna remarks (p, 36, 12) that, among plane numbers, at/
squares are similar, while of lTtpofi-iif!.<! those are similar `` whose sides, that
is, the numbers containing them, are proportional.'' Here irfpofirfKiji must
evidently be used, not in the sense of a number of the form n(n + i), but, as
synonymous with irpofHjieij;, any oblong number ; so that on this occasion
Theon follows the terminology of Plato and (according to lamblichus) of
Euclid. Obviously, if the strict sense of rrfpo/ufitiji is adhered to, no two
numbers of that form can be similar unless they are also cuat. We may
compare lamblichus' elaborate contrast of the square and the irfpo/tijuj.
Since the two sides of the square are equal, a square number might, as he
says (p. 8a, 9), be fitly called ISioiJxj;! (Nicomachus uses raurojjiojt) in
contrast to iripoiiTJinji ; and the ancients, according to him, called square
numbers `` the same `` and `` similar `` (rajirou; re nai ofiotavt), but iTtpo/iiKtK
numbers `` dissimilar and other `` (ovooi'ous koI Baripov),

With regard to solid numbers, Theon remarks in like manner (p. 37, 2)
that atl cube numbers are similar, while of the others those are similar whose
sides are profwrtional,- i.e. in which, as length is to length, so is breadth to
breadth and height to height.

\section*{Definition 22}

Theon of Smyrna (p. 45, 9 sqq.) and Nicomachus (i. 16) both give
the same definition of a perfect number, as well as the law of formation of
such numbers which Euclid proves in the later proposition, ix. 36. They
add however definitions of two other kinds of numbers in contrast with it,
(i) the oTier-pnfect (wVfpTAijv in Nicomachus, virtprtXttot in Theon), the
sum of whose parts, i.e. submultiples, is greater than the numbttr itself, e.g. 1 1,
24 etc., the sum of the parts of 12 being 6+4 + 3 + 2+1 = 16, and the
sum of the parts of 34 being 12 + 8 + 6 + 4 + 3 + 2 + i = 36, (2) the defective
(AAiTTiTv), the sum of whose parts is less than the whole, e.g. S or 14, the
parts in the first case adding up to 4 + a + i, or 7, and in the second case to
7 + 2 + I, or 10. All three classes are however made by Theon subdivisions
of numbers in general, but by Nicouiachus subdivisions of even numbers.

The term perfect was used by the Pythagoreans, but in another sense, of
10; while Theon tells us (p. 46, 14) that 3 was also called perfect ``because
it is the first number that has beginning, middle and extremity; it is also both
a line and a plcme (for it is an equilateral triangle having each side made up
of two units), and it is the first Imk and potentiality of the solid (for a solid
must be conceived of in three dimensions).''

There are certain unexpressed axioms used in Book vii. as there are in
earlier Books.

The following may be noted,

I. If j measures B, and  measures C, A will measure C.

a. l A measures B, apd also measures C, A will measure the difference
between B and C when they are unequal.

3, If A measures B, and also measures C, A will measure the sum of B
and C.

It is clear, from what we know of the Pythagorean theory of numbers, of
musical intervals expressed by numbers, of difTerent kinds of means etc., that
the substance of Euclid Books vii.— ix. was no new thing but goes back, at
least, to the Pythagoreans. It is well known that the mathematics of Plato's
Jlmaeus is essentially Pythagorean. It is therefore \emph{a priori} probable (if not
perhaps quite certain) that Plato irvSayopi'iet even in the passage (32 a, a) where
he speaks of numbers `` whether solid or square `` in continued proportion,
and proceeds to say that between planes one mean suffices, but to connect
two solids two means are necessary. This passage has been much discussed,
but I think that by `` planes `` and `` solids `` Plato certainly meant square and
iolid ntimers respectively, so that the allusion must be to the theorems
estabtished in Eucl.\ viii. ti, 12, that between two square numbers there is
one mean proportional number, and between two cube numbers there are
two mean proportional numbers'.

..-1. . . Iv

' It is true that similar p!ine and solid numbers have the arne property (Eucl, viii. 18,
19) ; but, if Plato had meant similar pkne and solid numbers generally, I think il would
have been necessary to specify that they were `` similir,'' whereas, seeing that the Timams a
as a whole concerned with regular fi|,nires, there is nothing unnatural in allowing rclar or
equilaleral to be understood. Further Plato speaks first of Juni/ini and iytoi and then of
``planes'' [tTrlriia) and ``solids'' (Fttpti.) in such a way as to surest that ixiviiua cor-
respoiid to iwlTttSu, and 6n/Kot to vripti. Now the regular meaning m Jaii is square (or
sometimes square rant), and I think it is here used m the sense of sauarr, notwithstanding
that Plato seems to speak of lAm squares in continued proportion, whereas, in general, the
mean between two squares as eitremes would not be square but olJong. And, if Suti/uu are
squares, it is reasonable to suppose that the tyKot afe also ei/iiiiateral, i.e. the ``sulids'' are
cubes. 1 am aware that Tb. Habler (Biilisthtia Malkimatita, VIII3, 1008, pp. 173—4)
thinks that [he passage is to be explained by reference to the problem of the duplication of
the cube, and does not refer to numbers at all. Against this we have to put the evidence of
Nicomachus (It. 54, 6) who, in speaking of ``a certain Platonic theorem,'' quotes the very
same results of Eud. VIii. 11, n. Secondly, it is worth noting that Hiiblet's explanation is
dulinctly raled out by Democritus tb« Platonjst (jrd cettt. A,D.) who, according to Proclus

\chapter*{Historical Note}

It is no less clear that, in his method and line of argument, Euclid was
following earlier models, though no doubt making improvements in the ex-
position. The tract on the ciio Cnnnis, KOTaro/ii; KavoviK (as to the genuine-
ness of which see above, Vol. !., p. 17) is in style and in the form of the
propositions generally akin to the Ekmenis. In one proposition (2) the author
says ``ife learned ((ftaSoei-) that, if as many numbers as we please be in (con-
tinued) proportion, and the first measures the last, the first will also measure
the intermediate numbers `` ; here he practically quotes EUm. viii. 7- In the
3rd proposition he proves that no number can be a mean between two
numbers in the ratio known as hrmopitu, the ratio, that is, of »  (  i to n, where
« is any integer greater than unity. Now, fortunately, Boethius, De vistitufione
viuska. III. 1 1 (pp. 885—6, ed. Friedlein), has preserved a proof by Archytas
of this same proposition ; and the proof is substantially identical with that
of Euclid. The two proofs are placed side by side in an article by Tannery
(Sibliothcta Mathematiea, vr,, 1905/6, p. 227). Archytas writes the smaller
term of the proportion first (instead of the greater, as Euclid does). Let, he
says, ,  be the `` superparticularis proportio `` (iu-ifiopio!' hasmfii-a in Euclid).
Take C, £>£ the smallest numbers which are in the ratio of A to B. [Here
DE means D + E: and in this respect the notation is different from that of
Euclid who, as usual, takes a line DF divided into two parts at G, GF
corresponding to E, and DG to D, in Archytas' notation. The step of taking
C, DE, the smallest numbers in the ratio of A to B, presupposes Eucl.\ vii.
33 J Then DE exceeds C by an aliquot part of itself and of C [cf the
definition of iTcifiopuK dpi/ia in Nicomachus, i. 19, i]. Let D be the excess
[i.e. E is supposed equal to C]. `` I say that D is not a number but an uniL''

For, if Z'' is a number and a part of DE, it measures JJE ; hence it
measures E, that is, C- Thus JD measures both C and BE, which is
impossible ; for the smallest numbers which are in the same ratio as any
numbers are prime to one another. [This presupposes Eucl, vei. 2a.] There-
fore -D is an unit ; that is, DE exceeds C by an unit. Hence no number can
be found which is a mean between two numbers C, DE. Therefore neither
can any number be a mean between the original numbers A, B which are in
the same ratio [this implies Eucl.\ vii. 20].

We have then here a clear indication of the existence at least as early as
the date of Archytas (about 430 — 365 B.C.) of an Eitments of Aritkmetie in
the form which we call Euclidean ; and no doubt text-books of the sort
existed even before Archytas, which probably Archytas himself and Eudoxus
improved and developed in their turn.

(In P!at<snis Tiaamm (ommtntaria, [+9 c), said that the dlfficultiep of the passage of the
TitHtuui tiad misUd some people into connecting it with tbe duplication of the cube,
whereas it really referred to similar planes and solids with sides in ra/iotial numbers
Thirdly, I do not think that, under tht supposition that the Delian problem is referred to,
we get the required sense. The problem in that case is not that of finding two mean
proportionals Between two eudii Ijut that of finding a second cybe the content of which
ahall Ue equal to twice, or k times (where  is any numtier not  complete cube), the content
of a given Oix (. Two mean proportionals are found, not between cubes, but between
two siraight linn in the ratio of 1 to k, or between a and ks. Unless .( is a culje, there
would lie no point in saying that two means are necessary to connect t aad k, and not one
mean ; for ijk is no more naluial than .Ji, and would be less natural in the case where *
happened to t>e square. On the other hand, if  is a cube,  that it Is a question of finding
means between tuhe numbers, the dictum of Plato is perfectly intelligible ; nor is any real
difficulty caused by the generality of the statement that two means are al-ivays necessary to
connect them, because any property enunciated generally of two cut>e numbers should
obviously be true of cubes 9S sikH, that is, it must hold in the extreme ease of two cubes
which are me to aw attihir.

\part*{Book VII. Propositions}

\begin{proposition}
\label{prop:VII_1}

\begin{statement}
Two UHsqtiai numbers being set out, and the less being
continually subtracted in turn from the greater, if the number
which is left never measures the one before it until an unit is
left, the original numbers mill be prime to one another.
\end{statement}

\begin{proof}

For, the less of two unequal numbers AB, CD being
continually subtracted from the greater, let the
number which is left never measure the one
before it until an unit is left ;

I say that AB, CD are prime to one another,
that is, that an unit alone measures AB, CD.

For, liAB, CD are not prime to one another,
some number will measure them.

Let a number measure them, and let it be
E\ let CD, measuring BF, leave FA less than
itself,

let AF, measuring DG, leave GC less than Itself,

and let GC, measuring FH, leave an unit HA.

Since, then, E measures CD, and CD measures BF,
therefore E also measures BF.

But it also measures the whole BA ;
therefore it will also measure the remainder ..<.

But AF measures DG ;
therefore E also measures DG.

But it also measures the whole DC
therefore it will also measure the remainder CG.

But CG measures FH ;
therefore E also measures FH.

But it also measures the whole FA ;

therefore it will also measure the remainder, the unit AH,
though it is a number : which is impossible.

Therefore no number will measure the numbers 5, CD;
therefore AB, CD are prime to one another. [vn. Def. i a]
\end{proof}

\begin{notes}

It is proper to remark here that the representation in Books vn. to ix. of
numbers by straight lines is adopted by Heiberg from the mss. The method
of those editors who substitute poirtti for lines is open to objection because it
practically necessitates, in many cases, the use of specific numbers, which is
contrary to Euclid's manner.

``Let CD, measuring BF, leave FA less than itself.'' This is a neat
abbreviation for saying, measure along BA successive lensths equal to CD
until a point F is reached such that the length FA remaining is less than
CD ; in other words, let BF be the largest eitact multiple of CD corstained
in BA.

Euclid's method in this proposition is an application to the particular
case of prime numbers of the method of finding the greatest common measure
of two numbers not prime (t) one another, which we shall find in the next
proposition. With our notation, the method may be shown thus. Supposing
the two numbers to be a, b, we have, say,

t '

If now a, 3 are not prime to one another, they must have a commcm
measure t, where e is some integer, not unity.

And since t measures a, 6, it measures a ~pb, i.e. (,

Agn, since t measures b, c, it measures 6-fc, i.e. d,

and lastly, since * measures < , d, it measures e~rd, i.e. i; i . 1. if

which is impossible.

Therefore there is no integer, except unity, that measures a, b, which are
accordingly prime to one another.

Observe that Euclid assumes as an axiom that, if a, b are both divisible by
f, so is a -pb. In the next proposition he assumes as an axiom that c will in
the case supposed divide a +pb,

\end{notes}

\end{proposition}

\begin{proposition}
\label{prop:VII_2}

\begin{statement}
Given (wo numbers not prime to one another, to find their
greatest contnion measure.
\end{statement}

\begin{proof}

Let AB, CD b« the two given numbers not prime to one
another.

Thus it is required to find the greatest J
common measure of AB, CD.

If now CD measures AH — and it also
measures itself — CD is a common measure of
CD, AB.

And it is manifest that it is also the greatest ;
for ao greater number than CD will measure
CD.

But, if CD does not measure AB, then, the less of the
numbers AB, CD being continually subtracted from the
greater, some number will be left which will measure the one
before it.

For an unit will not be left ; otherwise AB, CD will be
prime to one another [vii. i], which is contrary to the
hypothesis.

Therefore some number will be left which will measure
the one before it.

Now let CD, measuring BE, leave EA less than itself,
let EA, measuring DF, leave EC less than itself,

and let CA'' measure AE.

Since then, CF measures AE, and AE measures DF
therefore CF will also measure DF.

But it also measures itself;
therefore it will also measure the whole CD.

But CD measures BE ;
therefore C/also measures BE.

But it also measures EA ;
therefore it will also measure the whole BA.

But it also measures CD ;
therefore CF measures AB, CD.

Therefore CF'is a common measure oi AB, CD,

I say next that it is also the greatest.

For, if CF is not the greatest common measure of AB,
CD, some number which is greater than CF will measure the
numbers AB, CD.

Let such a number measure them, and let it be G.

Now, since G measures CD, while CD measures BE,
G also measures BE. ``'``

But it als© measures the whole BA ;

therefore it will also measure the remainder AB,

But . measures /?y;
therefore G will also measure DF.

But it also measures the whole DC ;

therefore it will also measure the remainder CF, that is, the
greater will measure the less : which is impossible.

Therefore no number which is greater than Ci will measure
the numbers AB, CD ;

therefore CF is the greatest common measure of AB, CD,

\begin{porism*}
From this it is manifest that, if a number
measure two numbers, it will also measure their greatest
common measure.
\end{porism*}
\end{proof}

\begin{notes}

Here we have the exact method of finding the greatest common measure
given in the text-books of algebra, including the reductio ad abmrdum proof
that the number arrived at is not only a common measure but the greatest
common measure. The process of finding the greatest common measure
is simply shown thus :

P±
€)biq   -.

We shall arrive, says Euclid, at some number, say d, which measures the one
before it, i.e. such that e = rd. Otherwise the process would go on until we
arrived at unity. This is impossible because in that case a, b would be prime
to one another, which is contrary to the hypothesis.

Next, like the text-books of algebra, he goes on to show that d will be some
common measure of a, b. For d measures c ;
therefore it measures jff + i, that is, ,   '

and hence it measures pb + , that is, a.

Lastly, he proves that d is the greatest common measure of a, b as follows.

Suppose that ? is a common measure greater than d.

Then e, measuring a, i, must measure a-fb, or c.

Similarly « must measure 6 -qe, that is, d: which is impossible, since e is

by hypothesis greater than d. . . i -  . ;

Therefore etc. `` .-! - .

Euclid's proposition is thus identical with the algebrmical proposidon as
generally given, e.g. in Todhunter's algebra, except that of course Euclid's
numbers are integers.

Niconiachus gives the same rule (though without proving it) when he
shows how to determine whether two given odd numbers are prime or not
prime to one another, and, if they are not prime to one anothet, what is their
common measure. We are, he says, to compare the numbers in turn by
continually taking the less from the greater as many times as possible,
then taking the remainder as many times as jKwsible from the less of the
original numbers, and so on ; this process `` will finish either at an unit or at
some one and the same number,'' by which it is implied that the division of a
greater number by a less is done by separate snbfraetiens of the less. Thus,
with regard to 2 1 and 49, Nicomachus says, `` I subtract the less from the
greater ; a8 is left ; then ;ain I subtract from this the same 2 r (for this is
possible); 7 is left; I subtract this from 2t, 14 is left; from which I again
subtract 7 (for this is possible); 7 will be left, but 7 cannot be subtracted from
7.'' The last phrase is curious, but the meaning of it is obvious enough, as
also the meaning of the phrase about ending `` at one and the same number.''

The proof of the Porism is of course contained in that part of the propo-
sition which proves that G, a common measure different from CF must
measure CF. The supposition, thereby proved to be false, that G is greater
than CFdxxA not affect the validity of the proof that G measures CF'xn any
case.

\end{notes}

\end{proposition}

\begin{proposition}
\label{prop:VII_3}

\begin{statement}
``  . Given three numbers noi prime to one another, to find their
greatest common measure.
\end{statement}

\begin{proof}

Let A, B, C be the three given numbers not prime to
one another ;

thus it is required to find the greatest
common measure oi A, B, C.

For let the greatest common measure,
D, of the two numbers ., .5 be taken ;

[vu. l]

then D either measures, or does not
measure, C,

First, let it measure iL

But it measures A, B also ;
therefore D measures A, B, C ;
therefore /? is a common measure of ., B, C,

I say that it is also the greatest.

For, if i? is not the greatest commoQ measure of, B, C,
some number which is greater than D will measure the numbers

A, B, C.

Let such a number measure them, and let it be E.
Since then E measures A, B, C,  ,

it will also measure A, B ;

therefore it will also measure the greatest common measure
of A, B. [vii. 2, For.]

But the greatest common measure of A, B is D ;
therefore E measures D, the greater the less : which is
impossible.

Therefore no number which is greater than Z? will measure
the numbers A, B, C;

therefore D is the greatest common measure of A, B, C.

Next, let D not measure C ;

I say first that C, D are not prime to one another.

For, since A, B, C are not prime to one another, some
number will measure them.

Now that which measures A, B, C will also measure A,

B, and will measure D, the greatest common measure oi A, B,

[vn.  , Por.]
But it measures C also ;

therefore some number will measure the numbers D, C

therefore D, C are not prime to one another.

Let then their greatest common measure E be taken.

[vii. i
Then, since E measures D,

and /? measures , ,

therefore E also measures A, B. ,

But it measures C also ; '

therefore E measures A, B, C; ,

therefore .£'' is a common measure of A, B, C,

I say next that it is also the greatest.

For, if E is not the greatest common measure of A, B, C,
some number which is greater than E will measure the
numbers A, B, C.

Let such a number measure them, and let it be F.

Now, since /measures A, B, C,   u * '-i. '
it also measures /, Z? ; -' >.i''.

therefore it will also measure the greatest common measure
of W, B. [vn. 2, Por.]

But the greatest common measure of A, B is D ;
therefore J measures D.

And it measures C also ; '``' . ''

therefore /measures Z>, C;

therefore it will also measure the greatest common measure
of D, C. [vn. i, Por.]

But the greatest common measure of Z?, C is  ;
therefore /'' measures £, the greater the Jess: which is
impossible.

Therefore no number which is greater than S will measure
the numbers A, B, C;
therefore £ is the greatest common measure of A, B, C,
\end{proof}

\begin{notes}

Euclid's proof is here longer than we should make it because he
distinguishes two cases, the simpler of which is really included in the other-
Having taken the greatest common measure, say d, of a, t>, two of the
three given numbers a, b, c,h distinguishes the cases

(i) in which d measures (,

(2) in which d does not measure c.

In the first case the greatest common measure of d, e'\ d itself; in the
second case it has to be found by a repetition of the process of v 11. 2. In
either case the greatest common measure oi a, 6, ( is the greatest common
measure of d, c.

But, after disposing of the simpler case, Euclid thinks it necessary to
prove that, if d does not measure c, d and ( must necessarily have a greatest
common measure. This he does by means of the original hypothesis that
tf, b, c are not prime to one another. Since they are not prime to one another,
they must have a common measure; any common measure of a, * is a measure
of d, and therefore any common measure of a, *, r is a common measure of
d, c \ hence d, c must have a common measure, and are therefore not prime to
one another.

The proofs of cases (i) and (2) repeat exactly the same alignment as we
saw in vii. 3, and it is proved separately for <f in case (i) and t in case (3),
where < is the greatest common measure of d, (,

(a) that it is a common measure of a, b, e,

(fi) that it is the greaiesi common measure.

Heron remarks (an-Nairīzī, ed. Curtze, p. 191) that the method does
not only enable us to find the greatest common measure of /Arte numbers ;
it can be used to find the greatest common measure of as many numbers
as we please. This is because any number measuring two numbers also
measures their greatest common measure ; and hence we can find the g.c.m.
of pairs, then the g.c.m, of pairs of these, and so 00, until only two numbers
are lefl and we find the g.c.m. of these. Euchd tacitly assumes this extension
in VII. 33, where he takes the greatest common measure ofay many numbers
as we phase.

\end{notes}

\end{proposition}

\begin{proposition}
\label{prop:VII_4}

\begin{statement}
Any number is either a pari or parts of any number, the
less of the greater.
\end{statement}

\begin{proof}

Let A, BC be two numbers, and let BC be the less ;
I say that BC is either a part, or parts, of A.

For A, BC are either prime to one another
or not.

First, let A, BC be prime to one another.

Then, if BC be divided into the units in it,
each unit of those in BC will be some part of A ;
so that BC is parts of .

Next let A, BC not be prime to one another;
then .5C either measures, or does not measure, A.

If now BC measures A, BC is a part oi A,

But, if not, let the greatest common measure D of A, BC
be taken ; [vn. 2]

and let BC be divided into the numbers equal to D, namely
BE, EF, FC.

Now, since D measures 4, /? is a part of A.

But D is equal to each of the numbers BE, EF, FC; -i*
therefore each of the numbers BE, EF, FC is also a part of A ;
so that BC is parts of A.

Therefore etc.
\end{proof}

\begin{notes}

The meaning of the enunciation is of course that, if a, b be two numbers
of which i is the less, then b is either a submultiple or soirn proper fraction of a.

(i) If a, b are prime to one another, divide each into its units ; then b
contains b of the same parts of which a contains a. Therefore b is `` parts `` or
i. proper Jradion of a.

(i) If a, b be not prime to one another, either b measures a, in which
case i is a submuttiple or `` part `` of a, or, if  be the greatest common
measure of a, b, we may put a — mg and h=ng, and h will contain « of the
same parts (£) of which a contains m, so that b is again ``parts,'' or a. proper
fraction, of a.

\end{notes}

\end{proposition}

\begin{proposition}
\label{prop:VII_5}

\begin{statement}
If a number be a part of a number, and another be the
same part of another, the sum will also be the same part of the
sum that the one is of the one.
\end{statement}

\begin{proof}

For let the number -(4 be a part of BC,

and another, D, the same part of another EF that A is of C;

I say that the sum of A, D  also the same
part of the sum of BC, EF that A is of BC.

For since, whatever part A is of BC, D
IS also the same part oi EF,

therefore, as many numbers as there are in
j5C equal to A, so many numbers are there
also in EF equal to D.

Let BC be divided into the numbers equal to A, namely
BG, GC,

and EF into the numbers equal to D, namely EH, HF

then the multitude of BG, GC will be equal to the multitude
of EH, HF.

And, since BG is equal to A, and EH to D,
therefore BG, EH are also equal to A, D.

For the same reason
GC, HF 2iX also equal to A, D.

Therefore, as many numbers as there are in BC equal to
A, so many are there also in BC, £''7 equal to A, D.

Therefore, whatever multiple BC o\ A, the same multiple
also is the sum of BC, EF of the sum of A, D.

Therefore, whatever part A is of BC, the same part also
is the sum of A, D of the sum of BC, EF.
\end{proof}

\begin{notes}

If a = -6, and e = -d, then

it
The proposition is of course true for any quantity of pairs of numbers
similarly related, as is the next proposition slso ; and both propositions aie
used in the extended form in vil 9, 10.

A

C

D

Q

H

B

E

\end{notes}

\end{proposition}

\begin{proposition}
\label{prop:VII_6}

\begin{statement}
If a number be parts of a number, and another be the same
parts of another, (he sum will also be the same parts of the sum
that the one is of the one.
\end{statement}

\begin{proof}

For let the number AB be parts of the number C,
and another, DE, the same parts of another,
F, that AB is of C ;

I say that the sum of AB, DE is also the
same parts of the sum of C, F that AB is
of C

For since, whatever parts AB is of C,
DE is also the same parts of F,
therefore, as many parts of C as there are
in AB, so many parts of / are there also in DE.

Let AB be divided into the parts of C, namely AG, GB,
and DE into the parts of , namely DH, HE;
thus the multitude of AG, GB will be equal to the multitude
of Z?jy, HE,

And since, whatever part AG is of C, the same part is
/JiYof Aalso,

therefore, whatever part AG o( C, the same part also is the
sum of AG, DH of the sum of C, F. [vii. 5]

For the same reason,    . ;

whatever part GB is of C, the same part also js the sum of
GB, HE of the sum of C, F,

Therefore, whatever parts AB is of C, the same parts also
is the sum of AB, DE of the sum of C, F.
\end{proof}

\begin{notes}

If a = — i, ana c = — d,

r n n

then a + €=- (fi + d).

More generally, if

OT . Iff V fft f

a = ~ b, c= — a, «-- /,
n ft n

then (a +(+ ( + g+ ,,,) = - (i-*-t/+/-i

In Euclid's proposition m<.n, but tht generality of the result is of course
not aiTected. This proposition and the last are complementary to v, i, which
proves the corresponding result with multiple substituted for ``pari'' or

\end{notes}

\end{proposition}

\begin{proposition}
\label{prop:VII_7}

\begin{statement}
If a number be that part of a number, which a number
subtracted is of a number subtracted, the remainder will also
be the same part of the remainder thai the whole is of the
whole.
\end{statement}

\begin{proof}

For let the number AB be that part of the number CD
which AE subtracted is of CF subtracted ;

I say that the remainder EB is also the same part of the
remainder FD that the whole AB is of the whole CD.

For, whatever part AE is of CF, the same part also let
EB be of CG.

Now siQce, whatever part AE is of CF, the same part
also is EB of CG,

therefore, whatever part AE is of CF, the same part also is
AB of GF. [vii. s]

But, whatever part AE is of CF, the same part also, by
hypothesis, is AB of CD ;

therefore, whatever part AB   o GF, the same part is it of
CD also ; ,. .

therefore GF is equal to CD.

Let CF be subtracted from each ;
therefore the remainder GC is equal to the remainder FD.

Now since, whatever part AE is of CF, the same part
also is EB of GC,

while GC is equal to FD,

therefore, whatever part AE is of CF, the same part also is
of/'Z>.

But, whatever part AE is of CF, the same part also is AB
oiCD;
therefore also the remainder EB is the same part of the
remainder FD that the whole AB is of the whole CD.
\end{proof}

\begin{notes}

If a= -li and f=-rf, we are to prove that

a~c = -(6-d),

a result differing from that of vji. 5 in that minus is substituted for /Ar.
Euclid's method is as follows.

Suppose that e is taken such that

a- = -e. (i)

Now e=-d.

n

Therefore a = -(d-¥e), [vii. 5]

whence, from the hypothesis, d-ve = b,
so that e = b-d,

and, substituting this value of in (i), we have

.i , a-e=-(b'-d).

\end{notes}

\end{proposition}

\begin{proposition}
\label{prop:VII_8}

\begin{statement}
If a number be the same parts of a number that a number
subtracted is of a number subtracted, the remainder will also
be the same parts of the remainder that the whole is of the
whole.
\end{statement}

\begin{proof}

For let the number AB be the same parts of the number
CD that AE subtracted is of CF
subtracted ;

I say that the remainder EB is
also the same parts of the re-
mainder FD that the whole AB
is of the whole CD.

For let GH be made equal to AB,

Therefore, whatever parts GH is of CD the same parts
also is AE of CF.

Let GHhe, divided into the parts of CD, namely GK, KH,
and AE into the parts of CF, namely AL, LE;

thus the multitude of GK, KH-\ be equal to the multitude
of AL, LE.

c

f =

Q

M K

N H

A

L

t B

Now since, whatever part GK is of CD, the same part

also is Z of C/s ``-<

while CD is greater than CF,
therefore GK is also greater than AL,

Let GM be made equal to AL.

Therefore, whatever part GK is of CD, the same part also
is GM oi CF;

therefore also the remainder MK is the lame part of the
remainder FD that the whole GK is of the whole CD. [vii. 7]

Again, since, whatever part KH is of CD, the same part
also is EL of CF,
while CD is greater than CF'',
therefore HK is also greater than EL.

Let KN be made equal to EL .

Therefore, whatever part KN is of CD, the same part
also is KJV of CF;

therefore also the remainder JV/f is the same part of the
remainder FD that the whole A'If is of the whole CD,

[vn. 7]

But the remainder MK was also proved to be the same
part of the remainder FD that the whole GK is of the whole
CD;

therefore also the sum of MK, NH is the same parts of DF
that the whole HG is of the whole CD.

But the sum of MK, NH is equal to EB,
and HG is equal to BA ;

therefore the remainder EB is the same parts of the remainder
FD that the whole AB is of the whole CD.
\end{proof}

\begin{notes}

It - - «--<( and i==~d.

(m <n)

then fl-f=(i-i).

Euclid's proof amounts to the following.

Take e equal to - b, and /equal to - d.

ft H

Then since, by hypothesis, b->d.

and, by VII. 7, e -f- -(b- d).

Repeat this for all the parts equal to t and/that there are in a, b respec-
tively, and we have, by addition (a, b containing m of such parts respectively),

»'i.:~n = ``'-(b-i).

But m(i—/) = a-c.

Therefore a-e- (d-d).

n

The propositions v[i, 7, 8 are complementary to v. 5 which gives the
corresponding result with multipit in the place of `` part `` or `` parts.''

\end{notes}

\end{proposition}

\begin{proposition}
\label{prop:VII_9}

\begin{statement}
If a number be a pari of a number, and another be the
same part of another, alternately also, whatever part or parts
ike first is of the third, the same part, or the same parts, will
the second also be of the fourth.
\end{statement}

\begin{proof}

For let the number A he a part of the number BC,
and another, D, the same part of another, £/, .. ,

that A is of BC ;

I say that, alternately also, whatever part or

parts A is of D, the same part or parts is BC , a
oi£F3.so.

For since, whatever part A is of BC, the
same part also is D of £/,

therefore, as many numbers as there are in BC equal to A,
so many also are there in £1 equal to D.

Let BC be divided into the numbers equal to A, namely
BG, GC,

and /into those equal to D, namely £If, HF\ -w ,„
thus the multitude of BG GC will be equal to the multitude
of EH, HF.

Now, since the numbers BG, GC are equal to one another,

and the numbers EH, HF are also equal to one another,

while the multitude of BG, GC is equal to the multitude of
EH, HF,

therefore, whatever part or parts BG is of EH, the same
part or the same parts is GC oi HF also ;

so that, in addition, whatever part or parts BG is of EH,
the same part also, or the same parts, is the sum BC of the
sum £F. [vit. s, 6]

But BG is equal to A, and EH to D\ '

therefore, whatever part or parts A is of D, the same part or
the same parts is BC of EF also.
\end{proof}

\begin{notes}

If a = - b and e = ~ d, then, whatever fraction (`` part `` or `` parts'') a is of
ft ft

c, the same fraction will ihe of ti.

Dividing i into each of its parts equal to a, and d into each of its parts
equal to i, it is clear that, whatever fraction one of the parts a is of one of the
parts c, the same fraction is any other of the parts a of any other of the parts .

And the number of the parts a is equal to the number of the parts /:, viz. n.

Therefore, by vii. 5, 6, na is the same fraction of »c that a is of (, i.e. 6 is
the same fraction of d that o is of <:.

\end{notes}

\end{proposition}

\begin{proposition}
\label{prop:VII_10}

\begin{statement}
If a numier be parts of a number, and another be ike
same Paris 0/ another, alternately also, wliatever parts or part
the first is of the third, the same parts or the same part will
the second also be of t/ie fourth,
\end{statement}

\begin{proof}

For let the number AB he parts of the number C,
and another, DE, the same parts of another,
;

I say that, alternately also, whatever parts or
part AB is of -DE, the same parts or the
same part is C of E also.

For since, whatever parts AB is o( C, ''
the same parts also is DE of E,
therefore, as many parts of C as there are
in AB, so many parts also of E are there in DE.

Let AB be divided into the parts of C, tiamely AG, GB,
and DE into the parts of E, namely DIf, HE ;
thus the multitude of AG, GB will be equal to the multitude
oiDH, HE.

Now since, whatever part AG xsoi C, the same part also
IsDHoE,

alternately also, whatever part or parts AGxtS DH,

the same part or the same parts is C of E also. [vu. 9]

For the same reason also,
whatever part or parts GB is of HE, the same part or the
same parts is C of J also ;

so that, in addition, whatever parts or part AB is of /?£',
the same parts also, or the same part, is C of F> [vii. 5, 6]
\end{proof}

\begin{notes}

( a = — b and e ~ —d, then, whatever fraction a is of c, the same fraction

is j of d.

To prove this, a is divided into its m parts equal to t>jn, and c into its
»J parts equal to djn.

Then, by v[i. 9, whatever fraction one of the m parts of a is of one of the
m parts of , the same fraction is a of d.

And, by vii, 5, 6, whatever fraction one of the If/ parts of a is of one of
the t» parts of c, the same fraction is the sum of the parts of a (that is, o) of
the sum of the parts ol c (that is, i).

Whence the resliit follows.

Fn the Greek text, after the words `` so that, in addition `` in the last line
but one, is an additional explanation making the reference to vii. 5, 6 clrer,
as follows: ``whatever part or parts AG is of DII, the same part or the
same parts is GS of HE also ;

therefore also, whatever part or parts C is of DH, the same part or the same
parts is AB of DE also. [vii. 5, 6]

But it was proved that, whatever part or parts AG s of DH, the same
part or the same parts is C of Faho ;
therefore also `` etc, as in the last two lines of the text.

Heiberg concludes, on the authority of P, which only has the words in
the margin in a later hand, that they may be attributed to Theon.

\end{notes}

\end{proposition}

\begin{proposition}
\label{prop:VII_11}

\begin{statement}
If, as whole is to whole, so is a number subtracted to a
number subtracted, the retnainder will also be to the remainder
as whole to whole.
\end{statement}

\begin{proof}

As the whole 5 is to the whole C£>, so tet A£ subtracted
be to C/ subtracted ;

I say that the remainder £B is also to the remainder
FD as the whole AB to the whole CD.

Since, as AB is to C£), so is A£ to CF,
whatever part or parts AB is of CD, the same part
or the same parts is AB of CF a.lso ; [vn. Def. 20]

Therefore also the remainder BB is the same
part or parts of FD that AB is of CD. [vn, 7, 8]

Therefore, as £B is to FD, so is AB to CD. [vn. Def. 30]
\end{proof}

\begin{notes}

It will be observed that, in dealing with the proportions in Props, 11—13,
Euclid only contemplates the case where the first number is ``a part'' or
``parts'' of the second, vhile in Prop. 13 he assumes the first to be ``a part''
or ``parts'' of the third also; that is, the first number ts in all three propositions
assumed to be less than the second, and in Prop. 13 less than the third also.
Yet the figures in Props. 1 1 and 1 3 are inconsistent with these assumptions.
If the facts are taken to correspond to the figures in these propositions, it is
necessary to take account of the other possibilities involved in the definition
of proportion (vii. Def 20), that the first number may also he a multiple, or
a multiple Ifkj ``a part'' or `` parts'' (including owe as a multiple in this case),
of each number with which it is compared. Thus a number of different cases
would have to be considered. The remedy is to make the ratio which is in
the lower terms the first ratio, and to invert the ratios, if necessary, in order
to make `` a part ``or `` parts `` literally apply.

If a : i   : d, (a > c, b > tf)

then (a — c):(b~d) = a;b.

This proposition for numbers corresponds to v. 19 for magnitudes. The
enunciation is the same except that the masculine (agreeing with apiS/io!)
takes the place of the neuter (agreeing with niytdm).

The proof is no more than a combination of the arithmetical definition of
proportion (vu. Def. 20) with the results of vii, 7, 8. The language of propor-
tions is turned into the language of fractions by Def. zo ; the results of vii. 7, 8
are then used and the language retransformed by Def. 20 into the language of
proportions.

\end{notes}

\end{proposition}

\begin{proposition}
\label{prop:VII_12}

\begin{statement}
If there be as Tnany numbers as we please in proportion,
then, as one of ike antecedents is to one of the consequents, so
are all the antecedents to ail the consequents.
\end{statement}

\begin{proof}

Let A, By C, D be as many numbers as we please in
proportion, so that,

as A is to By so is C to Z? ;
I say that, as . is to B, so are A, C to B, D.

For since, as A is to B, so Is C to D, aI bI c

whatever part or parts A is of B, the same part
or parts is C q( D also. [vn. Def. 20]

Therefore also the sum of A, C is the same
part or the same parts of the sum of B, D that A is of B.

[vii. 5, 6]

Therefore, as A is to B, so are A, C to B, D. [vn. Def. 20]
\end{proof}

\begin{notes}

If a:(( = l,:V = c:i;=...,

then each ratio is equal to (a +  + + ...) : (0' + *'+/ + ,..).

The proposition corresponds to v, 1 2, and the enunciation is word for word
the same with that of v. 12 except that apiSfio? takes the place of fiytoi.

Again the proof merely connects the arithmetical definition of proportion
(vn. Def 20) with the results of vii, 5, 6, which are quotttd as true for any
number of numbers, and not merely for two numbers as in the enunciations of
VII. s, 6.

\end{notes}

\end{proposition}

\begin{proposition}
\label{prop:VII_13}

\begin{statement}
If four numbers be proportional, they will also be propor-
tional alternately.
\end{statement}

\begin{proof}

Let the four numbers A, B, C, D ha proportional, so that,
as A is to B, so is C to Z? ;
I say that they will also be proportional alternately, so that,
as A is to C, so will B be to D.
For since, as A is to B, so is C to D,
therefore, whatever part or parts A is of B,
the same part or the same parts is C of Z? also.

[v[i. Def. ao]
Therefore, alternately, whatever part or
parts A is of C, the same part or the satne
parts 5 B oi D also.

Therefore, as  is to C, so is B to D.

If

[vii. 10]
[y. Def. so]
\end{proof}

\begin{notes}

a : b = e : d,
then, alternately, a : c = b id.

The proposition corresponds to v. 16 for magnitudes, and the proof
consists in connecting vii. Def. 20 with the result of vii. 10.

\end{notes}

\end{proposition}

\begin{proposition}
\label{prop:VII_14}

\begin{statement}
If there be as many numbers as me please, and others equal
to them in multitude, which taken two and two are in the same
ratio, they will also be in the same ratio ex aequali.
\end{statement}

\begin{proof}

Let there be as many numbers as we please A, B, C,
and others equal to them in multitude D, E, F, which taken
two and two are in the same ratio, so ihat,

as .(4 is to B, so is Z? to ,
and, as  is to C, so is E to F\ m   f .

I say that, ex aequali,

as A is to C, so also is D to F.

``B''

D

c — f

For, since, as A is to B, so is D to E,
therefore, alternately,

as A is to D, so is B to E.

[vn. 13]

Again, since, as B is to C, so is E to F,
therefore, alternately,

as B is to Ey so is C to F. [vii. 13]

But, as J? is to E, so is W to D;
therefore also, as y is to D, so is C to F.
Therefore, alternately,

as A is to C, so is D to F. . [iij
\end{proof}

\begin{notes}

If a: b = d.e,

and d : c = t -.ft

tKen, ex aepiali, a ; c = d ./;

and the same is true however many successive numbers are so related.

The proof is simphcity itself.

By VII. 13, alternately, a : il - i : f,

and b : e = c '. f. .

Therefore ad = c:f, 1

and, again alternately, a : cd if.

Observe that this simple method cannot be used to prove the corresponding
proposition for magnitudes, v. 22, although v. 22 has been preceded by the
tivo propositions in that Book corresponding to the propositions used here,
viz, V. 16 and v, 1 1. The reason of this is that this method would only prove
V. 22 for six magnitudes all t>f tht sttmt kind, whereas the magnitudes in v. jj
are not subject to this limitation.

Heiberg remarks in a note on V[r. 19 that, while Luclid has proved
several propositions of Book v. over again, by a separate proof, for numbers,
he has neglected to do so in certain cases; e.g., he often uses v. 1 1 in these pro-
positions of Book VII., V. 9 in vii. 19, v. 7 in the same proposition, and so on.
Thus Heiberg would apparently suppose Euclid ``o use v, 1 1 in the last step
of the present proof (Raiies whkk art the same with ike same ratio are also the
same with one another). I think it preferable to suppose that Euclid regarded
the last step as axiomatic ; since, by the definition of proportion, the first
number is the same multiple or the same part or the same parts of the second
that the third is of the fourth : the assumption is no more than an assumption
that the numbers or proper fractions which arc respectively equal to the same
number or proper fraction are equal to one another.

Though the proposition is only proved of six numbers, the extension to as
many as we please (as expressed in the enunciation) is obvious.

\end{notes}

\end{proposition}

\begin{proposition}
\label{prop:VII_15}

\begin{statement}
If an unit measure any number, and another number measure
any other number tfie same number of times, alternately also,
the unit ivill measure the third number the same number of
times that the second measures the fourth.
\end{statement}

\begin{proof}

For let the unit A measure any number BC,
and let another number D

measure any other number EF ~— + !l! — ?

the same number of times ;

I say that, alternately also, the  f L F

unit A measures the number

D the same number of times that BC measures EF.

For, since the unit A measures the number BC the same
number of times that D measures /:F,

therefore, as many units as there are in BC, so many numbers
equal to D are there in also.

Let BC be divided into the units in it, BG, GH, HC,
and EF into the numbers EK, KL, Z./ equal to D.

Thus the multitude of BG, GH, HC will be equal to the
multitude of EK, KL, LF.

And, since the units C GH, HCzx equal to one another,

and the numbers EK, KL, LF are also equal to one another,

while the multitude of the units BG, GH, HC is equal to the
multitude of the numbers EK, KL, LF,

therefore, as the unit BG is to the number EK, so will the
unit GH be to the number KL, and the unit HC to the
number LF,

Therefore also, as one of the antecedents is to one of
the consequents, so will all the antecedents be to all the
consequents ; [vii. iz]

therefore, as the unit BG is to the number EK, so is BC to
EF.

But the unit BG is equal to the unit A,

and the number EK to the number D.

Therefore, as the unit A is to the number D, so is BC to
EF.

Therefore the unit A measures the number D the same
number of times that BC measures EF.
\end{proof}

\begin{notes}

If there be four numbers ,m,a, ma (such that i measures m the same
number of times that a measures ma), i measures a the same number of
times that m measures ma.

Except that the first number is unity and the numbers are said to tfieasure
instead of being a art of others, this proposition and its proof do not differ
from VII, 9 J in fact this proposition is a particular case of the other.

\end{notes}

\end{proposition}

\begin{proposition}
\label{prop:VII_16}

\begin{statement}
If two numbers by multiplying one another make certain
numbers, the numbers so produced mill be equal to one another.
\end{statement}

\begin{proof}

Let A, B he two numbers, and let A by multiplying B
make C and  by multiplying ,j.

A make D ; f

I say that C is equal to D, b

For, since A by multiply- c —

ing B has made C, o

therefore B measures C ac- —
cording to the units in A.

But the unit E also measures the number A according to
the units in it ;

therefore the unit E measures A the same number of times
that B measures C.

Therefore, alternately, the imit E measures the number B
the same number of times that A measures C. [vn. ij]

Again, since B by multiplying A has made D,

therefore A measures D according to the units in . '

But the unit E also measures B according to the units
in it ;

therefore the unit E measures the number B the same
number of times that A measures D.

But the unit E measured the number B the same number
of times that A measures C ;

therefore A measures each of the numbers C, D the same
number of times.

Therefore C is equal to i?.
\end{proof}

\begin{annotations}

 1. The numbers > produced. The Gieek hu al yoituw i( niriir, `` the (numbers)
produced /rffjw fJutJt.'' By *'from them'* Euclid means ``from the original numbers,'' though
this is not very clear even in the Greek. I think ambiguity is best avoided by leaving out
the words.

\end{annotations}

\begin{notes}

This proposition proves that, if any nttmbers bt fmtttiplied together, the order
of muUiplication is indifferent, ox ab-ba.

It 's important to get a clear understanding of what Euclid means when
he speaks of sne number multiplying another, vti. Def, 15 states that the
effect of ``a multiplying b'' is taking a times b. We shall always represent
`` a times b `` by ab and `` b times a `` by ba. This being premiseidj the proof
that ab = ba may be represented as follows in the language of proportions.

. ... Ku
[vii. 13]

By V!i. Def. 10,

I : a = i : ai.

Therefore, alternately,

I : t = a : a.

Again, by vii. Def. 10,

X : i = a : ba.

Therefore

a : ab = a \ ba.

or ttb-ba.

Euclid does not use the language of proportions but that of fractions or
their equivalent measures, quoting vn, 15, a particular case of vii, 13
differently expressed, instead of vii, 13 itself.

\end{notes}

\end{proposition}

\begin{proposition}
\label{prop:VII_17}

\begin{statement}
If a number by muUiplymg itvo 7iuml>ers make certain
numbers, the numbers so produced will have ike same ratio
as the numbers multiplied.
\end{statement}

\begin{proof}

For let the number A by multiplying the two numbers B,
C make D, B\ - '

I say that, as .5 is to C, so is D to E.

For, since A by multiplying B has made D,
therefore B measures D according to the units in A.

A

B C-

F

But the unit /also measures the number A according to
the units in it ;

therefore the unit F measures the number A the same number
of times that B measures D.

Therefore, as the unit P is to the number , so is  to D.

[vii. Def. 30]

For the same reason,
as the unit F is to the number A, so also is C to £  ;
therefore also, as .5 is to /?, so is C to £.

Therefore, alternately, as B is to C, so is D to E. [vn. 13]
\end{proof}

\begin{notes}

In this case Euclid translates the language of measures into that of
proportions, and the proof is exactly like that set out in the last note.
By VII. Def. so, i : a = b : ai,

and X -.a-c: ac.

Therefore b : ahc ac,

and, altematdy, 6:c = ai:at. [*'' 'S]

\end{notes}

\end{proposition}

\begin{proposition}
\label{prop:VII_18}

\begin{statement}
If two numbers by multiplying any number make certain
numbers, the numbers so produced will have the same ratio as
the multipliers.
\end{statement}

\begin{proof}

For let two numbers A, B yj multiplying any number C
make D, E ;
I say that, as A is to B, so is D c

For, since A by multiplying e

C has made Z?,

therefore also C by multiplying A has made Z?. [vil 16]

For the same reason also  ,. ,

C by multiplying B has made E.

Therefore the number C by multiplying the two numbers
A, B has made D, E.

Therefore, as j4 is to B, so is D to E. [vii. 17]
\end{proof}

\begin{notes}

It is here proved that a:b=€ie:be.

The argument is as follows.

ac = ea. [vil, i6]

Similarly 6( - cb.

And a:i = ca:€bi [``I- 17]

therefcne a : b = tu : 6e.

\end{notes}

\end{proposition}

\begin{proposition}
\label{prop:VII_19}

\begin{statement}
If four numbers be proportional, the number produced from
the first and fourth will be equal to the number produced from
the second and third; and, if the number produced from the
first and fourth be equal to that produced from the second and
third, the four numbers will be proportional.
\end{statement}

\begin{proof}

Let A, B, C, Dhe four numbers in proportion, so that,
as j4 is to B, so is C to i? ;
and let A by multiplying D make £, and let B by multiply-
ing C make E;
I say that E is equal to E.

For let A by multiplying C make G,

Since, then, A by multiplying C has made G, and by
multiplying D has made E,
the number A by multiplying the two
numbers C, D has made G, E.

Therefore, as C is to Z?, so is G to E.

[vu. 17]

But, as C is to V, sols A 10 B ;
therefore also, as  is to 5, so is C
to E.

Again, since A by multiplying C
has made G,

but, further, B has also by multiplying
C made F,

the two numbers A, B by multiplying a certain number C
have made G, E.

Therefore, as A is to B,sos G to E. [vii. 18]

But further, as A is to B, i is G to E also ;
therefore also, as G Is to E, so is G to F.

Therefore G has to each of the numbers E, E the same
ratio ;

therefore E is equal to E.

Again, let E be equal to E;
I say that, as . is to B, so is C to D.

For, with the same construction,
since E is equal to F,
therefore, as (? is to E, so is G to F.

But, as £7 is to E, so is C to D,

and, as G is to E, so is A to B.

Therefore also, as A is to , so is C to D.
\end{proof}

\begin{notes}

If  ; I:. a :b = e:d,

then ad=kc; and conversely.
The proof is equivalent to the following,

(i) at : ad—( : d

 f\ if.xhi  . I I =a : i, ``

Bui `` ,W a:»acii(.

Therefore   i    m : ad = ae : be,

or ad=be.

[cf. V. 9]

  a y

 'l-.v . ,

[cf. V. 7]

[vn. 17]

[VII. 18]

[vn. 17]
'``  (vn. 18]

Since

ad = b<i.

ac \ ad = ae ; be.

But

ac : ad- e : d.

and

ac;bc~a : b.

Therefore

a b = c \ d.

[vii. 17]
[VII. iB]

As indicated in the note on vii. 14 above, Heiberg regards Euclid t
basing the inferences contained in the last step of part (i) of this proof and
in the first step of part (2) on the propositions v, 9 and v. 7 respectivefy,
since he has not proved those propositions separately for numbers in this
Book. I prefer to suppose that he regarded the inferences as obvious and
not needing proof, in view of the definition of numbers which are in pro-
portion. E.g., if at is the same fraction (`` part `` or `` parts ``) of ad that at is
of be, it is obvious that ad must be eiual to be.

Heiberg omits from his text here, and relegates to an Appendix, a
proposition appearing in the manuscripts V, p,  to the effect that, if ihret
numbers be proportional, the product of the extremes is equal to the square
of the mean, and conversely. It does not appear in P in the first hand, B has
it in the margin only, and Campanus omits it, remarking that Euclid does
not give the proposition about three proportionals as he does in vi. 17, since
it is easily proved by the proposition just given. Moreover an-Nairiri quotes
the proposition about three proportionals at an obsemaiion on vii. 19 probably
due to Heron (who is mentioned by name in the preceding paragraph).

\end{notes}

\end{proposition}

\begin{proposition}
\label{prop:VII_20}

\begin{statement}
The least numbers of those whitk have ike same ratio with
them measure those which Jtave the same ratio the same number
of times, the greater the greater and t lie less the less.
\end{statement}

\begin{proof}

For let CD, EF be the least niimbens of those which have
the same ratio with A, B ;
I say that CD measures A the same number
of times that EF measures B.

Now CD is not parts of .

For, if possible, let it be so ;
therefore EF is also the same parts of B
that CD is of . [vn. 13 and Def. ao]

Therefore, as many parts of A as there
are in CD, so many parts of  are there also
in EF,

Let CD be divided into the parts of A, namely CG, GD,
and £''into the parts oi B, namely EH, HF
thus the multitude of CG, GD will be equal to the multitude
of EH, HF.

Now, since the numbers CG, GD are equal to one another,
and the nunnbers EH, HF are also equal to one another,

while the multitude of CG, GD is equal to the multitude of
EH, HF,

therefore, as CG is to EH, so is GD to HF, `` ``

Therefore also, as one of the antecedents is to one of
the consequents, so will all the antecedents be to all the
consequents. [vu. u]

Therefore, as CG is to EH, so is CD to EF.

Therefore CG, EH are in the same ratio with CD, EF,
being less than they :

which is impossible, for by hypothesis CD, EF are the least
numbers of those which have the same ratio with them.

Therefore CD is not parts of A ;

therefore it is a part of it. [vit, 4]

And EF is the same part of B that CD is of  ;

[vii. 13 and Def. 20]
therefore CD measures A the same number of times that EF
measures B.
\end{proof}

\begin{notes}

If a, b are the least numbers among those which have the same ratio
(i.e. if ajb is a fraction in its lowest terms), and t, d are any
others in the same ratio,, i.e. if

a: b't :d,

then o = - f and b = - d, where tt is some inter.

ft tt , 1 .  I

The proof is by \emph{reductio ad absurdum}, thus.

[Since a<c,a  some proper fraction (`` part `` or ``parts ``) of e, l vii. 4.]

Now a cannot be equal to —e, where m is an integer less than n but

greater than i.

For, if a = -f, b= -rfalso. [vii, 13 and Def. aol

Take each of the m parts of a with each of the m parts of b, two and two ;

the latio of the members of all pairs is the same ratio ~ a :  - b.

mm

Therefore

- ix : - 6 = a:b. fvit. la]

m nt

But — a and — b are respectively less than a, b and they are in the same
tit tti

ratio : which contradicts the hypothesis.

Hence a can only be `` a part `` of r, or

a is of the form - c,
ft

and therefore d is of the form - rf.  , .

Here also Heibe omits a proposition which was no doubt interpolated
by Theon (B, V, p,  have it as vii. 22, hut P only has it in the margin
and in a later hand ; Campanus also omits it) proving for numbers the m
aequali proposition when ``the proportion is perturbed,'' i.e. (cf. enunciation
of V. jj) if

a:b = ef, (i)

and 6:e = d:e, (2)

then a .e = d:/.

The proof (see Heiberg's Appendix) depends on vii. 19.
From (i) we have of —be,

and from (2) bt — cd. [vii. 19]

Therefore af= cd,

and accordingly a:c-d:f. ,. [vii. 19]

\end{notes}

\end{proposition}

\begin{proposition}
\label{prop:VII_21}

\begin{statement}
Numbers prime io one another are the least of those which
have the same ratio with them,
\end{statement}

\begin{proof}

Let A, B be numbers prime to one another;
I say that A, B are the least of
those which have the same ratio
with them.

For, if not, there will be some
numbers less than A, B which are
in the same ratio with A, B.

Let them be C, D.

Since, then, the least numbers of those which have the
same ratio measure those which have the same ratio the
same number of times, the greater the greater and the less
the less, that is, the antecedent the antecedent and the
consequent the consequent, [vii. 20]

therefore C measures A the same number of times that D
measures B.

Now, as many times as C measures A, so many units let
there be in E.

Therefore D also measures B according to the units in E.

And, since C measures  according to the units in £,
therefore £ also measures A according to the units in C.

[vn. 16]
For the same reason

£ also measures B according to the units in D, [vii. 16]

Therefore £ measures A, B which are prime to one

another : which is impossible. [vn. Def. u]

Therefore there will be no numbers less than A, B which

are in the same ratio with A, B.

Therefore A, B are the least of those which have the same

ratio with them.
\end{proof}

\begin{notes}

In other words, \ a,b are prime to one another, the ratio a : J is `` in its
lowest terms.''

The proof is equivalent to the following.

If not, suppose that f, </ are the least numbers for which
a b-e \ d.
[Euclid only supposes tome numbers f, d in the ratio of o to i such that
ir<a, and (consequendy) d-b. It Js however necessary to suppose that
f, d are the least numbers in that ratio in order to enable vn. 3o to be
used m the proof.]

Then [vn. ao] a = mt, and b = md, where m is some integer.

Therefore o = cm, b - dm, [vn. 16]

and m is a common measure of a, b, though these ate prime to one another .
which is impossible. [vn. Def. la]

Thus the least numbers in the ratio of <7 to  cannot be less than a, i
thetnselves.

Where I have quoted vn. i6 Heiberg regards the reference as being to
VII.   5. I think the phraseology of the text combined with that of Def. 15
suggests the former rather than the latter.

\end{notes}

\end{proposition}

\begin{proposition}
\label{prop:VII_22}

\begin{statement}
T/te hast numbers of those which have the same ratio with
them are prime to one another.
\end{statement}

\begin{proof}

Let .,  be the least numbers of those which have the
same ratio with them ;

I say that A, B are prime to one g

another.

c

For, if they are not prime to one d

another, some number will measure

them.

Let some number measure them, and let it be C

And, as many times as C measures A, so many units
let there be in D, „:. -t rw-h

and, as many times as C measures B, so many units let there
be in E

<s Since C measures A according to the units in D, .<w'ik
therefore C by multiplying D has made A, [vii, Def. 15]

For the same reason also
C by multiplying E has made B. '

Thus the number C by multiplying the two numbers /?,
E has made A, B ;

therefore, as D is to E, so is  x.o B\ [vii. 17]

therefore D, E are in the same ratio with A, B, being less
than they : which is impossible.

Therefore no number will measure the numbers A, B.

Therefore A, B are prime to one another.
\end{proof}

\begin{notes}

i a: b `` in its lowest terms,'' a, b are prime to one another.

Again the proof is indirect.

If a, b are not prime to one another, they have some common measure f.
Mid .„ . ,

a = m, b = ne.

Therefore m : n-a : b. [vii. 17 or iS]

But m, n are less than a, b respectively, so that a ; A is not in its lowest
terms : which is contrary to the hypothesis.

Therefore etc.

\end{notes}

\end{proposition}

\begin{proposition}
\label{prop:VII_23}

\begin{statement}
If two numbers be prime to one another, (he number which
measures the one of them wilt be prime to the remaining
number.
\end{statement}

\begin{proof}

Let A, B be two numbers prime to one another, and let
any number C measure A ;
I say that C, B are also prime to one another.

For, if C, B are not prime to one another,
some number will measure C, B.

Let a number measure them, and let it be D.

Since D measures C, and C measures A,
therefore D also measures A. a a c 6

But it also measures B;

therefore Z? measures A, B which are prime to one another :
which is impossible, [vii, Def. la)

Therefore no number will measure the numbers C B.

Therefore C, B are prime to one another.
\end{proof}

\begin{notes}

If a, mil are prime to one another, b is prime to a. For, if not, some
number d will measure both a and *, and therefore both a and mb ; which is
contrary to the hypothesis.

Therefore etc.

\end{notes}

\end{proposition}

\begin{proposition}
\label{prop:VII_24}

\begin{statement}
If two numbers be prime to any number, their product also
will be prime to the same.
\end{statement}

\begin{proof}

For let the two numbers A, B be prime to any number C,
and let A by multiplying B make D
I say that C, D are prime to one another.

For, if C, D are not prime to one another,
some number will measure C, D.

Let a number measure them, and let it
be.

Now, since C, A are prime to one
another,

and a certain number E measures C,

therefore A, £ 3.re prime to one another. [v:t. 23]

As many times, then, as E measures Z?, so many units let
there hem E;

therefore E also measures £> according to the units in E.

[vii. 16]

Therefore E by multiplying E has made 2?, [vit. Def. 15]

But, further, A by multiplying B has also made D ;
therefore the product of E, E is equal to the product of A, B.

But, if the product of the extremes be equal to that of the
means, the four numbers are proportional ; [vn. 19]

therefore, as £ is to A, so is B to E. '* ''

But A, £ are prime to one another,
numbers which are prime to one another are also the least of
those which have the same ratio, [vn. 21J

and the least numbers of those which have the same ratio
with them measure those which have the same ratio the same

jalL BOOK VII [vii. n, a. 5

number of times, the greater the greater iind the less the less,
that is, the antecedent the antecedent and the consequent the
consequent ; [vn. ao]

therefore £ measures B.

But it also measures C ;
therefore £ measures B, C which are prime to one another :
which is impossible. [vn. Def. i*]

Therefore no number will measure the numbers C, D.

Therefore C D are prime to one another.
\end{proof}

\begin{annotations}

1. their product. A ii viitSir ivtbpunt, literal); `` the (number) pradoced rrom ihem,''
will liencefart)) be translated as ``tjieir product.''

\end{annotations}

\begin{notes}

If n, b a both prime to c, then ab, c are prime to one another.
The proof is again by reduciis ad absurdum.

If ab, c are not prime to one another, let them be measured by a nd be
equal to md, mi, say, respectively.

Now, since a, c are prime to one another and d measures e,

a, d are prime to one anbther. [vn. 33]

But, since ab = md,

d;a = b:m. [vn. 19]

Therefore [vit. *o] d msures *,

or b =pd, say. . 1 ,.

But e = ttd.

Therefore d measures both i and c, which are therefore not prime to one
another : which is impossible.

Therefore etc     " '`` ;!j

\end{notes}

\end{proposition}

\begin{proposition}
\label{prop:VII_25}

\begin{statement}
If two numbers bt prime to one another, the product of one
of them into itself will be prime to the remaining one.
\end{statement}

\begin{proof}

Let A, B be two numbers prime to one another, . .,
and let A by multiplying itself make C: „, „ j

I say that B, C are prime to one another.

For let D be made equal to A.

Since A, B are prime to one another,
and A is equal to D,
therefore D, B are also prime to one another.

Therefore each of the two numbers D, A is
prime to B ;
therefore the product of D, A will also be prime to B. [vn. 34]

But the number which is the product of D, A is C
Therefore C, B are prime to one another.
\end{proof}

\begin{annotations}

1. the product of one of them into Itself. The Greeks h in raO hfht a6r«iw ytrfurott
literatlj 'the number produced from the one of them/' leaves '* multiplied into it«lf to be
understood.

\end{annotations}

\begin{notes}

If a, i are prime to one another, - v     .' >'   '

a' is prime to i. v 1 - 1

Euclid takes d equal to a, so that d, a are both prime to .

Hence, by vii. 24, da, i.e. a', is prime to i.

The proposition is a particular case of the preceding proposition ; and the
method of proof is by substitution of different numbers in the result of that
proposition.

\end{notes}

\end{proposition}

\begin{proposition}
\label{prop:VII_26}

\begin{statement}
If two numbers ie prime to two numbers, both to each, their-
products also will be prime to one another.
\end{statement}

\begin{proof}

For let the two numbers A, B he prime to the two
numbers C, D; both to each,

and let A by multiplying B  q

make E, and let C by multi- g t,

plying Z* make ./;

I say that E, F are prime to p

one another.

For, since each of the numbers A, B is prime to C,
therefore the product oi A, B will also be prime to C. [vn. *4]

But the product of A, B is E ; >

therefore B, C are prime to one another. , ,

For the same reason , - m..-; /.

Et D are also prime to one another. ., . .-n- u

Therefore each of the numbers C, D is prime to E.

Therefore the product of C, D will also be prime to E.

[vii. *4]

But the product of C, D is F.

Therefore E, Fa.Tc prime to one another.
\end{proof}

\begin{notes}

If both a and i are prime to each of two numbers e, d, then ai, cd will be
prime to one another.

Since a, b are both prime to c,

ab, t are prime to one another. [vn. 34]

Similarly ab, d are prime to one another.

Therefore c. d are both prime to ab,

and so theiefoK is ed- [vn. 34]

\end{notes}

\end{proposition}

\begin{proposition}
\label{prop:VII_27}

\begin{statement}
If two numbers be prime to one another, and each by
multiplying itself make a certain number, the products V)iU be
prime to one another; and, if the original numbers by multi-
plying the products make certain numbers, the latter will also
be prime to one another [and this is always the case with the
extremes
\end{statement}

\begin{proof}

Let A, B be two numbers prime to one another,
let A by multiplying itself make C, and by
multiplying C make D,
and let B by multiplying itself make E, and
by multiplying E make F;
I say that both C, E and D, F are prime
to one another.

For, since A, B are prime to one another,
and v4 by multiplying itself has made C,
therefore C, B are prime to one another. [vn. 25]

Since then C, B are prime to one another,
and B by multiplying itself has made B,
therefore C, E are prime to one another. [«£)

Again, since A, B are prime to one another,
and B by multiplying itself has made E,
therefore A, E are prime to one another. id:

Since then the two numbers A, C are prime to the two
numbers S, E, both to each,

therefore also the product of A, C is prime to the product of
B, E. [vn. 26]

And the product of A, C is D, and the product of B, E
is F.

Therefore D, F are prime to one another.
\end{proof}

\begin{notes}

If a, b are prime to one another, so are a',  and so are a*, ; and,
generally, a'', i'' are prime to one another.

The words in the enunciation which assert the truth of the proposition for
any powers are suspected and bracketed by Heiberg because (i) in jripl rots
Bitpout the use of ojtpoi is peculiar, for it can only mean `` the last products,''
and (*) the words have nothing corresponding to them in the proof, much
less ts the generalisation proved. Campanus omits the words tn the enuncia'
tion, though he adds to tht proof a remark that the proposition is true of any,
the same or different, powers of «, i. Heiberg concludes that the words are
an interpolation of date earlier than Theon.

Euclid's proof amounts to this.

Since a, i are prime to one another, so are a*, * [vii. a si. and therefore
also a', ff. [vii, as]

Similarly [vii. as] «,  are prime to one another.

Therefore a, a* and b, *  satisfy the description in the enunciation of
VII. a 6.

Hence a',  are prime to one another. .  .  r >  > ...,..,,,

\end{notes}

\end{proposition}

\begin{proposition}
\label{prop:VII_28}

\begin{statement}
If two numbers be prime to one another, the sum will also
be prime to each of tkem ; and, if the sum of two numbers be
prime to any one of them, ike original numbers will also be
prime to one another.
\end{statement}

\begin{proof}

For let two numbers AB, BC prime to one another be
added ;

I say that the sum AC s also prime a''   ~S 6

to each of the numbers AB, BC.

D

For, if CA, AB are not prime to . ,

one another,

some number will measure CA, AB.

Let a number measure them, and let it be Z).
Since then D measures CA, AB,

therefore it will also measure the remainder BC. ,  |

But it also measures BA ;

therefore D measures AB, BC which are prime to one another :
which is impossible. [vii, Def. u]

Therefore no number will measure the numbers CA, AB;
therefore CA, AB are prime to one another.
For the same reason

AC, CB are also prime to one another.

Therefore CA is prime to each of the numbers AB, BC.

Again, let CA, AB he prime to one another ;
I say that AB, BC are also prime to one another.

For, if AB, BC are not prime to one another,
some number will measure AB, BC.

330 BOOK VII [vn. zS, 19

Let a number measure them, and let it be />.

Now, since Z? measures each of the numbers AB, BC, it
will also measure the whole CA.

But it also measures AB ;
therefore D measures CA, AB which are prime to one another:
which is impossible. [vii. Def. la]

Therefore no number will measure the numbers AB, BC.

Therefore AB, BC are prime to one another.
\end{proof}

\begin{notes}

If a, b are prime to one another, a-¥6 will be prinm to both a and b ; and
conversely.

For supjwse (a + *), a are not prime to one another. They must then
have some common measure d.

Therefore d also divides the difference (a + li) - a, or b, as well as a ; and
therefore a, h are not prime to one another : which is contrary to the
hypothesis.

Therefore a + i is prime to a.

Similarly b + * is prime to . , , , 1,

The converse is proved in the same way.

Heibei remarks on Euclid's assumption that, if c measures both a and b,
it also measures a±b. But it has already (vji. i, j) been assumed, more
generally, as an axiom that, tn the case supposed, c measures a±J>b.

\end{notes}

\end{proposition}

\begin{proposition}
\label{prop:VII_29}

\begin{statement}
Any prime number is prime to any number which it does
not measure.
\end{statement}

\begin{proof}

Let  be a prime number, and let it not measure B ;
I say that B, A are prime to one another.

For, if B, A are not prime to one a

another, — b

some number will measure them. c

Let C measure them.

Since C measures jff,
and A does not measure B,
therefore C is not the same with A.

Now, since C measures B, A,
therefore it also measures A which is prime, though it is not
the same with it :

which is impossible.

Therefore no number will measure B, A.
Therefore A, B are prime to one another.
\end{proof}

\begin{notes}

If a is prime and does not mtsasure b, tht:n a, b axa priiiiu lo onu anuthur.
The proof is self-evident.

\end{notes}

\end{proposition}

\begin{proposition}
\label{prop:VII_30}

\begin{statement}
If two numbers by muliiplying one another utake some
number, and any prime number measure the product, it will
also measure one of the original numbers.
\end{statement}

\begin{proof}

For let the two numbers A, B by multlplvirig one another
make C, and let any prime number

D measure C ;  ,

I say that D measures one of the b

numbers A, B. c

For let it not measure A. x)

Now D is prime ; e

therefore A, D are prime to one

another. [vu. 29J

And, as many times as D measures C, so many units let
there be in .£''.

Since then D measures C according to the units in E,
therefore D by multiplying E has made C. [vk. IJef, 15]

Further, A by multiplying B has also made C
therefore the product of D, E is equal to the product of
A, B.

Therefore, as /? is to , so is .5 to E. [vu, 19]

But D, A are prime to one another,
primes are also least, [vu. zi]

and the least measure the numbers which have the same
ratio the same number of times, the greater the greater and
the less the less, that is, the antecedent the antecedent and
the consequent the consequent ; , -, [vu. 20]

therefore D measures B.

Similarly we can also show that, if D do not measure B,
it will measure A.

Therefore D measures one of the numbers A, B.
\end{proof}

\begin{notes}

If IT, a primt; number, measure ai, e will measure either a ur i.
Suppose c does not measure a.

Therefore , a are prime to one another. [vii, 19]

Suppcrae ab-me.

Therefore e:a-b:m. .- 1;   . 'i.i. >'< [vu. 19]

Hence [vii. JO, 2i] fmeasuresi,  -' ``

Similarly, if c does not measure l>, it measures c.
Therefore it measures one or other of the two numbers a, k

\end{notes}

\end{proposition}

\begin{proposition}
\label{prop:VII_31}

\begin{statement}
Any compost ie number is measured by some prime number.
\end{statement}

\begin{proof}

Let $A$ be a composite number ; . . '

I say that A is measured by some prime number.

For, since A is composite, '
S some number will measure it A

Let a number measure it, and let it b- —

be B. c—

Now, if B is prime, what was en-
joined will have been done.
10 But if it is composite, some number will measure it

Let a number measure it, and let it be C

Then, since C measures B,
and .5 measures .,  :-, . ''      '

therefore C also measures >(4.

IS And, if C is prime, what was enjoined will have been
done.

But if it is composite, some number will measure it
Thus, if the investigation be continued in this way, some
prime number will be found which will measure the number
a) before it, which will also measure A.

For, if it is not found, an infinite series of number will
measure the number A, each of which is less than the other;

which is impossible in numbers.

Therefore some prime number will be found which will
as measure the one before it, which will also measure A.

Therefore any composite number is imasured by some
prime number.
\end{proof}

\begin{annotations}

8. if B Is prime, what was enjoined vrlll have been done, i.e. the implied
prsiiim of finding n prime number which measures A*

18. some prime number will be found which will measure. In the Creek the
sentence stops here, but it U necessary to add the words `` the number before it, which will
also measure j4,'' which are found a few lines further down. It is possible that the words
ma have fidlen out of P here by a simple mistake due to biitnariivtae (Heiberg).

\end{annotations}

\begin{notes}

Heiberg relegates to the Appendix an alternative proof of this proposition,
to the following effect. Since A is composite, some number will measure it.
Let B be the least such number. I say that 3 is prime. For, if not, B is
composite, and some number will measure it, say C; so that C is less than B.
But, since C measures S, and B tneasures A, C must measure A. And C is
less than B -. which is contrary to the hypothesis,

\end{notes}

\end{proposition}

\begin{proposition}
\label{prop:VII_32}

\begin{statement}
Any number either is prime or is measured by some prime number,
\end{statement}

\begin{proof}

Let .4 be a number;
I say that A either is prime or is measured by some prime
number. , ,

If now A IS prime, that which was a '

enjoined will have been done.

But if it is composite, some prime number will measure it

[vii. 31]

Therefore any number either is prime or is measured by
some prime number.

Q, E. D,
\end{proof}

\end{proposition}

\begin{proposition}
\label{prop:VII_33}

\begin{statement}
Gwen as many numbers as we p/ease, to find the least of
those which have the same ratio with them.
\end{statement}

\begin{proof}

Let A, B, C be the given numbers, as many as we please ;
thus it is required to find the least of
s those which have the same ratio with
A, B, C. *

A, B, C are either prime to one
another or not.

Now, i A, B, C are prime to one I I
10 another, they are the least of those y I
which have the same ratio with them,

[vii. 3i]

But, if not, let D the g-eatest common measure of ., B, C
be taken, [vii. 3]

and, as many times as D measures the numbers A, B, C
IS respectively, so many units let there be in the numbers
£, F, G respectively.

Therefore the numbers E, F, G measure the numbers A,
B, C respectively according to the units in D. yu. 16]

Therefore E, E, G measure A, B, C the same number of
30 times ;

therefore E, E, G are in the same ratio with A, B, C.

[vii. Drf. 30]

I say next that they are the least that are in that ratio.
For, if E, E, G are not the least of those which have the
same ratio with A, B, C,

'S there will be numbers less than E, F, G which are in the

same ratio with A, B, C.

Let them )x. H, K, L ; .-..-.

therefore H measures A the same number of times that the

numbers K, L measure the numbers B, C respectively.
30 Now, as many times as H measures A, so many units let

there be in M;

therefore the numbers K, L also measure the numbers B, C
rpectively according to the units in M.

And, since H measures A according to the units in M,

35 therefore M also measures A according to the units in H.

[vn. 16]
For the same reason

M also measures the numbers B, C according to the units in
the numbers K, L respectively ;

; - Therefore M measures A, B, C.
40 Now, since H measures A according to the units in M,
therefore H by multiplying M has made A. [vn. Def. 15]

For the same reason also ,i .

E by multiplying D has made A.

Therefore the product oi E, D\ equal to the product of
4S H,M.

Therefore, as  is to H, so is M to D. [vn. 19]

But E is greater than H ;

therefore M is also greater than D,
It And it measures A, B, C

which is impossible, for by hypothesis D is the greatest
common measure of A, B, C.

Therefore there cannot be any numbers less than E, F, G
which are in the same ratio with A, B, C.

Therefore F, F, G are the least of those which have the
js same ratio with A, B, C.

rviT i.iii'w i,tTr.m nefH ``.n* ! r-;( uJ jwt.,-
\end{proof}

\begin{annotations}

17. the numbeis E, F, C measure the numbers A, B, C respectively,
literally (as usual) ``eacli of the numl>er» £, F, G measures each of the numbers A,

\end{annotations}

\begin{notes}

Given any numbets a, h, c, .,,, to find the least numbers that are in the
same ratio.

Euclid's method is the obvious one, and the result is verified by reductio
ad absurdum.

We wi!t, like Euclid, take three numbers only, a, h, c.
Letf, their greatest common measure, be found [vn. 3], and suppose that
a = mg, i.e. gm, , v> . . [vn. 16]

6 = ug, U.g«. ,j ,.

': = />i. I.e. . X , , ..„ -;

It follows, by vn. Def. 20, that  v 1 .

4 1  t .-* n :[w

H m : H  .p = a : i :e, j . . i

«, », / shall be the numbers required.

For, if not, let x, y, e be the least numbers in the same ratio as a, b, r,
being less than /«, n, p.

Therefore a = kx (or xk, vn. 16),

i> = ky (at yk), \ i,l -S

c = kz (or zk),

where k is some integer. [vii. ao]

Thus «lj(*= a = xk.

Therefore ``' 'i '' `` m:x = k:g. .fj rj'.! [vii. 19]

And m x; therefore k-> g.

Since then k measures a, A, c, it follows that g is not the greatest comaon
measure ; which contradicts the hypothesis.

Therefore etc. ``  '- -ifr-'I'T

It is to be observed thai Euclid merely supposes that x, y, g are smaller
numbers than m, n, p in the ratio of ab, c\ but, in order to justify the next
inference, which apparently can only depencl on vn. ao, x, y, t must also be
assumed to be the hait numbers in the ratio of «, b, c.

The inference from the last proportion that, since m> x, i >is supposed
by Hetberg to depend upon vit. 13 and v. 14 together. I prefer to regard
Euclid as making the inference quite independently of Book v. E.g., the
proportion could just as well be written

X : m=g : k, ''

when the definition of proportion in Book vn. (Def. 20) gives all that we want,
since, whatever proper fraction x is of m, the same proper fraction is g of k.

\end{notes}

\end{proposition}

\begin{proposition}
\label{prop:VII_34}

\begin{statement}
Given two numbers , to find the least number which th
measure.
\end{statement}

\begin{proof}

Let A, B be the two given numbers ;
thus it is required to find the least number which they
measure,

Now, are either prime to one  B

another or not.

First, let A, B be prime to one

another, and let A by multiplying B

make C;

therefore also B by multiplying A has .

made C, : .  [vil 16]

Therefore A, B measure C

I say next that it is also the least number they measure.

For, if not, A, B will measure some number which is less
than C

Let them measure D. , „ .

Then, as many times as  measures D, so many units let
there be in £,

and, as many times as B measures D, so many units let there
be in f;

therefore A by multiplying £ has made D,

and B by multiplying / has made Z? ; [vii. Def, 15]

therefore the product of , j6'' is equal to the product of ,5, F.

Therefore, as A is to B, so is / to £. [vn, 19]

But A, B are prime,
primes are also least, [vn, ti]

and the least measure the numbers which have the same ratio
the same number of times, the greater the greater and the less
the less ; [vn. 10]

therefore B measures B, as consequent consequent.

And, since A by multiplying B, E has made C. D,
therefore, as B is to E, so is C to Z>. [vn. 17]

But B measures E
therefore C also measures D, the greater the less :
which is impossible.

Therefore A, B do not measure any number less than C ;
therefore C is the least that is measured by A, B.

Next, let , B no*'' be prime to one another,
and let F, E, the least numbers of those which have the same
ratio with A, B, be taken ; [vil 33]

therefore the product of -,  is equal to the product of , F.

[vii. 19]

And let A by multiplying E " m,i«7>- ,

make C ; * b

therefore also B by multiplying F p e

has made C ;

therefore A, B measure C. d ,,

I say next that it is also the least h

number that they measure.

For, if not. A, B will measure some number which is less
than C.

Let them measure D.

And, as many times as A measures D, so many units let
there be in G,

and, as many times as B measures D, so many units let there
be in H.

Therefore A by multiplying G has made D,
and B by multiplying H has made D.

Therefore the product of A, G is equal to the product of
B,H;
therefore, as A is to B, so is H to G. [vil 19]

I, But, as y4 is to B, so is F to E. ,i, ..

Therefore also, as / is to £, so is li to G. ,

But F, E are least, ,j, ,

and the least measure the numbers which have the same ratio

the same number of times, the greater the greater and the

less the less ;   [vn. to]

therefore E measures G.

And, since A by multiplying E, G has made C, /?,

therefore, as £ Is to G, so is C to D. [vn. 17]

But £ measures G ; itj v .»».

therefore C also measures D, the greater the less :

which is impossible.

Therefore A, B will not measure any number which is less
than C.

Therefore C is the least that is measured by A, B.
\end{proof}

\begin{notes}

This is the problem of finding the itait common multipk of two numbers,

as a, b. .. , .

I, If «,  be prime to one another, the l.cm. is ab, ,

For, if not, let it be rf, some number less than aA
Then d-ma-nb, where wi, n are inters.

Therefore a.b-n:m, [vil. rg]

and hence, a, b being prime to one another,

b measures m. [vii. ao, »i)

But bM = iA;aM [vti. 17]

= ab:d.
Therefore ah measures d: which is impossible.

' TT. If a, i be not prime to one another, find the numbers which are the
least of those having the ratio of a to b, say «, « ; [vit. 33]

then a: bm -.n,

and an-bm (=f, say); [vir, 19]

e is then the i.c.m.

For, if not, let it be i/ (< c), so that

ap-bq = d, where/, q are integers.

Then a:6 = q:p, [vii. 19]

whence m : n = q -.p,

so that n measures/. [vii. 20, ai]

And n : p = an  ap = c : d,

90 that e measures d:

which is impossible. ``

Therefore etc.

By VII

33>

m
n

4

~~S

b
 s

Hence the

I.C.M

. is

ab

S

, where £ is the g.c.m. of a, b.

\end{notes}

\end{proposition}

\begin{proposition}
\label{prop:VII_35}

\begin{statement}
If two numbers measure any number, ike least number
measured by tkem will also measure the same.
\end{statement}

\begin{proof}

For let the two numbers A, B measure any number CD,

and let E be the least that they

measure; * p

I say that E also measures CD.

For, if E does not measure
CD, let E, measuring DF, leave C/less than itself.

Now, since A, B measure E, '

and E measures DF,
therefore A, B will also measure DF.

But they also measure the whole CD ;
therefore they will also measure the remainder CF which is
less than E:

which is impossible. - '.

Therefore E cannot fail to measure CD ; ,-i- ,.., -.j
therefore it measures it.
\end{proof}

\begin{notes}

The \emph{least} common multiple of any two numbers must measure any other
common multiple.

The proof is obvious, depending on the fact that, if any number divides »
and b, it also divides a -ph. i n. ;

\end{notes}

\end{proposition}

\begin{proposition}
\label{prop:VII_36}

\begin{statement}
Given three numbers, to find the least number which they
measure.
\end{statement}

\begin{proof}

Let A, B, C be the three given numbers ;
thus it is required to find the least . ,

number which they measure, a '``' .'

Let D, the least number mea- b

sured by the two numbers A, B, c —

be taken. [vn. 34] d-

Then C either measures, or e I'l

does not measure, D. , . ,

First, let it measure it.

But A, B also measure D\     . .-- - . - — t

therefore A, B, C measure D.

I say next that it is also the least that they measure.

For, if not, A, B, C will measure some number which is
less than D.

Let them measure E.

Since A, B, C measure E, '
therefore also A, B measure E.

Therefore the lease number measured by A, B will also
measure E. [vn. 35]

But D is the least number measured by , Z? ;
therefore D will measure E, the greater the less :
which is impossible.

Therefore A, B, C will not measure any number which is
less than D ;

therefore D is the least that A, B, C measure.

Again, let C not measure D, '

and let E, the least number measured by .

C, D, be taken. [vn. 34]

Since A, B measure D, q

and D measures E, '

therefore also A, B msure E, 1

But C also measures E ; - '

therefore also A, B, C measure E,

1 say next that it is also the least that they measure.
For, if not. A, B, C will measure some number which
is less than E.

Let them measure E. ;.rs

Since A, B, C measure E,

therefore also ,  measure /;

therefore the least number measured by A, B will also
measure E. [vn. 35]

But D is the least number measured by A,*B ;
therefore D measures F. , .,v -;

But C also measures /; ejiii-i.v ..''.

therefore A C measure ./
so that the least number measured by D, C will also measure E.

But £ is the least number measured by C, D;
therefore £ measures J, the greater the less :
which is impossible.

Therefore A, B, C will not measure any number which is
less than £.

Therefore £ is the least that is measured by A, B, C
\end{proof}

\begin{notes}

Euclid's rule for finding the ucm. of ihrte numbers a, j,  is the rule with
which we are familiar. The L.CI1. of a, b is first found, say d, and then the
L.C.H. of d and c is found.

Euclid distinguishes the cases (i) in which c measures d, (i) in which c
does not measure d. We need only reproduce the proof of the general case
(3). The method is that of rtducfio ad asurdum.

Let e be the L.C.M, of d, c.

Since a, i both measure d, And d measures «, '' `` '

a, b both measure e.

So does c.

Therefore e is iome common multiple of a, b, c.

If it is not the ioj, let/be the L.CM.

Now a, b both measure/;
therefore d, their L.aM., also measures/ [vu, 35]

Thus d, ( both measure/
therefore e, their l.c.m,, measures/: : ,, .  [vii. 35]

which is impossible, since /< e.

Therefore etc.

The process can be continued ad libitum, so that we can find the L.C.M.,
liot only of threes but of as many numbers as we please.

\end{notes}

\end{proposition}

\begin{proposition}
\label{prop:VII_37}

\begin{statement}
If a number be measured by any number, the number which
is measured will have a part called by the same name as the
measuring number.
\end{statement}

\begin{proof}

For let the number A be measured by any number B
I say that A has a part called by the same
name as B, a

For, as many times as B measures A, b

so many units let there be in C.

Since B measures A according to the p

units in C,

and the unit D also measures the number C according to the

units in it,
therefore the unit D measures the number Cthe same number
of times as B measures A.

Therefore, alternately, the unit D measures the number B
the same number of times as C measures A ; [vii. 15]

therefore, whatever part the unit D is of the number B, the
same part is C of - also.

But the unit D is a. part of the number B called by the
same name as it ;

therefore C is also a part of A called by the same name as B,
so that A has a part C which is called by the same name as B.
\end{proof}

\begin{notes}

If 6 measures a, then 7 th of ii is a whole number.

Let a-m.b.

Now m = m.x.

Thus It m, b, a satisfy the enunciation of vti. 15 ;
therefore m measures a the same number of times that i measures b.

But I is T th part of b ; ''

theiefore w is r th part of a.

\end{notes}

\end{proposition}

\begin{proposition}
\label{prop:VII_38}

\begin{statement}
If a number have any part whatever, it will be measured
by a number called by the same name as the part.
\end{statement}

\begin{proof}

For let the number A have any part whatever, B,
and let C be a number called by the same
name as the part B ;
I say that C measures A. * '

For, since  is a part of A called by

the same name as C,

and the unit D is also a part of C called
by the same name as it,

therefore, whatever part the unit D is of the number C,

the same part s B o( A also ;

therefore the unit D measures the number C the same number
of times that B measures A.

 Therefore, alternately, the unit D measures the number B
the same number of times that C measures A. [vit. 15]

Therefore C measures A.

n I . J >i/i 3d'
\end{proof}

\begin{notes}

This proposition is practically a. restatement of the preceding proposition.
It asserts that, if * is - th part of «, ,.,.,

i.e.,if <5 = -a, . J -  '-'i !

m . . ,

then m measures a,

We have . .; . b=-~ a, . ,

and !=—«(.

Therefore i, m, i, a, satisfy the enunciation of vit. 15, and thet«foTe m
measures n the same number of times as i measures i, or

I

' , n( = 1 a. . .. . . 1-

r. i   - '`` ``

\end{notes}

\end{proposition}

\begin{proposition}
\label{prop:VII_39}

\begin{statement}
To find the number which is the least that will have given
parts.
\end{statement}

\begin{proof}

Let A, B, C be the given parts ;
thus it is required to find the number which is the least thai
will have the parts A, B, C.

A B c,

D

Let D, £, F be numbers called by the same name as the
parts A, B, C,

and let G, the least number measured by D, E, 7 be taken.

[vii. 36]

Therefore G has parts called by the same name as D, E, F.

[vn. 37]
But A, B, C are parts called by the same name as Z>, E, F
therefore G has the parts A, 3, C.

I say next that it is also the least number that has.

For, if not, there will be some number less than G which
will have the parts A, B, C.

Let it be H. m - :n' *

Since H has the parts A, By C,
therefore H will be measured by numbers called by the same
name as the parts A, B, C. fvu. 38]

But D, E, F are numbers called by the same name as the
parts A, B, C
therefore H is measured by D, E, F.

And it is less than G : which is impossible.

Therefore there will be no number less than G that will
have the parts A, B, C.
\end{proof}

\begin{notes}

This again is practically a restatement in another form of the problem of
finding the L.C.M.

To find a number which has - th, t th and - th parts.

Let d be the l.c.m. of a, d, c.

Thus d has -eh, rth and -th parts. [vti. 37]

If it is not the least number which has, let the least such number be < .

Then, since e has those parts,
e is measured by a, 6,e; and e<d:
which is impossible.

\end{notes}

\end{proposition}

\part{Book VIII}

\begin{proposition}
\label{prop:VIII_1}

\begin{statement}
If there be as many numbers as we please in continued
proportion, and the extremes of them be prime to one another,
the numbers are the least of those which have the same ratio
with them.
\end{statement}

\begin{proof}

Let there be as many numbers as wk please, A B, C, D,
in continued proportion,

and let the extremes of them a- ~ g —

A, Dx. prime to one another; b f

I say that A, B,C, D are the Q

least of those which have the `` h

same ratio with them.

For, if not, let E, F, G, H h less than A, B, C, D, and
in the same ratio with them.

Now, since A, B, C, D are in the same ratio with E, F.
G,H,

and the multitude of the numbers A, B, C, D is equal to the
multitude of the numbers E, F, G, H,
therefore, ex aequali,

as A is to D, so is E to H. [vii. 14]

But A, D are prime,
primes are also least, [vii. ai]

and the least numbers measure those which have the same
ratio the same number of times, the greater the greater and
the less the less, that is, the antecedent the antecedent and
the consequent the consequent. [vn. ao]

Therefore W measures £, the greater the less :
which is impossible.

Therefore E, F, G, H which are Jess than A B, C, D
are not in the same ratio with them.

Therefore A, B, C, D are the least of those which have
the same ratio with them.
\end{proof}

\begin{notes}

What we call a geometrical progression is with Euclid a series of terms ``in
continued proportion `` (ifijt o'lutAo).

This proposition proves that, if a, f, <:,... ji are a series of numbers in
geometrical progression, and if o, k are prime to one another, the series is in
the lowest terms jxjible with the same common ratio.

The proof is in femi by redudio ad absurdum. We should no doubt
desert .form while retaining the substance. If «', iJ', c', . . . A' be any other
series of numbers in c.p. with the same common ratio as before, we have,
tx atquali,

a : k = a' : k', [vii. 14]

whence, since a, k are prime to one another, a, k measure a', k' respectively, so
that a', k' are greater than a, k respectively.

\end{notes}

\end{proposition}

\begin{proposition}
\label{prop:VIII_2}

\begin{statement}
To find numbers in continitd proportion, as many as may
be prescribed, and Ike least that are in a given ratio.
\end{statement}

\begin{proof}

Let the ratio of 4 to  be the given ratio in least
numbers ;

thus it is required to find numbers in continued proportion,
as many as may be prescribed, and the least that are in the
ratio of A to j9.

-0

Let four be prescribed ;
let A by multiplying itself make C, and by multiplying B let
it make D

let B by multiplying itself make E ;
further, let A by multiplying C, D, E make F, G, H,
and let B by multiplying E make K.

Now, since A by multiplying itself has made C, `` *'

and by multiplying B has made /?,
therefore, as A is to B, so is C to /?. [vii. 17]

Again, since A by multiplying B has made D,

and  by multiplying itself has made £,

therefore the numbers A, B hy multiplying £ have made the
numbers D, E respectively.

 'r'l

Therefore, as  is to B, so is D to E. [vii. 18]

But, as y4 is to .5, so is C to Z? ;
therefore also, as C is to D, so is D to E.

And, since A by multiplying C, D has made F, G,
therefore, as C is to D, so is F to C [vn. 17]

But, as C is to D, so was A to B
therefore also, as  is to B, so is F to G,

Again, since A by multiplying /?,  has made G, If,
therefore, as ZJ is to E, so is G to If'. [vii, 17]

But, as i? is to £'', so is /4 to B.

Therefore also, as A is to B, so is G to H.

And, since .4,  by multiplying E have made H, K,
therefore, as A is to B, so is H to . [vii, 18]

But, as A is to B, so is / to G, and (7 to jfiT.

Therefore also, as F is to G, so is G to H, and H io K;
therefore C, Z?, £'', and Z G, H, K are proportional in the
ratio of A to B.

I say next that they are the least numbers that are so,

For, since A, B are the least of those which have the
same ratio with them,

and the least of those which have the same ratio are prime
to one another, , ,,1 [vn. aa]

therefore A, B are prime to one another.

And the numbers A, B hy multiplying themselves re-
spectively have made the numbers C, E, and by multiplying
the numbers C, E respectively have made the numbers F, K
therefore C, E and F, A'are prime to one another respectively,

[vn. 27]

But, if there be as many numbers as we please in continued
proportion, and the extremes of them be prime to one another.
they are the least of those which have the same ratio with
them. [viii. i]

Therefore C, D, E and F, G, H, K are the least of those
which have the same ratio with A, B.
\end{proof}

\begin{porism*}
From this it is manifest that, if three numbers
in continued proportion be the least of those which have the
same ratio with them, the extremes of them are squares, and,
if four numbers, cubes.
\end{porism*}

\begin{notes}

To find a series of numbers in geometrical progression and In the least
terms which have a given common ratio (understanding by that term ilu ratio
of one term to the next).

Reduce the given itio to its lowest terms, say, a : i. (This can be done
by VII. 33.)

Then a'', a*-'i, a'-lr', ... a'*''-', u-', i-

is the required series of numbers if (« + i ) terms are required.

That this is a series of terms with the given common ratio is clear from
vit. 17, i8.

That the G.P. is in the smallest terms possible is proved thus.
- a, 6 are prime to one another, since the ratio a : i is in its lowest terms.

[vii. 2»]

Therefore o'', `` are prime to one another ; so are «'',  and, generally,
*'', .  [vil. tj]

Whence the g.p. is in the smallest possible terms, by viu. i.

The Porism observes that, if there are h terms in the series, the
extremes are («- i)th powers.

\end{notes}

\end{proposition}

\begin{proposition}
\label{prop:VIII_3}

\begin{statement}
If as many numbers as we please in continued proportion
be the least of those which have the same ratio mith them, the
extreines of them are prims to one anot/ter.
\end{statement}

\begin{proof}

Let as many numbers as we please, A, B, C, D, m con-
tinued proportion be the least of those which have the same
ratio with them ;

 i-i <

— E — F :> .  1

— O H K

-L M N

I say that the extremes of them A, D are prime to one
aaother.

For let two numbers E, F, the least that are in the ratio
o A, B, C, D, be taken, [vii. 33]

then three others G, H, K with the same property ;

and others, more by one continually, [vm. 2]

until the multitude taken becomes equal to the multitude of
the numbers A, B, C, D.

Let them be taken, and let them be L, M, N, 0.

Now, since E, F are the least of those which have the
same ratio with them, they are prime to one another, [vti. 22]

And, since the numbers E, F by multiplying themselves
respectively have made the numbers G, K, and by multiplying
the numbers G, K respectively have made the numbers L, O,

[vin. z, For.]
therefore both G, /f and L, O are prime to one another, [vii. 27]

And, since A, B, C, D are the least of those which have
the same ratio with them,

while Z,, Mi N, O are the least that are in the same ratio with
A, B, C, D,

and the multitude of the numbers A, B, C, D is equal to the
multitude of the numbers L, M, N, O,

therefore the numbers A, B, C D are equal to the numbers
Li M, N, O respectively ;

therefore A is equal to L, and D to O.
And Z., O are prime to one another,
 Therefore A, D are also prime to one another.
\end{proof}

\begin{notes}

The proof consists in merely equating the given numbers to the terms of
a series found in the manner of viii. 2.

i a, b,c, ... k (n terms) be a geometrical progression in the lowest terms
having a given common ratio, the terms must respectively be of t>e form

found by viii. 2, where a : j9 is the ratio a  6 expressed in its lowest terms, so

that a, J8 are prime to one another [vn. 21], and hence «"', fi*~ are prime

to one another [vii. 27], ., .

But the two series must be the same, so that

\end{notes}

\end{proposition}

\begin{proposition}
\label{prop:VIII_4}

\begin{statement}
Given as many ratios as we please in leasi numbers, to find
numbers in continued proportion which are ike least in the
given ratios.
\end{statement}

\begin{proof}

Let the given ratios in least numbers be that of A to B,
s that of C to D, and that q( E to F
thus it is required to find numbers in continued proportion
which are the least that are in the ratio of A to B, in the
ratio of C to D, and in the ratio of E to F,

A— B  '

D

E F

«— 5

H

M !i

P L

Let G, the least number measured by B, C, be taken.

in And, as many times as B measures G, so many times also
let A measure ,

and, as many times as C measures G, so many times also let
D measure /C,

Now E either measures or does not measure K.
15 First, let it measure it.

And, as many times as E measures A*, so many times let
E measure L also.

Now, since A measures If the same number of times that
jff measures G,
*> therefore, as y is to B, so is If to G. [vn. Def. io, vil 13]

For the same reason also,

as C is to D, so is G to K,
and further, as  is to /, so is A' to Z. ;
therefore If, G, K, L are continuously proportional in the
«s ratio of A to B, in the ratio of C to D, and in the ratio of E
XoF.

I say next that they are also the least that have this
property.

For, if H, G, K, L are not the least numbers continuously
30 proportional in the ratios of A to B of C to D, and of B
to F, let them be N, 0, M, P. :-  j »., :

Then since, as A is to B, so is N to O,
while ,  are least,

and the least numbers measure those which have the same
35 ratio the same number of times, the greater the greater and
the less the less, that is, the antecedent the antecedent and the
consequent the consequent ;

therefore B measures 0. [vn. ao]

' , .1

For the same reason
40 C also measures O; ;

therefore B, C measure O ;

therefore the least number measured by B, C will also
measure O. [vii. 35]

But G is the least number measured by J9, C ;
45 therefore G measures O, the greater the less :
which is impossible.

Therefore there will be no numbers less than H, G, K, L
which are continuously in the ratio of A to B, of C to D, and
oiBxaF.
Sf> Next, let £ not measure J.

A

B

Q —

E

F

.,, , 1 ,

0-

H

t 1 <

K —

M

N —
P —

Let M, the least number measured by B, K, be taken.

And, as many times as K measures M, so many times let
H, G measure A', respectively,

and, as many times as B measures M, so many times let F
ss measure P also.

Since H measures A'' the same number of times that G
measures O,
therefore, as /T is to 6, so is A to O, [vn. 13 and Def, so]

But, as  is to 6, SO is /4 to  ;
60 therefore also, as A is to B, so is A' to O. 1 io 1 . 1

For the same reason also,

as C is to D, so is O to M.
Again, since  measures  the same number of times that
F measures P,
65 therefore, as  is to / so is  to /* ; [vu. 13 and Def, zo)

therefore A', O, M, P are continuously proportional in the
ratios of A to B, of C to /?, and of E to F,

I say next that they are also the least that are in the ratios
A:B, C:D. E:F.
70 For, if not, there will be some numbers less than A, O,
M, P continuously proportional in the ratios AB, C.D,
E:F , ,

Let them be Q, R, S, T.
Now since, as Q is to R, so is A to B,
75 while A, B arc least,
and the least numbers measure those which have the same
ratio with them the same number of times, the antecedent the
antecedent and the consequent the consequent, [vii. jo]

therefore B measures R.
80 For the same reason C also measures R ;
therefore B, C measure R.

Therefore the least number measured by B, C will also
measure R. [vii. 35]

But G is the least number measured by , C
is therefore G measures R.

And, as G is to R, so is A'' to ,'' : [vii. 13J

therefore K also measures S.

But E also measures S;
therefore E, K measure S,
90 Therefore the least number measured by E, K will also
measure S. [vn. 35]

But M is the least number measured by E] K
therefore M measures 5, the greater the less :
which is impossible.
9S Therefore there will not be any numbers less than A', O,
Mt P continuously proportional in the ratios of A to B, of
C to D, and of E to F;

therefore N, O, M, P are the least numbers continuously
proportional in the ratios A:B,C:D,E:F.
\end{proof}

\begin{annotations}

So, 71, 09< the ratios A : B, C : D, B : F. TKu mbbreruted expression is in the
Greek dI AB, TA, EZ V-

\end{annotations}

\begin{notes}

The terms ``in continued proportion'' is here not used in its proper sense,
since a geometrical progression is not meant, but  series of terms each of
which brs to the succeeding term a given, but not the same, ratio.

The proposition furnishes a good example of the cumbrousness of the
Greek method of dealing with non-determinate numbers. The proof in fact
is not easy to follow without the help of modern symbotical notation. If
this be used, the reasoning can be made clear enough.

Euclid takes three given ratios and therefore requires to tsAfour numbers.
We will leave out the simpler particular case which he puts first, that nameljr
in which B accidentally measures K, the multiple of D found in the first few
lines ; and we will reproduce the general case with Mr« ratios.

Let the ratios in their lowest terms be

;  a:i,e:d,e:/ .Tjiir

Take li, the l.c.u. of i, e, and suppose that , ; \ , .'

/, = mi — ne.
Form the numbers ma, mli , rut, 1 . . ,.

= nc\   I

These are in the ratios of ji to d and of <r to rf respectively.  .)' 'I
Next, let /, be the l.c,m. of nd, e, and let ; ,'. J i

4 =/«rf = qe.
Now form the numbers

pma, pmb \ , ptid \ , qf
=pfie f =ge I
and these are the four numbers required.

If they are not the least in the given ratios, let

f' y> « , >\ ... . V. .l.o'->i1|
be less numbers in the given ratios. . ,

Since ii : i is in its lowest terms, and '``"  '-"' '``

a \ b~x : y

i measures V. ,h , .

Similarly, since e : a= : z,

t measures > ,

Therefore /j, the L.C.M. of l>, f, measures >.

But /, xnd~e:dy:

Therefore nd measures z.

And, since e :f~z ; u,

e measures z.

Therefore 4i the L.C.M. of nd, e, measures z : which is impossible, since
i</j or pnd.

The step (line 86) inferring that G : H = K ; S 'n of course alternando
from G:K[=C: D\ = Ji . S.

It will be observed that viii. 4 corresponds to the portion of vi, 33 which
shows how to compound two ratios between straight lines.

>i. i-('A

-:> ti.H

X.

  .-ll.ii

[ r'i

  'h

\end{notes}

\end{proposition}

\begin{proposition}
\label{prop:VIII_5}

\begin{statement}
Plane numbers have to one another the ratio compounded
of the ratios of their sides.
\end{statement}

\begin{proof}

Let -4,  be plane numbers, and let the numbers C, D
be the sides of A, and E, F oi B
s I say that A has to B the ratio com- g
pounded of the ratios of the sides. -.

For, the ratios being given which C e — p
has to E and D to F, let the least q
numbers G, HyKiht are continuously
10 in the ratios CE,D:Fh taken, so
that,

as C is to E, so is G to Hy
and, as D is to F, so is H to K. [viii. 4]

And let D by multiplying E make L.
IS Now, since D by multiplying C has made A, and by
multiplying £'' has made L,
therefore, as C is to -£'', so is  to Z,. ,. [vii. ij]

But, as C is to , so is (7 to If ;
therefore also, as G is to H, so is A to L.
20 Again, since E by multiplying Z? has made L, and further
by multiplying F has made ,
therefore, as /? is to /s so is Lkq B. - [vii. 17]

But, as Z? is to F so  H to K
therefore also, as H is to K, so is L to .
*s But it was also proved that,

as fz is to H, so is  to Z. ;
therefore, ex aeguali,

as G is to K, so is A to B. [vn. 14]

But G has to K the ratio compounded of the ratios of the
30 sides ;
therefore A also has to B the ratio compounded of the ratios
of the sides.
\end{proof}

\begin{annotations}

I, 5, 19, 31. compounded of the ratios of Ibeir sides. As in v[. 13, the Greek
has the less exact phitue, `` cornpounded of their sides.''

\end{annotations}

\begin{notes}

If a = ed, b = tf,

then a has to b the ratio compounded oi c le and d :/

Take three numbers the least which are continuously in the given ratios.

If / is the L.CII. of e, d and l=mt = nd, the three numbers are

m/:, me , nf. [vill. 4]

~nd)
Now dc:dt=cc [vii. 17]

= mc : me - me \ nd.
Also td:tf=df [vii. 17]

= ttdnf.
Therefore, ft* atguali, cd .ef=mc : nf

= (ratio compounded of 1: . e and d ;/).
It will be seen that this proof follows exactly the method of vi. 23 for
parallelograms.

\end{notes}

\end{proposition}

\begin{proposition}
\label{prop:VIII_6}

\begin{statement}
If there be as many numbers as we please in continued
proportion, and the first do not measure the second, neither
will any other measure any other.
\end{statement}

\begin{proof}

Let there be as many numbers as we please. A, B, C, D, E,
in continued proportion, and let A not measure B ;
I say that neither will any other measure any other.

-F
— a
H

Now it is manifest that A, B, C, D, E do not measure
one another in order ; for A does not even measure B.

I say, then, that neither will any other measure any other.

For, if possible, let A measure C.

And, however many A, B, C are, let as many numbers
F, G, H, the least of those which have the same ratio with
A, B, C, be taken. [vn. 33]

Now, since F, G, H are in the same ratio -ith A, S, C,
and the multitude of the numbers A, B, C is equal to the
multitude of the numbers F, G, H,
therefore, ex aequali, as A is to C, so is F to H. [vil 14]

And since, as A is to , so is 7 to G, a -       »>
while A does not measure B,

therefore neither does F measure G ; [vu. Def. ao]

therefore F is not an unit, for the unit measures any number.

``om F, H are prime to one another. [vm. 3]

And, as F is to H, so is  to C ;
therefore neither does A measure C

Similarly we can prove that neither will any other measure
any other.
\end{proof}

\begin{notes}

Let ab,c...kx, geometrical progression in which a does not measure h.

Suppose, if possible, that a measures some term of the series, as /

Take x,y, x, u, v, w the itcat numbers in the ratio a, b, c, d, e,f.

Since x  .y = a:b,

and a does not measure b, '

X does not measure ; therefore x cannot be unity.

And, ex aequali, x : w = a :/.

Now X, w are prime to one another. [viii. 3]

Therefore a does not measure/

We can of course prove that an intermediate term, as b, does not measure
a later term / by using the series b, e, d, e, f and remembering that, since
b -.c-a; b, b does not measure c.

\end{notes}

\end{proposition}

\begin{proposition}
\label{prop:VIII_7}

\begin{statement}
If there be as many numbers as we please in continued
proportion, and the first measure the last, it tvUl measure the
second also.
\end{statement}

\begin{proof}

Let there be as many numbers as we please, A, B, C, D,
in continued proportion ; and
let A measure D ; *

I say that A also measures B,

For, if A does not measure '

B, neither will any other of the ° — —''

numbers measure any other. [vm. 6]

But A measures D.

Therefore A also measures B.  ,''
\end{proof}

\begin{notes}

An obvious proof by redudia ad absurdum from vm. 6,

\end{notes}

\end{proposition}

\begin{proposition}
\label{prop:VIII_8}

\begin{statement}
If between two numbers there fall numbers in continued
proportion with them, then, however many numbers fall between
them in continued proportion, so many will also fall in con-
tinued proportion between the numbers which have the same
ratio with the original numbers.
\end{statement}

\begin{proof}

Let the numbers C, D fall between the two numbers A,
B in continued proportion with them, and let E be made in
the same ratio to  as  is to  ;

I say that, as many numbers as have fallen between A., B in
continued proportion, so many will also fall between E, F in
continued proportion.

A e

c M

O N

B F- ~-

G

"  K

L   - - ``

For, as many as A, B, C, D are in multitude, let so many
numbers G, H, K, L, the least of those which have the same
ratio with A, C, D, B, be taken ; [vn, 33]

therefore the extremes of them (9j L are prime to one another.

[vin. 3]

Now, since A, C, D, B are in the same ratio with G, H,
K,L,

and the multitude of the numbers A, C, D, B is equal to the
multitude of the numbers G, H, K, L,
therefore, ex aequali, as A is to B, so is G to L. [vii. 14]

But, as .(4 is to B, so is E to E;
therefore also, as G is to Z, so is  to A '

But G, L are prime, ``'~

primes are also least, [vn. si]

and the least numbers measure those which have the same
ratio the same number of times, the greater the greater and
the less the less, that is, the antecedent the antecedent and the
consequent the consequent. [vn. o]

Therefore G measures S the same number of times as L
measures F.

Next, as many times as G measures E, so many times let
/f, K also measure M, N respectively ;
therefore <7, H, K, L measure E, M, N, F the same number
of times.

Therefore G, H, K, L are in the same ratio with E, M,
N, F, [vn, Def. 20]

But G, H, K, L are in the same ratio with A, C, D, B ;
therefore A, C, D, B are also in the same ratio with E, M,
N, F ,

But A, Ct D, B are in Continued proportion ;
therefore E, M, N, FAre also in continued proportion.

Therefore, as many numbers as have fallen between A, B
in continued proportion with them, so many numbers have also
fallen between E, F in continued proportion,
\end{proof}

\begin{annotations}

t. blL The Gi«k word is iiarhmir, `` Ml in `` = ``can be
interpolited.''

\end{annotations}

\begin{notes}

If a:6 = e:/, and between a, b there are any number of geometric
means t, d, there will be as 'many such means between f, /.

Let 0, fi, y,   , S be the least possible terms in the same ratio as a,
c, d,...b.

Then o, 8 are prime to one another, [vtu. 3]

and, ex atquali, a:  = ab

= €:f.
Therefore t = ima,ys: m , where m is some inter. [vii. 10]

t Take the numbers ma, mjS, my, ... mS.

This is a series in the given ratio, and we have the same number of
geometric means between ma, mS, or e,/, that there are between a, b.

\end{notes}

\end{proposition}

\begin{proposition}
\label{prop:VIII_9}

\begin{statement}
If two numbers be prime to one another, and numbers fall
between them in continued proportion, then, however many
numbers fall between them in continued proportion, so many
will also fall between each of them and an unit in continued
proportion.
\end{statement}

\begin{proof}

Let A, B be two numbers prime to one another, and let
C, D fall between them in continued proportion,
and let the unit E be set out ;
I say that, as many numbers as fall between A, B in continued proportion, so many wi]l also fall between either of
the numbers A, B and the unit in continued proportion.

For let two numbers F, Gy the least that are in the ratio
of A, C, D, B, be taken,

three numbers H, K, L with the same property,

and others more by one continually, until their multitude is
equal to the multitude of , C, D, B. [vni. a]

A

H

ic    '

D

L

B

E-
F-
Q—

n -1

.

P

Let them be taken, and let them be M, N, O, P.

It is now manifest that F by multiplying itself has made
H and by multiplying If has made M, while G by multiplying
itself has made L and by multiplying L has made P.

[viii. 3, Por.]

And, since M, N, O, P are the least of those which have
the same ratio with F G,

and A, C, D, B are also the least of those which have the
same ratio with F, G, [viii. i]

while the multitude of the numbers M, N, C, P is equal to the
multitude of the numbers A, C, D, B,

therefore M, N, O, P are equal to A, C, £>, B respectively ;

therefore Af is equal to A, and P to B. ,,.

Now, since Fhy multiplying itself has made If, -  '-

therefore F measures If according to the units in F.

But the unit F also measures F according to the units in it;

therefore the unit F measures the number F the same number
of times as F measures If,

Therefore, as the unit F is to the number F, so is F to ff.

[vii. Def. »o]

Again, since F by multiplying If has made M,
therefore If measures M according to the units in F, v«.>

But the unit £ also measures the number F according to
the units in it ;

therefore the unit B measures the number F the same number
of times as If measures M.

Therefore, as the unit £ is to the number F,sosIf to M.

But it was also proved that, as the unit F is to the number
F, so is / to / ;

therefore also, as the unit F is to the number F, so is F to If,
and If to M.

But Jtf is equal ja A
therefore, as the unit E is to the number /'', so is F to H,
and H to A.

For the same reason also,
as the unit E is to the number G, so is C to Z and L to B.

Therefore, as many numbers as have fallen between A,
B in continued proportion, so many numbers also have fallen
between each of the numbers A, B and the unit E in continued
proportion,
\end{proof}

\begin{notes}

Suppose there are n geometric means between a, b, Iwo numbers prime to
one another ; there are the same number (n) of geometric means between i
and a and between i and b.

If c, d... are the n means between a, b,

a, i, d ... b
are the least numbers in that ratio, since a, b are prime to one another, [viii. ij

The terms are therefore respectively identical with

a''*', a-, d'-'jS' ... Q, iS''*',

where o,  is the common ratio in its lowest terms. [vni. a. For.]

Thus tf = <i''+', * = «+>.

Now I Lffl = a I a' = a' : q'... =a» :a''*',

and I:j8 = j3: = :j9'...=j8-;j8-*-;

whence there are « geometric means between i, a, and between i, b.

\end{notes}

\end{proposition}

\begin{proposition}
\label{prop:VIII_10}

\begin{statement}
If numbers fall between each of two numbers and an unit
in continued proportion, however many numbers fall between
each of tkem and an unit in continued proportion, so many
also will fall between the numbers themselves in continued
proportion.
\end{statement}

\begin{proof}

For let the numbers D, E and F, G respectively fall
between the two numbers A, B and the unit C in continued
proportion ;

I say that, as many numbers as have fallen between each of
the numbers A, B and the unit C in continued proportion, so
many numbers will also fall between A, B in continued pro-
portion.

For let D by multiplying F make H, and let the numbers
D, F by multiplying Ji make K, L respectively.

.  i.'

c —

D—

E —

F

Q

A-

H

K

L

I t

Now, since, as the unit C is to the number D.sois D to £,

therefore the unit C measures the number D the same number
of times as D measures E. [vn. Def. 10]

But the unit C measures the number £f according to the
units in D ;

therefore the number D also measures E according to the units
inZ?;

therefore I? by multiplying itself has made E.

Again, since, as C is to the number D, so is E to A,

therefore the unit C measures the number £> the same number
of times as E measures A,

But the unit C measures the number D according to the
units in D ;

therefore E also measures A according to the units in D ;

therefore D by multiplying E has made A. - '

For the same reason also

F by multiplying itself has made G, and by multiplying G has
made B.

And, since D by multiplying itself has made E and by
multiplying F has made If,

therefore, as Z> is to F, so is E to If. [vil 17]

3«a BOOK Vm [vin. lo

For the same reason also, -: .. '

as D is to /  so is If to C  `` [vii. i8]

Therefore also, as £ is to N, so h N to G.

Again, since D by multiplying the numbers £, H has
made A, K respectively,
therefore, as E is to H, so is  to A'. [vii. 17]

But, as E is to H, sos D 10 E;
therefore also, as D is to E, so is A to K.

Again, since the numbers D, E by multiplying H have
made K, L respectively,
therefore, as D is to F, so is K to L, [vn. 18)

But, as D is to F, q 'v A to K
therefore also, as 4 is to K, so is K to L.

Further, since F by multiplying the numbers If, G has
made L, B respectively,
therefore, as H is to G, so is L to B. [vil 17]

But, as If is to G, so is D to A;
therefore also, as /? is to 7  so is Z, to A     '

But it was also proved that,

as D is to E, so is  to  and K to L
therefore also, as A is to K, so is A'' to Z and L to B.

Therefore A, K, L, B are in continued proportion.

Therefore, as many numbers as fall between each of the
numbers A, B and the unit C In continued proportion, so
many also will fall between A, Bu. continued proportion.
\end{proof}

\begin{notes}

If there be n geometric nutans between i and a, and also between i and
i, there will be « geometric means between a and b.

The proposition is the converse of the preceding.

The n means with the extremes form two geometric series of the fonn
I, (I, a' ... a'', 0''+',
I, j8, ;8'..., /8-+'.
where a'' = n, j3''+' = b.

By multiplying the last term in the first line by the first in the second,
the last but one in the first line by the second in the second, and so on, we
get the series

and we have the n means between a arid b.

It will be observed that, when EucUd says `` Fsr tke sami naion also, as
i? is to /J so is .ff to G,'' the reference is really to vn. iS instid of vii. 17.

He infers narady that Dx F: Fx F=D : F. But since, by vn, 16, the

order of multiplication is indifferent, he is practically justified in saying `` for
the same reason.'' The same thing occurs in later propositions.

\end{notes}

\end{proposition}

\begin{proposition}
\label{prop:VIII_11}

\begin{statement}
Between two square numbers there is one mean proportional
number, and the square has to the square the ratio duplicate
of that which the side has to the sieU. , ,
\end{statement}

\begin{proof}

Let A, By square numbers, `` '

and let C be the side of W, and D o( B;

I say that between A, B there is one mean proportional

number, and A has to B the ratio

duplicate of that which C has to D. a

For let C by multiplying D make E. b ~

Now, since .< is a square and C is o

its side, £

therefore C by multiplying itself has ,

made A. . . . -i

For the same reason also   t'

D by multiplying itself has made B.

Since then C by multiplying the numbers C, D has made
A, E respectively,
therefore, as C is to /?, so is  to E, , [vii. 1 7]

For the same reason also,

as C is to /?, so is  to .  [vn, iS]

Therefore also, as A is to E so is E to B.

Therefore between A, B there is one mean proportional
number.

I say next that A also has to B the ratio duplicate of
that which C has to D,

For, since A, E, B are three numbers in proportion,

therefore A has to B the ratio duplicate of that which A has
to E. [v. Def. 9]

But, as A is to E, so is C to D,

Therefore A has to B the ratio duplicate of that which
the side C has to D.
\end{proof}

\begin{notes}

According to Nicomachus the theorems in this proposition and the next,
that two squares have one geometric mean, and two cubes two geometric
means, betweerf them are Platonic. Cf. TimamSt 32 a sqq. and the note
thereon, p. 294 above.

a*,  being two squares, it is only necessary to form the product ai and
to prove that

a'', ai,
are in geometrica) progression. Euclid proves that

 '. ai = ai : ff'
by means of vii. 17, 18, as usuaL

In assuming that, since a* is to  in the duplicate ratio of a* to at, a* is
to P in the duplicate ratio of a to 6, Euclid assumes that ratios which are
the duplicates of equal ratios are equal. This, an obvious inference from
V. 33, can be inferred just as easily for numbers from vii. 14.

'.1 ;i>lr *n» ``'J J Jirf iWli.

\end{notes}

\end{proposition}

\begin{proposition}
\label{prop:VIII_12}

\begin{statement}
Between two cube numbers there are two mean proportional
numbers, and tlie cube has to the cube the ratio triplicate of that
which the side has to the side.
\end{statement}

\begin{proof}

Let A, B> cube numbers,
and let C be the side of A, and D oi B;
I say that between A, B there are two mean proportional
numbers, and A has to B the ratio triplicate of that which C
has to /?.

A-

ttt K

' '.'''

C —

 A

D

K

.»

For let C by multiplying itself make £'', and by multiplying
D let it make F;

let D by multiplying itself make G,

and let the numbers C, D by multiplying F make H, K
respectively.

Now, since . is a cube, and C its side, - ... « « li

and C by multiplying itself has made E,
therefore C by multiplying itself has made E and by multiply-
ing E has made A,

For the same reason also
D by multiplying itself has made G and by multiplying G has
made B.

And, since C by multiplying the numbers C, D has made
E, /respectively,
therefore, as C is to D, so is E to F, ....   [vu. ij]

For the same reason also,

as C is to /?, so is . to G. [vii. iSJ

Again, since C by multiplying the numbers E, F has
made A, H respectively,
therefore, as E is to F, so is A to H, [vii. 17]

But, as >£' is to F, so is C to /?.

Therefore also, as C is to D, so is A to If.

Again, since the numbere C, D by multiplying F have
made H, K respectively,
therefore, as C is to D, so is H to K, [vii. 18]

Again, since D by multiplying each of the numbers F, G
has made K, B respectively,
therefore, as F is to G, so is K to B. [vii, 17]

But, as / is to G, so is C to Z? ;
therefore also, as C is to /?, so is Ava H,H to K, and K to B.

Therefore H, K are two mean proportionals between A, B.

1 say next that A also has to B the ratio triplicate of that
which C has to D,

For, since A, H, K, B are four numbers in proportion,
therefore A has to B the ratio triplicate of that which A has
to H, [v. Def. 10]

But, as y is to Ht so is  to Z* ;
therefore A also has to B the ratio triplicate of that which C
has to D.
\end{proof}

\begin{notes}

The cube numbere if, fi being given, Euclid forms the products nb, a
and then proves, as usual, by means of vii, 17, 18 that
are in continued proportion.

He assumes that, since a* has to  the ratio triplicate of o* : <b, the
ratio a' :  is triplicate of the ratio a : b which is equal to a* : (?b. lliis
is again an obvious inference from vii. 14.

\end{notes}

\end{proposition}

\begin{proposition}
\label{prop:VIII_13}

\begin{statement}
If there be as many numbers as we please in continued
proportion, and each by multiplying itself make some number,
the products  mill be proportional ; and, if the original numbers
by multiplying the products make certain numbers, the latter
will also be proportional.
\end{statement}

\begin{proof}

Let there be as many numbers as we please, A, B, C, in
continued proportion, so that, as -(4 is to Ji, so is /? to C;

let A, B, Chy multiplying themselves make D, E, F, and by
multiplying D, E, /let them make G, H, K

I say that D, E, F and G, H, K are in continued proportion.

A

a

B

H

K

M

N

p

Q

> For let A by multiplying B make L,   .'i ''

and let the numbers A, B by multiplying L make M. N
respectively.

And again let B by multiplying C make O, ''

and let the numbers B, C by multiplying O make P, Q
respectively.

Then, in manner similar to the foregoing, we can prove
that

D, L, E and G, M, N, H are continuously proportional in the
ratio of A to B,

and further £, O, F and H, P, Q, K are continuously propor-
tional in the ratio of B to C.

Now, as A is to B so is  to C;

therefore D, L, E are also in the same ratio with E, O, F,

and further G, M, N, H in the same ratio with H, P, Q, K.

And the multitude of D, L, E is equal to the multitude of

E, O. F, and that of G, M, N, H to that of H, P,Q,K;

therefore, ex acgualt,

as D is to -£'', so is £  to F, '''' '
and, as tr is to H, so is H to K. [vn. 14]
\end{proof}

\begin{notes}

U a,i, c ...he a. series in geometrical progression, then

J J 13 J f '' *l '`` geometncal progression.

Heiberg brackets the words added to the enunciation which extend the
theorem to any powers. The words are ``and this always occurs with the
extremes `` (jtai a<i -rtpi tov( ixpov; toGto tnififlairu). They seem to be rightly
suspected on the samt: grounds as the same words added to the enunciation
of viL 27. There is no allusion to them iii the proof, much less any proof
of the extension.

Euclid forms, besides the squares and cules of the given numbers, the
products ad, tfi, atf, be, tc, ic\ When he says that `` we prove in manner
similar to the foregoing,'' he indicates successive uses of vii. 17, iS as
in VIII. 13.

With our notation the prcx)f is as easy to sec for any powers as for squares
and cubes.

To prove that n'', , <*,.. are in geometrical progression, `` ``'``' '

Form all the means between a', *'', and set out the series
a*, a'-'/', a'-'lr ... air-', ff.
The common ratio of one term to the next is a ; /S.

Next take the geometrical progression

*''. tr- i'-V ... Ifc-\ i'', v

the common ratio of which i i : c.

Proceed thus for all pairs of consecutive terms, '

Now a : 6 = fi : c= ...

Therefore any pair of succeeding terms in one scries are in the same ratio as
any pair of succeeding terms in any other of the series. ,.,

And the number of terms in each is the same, namely (h + i).

Therefore, m atgiialiy

\end{notes}

\end{proposition}

\begin{proposition}
\label{prop:VIII_14}

\begin{statement}
1/ a square measure a square, the side will also measure
the side ; and, if the side measure the side, the square will also
measure the square.
\end{statement}

\begin{proof}

Let A, Bh square numbers, let C, D be their sides, and
let A measure B;
I say that C also measures D. A

For let C by multiplying D make E ; fl

therefore A, E, B are continuously pro- —  ``

portional in the ratio of C to D. [vm. nj e

And, since A, E, B are continuously
proportional, and A measures B, ``

therefore A also measures E.  '``- [vm, 7]

And, as A is to E, so is C-io D\ " '
therefore also C measures Z?, [vii. Def, ao]

Again, let C measure D\  ``

I say that A also measures B.

For, with the same construction, we can in a similar
manner prove that A E, B are continuously proportional in
the ratio of C to Z?.

And since, as C is to D, so is A to E
and C measures D,
therefore A also measures E. [vii. Def. ao]

And A, E, B are continuously proportional ;

therefore A also measures B.

Therefore etc.
\end{proof}

\begin{notes}

If «* measures , a measure l> ; and, if « measures b, a* measures P.
(1) *i', ab, i° are in continued proportion in the ratio of a to b.

(viii. 7]

Therefore, since

0' measures ,

..,-.

a' measures ab.

But

d : ab = a \ b.

Therefore

a measures b.

(2) since a measures b, a* measures ab.

And a', ab, P are continuously proportional.
Thus ab measures ,

And a' measures ab.

Therefore «' measures i''.

It will be seen that Euclid puts the last step shortly, saying that, since
(j measures ab, and a*, ab,  are in continued proportion, a* measures *.
The same thing happens in viii. 15, where the series of terms is one more
than here.

\end{notes}

\end{proposition}

\begin{proposition}
\label{prop:VIII_15}

\begin{statement}
If a cube number measure a cube number, the side will also
measure the side ; and, if the side measure the side, the cube
will also measure the cube.
\end{statement}

\begin{proof}

For let the cube number A measure the cube B,
and let C be the side of A and D oi B
I say that C measures D.

For let C by multiplying itself make E,
and let D by multiplying itself make G ;
further, let Cby multiplying D make f, '``*» --tt .-;, . >'>
and let C, D by multiplying F make H, K respectively,

A

o—  ; K  '

E — '

°ir~ ``---  '

Now it is manifest that E, F, G and A, H, K, B are
continuously proportional in the ratio of Cto D. [viii. 11, u]

And, since A, H, K, B are continuously proportional,
and A measures B,
therefore it also measures H. c- > ,   ["  ']

And, as y4 is to /f, so is C to Z? ; 1 ,. ..
therefore C also measures D. [vii, Dct ao]

Next, let C measure D ; '`` " `` ''' `` ''

I say that A will also measure B.

For, with the same construction, we can prove in a similar
manner that A, If, K, B are continuously proportional in the
ratio of C to D.

And, since C measures D,
and, as C is to D, so is A to H,

therefore A also measures H [vii, Def. ao]

so that A measures B also,
\end{proof}

\begin{notes}

If 0* measures V, a measures b \ and via versa. The proor is, mutatii
mutandis, the same as for squares.

(i) (f,a'6,ai,itK continuously proportional in the ratio of a to * ;
and a* measures .

Therefore « measures a*i ; [viii. 7]

and hence a measures i, ``'  -fi. >(  1

(i) Since a measures i, a* measures ci*j.

And, 0*, a*3, aJ*,  being continuously proportional, each term measures the
succeeding term ;
therefore a* measures .

\end{notes}

\end{proposition}

\begin{proposition}
\label{prop:VIII_16}

\begin{statement}
If a square numder do not measure a square number, neither
will the side measure the side ; and, if the side do not measure
the side, neither will the square measure the square.
\end{statement}

\begin{proof}

Let A, B be square numbers, and let C, Z? be their sides ;
and let A not measure B ;

I say that neither does C measure D. *

For, if C measures D, A will also

measure B. [vni. 14] c

But A does not measure B ; d

therefore neither will C measure D.

Again, let C not measure Z? ; . . '

I say that neither will A measure B.

For, if A measures B, C will also measure D. [vin. 14)

But C does not measure D ;
therefore neither will  measure B.
\end{proof}

\begin{notes}

If a' does not measure , a mil not measure b; and, if a does not
measure h t? will not measure .

The proof is a mere rtduciio ad absurdum using vin. 14.

\end{notes}

\end{proposition}

\begin{proposition}
\label{prop:VIII_17}

\begin{statement}
If a cube number do not measure a cube numher, neither
will the side measure the side ; and, if the side do not measure
the side, neither will the cube measure the cube.
\end{statement}

\begin{proof}

For let the cube number A not measure the cube
number B,

and let C be the side of A, and D a

of;

1 say that C will not measure D.

For if C measures Z), A will
also measure B. [viii. 15]

But A does not measure B ;
therefore neither does C measure D,

Again, let C not measure D ;
I say that neither will A measure B.

For, ( A measures B, C will also measure D. [vm. 1 5]
But C does not measure D ;
therefore neither will A measure B.
\end{proof}

\begin{notes}

If a* does not measure , a will not measure i ; and viee versa.
Proved by reducHo ad absurdum employing viii. 15,

\end{notes}

\end{proposition}

\begin{proposition}
\label{prop:VIII_18}

\begin{statement}
Between two similar plane numbers there is one mean
proportional number ; and the plane number has to the plane
number the ratio duplicate of that which the corresponding
side has to the corresponding side.
\end{statement}

\begin{proof}

Let A, B be two similar plane numbei, and let the numbers
C, D be the sides of A, and E, F of B.

K O

B D

E

, . . F

Now, since similar plane numbers are those which have
their sides proportional, [vii. Def. «i]

therefore, as C is to D, so   E x.o F.   '' '

I say then that between A, B there is one mean propor-
tional number, and A has to B the ratio duplicate of that
which C has to E or D to F, that is, of that which the corre-
sponding side has to the corresponding side.

Now since, as C is to D, so is E to F,
therefore, alternately, as C is to E, so is D to F. [vn, 13)

And, since A is plane, and C, D are its sides,
therefore D by multiplying C has made A.

For the same reason also
E by multiplying F has made B.

Now let D by multiplying E make G.

Then, since D by multiplying C has made A, and by
multiplying E has made G,
therefore, as C is to E, so is A to G. [vn. 17]

But, as C is to £'', SO is Z? to /;  `` '

therefore also, as /? is to J, so is A to G. ''*'

Again, since £ by multiplying D has made G, and by
multiplying F has made B,

therefore, as Z> is to /  so is C to B. '`` \ `` [vii. 17]

But it was also proved that,

as Z' is to F, so  A to G;
therefore also, as A is to (7, so is G to B.

Therefore A, G, B are in continued proportion.
' Therefore between A, B there is one mean proportional
number.

I say next that A also has to B the ratio duplicate of
that which the corresponding side has to the corresponding
side, that is, of that which C has to £'' or Z> to F.

For, since A, G, B are in continued proportion,

A has to B the ratio duplicate of that which it has to G.

[v. Def. 9]

And, as A is to G, so is C to E, and so is D to F.
Therefore A also has to B the ratio duplicate of that which
C has to £  or Z? to F.
\end{proof}

\begin{notes}

i ab, <rf be `` similar plane numbers,'' i.e. products of factors such that
a  b = c ; d,

thciie is one mean proportional between ai and at; and ab 'n to ed m the

duplicate ratio of a to 1: or of  to .

Fonn the product be (or ad, which is equal to it, by vii. 19).

Then ab, be] , cd

ab, be,
= ad)'

is a series of terms in geometrical progression.
For a : b=c   d.

Therefore a:c = b:d. [v''. 13]

Therefore ab : be = be : cd. [vil. 17 and 16]

Thus be (or ad) is a geometric mean between ah, cd. ,

And ab is to<rfin the duplicate ratio of ab to be or of be to <rf, that is, of
d to f or of  to d.

\end{notes}

\end{proposition}

\begin{proposition}
\label{prop:VIII_19}

\begin{statement}
Between two similar solid numbers there fall two mean
proportional numbers; and the solid number has to the similar
solid number the ratio triplicate of that which the corresponding
side has to the corresponding side.
\end{statement}

\begin{proof}

Let j4, B he two similar solid numbers, and let C, D, E
be the sides of A, and F, G, H of B.

Now, since similar solid numbers are those which have
their sides proportional, [vu. Def. si]

therefore, as C is to /?, so is Fxa G, : I'l : ';

and, as D is to , so Is 6 to ff.

I say that between A, B there fall two mean proportional
numbers, and A has to B the ratio triplicate of that which C

has to F, D to G, and also E to H.

  \ .

A-

B

C- F- N-

0- o-

E— H

K—

I

M

For let C by multiplying D make K, and let F by
multiplying G make X.
 > Now, since C, D are in the same ratio with F, G,

and /C is the product of C, V, and L the product of F, O,
K, L are similar plane numbers ; [vn, Def. ai]

therefore between K, L there is one mean proportional number.

. , . [vin. 18]

 Let it be   '`` - `` '

Therefore M is the product of D, F, as was proved in the

theorem preceding this. [vm. 18]

Now, since D by multiplying C has made K, and by

multiplying  has made M,

therefore, as C is to / so \ K Xa M. [vn. 17]

But, as K is to M, so is M to L.

Therefore K, M, L are continuously proportional in the
ratio of C to F,

And since, as C is to D, so is F to G,
alternately therefore, as C is to /  so is Z? to G. [vii. 13]

For the same reason also,

as D is to G, so is E to H.

Therefore K, M, L are continuously proportional in the
ratio of C to F in the ratio of D to G, and also in the ratio
of E to H.

Next, let E, H by multiplying M make A, O respectively.

Now, since  is a solid number, and C, D, E are its sides,
therefore E by multiplying the product of C, D has made A.

But the product of C, D is K;  .   - lis.- ;

therefore E by multiplying K has made A.   m=

For the same reason also '

H by multiplying L has made B.

Now, since E by multiplying A' has made A, and further
also by multiplying M has made N

therefore, as K is to M, so is j4 to N. [vii. 17]

But, as K is to , so is C to F, D to £7, and also E to If ;

therefore also, as C is to Z', Z? to G, and EtoH,ia A to A',

Again, since £'',  by multiplying M have made A, O
respectively,

therefore, as  is to If, so is N to O. [vii. 18]

But, as  is to H, so is C to / and D Xa G
therefore also, as C is to /  Z? to G, and E to H,a  A to
iV and A to a

Again, since If by multiplying has made f?, and further
also by multiplying L has made B,

therefore, as  is to Z,, so is f? to B, [vil 17]

But, as  is to Z, so is C to F, D to G, and  to H.
Therefore also, as C is to F, D to G, and E io H, so not

only is t? to , but also A to N and A'' to O.

Therefore A, N,0, B are continuously proportional in the

aforesaid ratios of the sides.

I say that A also has to B the ratio triplicate of that which
the corresponding side has to the corresponding side, that is,
of the ratio which the number C has to F, or D to G, and
also E to H.

For, since A, N, O, B are four numbers in continued
proportion,

therefore A has to B the ratio triplicate of that which A has
to N, [v. Def. ro]

But, as y is to A'', so it was proved that C is to F, D to (?,
and also E to H.

Therefore  also has to  the ratio triplicate of that which
the corresponding side has to the corresponding side, that is,
of the ratio which the number C has to F, D to G and also
E to H,
\end{proof}

\begin{notes}

In other words, M a:b : c=d : e :/, then there are two geometric means
between abc, def; and abc is to def in the triphcate ratio of a to d, or b to e,
or c ,of

Euclid first takes the plane numbers ab, dt (leaving out e, f) and foroos
the product bd. Thus, as in viu. 18,

ab, bd\ , de
-eaj
are three terms in geometrical progression in the ratio of a to d, or of i to e.

He next forms the products of f,/ respectively into the mean bd.

Then afic, ebd, fid, def

are in geometrical progression in the ratio of a to  etc.

Fm abc

bd
fbd

: cbdab ; bd=a : d'

.fbd = c:f -. [vii. 17]

.def=bd:de=b:e ]
And a : d-b : t = c -.f.

The ratio of abc to def is the ratio triplicate of tliat of abc to ebd, ie. of
that of a ta d etc.

\end{notes}

\end{proposition}

\begin{proposition}
\label{prop:VIII_20}

\begin{statement}
If <me mean proportional number fall between two numben,
the numbers will be similar plane numbers.
\end{statement}

\begin{proof}

For let one mean proportional number C fall between the
two numbers A, B
5 1 say that A, B are similar plane numbers.

Let D, E, the least numbers of those which have the same
ratio with A, C, be taken ; [vii. 33]

therefore D measures A the same number of times that B
measures C. [vu. no]

10 Now, as many times as D measures A, so many units let
there be in ; r

therefore F by multiplying Z? has made A,
so that A is plane, and D, F are its sides.

Again, since />, £ are the least of the numbers which have
IS the same ratio with C, B,
therefore D measures C the same number of times that E
measures B. [vn. 20]

A' O

B-

F-
Q-

As many times, then, as E measures B so many units let
there be in G
ao therefore E measures B according to the units in G;
therefore G by multiplying E has made B. ``

Therefore B is plane, and E, G are its sides,

Therefore A, B are plane numbers.

I say next that they are also similar. - »

«S For, t since F by multiplying D has made At and by
multiplying E has made C,
therefore, as D is to E, so is A to C, that is, C to B. [vii. 17]

Again,t since E by multiplying F, G has made C, B
respectively,

30 therefore, as F is to G, so is C to B. [vn. 1 7]

But, as C is to .ff, so is Z? to  ;
therefore also, as D is to E, so is F to G.

And alternately, as D is to F, so is E to G. [vn. 13]

Therefore A, B are similar plane numbers; for their sides

3S are proportional.
\end{proof}

\begin{annotations}

J J. For, since F 17. C to B. The tent has clearly suffered oomiplion here. It

is not neccssaty to inftr from otfaer facts tht,  2 is to £, so \ A Ui C\ for this is part of
the hypotheses (II. 6, 7). Again, there is no enplanation of the statement (1. 55) that  by
multiplying E has made C. It is the statement and explanation of this latter fact which are
atene wanted ; after which the proof proceeds as in 1. 18. We might therefore luiHtitute for
It. J5 — »8 Ibe following,

``For, since £ measures C the same number of times that D measures A [1. 8], that is,
according to the units in F [1. loj, therefore F by multiplying E has made C.

And, since E by moltiplying F, G,'' etc. etc.

\end{annotations}

\begin{notes}

This proposition is the converse of vi 11, 18. i a,e, b are in geometrical
pnession, a, b are `` similar plane numbers.''

Let a : j3 be the ratio a i ( (and therefore also the ratio f : i) in its lowest
tenns.

Then [vn. 20]

a ~ ma, c - mfi, where m is some integer, '

f = wo, i = »ft where « is some integer.

Thus a,  are both products of two factors, i.e. plane.   , , ,
,. Again, a : fi = a : e = t : A

= m:n.  <.,.. ,, [v''. i8]

Therefore, alternately, a:m=':n,  [vii, 13]

and hence ma, «j3 are similar plane numbers.

[Our notation makes the second part still more obvious, for  =»(jS=«(x.]

\end{notes}

\end{proposition}

\begin{proposition}
\label{prop:VIII_21}

\begin{statement}
If two mean proportional numbers fali between ia)9 numbers,
the numbers are similar solid numbers.
\end{statement}

\begin{proof}

For let two mean proportional numbers C, D fall between
the two numbers A, B ;
I say that A, B are similar solid numbers.

Ill'

A E-

B F ':   '

..  c a .,/,'

D M-

,i I, . , «— . K-

O L-

M

For let three numbers E, F, G, the least of those which
have the same ratio with A, C, D, be taken ; [vii. 33 or vin. a]

therefore the extremes of them £, G are prime to one another.

[vni. 3]

Now, since one mean proportional number F has fallen
between £, G,
therefore E, G are similar plane numbers. [vm. aoj

Let, then, H, K be the sides of E, and L, M of G,

Therefore it is manifest from the theorem before this that
£ , F, G are continuously proportional in the ratio oi H to L
and that of K to M.

Now, since £, F, G are the least of the numbers which
have the same ratio with A, C, D,

and the multitude of the numbers E, F, G Is equal to the
mtiltitude of the numbers A, C, D,
therefore, ex aequali, as E is to G, so is A to /?. [vu. 14]

But B, G are prime,
primes are also least, [vii. ai]

and the least measure those which have the same ratio with
them the same number of times, the greater the greater and
the less the less, that is, the antecedent the antecedent and the
consequent the consequent ; [vn. 20]

therefore £ measures A the same number of times that G
measures £>.

Now, as many times as E measures A, so many units let
there be in TV.

Therefore JV by multiplying £ has made A.

But £ is the product of If, K ;
therefore N by muhiplying the product of J/, K has made A,
,. Therefore A is solid, and H, K, N are. its sides. , .

Again, since E, F, G are the least of the numbers which
have the same ratio as C, D, B,

therefore E measures C the same number of times that G
measures B.

Now, as many times as E measures C so many units let
there be in O.

Therefore G measures B according to the units in 0
therefore O by muhiplying G has made B.

But G is the product of L, M;
therefore O by multiplying the product of L, M has made B,

Therefore B is solid, and L. M, O are its sides ;
therefore A, B are solid.

I say that they are also similar.

For since N, O by multiplying E have made A, C,
therefore, as N is to O, so is A to C, that is, E to F. [vn. 18]

But, as E is to F, so is H 10 L and K to M
therefore also, as If is to Z, so is A'' to  and N to O.

And H, K, N are the sides of A, and O, L, M the sides
of.

Therefore A, B are similar solid numbers.
\end{proof}

\begin{notes}

The converse of viu. 19. If a, e, d, b are in geometrical progression, a, b
are ``similar solid numbers.''

Let a, ;3, y be the least nunibers in the ratio of a, , d (and therefor* also
of i-, rf, b). [vii. 33 or vui. »

Therefore a, y are prime to one another, , .,.,., [vhi. 3

They are also `` similar plane numbers.'' [vni. 30

Let a. = mn, y =pf,

where m:n-p:q.  .-  -.

Then, by the proof of viii, 20,

a :j3 = m:/ = « : .

Now, ex aegua/i, a ; J-a : y, [vil. 1 4)

and, since a, y are prime to one another,

B = m, d-ry, where r is an integer.

But a = mn

therefore a = rw«, and therefore a is `` solid.''

Again, ex aequaii, c : - a. : y,

and therefore c = ja, b = sy, where J is an integer.

Thus b = spg, and d is therefore `` solid.''

Now a ; j3 = a : e = ra: so.

= r:s. [vii. 18]

And, from above, a: fi-m : p- n : g.

Therefore r \ s = mip = » -.g,

and hence a, b are similar sohd numbers.

\end{notes}

\end{proposition}

\begin{proposition}
\label{prop:VIII_22}

\begin{statement}
/ Ae numbers be in continued proportion, and the first
be sqxtare, the third will also be square.
\end{statement}

\begin{proof}

Let A, Bt C'' be three numbers in continued proportion,
and let A the first be square ;
I say that C the third is also square. ``'

For, since between A, C there is one

mean proportional number, B,

therefore A, C are similar plane numbers. [viii. 30]

But A is square ; `` '

therefore C is also square.
\end{proof}

\begin{notes}

A mere application of viu. zo to the particular case where one of the
`` similar plane numbers `` is square.

\end{notes}

\end{proposition}

\begin{proposition}
\label{prop:VIII_23}

\begin{statement}
I/four numbers be in continued proportion, and the first be
cube, the fourth will also be cube.
\end{statement}

\begin{proof}

Let A, B, C, i? be four numbers in continued proportion,
and let A be cube ;

I say that D is also cube. *

For, since between A, D there c

are two mean proporti onalnumbers q

B, C,

therefore A, D are similar solid numbers, [vui. ai]

But is cube; 1 .,k .. r <

therefore Z? is also cube. '
\end{proof}

\begin{notes}

A mere application of viii. a i to the case where one of the `` similar solid
numbers `` is a cube. , , j

\end{notes}

\end{proposition}

\begin{proposition}
\label{prop:VIII_34}

\begin{statement}
If two numbers have to one another the ratio which a square
number has to a square number, and the first be square, the
second wiU also be square.
\end{statement}

\begin{proof}

For let the two numbers A, B have to one another the
ratio which the square number C has

to the square number D, and let j4 be A

square ; B

I say that  is also square. q

For, since C, D are square,
C, D are similar plane numbers.

Therefore one mean proportional number falls between
C, D. [viii. 18]

And, as C is to Z?, so is .(4 to 5 ;
therefore one mean proportional number falls between A, B
also. [viii. 8]

And j4 is square ; . i'-

therefore B is also square. ; , . [vm, 21]
\end{proof}

\begin{notes}

If « ;  = ; (f'', and « is a square, then b is also a square.

For f', dy have one mean proportionai (d. [vm. 18]

TheKfoie a, b, which are in the same ratio, have one mean proportional.

[vm. 8]
And, since a is square, b must also be a square. , . [vm, 33]

\end{notes}

\end{proposition}

\begin{proposition}
\label{prop:VIII_25}

\begin{statement}
If two numbers have to one another the ratio which a cube
number has to a cube number, and the first be cube, the second
will also be cube.
\end{statement}

\begin{proof}

For let the two numbers A, B have to one another the
ratio which the cube number C has to the cube number D,
and let A be cube ;
1 say that B is also cube.    1 >ii

For, since C, D are cube,

C, /? are similar solid numbers.

Therefore two mean proportional numbers fall between
D. [viii. ig]

C, D

A E-

B F-

O

And, as many numbers as fall between C, D in continued
proportion, so many will also fall between those which have
the same ratio with them ; [vm. 8]

so that two mean proportional numbers fall between A, B
also.

Let £, Fso fall.

Since, then, the four numbers A, E, F, B are in continued
proportion,
and A is cube,
therefore B is also cube. [vm. aj]
\end{proof}

\begin{notes}

ir d : b = ('id', and a is a cube, then b is also a cube.
For c', d* have two mean proportionals, [viil. 19]

Therefore a, b also have two mean proportionals, [vin. 8]

And a is a cube :
therefore £ is a cube. [vm. 33]

\end{notes}

\end{proposition}

\begin{proposition}
\label{prop:VIII_26}

\begin{statement}
Similar plane numbers have to one another the ratio which
a square numier has to a square number. , ..
\end{statement}

\begin{proof}

Let A, B be similar plane numbers ;
I say that A has to B the ratio which a square number has
to a square number.

l» .1 «

c-
D e-

For, since A, B are similar plane numbers,

therefore one mean proportional number falls between A, B,

[vm. 18]

Let it so fall, and let it be C; —- - -

and let /), £, F, the least numbers of those which have the
same ratio with A, C, B, be taken ; [vu, 33 or vm. 2]

therefore the extremes of them D, F are square, fvm. i. Pot.]

And since, as D is to F, so is A to B,

and D, F are square,

therefore A has to B the ratio which a square number has to
a square number.
\end{proof}

\begin{notes}

If a,  are similar ``plane numbers,'' let  be the mean proportional
between them, [vui. 18

Take a, P,y the smallest numbers in the ratio of a, c, t. [vti. 33 or viii. 2
Then a, y are squares. [vii[, j, Por.

Therefore a, i are in the ratio of a square to a square.

\end{notes}

\end{proposition}

\begin{proposition}
\label{prop:VIII_27}

\begin{statement}
Similar solid numbers have to one another the ratio which
a cube number has to a cube number.
\end{statement}

\begin{proof}

Let A, Bh similar solid numbers ; . .

I say that A has to B the ratio which a cube number has to
a cube number.

A c-

B D-

E — F Q H

For, since A, B are similar solid numbers,

therefore two mean proportional numbers fall between A, B.

[viii. 19]
Let C, D so fall,

and let E, F, G, H, the least numbers of those which have
the same ratio with A, C, D, B, and equal with them in
multitude, be taken ; [vii. 33 or vm, j]

therefore the extremes of them E, H are cube. [vm. », Por.]

And, as EiXo H,o  A Xa B

therefore A also has to B the ratio which a cube number has

to a cube number. ,
\end{proof}

\begin{notes}

The sime thing as \prop{8}{26} with cubes. It is proved in the same way
except that viii. 19 is used instead of viii. 18,

The last note of an-Nairīzī in which the name of Heron is mentioned is
on this proposition. Heron is there stated (p. 194 — 5, ed. Curtze) to have
added the two propositions that,

I. If two numbtri havt to ons another the ratio 0/ a square to a square, the
numbers are similar f lane numbers ;

1, If two numbers have to one another the ratio of a cube to a cube, the numbers
are similar solid numbers.

The propositions are of course the converses of viii. a 6, 27 respectively.
They are easily proved

(i) If a:b = c':<P,

then, since theie is one mean proportional (ed) between c\ d*,

[viii. 1 1 or 18]
there is also orx mean proportional between a, b. [viii. 8]

Therefore a, b are simitar plane numbers. [viti. 20]

(1) is similarly proved by the use of viii. 12 or it, viii. 8, vtit. 31.

The insertion by Heron of the first of the two propositions, the converse
of vm. a6, is fwrhaps an argument in favour of the correctness of the text of
IX. 10, though (as remarked in the -note on that proposition) it does not give
the easiest proof Cf Heron's extension ni vii. 3 tacitly assumed by Euclid
in vii, 33.

\end{notes}

\end{proposition}

\part{Book IX}

\begin{proposition}
\label{prop:IX_1}

\begin{statement}
If two similar plane numbers by muUiplying one another
make some number, the product will be square.
\end{statement}

\begin{proof}

Let A, B be two similar plane numbers, and let A by
multiplying B make C;
I say that C is square.

For let A by multiplying itself
make D.

Therefore /? is square.

Since then A by multiplying itself has made D, and by
multiplying B has made C,

therefore, as  is to , so is Z? to C [vti. 17]

And, since A, B are similar plane numbers,

therefore one mean proportional number falls between A, B.

[vm. 18]

But, if numbers fall between two numbers in continued

proportion, as many as fall between thetn, so many also fall

between those which have the same ratio ; [vni. 8]

so that one mean proportional number falls between D, Calso.

And D is square ;
therefore C is also square. [vm. la]
\end{proof}

\begin{notes}

The product of two similar pkne numbers b a square.
Let a,  be two similar plane numbers.

Now « : i = fl'' ; oi. [vii. 1 7''

And between a, b there b one mean profwrtional. [vm, i

Therefore between «' : « there is one mean proportional. [vm, 8'

And < is square ;
therefore ab is square. [vm. ai]

\end{notes}

\end{proposition}

\begin{proposition}
\label{prop:IX_2}

\begin{statement}
If two numbers by multiplying one another make a square
number, they are similar plane numbers.
\end{statement}

\begin{proof}

Let A, B be two numbers, and let A by multiplying B

make the square number C\ , .

f I **!' ('  est* '

I say that A, B are similar plane '' '
numbers. b

For let A by multiplying itself
make D ; °

therefore D is square. `` ``'  '``''>

Now, since  by multiplying itself has made D, and by
multiplying B has made C,
therefore, as  is to , so is Z) to C  '' '`` '- [vn. 17]

And, since D is square, and C is so also,
therefore D, C are similar plane numbers.

Therefore one mean proportional number falls between
A C. [vin. 18]

And, as D is to C so is . to  ;
therefore one mean proportional number falls between v4,./?
also. [viii. 8]

But, if one mean proportional number fall between two
numbers, they are similar plane numbers ; [vim. 20]

therefore A, B are similar plane numbers.
\end{proof}

\begin{notes}

ir oj is a square number, «, b are similar plane numbers, (The converse
of IX. i<)

For a ; b - a* : al. [vn. ij]

And a', ab being square numbers, and therefore similar plane numbers,

they have one mean proportional. [vin. 18]

Therefore a, b also have one mean proportional. [vni. 8]

whence a. b are similar plane numbers. rvni. aol

\end{notes}

\end{proposition}

\begin{proposition}
\label{prop:IX_3}

\begin{statement}
If a cube number by multiplying itself make some number,
the product vnll be cube.
\end{statement}

\begin{proof}

For let the cube number A by multiplying itself make B ;
I say that B is cube.

For let C, the side of , be taken, and let C by multiplying
itself make D.

It is then manifest that Cby muhiplying a

D has made A.

Now, since C by multiplying itself has c- d —

made D,
therefore C measures D according to the units in itself.

But further the unit also measures C according to the units
in it;
therefore, as the unit is to C, so is C to D. [vn, Def. ao]

Again, since C by multiplying D has made A
therefore D measures A according to the units in C.

But the unit also measures C according to the units in it ;
therefore, as the unit is to C so is Z? to -(4.

But, as the unit is to C, so is C to Z? ;
therefore also, as the unit is to C, so is C to D, and D to A.

Therefore between the unit and the number A two mean
proportional numbers C, D have fallen in continued proportion.

Again, since A by multiplying itself has made B,
therefore A measures B according to the units in itself.

But the unit also measures A according to the units in it;
therefore, as the unit   .<a A , ?  A to B. [vu. Def. 20]

But between the unit and A two mean proportional numbers
have fallen ;

therefore two mean proportional numbers will also fall between
A, B. (vtu. 8]

But, if two mean proportional numbers fall between two
numbers, and the first be cube, the second will also be cube.

[vni. 23]

And A is cube ;
therefore B is also cube.
\end{proof}

\begin{notes}

The product of t into itself, or a'' , a', is a cube.

For I : rt = a : fl' = a* : b'.

Therefore between i and a' there aie two mean proportionals.

Also I : a = u' : a* . o*.

Therefore two mean proportionals fall between a' and «' . a*. [vui, 8]

(U is true that viii. 8 is only enunciated of two pairs of numbers, but the
proof is equally valid if one number of one pair is unity.)

And a* is a cube number:
therefore o . <t' is also cube. [viii. 13]

\end{notes}

\end{proposition}

\begin{proposition}
\label{prop:IX_4}

\begin{statement}
If a cuie number by multiplying a cube number make some
number, the product will be cube.
\end{statement}

\begin{proof}

For let the cube number A by multiplying the cube number
B make C ;
I say that C is cube. A

For let A by multiplying e— —

itself make D ; c

therefore D is cube. [ix. 3] °

And, since A by muUtply-
ing itself has made D, and by multiplying B has made C
therefore, as A is to B, so is D to C [vii. ij]

And, since A, B are cube numbers,
A, B are similar solid numbers.

Therefore two mean proportional numbers fall between
A, B ; [viii. 19]

so that two mean proportional numbers will fall between D,
C also. [viii. 8]

And D is cjbe;
therefore C is also cube [vm. 13]
\end{proof}

\begin{notes}

The product of two cubes, say «* . , is a cabe.

For a' : *' = a'.o' : d*.*''.'' [vii. 17]

And two mean proportionals fall between a*, f which are similar solid

numbers, [vili. 19'

Therefore two mean proportionals fall between 4' .i, .y [viii. 8'

B jt a* , o* is a cube : [ix. 3

therefore a*.  is a cube. [vili, 23

\end{notes}

\end{proposition}

\begin{proposition}
\label{prop:IX_5}

\begin{statement}
If a cube number by multiplying any number make a cube
number, Ike multiplied number will also be cube.
\end{statement}

\begin{proof}

For let the cube number A by multiplying any number B
make the cube number C;
I say that B is cube. /

For let A by multiplying s « .<'

itself make D ; c

therefore D is cube. [ix. 3] P

Now, since  by multiplying itself has made J?, and by-
multiplying £ has made C,
therefore, as >4 is to B, so is Z? to C. v<fv,a  : '.*' [vn. 17]

And since Z>, C are cube,

they are similar solid numbers.

Therefore two mean proportional numbers fall between
D, C [vill, ig]

And, as Z* is to C, so is  to ,5 ;
therefore two mean proportional numbers fall between A, B
also, [vin, 8]

And A is cube ;
therefore B is also cube, [viii. aj]
\end{proof}

\begin{notes}

If the product cfb is a cube number, b is cube. `` '

By IX. 3, the product a',<i' is a cube. . V.   ,.

And tf , < : tfb = tf i b. [vn. 17]

The first two terms are cubes, and therefore ``similar sohds''; therefore

there are two mean proportionals between them. [i''- 19]

Therefore there are two mean proportionals between a*, b. [vn:. 8J

And 0* is a cube :

therefore i is a cube number. [viiL 93]

\end{notes}

\end{proposition}

\begin{proposition}
\label{prop:IX_6}

\begin{statement}
If a nttmber by muUiplytng itself make a cube number, it
will itself also be cube.
\end{statement}

\begin{proof}

For let the number A by multiplying itself make the cube
number B ;

I say that A is also cube,   "'"''I-''. a

. For let A by multiplying B make C.

Since, then, A by multiplying itself o

has made B, and by multiplying B has

made C,

therefore C is cube. -.-  .u.av. -. -/w .. .<

And, since A by multiplying itself has made B,
therefore A measures B according to the units in itself.

But the unit also measures A according to the units in it
Therefore, as the unit is to A, so is A to B, [vn. Def. ao]
And, since A by multiplying B has made C,
therefore B measures C according to the units in A.

But the unit also measures A according to the units in it.

Therefore, as the unit is to A, so is B to C.

But, as the unit is to A, sols A to B ;
therefore also, as  is to B, so is  to C. .

And, since B, C are cube,
they are similar solid numbers.

Therefore there are two mean
between B, C.

And, as B is to C, so is  to B.

Therefore there are two mean
between A, B also.

And B is cube ;
therefore A is also cube.

3«9
[vii. Def. so]

proportional

proportional

numbers
[viii. 19]

numbers

[vm. 8]

[cf. vm. 23]
\end{proof}

\begin{notes}

If a'' IS a cube number, a is also a cube.

For I ; a = fl : a' = a' : a*.

Now a\ ( are both cubes, and therefore ``similar solids ``; therefore ther«
:je two mean proportionals between them. [vm. 19]

Therefore there are two mean proportionals between a, a', [vm. 8]

And a* is a cube :
therefore a is also a cube number. [vm. 23]

It will be noticed that the last step is not an exact quotation of the result
of vtn. «3, because it is there the first of four terms which is known to be a
cube, and the last which is proved to be a cube ; here the case is reversed.
But there is no difficulty. Without inverting the proportions, we have only
to refer to vm. 2r which proves that a, cf, having two mean proportionals
between them, are two similar solid numbers ; whence, since a'' is a cube,
a is also a. cube.

\end{notes}

\end{proposition}

\begin{proposition}
\label{prop:IX_7}

\begin{statement}
Tf a composite number by multiplying any number make
some number, the product will be solid.
\end{statement}

\begin{proof}

For let the composite number A by multiplying any number

B make C

f,

I say that C is solid.

For, since is composite, q

it will be measured by some

number. [vn. Def. 13]

Let it be measured hy D
and, as many times as D measures A, so many units let there
be in B.

[o Since then D measures A according to the units in £'',
therefore E by multiplying D has made A. ym Def. 15]

And, since A by multiplying B has made C,
and A is the product of D, E,
therefore the product of D, E by multiplying B has made C.

Therefore C is solid, and D, E, B are its sides.
\end{proof}

\begin{notes}

Since a composite number is the product of two factors, the result of
multiplying tt by another number is to produce a ~n umber which is the
product of three factors, i.e. a ``solid number.''

\end{notes}

\end{proposition}

\begin{proposition}
\label{prop:IX_8}

\begin{statement}
If as many numbers as we please beginning from an unit, be
in continued proportion, the third from the unit will be square,
as will also those which successively leave out one ; the fourth
will be cube, as will also all those which leave out two; and the
seventh will be at once cube and square, as will also those which
leave out five.
\end{statement}

\begin{proof}

Let there be as many numbers as we please, A, B, C, D,
B, F, beginning from an unit and in con-
tinued proportion ; A

I say that B, the third from the unit, is
square, as are also all those which leave p
out one ; C, the fourth, is cube, as are g
also all those which leave out two ; and p
E, the seventh, is at once cube and
square, as are also all those which leave out five.

For since, as the unit s to A , so is A to B,
therefore the unit measures the number A the same number
of times that A measures. .5, [vu. Def 26]

But the unit measures the number A according to the
units in it ;
therefore A also measures B according to the units in A,

Therefore A by multiplying itself has made B;
therefore B is square. 1 . .

And, since B, C, Z? are in continued proportion, and B is
square,
therefore D is also square. [vtit. a 3]

For the same reason v. , t .1. .u m.-

r IS also square.

Similarly we can prove that all those which leave out one
are square.

I say next that C, the fourth from the unit, is cube, as are
also all those which leave out two.

For since, as the unit is to W, so is 3 to C,

therefore the unit measures the number A the same number
of times that B measures C.

But the unit measures the number A according to the units
in A ;
therefore B also measures C according to the units in A.

Therefore A by multiplying B has made C.

Since then A by multiplying itself has made B, and by
multiplying B has made C, r I    I'l   . I
therefore C is cube.

And, since C, D, E, F are in continued proportion, and C
is cube,  ( .  11'*'' 'J''!  ':! '

therefore F is also cube. fvui. 33]

But it was also proved square ;

therefore the seventh from the unit is both cube and square.

Similarly we can prove that all the numbers which leave
out five are also both cube and square.
\end{proof}

\begin{notes}

If t, a, og, o,, ... be a geometrical progression, then a,, a, a„ ... are
uaies;

a,, a,, a,, ... are cubes ;
tf„ a„, ... are both squares and cubes.  '``  '* '``'

Since 1 : a=a : at,

a, = a''.

And. since a,, a,, a are in geometrical progression aitd a, (= a*) is a square
a, is a square. [viii. 31]

Similarly a,, a, ... are squares.

Next, I : a = o, ; uj ., ..;

whence aj=a', a cube number. ``' , ~``

And, since Oj, anOi, a, are in geometrical progression, and a, is a cube,
a, is a cube. [viii. 33]

Similarly a„ a,,, ... are cubes. Vi t

Clearly then a,, Ou, o„, ... are both squares and cubes.
The whole result is of course obvious if the geometrical progression is
written, with our notation, as

\end{notes}

\end{proposition}

\begin{proposition}
\label{prop:IX_9}

\begin{statement}
T/as many numbers as we please beginning from an unit be
in continued proportion, and the number after the unit be square,
all the rest will also be square. And if Ihs number after the
unit be cube, all the rest will also be cube.
\end{statement}

\begin{proof}

Let there be as many numbers as we please, A, B, C, D,
E, F, beginning from an unit and in con-
tinued proportion, and let A, the number a

after the unit, be square ;

I say that all the rest will also be square.

Now it has been proved that B, the £

third from the unit, is square, as are also f

all those which leave out one ; [ix. 8]

I say that all the rest are also square.

For, since A, B, C are in continued proportion,
and A is square, ,

therefore C is also square. [vm. n]

Again, since B, C, D are in continued proportion,
and B is square,
D is also square. [vm. 22]

Similarly we can prove that all the rest are also square.

Next, let A be cube ;
I say that all the rest are also cube.

Now it has been proved that C, the fourth from the unit,
is cube, as also are all those which leave out two ; [ix, 8]

I say that all the rest are also cube.

For, since, as the unit is to y, so is / to B,
therefore the unit measures A the same number nf times as A
measures B.

But the unit measures A according to the units in it ;
therefore A also measures B according to the units in itself;
therefore A by multiplying itself has made B,

And  is cube. - '

But, if a cube number by multiplying itself make some
number, the product is cube. [ix. 3]

Therefore B is also cube.

And, since the four numbers A, B, C, D are in continued
proportion,

and > is cube, . ., . - , , :.. ...
D also is cube. , ,   mii [v'm. 23]

For the same reason
E is also cube, and similarly all the rest are cube.
\end{proof}

\begin{notes}

If I, a*, tf,, a,, a, ... are in geometrical progression, t>,, a,, a, ... are all
squares;

and, if I, a'', di, Ott t   c in geomtCrical progression, o, a,, ... are all cubes,
(i) By IX. 8, IT,, 1I4, a,, ... are all squares.
And, a', ii„ a, being in geometrical progression, and o' being a square,

a'viA square. [viii, 23]

For the same reason a,, Ot, ... are at) squares.
(1) By IX. 8, 17,, a,, a„ ... are all cubes. .

Now I : o* = «* : «j.

Therefore <»i = (J* . o*, which is a cube, by ix. 3.

And, <, 011 `` > <it being in geometrical progression, and a* being cube,

a, is cube. [``i- '3]

Similarly we prove that o, is cube, and so on.
The results are of course obvious in our notation, the series being
(i) I, a', a*, a*. ... o*

\end{notes}

\end{proposition}

\begin{proposition}
\label{prop:IX_10}

\begin{statement}
l/as many numbers as we piease beginning from an unit be
in continued proportion, and the number after the unit be not
square, neither will any other be square except the third from
the unit and all those which leave out one. And, if the number
after the unit be not cube, neither will any other be cube except
the fourth from the unit and all those which leave out two.
\end{statement}

\begin{proof}

Let there be as many numbers as we please. A, B, C, D,
E, F, beginning from an unit and in continued proportion,
and let A the number after the unit, not be square ;

I say that neither will any other be square except the third

from the unit <and those which

leave out one > . a '

For, if possible, let C be square. b

But B is also square ; [«. 8] 2

[therefore B, C have to one another °
the ratio which a square number
has to a square number].

And, as B is to C, so is  to  ; ``'*''

therefore A, B have to one another the ratio which a square

number has to a square number ;

[so that Ay B are similar plane numbers]. [vin. 26, converse]

And B is square ;
therefore A is also square :
which is contrary to the hypothesis.

Therefore C is not square.

Similarly we can prove that neither is any other of the
numbers square except the third from the unit and those which
leave out one

Next, let A not be cube,

1 say that neither will any other be cube except the fourth
from the unit and those which leave out two.
For, if possible, let D be cube.

Now C is also cube ; for it is fourth from the unit. [ix. 8]
And, as C is to /), so is  to C ;

therefore B also has to C the ratio which a cube has to a cube.

And C is cube ;
therefore B is also cube. [vm. 15]

And since, as the unit is to A, so is A to B,

and the unit measures A according to the units in it,

therefore A also measures B according to the units in itself ;

therefore A by multiplying itself has made the cube number ,

But, if a number by multiplying itself make a cube number,
it is also itself cube. [ix. 6]

Therefore A is also cube :

which is contrary to the hypothesis.

Therefore D is not cube.

Similarly we can prove that neither is any other of the
numbers cube except the fourth from the unit and those which
leave out two.
\end{proof}

\begin{notes}

ir I, a, a, a,, at, ... be a geometrical progression, then (i), if a is nut a
square, none of the terms will be square except a, Of, n,>, ...;
and (a), if a is not a cube, none of the terms will be cube except «j, a„ a„,

With reference to the first part of the proof, viz. that which proves that, if
a, is a square, a must be a square, Heiberg remarks that the words which
I have bracketed are perhaps spurious; for it is easier to use vui. 24 than
the converst of viii, 26, and a use of vui. 24 would correspond better to the
use of VIII. 15 in the second part relating to cubes. I agree in this view and
have bracketed the words accordingly. (See however note, p, 383, on
converses of viii, 26, 27 given by Heron.) If this change be made, the
proof runs as follows.

(i) If possible, let 03 be square. '

Now Q;, ; dj = a : fij.

But , is a square. [ix. 8]

Therefore a is to a, in the ratio of a square to a square.

And <(j is square ;

therefore a is square [viii. 24] : which is impossible.

(3) If possible, let «< be a cube.

Now a, ; a, = a, : a,.

And a, is a cube. [ix. 8]

Therefore a, is to «, in the ratio of a cube to a cube.

And a, is a cube :
therefore a, is a cube. [viu. 35]

But, since i : a = a .oj,

tij = a'.

And, since a* is a cube,

a must be a cube [ix. 6] : which is impossible.

The propositions viii. 24, 35 are here not quoted in their exact form in
(hat the firsi and uccnd squares, or cubes, change places. But there is no
difficulty, since the method by which the theorems are proved shows that
either inference is equally correct

\end{notes}

\end{proposition}

\begin{proposition}
\label{prop:IX_11}

\begin{statement}
If as many numbers as we please beginning from an unit be
in continued proportion, the less measures the greater according
to some one of the numbers which have place among the propor-
tional numbers.
\end{statement}

\begin{proof}

In Let there be as many numbers as we please, B, C, D, E,
beginning from the unit A and in con-
tinued proportion ;

I say that B, the least of the numbers B, a

C, D, E, measures E according to some c-

one of the numbers C, D. o

For since, as the unit A is to B, so

is D to E,

therefore the unit A measures the number B the same number

of times as D measures E \  V'i''., . It. T ,

therefore, alternately, the unit A measures D the same number

of times as B measures E. [vii. 15]

But the unit A measures D according to the units in it ;
therefore B also measures E according to the units in D ;
so that B the less measures E the greater according to some
number of those which have place among the proportional
numbers. —

PoRiSM. And it is manifest that, whatever place the
measuring number has, reckoned from the unit, the same
place also has the number according to which it measures,
reckoned from the number measured, in the direction of the
number before it. —
\end{proof}

\begin{notes}

The proposition and the porism together assert that, if t, a, a,, ... a„ be a
geometrical progression, a, measures a» and gives the quotient o,., (r < it).
Eudid only proves that a = a.a.„ as follows.

Therefore i measures a the same number of times as «„., measures a,.
Hence i measures a„., the same number of times as a measures a, ;

. . [vn. 15)

that IS, a, = a,ai,,.

We can supply the proof of the porism as follovrs.
I : a = Br : a,,,
(J : n, = «,+! : a+j,

whence, tx aequaii,

I : a,, = fl, : a,. [vil. 14]

It follows, by the same argument as before, that

With our notation, we have the theorem of indices that

\end{notes}

\end{proposition}

\begin{proposition}
\label{prop:IX_12}

\begin{statement}
If as many numbers as we please beginning from an unit be
tn conlitmed proportion, by however many prime numbers (he
lasi is measured, the next to the unit will also be measured by
ike same.
\end{statement}

\begin{proof}

Let there be as many numbers as we please, A, B, C, D,
beginning from an unit, and in continued proportion ;
I say that, by however many prime numbers D is measured,
A will also be measured by the same.

A P- `` ,. ',

B Q

 O

For let /? be measured by any prime number £ ;
I say that £ measures A.

I'' For suppose it does not; `` - '

now £ is prime, and any prime number is prime to any which
it does not measure ; [vn. 29]

therefore £, A are prime to one another.

And, since E measures D, let it measure it according to F,
therefore £ by multiplying F has made D.

Again, since A measures D according to the units in C,

[ix, 1 1 and For,]
therefore A by multiplying C has made D.

But, further, £ has also by multiplying F made D ;
therefore the product of , C is equal to the product of £, F.

Therefore, as A is to B, so'is F to C. [vii> 19]

But .(4, .£ are prime, .< j :''i : >

primes are also least, [vn. 21]

and the least measure those which have the same ratio the
same number of times, the antecedent the antecedent and the
consequent the consequent ; .,  [vn. 20]

therefore E measures C.
., Let it measure it according to G; - '

therefore £ by multiplying G has made C.

But, further, by the theorem before this,
A has also by multiplying B made C. [ix. 1 1 and Por.]

Therefore the product of A, B is equal to the product of

Therefore, as -,4 is to £'', so is ff to J?. '' `` [vil 19]

But A, E are prime,
primes are also least, [vii. ai]

and the least numbers measure those which have the same
ratio with them the same number of times, the antecedent the
antecedent and the consequent the consequent : Lii. 10]

therefore H measures B.

Let it measure it accordinp; o H

therefore E by multiplying H has made B,

But further A has also by multiplyiner itself made B ;

[.X.81

therefore the product of E, H is equal to the square on A.

Therefore, as £'' is to , so is  to If''. [vii. 19]

But A, E are prime, ..; > I

primes are also least, [vn. zi]

and the least measure those which have the same ratio the
same number of times, the antecedent the antecedent and the
consequent the consequent ; [vn. so]

therefore E measures A, as antecedent antecedent

But, again, it also does not measure it : ,

which is impossible -iii..  '...if! . /',

Therefore E, A are not prime to one another.
Therefore they are composite to one another. '' ''' `` '
But numbers composite to one another are measured by

some number, i. Def. 14]

And, since E ij by hypothesis prime,

and the prime is not measured by any number other than itself,

therefore E measures A, E,

so that E measures A. >   . . ,

[But it also measures /? ;  ``*-'   . i 1. iv-

therefore E measures A, Z?.]

Similarly we can prove that, by however many prime

numbers D is measured, A will also be measured by the same.
\end{proof}

\begin{notes}

If I, a, 0,, ... ahea. geometrical progression, and a. be measured by any
prime number >, a will a.tso be measured by /.

For, if possible, suppose that p does not measure a J then,/ being prime,

/, a are prime to one another. .  [vii. 29]

Suppose am.p. ,.''.,. .,.,

Now  o, = a.a,.,. [ix. li]

Therefore o . o,., = « . /,

and a : p = m : a-i.  > [vil. 19]

Hence, n,/ being prime to one another,

p measures ii„i. [vil. zo, ir]

By a reptetition of the same process, we can prove that p measures a„.j
and therefore o,„ and so on, and finally that p measures a.

But, by hypothesis, / does not measure a : which is impossible.

Hence p, a are not prime to one another :
therefore they have some common factor. [vu. Def. 14]

Butp is the only number which measures p;
therefore/ measures a.

Heiberg remarks that, as, in the iii0iirK, Euclid sets himself to prove that
E measures j4, the words bracketed above are unnecessary and therefore
perhaps interpoiatedi   » .

\end{notes}

\end{proposition}

\begin{proposition}
\label{prop:IX_13}

\begin{statement}
Ifas many numbers as we piease beginning from an unit be
in continued proportion, and the number after the unit be prime,
the greatest will not be measured by any except those which, have
a place among the proportional numbers.
\end{statement}

\begin{proof}

Let there be as many numbers as we please. A, B, C, D,
beginning from an unit and in continued proportion, and let A,
the number after the unit, be prime ;

I say that Z?, the greatest of them, will not be measured by any
other number except A, B, C.

K e '''

B F   i' ``

C Q.

O H

For, if possible, let it be measured by E, and let E nqt be
the same with any of the numbers A, B, C.

It is then manifest that E is not prime. ``'

For, if £ is prime and measures D,
it will also measure A [ix. 12], which is prime, though it is not
the same with it :
which is impossible. -rvrii .- 1.  . '.

Therefore -£'' is not prime.  -: -  .m..  „.. . .1.

Therefore it is composite.

But any composite number is measured by some prime
number ; [vii. 31]

therefore B is measured by some prime number.

I say next that it will not be measured by any other prime
except A.

For, if £ is measured by another,   . .
and £ measures D,  , 1,

that other will also measure D

so that it will also measure A [ix. n], which is prime, though
it is not the same with it :

which is impossible.

Therefore A measures E.

And, since E measures D, let it measure it according to F.

I say that F is not the same with any of the numbers
A, B, C.

For, if F is the same with one of the numbers A, B,C,
and measures D according to A,

therefore one of the numbers A,B,C also measures D according
to E.

But one of the numbers A, B, C measures D according to
some one of the numbers A, B, C; [ix. 11]

therefore E is also the same with one of the numbers A, B,C:
which is contrary to the hypothesis.

Therefore F is not the same as any one of the numbers
A. B, C.

Similarly we can prove that F is measured by A, by
proving again that Fi not prime.

For, if it is, and measures Z?,

it will also measure A [ix, i *], which is prime, though It is not

the same with it :

which is impossible ;

therefore F is not prime.

Therefore it is composite.

But any composite number is measured by some prime
number ; [va. 31 j

therefore F is measured by some prime number.

I say next that it will not be measured by any other prime
except A.

For, if any other prime number measures F,
and F measures D,

that other will also measure D ;

so that it will also measure A [ix. u], which I's prime, though it
is not the same with it :

which is impossible. , .,,, ,,u .

Therefore A measures F.

And, since E measures D according to  ., .
therefore E by multiplying F has made D.   '

But, further, A has also by multiplying C made D; [ix, n]

therefore the product of A, C is equal to the product of F, F.

Therefore, proportionally, as A is to E, so is F to C.

[vii. 19]
But A measures E ; .

therefore F also measures C,

Let it measure it according to G. ``' '

Similarly, then, we can prove that G is not the same with
any of the numbers A, B, and that it is measured by A.
And, since F measures C according to G

therefore F by multiplying G has made C.

But, further, A has also by multiplying B made C \ [ix. it]
therefore the product of A, B is equal to the product of F, G.

Therefore, proportionally, as A is to F, so is G to B.

[v''' '9]
But y4 measures /; ``  . ', ,,

therefore G also measures . ``'

Let it measure it according to H.

Similarly then we can prove that H is not the same
with A.

And, since G measures B according to H,

therefore G by multiplying H has made B,

But further A has also by multiplying itself made B ;

[ix. 8]

therefore the product of H, G is equal to the square on A.

Therefore, as If is to , so is v4 to 6. [vil 19]

But A measures (7;   >   .   .  ,. 1

therefore H also measures A, which is prime, though it is not
the same with it :
which is absurd.

Therefore D the greatest will not be measured by any
other number except A, B, C.
\end{proof}

\begin{notes}

If I, a, a,, ... a„ b« a geometrical progression, and if a is prime, a. will not
be measured by any numbers except the preceding terms of the series.

ir possible, let a be measured by by a number different from all the
preceding terms.

Now b cannot be prime, for, if it were, it would measure a. [ix. 12]

Therefore b is composite, and hence will be measured by somt prime
number [vii. 31], say p.

Thus p must measure a„ and therefore a [ix. i i] ; so that / cannot be
different from a, and b is not measured by any prime number except a.

Suppose that a~b . r.

Now f cannot be identical with any of the terms a, a,, ... a-ti for, if it
were, 6 would bt identical with another of them: [ix, 11]

which is contrary to the hyfwthesis.

We car non- prove (just as for b) that c cannot be prime and cannot be
measured by any prime number except a.

Since l>.e-a-a . a.i, [ix. Ii]

a:i = e: a,, ,
whence, since a measures i,

IT measures a-i.

Let a, = t .d.

We now prove in the same way that d is not identical with any of the terms
0, a , ... a,i, is not prime, and is not measured by any prime except a, and
also that

d measures a,.|.

Proceeding in this way, we get a last factor, say k, which measures a
though different from it :
which is absurd, since a is prime.

Thus the original supposition that a„ can be measured by a number S
different from all the terms a, a, ... a„, must be incorrect.

Therefore etc.

\end{notes}

\end{proposition}

\begin{proposition}
\label{prop:IX_14}

\begin{statement}
Ifa number be (he least that is measured by prime numbers,
it will not be measured by any other prime number except those
originally measuring it.
\end{statement}

\begin{proof}

For let the number A be the least that is measured by the
prime numbers B, C, D;

I say that A v/iW not be measured by any other prime number
except B, C, Z>.

For, if possible, let it be measured by the prime number
£, and let £ not be the same with any one of the numbers
B. C, D.

A

B

F

Now, since E measures A, let It measure it according
to F

therefore E by multiplying F has made A.

And A is measured by the prime numbers B, C, D.

But, if two numbers by multiplying one another make some
number, and any prime number measure the product, it will
also measure one of the original numbers ; [vii. 30]

therefore B, C, D will measure one of the numbers E, F.

Now they will not measure E ;

for E is prime and not the same with any one of the numbers
B, C D.

Therefore they will measure F, which is less than A :

which is impossible, for A is by hypothesis the least number
measured by B, C, D.

Therefore no prime number will measure A except
B, C, D.
\end{proof}

\begin{notes}

In other words, a number can be resolved into prime factors in only
one way.

Let a be the least number measured by each of the prime numbers
i, t, d, ... k.

If possible, suppose that a has a prime factor/ different from />, c, d, ... k.

Let a-p.m.

Now /i,(,d, ... a, measuring «, must measure one of the two factors/, m.

[vii. 30]
rhey do not, by hypothesis, measure p ;

therefore they must measure iw, a number less than a: ``.)"'';.(-'.

which is contrary to the hypothesis.

Therefore a has no prime factors except i, (, d, ... k.

\end{notes}

\end{proposition}

\begin{proposition}
\label{prop:IX_15}

\begin{statement}
If three numbers in continued proportion be the least of
those which have the same ratio with them, any two whatever
added together will he prime to the remaining number.
\end{statement}

\begin{proof}

Let A, B, C, three numbers in continued proportion, be
the least of those which have the same

ratio with them ; * b

I say that any two of the numbers c

A, B, C whatever added together are 0— J — f

prime to the remainingnumber, namely .

A, B to C\ B, Cto A ; and further A, C to B.

For let two numbers DB, EF, the least of those which
have the same ratio with A, B, C, be taken. [vni. 1]

It is then manifest that I>B by multiplying itself has made
A, and by multiplying BB has made B, and, further. BB by
multiplying itself has made C. [vin. a]

Now, since DB, BBare least,
they are prime to one another. [vii. «]

But, if two numbers be prime to one another,
their sum is also prime to each ; [vii. a8]

therefore DB is also prime to each of the. numbers DB, BB.

But further D£ is also prime to BB ;

therefore DB, DB are prime to BB.

But, if two numbers be prime to any number,
their product is also prime to the other ; [vii. 84]

so that the product of BD, DB is prime to BF
hence the product of FD, DB is also prime to the square
on BF. [vii. »j]

But the product of BD, DE is the square on DB together
with the product of DB, BF; [n. 3]

therefore the square on DB together with the product of DB,
BF is prime to the square on BF.

And the square on DB is A,
the product of DE, BB is B, .,....,
and the square on BF is C;
therefore A, B added together are prime to C.

Similarly we can prove that B, C added together are
prime to A.

I say next that A, C added together are also prime to B.
For, since DF is prime to each of the numbers DE, EF,

the square on DF is also prime to the product of DE, EF.

[vii. 24, js]

But the squares on DE, EF together with twice the pro-
duct of DE, EFa.K equal to the square on DF; [11. 4]
therefore the squares on DE, EF together with twice the
product of DE, EF are prime to the product of DE, EF.

Separando, the squares on DE, EF together with once
the product of DE, EFe prime to the product of DE, EF.

Therefore, separando again, the squares on DE, EF are
prime to the product oi DE, EF. '' ``'``'< ``-

And the square on Z? is v4, :V . .' . .

the product of DE, EF is B,
and the square on EF is C. : ``

Therefore A, C added together are prime to B.
\end{proof}

\begin{notes}

If a, 6, c he a. geometrical progression in the least terms which have a
given common ratio, (i + f), (c+ 3), (a + *) are respectively prime to a, d, e.

Let a :  be the cominon ratio in its lowest terms, so that the geometrical
progression is

a\ a/9, f?. [VHI. »J

Now, a, fi being prime to one another, '

a-*-/) is prime to both a and j8, `` [vii. *8]

Therefore (0 + )1 <> t* tth prime to p.

Hence (a +  a is prime to j3, [vit. 24]

and therefore to ; [vii. 25]

Le- a' + afi is prime to ,

or a + i is prime to . ``

Similarly, c + jS* is prime to a', * *

or * + f is prime to o. .  **''

Lastly, a  «   being prime to both a and j9,

(a + )' is prime to a/9, [vii. 24, 25]

or o' + /S* + Mj9 is prime to afi :

whence a' + yS'' is prime to aj8.

The latter inference, made in two steps, may be proved by reducHo ad
absurdum as Commandinus proves it

If a* +  is not prime to o/S, let x measure them ;
therefore x measures a* +   «- lafi as well as o ;

hence a* +   «- 2 aj3 and aj3 are not prime to one another, which is contrary
to the hypothesis.

\end{notes}

\end{proposition}

\begin{proposition}
\label{prop:IX_16}

\begin{statement}
If two numbers be prime to one another, the second will not
be to any other number as the first is to the second.
\end{statement}

\begin{proof}

For let the two numbers A, Bhe. prime to one another ;
I say that B is not to any other number as
A is to B. A

For, if possible, as A is to B, so let B be b

to C. o

Now A, B are prime,
primes are also least, fvii. zi]

and the least numbers measure those which have the same
ratio the same number of times, the antecedent the antecedent
and the consequent the consequent ; [vii. 20]

therefore A measures B as antecedent antecedent.

But it also measures itself;
therefore A measures A, B which are prime to one another :
which is absurd.

Therefore B will not be to Q as A is to B.
\end{proof}

\begin{notes}

If a, h are prime to one another, they can have no integral third
proportional.
 If possible, let a : 6 = i ; x.

Therefore [vii. zo, 21] a measures i, and a. b have the commorv measure,
d, which is contrary to the hypothesis.  ,.,.,,. a , .

\end{notes}

\end{proposition}

\begin{proposition}
\label{prop:IX_17}

\begin{statement}
  - If there be as many numbers as we please in continued
proportion, and the extremes of them be prime to one another,
the last will not be to any other number as the first to the
second.
\end{statement}

\begin{proof}

For let there be as many numbers as we please, A,B,C,D,
in continued proportion,
and let the extremes of them, A,
D, be prime to one another ;

1 say that D is not to any other

number as A is to B,

For, if possible, as A is to B, so let D he to £ ;
therefore, alternately, as y is to Z?, so is .5 to .£'', [vii. 13]

: J-'

- 'i

-. j '-

,ii '-

'\ 13(1 til

'    1

I I'ur'- 1

 ?o; 1

ti

But A, D are prime, ~  1

primes are also least, if.' i'' i-* yj [vii.it]

and the least numbers measure those which have the same
ratio the same number of times, the antecedent the antecedent
and the consequent the consequent. [vn. 20]

Therefore A measures .

And, as A is to B, so is B to C.

Therefore B also measures C
so that A also measures C.

And since, as B is to C, so is C to D,
and  measures C,
therefore C also measures D. , , „

But y4 measured C ;
so that A also measures D. , ,

But it also measures itself;
therefore A measures A, D which are prime to one another :
which is impossible.

Therefore Z? will not be to any other number as A is to B.
\end{proof}

\begin{notes}

Ifd, 1(9, d„ ... Sh be a geometrical prciression, and a, a. are prime to one
another, then a, a„ a„ can have no integral fourth proportional.

For, if possible, let a ; a, = a, : jf,

Therefore a : a = a\ x,

and hence [vn. 20, 21] a measures at.

Therefore a, measures a,, [vii. Def. 20]

and hence a measures a,, and therefore also ultimately a,.

Thus a, a are both measured by a : which is contrary to the hypothesis.

\end{notes}

\end{proposition}

\begin{proposition}
\label{prop:IX_18}

\begin{statement}
Given two numbers, to investigate whether it is possible to
find a third proportional to them.
\end{statement}

\begin{proof}

Let A, B be the given two numbers, and let it be required
to investigate whether it is possible to find a third proportional
to them.

Now A, B are either prime to one another or not.

And, if they are prime to one another, it has been proved
that it is impossible to find a third proportional to them.

[tx. 16]

Next, let A, B not be prime to one another,
and let B by multiplying itself make C.

Then A either measures C or does not measure it.

A-
B-

First, let it measure it according to D ;
therefore A by multiplying D has made C.

But, further, B has also by multiplying itself made C ;
therefore the product of , D is equal to the square on B,

Therefore, as  is to B, so is /? to Z? ; [vii. 19]

therefore a third proportional number D has been found to
A,B.

Next, let A not measure C ;
I say that it is impossible to find a third proportional number
to A, B.

For, if possible, let /?, such third proportional, have been
found.

Therefore the product of A, D is equal to the square on B.

But the square on  is C;
therefore the product of A, D is equal to C.

Hence A by multiplying D has made C
therefore A measures C according to D.

But, by hypothesis, it also does not measure it :
which is absurd.

Therefore It is not possible to find a third proportional
number xja A B when A does not measure C.
\end{proof}

\begin{notes}

Given two numbers a, 6, to find the condition that they may have an
intral third proportional,

(t) a, i must not be prime to one another. [ix. 16]

(a) a must measure S'.

For, if a, , f be in continued proportion.

Therefore a measures .
Condition (i) is included in condition (1) since, if l = ma, a and S cannot
be prime to one another.

The result is of course easily seen if the three terras in continued
proportion be written

``' ``a' Hi)-

\end{notes}

\end{proposition}

\begin{proposition}
\label{prop:IX_19}

\begin{statement}
Given three numben, to investigate when it is possible to
find a fourth proportional to tkem.
\end{statement}

\begin{proof}

Let A, B,Cx: the given three numbers, and let it be
required to investigate when it is

possible to find a fourth proportional  ,

to them. g .,1'

Now either they are not in con- p , ,

tinued proportion, and the extremes

of them are prime to one another ;

or they are in continued proportion, and the extremes of them

are not prime to one another ;

or they are not in continued proportion, nor are the extremes

of them prime to one another ;

or they are in continued proportion, and the extremes of them

are prime to one another.

If then A, B, C are in continued proportion, and the
extremes of them A, C are prime to one another,
it has been proved that it is impossible to find a fourth pro-
portional number to them. [ix. 17]

tNext, let A, B, C not be in continued proportion, the
extremes being again prime to one another ;
I say that in this case also it is impossible to find a fourth
proportional to them.

For, if possible, let D have been found, so that,
as y is to , so is C to D,
and let it be contrived that, as  is to C, so is Z> to E.

Now, since, as A is to B, so is C to D,
and, as  is to C, so is Z? to .£'',
therefore, ex aequali, as 4 is to C, so is C to B. [vii. 14]

But A, C are prime,
primes are also least, [vit. li]

and the least numbers measure those which have the same
ratio, the antecedent the antecedent and the consequent the
consequent [vn, ao]

Therefore A measures C as antecedent antecedent.

But it also measures itself ;
therefore A measures A, C which are prime to one another :
which is impossible.

Therefore it is not possible to find a fourth proportional

to A. B, C.f

Next, et A, B, C be again in continued proportion,
but let A, C not be prime to one another.

I say that it is possible to find a fourth proportional to
them.

For let B by multiplying C make D ;
therefore A either measures D or does not measure it.

First, let it measure it according to E ;
therefore A by multiplying E has made D.

But, further, B has also by multiplying C made D ;
therefore the product of A, E is equal to the product of
B, C;

therefore, proportionally, as A is to B, so is C to  ; [vii. 19]
therefore E has been found a fourth proportional to A, B, C.

Next, let A not measure D ;
I say that it Is impossible to find a fourth proportional number
to A, B, C.

For, if possible, let E have been found ;

therefore the product of A, E is equal to the product of, C,

[vii. 19]

But the product if B, C is D ;
therefore the product of A, E is also equal to D.

Therefore A by multiplying E has made D ;
therefore A measures D according to E,
so that A measures D.

But it also does not measure it :
which is absurd.

Therefore it is not possible to find a fourth proportional
number to A, B, C when A does not measure D.

Next, let A, B, C not he in continued proportion, nor the
extremes prime to one another.

And let B by multiplying C make D. . 1

Similarly then it can be proved that, if A measures D,
it is possible to find a fourth proportional to them, but, if it
does not measure it, impossible.
\end{proof}

\begin{notes}

Given three numbers a, b, e, to find the condition that they may have an
integral fourth proportional.

The Greek text of part of this proposition is hopelessly corrupt. Accord-
ing to it Euclid takes four cases,

(i) a,b,e not in continued proportion, and a, c prime to one another.

(2) a, , c in continued proportion, and a, c not prime to one another.

(3) o, *i < not in continued proportion, and a, t not prime to one another.

(4) a,b,c in continued proportion, and a, c prime to one another.

(4) is the case dealt with in ix. 17, where it is shown that on hypothesis
(4) a fourth proportional cannot be found.

The text now takes case (i) and asserts that a fourth proportional cannot
be found in this case either. We have only to think of 4, 6, 9 in order to see
that there is something wrong here. The supposed proof is also wrong. If
possible, says the text, let rf Ije a fourth proportional to a, b, c, and lei e
be taken such that

b : c - d : e.

Then, ex auali, n '. c-c ; e,

whence a measures c : [vii, so, Ji]

which is impossible, since o, c are prime to one another.

But this does not prove that a fourth proportional d cannot be found ; it
only proves that, if if is a fourth proportional, no integer e can be found to
satisfy the equation , - .

b : e-d ; e.

Indeed it is obvious from ix. 16 that in the equation f-    .

a : € = t e ``

t cannot be integral. > .

The cases ( ) and (3) are correctly given, the first in full, and the other as
a case to be proved ``similarly'' to it.

These two cases really give all that is necessary.

Let the product be be taken.

Then, if a measures be, suppose bc = ad;
therefore a : b = c : d,

and  is a fourth proportional.

But, if a does not measure be, no fourth proportional can be found.
For, if X were a fourth proportional, ax would be equal to />c, and a would
measure be.

The sufficient condition in any case for the possibility of finding a fourth
proportional to a, b, e is that a should measure be.

Theon appears to have corrected the proof by leaving out the incorrect
portion which I have included between daggers and the last case (3) dealt
with in the last lines. Also, in accordance with this arrangement, he does not
distinguish four cases at the beginning but only two. `` Either A, B, C are
in continued proportion and the extremes of them A, C are prime to one
another; or not,'' Then, instead of introducing case (2) by the words
``Next let A, B, C.to find a fourth proportional to them,'' immediately
following the second dagger above, Theon merely says `` But if not,'' [i.e.
if it is not the case that a, b,  are in g.p. and a,  prime to one another] ``let
B by multiplying C make D,'' and so on.

August adopts Theon's fonn of the proof. Heibeig does not feel able to
do this, in view of the superiority of the authority for the text as given above
(P) ; he therefore retains the latter vrithout any attempt to emend it.

\end{notes}

\end{proposition}

\begin{proposition}
\label{prop:IX_20}

\begin{statement}
Prime numbers are more than any assigned muUitvde of
prime numbers. iutt t ji .-i -irrw )>.
\end{statement}

\begin{proof}

Let , jff, C be the assigned prime numbers ;

I say that there are more > i'

prime numbers than A, B, C. a— ''  '

For let the least number  Q

measured by , .5, C be c

taken, e F

and let it be DE ;

let the unit i?/ be added to ZJ.fi'.

Then EF is either prime or not. '

First, let it be prime ;
then the prime numbers A, B, C, EEhave been found which
are more than A, B, C.

Next, let `` not be prime; ; '-'a >  -i-

therefore it is measured by some prime number [vir. 31]

Let it be measured by the prime number G.

I say that G is not the same with any of the numbers
A,B,C.

For, if possible, let it be so.

Now A, B, C measure DE ;  `` '

therefore G also will measure DE. '\ , ',' ``,' |'

But it also measures EE.

Therefore G, being a number, will measure the remainder,
the unit i?: .., . , , . ,

which is absurd.

Therefore G is not the same with any one of the numbers
A,B,C. . I ,

And by hypothesis it is prime.
 Therefore the prime numbers A, B, C, G have been found
which are more than the assigned multitude of A, B, C.
\end{proof}

\begin{notes}

We have here the important proposition that the number of prime numiert
ts infinite.

The proof will be seen to be the same as that given in our algebraical
text-books. Let a, , e, ..  Ji be any prime numbers.

Take the product aie ... k and add unity.

Then (ai( ... i + i) is either a prime number or not a prime number.

(i) If it /r, we have added another prime number to those given.

(2) If it is not, it must be measured by some prime number [vii. 31], say/.

Now/ cannot be identical with any of the prime numbers a, b,e, ... k.

For, if it [s, it will divide abc ... k.
Therefore, since it divides labc...'k. i) also, it will measure the difference,
or unity :
which is impossible.

Therefore in any case we have obtained one fresh prime number.

And the process can be carried on to any extent.

\end{notes}

\end{proposition}

\begin{proposition}
\label{prop:IX_21}

\begin{statement}
If as many even numbers as we please be added together,
tke whole is even.
\end{statement}

\begin{proof}

For let as many even numbers as we please, AB, BC, CD,
DE, be added together ;

I say that the whole AE * b c o e

is even.

For, since each of the numbers AB, BC, CD, DE is even,
it has a half part ,; [vii, Def. 6]

so that the whole AE also has a half part.

But an even number is that which is divisible into two
equal parts ; [«A]

therefore AE is even,
\end{proof}

\begin{notes}

In this and the following propositions up to ix. 34 inclusive we have a
number of theorems about odd, even, ``even-times even'' and ``even- times
odd'' numbers respectively. They are all simple and require no explanation
in order to enable them to be followed easily.

\end{notes}

\end{proposition}

\begin{proposition}
\label{prop:IX_22}

\begin{statement}
If as many odd numbers as we please be added together, and
their multitude be even, the whole will be even.
\end{statement}

\begin{proof}

For let as many odd numbers as we please, AB, BC, CD,
DE, even in multitude, be added together ; ,

I say that the whole AE is even.

For, since each of the numbers AB, BC, CD, DE is odd,
if an unit be subtracted from each, each of the remainders will
be even ; [vn, ]>ef. 7]

so that the sum of them will be even. > [ix. 31]

*  p D E

But the multitude of the units is also even.

Therefore the whole AE is also even. [ix. ai]
\end{proof}

\end{proposition}

\begin{proposition}
\label{prop:IX_23}

\begin{statement}
If as many odd numders as we please be added together,
and their multitude be odd, the whole will also be odd.
\end{statement}

\begin{proof}

For let as many odd numbers as we please, AB, BC, CD,
the multitude of which is odd,

be added tcether ; a b c e d

I say that the whole AD is *'' ``

also odd.

Let the unit DE be suht»cted from CD ;
therefore the remainder CE is even.

But CA is also even ;
therefore the whole AE is also even.

And DE is an unit.

Therefore AD is odd,
\end{proof}

\begin{annotations}

3. LiteniUj' '* let there c as many numbers u we please, of which /tf the muLtUude it
odd.'' Thb forai, natural in Greek, is awkward in English,

\end{annotations}

\end{proposition}

\begin{proposition}
\label{prop:IX_24}

\begin{statement}
If from an even number an even number be subtracted, the
remainder will be even.
\end{statement}

\begin{proof}

For from the even number AB let the even number BC
be subtracted :
1 say that the remainder CA is even. a 9 a

For, since AB is even, it has a half
part, [vn. I>ef. 6]

[vii

1. Def.

7]

[ X.

..]

t.x.

``]

[v.>

:. Def.

71

Q.

, E.

D.

For the same reason BC also has a half part ;

so that the remainder [CA also has a half part, and] AC i
therefore even.
\end{proof}

\end{proposition}

\begin{proposition}
\label{prop:IX_25}

\begin{statement}
1/ from an even number an odd number be subtracted, the
remainder will be odd.
\end{statement}

\begin{proof}

For from the even number AB let the odd number BC be
subtracted ;

I say that the remainder CA is odd. /k c d b

For let the unit CD be sub-
tracted from BC ;

therefore DB is even.  [vn. Def. 7]

But AB is also even ;
therefore the remainder AD is also even. [ix. 84]

And CD is an unit ;
therefore CA is odd. [vn. Def. 7]

I'
\end{proof}

\end{proposition}

\begin{proposition}
\label{prop:IX_26}

\begin{statement}
If from an odd number an odd number be subtracted, the
remainder will be even.
\end{statement}

\begin{proof}

For from the odd number AB let the odd number BC be
subtracted

I say that the remainder CA is even. p o d a

For, since AB is odd, let the unit
BD be subtracted ; ...

therefore the remainder AD is even. [vii, Def. 7]

For the same reason CD is also even ; [vn. Def. 7]

so that the remainder CA is also even, [ix. 14]
\end{proof}

\end{proposition}

\begin{proposition}
\label{prop:IX_27}

\begin{statement}
If from an odd number an even number be subtracted, the
remainder will be odd.
\end{statement}

\begin{proof}

For from the odd number j45 let the even number BC be
subtracted ;
I say that the remainder CA is odd.

Let the unit AD be subtracted ; —

therefore DB is even. [vn. Def, 7]

But BC is also even ;
therefore the remainder CD is even. [ix. 4]

Therefore CA is odd. [vil IM, 7]
\end{proof}

\end{proposition}

\begin{proposition}
\label{prop:IX_28}

\begin{statement}
If an odd number by multiplying an even number make
some number, the product will be even.
\end{statement}

\begin{proof}

For let the odd number A by multiplying the even number
B make C\ „ , .
I say that C is even.

For, since A by multiplying B has

made C,

therefore C is made up of as many numbers equal to B as

there are units in A. [vn. Def. 15]

And B is even ;
therefore C is made up of even numbers.

But, if as maoy even numbers as we please be added
together, the whole is even. [ix. ai]

Therefore C is even.
\end{proof}

\end{proposition}

\begin{proposition}
\label{prop:IX_29}

\begin{statement}
If an odd number by multiplying an odd number muke
some number, the product will be odd.
\end{statement}

\begin{proof}

For let the odd number A by multiplying the odd number
B make C ;
I say that C is odd. *

For, since A by multiplying B has g

made C,

therefore C is made up of as many numbers equal to B as
there are units in A. [vii. Def. 15]

And each of the numbers A, B is odd ;
therefore C is made up of odd numbers the multitude of which
is odd.

'''tins C is odd. ,., .r, --li-Ki.- b'ii
\end{proof}

\end{proposition}

\begin{proposition}
\label{prop:IX_30}

\begin{statement}
If an odd number measure an even number it wiU also
measure Ike half of it.
\end{statement}

\begin{proof}

For let the odd number A measure the even number B ;
I say that it will also measure the half
of it.

For, since A measures B, a

let it measure it according to C ; o

I say that C is not odd.

For, if possible, let it be so. '

Then, since A measures B according to C, ``'- `` **
therefore A by multiplying C has made B.

Therefore B is made up of odd numbers the multitude
of which is odd.

Therefore .5 is odd : n ;: \   v.  [``c. 33]

which is absurd, for by hypothesis it is even.   -   .

Therefore C is not odd ; < :  '

therefore C is even. - '``' c» .our,

  , r

Thift A measures B an even number of times.
For this reason then it also measures the half of it.
\end{proof}

\end{proposition}

\begin{proposition}
\label{prop:IX_31}

\begin{statement}
If an odd number be prime to any nutter, it will also be
prime to the double of it.
\end{statement}

\begin{proof}

For let the odd number A be prime to any number B,

and let C be double of B ;

I say that A is prime to C.

For, if they are not prime

to one another, some number
will measure them. "

Let a number measure them, and let it be Z*.

Now - is odd ;
therefore D is also odd.

And since D which is odd measures C,
and C is even, ``''' '' '

therefore [ZJ] will measure the half of C also, [ix. 30]

But  is half of C;
therefore D measures B.

But it also measures A ;
therefore D measures /I, B which are prime to one another:
which is impossible.

Therefore A cannot but be prime to C.

Therefore A, C are prime to one another.
\end{proof}

\end{proposition}

\begin{proposition}
\label{prop:IX_32}

\begin{statement}
Each of the numbers which are continually doubled beginning
from a dyad is even-times even only.
\end{statement}

\begin{proof}

For let as many numbers as we please, B, C, D, have been
continually doubled beginning
from the dyad A ; * —

I say that B, C, D are even- '

times even only. .

Now that each of the
numbers B, C, D is even-times even is manifest ; for it is
doubled from a dyad.

I say that it is also even-times even only.

For let an unit be set out.

Since then as many numbers as we please beginning from
an unit are in continued proportion,
and the number A after the unit is prime,
therefore D, the greatest of the numbers A, B, C, D, will not
be measured by any other number except A, B, C. [ix. 13]

And each of the numbers A, B, C is even ;
therefore D is even-times even only. [vn. Def. 8]

Similarly we can prove that each of the numbers B, C is
even-times even only.
\end{proof}

\begin{notes}
See the notes on vti. Deff. 8 to 11 for a discussion of the difficulties
shown by lamblichus to be involved by the Euclidean definitions of `` even-
times even,'' ``eveo-timesodd'' and ``odd-times even,''

\end{notes}

\end{proposition}

\begin{proposition}
\label{prop:IX_33}

\begin{statement}
I/a number have its haif odd, it is even-times odd only.
\end{statement}

\begin{proof}

For let the number A have its half odd ;
I say that A Is even-times odd only.

Now that it is even-times odd is

manifest ; for the half of it, being odd,

measures it an even number of times. [vil Def. 9)

I say next that it is also even -times odd only.

For, if A is even-times even also,
it will be measured by an even number according to an even
number ; [vii. Def. 8]

so that the half of it will also be measured by an even number
though it is odd :
which is absurd.  , . ,, ,,

Therefore A is even-times odd only,  , .
\end{proof}

\end{proposition}

\begin{proposition}
\label{prop:IX_34}

\begin{statement}
If a numier neither be one of those which are continually
doubled from a dyad, nor have its half odd, it is both even-
times even and even-times odd.
\end{statement}

\begin{proof}

For let the number A neither be one of those doubled
from a dyad, nor have its half odd  ;
I say that A is both even-times even j.

and even-times odd.

Now that A is even-times even is manifest ;
for it has not its half odd. [vu. Def. 8]

I say next that it is also even-times odd.

For, if we bisect A, then bisect its half, and do this con-
tinually, we shall come upon some odd number which will
measure A according to an even number.

For, if not, we shall come upon a dyad,
and A will be among those which are doubled from a dyad ;
which is contrary to the hypothesis.

Thus A is even-times odd.

But it was also proved even-times even.

Therefore A is both even-times even and even-times odd.
\end{proof}

\end{proposition}

\begin{proposition}
\label{prop:IX_35}

\begin{statement}
If as many numbers as we please be in coniinved proportion,
and there be subtracted from the second and the last numbers
equal to the first, then, as ths excess of the second is to the
first, so will the excess of the last be to all those before it.
\end{statement}

\begin{proof}

Let there be as many numbers as we please in continued
proportion, A, BC, D, EF, . . ,
beginning from A as least, `` a-
and let there be subtracted B-0
from BC and ithe numbers °

BG, FH, each equal to A ; e - jF

I say that, as GC is to A, so
  EH .Q A, BC, D.

For let FK be made equal to BC, and FL equal to D,

Then, since FK is equal to BC,
and of these the part FH is equal to the part BG,
therefore the remainder HK is equal to the remainder GC.

And since, as EF is to /?, so is /? to BC, and BC to A,
while D is equal to FL, BC to FK, and A to FH,
therefore, as EF is to FL, so is LF to FK, and FK to FH.

Separando, as EL is to LF, so is LK to FK, and KH
to FH. [vii. II, 13]

Therefore also, as one of the antecedents is to one of the
consequents, so are all the antecedents to all the consequents ;

[vii. I a)

therefore, as KH is to FH, so are EL, LK, KH to LF,
FK, HF.

But KH is equal to CG, FH to A, and LF, FK, HFto
D, BC, A ;
therefore, as CG is to yj, so is EH to D, BC, A.

Therefore, as the excess of the second is to the first, so is
the excess of the last to all those before it
\end{proof}

\begin{notes}

This proposition is perhaps the most interesting in the arithmetical Books,
since it gives a method, and a very el;ant one. of summing any serin of
terms in geometriai progression.

Let a,, a,, a,,...a, a,., be a series of terms in geometrical progression.
Then Euclid's proposition proves that

(«ii+i-Oi> : (a, + i*. + ... +a,) = (o,-fl,) : Ui-

For clearness' sake we will on this occasion use the fractionul notation of
algebra to represent proportions.

Euclid's method then comes to this.

Since ??!!=.?!= ... = «», '

we have, separandi),

``n*! — °ii  -''n- i   tfj — i»a  gj-tfi

whence, since, as one of the antecedents is to one of the consequents, so is
the sum of all the antecedents to the sum of ail the consequents, [vti. 12]

< *\ - ``1  I)-''!

which gives a, + «,+ ... 4- a,, or S..

If, to compare the result with that arrived at in algebraical text-books, we
write the series in the form

a, ar, ar',.,.ar'~' (n terms),
a>* ~a ar~a

we have

5, a

alr'-l)

s,=

\end{notes}

\end{proposition}

\begin{proposition}
\label{prop:IX_36}

\begin{statement}
jy as many numbers as we please beginning fr(nn an unit
be set out continuously in double proportion, until the sum of all
becomes prime, and if the sum multiplied into the last make
some number, the product will be perfect.
\end{statement}

\begin{proof}

For let as many tiumbers as we please, A, B, C, D,
beginning from an unit be set out in double proportion, until
the sum of all becomes prime,

let E be equal to the sum, atid let E by multiplying Z?
make FG ;
I say that FG is perfect.

For, however many A, B, C, D are in multitude, let so
many, HK L, Mh taken in double proportion beginning
from £ ;
therefore, ex aeguali, as A is to D, so is E to M. [vii. 14]

Therefore the product of E, D is equal to the product of
A, M, [vii- 19]

And the product of E, D is FG ;
therefore the product of 4, M is also FG. , ; ri'

Therefore A by multiplying M has made FG ;
therefore M measures FG according to the units in A.

And  is a dyad ;
therefore FG is double of M. .

— A

!= E

M

F 1 Q H

Q-

But M, L, HK, E are continuously double of each other ;
therefore E, HK, L, M, FG are continuously proportional in
double proportion.

Now let there be subtracted from the second /f/C and the
last FG the numbers /W, FO, each equal to the first E ;
therefore, as the excess of the second is to the first, so is the
excess of the last to all those before it [ix. 35]

Therefore, as IfJC is to E, so is OG to M, L, KH, E.

And NK is equal to E ;
therefore OG is also equal to M, L, HK, E.

But FO is also equal Xa E, .

and E is equal to A, B, C, D and the unit

Therefore the whole FG is equal to E, HK, L, M and
A, B, C, D and the unit ;
and it is measured by them.

I say also that FG will not be measured by any other
number except A, B, C, Z>, E, HK, L, M and the unit

For, if possible, let some number P measure FG,
and let P not be the same with any of the numbers A, B, C,
D, E, HK, L, M,

And, as many times as P measures FG, so many units let
there be in i? ;
therefore Q by multiplying P has made FG.

:i But, further, £ has also by multiplying D made FG ;
therefore, as  is to 0, so is /* to D. [vn. 19]

And, since A, B, C, D are continuously proportional
beginning from an unit.

therefore D will not be measured by any other number except
A, B, C. [«. 13]

And, by hypothesis, P is not the same with any of the
numbers A, B, C;
therefore P will not measure D.

But, as P is to Z?, so is  to i2 ;
therefore neither does £ measure Q. [vii, Def. ao)

And £ is prime ;
and any prime number is prime to any number which it does
not measure. [vn. >9]

Therefore E, Q are prime to one another.

But primes are also least, [vii. *i]

and the least numbers measure those which have the same
ratio the same number of times, the antecedent the antecedent
and the consequent the consequent ; [vn, 30]

and, as £  is to 0, so is /* to Z? ;

therefore £ measures P the same number of times that Q
measures D.

But D is not measured by any other number except
A, B. C;
therefore Q is the same with one of the numbers A, B, C.

Let it be the same with B.

And, however many B, C, D are in multitude, let so many
£ HK, L be taken beginning from £.

Now £, HK, L are in the same ratio with B, C, D
therefore, ex aequali, as  is to D, so is  to Z. [vn, 14)

Therefore the product of B, L is equal to the product of
A E. [vii. 19]

But the product of D, E is equal to the product of Q, P;
therefore the product of Q, P is also equal to the product of
B,L.

Therefore, as is to B so is L to P. [vn. 19]

And Q is the same with B ;
therefore L is also the same with P :

434 BOOK IX [ix. 36

which is impossible, for by hypothesis P is not the same with
any of the numbers set out.

Therefore no number will measure FG except A B, C,
D, E, HK, L, J/ and the unit.

And FG was proved equal to A, B, C, D, E, HK, L, M
and the unit ;
and a perfect number is that which is equal to its own parts ;

[vii. D«f. 2»]

therefore FG is perfect,
\end{proof}

\begin{notes}

If the sum of any number of terms of the series
I, i, a*, ... 2"'
be prime, and the said sum ( e multiplied by the last term, the product will be
a ``perfect'' number, i.e. equul to the sum of all its factors.

Ijet I + I + j' + . . . + a"' (= 5,) be prime ;
then shall S . 2"' be ``perfect,''

Take (it - i) terms of the series

->ii> ``Jii, 2 Ob, ... i Ob.

These are then terr.is proportional to the terms
a, 3', 2', ... 2''-'.

Therefore, ex aequali, . ,

jra— = ;a-j;, , [vii. 14]

or 2 . 2''-'5',= j''-' . J,.   [vii. 19]

(This is of COttrse obvious algebraically, but Euclid's notation requires him to
prove it.)

Now, by IX. 35, we can sum the series S + 2 j + ., . + 2''-'i,
and (25. - S) : 5. = (a-' 5. - .S;) : (.S; + a; + ... + a—3.).

Therefore .S; + 25. + 2»5, + . . . + 2*-'5, = a''-'5, - S,,
or s''-'', = 5, + a5. + 2*5, + . . . + 2*-*Sn + J.

= S+tS„+ ... + 2'-»5, + (i + 2 + 2' + ... 4- a''-''),
and 2"' 5, is measured by every term of the right hand expression.

It is now necessary to prove that a*'''.?, cannot have any factor except
those terms.

Suppose, if possible, that it has a factor x difTerent from all of them,
and let 2'``''S'» = x . ut.

Therefore S: m = x : 2"'. [vii. 19]

Now 2''~' can only be measured by the preceding terms of the series
I, 2, a',.., a''-', [IX. 13J

and X is different from all of these ; j

therefore x does not measure a"',
so that S, dots not measure m. , [vii, Def. 20]

And S, is prime: therefore it is prime to m. [vii. 29]

It follows [vii. 20, 2r] that

m measures a"'.

Suppose that > <  w = J .i-

Now, ex aequaii, a' r i"' = S, : j''-'-' 5,. '

Therefore a'' . a"''-' 5« = s"' 5. [vji. 1 9]

- X . m, from above.

And m = a'' ;
therefore .r = 2''-'-' S», one of the terms of the series Sx,S,2*Sf, .,.i*~'S:
which contradicts the h)pothesis.

There a''~'5, has no factors except

5„ zSn, a'5„ ... i''-'., I, 2, a', ... i'-

Theon of Smyrna and Nicomachus both define a `` perfect `` number and
give the law of its formation, Nicomachus gives four perfect numbers and no
more, namely 6, 28, 496, 8138. He says they are formed in ``ordered''
fashion, there being one among the units (i.e. less than 10), one among the
tens (less than too), one among the hundreds (less than 1000) and one among
the thousands (less than loooo) ; he adds that they terminate in 6 or 8
alternately. They do all terminate in 6 or 8, as can easily be proved by
means of the formula (2''- 1)1"' (cf. Loria, Le tcienst tmtte neW antica
Grtda, pp. 840 — i), but not alternately, for the fifth and sixth perfect numbers
both end in 6, and the seventh and eighth both end in 8. lamblichus adds
a tentative suggestion that perhaps there may he, in like manner, one perfect
number among the ``first myriads'' (less than loooo'), one among the ``second
myriads'' (less than loooo'), and so on. This is, as we shall see, incorrect.

It is natural that the subject of perfect numbers should, ever since Enclid's
time, have had a fascination for mathematicians. Fermat (160 1—1655), in a
letter to Mersenne (CEuvres de Fermai, ed. Tannery and Henry, Vol. 11.,
1894, pp. 197 — 9), enunciated three propositions which much facilitate the
investigation whether a given number of the form a''-! is prime or not. If
we write in one line the exponents i, a, 3, 4, etc. of the successive powers of
2 and underneath them respectively the numbers representing the correspond-
ing powers of a diminished by i, thus,

1334567 S 9 10 II ...n
I 3 7 IS 3' 63 ``7 ass S'' >«23 2047...2''-i,
the following relations are found to subsist between the numbers in the first
line and those directly below them in the second line.

I. If the exponent is not a prime number, the corresponding number is
not a prime number either (since a'' ~ i is always divisible by a' — i as well
as by a' - 1 ).

a. If the exponent is a prime number, the corresponding number dimi-
nished by I is divisible by twice the exponent, [(a*- a)/2M = (a"' -i)/« ; so
that this is a special case of `` Fermat's theorem that, if/ is a prime number
and a is prime to/, then cf~' is divisible by/.]

3. If the exponent n is a prime number, the corresponding number is
only divisible by numbers of the form (tmn+ i). If therefore the corre-
sponding number in the second line has no factors of this form, it has no
inte>;ral factor.

The first and third of these propositions are those which are specially
useful for the purpose in question. As usual, Fermat does not give his proofs
but merely adds : `` Voilil trois fort belles propositions que j'ay trouvees et
prouves non sans peine. Je les puis apfteller les fondements de I'invention
des nombres parfaits.''

I append a few details o( discoveries of further perfect numbers after the
first four. The next are as follows :

fifth, 3''(3''-i) = 33 SSO 336 f

sixth, »'' (* '-!) = 8 589 869 oj6

seventh, i''(i'*-i)= 137 438 691 jaS

eighth, 2* (2''-!)= 2 305 843 008 139 95* ia8

ninth, 2'' (2*' - r ) = 2 658 455 99 1 569 83 1 744 654 691 6 1 5 95 3 842 1 76

tenth, a''(a''-i).
It has further been proved that s'``- 1 is prime, and so is a'``- r. Hence
3'`` (a'``-!) and 3'``(2'''-r) are two more perfect numbers.

The fifth perfect number may have been known to kmblichus, though he
does not give it ; it was however known, with all its factors, in the fifteenth
century, as appears from a tract written in German which was discovered by
Curtze (Cod. lat. Monac. 14908). The first eight perfect numbers were
calculated by Jean Frestet(d. 1670). Fermat had stated, and Euler proved,
that a''- I is prime. The ninth perfect number was found by P. Seelhoff
(Zatithrift filr Math, u, Physik, xxxi., 1886, pp. 174 — 8) and verified by
E, Lucas (Mat/Usis, vii,, 1887, pp. 45 — 6). The tenth was discovered by
R. E. Powers (see Bulletin of ifu Ameriam Mathematical Society, xvni,, 191*,
p. i6a), 2''*— I was proved to be prime by E. Fauqueuibergue and R. E.
Powers (191 4), while Fauquembergue proved that 2"-! is prime.

There have been attempts, so far unsuccessful, to solve the question
whether there exist other `` perfect numbers `` than those of Euclid, and, in
particular, p>erfect numbers which are odd. (Cf. several notes by Sylvester in
Comptes rtndus, cvi., t888 ; Catalan, `` Mdlanges mathrfmatiques `` In Mm. de
la Sec. dt Liige, 2* Srie, xv,, 1888, pp. ao5 — 7 ; C. Servais in Mathisii, vii.,
pp. 228 — 30 and VIII., pp. 92 — 93, 135; E, Cesiro in Mathsis, vii.,
pp. 245 — 6 ; E. Lucas in Mathisis, X., pp. 74 — 6).

For the detailed history of the whole subject see L. E. Dickson, History
o/the Theory of Numbers, Vol. 1., 19 19, pp. iii — iv, 3 — 33.

\end{notes}

\end{proposition}

\part{Book X}

\chapter*{Introductory Note}

We have seen (Vol. r., p. 351 etc.) that the discovery of the irrational is
due to the Pythagoreans. The first scholium on Book x. of the Element 1
states that the Pythagoreans were the first to address themselves to the in-
vestigation of commensurability, having discovered it by means of their obser-
vation of numbers. They discovered, the scholium continues, that not all
magnitudes have a common measure. `` They called all magnitudes measure-
able by the same measure commensurable, but those which are not subject to
the same measure incommensurable, and again such of these as are measured
by some other common measure commensurable with one another, and such
as are not, incommensurable with the others. And thus by assuming tbeir
measures they referred everything to different com mensurabili ties, but, though
they were different, even so (they proved that) not all magnitudes are com-
mensurable with any. (They showed that) all magnitudes can be rational
(pp-a) and all irrational (SXaya) in a relative sense ((Js wpoi t(); hence the
commensurable and the incommensurable would be for them natural (kinds)
(iiacrti), while the rational and irrational would rest on assumption or con-
vention (9itrn).'' The scholium quotes further the legend according to which
`` the first of the Pythagoreans who made public the investigation of these
matters perished in a shipwreck,'' conjecturing that the authors of this story
``perhaps spoke allegorical ly, hinting that everything irrational and formless
is properly concealed, and, if any soul should rashly invade this region of life
and lay it open, it would be carried away into the sea of becoming and be over-
whelmed by its unresting currents.'' There would be a reason also for keeping
the discovery of irrationals secret for the time in the fact that it rendered un-
stable so much of the groundwork of geometry as the Pythagoreans had based
upon the imperfect theory of proportions which applied only to numbers. We
have already, after Tannery, referred to the probability that the discovery
or incommensurability must have necessitated a great recasting of the whole
fabric of elementary geometry, pending the discovery of the general theory
of proportion applicable to incommensurable as well as to commensurable
magnitudes.

It seems certain that it was with reference to the length of the diagonal of
a square or the hypotenuse of an isosceles right-angled triangle that the irra-
tional was discovered. Plato (Theaetetus, 1470) tells us that Theodoras of
Cyrene wrote about square roots (Surau;), proving that the square roots of
three square feet and five square feet are not commensurable with that of one
square foot, and so on, selecting each such square root up to that of 1 7 square
feet, at which for some reason he stopped. No mention is here made of Ja,
doubtless for the reason that its incommensurability had been proved before.
Now we are told that Pythagoras invented a formula for finding right-angled
triangles in rational numbers, and in connexion with this it was inevitable that
the Pythagoreans should investigate the relations between sides and hypo-
tenuse in other right-angled triangles. They would naturally give special
attention to the isosceles right-angled triangle ; they would try to measure the
diagonal, would arrive at successive approximations, in rational fractions, to
the value of J 2, and would find that successive efforts to obtain an exact
expression for it failed. It was however an enormous step to conclude that
such exact expression was impossible, and it was this step which the Pytha-
goreans made. We now know that the formation of the side- and diagonal-
n umbers explained by Theon of Smyrna and others was Pythagorean, and
also that the theorems of Eucl.\ R. 9, 10 were used by the Pythagoreans in
direct connexion with this method of approximating to the value of Jx. The
very method by which Euclid proves these propositions is itself''an indication
of their connexion with the investigation of J 2, since he uses a figure made
up of two isosceles right-angled triangles.

The actual method by which the Pythagoreans proved the incommensura-
bility of J2 with unity was no doubt that referred to by Aristotle (Anal, prior.
1. 23, 41 a 26 — j), a redwtio ad aiisurdum by which it is proved that, if the
diagonal is commensurable with the side, it will follow that the same number
is both odd and even. The proof formerly appeared in the texts of Euclid as
x. 117, but it is undoubtedly an interpolation, and August and Heiberg
accordingly relegate it to an Appendix. It is in substance as follows.

Suppose AC, the diagonal of a square, to be com men- a

su ruble with AH, its side. Let a : j3 be their ratio expressed
in the smallest numbers.

Then a > fi and therefore necessarily > 1.

Now AC* ; A£* = a' ; ,

and, since AC = xAB*, [Eucl.\ 1. 47]

a.' = 2p.

Therefore a a is even, and therefore a is even.

Since a : J3 is in its lowest terms, it follows that /3 must be odd.

Put a = 2y ;

therefore qf = 2fp,

or 0> = 2v»,

so that ``, and therefore /J, must be even.

But was also odd :
which is impossible.-.

This proof only enables us to prove the incommensurability of the
diagonal of a square with its side, or of 2 with unity. In order to prove
the incommensurability of the sides of squares, one of which has three times
the area of another, an entirely different procedure is necessary ; and we find
in fact that, even a century after Pythagoras' time, it was still necessary to use
separate proofs (as the passage of the TAeaeietus shows that Theodorus did)
to establish the incommensurability with unity of 3, / 5, ... up to 17.

This fact indicates clearly that the general theorem in Eucl.\ x. 9 that squares
which have not to one another the ratio of a square number to a square number
have their sides incommensurable in length was not arrived at all at once, but
was, in the manner of the time, developed out of the separate consideration
of special cases (Hankel, p. 103).

The proposition x. 9 of Euclid is definitely ascribed by the scholiast to
Theaetetus. Theaetetus was a pupil of Theodorus, and it would seem clear
that the theorem was not known to Theodorus. Moreover the Platonic
passage itself (Theatt. 147 d sqq.) represents the young Theaetetus as striving
after a general conception of what we call a surd. ``The idea occurred to
me, seeing that square roots (Jim/hk) appeared to be unlimited in multitude,
to try to arrive at one collective term by which we could designate all these
square roots — I divided number in general into two classes. The number
which can be expressed as equal multiplied by equal (urw utokk) I likened
to a square in form, and I called it square and equilateral... .The intermediate
number, such as three, five, and any number which cannot be expressed as
equal multiplied by equal, but is either less times more or more times less, so
that it is always contained by a greater and less side, I likened to an oblong

figure and called an oblong number Such straight lines then as square the

equilateral and plane number I defined as length ((ijmt), and such as square
the oblong square roots (Svi-acts), as not being commensurable with the
others in length but only in the plane areas to which their squares are
equal. ``

There is further evidence of the contributions of Theaetetus to the theory
of incommensurables in a commentary on Eucl.\ x. discovered, in an Arabic
translation, by Woepcke (Aft moires prlsentis d fAcadimie des Sciences, xiv.,
1856, pp. 658 — 720). It is certain that this commentary is of Greek origin.
Woepcke conjectures that it was by Vettius Valens, an astronomer, apparently
of Antioch, and a contemporary of Claudius Ptolemy (and cent. a.d.).
Heiberg, with greater probability, thinks that we have here a fragment of the
commentary of Pappus (Euklid-studien, pp. 169 — 71), and this is rendered
practically certain by Suter (Die Afathemqtikcr und Astronomen der Araber
und ihre Werke, pp. 49 and in). This commentary states that the theory
of irrational magnitudes `` had its origin in the school of Pythagoras. It was
considerably developed by Theaetetus the Athenian, who gave proof, in this
part of mathematics, as in others, of ability which has been justly admired.
He was one of the most happily endowed of men, and gave himself up, with a
fine enthusiasm, to the investigation of the truths contained in these sciences,
as Plato bears witness for him in the work which he called after his name. As
for the exact distinctions of the above-named magnitudes and the rigorous
demonstrations of the propositions to which this theory gives rise, I believe
that they were chiefly established by this mathematician; and, later, the
great Apollonius, whose genius touched the highest point of excellence in
mathematics, added to these discoveries a number of remarkable theories
after many efforts and much labour.

``For Theaetetus had distinguished square roots [puissances must be the
Swdjiw! of the Platonic passage] commensurable in length from those which
are incommensurable, and had divided the well-known species of irrational
lines after the different means, assigning the medial to geometry, the binomial
to arithmetic, and the apotomc to harmony, as is stated by Eudemus the
Peripatetic.

`` As for Euclid, he set himself to give rigorous rules, which he established,
relative to commensurability and incommensurability in general; he made
precise the definitions and the distinctions between rational and irrational
magnitudes, he set out a great number of orders of irrational magnitudes, and
finally he clearly showed their whole extent.''

The allusion in the last words must apparently be to X. 115, where it is
proved that from the medial straight line an unlimited number of other
irrationals can be derived, all different from it and from one another.

The connexion between the media/ straight line and the geometric mean

is obvious, because it is in fact the mean proportional between two rational

straight lines ``commensurable in square only.'' Since (x+y) is the arithmetic

mean between x. y\ the reference to it of the binomial can be understood.

The connexion between the apotome and the harmonic mean is explained by

some propositions in the second book of the Arabic commentary. The

2xy
harmonic mean between x, y is - , and propositions of which Woepcke

quotes the enunciations prove that, if a rational or a medial area has for one
of its sides a binomial straight line, the other side will be an apotome of corre-
sponding order (these propositions are generalised from Euci. x. 1 r 1 — 4) ; the

fact is that -S~ = -~—. . (x- y).
x+y x* — yr

One other predecessor of Euclid appears to have written on irrationals,
though we know no more of the work than its title as handed down by
Diogenes Laertius 1 . According to this tradition, Democritus wrote iript
aXoyiov ypa/ifiar xal raa-rwr /3', two Books on irrational straight lints and
solids (or atoms). Hultsch (Neue Jahrbikhtr fur Phitologie vnd Padagogik,
1SS1, pp. 578 — 9) conjectures that the true reading may be -atpX iXoymy
ypa/ifium jcAmftw, `` on irrational broken lines.'' Hultsch seems to have
in mind straight lines divided into two parts one of which is rational
and the other irrational (`` Aus einer Art von Umkehr des Pythagoreischcn
Lehrsatzes iiber das rechtwtnklige Dreieck gieng zunchst mit Letchtigkeit
hervor, dass man eine Linie construtren konne, welche als irrational zu
bezeichnen ist, aber durch Brechung sich darstellen' lasst ais die Summe
einer rationalen und einer irrationalen Linie''). But I doubt the use of kXsoto?
in the sense of breaking one straight line into parts ; it should properly mean
a bent line, i.e. two straight lines forming an angle or broken short off at their
point of meeting. It is also to be observed that vamav is quoted as a
Democritean word (opposite to xivbv) in a fragment of Aristotle (202). I see
therefore no reason for questioning the correctness of the title of Democritus'
book as above quoted*.

I will here quote a valuable remark of Zeuthen's relating to the classifi-
cation of irrationals. He says (Gtschithte der Mathematik im Altertum und
Mittela/ter, p. 56) ``Since such roots of equations of the second degree as are
incommensurable with the given magnitudes cannot be expressed by means
of the latter and of numbers, it is conceivable that the Greeks, in exact
investigations, introduced no approximate values but worked on with the
magnitudes they had found, which were represented by straight lines obtained
by the construction corresponding to the solution of the equation. That is
exactly the same thing which happens when we do not evaluate roots but content
ourselves with expressing them by radical signs and other algebraical symbols.
But, inasmuch as one straight line looks like another, the Greeks did not get

1 Diog. Laert. ix. 47, p. 139 (ed. Cobet).
* Cf. unit. Vol. i., p. 41 j.

the same clear view of what they denoted (Le. by simple inspection) as our
system of symbols assures to us. For this reason it was necessary to under-
take a classification of the irrational magnitudes which had been arrived at by
successive solution of equations of the second degree.'' To much the same
effect Tannery wrote in 1883 (De la solution giometrique des firoblemes du
second degri avant Euclide in Aflmoirts de la Soci/li dts sciences physiques et
naturilUs de Bordeaux, 2* Serie, iv. pp. 395 — 416). Accordingly Book x.
formed a repository of results to which could be referred problems which
depended on the solution of certain types of equations, quadratic and biquad-
ratic but reducible to quadratics.
Consider the quadratic equations

x? ± iax . p ± 8 . p' = o,

where p is a rational straight line, and a, B are coefficients. Our quadratic
equations in algebra leave out the p ; but I put it in, because it has always to
be remembered that Euclid's x is a straight line, not an algebraical quantity,
and is therefore to be found in terms of, or in relation to, a certain assumed
rational straight tint, and also because with Euclid p may be not only of the

form a, where a represents a units of length, but also of the form */ —   <*>

which represents a length ``commensurable in square only'' with the unit of
length, or A where A represents a number (not square) of units of area.
The use therefore of p in our equations makes it unnecessary to multiply
different cases according to the relation of p to the unit of length, and has the
further advantage that, e.g., the expression p ± Jk , p is just as general as the
expression Jhp±J-p, since p covers the form Jh . p, both expressions
covering a length either commensurable in length, or ``commensurable in
square only,'' with the unit of length.

Now the positive roots of the quadratic equations
x? ± tax . p ± ff , p* = o
can only have the following forms

pfa + Va/S), jp-vVS) I
  r i = P (vV + B + a), X,' = p (>Ja r +~ -a) i '

The negative roots do not come in, since x must be a straight line. The
omission however to bring in negative roots constitutes no loss of generality,
since the Greeks would write the equation leading to negative roots in another
form so as to make them positive, i.e. they would change the sign of x in the
equation.

Now the positive roots jc,, x,', x t , x 3 may be classified according to the
character of the coefficents a, ft and their relation to one another.

I. Suppose that a, B do not contain any surds, i.e. are either integers or

of the form mjn, where m, n are integers.

Now in the expressions for x„ x,' it may be that

m*
( 1 ) 8 is of the form -= a 1 .
fr

Euclid expresses this by saying that the square on op exceeds the square

on pi/a 1 — B by the square on a straight line commensurable in length with ap.

In this case x, is, in Euclid's terminology, a first binomial straight line,

and x, a first afiotome.

m*
(a) In general, B not being of the form - a*,

x l is a. fourth binomial,

Xi 3. fourth apotomt.
Next, in the expressions for x„ x t it may be that
(1) B is equal to -j (V + 0), where w, « are integers, i.e. B is of the form

n*-m*

Euclid expresses this by saying that the square on pv<t'' + B exceeds the
sq uare o n op by the square on a straight line commensurable in length with
P-Jj + B.

In this case x, is, in Euclid's terminology, a second binomial,
x t a second apotomt.

(2) In general, B not being of the form -j , a',

x, is a fifth binomial,
x t a. fifth apotome.

II, Now suppose that a. is of the form »/ — , where m, n are integers, and

let us denote it by ,/A.
Then in this case

x, = p(J*+J-B), x( = p(,J-jJ>~),

xt m P ( jx+p + jk), x,' = P (jx~+p~- ,/A).

Thus x It x,' are of the same form as x t , x t '.

If <JX - 8 in jt,, *,' is not surd but of the form mjn, and if *A + B in x,, x f '
is not surd but of the form mjn, the roots are comprised among the forms
already shown, the first, second, fourth and fifth binomials and apotomes.

If s/k- B in jc,, Xi is surd, then

m*

(1) we may ha-ve of the form -j A, and in this case

x, is a rift/n/ binomial straight line,
*i' a third apotomt;

(2) in general, B not being of the form -3 A,

x, is a Jtxr/ji binomial straight line,
*l' a /£rM apotome.

With the expressions for *„ *,' the distinction between the third and sixth
binomials and apotomes is of course the distinction between the cases

(i) in which 8 = — (X + B), or 8 is of the form -* s A,

and <i) in which B is not of this form.

If we take the square root of the product of p and each of the six
binomials and six apotomes just classified, i.e.
in the six different forms that each may take, we find six new irrationals with
a positive sign separating the two terms, and six corresponding irrationals with
a negative sign. These are of course roots of the equations
x* ± 2a* 3 . p' + (3 . p* - o.
These irrationals really come before the others in Euclid's order (x. 56 —
41 for the positive sign and x. 73 — 78 for the negative sign). As we shall
see in due course, the straight lines actually found by Euclid are
r . p + JJb . p, the binomial (ij ix Su'o oyoimtiov)

and the apotome (aVorop,),
which are the positive roots of the biquadratic (reducible to a quadratic)
x>- 2 (1+p 1 . * s + (r -A)V = °-

2, i*p ± 6*p, the first bimedial (Ik Svo fito-wv irpcuny)
and the first apotomc of a medial (p,«7ijs a-voTo/ni) irpuinf),

which are the positive roots of

X* - 2 Jk (1 + k) p s . X> + k( I - A)V = o.

3. kp + r P, the second bimedial (in Buo p-iwav Stvripa)

and the second apotomc of a medial (/i«n;t (nroTo/17 Sivrtpa),
which are the positive roots of the equation

. k + \ . , jk-Xf ,

p

s/ i *jm t  '/ 1

Ti *

Jz V Ji + -zV 7T+-f

the major (irrational straight line) (ficifav)
and the m/nw (irrational straight line) ((Wm»),
which are the positive roots of the equation

X* - zp a . X 1 + r; p' = o.

r t + k 1 r

the ``side'' of a rational plus a medial (area) (pip-ov m! fUoov Svra/xmj)
and the `` *£ `` of a medial minus a rational area (in the Greek 7 /wni proS
jiitrov to oAor irotovcra),

which are the positive roots of the equation

2 JP

'** `` ~7=S P*  ** + V.T i p * = °'

fi P / k~ A*p /''

the ``side'' of the sum of two medial areas ( Suo /item BumpoTj)
and the `` j<aV `` a/ a medial minus a medial area (in the Greek ij pitra jUoou

pVcTOP to 0A0* TTOlOWa),

which are the positive roots of the equation

X l ~2jk.X 1 p'+\ -sp' = 0.

The above facts and formulae admit of being stated in a great variety of
ways according to the notation and the particular letters used. Consequently
the summaries which have been given of Eucl.\ x. by various writers differ
much in appearance while expressing the same thing in substance. The first
summary in algebraical form (and a very elaborate one) seems to have been
that of Cossali (Origin:, trasporto in Italia, primi progressi in essa del-
f Algebra, Vol. IL, pp. 242 — 65) who takes credit accordingly (p. 265). In
1794 Meier Hirsch published at Berlin an Algebraischcr Commcntar iiber das
sehente Buck der Ekmente des Euklides which gives the contents in algebraical
form but fails to give any indication of Euclid's methods, using modern forms
of proof only. In 1834 Poselger wrote a paper, Ueber das zthnte Buck der
Eletnente des Euklides, in which he pointed out the defects of Hirsch's repro-
duction and gave a summary of his own, which however, though nearer to
Euclid's form, is difficult to follow in consequence of an elaborate system of
abbreviations, and is open to the objection that it is not algebraical enough
to enable the character of Euclid's irrationals to be seen at a glance. Other
summaries will be found (1) in Nesselmann, Die Algebra der Griecken,
pp. 165 — 84; (2) in Loria, Le scienze esatte nelF antica Grecia, 19 14,
pp. 221—34 > (3) m Christensen's article ``Ueber Gleichungen vierten Grades
im zehnten Buch der Elemente Euklids `` in the Zeitsehri/t fiir Math, u,
Bhysik (Historisek-litteraristhe Abtneitung), xxxtv. (1889), pp. 201 — 17. The
only summary in English that I know is that in the Benny Cyclopaedia, under
`` Irrational quantity,'' by De Morgan, who yielded to none in his admiration of
Book x. `` Euclid investigates,'' says De Morgan, `` every possible variety of
lines which can be represented by J(Ja ± /b), a and b representing two
commensurable lines. ...This book has a completeness which none of the
others (not even the fifth) can boast of : and we could almost suspect that
Euclid, having arranged his materials in his own mind, and having completely
elaborated the loth Book, wrote the preceding books after it and did not live
to revise them thoroughly.''

Much attention was given to Book x. by the early algebraists. Thus
Leonardo of Pisa (fl. about 1 200 a.d.) wrote in the 14th section of his Liber
Abaci on the theory of irrationalities (de tractatu binomiorum et recisorum),
without however (except in treating of irrational trinomials and cubic irra-
tionalities) adding much to the substance of Book x. ; and, in investigating
the equation

3? + 2X* + I OX - 20,

propounded by Johannes of Palermo, he proved that none of the irrationals
in Eucl.\ x. would satisfy it (Hankel, pp. 344 — 6, Cantor, n„ p. 43). Luca
Paciuolo (about 1443 — 1514A.D.) in his algebra based himself largely, as he
himself expressly says, on Euclid x. (Cantor, n,, p. 293). Michael Stifel
(i486 or 1487 to 1567) wrote on irrational numbers in the second Book of
his Arithmetica Integra, which Book may be regarded, says Cantor (][,, p. 401),
as an elucidation of Eucl.\ x. The works of Cardano (1501 — 76) abound in
speculations regarding the irrationals of Euclid, as may be seen by reference to
Cossali (Vol. 11., especially pp. 268—78 and 383—99); the character of
the various odd and even powers of the binomials and apotomes is therein
investigated, and Cardano considers in detail of what particular forms of equa-
tions, quadratic, cubic, and biquadratic, each class of Euclidean irrationals can
be roots. Simon Stevin (48—1620) gave an Appendice des incommensurables
grandeurs en laqudle est sommairement (Uclari le content* du Dixiesme Livre
d'Euclide (Oevirrcs mathe''matiques, Leyde, 1634, pp. 218–22); he speaks thus
of the book : `` La difficulty du dixiesme Livre d'Euclide est a plusieurs
devenue en horreur, voire jusque  I'appeler la croix des mathematiciens,
matiere trop dure a digger, et en la quelle n'apercpivent aucune utility,'' a
passage quoted by Ijoria (op. at, p. 222).

It will naturally be asked, what use did the Greek geometers actually
make of the theory of irrationals developed at such length in Book x. ? The
answer is that Euclid himself, in Book X111., makes considerable use of the
second portion of Book x. dealing with the irrationals affected with a negative
sign, the apotomes etc. One object of Book xin. is to investigate the relation
of the sides of a pentagon inscribed in a circle arid of an icosahedron and
dodecahedron inscribed in a sphere to the diameter of the circle or sphere
respectively, supposed rational. The connexion with the regular pentagon of
a straight line cut in extreme and mean ratio is well known, and Euclid first
proves (xm. 6) that, if a rational straight line is so divided, the parts are the
irrationals called apotomes, the lesser part being a first apotome. Then, on
the assumption that the diameters of a circle and sphere respectively are
rational, he proves (xin. n) that the side of the inscribed regular pentagon is
the irrational straight line called minor, as is also the side of the inscribed
icosahedron (xin. 16), while the side of the inscribed dodecahedron is the
irrational called an apoiomt (xin. 17),

Of course the investigation in Book x, would not have been complete if
it had dealt only with the irrationals affected with a negative sign. Those
affected with the positive sign, the binomials etc., had also to be discussed,
and we find both portions of Book x., with its nomenclature, made use of by
Pappus in two propositions, of which it may be of interest to give th*- enun-
ciations here.

If, says Pappus (iv. p. 178), A3 be the rational diameter of a semicircle, and
if AB be produced to C so that B C is equal to the radius, if CD be a tangent,

if E be the middle point of the arc BD, and if CE be joined, then CE is the
irrational straight line called minor. As a matter of fact, if p is the radius,

If, again (p. 182), CD be equal to the radius of a semicircle supposed

rational, and if the tangent DB be drawn and the angle ADB be bisected by
DP meeting the circumference in F, then DF is the excess by which the
binomial exceeds the straight line which produces with a rational area a medial

IO BOOK X [x. DEKF, 1—4

whole (see EucL x. 77), (In the figure DK is the binomial and .O'' the other
irrational straight line.) As a matter of fact, if p be the radius,

Proclus tells us that Euclid left out, as alien to a selection of elements, the
discussion of the more complicated irrationals, ``the unordered irrationals which
Apollonius worked out more fully'' (Proclus, p. 74, 23), while the scholiast
to Book X. remarks that Euclid does not deal with all rational s and irrationals
but only the simplest kinds by the combination of which an infinite number
of irrationals are obtained, of which Apollonius also gave some. The author
of the commentary on Book x. found by Woepcke in an Arabic translation,
and above alluded to, also says that ``it was Apollonius who, beside the
ordered irrational magnitudes, showed the existence of the unordered nd by
accurate methods set forth a great number of them.'' It can only be vaguely
gathered, from such hints as the commentator proceeds to give, what the
character of the extension of the subject given by Apollonius may have been.
See note at end of Book.

\chapter8{Definitions}

\begin{enumerate}

\item\label{def:X_1} Those magnitudes are said to be commensurable
  which are measured by the same measure, and those incommensurable
  which cannot have any common measure.

\item\label{def:X_2} Straight lines are commensurable in square when
  the squares on them are measured by the same area, and
  incommensurable in square when the squares on them cannot possibly
  have any area as a common measure.

\item\label{def:X_3} With these hypotheses, it is proved that there
  exist straight lines infinite in multitude which are commensurable
  and incommensurable respectively, some in length only, and others in
  square also, with an assigned straight line. Let then the assigned
  straight line be called rational, and those straight lines which are
  commensurable with it, whether in length and in square or in square
  only, rational, but those which are incommensurable with it
  irrational,

\item\label{def:X_4} And let the square on the assigned straight line
  be called rational and those areas which are commensurable with it
  rational, but those which are incommensurable with it irrational,
  and the straight lines which produce them irrational, that is, in
  case the areas are squares, the sides themselves, but in case they
  are any other rectilineal figures, the straight lines on which are
  described squares equal to them.

\end{enumerate}

\section*{Definition 1}

 i'p.ificTpa pfyiOrj Xiyttat. ra ttZ avry t** T p*p ptftptrvfitva, AtrvfLptTpa €, w
div /vSi«t<u kqivov ftirpov ytvirrat

\section*{Definition 2}

KvO< tat owdfLU 0-vftji.tTpoi tUrtVf otav to. air* avrwv T<rpayu*a rtp artp wptw
fi.trpt]ttxt t antJ/urpot 8  OTav tow air' avrw T«Tpayti>fot9 ftYfStv cvSifflTal wpioj'
notfof nit pin' yiriuBtu.

Commensurable in square is in the Greek Wa/u« o-itrpM. In earlier
translations (e.g. Williamson's) 6i>ia/i« has been translated `` in power,'' but,
as the particular power represented by Swttpci in Greek geometry is square,
I have thought it best to use the latter word throughout. It will be observed
that Euclid's expression commensurable in square only (used in Def. 3 and
constantly) corresponds to what Plato makes Theaetetus call a square root
(Silt-nun) in the sense of a surd. If a is any straight line, a and ajm, or
aJm and a>jn (where m, n are integers or arithmetical fractions in their
lowest terms, proper or improper, but not square) are commensurable in square
only. Of course (as explained in the Porism to X. 10) all straight lines
commensurable in length ((irjKtt), in Euclid's phrase, are commensurable in
square also ; but not all straight lines which are commensurable in square are
commensurable in lengtk as well. On the other hand, straight lines incom-
mensurable in square are necessarily incommensurable in length also ; but not
all straight lines which are incommensurable in length are incommensurable
in square. In fact, straight lines which are commensurable in square only are
incommensurable in length, but obviously not incommensurable in square.

\section*{Definition 3}

TfWTGJV faOKUIlilriOV hf.tKVVta.lf OTl Tft irporttfan] tvBf.it;. vTtpyovtrw tvOtiai
TvkrjOtL axttpot trvfiftttpot tt Ktu aui'fifjutptn at fiiv p-rjKtt ftovov, at S* Kal (Wti/te t.
xaAiiirStii ovr ij ftcy rptrritura liim fn)rq, Kal til ruiJrj; froji/ittpoi tin /XTjKtt xal
owdjjLtt t iTt Stjraacr futvoy fiip-fit, at di tavrvj axrvftfittTpoi tlAoyut ku A<ttrtifirav.

The first sentence of the definition is decidedly elliptical. It should,
strictly speaking, assert that `` with a given straight line there are an infinite
number of straight lines which are (t) commensurable either (a) in square
only or (b) in square and in length also, and (2) incommensurable, either
(a) in length only or (b) in length and in square also.''

The relativity of the terms rational and irrational is well brought out in
this definition. We may set out any straight line and call it rational, and it
is then with reference to this assumed rational straight line that others are
called rational or irrational.

We should carefully note that the signification of rationale Euclid is wider
than in our terminology. With him, not only is a straight line commensurable in
length with a rational straight line rational, but a straight line is rational which
is commensurable with a rational straight line in square only. That is, if p is a

rational straight line, not only is — p rational, where m, n are integers and

mjn in its lowest terms is not square, but */ — .pis rational also. We should

in this case call . / — . p irrational. It would appear that Euclid's termino-
logy here differed as much from that of his predecessors as it does from
ours. We are familiar with the phrase apfajros hiAfurptn rift irt/urciSM by
which Plato (evidently after the Pythagoreans) describes the diagonal of a
square on a straight line containing 5 units of length. This `` inexpressible
diameter of five (squared) `` means V50, in contrast to the ptp-7 ap.cTp<«, the
`` expressible diameter `` of the same square, by which is meant the approxi-
mation 50–1, or 7. Thus for Euclid's predecessors /.p would

apparently not have been rational but apptyrm, `` inexpressible,'' i.e. irrational.

I shall throughout my notes on this Book denote a rational straight line in
Euclid's sense by p, and by p and cr when two different rational straight lines are
required. Wherever then I use p or tr, it must be remembered that p, cr may
have either of the forms a, 'i . a, where a represents a units of length, a being
either an integer or of the form m' n, where m, n are both integers, and k is an
integer or of the form mjn (where both m, n are integers) but not square. In
other words, p, ir may have either of the forms a or J A, where A represents
A units of arm and A is integral or of the form mjn, where m, n are both
integers. It has been the habit of writers to give a and J a as the alternative
forms of p, but I shall always use J A for the second in order to keep the
dimensions right, because it must be borne in mind throughout that p is an
irrational straight linr.

As Euclid extends the signification of rational (/Wtos, literally expressible),
so he limits the scope of the term dAo-yov (literally having no ratio) as applied
to straight lines. That this limitation was started by himself may perhaps be
inferred from the form of words `` let straight lines incommensurable with it
be tailed irrational.'' Irrational straight lines then are with Euclid straight lines
commensurable neither in length nor in square with the assumed rational
straight line- Jk . a where k is not square is not irrational; k . a is irrational,
and so (as we shall see later on) is (y/6± s/*)a.

\section*{Definition 4}

Kat to pep airo T17S TrpaTiQitoifi tvtiiis TCTpayowov pyrov, «ai t« rounp
(rvfijarpa /nTit, ra Si Toi-rw atrvp/icrpa dAoya Kakiitrdv, kcu ai cWa/Kwu aura
aAoyot, (i piv T«Tpay*>™ *''), aurai at irAtvpai, *£ St trtpa riva ivffiiypap-pa,, ai
ttra avrot? TCTpayaiva apaypa<owrai.

As applied to areas, the terms rational and irrational have, on the other
hand, the same sense with Euclid as we should attach to them. According
to Euclid, if p is a rational straight line in his sense, p 1 is rational and any
area commensurable with it, i.e. of the form kp* (where k is an integer, or of
the form mjn, where m, n are integers), is rational ; but any area of the form
n/h . p* is irrational. Euclid's rational area thus contains A units of area,
where A is an integer or of the form mjn, where m, n are integers ; and his
irrational area is of the form /h.A. His irrational area is then connected
with his irrational straight line by making the latter the square root of the
former. This would give us for the irrational straight line if A . J A, which of
course includes ijk.a.

at BwafitKu aura are the straight lines the squares on which are equal to
the areas, in accordance' with the regular meaning of SvyatrOai, It is scarcely
possible, in a book written in geometrical language, to translate Suwi/ifn) as
the square root (of an area) and Jwmrfai as to be the square root (of an area),
although I can use the term `` square root `` when in my notes I am using an
algebraical expression to represent an area ; I shall therefore hereafter use the
word ``side'' for Swa/ttVij and ``10 be the side of'' for huvaaOai, so that
``side'' will in such expressions be a short way of expressing the ``side of
a square equal to (an area).'' In this particular passage it is not quite practi-
cable to use the words `` side of'' or `` straight line the square on which is equal
to,'' for these expressions occur just afterwards for two alternatives which the
word SumijmVi) covers. I have therefore exceptionally translated `` the straight
lines which produce them `` (i.e. if squares are described upon them as sides).

<ii ``uro. avruw TtTjiiytava. avaypuHiiwat, literally `` the (straight lines) which
describe squares equal to them'' : a peculiar use of the active of draypdijmr,
the meaning being of course `` the straight lines on which are described the
squares'' which are equal to the rectilineal figures.

\part{Book X. Propositions}

\begin{proposition}
\label{prop:X_1}

\begin{statement}
Two unequal magnitudes being set out, if from the greater
there be subtracted a magnitude greater than its half, and from
that -which is left a magnitude greater than its half and if
this process be repeated continually, there will be left some
magnitude which will be less than the lesser magnitude set out.
\end{statement}

\begin{proof}

Let AB, C be two unequal magnitudes of which AB is
the greater :

I say that, if from AB there be * — B °

subtracted a magnitude greater o -t + e

than its half, and from that which

is left a magnitude greater than its half, and if this process be
repeated continually, there will be left some magnitude which
will be less than the magnitude C.

For C if multiplied will sometime be greater than AB.

[cf. v, Def. 4]

Let it be multiplied, and let DE be a multiple of C, and
greater than. AB ;

let DE be divided into the parts DE, EG, GE equal to C,
from AB let there be subtracted BH greater than its half,
and, from AH, HK greater than its half,
and let this process be repeated continually until the divisions
in AB are equal in multitude with the divisions in DE.

Let, then, AK, KH, HB be divisions which are equal in
multitude with DE, EG, GE.

Now, since DE is greater than AB,
and from DE there has been subtracted EG less than its
half,

and, from AB, BH greater than its half,
therefore the remainder GD is greater than the remainder HA.

And, since GD is greater than HA,
and there has been subtracted, from GD, the half GP,
and, from HA, HK greater than its half,
therefore the remainder DFs greater than the remainder A K.

But DF is equal to C ;
therefore C is also greater than A K.

Therefore AK is less than C.

Therefore there is left of the magnitude AB the magnitude
AK which is less than the lesser magnitude set out, namely C.

Q.E.D.

And the theorem can be similarly proved even if the parts
subtracted be halves.
\end{proof}

\begin{notes}

This proposition will be remembered because it is the lemma required in
Euclid's proof of XII. 2 to the effect that circles are to one another as the
squares on their diameters. Some writers appear to be under the impression
that xii. 2 and the other propositions in Book xii. in which the method of
exhaustion is used are the only places where Euclid makes use of X. t ; and it
is commonly remarked that x. 1 might just as well have been deferred till the
beginning of Book XII. Even Cantor (Geseh. d. Math, ij, p. 269) remarks
that `` Euclid draws no inference from it [x. 1], not even that which we should
more than anything else expect, namely that, if two magnitudes are incom-
mensurable, we can always form a magnitude commensurable with the first
which shall differ from the second magnitude by as little as we please.'' But,
so far from making no use of x. 1 before xii. 2, Euclid actually uses it in the
very next proposition, x. 2. This being so, as the next note will show, it
follows that, since x. 2 gives the criterion for the incommensurability of two
magnitudes (a very necessary preliminary to the study of in com m ens u rabies),
X: 1 comes exactly where it should be.

Euclid uses X. 1 to prove not only xn. 2 but xii. 5 (that pyramids with the
same height and triangular bases are to one another as their bases), by means
of which he proves (xn. 7 and Por.) that any pyramid is a third part of the
prism which has the same base and equal height, and xn. 10 (that any cone
is a third part of the cylinder which has the same base and equal height),
besides other similar propositions. Now xn. 7 Por. and xii. 10 are theorems
specifically attributed to Eudoxus by Archimedes (On the Sphere and Cylinder,
Preface), who says in another place (Quadrature of the Parabola, Preface) that
the first of the two, and the theorem that circles are to one another as the
squares on their diameters, were proved by means of a certain lemma which
he states as follows : ``Of unequal lines, unequal surfaces, or unequal solids,
the greater exceeds the less by such a magnitude as is capable, if added
[continually] to itself, of exceeding any magnitude of those whfch are
comparable with one another,'' i.e. of magnitudes of the same kind as the
original magnitudes. Archimedes also says (he. eit.) that the second of
the two theorems which he attributes to Eudoxus (End. xii. 10) was
proved by means of ``a lemma similar to the aforesaid.'' The lemma
stated thus by Archimedes is decidedly different from x. i, which, however,
Archimedes himself uses several times, while he refers to the use of it
in xii. 2 (On the Sphere and Cylinder, I. 6). As I have before suggested
( The Works of Archimedes, p. xlviii), the apparent difficulty caused by the
mention of two lemmas in connexion with the theorem of Eucl.\ xn. 2 may be
explained by reference to the proof of x. 1. Euclid there takes the lesser
magnitude and says that it is possible, by multiplying it, to make it some time
exceed the greater, and this statement he clearly bases on the 4th definition of
Book v., to the effect that ``magnitudes are said to bear a ratio to one another
which can, if multiplied, exceed one another.'' Since then the smaller
magnitude in x. 1 may be regarded as the difference between some two
unequal magnitudes, it is clear that the lemma stated by Archimedes is in
substance used to prove the lemma in x. 1, which appears to play so much
larger a part in the investigations of quadrature and cubature which have come
down to us.

Besides being employed in Eucl.\ x. 1, the ``Axiom of Archimedes'' appears
in Aristotle, who also practically quotes the result of x. 1 itself. Thus he
says, Physics vm. 10, 266 b 2, `` By continually adding to a finite (magnitude)
1 shall exceed any definite (magnitude), and similarly by 'continually subtract-
ing from it I shall arrive at something less than it,'' and Hid. til. 7, 207 b 10
`` For bisections of a magnitude are endless.'' It is thus somewhat misleading
to use the term ``Archimedes' Axiom'' for the ``lemma'' quoted by him,
since he makes no claim to be the discoverer of it, and it was obviously much
earlier.

Stolz (see G. Vitali in Questioni riguardanti le matematiche elemenfari, 1.,
pp. 1 29—30) showed how to prove the so-called Axiom or Postulate of Archi-
medes by means of the Postulate of Dedekind, thus. Suppose the two magni-
tudes to be straight lines. It is required to prove that, given two straight lines,
(here always exists a multiple 0/ the smaller which is greater than the other.

Let the straight lines be so placed that they have a common extremity and
the smaller lies along the other on the same side of the common extremity.

If AC be the greater and AB the smaller, we have to prove that there
exists an integral number n such that n. AB> AC.

Suppose that this is not true but that there are some points, like B, not
coincident with the extremity A, and such that, n being any integer however
great, n . AB-cAC; and we have to prove that this assumption leads to an
absurdity.

h iyi k i

The points of AC may be regarded as distributed into two ``parts,'' namely

(1) points Iffor which there exists no integer n such that «'. AH> AC-,

(2) points K for which an integer n does exist such that n, AK> AC.

This division into parts satisfies the conditions for the application of
Dedekind's Postulate, and therefore there exists a point M such that the
points of AM belong to the first part and those of MC to the second part.

Take now a point Kon MC such that MY< AM. The middle point (X)
of A Kwill fall between A and M and will therefore belong to the first part;
but, since there exists an integer a such that n . AV> AC, it follows that
2n . AX>- AC: which is contrary to the hypothesis.

\end{notes}

\end{proposition}

\begin{proposition}
\label{propX_2}

\begin{statement}
If when the less of two unequal magnitudes is continually
subtracted in turn from the greater, that which is left never
measures the one before it, the magnitudes will be incom-
mensurable.
\end{statement}

\begin{proof}

For, there being two unequal magnitudes AB, CD, and
AB being the less, when the less is continually subtracted
in turn from the greater, let that which is left over never
measure the one before it ;
I say that the magnitudes AB, CD are incommensurable.

c £ o

For, if they are commensurable, some magnitude will
measure them.

Let a magnitude measure them, if possible, and let it be E ;
let AB, measuring FD, leave CTless than itself,
let CF measuring BG, leave AG less than itself,
and let this process be repeated continually, until there is left
some magnitude which is less than E.

Suppose this done, and let there be left AG less than E.

Then, since E measures AB,
while AB measures DF,
therefore E will also measure FD.

But it measures the whole CD also ;
therefore it will also measure the remainder CF.

But CF measures BG ;
therefore E also measures BG.

But it measures the whole AB also ;
therefore it will also measure the remainder AG, the greater
the less :
which is impossible.

Therefore no magnitude will measure the magnitudes AB,
CD;
therefore the magnitudes AB, CD are incommensurable.

[x. Def. 1]

Therefore etc.
\end{proof}

\begin{notes}

This proposition states the test for incommensurable magnitudes, founded
on the usual operation for rinding the greatest common measure. The sign
of the incommensurability of two magnitudes is that this operation never
comes to an end, while the successive remainders become smaller and smaller
until they are less than any assigned magnitude.

Observe that Euclid says `` let this process be repeated continually until
there is left some magnitude which is less than E.'' Here he evidently
assumes that the process will some time produce a remainder less than any
assigned magnitude E. Now this is by no means self-evident, and yet
Heiberg (though so careful to supply references) and Lorenz do not refer to
the basis of the assumption, which is in reality x. I, as Billingsley and
Williamson were shrewd enough to see. The fact is that, if we set off a
smaller magnitude once or oftcner along a greater which it does not exactly
measure, until the remainder is less than the smaller magnitude, we take away
from the greater more than its half. Thus, in the figure, FD is more than the
half of CD, and BG more than the half of AB. If we continued the process,
AG marked off along CF as many times as possible would cut off more than
its half; next, more than half AG would be cut off, and so on. Hence along
CD, AB alternately the process would cut off more than half, then more than
half the remainder and so on, so that on both lines we should ultimately
arrive at a remainder less than any assigned length.

The method of finding the greatest common measure exhibited in this
proposition and the next is of course again the same as that which we use and
which may be shown thus :

b)u(p

Pi
')*(?

d)c(r
rd

The proof too is the same as ours, taking just the same form, as shown in the
notes to the similar propositions vii. i, 2 above. In the present case the
hypothesis is that the process never stops, and it is required to prove that a, b
cannot in that case have any common measure, as/ For suppose that/ is a
common measure, and suppose the process to be continued until the remainder
t, say, is less than/

Then, since /measures a, b, it measures a -pb, or c.

Since/ measures b, c, it measures b-qc, or d; and, since/ measures c, d,
it measures c—rd, or e \ which is impossible, since e<f.

Euclid assumes as axiomatic that, if/ measures a, />, it measures ma ± nb.

In practice, of course, it is often unnecessary to carry the process far in
order to see that it will never stop, and consequently that the magnitudes are
incommensurable. A good instance is pointed out by Allman (Greek Geometry
from Thates to Euclid, pp. 4a, 137—8). Euclid proves in XHI. 5 that, if AB
be cut in extreme and mean ratio at C, and if

DA equal to AC be added, then DB is also cut D A c B

in extreme and mean ratio at A. This is indeed

obvious from the proof of 11. t r. It follows conversely that, if BD is cut into
extreme and mean ratio at A, and AC, equal to the lesser segment AD, he
subtracted from the greater AB, AB is similarly divided at C. We can then
mark off from AC a. portion equal to CB, and AC will then be similarly
divided, and so on. Now the greater segment in a line thus divided is greater
than half the line, but it follows from xni. 3 that it is less than twice the
lesser segment, i.e. the lesser segment can never be marked off more than
once from the greater. Our process of marking off the lesser segment from the
greater continually is thus exactly that of finding the greatest common measure.
If, therefore, the segments were commensurable, the process would stop. But
it clearly does not ; therefore the segments are incommensurable.

Allman expresses the opinion that it was rather in connexion with the line
cut in extreme and mean ratio than with reference to the diagonal and side
of a square that the Pythagoreans discovered the incommensurable. But the
evidence seems to put it beyond doubt that the Pythagoreans did discover
the incommensurability of J2 and devoted much attention to this particular
case. The view of Allman does not therefore commend itself to me, though
it is likely enough that the Pythagoreans were aware of the incommensura-
bility of the segments of a line cut in extreme and mean ratio. At all events
the Pythagoreans could hardly have carried their investigations into the in-
commensurability of the segments of this line very far, since Theaetetus is
said to have made the first classification of irrationals, and to him is also, with
reasonable probability, attributed the substance of the first part of Eucl.\ xni.,
in the sixth proposition of which occurs the proof that the segments of a
rational straight line cut in extreme and mean ratio are apotomes.

Again, the incommensurability of Ji can be proved by a method
practically equivalent to that of x. 2, and without carrying the process very
far. This method is given in Chrystal's text-book of Algebra (1. p. 270). Let d, a be the
diagonal and side respectively of a square
A BCD. Mark off AF along A C equal to a.
Draw FE at right angles to AC meeting BC
in E.

It is easily proved that

BE = EF= FC, ©/

CF=AC-AB = d-a (r). v

CE=C3- CF=a-(d-a)

= xa-d (z).

Suppose, if possible, that d, a are commensurable. If d, a are both
commensurably expressible in terms of any finite unit, each must be an
integral multiple of a certain finite unit

But from (1) it follows that CF, and from (2) it follows that CE, is an
integral multiple of the same unit.

And CF, CE are the side and diagonal of a square CFEG, the side of
which is less than half the side of the original square. 1 f a u d x are the side and
diagonal of this square,

di = ia-d I '

Similarly we can form a square with side a   and diagonal d, which are less
than half a„ d, respectively, and a„ d, must be integral multiples of the same
unit, where

a t = d x -a lt
and this process may be continued indefinitely until (x. i) we have a square
as small as we please, the side and diagonal of which are integral multiples of
a finite unit : which is absurd.

Therefore a, d are incommensurable.

It will be observed that this method is the opposite of that shown in the
Pythagorean series of side- and diagonal-numbers, the squares being
successively smaller instead of larger.

\end{notes}

\end{proposition}

\begin{proposition}
\label{propX_3}

\begin{statement}
Given two commensurable magnitudes, to find their greatest
common measure.
\end{statement}

\begin{proof}

Let the two given commensurable magnitudes be AB, CD
of which AB is the less ;

thus it is required to find the greatest common measure of
AB, CD.

Now the magnitude AB either measures CD or it does
not.

If then it measures it — and it measures itself also — AB is
a common measure of AB, CD,

And it is manifest that it is also the greatest ;
for a greater magnitude than the magnitude AB will not
measure AB.

-a
-a

-9- .4

Next, let AB not measure CD.

Then, if the less be continually subtracted in turn from
the greater, that which is left over will sometime measure
the one before it, because AB, CD are not incommensurable;

[cf. x. 2]
let AB, measuring ED, leave EC less than itself,

let EC, measuring FB, leave AF less than itself,
and let AF measure CE.

Since, then, AF measures CE,
while CE measures FB,
therefore AF will also measure FB.

But it measures itself also ;
therefore AF will also measure the whole AB.

But AB measures DE ;
therefore AF will also measure ED.

But it measures CE also ;
therefore it also measures the whole CD.

Therefore AF is a common measure of AB, CD.

I say next that it is also the greatest.

For, if not, there will be some magnitude greater than AF
which will measure AB, CD.

Let it be G.

Since then G measures AB,
while AB measures ED,
therefore G will also measure ED.

But it measures the whole CD also ;
therefore G will also measure the remainder CE.

But CE measures FB ;
therefore G will also measure FB.

But it measures the whole AB also,
and it will therefore measure the remainder AF, the greater
the less :
which is impossible.

Therefore no magnitude greater than AF will measure
AB, CD;
therefore AF'is the greatest common measure of AB, CD.

Therefore the greatest common measure of the two given
commensurable magnitudes AB, CD has been found.
\end{proof}

\begin{porism*}

From this it is manifest that, if a magnitude
measure two magnitudes, it will also measure their greatest
common measure.

\end{porism*}

\begin{notes}

This proposition for two commensurable magnitudes is, mutatis mutandis,
exactly the same as vu. 2 for numbers. We have the process

b)a(p
pb

T)b(q
qc

rd
where c is equal to rd and therefore there is no remainder,

It is then proved that d is a common measure of a, 6; and next, by a
reiurtio ad adsardum, that it is the greatat common measure, since any
common measure must measure d, and no magnitude greater than d can
measure d. The rtductw ad abmrdum is of course one of form only.

The Porism corresponds exactly to the Porism to vn. a.

The process of finding the greatest common measure is probably given in
this Book, not only for the sake of completeness, but because in x. 5 a
common measure of two magnitudes A, B is assumed and used, and therefore
it is important to show that such a measure can be found if not already
known.

\end{notes}

\end{proposition}

\begin{proposition}
\label{propX_4}

\begin{statement}
Given three commensurable magnitudes, to find their greatest
common measure.
\end{statement}

\begin{proof}

Let A, B, C be the three given commensurable magnitudes;
thus it is required to find the greatest
common measure of A, B, C. A

Let the greatest common measure b

of the two magnitudes A, B be taken, c-

and let it be D ; [x. 3] D E F

then D either measures C, or does
not measure it.

First, let it measure it.

Since then D measures C,
while it also measures A, B,
therefore D is a common measure of A, B, C.

And it is manifest that it is also the greatest ;
for a greater magnitude than the magnitude D does not
measure A, B.

Next, let D not measure C.

\ say first that C, D are commensurable.

For, since A, B, C are commensurable,

some magnitude will measure them,

and this will of course measure A, B also ;

so that it will also measure the greatest common measure of
A, B, namely D. [x. 3, Por.]

But it also measures C;

so that the said magnitude will measure C, D ;

therefore C, D are commensurable.

Now let their greatest common measure be taken, and let
it be E. [x. 3]

Since then E measures D,
while D measures A, B t
therefore E will also measure A, B.

But it measures C also ;
therefore E measures A, B, C ;
therefore E is a common measure of A, B, C.

I say next that it is also the greatest.

For, if possible, Jet there be some magnitude F greater than
E, and let it measure A, B, C.

Now, since F measures A, B, C,
it will also measure A, B,
and will measure the greatest common measure of A, B.

[x. 3. Por.]

But the greatest common measure of A, B is D;

therefore F measures D.

But it measures C also ;
therefore F measures C, D ;

therefore F will also measure the greatest common measure
of C, D. [x, 3, Por.]

But that is E;
therefore F will measure E, the greater the less :
which is impossible.

Therefore no magnitude greater than the magnitude E
will measure A, B, C;

therefore E is the greatest common measure of A, B, C i D
do not measure C,
and, if it measure it, D is itself the greatest common measure.

Therefore the greatest common measure of the three given
commensurable magnitudes has been found.

Porism. From this it is manifest that, if a magnitude
measure three magnitudes, it will also measure their greatest
common measure.

Similarly too, with more magnitudes, the greatest common
measure can be found, and the porism can be extended.
\end{proof}

\begin{notes}

This proposition again corresponds exactly to vn, 3 for numbers. As
there Euclid thinks it necessary to prove that, a, b, c not being prime to one
another, d and c are also not prime to one another, so here he thinks it
necessary to prove that d, c are commensurable, as they must be since any
common measure of a, b must be a measure of their greatest common
measure d (x. 3, Por.).

The argument in the proof that t, the greatest common measure of d, c, is
the greatest common measure of a, b, c, is the same as that in vn. 3 and x. 3.

The Porism contains the extension of the process to the case of four
or more magnitudes, corresponding to Heron's remark with regard to the
similar extension of vn. 3 to the case of four or more numbers.

\end{notes}

\end{proposition}

\begin{proposition}
\label{propX_5}

\begin{statement}
Commensurable magnitudes have to one another the ratio
which a number has to a number.
\end{statement}

\begin{proof}

Let A, B be commensurable magnitudes ;

I say that A has to B the ratio which a number has to a
number.

For, since A, B are commensurable, some magnitude will
measure them.

Let it measure them, and let it be C,

And, as many times as C measures A, so many units let
there be in D ;

and, as many times as C measures B, so many units let there
be in E.

Since then C measures A according to the units in D,

while the unit also measures D according to the units in it,

therefore the unit measures the number D the same number
of times as the magnitude C measures A ;

therefore, as C is to A, so is the unit to D ; [vn, Def. *o)

therefore, inversely, as A is to C, so is D to the unit.

[cf. v. 7, Por.]
Again, since C measures B according to the units in E,
while the unit also measures E according to the units in it,
therefore the unit measures E the same number of times as C
measures B ;

therefore, as C is to B, so is the unit to E.
But it was also proved that,

as A is to C, so is D to the unit ;
therefore, ex aequali,

as A is to B, so is the number D to E. [v. 22J

Therefore the commensurable magnitudes A, 3 have to
one another the ratio which the number D has to the number E.
\end{proof}

\begin{notes}

The argument is as follows. If a, b be commensurable magnitudes, they
have some common measure c, and

a-mc t
b=t«e,
where m, n are integers.

It follows that c:a= 1 :m (i),

or, inversely, a : c = m : t j

and also that c : b = 1 : «,

so that, ex aequali, a :b = m;n.

It will be observed that, in stating the proportion (1), Euclid is merely
expressing the fact that a is the same multiple of c that m is of 1. In other
words, he rests the statement on the definition of proportion in vil, Def. 20.
This, however, is applicable only to four numbers, and c, a are not numbers but
magnitudes. Hence the statement of the proportion is not legitimate unless
it is proved that it is true in the sense of v. Det. 5 with regard to magnitudes
in general, the numbers 1, m being magnitudes. Similarly with regard to the
other proportions in the proposition.

There is, therefore, a hiatus. Euclid ought to have proved that magnitudes
which are proportional in the sense of vn. Def. 10 are also proportional in the
sense of v. Def. 5, or that the proportion of numbers is included in the
proportion of magnitudes as a particular case. Simson has proved this in his
Proposition C inserted in Book v. (see Vol. 11. pp. 1 26 — 8). The portion of
that proposition which is required here is the proof that,
if a = mb

c = md)'
then a 1 W :   M the sense of v. Def.  .

Take any equimultiples pa, pe of a, c and any equimultiples qb, qd of b, d.

Now pa =pmb 1

pc=pmd) '

But, according as pmb > = <qb, pmd > = <qd.
Therefore, according as pa > = < qb, pm > = < qd.

And pa, pc are any equimultiples of a, c, and qb, qd any equimultiples
of b, d.

Therefore a:b = ed. [v, Def. S-]

\end{notes}

\end{proposition}

\begin{proposition}
\label{propX_6}

\begin{statement}
If two magnitudes have to one another the ratio which a
number has to a number, the magnitudes will be commensurable.
\end{statement}

\begin{proof}

For let the two magnitudes A, B have to one another the
ratio which the number D has to the number E
s I say that the magnitudes A, B are commensurable.
A -> i b c

For let A be divided into as many equal parts as there
are units in D,

and let C be equal to one of them ;

and let F be made up of as many magnitudes equal to C as
io there are units in E.

Since then there are in A as many magnitudes equal to C
as there are units in D,

whatever part the unit is of D, the same part is C of A also ;
therefore, as C is to A, so is the unit to D. [vn. Def. 20]

But the unit measures the number D ;
therefore Calso measures A.

And since, as C is to A, so is the unit to D,
therefore, inversely, as A is to C, so is the number D to the
unit. [cf. v. 7, Por.]

to Again, since there are in F as many magnitudes equal
to C as there are units in E,

therefore, as C is to F, so is the unit to E. [vu. Def. 20]

But it was also proved that,

as A is to C, so is D to the unit ;
therefore, ex aeouali, as A is to F, so is D to E. [v. 22]

But, as D is to E, so is A to B ;
therefore also, as A is to B t so is it to also. [v. n]

Therefore A has the same ratio to each of the magnitudes
B, F;
therefore B is equal to F. [v. 9]

But C measures F;
therefore it measures B also.

Further it measures A also ;
therefore C measures A, B.

Therefore A is commensurable with B.
Therefore etc.
\end{proof}

\begin{porism*}

From this it is manifest that, if there be two
numbers, as D, E, and a straight line, as A, it is possible to
make a straight line F''\ such that the given straight line is to
it as the number D is to the number E,

And, if a mean proportional be also taken between A, F,
as B,

as A is to F, so will the square on A be to the square on B,

that is, as the first is to the third, so is the figure on the first

4s to that which is similar and similarly described on the second.

[vj. 19, Por.]
But, as A is to F, so is the number D to the number E
therefore it has been contrived that, as the number D is to
the number E, so also is the figure on the straight line A to
the figure on the straight line B.

Q.E.D.

\end{porism*}

\begin{annotations}

15. But the unit measures the number D ; therefore C also measures A.
These words are redundant, thmigh they are apparently found in all the MSs.
\end{annotations}

\begin{notes}

The same link to connect the proportion of numbers with the proportion
of magnitudes as was necessary in the last proposition is necessary here. This
being premised, the argument is as follows.

Suppose a : b = m:rt,

where m, n are (integral) numbers.

Divide a into m parts, each equal to e, say,
so that a = mc.

Now take d such that d = nc.

Therefore we have ac-m;,

and c : d = 1 : «,

so that, ex aequali, a :d=m : »

= a : , by hypothesis.

Therefore b-d-- ne,
so that c measures b n times, and a, b are commensurable.

The Porism is often used in the later propositions. It follows (1) that, if
a be a given straight line, and m, n any numbers, a straight line x can be
found such that

a : x — m : n.
(a) We can find a straight line such that

a* s y* = m : n.
For we have only to take y, a mean proportional between a and x, as
previously found, in which case a, y, x are in continued proportion and
[v. Def. 9]

t'.y = a : x
= m : «.

\end{notes}

\end{proposition}

\begin{proposition}
\label{propX_7}

\begin{statement}
Incommensurable magnitudes have not to one another the
ratio which a number has to a number.
\end{statement}

\begin{proof}

Let A, B be incommensurable magnitudes ;
I say that A has not to B the ratio which a number has to a
number.

For, if A has to B the ratio which a number has to a
number, A will be commensurable with B. [x 6]

But it is not ;
therefore A has not to B the ratio which a

number has to a number.

Therefore etc.
\end{proof}

\end{proposition}

\begin{proposition}
\label{propX_8}

\begin{statement}
If two magnitudes have not to one another the ratio which
a number has to a number, the magnitudes will be incom-
mensurable.
\end{statement}

\begin{proof}

For let the two magnitudes A, B not have to one another
the ratio which a number has to a number ;

I say that the magnitudes A, B are incom- — -

mensurable. ~

For, if they are commensurable, A will have to B the
ratio which a number has to a number. [x. 5]

But it has not ;
therefore the magnitudes A, B are incommensurable.

Therefore etc.
\end{proof}

\end{proposition}

\begin{proposition}
\label{propX_9}

\begin{statement}
The squares on straight lines commensurable in length have
to one another the ratio which a square number has to a square
number; and squares which have to one another the ratio
which a square number has to a square number will also have
their sides commensurable in length. But the squares on
straight lines incommensurable in length have not to one
another the ratio which a square number has to a square
number ; and squares which have not to one another the ratio
which a square number has to a square number will not have
their sides commensurable in length either.
\end{statement}

\begin{proof}

For let A, B be commensurable in length ;
I say that the square on A

has to the square on B the ~

ratio which a square number ~~

has to a square number.

For, since A is commensurable in length with B,
therefore A has to B the ratio which a number has to a
number. [x. 5]

Let it have to it the ratio which C has to D.

Since then, as A is to B, so is C to D,
while the ratio of the square on A to the square on B is
duplicate of the ratio of A to B,

for similar figures are in the duplicate ratio of their corre-
sponding sides; [vi. zo, Por.]
and the ratio of the square on C to the square on D is duplicate
of the ratio of C to D,

for between two square numbers there is one mean proportional
number, and the square number has to the square number the
ratio duplicate of that which the side has to the side ; [vm. n]
therefore also, as the square on A is to the square on B, so
is the square on C to the square on D.

Next, as the square on A is to the square on B, so let
the square on C be to the square on D ;
I say that A is commensurable in length with B.

For since, as the square on A is to the square on B, so is
the square on C to the square on D,

while the ratio of the square on A to the square on B is
duplicate of the ratio of A to B,

and the ratio of the square on C to the square on D is duplicate
of the ratio of C to D,
therefore also, as A is to B, so is C to D.

Therefore A has to B the ratio which the number C has
to the number D ;
therefore A is commensurable in length with B. [x. 6]

Next, let A be incommensurable in length with B ;
I say that the square on A has not to the square on B the
ratio which a square number has to a square number.

For, if the square on A has to the square on B the ratio
which a square number has to a square number, A will be
commensurable with B.

But it is not ;
therefore the square on A has not to the square on B the
ratio which a square number has to a square number.

Again, let the square on A not have to the square on B
the ratio which a square number has to a square number ;
I say that A is incommensurable in length with B.

For, if A is commensurable with B, the square on A will
have to the square on B the ratio which a square number has
to a square number.

But it has not ;
therefore A is not commensurable in length with B.

Therefore etc.

Porism. And it is manifest from what has been proved
that straight lines commensurable in length are always com-
mensurable in square also, but those commensurable in square
are not always commensurable in length also.

[Lemma. It has been proved In the arithmetical books
that similar plane numbers have to one another the ratio
which a square number has to a square number, [vm. 26]

and that, if two numbers have to one another the ratio which
a square number has to a square number, they are similar
plane numbers. [Converse of vm. a6]

And it is manifest from these propositions that numbers
which are not similar plane numbers, that is, those which
have not their sides proportional, have not to one another
the ratio which a square number has to a square number.

For, if they have, they will be similar plane numbers :
which is contrary to the hypothesis.

Therefore numbers which are not similar plane numbers
have not to one another the ratio which a square number has
to a square number.]
\end{proof}

\begin{notes}

A scholium to this proposition (Schol. x. No. 62) says categorically that
the theorem proved in it was the discovery of T/heaetetus.

If a, b be straight lines, and

a : b = fit : «,
where m, n are numbers,
then f-.PnP-.tt;

and conversely.

This inference, which looks so easy when thus symbolically expressed, was
by no means so easy for Euclid owing to the fact that a, b are straight lines,
and m, n numbers. He has to pass from a : b to a 1 : b* by means of vi. 20, Por.
through the duplicate ratio; the square on a is to the square on b in the
duplicate ratio of the corresponding sides a, b. On the other hand, mi, m
being numbers, it is vm. \ 1 which has to be used to show that «' : n   is the
ratio duplicate of »i : «.

Then, in order to establish his result, Euclid assumes that, if two ratios art
equal, the ratios which are their duplicates are also equal. This is nowhere
proved in Euclid, but it is an easy inference from v. 22, as shown in my note
on vi. 22.

The converse has to be established in the same careful way, and Euclid
assumes that ratios the duplicates of which are equal are themselves equal.
This is much more troublesome to prove than the converse; for proofs I refer
to the same note on vi. 22.

The second part of the theorem, deduced by \emph{reductio ad absurdum} from
the first, requires no remark.

In the Greek text there is an addition to the Porism which Heiberg
brackets as superfluous and not in Euclid's manner. It consists (1) of a sort
of proof, or rather explanation, of the Porism and (2) of a statement and
explanation to the effect that straight lines incommensurable in length are
not necessarily incommensurable in square also, and that straight lines
incommensurable in square are, on the other hand, always incommensurable
in length also.

The Lemma gives expressions for two numbers which have to one another
the ratio of a square number to a square number. Similar plane numbers
are of the form pm . pn and q m . qn, or mnp 1 and mnq 1 , the ratio of which is
of course the ratio of p* to a*.

The converse theorem that, if two numbers have to one another the ratio
of a square number to a square number, the numbers are similar plane
numbers is not, as a matter of fact, proved in the arithmetical Books. It is
the converse of vui. 26 and is used in ix. 10. Heron gave it (see note on
vim. 27 above).

Heiberg however gives strong reason for supposing the Lemma to be an
interpolation. It has reference to the next proposition, x. 10, and, as we shall
see, there are so many objections to x, ro that it can hardly be accepted as
genuine. Moreover there is no reason why, in the Lemma itself, numbers
which are not similar plane numbers should be brought in as they are.

\end{notes}

\end{proposition}

\begin{proposition}
\label{propX_10}

\begin{statement}
To find two straight lines incommensurable, the one in
length only, and the other in square also, with an assigned
straight line,
\end{statement}

\begin{proof}

Let A be the assigned straight line ;
thus it is required to find two straight lines incommensurable,
the one in length only, and the other in square also, with A.

Let two numbers B, C be set out which have not to one
another the ratio which a square number has to a square
number, that is, which are not similar plane

numbers ; A

and let it be contrived that, o

as B is to C, so is the square on A to * ``

the square on D

— for we have learnt how to do this —

[x. 6, Por.]

therefore the square on A is commensurable with the square
on D. [x. 6]

And, since B has not to C the ratio which a square number
has to a square number,

therefore neither has the square on A to the square on D the
ratio which a square number has to a square number ;
therefore A is incommensurable in length with D. [x. 9]

Let E be taken a mean proportional between A, D ;

therefore, as A is to D, so is the square on A to the square
on E. [v. Def. 9]

But A is incommensurable in length with D ;

therefore the square on A is also incommensurable with the
square on E ; [x. 11]

therefore A is incommensurable in square with E.

Therefore two straight lines D, E have been found in-
commensurable, D in length only, and E in square and of
course in length also, with the assigned straight line A,~
\end{proof}

\begin{notes}

It would appear as though this proposition was intended to supply a
justification for the statement in x. Def. 3 that it is proved that there are an
infinite number of straight lines (a) incommensurable in length only, or
commensurable in square only, and (A) incommensurable in square, with any
given straight line.

But in truth the proposition could well be dispensed with ; and the
positive objections to its genuineness are considerable.

In the first place, it depends on the following proposition,'' x. 1 1 ; for the
last step concludes that, since

a* ; y = a ; x,

and a, x are incommensurable in length, therefore a', y are incommensurable.
But Euclid never commits the irregularity of proving a theorem by means of
a later one. Gregory sought to get over the difficulty by putting x, 10 after
x. ti ; but of course, if the order were so inverted, the Lemma would still be
in the wrong place.

Further, the expression Ifinffopfv yap, ``for we have learnt (how to do this),''
is not in Euclid's manner and betrays the hand of a learner (though the same
expression is found in the Sectia Canenis of Euclid, where the reference is
to the Elements).

Lastly the manuscript P has the number io, in the first hand, at the top
of x. it, from which it may perhaps be concluded that x. io had at first no
number.

It seems best therefore to reject as spurious both the Lemma and x. io.

The argument of X. io is simple. If a be a given straight line and «, n
numbers which have not to one another the ratio of square to square, take x
such that

a' :x' = m : n, [x. 6, Por.]

whence a, x are incommensurable in length. [x, 9]

Then take y a mean proportional between a, x, whence

a* : y' = a ; x [v. Def. 9]

[- Jm : J>t],
and x is incommensurable in length only, white y is incommensurable in
square as well as in length, with a.

\end{notes}

\end{proposition}

\begin{proposition}
\label{propX_11}

\begin{statement}
If four magnitudes be proportional, and the first be com-
mensurable with the second, the third will also be commensurable
with the fourth ; and, if the first be incommensurable with the
second, the third will also be incommensurable with the fourth.
\end{statement}

\begin{proof}

Let A y B, C, D be four magnitudes in proportion, so
that, as A is to B, so is C

to D, A b

and let A be commensurable c D

with B ;

I say that C will also be commensurable with D.

For, since A is commensurable with B,
therefore A has to B the ratio which a number has to a
number. [x. 5]

And, as A is to B, so is C to D ;
therefore C also has to D the ratio which a number has to a
number ;
therefore C is commensurable with D, [x. 6]

Next, let A be incommensurable with B ;
I say that C will also be incommensurable with D.

For, since A is incommensurable with B,
therefore A has not to B the ratio which a number has to a
number. [x. 7]

And, as A is to B, so is C to D ;
therefore neither has C to D the ratio which a number has to
a number ;
therefore C is incommensurable with D. [x. 8]

Therefore etc.
\end{proof}

\begin{notes}

I shall henceforth, for the sake of brevity, use symbols for the terms
`` commensurable (with) `` and `` incommensurable (with) `` according to the
varieties described in x. Deff. i — 4. The symbols are taken from Lorenz
and seem convenient.

Commensurable and commensurable with, in relation to areas, and com-
mensurable in length and commensurable in length withy in relation to straight
lines, will be denoted by <->,

Commensurable in square only or commensurable in square only with (terms
applicable only to straight lines) will be denoted by ``-.

Incommensurable (with), of areas, and incommensurable (with), of straight
lines will be denoted by *j.

Incommensurable in square (with) (a term applicable to straight lines only)
will be denoted by <*-,

Suppose a, b, c, d to be four magnitudes such that
a : > m e i t,

Then (1), if a ~ i, a : 6 = m : n, where m, n are integers, [x. 5]

whence c \ d = m : tt,

and therefore c « d. [x. 6]

(j) If a u b, a :b*m : n, [x. 7]

so that c:d*m:n,

whence c u d, [x. 8]

\end{notes}

\end{proposition}

\begin{proposition}
\label{propX_12}

\begin{statement}
Magnitudes commensurable with the same magnitude are
commensurable with one another also.
\end{statement}

\begin{proof}

For let each of the magnitudes A, B be commensurable
with C ;
I say that A is also commensurable with B.

-D

— E H

-F K

-G — L

For, gince A is commensurable with C,
therefore A has to C the ratio which a number has to a
number. [x. 5]

Let it have the ratio which D has to E.

Again, since C is commensurable with B,

therefore C has to B the ratio which a number has to a
number. [x. 5]

Let it have the ratio which F has to G.
And, given any number of ratios we please, namely the
ratio which D has to E and that which F has to G,

let the numbers H, K, L be taken continuously in the given
ratios ; [cf. vin. 4]

so that, as D is to E, so is H to K,

and, as F is to G, so is K to L.

Since, then, as A is to C, so is D to E,
while, as D is to E, so is H to K,
therefore also, as A is to C, so is H to K. ( [v. n]

Again, since, as C is to 2?, so is 7*'' to G,
while, as /*'' is to (r, so is K to Z.,
therefore also, as C is to B, so is A' to L. [v. 11]

But also, as A is to C, so is If to K
therefore, ex aequali, as A is to /?, so is H to Z. [v. «]

Therefore -*4 has to B the ratio which a number has to a
number ;
therefore A is commensurable with B. [x. 6]

Therefore etc.
\end{proof}

\begin{notes}

We have merely to go through the process of compounding two ratios in
numbers.

Suppose
Therefore

0, 6 each *''> c.
a : c - m : 71, say,

[*-s]

Now

e:b = P-g, say.
m  h = tup : np,

and

Therefore

p : g = «/ : nq.
a ve=mp : «/,

whence, ex aeqvati*
so that

c \ b = np : nq,
a : b-mp : ruf,
a * A,

\end{notes}

\end{proposition}

\begin{proposition}
\label{propX_13}

\begin{statement}
If two magnitudes be commensurable, and the one of them
be incommensurable with any magnitude, the remaining one
will also be incommensurable with the same.
\end{statement}

\begin{proof}

Let A, B be two commensurable magnitudes, and let one
of them, A, be incommensurable with

any other magnitude C; A—

I say that the remaining one, B, will

also be incommensurable with C.

For, if B is commensurable with C,
while A is also commensurable with B,
A is also commensurable with C. [x. 1*]

But it is also incommensurable with it :
which is impossible.

Therefore B is not commensurable with C
therefore it is incommensurable with it.

Therefore etc.

L-j Erfrl MA.

Given two unequal straight lines, to find by what square the
square on the greater is greater than the square on the less.

Let AB, C be the given two unequal straight lines, and
let AB be the greater of them ;
thus it is required to find by what
square the square on AB is greater
than the square on C.

Let the semicircle ADB be de-
scribed on AB,

and let AD be fitted into it equal to C; [iv. 1]

let DB be joined.

It is then manifest that the angle ADB is right, [m. 31]

and that the square on AB is greater than the square on
AD, that is, C, by the square on DB. [i- 47]

Similarly also, if two straight lines be given, the straight
line the square on which is equal to the sum of the squares
on them is found in this manner.

Lemma, x. 14]

Let AD, DB be the given two straight lines, and let it be
required to find the straight line the square on which is equal
to the sum of the squares on them.

Let them be placed so as to contain a right angle, that
formed by AD, DB ;
and let AB be joined.

It is again manifest that the straight line the square on
which is equal to the sum of the squares on AD, DB is AB.
\end{proof}

\begin{notes}

The lemma gives an obvious method of finding a straight line (e) equal to
J<? t £*, where a, b are given straight lines of which a is the greater.

\end{notes}

\end{proposition}

\begin{proposition}
\label{propX_14}

\begin{statement}
If four straight lines be proportional, and the square on
the first be greater than the square on the second by the square
on a straight line commensurable with the first, the square on
the third will also be greater than the square on the fourth by
the square on a straight line commensurable with the third.

And, if the square on the first be greater than the square
on the second by the square on a straight line incommensurable
with the first, the square on the third will also be greater than
the square on the fourth by the square on a straight line in-
to commensurable with the third.
\end{statement}

\begin{proof}

Let A, B, C, D be four straight lines in proportion, so

that, as A is to B, so is C to D ;

and let the square on A be greater than

the square on B by the square on E, and
islet the square on C be greater than the

square on D by the square on F;

I say that, if A is commensurable with E,

C is also commensurable with F,

and, if A is incommensurable with E, C is
kj also incommensurable with F.

For since, as A is to B, so is C to D,

therefore also, as the square on A is to the square on B, so is

the square on C to the square on D. [vi. 22]

But the squares on E, B are equal to the square on A,
15 and the squares on D, F are equal to the square on C.

A B

C D

Therefore, as the squares on E, B are to the square on
B, so are the squares on D, F to the square on D ;

therefore, separando, as the square on E is to the square on
B, so is the square on F to the square on D ; [v. 17]

30 therefore also, as E is to B, so is F to D ; [vi. 22]

therefore, inversely, as B is to E, so is D to F.

But, as v4 is to B, so also is C to D

therefore, ex aequali, as A is to .£'', so is C to T 7 ! [v. ta]

Therefore, if 4 is commensurable with ZT, C is also com-
35 mensurable with F,

and, if A is incommensurable with E, C is also incommen-
surable with F. fx. iil

Therefore etc.
\end{proof}

\begin{annotations}

3, 5, 8, 1o< Euclid speaks of the square on the lint (third) being greater thin the square
on the second (fourth) by the square on a straight line commensurable (incommensurable)
``with if*,If' tiat'Tri),'* and similarly in all Like phrases throughout the Book, For clearness
sake I substitute ``the first,'' `` the third,'' or whatever it may be, for `` itself'' in these cases.

\end{annotations}

\begin{notes}

Suppose a, i, c, d to be straight lines such that

a : b = c ;d , /i).

ft follows [vt. u] that d> :P = c'':d 1 (2).

In order to prove that, eenvcrtendo,

a' : (a i -P) = £* : (S-d*)

Euclid has to use a somewhat roundabout method owing to the absence of a
coimertcnda proposition in his Book v. (which omission Simson supplied by
his Prop. E).

It follows from (2) that

((* ->) + *») :P(tf-<P) + d*) :d

whence, separando, (a 1 - **) ;  = (/' -d 1 ) ; d\ [v. 17]

and, inversely, P : (a* - P) = <T : (V - d*).

From this and (2), ex aeguali,

a*: (a?-)« : (c'-d*). [v, 22]

Hence a :  Ja*-P = e : -Jc' — d*. [vi. 22]

According therefore as a«or u Ja' — P,

eoTJf-d'. [x. 11]

If a « /(7 J - P. we may put J a- - P = Aa, where k is of the form mjn
and m, n are integers. And if 'a- - P = ka, it follows in this case that
 fF-d* = kc.

\end{notes}

\end{proposition}

\begin{proposition}
\label{propX_15}

\begin{statement}
If two commensurable magnitudes be added together, the
whole will also be commensurable with each of them / and, if
the whole be commensurable with one of them, the original
magnitudes will also be commensurable.
\end{statement}

\begin{proof}

For let the two commensurable magnitudes AB, BC be
added together ;

I say that the whole AC is also A   c

commensurable with each of the
magnitudes AB, BC.

For, since AB, BC are commensurable, some magnitude
will measure them.

Let it measure them, and let it be D.

Since then D measures AB, BC, it will also measure the
whole AC.

But it measures AB, BC also ;
therefore D measures AB, BC, AC ;

therefore AC is commensurable with each of the magnitudes
AB, BC. [x. Def. 1]

Next, let AC be commensurable with AB;
I say that AB, BC are also commensurable.

For, since AC, AB are commensurable, some magnitude
will measure them.

Let it measure them, and let it be D.

Since then D measures CA, AB, it will also measure the
remainder BC.

But it measures AB also ;

therefore D will measure AB, BC;

therefore AB, BC are commensurable, [x. Def. r]

Therefore etc.
\end{proof}

\begin{notes}

(1) If a, b be any two commensurable magnitudes, they are of the form
im, ne, where c is a common measure of a, b and m, n some integers.

It follows that a + b = (m + n) c ;

therefore (a + b), being measured by c, is commensurable with both a and /'.

(2) If a + b is commensurable with either a or b, say a, we may put
a + b = mc, a = nc, where c is a common measure of (a + b), a, and m, n are
integers.

Subtracting, we have b = («— ») t,

whence /' * a.

\end{notes}

\end{proposition}

\begin{proposition}
\label{propX_16}

\begin{statement}
If two incommensurable magnitudes be added together, the
whole will also be incommensurable with each of them ; and, if
the whole be incommensurable with one of them, the original
magnitudes will also be incommensurable.
\end{statement}

\begin{proof}

For let the two incommensurable magnitudes AB, BC be
added together ;

I say that the whole AC is also incommensurable
with each of the magnitudes AB, BC.

For, if CA, AB are not incommensurable, some
magnitude will measure them.

Let it measure them, if possible, and let it be D.

Since then D measures CA, AB,

therefore it will also measure the remainder BC.

But it measures AB also ;
therefore D measures AB, BC.

Therefore AB, BC are commensurable ;
but they were also, by hypothesis, incommensurable :
which is impossible.

Therefore no magnitude will measure CA, AB ;
therefore CA, AB are incommensurable. [x. Def. 1]

Similarly we can prove that AC, CB are also incom-
mensurable.

Therefore ACh incommensurable with each of the magni-
tudes AB, BC.

Next, let AC be incommensurable with one of the magni-
tudes AB, BC.

First, let it be incommensurable with AB ;
I say that AB, BC are also incommensurable.

For, if they are commensurable, some magnitude will
measure them.

Let it measure them, and let it be D.

Since then D measures AB, BC.
therefore it will also measure the whole AC.

But it measures AB also ;
therefore D measures CA, AB,

Therefore CA, AB are commensurable ;
but they were also, by hypothesis, incommensurable :
which is impossible.

Therefore no magnitude will measure AB, BC ;
therefore AB, BC are incommensurable. [x. Def. t]

Therefore etc.
\end{proof}

\begin{lemma*}

\begin{statement}
If to any straight line there be applied a parallelogram
deficient by a square figure, the applied parallelogram is equal
to the rectangle contained by the segments of the straight line
resulting from the application.
\end{statement}

For let there be applied to the straight line AB the
parallelogram AD deficient by the
square figure DB;

I say that AD is equal to the rectangle
contained by AC, CB.

This is indeed at once manifest ;

for, since DB is a square,

DC is equal to CB ;

and AD is the rectangle AC, CD, that is, the rectangle AC,
CB,

Therefore etc.

\end{lemma*}

\begin{notes}

If a be the given straight line, and x the side of the square by which the
applied rectangle is to be deficient, the rectangle is equal to ax- x*, which is
of course equal to x (a - x). The rectangle may be written xy, where
* + y = a. Given the area x(a-x), or xy (where x+y = a), two different
applications will give rectangles equal to this area, the sides of the defect
being x or a - x (x or y) respectively ; but the second mode of expression
shows that the rectangles do not differ in form but only in position.

\end{notes}

\end{proposition}

\begin{proposition}
\label{propX_17}

\begin{statement}
If there be two unequal straight lines, and to the greater
there be applied a parallelogram equal to the fourth part of
the square on the less and deficient by a square figure, and if
it divide it into parts which are commensurable in length, then
s the square on the greater will be greater than the square on
the less by the square on a straight line commensurable with
the greater.
\end{statement}

\begin{proof}

And, if the square on the greater be greater than the square
on the less by the square on a straight line commensurable with
the greater, and if there be applied to Ike greater a parallelogram
equal to the fourth part of the square on the less and deficient
by a square figure, it will divide it into parts which are com-
mensurable in length.

Let A, BC be two unequal straight lines, of which BC is

15 the greater,

and let there be applied to BCs. parallel-

ogram equal to the fourth part of the

square on the less, A, that is, equal to

the square on the half of A, and deficient
to by a square figure. Let this be the k f e o 6

rectangle BD, DC, [cf. Lemma]

and let BD be commensurable in length with DC;

I say that the square on BC is greater than the square on A

by the square on a straight line commensurable with BC.
*5 For let BC be bisected at the point E,

and let EF be made equal to DE.

Therefore the remainder DC is equal to BF.

And, since the straight line BC has been cut into equal

parts at E, and into unequal parts at D,

30 therefore the rectangle contained by BD, DC, together with
the square on ED, is equal to the square on EC ; [n. 5]

And the same is true of their quadruples ;
therefore four times the rectangle BD, DC, together with
four times the square on DE, is equal to four times the square
35 on EC.

But the square on A is equal to four times the rectangle
BD, DC;

and the square on DF is equal to four times the square on
DE, for DF is double of DE.
40 And the square on BC is equal to four times the square
on EC, for again BC is double of CE.

Therefore the squares on A, DE are equal to the square
on BC,

so that the square on BC is greater than the square on A by
45 the square on DF.

It is to be proved that BC is also commensurable with DF.
Since BD is commensurable in length with DC,
therefore BC is also commensurable in length with CD. [x. 15]

But CD is commensurable in length with CD, BF, for

so CD is equal to BF. [x. 6]

Therefore BC is also commensurable in length with BF,

CD, [x. 12]

so that BC is also commensurable in length with the remainder
FD; [x. .5]

55 therefore the square on BC is greater than the square on A
by the square on a straight line commensurable with BC.

Next, let the square on BC be greater than the square on
A by the square on a straight line commensurable with BC,
let a parallelogram be applied to BC equal to the fourth part
60 of the square on A and deficient by a square figure, and let
it be the rectangle BD, DC.

It is to be proved that BD is commensurable in length
with DC.

With the same construction, we can prove similarly that
65 the square on BC is greater than the square on A by the
square on FD.

But the square on BC is greater than the square on A
by the square on a straight line commensurable with BC.
Therefore BC is commensurable in length with FD,
70 so that BC is also commensurable in length with the remainder,
the sum of BF, DC. [x. 15]

Bat the sum of BF, DC is commensurable with DC, [x. 6]
so that BC is also commensurable in length with CD \ [x. 12]

and therefore, separando, BD is commensurable in length
n with DC. [x 15]

Therefore etc.
\end{proof}

\begin{annotations}

15- After raying literally that ``the square on BC is greater than the square on A by the
square on DF'' Euclid adds the equivalent expression with vhvf*l in its technical sense,
it UP apa Tiji A a<.t(w 6yarat t£ AZ. As this is untranslatable in English except by a
paraphrase in practically the< same words as have preceded, 1 have not attempted to
reproduce it.

\end{annotations}

\begin{notes}

This proposition gives the condition that the roots of the equation in x,
ax-x* = p(= -, sayV

are commensurable with a, or that x is expressible in terms of a and integral

numbers, i.e. is of the form — a. No better proof can be found for the fact

that Euclid and the Greeks used their solutions of quadratic equations for
ntoaerica! problems. On no other assumption could an elaborate discussion
of the conditions of incommensurability of the roots with given lengths o:
with a given number of units of length be explained. In a purely geometrical
solution the distinction between commensurable and incommensurable roots
has no point, because each can equally easily be represented by straight lines.
On the other hand, on the assumption that the numerical solution of quadratic
equations was an important part of the system of the Greek geometers,
the distinction between the cases where the roots are commensurable and
incommensurable respectively with a given length or unit becomes of great
importance. Since the Greeks had no means of expressing what we call an
irrational number, the case of an equation with incommensurable roots could
only be represented by them geometrically; and the geometrical representations
had to serve instead of what we can express by formulae involving surds,

Euclid proves in this proposition and the next that, x being determined
from the equation

x(a-x) = - (i

4

x, (a~x) are commensurable in length when Ja' — P, tf are so, and incom-
mensurable in length when Va* — P,a are incommensurable ; and conversely.
Observe the similarity of his proof to our algebraical method of solving
the equation, a being represented in the figure by BC, and x by CD,

EF=ED=°~x

2

— x\ - — , by Eucl.\ u. 5.

If we multiply throughout by 4,

whence, bv (1),  + (a — txf = a',

or J-P=(a-2xY,

and  Ja , -6* = a-ix.

We have to prove in this proposition

(1) that, if x, (a — x) are commensurable in length, so are a, J a' — P,
(a) that, if a, >/a* - *'' are commensurable in length, so are x, (a — x).

(1) To prove that a, a- 2X are commensurable in length Euclid employs
several successive steps, thus.

Since (a - x) « x, a « x.

But x « 2X.

Therefore a » »x

« (a - »x).
That is, a « J*? -P.

(2) Since a « */a* — ``, a `` a- ix,
whence `` `` '*

But 2X»x;

therefore * ** x,

and hence (*-*) `` *

[X.

5]

[X

.«]

[X.

n]

[X.

 5]

[X.

 5]

[*

.61

[X.

12]

[X.

'5]

It is often more convenient to use the symmetrica.) form of equation in

this and similar cases, viz.

x + y = a j

The result with this mode of expression is that

(1) if x `` y, then a « nJa'-P; and

(3) if a « iJj—, then x * y.

The truth of the proposition is even easier to see in this case, since
( X -yf = (<?-?).

\end{notes}

\end{proposition}

\begin{proposition}
\label{propX_18}

\begin{statement}
If there be two unequal straight lines, and to the greater
there be applied a parallelogram equal to the fourth part of
the square on the less and deficient by a square figure, and
if it divide it into parts which are incommensurable, the square
on the greater will be greater than the square on the less by
the square on a straight line incommensurable with the greater.

And, if the square on the greater be greater titan the square
on the less by the square on a straight line incommensurable
with the greater, and if there be applied to the greater a
parallelogram equal to the fourth part of the square on the
less and deficient by a square figure, it divides it into parts
which are incommensurable.
\end{statement}

\begin{proof}

Let A, BC be two unequal straight lines, of which BC is
the greater,

and to BC let there be applied a parallelogram equal
to the fourth part of the square on the less, A, and
deficient by a square figure. Let this be the rect-
angle BD, DC, [cf. Lemma before x. 17]
and let BD be incommensurable in length with DC;
I say that the square on BC is greater than the
square on A by the square on a straight line incom-
mensurable with BC.

For, with the same construction as before, we can prove
similarly that the square on BC is greater than the square on
A by the square on FD.

It is to be proved that BC is incommensurable in length
with DF.

Since BD is incommensurable in length with DC,

therefore BC is also incommensurable in length with CD.

[x. .6]

But DC is commensurable with the sum of BF, DC ; [x. 6]
therefore BC is also incommensurable with the sum of BF,
DC; [x. I3 )

so that.5C is also incommensurable in length with the remainder
FD. [x. 16]

And the square on BC is greater than the square on A
by the square on FD

therefore the square on BC is greater than the square on A
by the square on a straight line incommensurable with BC.

Again, let the square on BC be greater than the square on
A by the square on a straight line incommensurable with BC,
and let there be applied to BC a parallelogram equal to the
fourth part of the square on A and deficient by a square figure.
Let this be the rectangle BD, DC.

It is to be proved that BD is incommensurable in length
with DC.

For, with the same construction, we can prove similarly
that the square on BC is greater than the square on A by
the square on FD.

But the square on BC is greater than the square on A by
the square on a straight line incommensurable with BC;
therefore BC is incommensurable in length with FD,
so that BC is also incommensurable with the remainder, the
sum of BF, DC, [x. 16]

But the sum of BF, DC is commensurable in length with
DC; [x. 6]

therefore BC is also incommensurable in length with DC,

[x. n
so that, separando, BD is also incommensurable in length with
DC. [x. 16]

Therefore etc.

With the same notation as before, we have to prove in this proposition that
(1) if (a - x), x a re incommensurable in length, so are a, J a' - ?, and
(a) if a, V«'' — P are incommensurable in length, so are (a - x), x.

Or, with the equations

P
xy = -
4 >
x +y = a

(1) if x -jy, then a w Ja' — P, and

(2) if a 1/ Va* - ``, then .*  j>.

The steps are exactly the same as shown under (1) and (2) of the last
note, with u instead of n , except only in the lines ``x 'ft zx'' and ``m `` x''
which are unaltered, while, in the references, x. 13, 16 take the place of x.
12, 15 respectively.
\end{proof}

\begin{lemma*}

[Since it has been proved that straight lines commen-
surable in length are always commensurable in square also,
while those commensurable in square are not always com-
mensurable in length also, but can of course be either
commensurable or incommensurable in length, it is manifest
that, if any straight line be commensurable in length with a
given rational straight line, it is called rational and commen-
surable with the other not only in length but in square also,
since straight lines commensurable in length are always
commensurable in square also.

But, if any straight line be commensurable in square with
a given rational straight line, then, if it is also commensurable
in length with it, it is called in this case also rational and
commensurable with it both in length and in square ; but, if
again any straight line, being commensurable in square with a
given rational straight line, be incommensurable in length
with it, it is called in this case also rational but commensurable
in square only.]

\end{lemma*}

\end{proposition}

\begin{proposition}
\label{propX_19}

\begin{statement}
The rectangle contained by rational straight lines commen-
surable in length is rational.
\end{statement}

\begin{proof}

For let the rectangle AC be contained by the rational
straight lines AB, BC commensurable in
length ;
I say that AC is rational.

For on AB let the square AD be de-
scribed ;
therefore AD is rational. [x. Def. 4]

And, since AB is commensurable in
length with BC,
while AB is equal to BD,
therefore BD is commensurable in length with BC.

And, as BD is to BC, so is DA to AC. [vi. i]

Therefore ZM is commensurable with /4C. [x. iij

But DA is rational ;
therefore AC is also rational. [x. Def, 4]

Therefore etc.
\end{proof}

\begin{notes}

There is a difficulty in the text of the enunciation of this proposition.

The Greek runs to vwq pifnav fXTjKti vp,p£Tp*iiv Kara T*pa totv TrpouprfpAvrnv
Tpoirwv tv9tifj>v Tripitxpfifvov op6oywvior prjrav «rrcy, where the rectangle is
said to be contained by `` rational straight lines commensurable in length in
any of the aforesaid ways.'' Now straight lines can only be commensurable
in length in one way, the degrees of commensurability being commensurability
in length and commensurability in square only. But a straight line may be
rational in two ways in relation to a given rational straight line, since it may
be either commensurable in length, or commensurable in square only, with the
latter. Hence Billingsley takes Kara rtva Tajf irpofiprfitaitiiv jporuv with ptpw,
translating `` straight lines commensurable in length and rational in any of the
aforesaid ways,'' and this agrees with the expression in the next proposition
``a straight line once more rational in any of the aforesaid ways''; but the
order of words in the Greek seems to be fatal to this way of translating
the passage.

The best solution of the difficulty seems to be to reject the words ``in
any of the aforesaid ways `` altogether. They have reference to the I»emma
which immediately precedes and which is itself open to the gravest suspicion.
It is very prolix, and cannot be called necessary ; it appears moreover in
connexion with an addition clearly spurious and therefore relegated by
Heiberg to the Appendix. The addition does not even pretend to be Euclid's,
for it begins with the words ``for he calls rational straight lines those — ``
Hence we should no doubt relegate the Lemma itself to the Appendix.
August does so and leaves out the suspected words in the enunciation, as I
have done.

Exactly the same arguments apply to the Lemma added (without the
heading `` Lemma ``) to x. 33 and the same words `` in any of the aforesaid
ways `` used with `` medial straight lines commensurable in length `` in the
enunciation of x. 24. The said Lemma must stand or fall with that riow in
question, since it refers to it in terms: ``And in the same way as was explained
in the case of rational*.. ..''

Hence I have bracketed the Lemma added to x. 23 and left out the
objectionable words in the enunciation of x. 24.

If p be one of the given rational straight lines (rational of course in the
sense of x. Def. 3), the other can be denoted by kp, where * is, as usual, of
the form mjn (where jw, « are integers). Thus the rectangle is kp\ which is
obviously rational since it is commensurable with p'. [x. Def. 4.]

A rational rectangle may have any of the forms ab, ha'', kA or A, where
a, b are commensurable with the unit of length, and A with the unit of area.

Since Euclid is not able to use ip as a symbol for a straight line
commensurable in length with p, he has to put his proof in a form corre-
sponding to

p ] : V = P *P.
whence, p, kp being commensurable, p 5 , hp- are so also. [x. 11]

\end{notes}

\end{proposition}

\begin{proposition}
\label{propX_20}

\begin{statement}
If a rational area be applied to a rational straight line, it
produces as breadth a straight line rational and commensurable
in length with the straight line to which it is applied.
\end{statement}

\begin{proof}

For let the rational area AC he. applied to AB, a straight
line once more rational in any of the aforesaid
ways, producing BC as breadth ;
I say that BC is rational and commensurable in
length with BA.

For on AB let the square AD be described ;
therefore AD is rational. [x. Def. 4]

But AC is also rational ;
therefore DA is commensurable with AC.

And, as DA is to AC, so is DB to BC.

[v.. il

Therefore DB is also commensurable with BC ; [x. n]

and DB is equal to BA ;

therefore AB is also commensurable with BC.

But AB is rational ;
therefore BC is also rational and commensurable in length
with AB.

Therefore etc.
\end{proof}

\begin{notes}

The converse of the last. If p is a rational straight line, any rational area
is of the form kp\ If this be ``applied'' to p, the breadth is kp commensurable
in length with p and therefore rational. We should reach the same result if
we applied the area to another rational straight line <r. The breadth is then

\end{notes}

\end{proposition}

\begin{proposition}
\label{propX_21}

\begin{statement}
The rectangle contained by rational straight lines commen-
surable in square only is irrational and the side of the square
equal to it is irrational. Let the latter be called medial.
\end{statement}

\begin{proof}

For let the rectangle AC be contained by the rational
straight lines AB, BC commensurable in square only ;

I say that AC is irrational, and the side of the square equal
to it is irrational ;

and let the latter be called medial.

For on AB let the square AD be described ;
therefore AD is rational. [x. Def. 4]

And, since AB is incommensurable in length
with BC,

for by hypothesis they are commensurable in
square only,
while AB is equal to BD,
therefore DB is also incommensurable in length with BC.

And, as DB is to BC, so is AD to AC; [n, 1]

therefore DA is incommensurable with AC. [x. it]

But DA is rational ;
therefore AC is irrational,

so that the side of the square equal to AC is also irrational.

[. Def. 4]
Ana Jet the latter be called medial.
\end{proof}

\begin{notes}

A medial straight line, now defined for the first time, is so called because
it is a mean proportional between two rational straight lines commensurable
in square only. Such straight lines can lie denoted by p, p ,J. A medial

straight line is therefore of the form Jp r jk or £*p. Kuclid's proof that this is
irrational is equivalent to the following. Take p, p,JA commensurable in
square only, so that they are incommensurable in length.

Now p : p*]k - p' : p'k,

whence [x. 11] p' i is incommensurable with p 1 and therefore irrational
[x. Def. 4], so that JpTJk is also irrational [.].

A medial straight line may evidently take either of the forms JaJB or
XI AB, where of course B is not of the form k'A.

\end{notes}

\begin{lemma*}

If there be two straight lines, then, as the first is to the
second, so is the square on the first
to the rectangle contained by the
two straight lines.

Let FE, EG be two straight
lines.

I say that, as FE is to EG, so is the square on FE to
the rectangle FE, EG.

For on FE let the square DF be described,
and let GD be completed.

Since then, as FE is to EG, so is FD to DG, [vi. i]

and FD is the square on FE,

and DG the rectangle DE, EG, that is, the rectangle FE, EG,
therefore, as FE is to EG, so is the square on FE to the
rectangle FE, EG.

Similarly also, as the rectangle GE, EF is to the square
on EF, that is, as GD is to FD, so is GE to EF.

Q.E.D.

\end{lemma*}

\begin{notes}

If a, b be two straight lines,

a : h - a* : ab.

\end{notes}

\end{proposition}

\begin{proposition}
\label{propX_22}

\begin{statement}
The square on a medial straight line, if applied to a
rational straight line, produces as breadth a straight line
rational and incommensurable in length with that to which it
is applied.
\end{statement}

\begin{proof}

Let A be medial and CB rational,
and let a rectangular area BD equal to the square on A be
applied to BC, producing CD as
breadth ;

I say that CD is rational and incom-
mensurable in length with CB.

For, since A is medial, the square
on it is equal to a rectangular area
contained by rational straight lines
commensurable in square only.

[x 21]

Let the square on it be equal to GF.

But the square on it is also equal to BD ;
therefore BD is equal to GF.

But it is also equiangular with it ;
and in equal and equiangular parallelograms the sides about
the eual angles are reciprocally proportional ; [vi. 14)

therefore, proportionally, as BC is to EG, so is EF to CD.

Therefore also, as the square on BC is to the square on
EG, so is the square on EF to the square on CD. [vi. 22]

But the square on CB is commensurable with the square
on EG, for each of these straight lines is rational ;

therefore the square on EF is also commensurable with the
square on CD. [x. n]

But the square on EF is rational ;
therefore the square on CD is also rational j [x. Uef. 4J

therefore CD is rational.

And, since EF is incommensurable in length with EG,

for they are commensurable in square only,

and, as EF is to EG, so is the square on EF to the rectangle
FE, EG, [Lemma]

therefore the square on EF is incommensurable with the
rectangle FE, EG. [x. 11]

But the square on CD is commensurable with the square
on EF, for the straight lines are rational in square ;
and the rectangle DC, CB is commensurable with the rect-
angle FE, EG, for they are equal to the square on A j
therefore the square on CD is also incommensurable with the
rectangle DC, CB. [x. 13]

But, as the square on CD is to the rectangle DC, CB, so
is DC to CB ; [Umma]

therefore DC is incommensurable in length with CB. [x. 1 1]

Therefore CD is rational and incommensurable in length
with CB.
\end{proof}

\begin{notes}

Our algebraical notation makes the result of this proposition almost self-
evident. We have seen that the square of a medial straight line is of the form
J A . p 1 . If we ``apply'' this area to another rational straight line <r, tin-
breadth is .

This is equal to . o- = Jk . — <r, where m, n are integers. The latter
<r ft

straight line, which we may express, if we please, in the form Jk' . <r, is clearly
commensurable with 0- in square only, and therefore rational but incom-
mensurable in length with tr.

Euclid's proof, necessarily longer, is in two parts.

Suppose that the rectangle fo. p*~tr.x.

Then (r) <r ;p= Jk.p:x, [vi. 14]

whence o- 1 : p' = kf? ; Jr*. [vi. «]

But <r* `` p\ and therefore V « a*. [x. 1 1]

And kp* is rational ,
therefore x*, and therefore x, is rational.

(2) Since JJk . p < — p, ,Jk . p v p.

But [Lemma] Jk . p : p = kp* : Jt, p 1 ,

whence ip»  Jk . p'.

But Jk.p' = trx, and ip' « ;* * (from above) ;
therefore x**/<rx ;

and, since «'' : trx = x : o-,

X  «r.

53
[x. Def. 4)

[x. II]
[Lemma]

O F

\end{notes}

\end{proposition}

\begin{proposition}
\label{propX_23}

\begin{statement}
W straight line commensurable with a medial straight line
is medial.
\end{statement}

\begin{proof}

Let A be medial, and let B be commensurable with A ;
I say that B is also medial.

For let a rational straight line CD

be set out,

and to CD let the rectangular area CE
equal to the square on A be applied,
producing ED as breadth ;
therefore ED is rational and incommen-
surable in length with CD. [x. 22]

And let the rectangular area CF
equal to the square on B be applied to
CD, producing DF as breadth.

Since then A is commensurable with B,
the square on A is also commensurable with the square on B.

But EC is equal to the square on A,
and CF is equal to the square on B ;
therefore EC is commensurable with CF.

And, as EC is to CF, so is ED to DF; fvi. 1]

therefore .£''/? is commensurable in length with DF. [x. n]

But ED is rational and incommensurable in length with
DC;

therefore DF is also rational [x. Def. 3] and incommensurable
in length with DC. [x- 13]

Therefore CD, DF are rational and commensurable in
square only.

But the straight line the square on which is equal to the
rectangle contained by rational straight lines commensurable
in square only is medial ; [x. 21]

therefore the side of the square equal to the rectangle CD,
DB''is medial.

And B is the side of the square equal to the rectangle
CD, DF;
therefore B is medial.
\end{proof}

\begin{porism*}

From this it is manifest that an area commensurable with a medial area is medial.

[And in the same way as was explained in the case of
rationals [Lemma following x. 18] it follows, as regards medials,
that a straight line commensurable in length with a medial
straight line is called medial and commensurable with it not
only in length but in square also, since, in general, straight
lines commensurable in length are always commensurable in
square also.

But, if any straight line be commensurable in square with
a medial straight line, then, if it is also commensurable in
length with it, the straight lines are called, in this case too,
medial and commensurable in length and m square, but, if in
square only, they are called medial straight lines commen-
surable in square only.]

\end{porism*}

\begin{notes}

As explained in the bracketed passage following this proposition, a straight
line commensurable with a medial straight line in square only, as well as a
straight line commensurable with it in length, is medial.

Algebraical notation shows this easily.

If i*p be the given straight line, Xi'p is a straight line commensurable
in length with it and Jk . jb*p a straight line commensurable with it in square
only.

But if and Jk.p are both rational [x. Def. 3] and therefore can be

expressed by p', and we thus arrive at *p, which is clearly medial.

Euclid's proof amounts to the following.

Apply both the areas J-p' and k'Ji.p' (or k/Jk.p r ) to a rational
straight line a.

The breadths Jk . - and k'Ji . - f or kji . — I are in the ratio of the

areas *Jk.p* and k*Jk . p 1 (or kji.p 1 ) themselves and are therefore com-
mensurable.

Now [x. 22] y/i . — is rational but incommensurable with <r.
Therefore A»£ . `` for k/Jt . —I is so also;

whence the area X'/k , />* (or a/£ . p') is contained by two rational straight
lines com mensurable in square only, so that k*p (or /\ , k*p) is a medial
straight line.

It is in the Porism that we have the first mention of a medial area. It is
the area which is equal to the square on a medial straight line, an area, there-
fore, of the form Iflf?, which !s, as a matter of fact, arrived at, though not
named, before the medial straight line itself (x. 21).

The Porism states that Ai'p s is a medial area, which is indeed obvious.

\end{notes}

\end{proposition}

\begin{proposition}
\label{propX_24}

\begin{statement}
The rectangle contained by medial straight lines commen-
surable in length is medial.
\end{statement}

\begin{proof}

For let the rectangle AC be contained by the medial
straight lines AB, BC which are commensurable
in length ;
I say that AC is medial.

For on AB let the square AD be described ;
therefore AD is medial.

And, since AB is commensurable in length
with BC,

while AB is equal to BD,
therefore DB is also commensurable in length
with BC;
so that DA is also commensurable with AC, [vi. 1, x. 11]

But DA is medial ;
therefore AC is also medial. [x. 23, Por.]
\end{proof}

\begin{notes}

There is the same difficulty in the text of this enunciation as in that of
X. 19. The Greek says `` medial straight lines commensurable in length in
any of the aforesaid ways''; but straight lines can only be commensurable in
length in one way, though they can be medial in two ways, as explained in the
addition to the preceding proposition, i.e. they can be either commensurable
in length or commensurable in square only with a given medial straight line.
For the same reason as that explained in the note on x. 19 I have omitted
``in any of the aforesaid ways `` in the enunciation and bracketed the addition
to x. 33 to which it refers.

k*p and lrp are medial straight lines commensurable in length. The

rectangle contained by them is k*p\ which may be written *p''' and is there-
fore clearly medial.

Euclid's proof proceeds thus. Uti x, x be the two medial straight lines
commensurable in length.

Therefore «* : x . x = x : Xx.

[x. 24, 25
[x. ,«]

But x « Ajt, so that x 1 « x . Ajc,

Now x? is medial (x. a i ] ;
therefore x i Kx is also medial. [x. 23, For.)

We may of course write two medial straight lines commensurable in length
in the forms m/irp, ntrp; and these may either be mJaJB, nJaJB, or
mtlAB, ni/AB.

\end{notes}

\end{proposition}

\begin{proposition}
\label{propX_25}

\begin{statement}
rke rectangle contained by medial straight lines commen-
surable in square only is either rational or medial.
\end{statement}

\begin{proof}

For let the rectangle AC be contained by the medial
straight lines AB, BC which are
commensurable in square only ;
1 say that AC is either rational
or medial.

For on AB, BC let the
squares AD, BE be described ;
therefore each of the squares
AD, BE is medial.

Let a rational straight line
FG be set out,
to FG let there be applied the rectangular parallelogram GH
equal to AD, producing FH as breadth,
to HM let there be applied the rectangular parallelogram MK
equal to AC, producing HK as breadth,
and further to K N let there be similarly applied NL equal to
BE, producing KL as breadth ;
therefore FH, HK, KL are in a straight line.

Since then each of the squares AD, BE is medial,
and AD is equal to GH, and BE to NL,
therefore each of the rectangles GH, ML is also medial.

And they are applied to the rational straight lino FG ;
therefore each of the straight lines FH, KL is rational and
incommensurable in length with FG.

And, since AD is commensurable with BE,
therefore GH is also commensurable with NL.

And, as GH is to NL, so is FH to KL ;
therefore FH is commensurable in length with KL.

[x. 2 2]

[v,. ,]

[x 1 ,]

Therefore FH, KL are rational straight lines commen-
surable in length ;
therefore the rectangle FH, KL is rational. [x. 19]

And, since DB is equal to BA, and OB to BC,
therefore, as DB is to BC, so is AB to BO.

But, as DB is to BC, so is DA to AC, [vt. 1]

and, as AB is to BO, so is AC to CO ; [»rf.]

therefore, as DA is to A C, sots AC to CO.

But AD is equal to GH, AC to J/A'' and CO to Z ;
therefore, as GH is to A/A', so is MK to AX ;
therefore also, as FH is to HK, so is HK to A'Z. ; [vi. 1, v. n]
therefore the rectangle FH, KL is equal to the square on HK.

fvi. 17]

But the rectangle FH, KL is rational ;
therefore the square on HK is also rational.

Therefore HK is rational.

And, if it is commensurable in length with FG,

HN is rational ; [*. 19]

but, if it is incommensurable in length with FG,

KH, HM atz rational straight lines commensurable in square
only, and therefore HN is medial. [x. 21]

Therefore HN is either rational or medial.

But HN is equal to AC;

therefore AC is either rational or medial.

Therefore etc.
\end{proof}

\begin{notes}

Two medial straight lines commensurable in square only are of the form

The rectangle contained by them is ,/A   *V*. Now this is in general
medial; but, if N /X = k' Jk, the rectangle is kk'/?, which is rational.

Euclid's argument is as follows. Let us, for convenience, put x for frp, sc
that the medial straight lines are x, J\ . x.

Form the areas x*, x . J\ . x, Xx*,
and let these be respectively equal to <ru, <ru, mv, where a is a rational
straight line.

Since X s , Kx' are medial areas,
so are av, <rw,
whence v, w are respectively rational and <*- <r.

But a? n kx 1 ,
so that o-B `` oro,
or u * w

Therefore, u, w being both rational, «w is rational

Now * ; J.a* = J.3*;

Or <M:trv = tTV: trw,

so that u:v = v:w,

and arc = p''.

Hence, by (2), »>, and therefore v, is rational

Now (a) if »<''» <r, (tv or ,/A .  is rational;

(0) if I* v (r, so that v - <r, trv or J\ . x 1 is medial.

,(,).
.<i).

 (3).

\end{notes}

\end{proposition}

\begin{proposition}
\label{propX_26}
4 medial area does not exceed a medial area by a rational area.
\begin{statement}

\end{statement}

\begin{proof}

For, if possible, let the medial area AB exceed the medial
area AC by the rational area
DB,

and let a rational straight line
EF be set out ;

to EF let there be applied the
rectangular parallelogram FH
equal to AB, producing EH as
breadth,

and let the rectangle FG equal to AC be subtracted ;
therefore the remainder B> > is equal to the remainder KH.

But DB is rational ;
therefore KH is also rational.

Since, then, each of the rectangles AB, AC is medial,
and AB is equal to FH, and AC to FG,
therefore each of the rectangles FH, FG is also medial.

And they are applied to the rational straight line EF;
therefore each of the straight lines HE, EG is rational and
incommensurable in length with MF. [x. a 2]

And, since DB is rational and is equal to KH,
therefore] KH is [also] rational ;
and it is applied to the rational straight line EF;
therefore GH is rational and commensurable in length with
EF. [x. 10]

But EG is also rational, and is incommensurable in length
with EF;

therefore EG is incommensurable in length with GH. [x. 13]
And, as EG is to GH, so is the square on EG to the
rectangle EG, GH;

therefore the square on EG is incommensurable with the
rectangle EG, GH. [x. n]

But the squares on EG, GH are commensurable with the
square on EG, for both are rational ;

and twice the rectangle EG, GH is commensurable with the
rectangle EG, GH, for it is double of it ; [x. 6]

therefore the squares on EG, GH are incommensurable with
twice the rectangle EG, GH \ [x. 13]

therefore also the sum of the squares on EG, GH and twice
the rectangle EG, GH, that is, the square on EH [11. 4], is
incommensurable with the squares on EG, GH. [x. 16]

But the squares on EG, GH are rational ;
therefore the square on EH is irrational. [x. Def. 4]

Therefore EH is irrational.

But it is also rational :
which is impossible.

Therefore etc.
\end{proof}

\begin{notes}

``Apply'' the two given medial areas to one and the same rational straight
line p. They can then be written in the form p , k*p, p . X>.

The difference is then taji- Jtyp' ; and the proposition asserts that this
cannot be rational, i.e. (y/i- JA) cannot be equal to '. Cf. the proposition
corresponding to this in algebraical text-books.

To make Euclid's proof clear we will put x for A*p and y for X'p.

Suppose p (x -y) = pi,

and, if possible, let pz be rational, so that a must be rational and `` p ..(1).

Since px, py are medial,

x and y are respectively rational and v p (z ).

From (t) and (2), y  j s.

Now y =y* :yz,

so that y* tyz.

But

/t2 , ``/|

and

2JI2 1

-.jw

Therefore

y + v tyf,

whence

(y + zf »(?*),

or

X 1 «(/+**).

And (/ + **) is

rational ;

therefore x 1 , and consequently

x, is irrational.

But, by (2), x is

rational :

which is impossible.

Therefore ps is

not rational.

[x. j 6, a 7

\end{notes}

\end{proposition}

\begin{proposition}
\label{propX_27}

\begin{statement}
To find medial straight lines commensurable in square only
which contain a rational rectangle.
\end{statement}

\begin{proof}

Let two rational straight lines A, B commensurable in

square only be set out ;

let C be taken a mean proportional between
A, B, [v,. I3 ]

and let it he contrived that,

as A is to B, so is C to D. [vx 12'J
Then, since A, B are rational and com-
mensurable in square only,

the rectangle A, B, that is, the square on C
[vi 17], is medial. [x. 21]

Therefore C is medial. [x. 21]

And since, as A is to B, so is C to D,
and A t B are commensurable in square only,
therefore C, D are also commensurable in square only. [x. n]

And C is medial ;
therefore D is also medial. [x. 23, addition]

Therefore C, D are medial and commensurable in square
only,

I say that they also contain a rational rectangle.

For since, as A is to B, so is C to D,
therefore, alternately, as A is to C, so is B to D. [v. 16]

But, as A is to C, so is C to B ;
therefore also, as C is to B, so is B to Z? ;
therefore the rectangle C, Z? is equal to the square on B,

x. a?. 28] PROPOSITIONS 26—28 61

But the square on B is rational ;
therefore the rectangle C, D is also rational.

Therefore medial straight lines commensurable in square
only have been found which contain a rational rectangle.
\end{proof}

\begin{notes}

Euclid takes two rational straight lines commensurable in square only, say
p, k*p.

Find the mean proportional, i.e. i*p.

Take x such that p : p =  p . x (,j

This gives x - Irp,
and the lines required are krp, A*p.

For (a) *p is medial.

And (j8), by (1), since p «- £'p,

4*p ~- JP,
whence [addition to x. 23], since *p is media),

it*p is also medial.
The medial straight lines thus found may take either of the forms

< ) -iJs, J 23* or (2) Tab, J~ a -

\end{notes}

\end{proposition}

\begin{proposition}
\label{propX_28}

\begin{statement}
To find medial straight lines commensurable in square only
which contain a medial rectangle.
\end{statement}

\begin{proof}

Let the rational straight lines A, B, C commensurable in
square only be set out ;

let D be taken a mean proportional between A, B, [vi, 13]
and let it be contrived that,

as B is to C, so is D to E. [vi. 12]

Since A, B are rational straight lines commensurable in
square only,

therefore the rectangle A, B, that is, the square on D [vi. 17],
is medial. [x. at)

61 BOOK X [x. 18

Therefore D is medial. [x. *i]

And since B, C are commensurable in square only,

and, as B is to C, so is D to E,

therefore D, E are also commensurable in square only. fx. u]
But Z> is medial ;

therefore E is also medial. [x. 13, addition]

Therefore D, E are medial straight lines commensurable
in square only.

I say next that they also contain a medial rectangle.

For since, as B is to C, so is D to E,
therefore, alternately, as B is to D, so is C to E. [v. 16]

But, as B is to D, so is D to A ;

therefore also, as Z? is to A, so is C to E ;

therefore the rectangle A, C is equal to the rectangle D, E.

[v.. ,6]
But the rectangle A, C is medial ; [x. zr]

therefore the rectangle D, E is also medial.

Therefore medial straight lines commensurable in square
only have been found which contain a medial rectangle.
\end{proof}

\begin{notes}

Euclid takes three straight lines commensurable in square only, i.e. of the
form p, Vi A ft and proceeds as follows.

Take the mean proportional to p, A'p, i.e. £*p.

Then take a: such that

*V : *p = t*p : x (r),

so that x = A''p/A*.

trp, A*p/£* are the required medial straight lines.

For JPp is medial.

Now, by (1), since t?p «- A*p,

trp ``- x,

whence x is also medial [x. 33, addition], while «- £*p.

Next, by (1), *p : jc = **p : £*p

= <i*p:p,

whence x . A*p = A'p', which is medial.

The straight lines Irp, X*p/£* of course take different forms according as
the original straight lines are of the forms (1) a, ,JB, ,JC, (2) J A, JB, JC,
(3) -J A, b, JC, and (4) J A, JB, e.

E.g. in case (1) they are JaJB, <sj -jgi

in case (2) they are VAB, */-—

and so on.

\end{notes}

\begin{lemma*}

Lemma 1.

To find (wo square numbers such (hat their sum is also
square.

Let two numbers AB, BC be set out, and let them be
either both even or both odd.

Then since, whether an even a 6 S B

number is subtracted from an

even number, or an odd number from an odd number, the
remainder is even, [ix. 24, 26]

therefore the remainder AC is even.

Let AC be bisected at D.

Let AB, BC also be either similar plan„ numbers, or
square numbers, which are themselves also similar plane
numbers.

Now the product of AB, BC together with the square on
CD is equal to the square on BD. [11. 6]

And the product of AB, BC is square, inasmuch as it
was proved that, if two similar plane numbers by multiplying
one another make some number the product is square, [tx. 1]

Therefore two square numbers, the product of AB, BC,
and the square on CD, have been found which, when added
together, make the square on BD.

And it is manifest that two square numbers, the square
on BD and the square on CD, have again been found such
that their difference, the product of AB, BC, is a square,
whenever AB, BC are similar plane numbers.

But when they are not similar plane numbers, two square
numbers, the square on BD and the square on DC, have been
found such that their difference, the product of AB, BC, is
not square.

Q.E.D.

\end{lemma*}

\begin{notes}

Euclid's method of forming right-angled triangles in integral numbers.
already alluded to in the note on 1. 47, is as follows.

Take two similar plane numbers, e.g. mn/', mnq\ which are either both even
or both odd, so that their difference is divisible by 1.

Now the product of the two numbers, or wiW'f 3 , is square, [ix, i]

and, by n. 6,

»,nf . mtf + (*»?---*»?)' = (WJ,

so that the numbers mnpq, £ (mitj? - mnq*) satisfy the condition that the sum
of their squares is also a square number.

It is also clear that \ (mnp* -t mnij*), mnpq are numbers such that the
difftrcnce of their squares is also square.

\end{notes}

\begin{lemma*}

Lemma 2.

To find two square numbers such that their sum is not
square.

For let the product of AB, BC, as we said, be square,
and CA even,
and let CA be bisected by D.

E

a <£ A 6 p ~~c'' b

It is then manifest that the square product of AB, BC
together with the square on CD is equal to the square on BD.

[See I-enima 1]

Let the unit DE be subtracted ;
therefore the product of AB, BC together with the square on
CE is less than the square on BD.

I say then that the square product of AB, BC together
with the square on CE will not be square.

For, if it is square, it is either equal to the square on BE,
or less than the square on BE, but cannot any more be
greater, lest the unit be divided.

First, if possible, let the product of AB, BC together
with the square on CE be equal to the square on BE,
and let GA be double of the unit DE.

Since then the whole AC is double of the whole CD,
and in them AG is double of DE,

therefore the remainder GC is also double of the remainder EC;
therefore GC is bisected by E.

Therefore the product of GB, BC together with the square
on CE is equal to the square on BE. [it. 6]

But the product of AB, BC together with the square on
CE is also, by hypothesis, equal to the square on BE ;
therefore the product of GB, BC together with the square on
CE is equal to the product of AB, BC together with the
square on CE.

And, if the common square on CE be subtracted,
it follows that AB is equal to GB :
which is absurd.

Therefore the product of AB, BC together with the square
on CE is not equal to the square on BE.

I say next that neither is it less than the square on BE.

For, if possible, let it be equal to the square on BE,
and let HA be double of DF.

Now it will again follow that HC is double of CF;
so that CH has also been bisected at F,
and for this reason the product of HB, BC together with the
square on FC is equal to the square on BE. [11. 6]

But, by hypothesis, the product of AB, BC together with
the square on CE is also equal to the square on BE.

Thus the product of HB, BC together with the square
on CE will also be equal to the product of AB, BC together
with the square on CE :

which is absurd.

Therefore the product of AB, BC together with the square
on CE is not less than the square on BE.

And it was proved that neither is it equal to the square
on BE.

Therefore the product of AB, BC together with the square
on CE is not square.

Q.E.D.

\end{lemma*}

\begin{notes}

We can, of course, write the identity in the note on Lemma 1 above (p. 64)
in the simpler form

where, as before, mp* t mf are both odd or both even.
Now, says Euclid,

mp* . mq* + vS— — *- - 1 J is not a square number.

This is proved by \emph{reductio ad absurdum}.

The number is dearly less than to/ 1 , mtf + (-- — 1, i.e. less than

If then the number is square, its side must be greater than, equal to, or

fmt? + rap 1 V j. , ... , . w/ 1 + to/

less than I >* — —* - 1 L the number next less than — — .

But (1) the side can no l oe > I — — — 1) without being equal to

-- — —''- since they are consecutive numbers.
2

( 2 ) («/>- 2) TO?» + -~-— ~ l) = (-7- `` ')  I''- 6 1

If then rnf.tnf+ C ~ M *~ - 1 J is also equal to f ffi*'``*  , J j

we must have (»»/* - 2) «/ = to/ 1 . to? 1 ,

or to/* - 2 = m/* :

which is impossible.

( 3) ir . + (-«!z*t ,  I ) , <(at±itf I ) , j

suppose it equal to f - 1 - — r) .

But [,,. 61 (to/ - «,) to/ + (5tf  r )' = fW . r )*
Therefore

which is impossible.

Hence all three hypotheses are false, and the sum of the squares

j   j / m ?* ~ <*F *
«(/ . toj* and I -i— — J-  1 1 is not square.

\end{notes}

\end{proposition}

\begin{proposition}
\label{propX_29}

\begin{statement}
To find two rational straight lines commensurable in square
only and such that the square on the greater is greater than
the square on the less by the square on a straight line commen-
surable in length with the greater.
\end{statement}

\begin{proof}

For let there be set out any rational straight line AB,
and two square numbers CD, DE such that their difference
CE is not square ; [Lemma 1]

let there be described on AB the semicircle AFB,
and let it be contrived that,

as DC is to CE, so is the square on BA to the square
on AF. [x. 6, Pot.]

Let FB be joined.

Since, as the square on BA is to
the square on AF, so is DC to CE,
therefore the square on BA has to
the square on AF the ratio which the

number DC has to the number CE ; q g =

therefore the square on BA is com-
mensurable with the square on AF. [x. 6]

But the square on AB is rational ; [x, Def. 4]

therefore the square on AF is also rational ; [M.]

therefore AFis also rational.

And, since DC has not to CE the ratio which a square
number has to a square number,

neither has the square on BA to the square on AF the ratio
which a square number has to a square number ;
therefore AB is incommensurable in length with AF. [x. 9]

Therefore BA, AF are rational straight lines commen-
surable in square only.

And since, as DC is to CE, so is the square on BA to
the square on AF,

therefore, converlendo, as CD is to DE, so is the square on
AB to the square on BF. [v. 19, Por., m. 31, 1. 47]

But CD has to DE the ratio which a square number has
to a square number :

therefore also the square on AB has to the square on BF
the ratio which a square number has to a square number ;
therefore AB is commensurable in length with BF. [x. 9]

And the square on AB is equal to the squares on AF, FB ;
therefore the square on AB is greater than the square on AF
by the square on BF commensurable with AB.

Therefore there have been found two rational straight
lines BA, AF commensurable in square only and such that
the square on the greater AB is greater than the square on
the less AF by the square on BF commensurable in length
with AB.
\end{proof}

\begin{notes}

Take a rational straight line p and two numbers m\ «* such that (m* - n')
is not a square.

Take a straight line x such that

»* , :iW , ,-« , = p , :* , (1),

whence x? = r— p 1 ,

x=pifl¥ t where i -

Then p, p>Ji — are the straight lines required.

It follows from (1) that  `` p*,

and ;r is rational, but x <j p.

By (1), cmverUndo, m*  «* = p* : p' - xt,
so that tjp' -a? « p, and in fact = £p.

According as p is of the form a or , the straight Hnes are (1) a, J a 1 - P
or (a) JA, -J A - PA.

\end{notes}

\end{proposition}

\begin{proposition}
\label{propX_30}

\begin{statement}
To find two rational straight lines commensurable in square
only and such that the square on the greater is greater than
the square on the less by the square on a straight line incom-
mensurable in length with the greater.
\end{statement}

\begin{proof}

Let there be set out a rational straight line AB,
and two square numbers CE, ED
such that their sum CD is not
square ; [Lemma a]

let there be described on AB the
semicircle AFB,
let it be contrived that,
as DC is to CE, so is the square

on BA to the square on AF, <j e — — b

[x. 6, Pot.]
and let FB be joined.

Then, in a similar manner to the preceding, we can prove
that BA, AF are rational straight lines commensurable in
square only.

And since, as DC is to CE, so is the square on BA to
the square on AF,

therefore, convertendo, as CD is to DE, so is the square on
AB to the square on BF. [v. 19, Por., in. 31, 1. 47]

But CD has not to DE the ratio which a square number
has to a square number ;
therefore neither has the square on AB to the square on BF
the ratio which a square number has to a square number ;
therefore AB is incommensurable in length with BF. [x. 9]

And the square on AB is greater than the square on AF
by the square on FB incommensurable with A B,

Therefore AB, AF are rational straight lines commen-
surable in square only, and the square on AB is greater than
the square on AF by the square on FB incommensurable in
length with AB.
\end{proof}

\begin{notes}

In this case we take m\ r? such that «' + »* is not square.
Find x such that m* + «* : m 1 - p' : x 3 ,

—A

whence x* =

m'' + n> '

Then />,

x- p

where k — —
m

itisfy the condition.

JWp

The proof is after the manner of the proof of the preceding proposition
and need not be repeated.

According a s p is of the form a or J A, the straight lines take the

y''  > a i~

a a  , that is, a, vV — B, or (2) J A, 'J A - B and

JA, J~4¥.

\end{notes}

\end{proposition}

\begin{proposition}
\label{propX_31}

\begin{statement}
To find two medial straight lines commensurable in square
only, containing a rational rectangle, and such that the square
on the greater is greater than the square on the less by the
square on a straight line commensurable in length with the
greater.
\end{statement}

\begin{proof}

Let there be set out two rational straight lines A, B
commensurable in square only and such that the
square on A, being the greater, is greater than
the square on B the less by the square on a
straight line commensurable in length with A.

[x. 29]

And let the square on C be equal to the
rectangle A, B. D

Now the rectangle A, B is medial ; [x. 21]
therefore the square on C is also medial ;
therefore C is also medial. [x. ii]

Let the rectangle C, D be equal to the square on B.

Now the square on B is rational ;
therefore the rectangle C, D is also rational.

And since, as A is to B, so is the rectangle A, B to the
square on B,

while the square on C is equal to the rectangle A, B,
and the rectangle C, D is equal to the square on B,
therefore, as A is to B, so is the square on C to the rectangle
C,D.

But, as the square on C is to the rectangle C, D, so is C
toZ>;
therefore also, as A is to B, so is C to D.

But A is commensurable with B in square only ;
therefore C is also commensurable with D in square only. [x. 1 1]

And C is medial ;
therefore D is also medial. [x. 23, addition]

And since, as A is to B, so is C to D,
and the square on A is greater than the square on B by the
square on a straight line commensurable with A y
therefore also the square on C is greater than the square on
D by the square on a straight line commensurable with C.

[x. 14]

Therefore two medial straight lines C, D, commensurable
in square only and containing a rational rectangle, have been
found, and the square on C is greater than the square on D
by the square on a straight Line commensurable in length
with C.

Similarly also it can be proved that the square on C
exceeds the square on D by the square on a straight line
incommensurable with C, when the square on A is greater
than the square on B by the square on a straight line incom-
mensurable with A. [x. 30)
\end{proof}

\begin{notes}

I. Take the rational straight lines commensurable in square only found
in x. 29, i.e. p, p Vi - .

Take the mean proportional p(i - ¥)* and jc such that
P (1 - )* : p vTP = p vTP ! *

Then P ( 1 1 -  )*, x, or p (i - )*, p (i - 4?)* are straight lines satisfying the
given conditions.

For (a) p'v'i -<P is a medial area, and therefore p(i -.**)* is a media!

straight line (1);

and x. p(i - *)* = p*(i~) and is therefore a rational area.

(fi) p, p(i - J?)*, p V 1 - , x are straight lines in continued proportion, by
construction.

Therefore p : p Vi -£* = p(r -)* : x (a).

(This Euclid has to prove in a somewhat roundabout way by means of the
lemma after X. 21 to the effect that a : b = ab : P.)

From (2) it follows [x. n] that x . — p(i -£*)*; whence, since p(\ -.4*)* is

medial, x or p(i — £'')* is medial also.

( y) From (2), since p, pVt -i* satisfy the remaining condition of the

problem, p(i -£*)*, p(t-ff)° do so also [x. 14].

According as p is of the form a or J A, the straight lines take the forms

(a) UA (A - PA),

A-A

UA (A - PA) '

II. To find medial straight lines commens irable in square only contain-
ing a rational rectangle, and such that the square on one exceeds the square
on the other by the square on a straight line incommensurable with the former,
we simply begin with the rational straight lines having the corresponding

property [x. 30], viz. p,       , and we arrive at the straight lines

VI +*F

P P

According as p is of the form a or /A, these (if we use the same
transformation as at the end of the note on x. 30) may take any of the forms

JTjj, - a '~ jg  .

(1)

(a) KlAlAB),

or !A(A-P),

A- B

Va(A-b)'

A-P

Ma (A-p)

\end{notes}

\end{proposition}

\begin{proposition}
\label{propX_32}

\begin{statement}
To find two medial straight lines commensurable in square
only, containing a medial rectangle, and such that the square
on the greater is greater than the square on the less Sy the
square on a straight line commensurable with the greater.
\end{statement}

\begin{proof}

Let there be set out three rational straight lines A, B, C
commensurable in square only, and such that the square on A
is greater than the square on C by the square on a straight
line commensurable with A, [x. 29]

and let the square on D be equal to the rectangle A, B.

Therefore the square on D is medial ;
therefore D is also medial. [*  ``]

Let the rectangle D, E be equal to the rectangle B, C.

Then since, as the rectangle A, B is to the rectangle B, C,
so is A to O,

while the square on D is equal to the rectangle A, B,
and the rectangle D, E is equal to the rectangle B, C,
therefore, as A is to C, so is the square on D to the rectangle
D,E.

But, as the square on D is to the rectangle D, E, so is D
to E ;
therefore also, as A is to C, so is D to E.

But A is commensurable with C in square only ;
therefore D is also commensurable with E in square only. jx. 1 1]

But D is medial ;
therefore E is also medial. [x. 23, addition]

And, since, as A is to C, so is D to E,
while the square on A is greater than the square on C by
the square on a straight line commensurable with A,
therefore also the square on D will be greater than the square
on E by the square on a straight line commensurable with D.

[x. 14]

I say next that the rectangle D, E is also medial.

For, since the rectangle B, C is equal to the rectangle D, E,
while the rectangle B, C is medial, [x. 31]

therefore the rectangle D, E is also medial.

Therefore two medial straight lines D, E, commensurable
in square only, and containing a medial rectangle, have been
found such that the square on the greater is greater than the
square on the less by the square on a straight line commen-
surable with the greater.

Similarly again it can be proved that the square on D
is greater than the square on E by the square on a straight
line incommensurable with D, when the square on A is
greater than the square on C by the square on a straight line
incommensurable with A. [x. 30]
\end{proof}

\begin{notes}

I. Euclid takes three straight lines of the form p, p Jk, p vi-/*,

takes the mean proportional pk* between the first two ( i ),

and then finds -t such that

pX i :p = pi¥:x (2),

whence x = pk* Ji -M,

and the straight lines pk*, pk* Ji — 4? satisfy the given conditions.
Now (a) pk* is medial.
() We have, from (i) and (2),

p:p*Ji- = pk i :x (3),

whence x . — pk* ; and x is therefore medial and «- pk*.

(y) x.pk* = Pt Jk.pJi-£'.

But the latter is media! ; [x. 2 1 J

therefore x . pk*, or pk* . pk* Ji - JP, is medial.

Lastly (8) p, p Ji — ¥ have the remaining property in the enunciation ;
therefore pk*, pX <Ji —JP have it also. [x. 14]

(Euclid has not the assistance of symbols to prove the proportion (3) above.
He therefore uses the lemmas ab;bc=ac and d 1 de = de to deduce from
the relations

06 = 4*
and d : b = e : e J

that a:c = d:e.)

The straight lines pk*, p*Ji- may take any of the following forms
according as the straight lines first taken are

(1) «i JB, j±, f» JA, JB, JAVA , (3) JA, b, J A -PA.

, jB(a>-S)

(1) JaJB,

(2) AB,

(3) -Jb~[A,

JaJB '
>JB(A-A)
ifAB '

bJA-A

JbJA

74

II. If the other conditions are the same, but the square on the first
medial straight line is to exceed the square on the second by the square on a
straight line ituommensxrable with the first, we begin with the three straight

lines p, pj, -,—-- , and the medial straight lines are
vi +

->*

P ki,

The possible forms are even more various in this case owing to the more
various forms that the original lines may take, e.g.

(I) a, JB> JtfCi

(z) J A, b, -JA~-?;

(3) sM> 4 -JaC;

< 4 ) J A, JB, JA-?
() JA, JB, JJC;
the medial straight lines corresponding to these being

slB(a*-C).

(r) JJB,

(2) JFJA,

(3) VTO

( 4 ) Has,

(s) 2B,

 JTjB '

b Ja-c'
JbijA '

b-J A~C
Jb~J2 '

JB(A~)

Hab '
JbJaJC)
Sab '

\end{notes}

\begin{lemma*}

Let ABC be a right-angled triangle having the angle A
right, and let the perpendicular AD be
drawn ; A

I say that the rectangle CB, BD is
equal to the square on BA,
the rectangle BC, CD equal to the
square on CA,

the rectangle BD, DC equal to the square on AD,

and, further, the rectangle BC, AD equal to the rectangle

And first that the rectangle CB, BD is equal to the square
on BA.

For, since in a right-angled triangle AD has been drawn
from the right angle perpendicular to the base,
therefore the triangles ABD, ADC are similar both to the
whole ABC and to one another. [vi. 8]

And since the triangle ABC is similar to the triangle ABD,
therefore, as CB is to BA, so is BA to BD ; [vi. 4]

therefore the rectangle CB, BD is equal to the square on AB.

[vi. 17]

For the same reason the rectangle BC, CD is also equal
to the square on AC.

And since, if in a right-angled triangle a perpendicular
be drawn from the right angle to the base, the perpendicular
so drawn is a mean proportional between the segments of the
base, [vi. 8, Por.]

therefore, as BD is to DA, so is AD to DC;
therefore the rectangle BD, DC is equal to the square on AD.

[vi. 17]

I say that the rectangle BC, AD is also equal to the rect-
angle BA, AC.

For since, as we said, ABC is similar to ABD,
therefore, as BC is to CA, so is BA to AD. [vi. 4]

Therefore the rectangle BC, AD is equal to the rectangle
BA, AC. [vi. 16]

Q.E.D.

\end{lemma*}

\end{proposition}

\begin{proposition}
\label{propX_33}

\begin{statement}
To find two straight lines incommensurable in square which
make the sum of the squares on them rational hut the rectangle
contained by them medial.
\end{statement}

\begin{proof}

Let there be set out two rational straight lines AB, BC
commensurable in square only
and such that the square on the
greater A Bis greater than the
square on the less BC by the
square on a straight line in-
commensurable with AB,

let BC be bisected at D,

let there be applied to AB a parallelogram equal to the square

on either of the straight lines BD, DC and deficient by a

square figure, and let it be the rectangle AE, EB j [vi. 18]

let the semicircle AFB be described on AB,

let EF be drawn at right angles to AB,

*nd let AF, FB be joined.

Then, since AB, BC are unequal straight lines,

and the square on AB is greater than the square on BC by
the square on a straight line incommensurable with AB,

while there has been applied to AB a parallelogram equal to
the fourth part of the square on BC, that is, to the square on
half of it, and deficient by a square figure, making the rect-
angle AE, EB,
therefore AE is incommensurable with EB. [x. 18]

And, as AE is to EB, so is the rectangle BA, AE to the
rectangle AB, BE,

while the rectangle BA, AE is equal to the square on AF,
and the rectangle AB, BE to the square on BF;

therefore the square on AF is incommensurable with the
square on FB ;

therefore AF, FB are incommensurable in square.

And, since AB is rational,
therefore the square on AB is also rational ;
so that the sum of the squares on AF, FB is also rational.

[>  47]

And since, again, the rectangle AE, EB is equal to the
square on EF,

and, by hypothesis, the rectangle AE, EB is also equal to the
square on BD,

therefore FE is equal to BD :

therefore BC is double of FE,

so that the rectangle AB, BC is also commensurable with the
rectangle AB, EF.

But the rectangle AB, BC is medial ; [x. n]

therefore the rectangle AB, EF is also medial. [x. 23, Por.]

But the rectangle AB, EF is equal to the rectangle AF,

FB ', [Lemma]

therefore the rectangle AF, FB is also medial.

But it was also proved that the sum of the squares on these
straight lines is rational.

Therefore two straight lines AF, FB incommensurable
in square have been found which make the sum of the
squares on them rational, but the rectangle contained by them
medial.
\end{proof}

\begin{notes}

Euclid take the straight lines found in x. 30, viz. p,

*/! + '

He then solves geometrically the equations

x+y = p

xy = -

 ()

If x, y are the values found, he takes u, v such that

*: )    »

and u, v art; straight lines satisfying the conditions of the problem.
Solving algebraically, we get (if x >y)

whence

y ;

V* v s/TT* 1

``AV

(3)-

V 2 v Vi+,

Euclid's proof that these straight lines fulfil the requirements is as follows.

(a) The constants in the equations (1) satisfy the conditions of x. 18 ;
therefore x y.

But x :y = u* :tf.

Therefore «* « e*,

and a, v are thus incommensurable in square.

(0) ** + = p*, which is rational.
( r ) By (1), V= /

By (1), uv = p.Jxy

But — ?£=. is a medial area.

therefore uv is medial.

Since p,   .: - may have any of the three forms

(!) a, JfZTB, K ,) JA, JaB, (3) JA, -JAP,
u, v may have any of the forms

, , teF+ajM Is'-ajB

M V --5—   V-* '

/a + >Ja£ I a- 4ab

<«) V ~ > V - -   ;

, . fATFJl /a~VJa

w V — , V — - —

\end{notes}

\end{proposition}

\begin{proposition}
\label{propX_34}

\begin{statement}
To find two straight lines incommensurable in square which
make the sunt of the squares on them medial but the rectangle
contained by them rational.
\end{statement}

\begin{proof}

Let there be set out two medial straight lines AB, BC,
commensurable in square only, such that the rectangle which
they contain is rational, and the square on AB is greater than
the square on BC by the square on a straight line incom-
mensurable with AB ; [x. 31, adfin.

let the semicircle ADB be described on AB,

let BC be bisected at E,

let there be applied to AB a parallelogram equal to the square

on BE and deficient by a square figure, namely the rectangle

AF, FB; [vi. 28]

therefore AF is incommensurable in length with FB. [x. 18]

Let FD be drawn from F at right angles to AB,
and let AD, DB be joined.

Since AF is incommensurable in length with FB,

therefore the rectangle BA, AF is also incommensurable with
the rectangle AB, BF. [x. 1 1]

But the rectangle BA, AF s equal to the square on AD,
and the rectangle AB, BF to the square on DB ;
therefore the square on AD is also incommensurable with the
square on DB.

And, since the square on AB is medial,
therefore the sum of the squares on AD, DB is also medial.

[in. 31, 1. 47]

And, since BC is double of DF,

therefore the rectangle AB, BC is also double of the rectangle
AB, FD.

But the rectangle AB, BC is rational ;
therefore the rectangle AB, FD is also rational. [x. 6]

But the rectangle AB, FD is equal to the rectangle AD,
DB ; [Lemma]

so that the rectangle AD, DB is also rational.

Therefore two straight lines AD, DB incommensurable
in square have been found which make the sum of the squares
on them medial, but the rectangle contained by them rational,
\end{proof}

\begin{notes}

In this case we take [x. 31, and part] the medial straight lines
P P

Solve the equations

P

x+y = ,

(1 +)*

Take u, v such that, if x, y be the result of the solution,

.(!>.

and u, v are straight lines satisfying the given conditions.

Euclid's proof is similar to the preceding,
(a) From (() it follows [x. 18] that

x„y,
whence u* v »*,

and <r, v are thus incommensurable in square.

 (*)

So BOOK X [x. 34, 35

(Jl) a* + 1 — -£ — which is a medial area.

IP 1 .

= - . — - .. . which is a rational area.

Therefore uv is rational.

To find the actual form of u, v, we have, by solving the equations (i)
(if x>y),

2 — iiJT+P + i),

x =

and hence « = -  p - -   JjTW+k.

Ja(iTF)

Bearing in mind the forms which — £ — - , - — * may take (see note

ft + iPJ* (i+)*
on x, 31), we shall find that u, v may have any of the forms

, /( a + JB) «/~g J ( a -JB)>]7=li

, . / uA + JB)*lA-£ hjA-JB)JA-B .

\end{notes}

\end{proposition}

\begin{proposition}
\label{propX_35}

\begin{statement}
To find two straight lines incommensurable in square which
make the sum of the squares on them medial and the rectangle
contained by them medial and moreover incommensurable with
the sum of the squares on them.
\end{statement}

\begin{proof}

Let there be set out two medial straight lines AB, BC
commensurable in square only, containing a medial rectangle,
and such that the square on AB is greater than the square on
BC by the square on a straight line incommensurable with
AB; [x.t,a£Jkt.]

let the semicircle ADB be described on AB,
and let the rest of the construction be as above.

Then, since AF is incommensurable in length with FB,

[x. 18]
AD is also incommensurable in square with DB. [x. u]

And, since the square on AB is medial,
therefore the sum of the squares on AD, DB is also medial.

[in. 31, 1.47]
And, since the rectangle AF, FB is equal to the square
on each of the straight lines BE, DF,

therefore BE is equal to DF;

therefore BC is double of FD,

so that the rectangle AB, BC is also double of the rectangle
AB, FD.

But the rectangle AB, BC is medial ;

therefore the rectangle AB, FD is also medial. [x, 3*, Por.]

And it is equal to the rectangle AD, DB ;

[Lemma after x. 32]

therefore the rectangle AD, DB is also medial.

And, since AB is incommensurable in length with BC,
while CB is commensurable with BE,
therefore AB is also incommensurable in length with BE,

[x. 13]
so that the square on AB is also incommensurable with the
rectangle AB, BE. [x. n]

But the squares on AD, DB are equal to the square on

AB, [K47]

and the rectangle AB, FD, that is, the rectangle AD, DB, is
equal to the rectangle AB, BE;

therefore the sum of the squares on AD, DB is incommen-
surable with the rectangle AD, DB.

Therefore two straight lines AD, DB incommensurable
in square have been found which make the sum of the squares
on them medial and the rectangle contained by them medial
and moreover incommensurable with the sum of the squares
on them.
\end{proof}

\begin{notes}

Take the medial straight lines found in x. 3a (and part), viz.

Solve the equations

x+y- pX*

„ pA [ (I> '

and then put u*- r .

« , = pX i .jcl

ll> = pk* .y

where x, y are the ascertained values of x, y.

Then u, v. are straight lines satisfying the given conditions.
Euclid proves this as follows.

(«) From (1) it follows [x. 18] that x y.
Therefore k*  t,

arid u ~- v.

(p3) v* + »'' = p , ,/A, which is a medial area (3).

(y) uv = p* . <Jxy

I n* /X

= -   / -—   ) which is a medial area (4);

therefore uv is medial.

(«) t**~i-dL,

whence ?Jk~~ -4*

That is, by (3) and (4),

The actual values are found thus. Solving the equations (1), we have
pA*/ i

pX* / h

whence ,.!*/, 4 ,

pA* / A

According as p is of the form a or J A, we have a variety of forms for
u, v, arrived at by using the same transformations as in the notes on X. 30
and X. 3* (second part), e.g.

( 3 ) J SBH Ji, /QaEam.

and the expressions in (z), (3) with b in place of JB.

\end{notes}

\end{proposition}

\begin{proposition}
\label{propX_36}

\begin{statement}
If two rational straight lines commensurable in square only be added
together, the whole is irrational; and let it be called
\textbf{binomial}.
\end{statement}

\begin{proof}

For let two rational straight lines AB, BC commen-
5 surable in square only be added
together ;

I say that the whole AC is ir- A 8 c

rational.

For, since AB is incommensurable in length with BC —
10 for they are commensurable in square only —
and, as AB is to BC, so is the rectangle AB, BC to the
square on BC,

therefore the rectangle AB, BC is incommensurable with the

square on BC. [x. itj

15 But twice the rectangle AB, BC is commensurable with

the rectangle AB, BC [x. 6], and the squares on AB, BC are

co- n mensurable with the square on BC— for AB, BC are

rational straight lines commensurable in square only — [x. 15]

therefore twice the rectangle AB, BC is incommensurable

» with the squares on AB, BC. [x. 13]

And, componendo, twice the rectangle AB, BC together

with the squares on AB, BC, that is, the square on AC [n. 4],

is incommensurable with the sum of the squares on AB, BC.

[x .6]
But the sum of the squares on AB, BC is rational ;

25 therefore the square on A C is irrational,
so that AC is also irrational. [x. Def. 4]

And let it be called binomial.
\end{proof}

\begin{notes}

Here begins the first hexad of propositions relating to compound irrational
straight lines. The six compound irrational straight lines are formed by
adding two parts, as the corresponding six in Props. 73 — 78 are formed by
subtraction, 1'he relation between the six irrational straight lines in this and
the next five propositions with those described in Definitions 11. and the
Props. 48 — S3 following thereon (the first, second, third, fourth, fifth and
sixth binomials) will be seen when we come to Props. 54 — 59 ; but it may be
stated here that the six compound irrationals in Props. 36 — 41 can be found
by means of the equivalent of extracting the square root of the compound
irrationals in x. 48 — 53 (the process being, strictly speaking, the finding of the
sides of the squares equal to the rectangles contained by the latter irrationals
respectively and a rational straight line as the other side), and it is therefore
the further removed compound irrational, so to speak, which is treated first.

In reproducing the proofs of the propositions, I shall for the sake of
simplicity call the two parts of the compound irrational straight line x, y,
explaining at the outset the forms which x, y really have in each case 5 x will
always be supposed to be the greater segment.

In this proposition x, y are of the form p, Jk . p, and (x +y) is proved to
be irrational thus.

x ~-y, so that x uj.

Now x :y = x t : xy,

so that x* v xy.

But x 1 f (x* +y), and xy « ixy ;
therefore (.x* +> *) « 2xy,

and hence («'' +y + axy) « (a +y).

But (x? +y) is rational ;
therefore (x +yf, and therefore (* +y), is irrational.

This irrational straight line, p + Jk . p, is called a binomial straight line.

This and the corresponding apatomt (p — /Jb . p) found in x. 73 are the
positive roots of the equation

\end{notes}

\end{proposition}

\begin{proposition}
\label{propX_37}

\begin{statement}
ff two medial straight lines commensurable in square only
and containing a rational rectangle be added together, the
whole is irrational ; and let it be called a first bi medial
straight line.
\end{statement}

\begin{proof}

For let two medial straight lines AB, BC commensurable
in square only and containing

a rational rectangle be added  g — — — c

together ;

] say that the whole AC is irrational.

For, since AB is incommensurable in length with BC,
therefore the squares on AB, BC are also incommensurable
with twice the rectangle AB, BC; [cf. x. 36, 11. 9— *o]
and, componendo, the squares on AB, BC together with twice
the rectangle AB, BC, that is, the square on AC [11. 4], is
incommensurable with the rectangle AB, BC. [x. 16]

But the rectangle AB, BC is rational, for, by hypothesis,
AB, BC are straight lines containing a rational rectangle ;
therefore the square on AC is irrational ;
therefore AC is irrational. [x. Def. 4]

And let it be called a first bimedial straight line.
\end{proof}

\begin{notes}

Here x,y have the forms *p, p respectively, as found in x. 27.

Exactly as in the last case we prove that
x* +y* u zxy,
whence (* +y)* « txy.

But xy is rational ,
therefore (x +yf, and consequently (x +y), is irrational.

The irrational straight line trp + k*p is called a first bimedial straight line.

This and the corresponding first apotome of a medial (Jrp - A*p) found in
x. 74 are the positive roots of the equation

X* - 2 jk < I + k) p* . x' + k (1 - kfp',= o.

\end{notes}

\end{proposition}

\begin{proposition}
\label{propX_38}

\begin{statement}
If two medial straight lines commensurable in square only
and containing a medial rectangle be added together, the whole
is irrational; and let it be called a second bimedial straight
line.
\end{statement}

\begin{proof}

For let two medial straight lines AB, BC commensurable

in square only and containing

a medial rectangle be added a B o

together; p H Q

I say that AC is irrational.
10 For let a rational straight

line DE be set out, and let the

parallelogram DF equal to the

square on AC be applied to DE,

producing DG as breadth. [1. 44]

15 Then, since the square on A C is equal to the squares on

AB, BC and twice the rectangle AB, BC, [11. 4]

let EH, equal to the squares on AB, BC, be applied to DE;
therefore the remainder HF is equal to twice the rectangle
AB, BC.

jo And, since each of the straight lines AB, BC is medial,
therefore the squares on AB, BC are also medial.

But, by hypothesis, twice the rectangle AB, BC is also
medial.

And EH is equal to the squares on AB, BC,

»s while FH is equal to twice the rectangle AB, BC;

therefore each of the rectangles EH, HF is medial.

And they are applied to the rational straight line DE ;

therefore each of the straight lines DH, HG is rational and
incommensurable in length with DE. [x. n]

3f> Since then AB is incommensurable in length with BC,

and, as AB is to BC, so is the square on AB to the rectangle
AB, BC,

therefore the square on AB is incommensurable with the rect-
angle AB, BC. [x. 11]

35 But the sum of the squares on AB, BC is commensurable
with the square on AB, [x. 15]

and twice the rectangle AB, BC Is commensurable with the
rectangle AB, BC. [x. 6]

Therefore the sum of the squares on AB, BC is incom-
40 mensurable with twice the rectangle AB, BC. [x. 13]

But EH is equal to the squares on AB, BC,

and HF is equal to twice the rectangle AB, BC.

Therefore EH is incommensurable with HF,

so that DH is also incommensurable in length with HG.

[vi. t, x. 1 1]

45 Therefore DH, HG are rational straight lines commen-
surable in square only ;

so that DG is irrational. [x. 36]

But DE is rational ;

and the rectangle contained by an irrational and a rational
50 straight line is irrational ; [cf. x. 20]

therefore the area DF is irrational,
and the side of the square equal to it is irrational. [x. Def. 4]

But AC is the side of the square equal to DF
therefore AC is irrational.
s And let it be called a second bimedial straight Hue.
\end{proof}

\begin{notes}

After proving (1. 2 1 ) that each of the squares on AB, BC is medial, Euclid
states (II. 24, 26) that EH, which is equal to the sum of the squares, is a
medial area, but does not explain why. It is because, by hypothesis, the
squares on AB, BC are commensurable, so that the sum of the squares is
commensurable with either [x. 15] and is therefore a medial area [x. 23, Por.J.

In this case [x. 28, note] x, y are of the forms Irp, X'p/ respectively.
Apply each of the areas (x* +y) and zxy to a rational straight line <r, i.e.
suppose

axy = av.

Now it follows from the hypothesis, x. 15 and x. 13, Por. that (x t +y t ) is
a medial area ; and so is 2 xy, by hypothesis ;
therefore vu, av are medial areas.

Therefore each of the straight lines u, v is rational and w a (r).

Again x u y ;

therefore x* u xy.

But x* `` x* +y* and xy « txy ;
therefore x'+y i v3xy,
or <ru  av,
whence «  v (*)

Therefore, by (t), (2), u, ware rational and «-.

It follows, by x. 36, that (« + v) is irrational.

Therefore (u + v) a is an irrational area [this can be deduced from x, ao
by \emph{reductio ad absurdum}

whence (x +y) 1 , and consequently (x+y), is irrational.

The irrational straight line Jp + — j is called a second bimedial straight
line.

This and the corresponding second apotome of a media! (irp— r p)
found in x. 75 are the positive roots of the equation

\end{notes}

\end{proposition}

\begin{proposition}
\label{propX_39}

\begin{statement}
If two straight lines incommensurable in square which
make the sum of the squares on them rational, but the rectangle
contained by them medial, be added together, the whole straight
line is irrational : and let it be called major.
\end{statement}

\begin{proof}

For let two straight lines AB, BC incommensurable in
square, and fulfilling the given con-
ditions [x. 33], be added together; j g 6
I say that AC is irrational.

For, since the rectangle AB, BC is medial,
twice the rectangle AB, BC is also medial. [x. 6 and 23, Por.]

But the sum of the squares on AB, BC is rational ;
therefore twice the rectangle AB, BC is incommensurable
with the sum of the squares on AB, BC,
so that the squares on AB, BC together with twice the rect-
angle AB, BC, that is, the square on AC, is also incommen-
surable with the sum of the squares on AB, BC ; [x. 16]

therefore the square on A C is irrational,
so that AC is also irrational, [x Def. 4]

And let it be called major.
\end{proof}

\begin{notes}

Here x, y are of the form found in x. 33, vnz.

j*v 1 + jt+p' V2V 1 vnr

By hypothesis, the rectangle xy is media] ,
therefore txy is medial.

Also [x* +y T ) is a rational area.

Therefore x* -i-y* « ixy,

whence (x +y) s « (x* +y

so that (x +y)\ and therefore (x +y), is irrational.

The irrational straight line -?- */ 1

called a major (irrational) straight line.

This and the corresponding minor irrational found in x. 76 are the

positive roots of the equation

W

x*—2p'.x' + iia

r 1 + P r

\end{notes}

\end{proposition}

\begin{proposition}
\label{propX_40}

\begin{statement}
If two straight lines incommensurable in square which
make the sum of the squares on them medial, but the rectangle
contained by them rational, be added together, the whole straight
line is irrational ; and let it be called the side of a rational
plus a medial area.
\end{statement}

\begin{proof}

For let two straight lines AB, BC incommensurable in
square, and fulfilling the given con-
ditions [x. 34], be added together ; a b o

I say that AC is irrational.

For, since the sum of the squares on AB, BC is medial,
while twice the rectangle AB, BC is rational,
therefore the sum of the squares on AB, BC is incommen-
surable with twice the rectangle AB, BC;
so that the square on AC is also incommensurable with twice
the rectangle AB, BC, [x. 16]

But twice the rectangle AB, BC is rational ;
therefore the square on A C is irrational.

Therefore AC is irrational. [x. Def. 4]

And let it be called the side of a rational plus a
medial area.
\end{proof}

\begin{notes}

Here x,y have [x. 34] the forms

In this case (x* +y) is a medial, and ixy a rational, area ; thus

** +j? v ixy.
Therefore (x +yf  axy,

whence, since sxy is rational,

(x + y) t , and consequently (x +) t is irrational.
The irrational straight line

is called (for an obvious reason) the `` side `` of a rational plus a medial (area).
This and the corresponding irrational with a minus sign found in x. 77
are the positive roots of the equation

x*- . . -- 2 p*.x*+ . ,. K . p l = o.

\end{notes}

\end{proposition}

\begin{proposition}
\label{propX_41}

\begin{statement}
If two straight lines incommensurable in square which
make the sum 0/ the squares on them medial, and the rectangle
contained by them medial and also incommensurable with the
sum of the squares on them, be added together, Ike whole straight
line is irrational; and let it be called the side of the sum
of two medial areas.
\end{statement}

\begin{proof}

For let two straight lines AB, BC incommensurable in
square and satisfying the given conditions
[x. 35] be added together ;
I say that AC is irrational.

Let a rational straight line DE be set out,
and let there be applied to DE the rectangle
DF equal to the squares on A B, BC, and
the rectangle GH equal to twice the rectangle
AB,BC;

therefore the whole DH is equal to the square
on AC. [h- 4]

Now, since the sum of the squares on
AB, BC is medial, /''

and is equal to DF, A S ~~°

therefore DF is also medial.

And it is applied to the rational straight line DE ;
therefore DG is rational and incommensurable in length with
DE. [x. «]

For the same reason GK is also rational and incommen-
surable in length with GF, that is, DE.

And, since the squares on AB, BC are incommensurable
with twice the rectangle AB, BC,
DF is incommensurable with GH;
so that DG is also incommensurable with GK. [vi. t, x. u]

And they are rational ;
therefore DG, GK are rational straight lines commensurable
in square only ;
therefore DK is irrational and what is called binomial, [x. 36]

But DE is rational ;
therefore DM is irrational, and the side of the square which
is equal to it is irrational. [x. Def. 4]

But AC is the side of the square equal to HD ;
therefore AC is irrational.

And let it be called the side of the sum of two medial
areas.

In this case x, y are of the form
\end{proof}

\begin{notes}

l=<M n (.).

By hypothesis, (x 1 + ,r) and   xy are medial areas, and

x* +y* v 2xy (1).

' Apply these areas respectively to a rational straight line <r, and suppose
x* +y — <ru 1
axy

Since then au and av are both medial areas, u, v are rational and both
are  a (3).

Now, by (1) and (i),

so that guv.

By this and (3), «, w are rational and ``-.
Therefore [x. 36] (» + v) is irrational.
Hence <r(u +v) is irrational [deduction from x. 20].
Thus (jc +J>Y, and therefore (x + y), is irrational.
The irrational straight line

?iV V.+-4*  /'' V

is called (again for an obvious reason) the ``side'' of the sum 0/ two mtttials
(medial areas).

This and the corresponding irrational with a minus sign found in X. 78
are the positive roots of the equation

X* - 2 Jk . X'p' + k r; p* = 0.

I + K

Lemma.

And that the aforesaid irrational straight lines are divided
only in one way into the straight lines of which they are the
sum and which produce the types in question, we will now
prove after premising the following lemma.

Let the straight fine AB be set out, let the whole be cut
into unequal parts at each of

the points C, D, .

and letfTbe supposed greater A dec

than DB ;

I say that the squares on A C, CB are greater than the squares
on AD, DB.

For let AB be bisected at E.

Then, since AC is greater than DB,
let DC be subtracted from each ;
therefore the remainder AD is greater than the remainder CB,

But AE is equal to EB ;
therefore DE is less than EC ;

therefore the points C, D are not equidistant from the point
of bisection.

And, since the rectangle AC, CB together with the square
on EC is equal to the square on EB, [»  5]

and, further, the rectangle AD, DB together with the square
on DE is equal to the square on EB, [id.]

therefore the rectangle AC, CB together with the square on
EC is equal to the rectangle AD, DB together with the
square on DE.

And of these the square on DE is less than the square
on EC;

therefore the remainder, the rectangle AC, CB, is also less
than the rectangle AD, DB,

so that twice the rectangle AC, CB is also less than twice
the rectangle AD, DB.

Therefore also the remainder, the sum of the squares on
AC, CB, is greater than the sum of the squares on AD, DB.

Q.E.D.

3. and which produce the types in question. The Greek is rnnnmar t vpttflpcvti
rfip and I have taken eKij to mean ``types (of irrational straight lines),'' though the expression
might perhaps mean `` satisfying the conditions in question,''

This proves that, if x +y = u + v, and if u, ti are more nearly equal than
x, y (Le. if the straight line is divided in the second case nearer to the point
of bisection), then

(*+/)>(«* + i/«).
It is first proved by means of 11. 5 that

zxy < tuv,
whence, since (x +yy = (u + v)\ the required result follows.

\end{notes}

\end{proposition}

\begin{proposition}
\label{propX_42}

\begin{statement}
A binomial straight line is divided into its terms at one
point only.
\end{statement}

\begin{proof}

Let AB be a binomial straight line divided into its terms
at C;

therefore AC, CB are rational  -5 g b

straight lines commensurable in

square only. [*  3t>]

I say that AB is not divided at another point into two
rational straight lines commensurable in square only.

For, if possible, let it be divided at D also, so that AD,
DB are also rational straight lines commensurable in square
only.

It is then manifest that AC is not the same with DB.

For, if possible, let it be so.

Then AD will also be the same as CB,

and, as AC is to CB, so will BD be to DA;

thus AB will be divided at D also in the same way as by the
division at C:

which is contrary to the hypothesis.

Therefore AC is not the same with DB.

For this reason also the points C, D are not equidistant
from the point of bisection.

Therefore that by which the squares on AC, CB differ
from the squares on AD, DB is also that by which twice
the rectangle AD, DB differs from twice the rectangle

AC, CB,

because both the squares on AC, CB together with twice the
rectangle AC, CB, and the squares on AD, DB together
with twice the rectangle AD, DB, are equal to the square
on AB. [a. 4]

But the squares on AC, CB differ from the squares on

AD, DB by a rational area,

for both are ration, 1! ;

therefore twice the rectangle AD, DB also differs from twiGe
the rectangle AC, CB by a rational area, though they are
medial [x. n] :

which is absurd, for a medial area does not exceed a medial
by a rational area. [x. 26]

Therefore a binomial straight line is not divided at different
points ;
therefore it is divided at one point only.
\end{proof}

\begin{notes}

This proposition proves the equivalent of the well-known theorem in
surds that,

if a + Ji - x + Jy,

then a = x, b =y,

and if Ja + Ji = Jx+- Jy,

then a = x, b=y (or a=y, h = x).

The proposition states that a binomial straight line cannot be split up into
terms (ovo/wto.) in two ways. For, if possible, let

x+y-x' +y,
where x, y, and also x', y\ are the terms of a binomial straight line, x', y'
being different from x, y (or y, x).

One pair is necessarily more nearly equal than the other. Let x',y be
more nearly equal than x, y,

Then (a* +y*) - (*' a +/*) = tx'y' - 2xy.

Now by hypothesis (x? +y), (xP +y ) are rational areas, being of the form

p'+V;

but 2x'y, zxy are medial areas, being of the form jk.ff;
therefore the difference of two medial areas is rational i
which is impossible. [x. 26]

Therefore x'', y cannot be different from x, y (oi'y, x r

\end{notes}

\end{proposition}

\begin{proposition}
\label{propX_43}

\begin{statement}
A first bimedial straight line is divided at one point only.
\end{statement}

\begin{proof}

Let AB be a first bimedial straight line divided at C, so
that AC, CB are medial straight

lines commensurable in square , —   .

only and containing a rational A ° c B

rectangle ; [x. 37]

I say that AB is not so divided at another point.

For, if possible, let it be divided at D also, so that AD,
DB are also medial straight lines commensurable in square
only and containing a rational rectangle.

Since, then, that by which twice the rectangle AD, DB
differs from twice the rectangle AC, CB is that by which the
squares on AC, CB differ from the squares on AD, DB,
while twice the rectangle AD, DB differs from twice the
rectangle AC, CB by a rational area — for both are rational —

therefore the squares on A C, CB also differ from the squares
on AD, DB by a rational area, though they are medial :

which is absurd, [x. 16]

Therefore a first bimedial straight line is not divided into
its terms at different points ;

therefore it is so divided at one point only.
\end{proof}

\begin{notes}

In this case, with the same hypothesis, viz. that

x +v = *' +y,

and *', / are more nearly equal than x, y,

we have as before (x* +y'') - (x'*+y u ) = zx'y - 2xy.

But, from the given properties of x, y, and at, y', it follows that txy, sx'y
are rational, and (x* +>*), (x'' + / s ) medial, areas.

Therefore the difference between two medial areas is rational :
which is impossible. [x. 26]

\end{notes}

\end{proposition}

\begin{proposition}
\label{propX_44}

\begin{statement}
A second bimedial straight line is divided at one point only.
\end{statement}

\begin{proof}

Let AB be a second bimedial straight line divided at C,
so that AC, CB are medial straight lines commensurable in
square only and containing a medial rectangle ; [x. 38]

it is then manifest that C is not at the point of bisection,
because the segments are not commensurable in length.

I say that AB is not so divided at another point.

A O C B

C

— 1—

M

F L Q K

For, if possible, let it be divided at D also, so that AC is
not the same with DB, but AC is supposed greater ;
it is then clear that the squares on AD, DB are also, as we
proved above [Lemma], less than the squares on AC, CB ;
and suppose that AD, DB are medial straight lines commen-
surable in square only and containing a medial rectangle.

Now let a rational straight line EF be set out,
let there be applied to EF the rectangular parallelogram EK
equal to the square on AB,

and let EG equal to the squares on AC, CB be subtracted ;
therefore the remainder HK is equal to twice the rectangle
AC, CB, [it. 4]

Again, let there be subtracted EL, equal to the squares
on AD, DB, which were proved less than the squares on
AC, CB [Lemma] ;

therefore the remainder MK is also equal to twice the rect-
angle AD, DB.

Now, since the squares on AC, CB are medial,
therefore EG is medial.

And it is applied to the rational straight line EF;

therefore EH is rational and incommensurable in length with
EF, [x, 22]

For the same reason

HN is also rational and incommensurable in length with EF.

And, since AC, CB are medial straight lines commen-
surable in square only,
therefore AC vs. incommensurable in length with CB,

But, as A C is to CB, so is the square on AC to the rect-
angle AC, CB;

therefore the square on AC is incommensurable with the rect-
angle AC, CB, [x. 11]

But the squares on AC, CB are commensurable with the
square on AC ; for AC, CB are commensurable in square.

[X 15]

And twice the rectangle AC, CB is commensurable with
the rectangle AC, CB, [x. 6]

Therefore the squares on AC, CB are also incommen-
surable with twice the rectangle AC, CB. [x. 13]

But EG is equal to the squares on A C, CB,
and HK is equal to twice the rectangle AC, CB ;
therefore EG is incommensurable with HK,
so that EH is also incommensurable in length with HN.

[vl. I, X. Il]

And they are rational ;
therefore EH, HN are rational straight lines commensurable

in square only.

But, if two rational straight lines commensurable in square
only be added together, the whole is the irrational which is
called binomial. [x. 3 6 ]

Therefore EN is a binomial straight line divided at H.

In the same way EM, MN will also be proved to be
rational straight lines commensurable in square only ;
and EN will be a binomial straight line divided at different
points, H and M.

And EH is not the same with MN.

For the squares on AC, CB are greater than the squares
on AD, DB.

But the squares on AD, DB are greater than twice the
rectangle AD, DB ;

therefore also the squares on AC, CB, that is, EG, are much
greater than twice the rectangle AD, DB, that is, MK,
so that EH is also greater than MN.

Therefore EH is not the same with MN.
\end{proof}

\begin{notes}

As the irrationality of the second bimedial straight line [x. 38] is proved by
means of the irrationality of the binomial straight line [x. 36], so the present
theorem is reduced to that of x. 43.

Suppose, if possible, that the second bimedial straight line can be divided
into its terms as such in two ways, i.e. that

x +y = li +y,
where *', / are nearer equality than x, y.

Apply x 1 +v'', 2xy to a rational straight line a-, i.e. let
x* +v* = <r»,
xxy — <rv.

Then, as in x. 38, the areas x?+y, ixy are medial, so that au, ov are
medial;
therefore u, V are both rational and u a (t).

Again, by hypothesis, x, y are medial straight lines commensurable in
square only ;
thetefore xy.

Hence x*vxy.

And x 1 `` (** +v*), while xytxy;
therefore («* -+y*) ~ txy,
or <r» v <tv,
and hence u*> v (1).

Therefore, by (1) and (1), u, v are rational straight lines commensurable
in square only;

therefore u + v is a binomial straight line-
Similarly, if x'' ty* = w' and ix'y = <rv',
u + v' will be proved to be a binomial straight line.

And, since (x +y)* = (*' +>')', and therefore (« + ») = («' + v'), it follows that
a binomial straight line is divided as such in two ways :
which is impossible. [x. 42]

Therefore x + y, the given second bimedial straight line, can only be so
divided in one way.

In order to prove that u + v, u + v' represent a different division of the
same straight line, Euclid assumes that x'+y t > ixy. This is of course an
easy inference from 11. 7; but the assumption of it here renders it probable
that the Lemma after x. 59 is interpolated.

\end{notes}

\end{proposition}

\begin{proposition}
\label{propX_45}

\begin{statement}
A major straight line is divided at one and the same point
only.
\end{statement}

\begin{proof}

Let AB be a major straight line divided at C, so that
AC, CB are incommensurable in

square and make the sum of the j 1 < B

squares on AC, CB rational, but the

rectangle AC, CB medial ; [x. 39]

I say that AB is not so divided at another point.

For, if possible, let it be divided at D also, so that AD,
DB are also incommensurable in square and make the sum
of the squares on AD, DB rational, but the rectangle con-
tained by them medial.

Then, since that by which the squares on AC, CB differ
from the squares on AD, DB is also that by which twice the
rectangle AD, DB differs from twice the rectangle AC, CB,
while the squares on AC, CB exceed the squares on AD,
DB by a rational area — for both are rational —
therefore twice the rectangle AD, DB also exceeds twice the
rectangle AC, CB by a rational area, though they are medial :
which is impossible. [x. *6]

Therefore a major straight line is not divided at different
points;

therefore it is only divided at one and the same point.
\end{proof}

\begin{notes}

If possible, let the major irrational straight line be divided into terms in
two ways, viz. as (i +y) and (x +y), where x 1 , y are supposed to be nearer
equality than x, y.

We have then, as in x. 42, 43,

(x* +y) - (?P +/») = 2 *y - xxy.

But, by hypothesis, (x*+y), (x'+y'') are both rational, so that their
difference is rational.

Also, by hypothesis, 2x'y, xxy are both medial areas ;
therefore the difference of two medial areas is a rational area :
which is impossible. [x. 26]

Therefore etc.

\end{notes}

\end{proposition}

\begin{proposition}
\label{propX_46}

\begin{statement}
The side of a rational plus a medial area is divided at one
point only.
\end{statement}

\begin{proof}

Let AB be the side of a rational plus a medial area
divided at C, so that AC, CB are

incommensurable in square and make a d~c B

the sum of the squares on AC, CB

medial, but twice the rectangle A C, CB rational ; [x. 40]

I say that AB is not so divided at another point.

For, if possible, let it be divided at D also, so that AD,
DB are also incommensurable in square and make the sum
of the squares on AD, DB medial, but twice the rectangle
AD, DB rational.

Since then that by which twice the rectangle AC, CB
differs from twice the rectangle AD, DB is also that by
which the squares on AD, DB differ from the squares on

AC, CB,

while twice the rectangle AC, CB exceeds twice the rectangle

AD, DB by a rational area,

therefore the squares on AD, DB also exceed the squares
on AC, CB by a rational area, though they are medial :
which is impossible. [x. 26]

Therefore the side of a rational plus a medial area is not
divided at different points ;
therefore it is divided at one point only.
\end{proof}

\begin{notes}

Here, as before, if we use the same notation,

(*  4 f) - (x +/*) = 2X'/ - 2xy,
and the areas on the left side are, by hypothesis, both media), while the areas
on the right side are both rational.

Thus the result of x. 26 is contradicted, as before.

Therefore etc.

\end{notes}

\end{proposition}

\begin{proposition}
\label{propX_47}

\begin{statement}
The side of the sum of two medial areas is divided at one
point only.
\end{statement}

\begin{proof}

Let AB be divided at C, so that AC, CB are incommen-
surable in square and make the sum of the squares on AC,

CB medial, and the rectangle AC, CB medial and also in-
commensurable with the sum of the squares on them ;
I say that AB is not divided at another point so as to fulfil
the given conditions,

For, if possible, let it be divided at D, so that again AC
is of course not the same asBD, but AC is supposed greater;
let a rational straight line EF be set out,
and let there be applied to EF the rectangle EG equal to the
squares on AC, CB,

and the rectangle HK equal to twice the rectangle AC, CB ;
therefore the whole EK is equal to the square on AB. |n. 4]

Again, let EL, equal to the squares on AD, DB, be applied
to EF;

therefore the remainder, twice the rectangle .fZ), DB, is equal
to the remainder MK.

And since, by hypothesis, the sum of the squares on AC,
CB is medial,
therefore EG is also medial.

And it is applied to the rational straight line EF;
therefore HE is rational and incommensurable in length with
EF. [x. 22]

For the same reason
HN is also rational and incommensurable in length with EF.

And, since the sum of the squares on AC, CB is incom-
mensurable with twice the rectangle AC, CB,
therefore EG is also incommensurable with GN,
so that EH is also incommensurable with HN. [vi. i, x. n]

And they are rational ;

therefore EH, HN are rational straight lines commensurable

in square only ;

therefore EM is a binomial straight line divided at H. [x. 36]

Similarly we can prove that it is also divided at M.

And EH is not the same with MM ;
therefore a binomial has been divided at different points :
which is absurd. [x. 43]

Therefore a side of the sum of two medial areas is not
divided at different points ;
therefore it is divided at one point only.

Using the same notation as in the note on x. 44, we suppose that, if
possible,

x+y = x' +/,
and we put

2xy = <rv ) zxy -trv J

Then, since *' +y\ ixy are medial areas, and <r rational,

u, v are both rational and w a (1).

Also, by hypothesis, x*+y* xxy,
whence u-jV ( 2 ),

Therefore, by (i) and (*), u, v are rational and <*-,

Hence u + v is a binomial straight line. fx. 36]

Similarly u + ©' is a binomial straight line.

But u + v=u'+v';

therefore a binomial straight line is divided into terms in two ways :
which is impossible. [x. 41]

Therefore etc.
\end{proof}

\end{proposition}

\chapter*{Definitions II}

\begin{enumerate}

\item Given a rational straight line and a binomial, divided
into its terms, such that the square on the greater term is
greater than the square on the lesser by the square on a
straight line commensurable in length with the greater, then,
if the greater term be commensurable in length with the
rational straight line set out, let the whole be called a first
binomial straight line;

\item but if the lesser term be commensurable in length
with the rational straight line set out, let the whole be called
a second binomial ;

\item and if neither of the terms be commensurable in length
with the rational straight line set out, let the whole be called
a third binomial.

\item Again, if the square on the greater term be greater
than the square on the lesser by the square on a straight line
incommensurable in length with the greater, then, if the
greater term be commensurable in length with the rational
straight line set out, let the whole be called a fourth
binomial ;

\item if the lesser, a fifth binomial ;

\item and if neither, a sixth binomial.

\end{enumerate}

\begin{proposition}
\label{prop:X_48}

\begin{statement}
To find the first binomial straight line.
\end{statement}

\begin{proof}

Let two numbers AC, CB be set out such that the sum
of them AB has to BC the ratio
which a square number has to a D— H

square number, but has not to CA *

the ratio which a square number

has to a square number ; a c b

[Lemma 1 after x. 38]
let any rational straight line D be set out, and let EF be
commensurable in length with D t

Therefore EF is also rational.

Let it be contrived that,
as the number BA is to AC, so is the square on EF to the
square on FG. [x. 6, Pot.]

But AB has to AC the ratio which a number has to a
number ;

therefore the square on EF also has to the square on FG
the ratio which a number has to a number,
so that the square on EF is commensurable with the square
on FG. [x, 6]

And EF is rational ;
therefore FG is also rational.

And, since BA has not to AC the ratio which a square
number has to a square number.
neither, therefore, has the square on EF to the square on FG
the ratio which a square number has to a square number ;
therefore EF is incommensurable in length with FG. [x. 9]
Therefore EF, FG are rational straight lines commen-
surable in square only ;
therefore EG is binomial [x. 36]

I say that it is also a first binomial straight line.

For since, as the number BA is to A C, so is the square
on EF to the square on FG,
while BA is greater than AC,

therefore the square on EF is also greater than the square
on FG.

Let then the squares on FG, H be equal to the square on
EF.

Now since, as BA is to A C, so is the square on EF to the
square on FG,
therefore, convertendo,

as AB is to BC, so is the square on EF to the square on H.

[v. 19, Por.]

But AB has to BC the ratio which a square number has
to a square number ;

therefore the square on EF also has to the square on H the
ratio which a square number has to a square number.

Therefore EF is commensurable in length with H '; [x. 9]
therefore the square on EF is greater than the square on FG
by the square on a straight Hne commensurable with EF.

And EF, FG are rational, and EF is commensurable in
length with D.

Therefore EF is a first binomial straight line.
\end{proof}

\begin{notes}

Let kp be a straight line commensurable in length with p, a given rational
straight line.

The two numbers taken may be written p («* - it 1 ), /J* 1 , where (m'-n 1 ) is
not a square.

Take x such that

pm i :p(m i - n<) = ? :x> (1),

whence x = kp .

*

Then ip + x, or kp + kp   , is  first binomial straight li ne (2).

To prove this we have, from (i),

and x is rational, but Xvkp;

that is, x is rational and ``- kp,

SO that kp -t x is a binomial straight line.

Also, £*/)' being greater than a?, suppose Pp 1 - x* =y.

Then, from ( 1 ), pm* : pt? = **/>' : >'',

whence is rational and * kp.

Therefore kp + x is a jfw/ binomial straight line [x. DefT. 11. 1].

This binomial straight line may be written thus,
kp + kp *J 1 - X'.

When we come to x, 85, we shall find that the corresponding straight line
with a negative sign is Hog first apotome,

kp-kp>jW.
Consider now the equation of which these two expressions are the roots.
The equation is

x 1 - ikp . x + A 5 £V = o-
In other words, the first binomial and the first apotome correspond to the
roots of the equation

JC* — 20JC + X'o s = o,
where a = kp.

\end{notes}

\end{proposition}

\begin{proposition}
\label{propX_49}
To find the seco
\begin{statement}nd binomial straight line.

Let two numbers AC, CB be set out such that the sum
of them AB has to BC the ratio which
a square number has to a square number,
but has not to AC the ratio which a
square number has to a square number ;
let a rational straight line D be set out,
and let EF be commensurable in length
with D ;
therefore EF is rational.
\end{statement}

\begin{proof}

Let it be contrived then that,
as the number CA is to AB, so also is the square on EF to
the square on FG ; [x. 6, Por.]

therefore the square on EF is commensurable with the square
on FG. [x. 6]

Therefore FG is also rational.

Now, since the number CA has not to AB the ratio which
a square number has to a square number, neither has the
square on EF to the square on FG the ratio which a square
number has to a square number.

Therefore EF is incommensurable in length with FG

[*-9]
therefore EF, FG are rational straight lines commensurable
in square only ;
therefore EG is binomial. [x. 36]

It is next to be proved that it is also a second binomial
straight line.

For since, inversely, as the number BA is to AC, so is
the square on GF to the square on FE,
while BA is greater than AC,
therefore the square on GF is greater than the square on FE.

Let the squares on EF, H be equal to the square on GF;

therefore, convertendo, as AB is to BC, so is the square on
FG to the square on H. [v. 19, Por.]

But AB has to BC the ratio which a square number has
to a square number ;

therefore the square on FG also has to the square on H the
ratio which a square number has to a square number.

Therefore FG is commensurable in length with H ; [x. 9]
so that the square on FG is greater than the square on FE
by the square on a straight line commensurable with FG.

And FG, FE are rational straight lines commensurable
in square only, and EF, the lesser term, is commensurable in
length with the rational straight line D set out.

Therefore EG is a second binomial straight line.
\end{proof}

\begin{notes}

Taking a rational straight line kp commensurable in length with p, and
selecting numbers of the same form as before, viz. p (m* - re 1 ), /« J , we put
fi(m i -tt') :/>m , = ¥p i :  (1),

so that x = kp

Vm*-«'

= kp .-1— , say (a).

/I —A*

Just as before, x is rational and ``- kp,
whence kp + * is a binomial straight line.
By(i) f x 1 >k , p

Let **-iy=y (

whence, from (r), fm 1 :pn , = x 1 :y*,

and y is therefore rational and « x.

The greater term of the binomial straight line is x and the lesser kp, and

Ji -A'
satisfies the definition of the second binomial straight line.
The corresponding second apotome [x. 86] is

-£= -k P .

The equation of which the two expressions are the roots is

or
where

\end{notes}

\end{proposition}

\begin{proposition}
\label{propX_50}
To find the thir
\begin{statement}d binomial straight line.

Let two numbers AC, CB be set out such that the sum
of them AB has to BC the ratio which a square number has
to a square number, but has not to AC the ratio which a square
number has to a square number.
\end{statement}

\begin{proof}

Let any other number D, not square, be set out also, and
let it not have to either of the numbers BA. AC the ratio
which a square number has to a square number.

Let any rational straight line E be set out,

and let it be contrived that, as D is to AB, so is the square
on E to the square on FG ; [x. 6, Por,]

therefore the square on E is commensurable with the square
on FG. [x. 6]

And E is rational ;

therefore FG is also rational.

And, since D has not to AB the ratio which a square
number has to a square number,

neither has the square on E to the square on FG the ratio
which a square number has to a square number ;
therefore E is incommensurable in length with FG. [x. 9]

Next let it be contrived that, as the number BA is to AC,
so is the square on FG to the square on GH; [x. 6, Por.]

therefore the square on FG is commensurable with the square
on GH, [x. 6]

But FG is rational ;
therefore GH is also rational.

And, since BA has not to A C the ratio which a square
number has to a square number,

neither has the square on FG to the square on HG the ratio
which a square number has to a square number ;
therefore FG is incommensurable in length with GH. [x. 9]

Therefore FG, GH are rational straight lines commen-
surable in square only ;
therefore FH is binomial. [x. 36]

I say next that it is also a third binomial straight line.

For since, as D is to AB, so is the square on E to the
square on FG,

and, as BA is to AC, so is the square on FG to the square
on GH,

therefore, ex aequali, as D is to A C, so is the square on E to
the square on GH. [v. z*]

But D has not to AC the ratio which a square number
has to a square number;

therefore neither has the square on E to the square on GH
the ratio which a square number has to a square number ;
therefore E is incommensurable in length with GH. [x. 9]

And since, as BA is to AC, so is the square on FG to
the square on GH,
therefore the square on FG is greater than the square on GH.

Let then the squares on GH, K be equal to the square
on FG ;

therefore, converlendo, asAB is to BC, so is the square on FG
to the square on K. [v. 19, Por.]

But AB has to BC the ratio which a square number has
to a square number ;

therefore the square on FG also has to the square on K the
ratio which a square number has to a square number ;
therefore FG is commensurable in length with K. [x. 9]

Therefore the square on FG is greater than the square on
GH by the square on a straight line commensurable with FG.

And FG, GH are rational straight lines commensurable
in square only, and neither of them is commensurable in length
with E.

Therefore FH is a third binomial straight line.
\end{proof}

\begin{notes}

Let p be a rational straight line.

Take the numbers q (»' — «''), qn 1 ,
and let p be a third number which is not a square and which has not to qm 1
or q (m* - «') the ratio of square to square.

Take x such that / : qm' = p' : x* (i),

Thus x is rational and « p (2).

Next suppose that qm' :?(*»* — «*) ~x? -,y* (3).

It follows that y is rational and «- * (4),

Thus (x +y) is a binomial straight line.

Again, from (1) and (3), ex aequaU,

/;q(m'-«') = p>;y> (5),

whence yp (6).

Suppose that x? -f = **.

Then, from (3), eonverUndo,

qm 1 : qn   m x? : i 1 ,
whence £ rt *

Thus  Jx*-y>''x,

and x, y are both w p ;
therefore 1 +c is a third binomial straight line.

Now, from (i), x = p. — JO-i

.... - Jm* -n*. Jq

and, by (5), y = p. — ``-£ - —

Thus the third binomial is

A

p(m+ Jm* - «'')»

which we may write in the form

m */ . p + m Jh . p */i - X*.

The corresponding third apotomt [x. 87] is

m Jk . p — m JJt . p Vi - A*.
The two expressions are accordingly the roots of the equation
** - in Jk . px + AWiip'' = o,
or x* — tax + X'o* - o,

where a = m „JA . p.

See also note on x. 53 (ad fin.).

\end{notes}

\end{proposition}

\begin{proposition}
\label{propX_51}

\begin{statement}
To find I he fourth binomial straight line.
\end{statement}

\begin{proof}

Let two numbers AC, CB be set out such that AB
neither has to BC, nor yet to AC, the ratio
which a square number has to a square number.

Let a rational straight line O be set out,
and let EF be commensurable in length with D ;
therefore EF is also rational.

Let it be contrived that, as the number BA
is to AC, so is the square on EF to the square
on FG ; [x, 6, For,]

therefore the square on EF is commensurable
with the square on FG ; [x. 6]

therefore FG is also rational.

Now, since BA has not to AC the ratio which a square
number has to a square number,

neither has the square on EF to the square on FG the ratio
which a square number has to a square number ;
therefore EF is incommensurable in length with FG. [x. 9]

Therefore EF, FG are rational straight lines commen-
surable in square only ;
so that EG is binomial.

I say next that it is also a fourth binomial straight line.

For since, as BA is to AC, so is the square on EF to the
square on FG,
therefore the square on EF is greater than the square on FG.

Let then the squares on FG, H be equal to the square
on EF;

therefore, convertendo, as the number AB is to BC, so is the
square on EF to the square on H. [v. 19, Por.]

But AB has not to BC the ratio which a square number
has to a square number ;

therefore neither has the square on EF to the square on H
the ratio which a square number has to a square number.

Therefore EF is incommensurable in length with H \ [x. 9]
therefore the square on EF is greater than the square on GF
by the square on a straight line incommensurable with EF.

And EF, FG are rational straight lines commensurable in
square only, and EF is commensurable in length with D.

Therefore EG is a fourth binomial straight line.
\end{proof}

\begin{notes}

Take numbers m, n such that (m + n) has not to either m or n the ratio of

square to square.

Take x such that <« + »):« = Ay ; x*,

I

whence i=A)»/

kp

[ + A

say.

kp
Then kp + x, or kp +   , is s. fourth binomial straight line.

For Jp'-x? is incommensurable in length with kp, and kp is com-
mensurable in length with p.

The corresponding/twrr'ii afotome [x, 88] Is

Ji + K
The equation of which the two expressions are the loots is

X*-2kp.X+j-k>p'=0,

Or X*— 2MT + - rB 1 =0,

I + A
where a=kp.

\end{notes}

\end{proposition}

\begin{proposition}
\label{propX_52}

\begin{statement}
To find the fifth binomial straight line.
\end{statement}

\begin{proof}

Let two numbers AC, CB be set out such that AB has
not to either of them the ratio which a square number has
to a square number ;
let any rational straight line D be set out,

and let EF be commensurable with D ;
therefore EF is rational.

Let it be contrived that, as CA is to AB, so is the
square on EF to the square on FG. [x. 6, Par.]

But CA has not to AB the ratio which a
square number has to a square number ;
therefore neither has the square on EF to the
square on FG the ratio which a square number
has to a square number.

Therefore EF, FG are rational straight
lines commensurable in square only ; [* 9]
therefore EG is binomial. [x. 36]
t say next that it is also a fifth binomial straight line.

For since, as CA is to AB, so is the square on EF tc
the square on FG,

inversely, as BA is to AC, so is the square on FG to the
square on FE ;

therefore the square on GF is greater than the square on FE.

Let then the squares on EF, H be equal to the square
on GF;

therefore, conventendo, as the number AB is to BC, so is the
square on GF to the square on H. [v. 19, Por.]

But AB has not to BC the ratio which a square number

has to a square number ;

therefore neither has the square on EG to the square on H
the ratio which a square number has to a square number.

Therefore FG is incommensurable in length with H \ [x. 9]

so that the square on FG is greater than the square on FE
by the square on a straight line incommensurable with FG.

And GF, FE are rational straight lines commensurable
in square only, and the lesser term EF is commensurable in
length with the rational straight line D set out.

Therefore EG is a fifth binomial straight line.
\end{proof}

\begin{notes}

If m, n be numbers of the kind taken in the last proposition, take x such
that

m : (m + ri) = ifp' : srt

In this case

= kp Jt + A, say,

[* 5*. S3

and x =- kp.

Then ip Vi + A + Ap is ay£A binomial straight line.

For <Jx* — p\ or .kp, is incommensurable in length with Up si + A,
or ;

and £p, but not £p Vi + A, is commensurable in length with p.

The corresponding/* apotome [x. 89] is

>/>i +X-V

The equation of which the fifth binomial and the fifth apotome are the
roots is

or

where

x* — ikp <J\ + A . x + Av5y = o,

ar — 2«x + r o = o,

1 + A

a   kp V I + A*

\end{notes}

\end{proposition}

\begin{proposition}
\label{propX_53}

\begin{statement}
To find the sixth binomial straight line.
\end{statement}

\begin{proof}

Let two numbers AC, CB be set out such that AB has
not to either of them the ratio which a
square number has to a square number ;
and let there also be another number D
which is not square and which has not to
either of the numbers BA, AC the ratio
which a square number has to a square
number.

Let any rational straight line E be set
out,

and let it be contrived that, as D is to AB,
so is the square on E to the square on EG ; [x. 6, Por.)

therefore the square on E is commensurable with the square
on EG. [x. 6]

And E is rational ;

therefore EG is also rational.

Now, since D has not to AB the ratio which a square
number has to a square number,

neither has the square on E to the square on FG the ratio

which a square number has to a square number ;

therefore E is incommensurable in length with FG, [x. 9]

Again, let it be contrived that, as BA is to AC, so is the
square on FG to the square on GH. [x. 6, Por.]

Therefore the square on FG is commensurable with the
square on HG, [x. 6]

Therefore the square on HG is rational ;
therefore HG is rational.

And, since BA has not to AC the ratio which a square
number has to a square number,

neither has the square on FG to the square on GH the ratio
which a square number has to a square number ;
therefore FG is incommensurable in length with GH. [x. 9]

Therefore FG, GH are rational straight lines commen-
surable in square only ;
therefore FH is binomial. [x. 36]

It is next to be proved that it is also a sixth binomial
straight line.

For since, as D is to AB, so is the square on E to the
square on FG,

and also, as BA is to AC, so is the square on FG to the
square on GH,

therefore, ex aequaii, as D is to AC, so is the square on E
to the square on GH. [v. 12]

But D has not to AC the ratio which a square number
has to a square number ;

therefore neither has the square on E to the square on GH
the ratio which a square number has to a square number ;
therefore E is incommensurable in length with GH. [x. 9]

But it was also proved incommensurable with FG ;
therefore each of the straight lines FG, GH is incommen-
surable in length with E.

And, since, as BA is to AC, so is the square on FG to
the square on GH,
therefore the square on FG is greater than the square on GH.

Let then the squares on GH, K be equal to the square
on FG;

therefore, convertendo, as AB is to BC, so is the square on FG
to the square on K. [v. iy, For.]

But AB has not to BC the ratio which a square number
has to a square number ;

so that neither has the square on FG to the square on K the
ratio which a square number has to a square number.

Therefore FG is incommensurable in length with K; [x. 9]
therefore the square on FG is greater than the square on GH
by the square on a straight line incommensurable with FG.

hxAFG t GHaxz. rational straight lines commensurable in
square only, and neither of them is commensurable in length
with the rational straight line E set out.

Therefore FN is a sixth binomial straight line.
\end{proof}

\begin{notes}

Take numbers m, n such that (m + 11) has not to either of the numbers
m, n the ratio of square to square ; take also a third number /, which is not
square, and which has not to either of the numbers (m + «), m the ratio of
square to square.

Let /> : (»t + n) = p'': a* (1)

and (m + ») : m- x* : y* (*)

Then shall (x +y) be a sixth binomial straight line.

For, by (1), x is rational and <j p.
By (2), since x is rational,

y is rational and u x.

Hence x, y are rational and commensurable in square only, so that (x +y)
is a binomial straight line.

Again, ex aequali, from (1) and (z),

t :m = p*:y   (3).

whence y u p.

Thus x, y are both incommensurable in length with p.
Lastly, from (a), comtrftndo,

(m + ») : n = x* ; (x' — y 1 ),
so that fx* — y' u jr.

Therefore (x + y) is a sixth binomial straight line.
Now, from (1) and (3),

 n

t(tM say,

ym +
``7

/ J*- */j=P>J*i say.

and the «M5i trinomial straight line may be written

Jh . p +  jk . p.

The corresponding sixth apotomt is [x. 90]

«jk.p-J.p- )

and the equation of which the two expressions are the toots is

x* - a Jk . px + (i — A) p* » o,

*-* ,
or ar — 2<l£ + — j— o = o,

ft

where o = Jk . p.

Tannery remarks (``De la solution geometrique des problemes du second
degr£ avant Euclide'' in Mimoires de fa Sociitides sciences physiques el naturdies
it Bordeaux, 2= Serie, T. iv.) that Euclid admits as binomials and apotomes
the third and sixth binomials and apotomes which are the square roots of first
binomials and apotomes respectively. Hence the third and sixth binomials
and apotomes are the positive roots of biquadratic equations of the same form
as the quadratics which give as roots the first and fourth binomials and
apotomes. But this remark seems to be of no value because (as was pointed
out a hundred years ago by Cossali, 11. p. 160) the squares of ail the six
binomials and apotomes (including the first and fourth) give first binomials
and apotomes respectively. Hence we may equally welt regard them all as
roots of biquadratics reducible to quadratics, or generally as roots of equations
of the form

** + aa. a*" 1 ± = 0;
and nothing is gained by raising the degree of the equations in this way.

It is, of course, easy to see that the most general form of binomial and
apotome, viz.

p . h ± p . ,/A,
give first binomials and apotomes when squared.

For the square is p((A + A)p + s */hk . p\ ; and the expression within the
bracket is a first binomial or apotome, because

(1) k~k>?jkK,

(2) V(A + A)*-+iA = k - A, which is « (A + A),

(3) (* + *)p''f-

\end{notes}

\begin{lemma*}

Let there be two squares AB, BC, and let them be placed
so that DB is in a straight line with BE ;
therefore FB is also in a straight line with
BG.

Let the parallelogram AC be completed;
I say that AC is a square, that DG is a
mean proportional between AB, BC, and
further that DC is a mean proportional
between AC, CB.

For, since DB is equal to BF, and BE to BG,
therefore the whole DE is equal to the whole FG.

But DE is equal to each of the straight lines AH, KC,
and FG is equal to each of the straight lines AK, HC ; [1. 34]

therefore each of the straight lines AH, KC is also equal to
each of the straight lines AK, HC.

Therefore the parallelogram A C is equilateral.

And it is also rectangular ;
therefore AC is a square.

And since, as FB is to BG, so is DB to BE,
while, as FB is to BG, so is AB to DG,
and, as DB is to BE, so is DG to BC, [vi. i]

therefore also, as AB is to DG, so is DG to i?C. [v. 1 1]

Therefore DG is a mean proportional between -<4Z?, Z?C

I say next that DC is also a mean proportional between
AC, CB.

For since, as AD is to DK, so is KG to Ct —
for they are equal respectively —

and, componendo, as AK is to KD, so is KC to CC, [v. 18]
while, as AK is to KD, so is 4 C to CM
and, as KC is to CG, so is DC to CZ?, [vi. i]

therefore also, as AC is to DC, so is DC to i?C [v. n]

Therefore DC is a mean proportional between AC, CB.
Being what it was proposed to prove.

\end{lemma*}

\begin{notes}

It is here proved that

x* :xy-xy -.y,

and (* +>)* : (x +y)y = (* + y)y -.y.

The first of the two results is proved in the course of x. 25 (lines 6—8 on
p. 57 above). This fact may, I think, suggest doubt as to the genuineness
of this Lemma.

\end{notes}

\end{proposition}

\begin{proposition}
\label{propX_54}

\begin{statement}
If an area be contained by a rational straight line and the
first binomial, the ``side'' of the area is the irrational straight
line which is called binomial.
\end{statement}

\begin{proof}

For let the area A C be contained by the rational straight
line AB and the first binomial AD ;

I say that the ``side'' of the area AC is the irrational straight
line which is called binomial.

For, since AD is a first binomial straight line, let it be
divided into its terms at E,
and let AE be the greater term.

It is then manifest that AE, ED are rational straight lines
commensurable in square only,

the square on AE is greater than the square on ED by the
square on a straight line commensurable with AE,
and AE is commensurable in length with the rational straight
line A3 set out [x. Deff. h. i]

Let ED be bisected at the point F.

Then, since the square on AE is greater than the square
on ED by the square on a straight line commensurable with
AE,

therefore, if there be applied to the greater AE a. parallelogram
equal to the fourth part of the square on the less, that is, to
the square on EF, and deficient by a square figure, it divides
it into commensurable parts. [x. 17]

Let then the rectangle AG, GE equal to the square on
EF be applied to AE ;
therefore AG is commensurable in length with EG.

Let GH, EK, FL be drawn from G, E, F parallel to
either of the straight lines AE, CD ;

let the square SN be constructed equal to the parallelogram
AH, and the square NQ equal to GK, [11. 14]

and let them be placed so that MN is in a straight line with
NO;
therefore RN is also in a straight line with NP.

And let the parallelogram SQ be completed ;
therefore SQ is a square. [Lemma]

Now, since the rectangle AG, GE is equal to the square
on EF,

therefore, as A G is to EF, so is FE to EG ; [vi. 1 7)

therefore also, as AH is to EL, so is EL to KG ; [vi. 1]

therefore EL is a mean proportional between AH, GK.

But A His equal to SN, and GK to NQ ;
therefore EL is a mean proportional between SN, NQ.

But MR is also a mean proportional between the same
SN, NQ; [Lemma]

therefore EL is equal to MR,
so that it is also equal to PO.

But AH, GK are also equal to SN, NQ ;
therefore the whole AC is equal to the whole SQ, that is, to
the square on MO ;
therefore MO is the ``side'' of AC.

I say next that MO is binomial.

For, since AG is commensurable with GE,
therefore AE is also commensurable with each of the straight
lines AG, GE. [x. 15]

But A E is also, by hypothesis, commensurable with AB ;
therefore AG, GE are also commensurable with AB. [x. it]

And AB is rational ;
therefore each of the straight lines AG, GE is also rational ;
therefore each of the rectangles AH, GK is rational, [x. 19]
and AH is commensurable with GK.

But AH is equal to SN, and GK to NQ ;
therefore SN, NQ, that is, the squares on MN, NO, are
rational and commensurable.

And, since AE is incommensurable in length with ED,
while AE is commensurable with AG, and DE is commen-
surable with EF,

therefore AG is also incommensurable with EF,
so that AH is also incommensurable with EL. [vi.

But WIf is equal to S/V, and EL to AW ;
therefore SN is also incommensurable with MR.

But, as SAf is to MR, so is PN to A'j? ;
therefore PN is incommensurable with ATtf.

But PN is equal to MN, and A>7? to A ;
therefore MN is incommensurable with NO.

And the square on MN is commensurable with the square
on NO,

and each is rational ;

therefore MN, NO are rational straight lines commensurable
in square only.

Therefore MO is binomial [x. 36] and the ``side'' of AC,
\end{proof}

\begin{annotations}

1. ``side,'' I us? the word `` side `` in the sense explained in the note on x. Def. 4
(p. 13 above), i.e. as short for ``side or a square equal to.'' The Greek is  ri xter

\end{annotations}

\begin{notes}

A first binomial straight line being, as we have seen in x. 48, of the form
ip + kpjik 1 ,
the problem solved in this proposition is the equivalent of finding the square
root of this expression multiplied by p, or of

p(hp + kp-Ji -A 1 ),

and of proving that the said square root represents a binomial straight line
as defined in x. 36.

The geometrical method corresponds sufficiently closely to the algebraical
one which we should use.
First solve the equations

u + vkp I „

iw=1*V(i-A') I K * h

Then, if u, v represent the straight lines so found, put

x* = pu

y

and the straight line (x +y) is the square root required.
The actual algebraical solution of (1) gives

u - V = kp . X,
so that u - kp (1 + X),

v = kp(t -X),

n

and therefore * = P*/-( I + A),

': ) »

>~*s/

'(I-*).

and

*+ = P/-( I + X ) + Py -(i-*)-

This is clearly a binomial straight line as defined in x. 36.
Since Euclid has to express his results by straight lines in his figure, and
has no symbols to make the result obvious by inspecti on, he is obliged to

Strove (1) that (* + /) is the square root of p(kp + kp >J 1 - A*), and (1) that
x + y) is a binomial straight line, in the following manner.
First, he proves, by means of the preceding Lemma, that

qr«|/«/t=? (3);

therefore (* *yf = x*+y + xxy

= p(u + v) + ixy
= V + kp* Vi - A», by (1) and (3),

so that x+yJp(kp + kp>li- X').

Secondly, it results from (i), [by x. 17], that

M n V,

so that u, v are both « (u + v), and therefore « p (4);

thus «, n are rational,

whence pu, pv are both rational, and

pu n pv.

Therefore a?, v 1 are rational and commensurable (5).

Next, kp yj kp 1/ 1 — if,
and kp « u, while kp Vr - X'' « £,*p -Ji-X';
therefore » ~ kp J\ - X',

whence pv ~ kp' «/r -X*,

or j u ay,

so that x uf.

By this and (5), x, y are rational and «-, so that (jr+.p) is a binomial
straight line. [x. 36]

X. 91 will prove in like manner that a like theorem holds for apotomes,
viz. that

P J\ (t * A)-* v/fo*) = Vp(*P-Wi''-A*).

Since the rri binomial straight line and the ./frrf apotomt are the root* of
the equation

jc 5 — zip . x + Wp* = o,
this proposition and x. 91 give us the solution of the biquadratic equation

X 4 -2kp'.X* + XWp' - o.

\end{notes}

\end{proposition}

\begin{proposition}
\label{propX_55}

\begin{statement}
If an area be contained by a rational straight line and the
second binomial, the ``side'' of the area is the irrational straight
line vthich is called a first bimediat.
\end{statement}

\begin{proof}

For let the area ABCD be contained by the rational
s straight line AB and the second binomial AD ;

I say that the ``side'' of the area AC is a first bimedial straight

line.

For, since AD is a second binomial straight line, let it be

divided into its terms at E, so that AE is the greater term ;
10 therefore AE, ED are rational straight lines commensurable

in square only,

the square on AE is greater than the square on ED by the

square on a straight line commensurable with AE,

and the lesser term ED is commensurable in length with AB.

[x. Deff. 11. 1]
ij Let ED be bisected at E,

and let there be applied to AE the rectangle AG, GE equal
to the square on EF and deficient by a square figure ;
therefore AG is commensurable in length with GE. [x. 17]
Through G, £, F let GH, EK, FL be drawn parallel to
*>AB, CD,
let the square SN be constructed equal to the parallelogram
AH, and the square NQ equal to GK,
and let them be placed so that MN is in a straight line with

N0 >
is therefore RN is also in a straight line with NP.

Let the square SQ be completed.

It is then manifest from what was proved before that MR
is a mean proportional between SN, NQ and is equal to EL,
and that MO is the ``side'' of the area AC.

It is now to be proved that MO is a first bimedial straight line.

Since AE is incommensurable in length with ED,
while ED is commensurable with AS,
therefore AE is incommensurable with AB. [x. 13]

And, since AG is commensurable with EG,
35 AE is also commensurable with each of the straight lines
AG, GE. (x. 15]

But AE is incommensurable in length with AB ;
therefore AG, GE are also incommensurable with AB. [x. 13]

Therefore BA, AG and BA, GE are pairs of rational
40 straight lines commensurable in square only ;
so that each of the rectangles AH, GK is medial. [x. ai]

Hence each of the squares SN, NQ is medial.

Therefore MN, NO are also medial.

And, since AG is commensurable in length with GE,
45 AH is also commensurable with GK, [vi. 1, x. 11]

that is, SN is commensurable with NQ,
that is, the square on MN with the square on NO.

And, since AE is incommensurable in length with ED,

while AE is commensurable with AG,
so and ED is commensurable with EF,

therefore AG is incommensurable with EF; [x. 13]

so that AH is also incommensurable with EL,

that is, SN is incommensurable with MR,

that is, PN with NR, [vi. i, x. n]

ss that is, MN is incommensurable in length with NO.

But MN, NO were proved to be both medial and com-
mensurable in square ;

therefore MN, NO are medial straight lines commensurable
in square only.

60 1 say next that they also contain a rational rectangle.

For, since DE is, by hypothesis, commensurable with each
of the straight lines AB, EF,

therefore EF is also commensurable with EK. [x. u]

And each of them is rational ;

65 therefore EL, that is, MR is rational, L x - '9]

and MR is the rectangle MN, NO.

But, if two medial straight lines commensurable in square

only and containing a rational rectangle be added together, the

whole is irrational and is called a first bimedial straight line.

[*  37]
70 Therefore MO is a first bimedial straight line.
\end{proof}

\begin{annotations}

39. Therefore BA, AG and BA, OE are pairs of rational straight line a com-
mensurable in square only. The text has ``Therefore BA, AG, GE are rational straight
lines commensurable in square only.'' which I have altered because it would naturally convey
the impression that any two of the three straight lines ate com mensurable in square only,
whereas AG, GE are commensurable in length (1. 18), and it is only the other two pain
which are commensurable in square only.

\end{annotations}

\begin{notes}

A second binomial straight line being [x. 49] of the form
the present proposition is equivalent to finding the square root of the expression

As in the last proposition, Euclid finds u, o from the equations
u + v = -

then finds x, y from the equations

x* = pu

(1),
-(a).

and then proves (a) that

*+J=vA (;/== + ).

and (/J) that (.v + v) is a first bimedial straight line [x. 37].

The steps in the proof are as follows.

For (a) reference to the corresponding part of the previous proposition
suffices.

09) By (!) and X. 17,

«« w;

therefore u, v are both rational and rt (« + f), and therefore */ f> [by (1 )]...( 3).

Hence pu, pv, or x*, y, are medial areas,
so that x,y are also medial (4)-

But, since u « p,

`` (5)-

Again (« + w), or 7==, yip,

so that it u ip ,

whence pa v lip',

or x? v xy,

and *> (6).

Thus [(4), (5), (6)] *, ,v are medial and ``-.
Lastly, acy = iip 1 , which is rational.
Therefore (x + y) is  first bimedial straight line.
The actual straight lines obtained from (1) are

U=-=:kp )

vi-i 1 I
1 >-* 1 I''

sothat *+J' = `` N /»(T) + W(tTx)'

The corresponding jfrj/ apolome of a medial straight line found in x, 92
being the same thing with a minus sign between the terms, the two expressions
are the roots of the biquadratic

being the equation in r 1 corresponding to that in x in x. 49.

\end{notes}

\end{proposition}

\begin{proposition}
\label{propX_56}

\begin{statement}
If an area be contained by a rational straight line and the
third binomial, the ``side'' 0/ the area is the irrational straight
line called a second bimedial.
\end{statement}

\begin{proof}

For let the area ABCD be contained by the rational
straight line AB and the third binomial AD divided into its
terms at E, of which terms AE is the greater ;
I say that the **side'' of the area AC is the irrational straight
line called a second bimedial.

For let the same construction be made as before.

Now, since AD is a third binomial straight line,
therefore AE, ED are rational straight lines commensurable
in square only,

the square on AE is greater than the square on ED by the
square on a straight line commensurable with AE,
and neither of the terms AE, ED is commensurable in length
with AB. I*. Deff. 11. 3]

Then, in manner similar to the foregoing, we shall prove
that MO is the ``side'' of the area AC,

and MN, NO are medial straight lines commensurable in
square only ;
so that MO is bimedial.

It is next to be proved that it is also a second bimedial
straight line.

Since DE is incommensurable in length with AB, that is,
with EK,

and DE is commensurable with EE,
therefore EE is incommensurable in length with EK. [x. 13]

And they are rational ;

therefore FE, EK are rational straight lines commensurable
in square only.

Therefore EL, that is, MR, is medial. [x. ji]

And it is contained by MN, NO ;
therefore the rectangle MN, NO is medial.

Therefore MO is a second bi medial straight line. [x. 3 8 ]
\end{proof}

\begin{notes}

This proposition in like manner is the equivalent of finding the square
1 jot or the product of p and the third Hnomial [x. 50], i.e. of the expression

pUk.p+JM.pJTX).
As before, put

« + v=Jk.p 1

uv = kp l (i-k t ) J w '

Next, «, v being found, let

x* = pu,

then (a- + v) is the square root required and is a second bimedial straight line.

t x -38]
For, as in the last proposition, it is proved that (x +v) is the square root,
and x, y are medial and «-»

Again, xy = § Jk . p'' ,/i -A*, which is media/.
Hence (* +v) is a second bimedial straight line.
By solving equations (1), we find

*«1<V*. J. + *</*.*>),

and x+y = p J  (. + X) + p J (l - X).

The corresponding second apotome of a medial found in X. 93 is the same
thing with a minus sign between the terms, and the two are the roots (cf. note
on x. 50) of the biquadratic equation

x 4 -3ji.p*x> + *»V = »

\end{notes}

\end{proposition}

\begin{proposition}
\label{propX_57}

\begin{statement}
If an area be contained by a rational straight line and the
fourth binomial, the ``side'' of the area is the irrational straight
line called major.
\end{statement}

\begin{proof}

For let the area AC he. contained by the rational straight
line AB and the fourth binomial AD divided into its terms
at E, of which terms let AE be the greater ;
I say that the ``side'' of the area AC is the irrational straight
line called major.

For, since AD is a fourth binomial straight line,
therefore AE, ED are rational straight lines commensurable
in square only,

the square on AE is greater than the square on ED by the
square on a straight line incommensurable with AE,
and AE is commensurable in length with AB. [x. Deff. 11. 4]

Let DE be bisected at F,
and let there be applied to AE a parallelogram, the rectangle
AG, GE, equal to the square on EF;
therefore AG is incommensurable in length with GE, [x. 18]

Let GH, EK, FL be drawn parallel to AB,
and let the rest of the construction be as before ;
it is then manifest that MO is the ``side'' of the area AC.

It is next to be proved that MO is the irrational straight
line called major.

Since AG is incommensurable with EG,
AH is also incommensurable with GK, that is. SN with NQ ;

[VI. I, X. It]

therefore MN, NO are incommensurable in square.

And, since AE is commensurable with A3,
AK is rational ; [x. 19]

and it is equal to the squares on MN, NO ;
therefore the sum of the squares on MN, NO is also rational.

And, since DE is incommensurable in length with AB,
that is, with EK,

while DE is commensurable with EF,
therefore EF is incommensurable in length with EK. [x. 13]

Therefore EK, EF are rational straight lines commen-
surable in square only;
therefore LE, that is, MR, is medial. [x. 21]

And it is contained by MN, NO ;
therefore the rectangle MN, NO is medial.

And the [sum] of the squares on MN, NO is rational,
and MN, NO are incommensurable in square.

But, if two straight lines incommensurable in square and
making the sum of the squares on them rational, but the
rectangle contained by them medial, be added together, the
whole is irrational and is called major. [x. 39]

Therefore MO is the irrational straight line called major
and is the ``side'' of the area AC.
\end{proof}

\begin{notes}

The problem here is to find the square root of the expression [cf. x. 51]

\ -Ji + k/

The procedure is the same.
Find », v from the equations

u + v = kp

W = l— pT f

4 i + M

-M.

and, if x* = pu 1 , .

y=pW *''*

(x +y) is the required square root.

To prove that ( +>) is the major irrational straight line Euclid argues
thus.

By x, 18, u w v,

therefore pu w pv,

or y,

sothat x-y .... (3).

Now, since (u + v)* p,

(u + v)p, or (x'+y), is a rational area (4).

An*

Lastly, xy=;l-~==, which is a medial area (5).

Vi + X

Thus [(3)1 (4), (5)] ( x +y) is  major irrational straight line. [x. 39]

Actual solution gives

and *+j--pVK i + /7 + 'V;( i- v7ta)

The corresponding square root found in x. 94 is the minor irrational
straight line, the terms being separated by a minus sign, and the two straight
lines are the roots (cf. note on x. 5 1) of the biquadratic equation

x* - lip 1 . x* + r iff* = o.

r 1 + A r

\end{notes}

\end{proposition}

\begin{proposition}
\label{propX_58}

\begin{statement}
If an area be contained by a rational straight line and the
fifth binomial, the ``side `` of the area is the irrational straight
line called the side of a rational plus a medial area.
\end{statement}

\begin{proof}

For let ihe area AC be contained by the rational straight
line AB and the fifth binomial AD divided into its terms at
E, so that AE is the greater term ;

I say that the ``side'' of the area AC is the irrational straight
line called the side of a rational plus a medial area.

For let the same construction be made as before shown ;
it is then manifest that MO is the ``side'' of the area AC.

It is then to be proved that MO is the side of a rational
plus a medial area.

For, since AG is incommensurable with GE,  . 18]

therefore AH is also commensurable with HE, [vi. 1, x. n]
that is, the square on MN with the square on NO ;
therefore MN, NO are incommensurable in square.

And, since AD is a fifth binomial straight line, and ED
the lesser segment,
therefore ED is commensurable in length with AB.

fx. Deff. 11. 5]

But AE is incommensurable with ED ;
therefore AB is also incommensurable in length with AE.

[* '3]

Therefore AK, that is, the sum of the squares on MN,
NO, is medial. [x. 21]

And, since DE is commensurable in length with AB, that
is. with EK,

while DE is commensurable with EF,
therefore EF is also commensurable with EK. [x. 1a)

)   < >

; (ft

And EK is rational ;
therefore EL, that is, MR, that is, the rectangle MN, NO, is
also rational. [x. 19]

Therefore MN, NO are straight lines incommensurable
in square which make the sum of the squares on them medial,
but the rectangle contained by them rational.

Therefore MO is the side of a rational plus a medial area
[x. 40] and is the ``side'' of the area AC.
\end{proof}

\begin{notes}

We have here to find the square root of the expression [cf. x. 52]

p (hp J 1 + A + kp).
As usual, we put

u + v = kpji + A

uv = i *y

Then, u, v being found, we take

x* = pu
f = pv
and (x +y), so fouuu, is our required square root.

Euclid's proof of the class of (x +>>) is as follows :

By x. 18, « u v;
therefore pu v pv,
so that x 1 ~y,
and x~y (3).

Next u + v -j kp

``ft

whence p (u + v), or ( 4-y 1 ), is a medial area (4).

Lastly, xy = kf?, which is a rational area (5).

Hence [(3), (4), (5)] (x+y) is the side of a rational plus a medial area.

|x, 40]
If we solve algebraically, we obtain

* = U/rTA+,/A),
and x +y = P y - (7i + A + ,A) + P V - U l + * '

2 `` '-.*)'' v a-T* v '

The corresponding `` side `` found in x. 95 is a straight lint which produces

with a rational area a medial whole, being of the form [x -y), where x, y

have the same values as above.

The two square roots are (cf. note on x. 52) the roots of the biquadratic

equation

\end{notes}

\end{proposition}

\begin{proposition}
\label{propX_59}

\begin{statement}
If an area be contained by a rational straight line and the
sixth binomial, the ``side'' of the area is the irrational straight
line called the side of the sum of two medial areas.
\end{statement}

\begin{proof}

For let the area ABCD be contained by the rational
straight line AB and the sixth binomial AD, divided into its
terms at E, so that AE is the greater term ;
I say that the ``side'' of AC is the side of the sum of two
medial areas.

Let the same construction be made as before shown.

A Q E F D R Q

H K

N

It is then manifest that MO is the ``side'' of AC, and
that MN is incommensurable in square with NO.

Now, since EA is incommensurable in length with AB,
therefore EA, AB are rational straight lines commensurable
in square only ;

therefore AK, that is, the sum of the squares on MN, NO,
is medial. [x. 21]

Again, since ED is incommensurable in length with AB,
therefore FE is also incommensurable with EK; [x. 13]

therefore FE, EK are rational straight lines commensurable
in square only ;

therefore EL, that is, MR, that is, the rectangle MN, NO, is
medial. [x. 21]

And, since AE is incommensurable with EF,
A A' is also incommensurable with EL. [vi. i, x. it]

But AK is the sum of the squares on MN, NO,
and EL is the rectangle MN, NO

therefore the sum of the squares on MN, NO is incommen-
surable with the rectangle MN. NO,

And each of them is medial, and MN, NO are incom-
mensurable in square.

Therefore MO is the side of the sum of two medial areas
[x. 41], and is the `` side `` of AC.
\end{proof}

\begin{notes}

Euclid here finds the square root of the expression [cf, x. 53]

p(JA.p + J. p).
As usual, we solve the equations

* + P = V*. P i

p 5 J (h

) (*).

uv- kp 1
then, », v being found, we put

y = pv
and (* + y) is the square root required.

Euclid proves that (x + y) is the side of (the sum of) two medial areas, as
follows.

As in the last two propositions, x, y are proved to be incommensurable
in square.

Now Jk , p, p are. commensurable in square only ;
therefore p(u + v), or (x* +y 1 ), is a media/ area (3),

Next, xy m   ,/A \ p\ which is again a media/ area (4).

Lastly, Jfc . p « \ Jk . p,
so that V*V ~i*/k.p*;
'hat is, (x*+?) v xy „ (5).

Hence [(3), (4), (5)] (x +y) is the side of the sum of two medial areas.

Solving the equations algebraically, we have

and x+y = P J\ (Jk + <Jk=k) + p JU* ~ J*>-

The corresponding square root found in x. 96 is x-y, where x,y are the

same as here.

The two square roots are (cf. note on x. 53) the roots of the biquadratic

equation

**-a./*.pV + (A-XV = o.

t

\end{notes}

\begin{lemma*}

[Lemma.

If a straight line be cut into unequal parts, the squares
on the unequal parts are greater

than twice the rectangle con- £ ? — £ P

tained by the unequal parts.

Let AB be a straight line, and let it be cut into unequal
parts at C, and let A C be the greater ;

I say that the squares on AC, CB are greater than twice the
rectangle AC, CB.

For let AB be bisected at D.

Since then a straight line has been cut into equal parts
at D, and into unequal parts at C,

therefore the rectangle AC, CB together with the square on
CD is equal to the square on AD, [n. 5]

so that the rectangle AC, CB is less than the square on AD;
therefore twice the rectangle AC, CB is less than double of
the square on AD.

But the squares on AC, CB are double of the squares on
AD, DC; [11. 9]

therefore the squares on AC, CB are greater than twice the
rectangle AC, CB.

Q.E.D.]

\end{lemma*}

\begin{notes}

We have already remarked (note on x. 44) that the Lemma here proving
that

x* + jp > 2xy
can hardly be genuine, since the result is used in x. 44.

\end{notes}

\end{proposition}

\begin{proposition}
\label{propX_60}

\begin{statement}
The square on the binomial straight line applied to a
rational straight line produces as breadth the first binomial.
\end{statement}

\begin{proof}

Let AB be a binomial straight line divided into its terms
at C, so that AC is the greater term ;

let a rational straight line DE be ° £— y

set out,

and let DEFG equal to the square

on AB be applied to DE producing

DG as its breadth ;

I say that DG is a first binomial

straight line.

For let there be applied to DE the rectangle DH equal
to the square on AC, and KL equal to the square on BC;
therefore the remainder, twice the rectangle AC, CB, is equal
to MF.

Let MG be bisected at A*'', and let NO be drawn parallel
[to ML or GF

Therefore each of the rectangles MO, NF is equal to
once the rectangle AC, CB.

Now, since AB is a binomial divided into its terms at C,

therefore AC, CB are rational straight lines commensurable
in square only ; [x. 36]

therefore the squares on AC, CB are rational and commen-
surable with one another,
so that the sum of the'' squares or A C, CB is also rational.

[x. , S ]
And it is equal to DL ;

therefore DL is rational.

And it is applied to the rational straight line DE ;
therefore DM is rational and commensurable in length with
DE. [x. to]

Again, since AC, CB are rational straight lines commen-
surable in square only,
therefore twice the rectangle AC, CB, that is ME, is medial.

[X. 2l]

And it is applied to the rational straight line ML ;
therefore MG is also rational and incommensurable in length
with ML, that is, DE. (x. 22]

But MD is also rational and is commensurable in length
with DE ;
therefore DM is incommensurable in length with MG. [x. 13]

And they are rational ;

therefore DM, MG are rational straight lines commensurable

in square only;

therefore DG is binomial. [x. 36]

It is next to be proved that it is also a first binomial
straight line.

Since the rectangle AC, CB is a mean proportional between
the squares on AC, CB, [cf. Lemma after x. 53]

therefore MO is also a mean proportional between DH, KL.

Therefore, as DH is to MO, so is MO to KL,

that is, as DK is to MN, so is MN to MK\ [vi. 1]

therefore the rectangle DK, KM is equal to the square
on MN. [vi. <(

And, since the square on AC is commensurable with the
square on CB,

DH is also commensurable with KL,
so that DK is also commensurable with KM. [vi. 1, x. n]

And, since the squares on AC, CB are greater than twice
the rectangle AC, CB, [Lemma]

therefore DL is also greater than MF,
so that DM is also greater than MG, [vi. 1]

And the rectangle DK, KM is equal to the square on
MN, that is, to the fourth part of the square on MG,
and DK is commensurable with KM.

But, if there be two unequal straight lines, and to the greater
there be applied a parallelogram equal to the fourth part of
the square on the less and deficient by a square figure, and
if it divide it into commensurable parts, the square on the
greater is greater than the square on the less by the square
on a straight line commensurable with the greater ; [x. 17]
therefore the square on DM is greater than the square on
MG by the square on a straight line commensurable with DM,

And DM, MG are rational,

and DM, which is the greater term, is commensurable in length
with the rational straight line DE set out.

Therefore DG is a first binomial straight line. [x. Deft'', n. i]
\end{proof}

\begin{notes}

In the hexad of propositions beginning with this we have the solution of
the converse problem to that of X. 54 — 59. We find the squares of the
irrational straight lines of x. 36 — 41 and prove that they are respectively equal
to the rectangles contained by a rational straight line and the first, second,
third, fourth, fifth and sixth binomials.

In x. 60 we prove that, p + Jh . p being a binomial straight line [x. 36],

is a first binomial straight line, and we find it geometrically.
The procedure may b<
Take x,y, t such that

The procedure may be represented thus.

ax = p*,

try = kf?
it . 2  = 2 Jk . p',
p\ hp   being of course the squares on the terms of the original binomial,
and 2 Jk.p 1 twice the rectangle contained by them.

Then tf ) t * r fe±d£,

and we have to prove that (x+y) + 21 is a. first binomial straight line of which
(x+y), 2z are the terms and (x+y) the greater.

Euclid divides the proof into two parts, showing first that (x + y) + 2 is
some binomial, and secondly that it is the first binomial.

(a) p 'w <J . p, so that p 5 , ip* are rational and commensurable ;

therefore p* + kp\ or cr (je +), is a rational area,

whence (x +y) is rational and `` a (1).

Next, 2p . Ji . p is a medial area,
so that o- . 25 is a medial area,
whence 2s is rational but  <r (2).

Hence [(t), (2)] (-v+), iz are rational and commensurable in square

only - (3);

thus (x +y) + tz is a binomial straight line. [x. 36]

W pO-.jk.pjk.t?; kp

SO that <r x ; trz = az : try,

and x : z = z :y,

or .xyz'-liizf (4).

Now p 1 , Ap* are commensurable, so that <r.v, try are coin mensurable, and
therefore

**y (s)-

And, since [Lemma] p 3 + ip' > 2 JA . p a ,
x + y > 2Z.

But (x + y) is given, being ei|ual to (6).

Therefore [(4), (), (6), and x. 17] J(x +yy~(tjif « ( x +y).
And (x +y), 2t are rational and ``- [(3)],
while (*+j») -« [(!)].

Hence (x +y) + 2z is  first binomial.
The actual value of (x +y) + 22 is, of course,
pt>

- (l +-4 + 2 Jk).

\end{notes}

\end{proposition}

\begin{proposition}
\label{propX_61}

\begin{statement}
The square on the first bimedial straight line applied to a
rational straight line produces as breadth the second binomial.
\end{statement}

\begin{proof}

Let AB be a first bimedial straight line divided into its
medials at C, of which medials AC

is the greater;  D K m ng

let a rational straight line DE be set
out,

and let there be applied to DE the
parallelogram DF equal to the square

on AB, producing DG as its breadth; e h l o f

I say that DG is a second binomial a ~~6 8

straight line.

For let the same construction as before be made.

Then, since AB is a first bimedial divided at C,
therefore AC, CB are medial straight lines commensurable in
square only, and containing a rational rectangle, [x. 37]

so that the squares on AC, CB are also medial. [x, 21]

Therefore DL is medial. [x, 15 and 23, Por.]

And it has been applied to the rational straight line DE ;
therefore MD is rational and incommensurable in length
with DE. [x. 22]

Again, since twice the rectangle A C, CB is rational, ME is
also rational.

And it is applied to the rational straight line ML ;
therefore MG is also rational and commensurable in length
with ML, that is, DE ; [x. 20]

therefore DM is incommensurable in length with MG. [x. 13]

And they are rational ;
therefore DM, MG are rational straight lines commensurable
in square only ;
therefore DG is binomial. [x. 36]

It is next to be proved that it is also a second binomial
straight line.

For, since the squares on AC, CB are greater than twice
the rectangle AC, CB,
therefore DL is also greater than ME,
so that DM is also greater than MG. [vi. 1]

And, since the square on AC is commensurable with the
square on CB,

DH is also commensurable with KL,
so that DK is also commensurable with KM. [vi. 1, x. n]

And the rectangle DK, KM is equal to the square on MN;

therefore the square on DM is greater than the square on

MG by the square on a straight line commensurable with DM,

[x- 17]
And MG is commensurable in length with DE,

Therefore DG is a second binomial straight line. [x. Deff. 11. z
\end{proof}

\begin{notes}

In this case we have to prove that, ( ? + p) being a first bimedial
straight line, as found in X. 37,

a

is a stand binomial straight line.

The form of the proposition ana the figure being similar to those of x. 60,
I can somewhat abbreviate the reproduction of the proof.
Take x, y, z such that

ax = k 'p,

a.xz = tip'.
Then shall (x + y) + iz be a second binomial.

(a) A*p, k'p.aie medial straight lines commensurable in square only and
containing a rational rectangle. [x. 37)

The squares £*p*> V are medial ;
thus the sum, or o-(je +y), is medial. (x. 23, Por.]

Therefore (x +y) is rational and  <r.

And it . 2: is rational ;
therefore 22 is rational and « <r (1).

Therefore (£ + > ), as are rational and ``- (2),

so that (x +y) + 2z is a binomial.

<fi) As before, (* +) > as.

Now, JPff, iPff being commensurable,

And jry = s',

... *v + *v

while *+ v= — — .

Hence [x. 17] J(x+y)' -sfix+y) (3).

But U « <r, by (1).

Therefore [(1), (2), (3)] (jc +>>) + a* is a second binomial straight line.

Of course (x+y) + az = - ((1 +/4) + a*).

\end{notes}

\end{proposition}

\begin{proposition}
\label{propX_62}

\begin{statement}
7''j4e square an the second bimedial straight line applied to
a rational straight line produces as breadth the third binomial.
\end{statement}

\begin{proof}

Let AB be a second bimedial straight line divided into
its medials at C, so that AC is the
greater segment ; '- '

let DE be any rational straight line,
and to DE let there be applied the
parallelogram DF equal to the square
on AB and producing DG as Its
breadth ;

I say that DG is a third binomial
straight line.

Let the same construction be made as before shown

Then, since AB is a second bimedial divided at C,
therefore AC, CB are medial straight lines commensurahle in
square only and containing a medial rectangle, [x, 38]

so that the sum of the squares on AC, CB is also medial.

[x. 15 and 23 Por.]
And it is equal to DL ;

therefore DL is also medial.

And it is applied to the rational straight line DE ;
therefore MD is also rational and incommensurable in length
with DE. [x. zx

For the same reason,
MG is also rational and Incom mensurable in length with ML,
that is, with DE ;

therefore each of the straight lines DM, MG is rational and
incommensurable in length with DE.

And, since AC is incommensurable in length with CB,
and, as AC is to CB, so is the square on AC to the rectangle
AC, CB,

therefore the square on A C is also incommensurable with the
rectangle AC, CB. [x. n]

Hence the sum of the squares on AC, CB is incommen-
surable with twice the rectangle AC,' CB, [x. 12, 13]
that is, DL is incommensurable with ME,
so that DM is also incommensurable with MG. [vi. 1, x. 11]

And they are rational ;
therefore DG is binomial. [x. 36]

It is to be proved that it is also a third binomial straight line.

In manner similar to the foregoing we may conclude that
DM is greater than MG,
and that DK is commensurable with KM.

And the rectangle DK, KM is equal to the square on
MN;

therefore the square on DM is greater than the square on
MG by the square on a straight line commensurable with
DM.

And neither of the straight lines DM, MG is commen-
surable in length with DE.

Therefore DG is a third binomial straight line, [x, Deff. 11. 3]
\end{proof}

\begin{notes}

We have to prove that [cf. x, 38]

1 /.i X W

is a third binomial straight line.
Take x, y, z such that

<TX = A V,

V

'

CF . 2Z = 2 ,/A. . p>.

(o) Now £*p, — ~ are medial straight lines commensurable in square only

k*
and containing a medial rectangle. [x. 38]

The sum of the squares on them, or o- (* +y), is media/;

therefore (x +y) is rational and ~ a (1).

And ..: . 22 being medial also,

2Z is rational and U a i 2 )-

Now (V-f

= trx : trz,
whence trx ~ o-z.

But (A*pY «(y + ra''), or

trjc a tr (jc + y), and <tz « cr . 2a ;

therefore <r(x +y) ~ a . tz,

or (x+y)> 2Z (3).

Hence [(0, (2), (3)] (x +y) + 2z is a binomial straight line (4).

(fi) As before, (x +y) > 2z,

and *«.

Also xy = §*.

Therefore [x. 17] J(x+yf-(*z) 1 « (x+jr).

And [(1), (2)] neither (jc + j') nor 22 is « tr.

Therefore (* +.?) + 2Z is a ZiVrf binomial straight line.

Obviously (x +y) + 2z   - j —jr + 2 V*[

\end{notes}

\end{proposition}

\begin{proposition}
\label{propX_63}

\begin{statement}
The square on the major straight line applied to a rational
straight line produces as breadth the fourth binomial.
\end{statement}

\begin{proof}

Let AB be a major straight line divided at C, so that AC
is greater than CB ;
let DE be a rational straight line,

and to DE let there be applied the parallelogram DF equal
to the square on AB and producing DG as Its breadth ;
I say that DG is a fourth binomial
straight line.

Let the same construction be
made as before shown.

Then, since AB is a major
straight line divided at C,
AC, CB are straight lines incom-
mensurable in square which make c e
the sum of the squares on them
rational, but the rectangle contained by them medial. [x. 39]

Since then the sum of the squares on AC, CB is rational,
therefore DL is rational ;

therefore DM is also rational and commensurable in length
with DE. [x. 20]

Again, since twice the rectangle AC, CB, that is, ME, is
medial,

and it is applied to the rational straight line ML,
therefore MG is also rational and incommensurable in length
with DE ; [x. 22]

therefore DM is also incommensurable in length with MG.

Therefore DM, MG are rational straight lines commen-
surable in square only ;
therefore DG is binomial. [x. 36]

It is to be proved that it is also a fourth binomial straight line.

In manner similar to the foregoing we can prove that
DM is greater than MG,
and that the rectangle DK, KM is equal to the square on MN.

Since then the square on AC is incommensurable with the
square on CB,

therefore DH is also incommensurable with KL,
so that DK is also incommensurable with KM. [vi. 1, x. n]

But, if there be two unequal straight lines, and to the
greater there be applied a parallelogram equal to the fourth
part of the square on the less and deficient by a square
figure, and if it divide it into incommensurable parts, then the
square on the greater will be greater than the square on the
less by the square on a straight line incommensurable in
length with the greater ; [x. 18]

therefore the square on DM is greater than the square on
MG by the square on a straight line incommensurable with
DM.

And DM, MG are rational straight lines commensurable
in square only,

and DM is commensurable with the rational straight line DE
set out.

Therefore DG is a fourth binomial straight line. [x. Deff. 11. 4]
\end{proof}

\begin{notes}

We y v/e to prove that [cf, x. 39]

is a fourth binomial straight line.

For brevity we must call this expression

-im  vf.

tr v
Take x, y, z such that

<MC = «*

try = 1?

tr , 23 ZUV

wherein it has to be remembered [x, 39] that u, v are incommensurable in

square, (« ! + sr) is rational, and uv is medial.

(a) («* + (/), and therefore a (x +y), is rational ;

therefore (x +y) is rational and < » a (1).

3uv, and therefore a . zz, is medial ;
therefore zz is rational and v a (2).

Thus ( * +>), 2 * are rational and - (3),

so that (x +y) + tz is a binomial straight line.
() As before, x+y> iz,

and y = *''

Now, since a'  tr 1 ,
ax u oj 1 , or x y.

Hence [x. 18] *J(x +yy -(**)* ~(x +y) (4).

And (x +y) « <r, by (1).

Therefore [(3), (4)] (.* +.y) + 2* is a fourth binomial straight line.

It is of course j» 1 1 + , 5 l .

\end{notes}

\end{proposition}

\begin{proposition}
\label{propX_64}

\begin{statement}
The square on the side of a rational plus a medial area
applied to a rational straigltt line produces as breadth the fifth
binomial.
\end{statement}

\begin{proof}

Let AB be the side of a rational plus a medial area,
divided into its straight lines at C,
so that AC is the greater ;
let a rational straight line DE be set
out,

and let there be applied to DE the
parallelogram DF equal to the square
on AB, producing DG as its breadth ;
I say that DG is a fifth binomial
straight line.

Let the same construction as before be made.

Since then AB is the side of a rational plus a medial
area, divided at C,

therefore AC, CB are straight lines incommensurable in square
which make the sum of the squares on them medial, but the
rectangle contained by them rational. [x. 40]

Since then the sum of the squares on AC, CB is medial,
therefore DL is medial,

so that DM is rational and incommensurable in length with
DE. [x. »]

Again, since twice the rectangle AC, CB, that is ME, is
rational,
therefore MG is rational and commensurable with DE. [x. 20]

Therefore DM is incommensurable with MG ; [x. 13]

therefore DM, MG are rational straight lines commensurable
in square only ;
therefore DG is binomial. [x. 36]

I say next that it is also a fifth binomial straight line.

For it can be proved similarly that the rectangle DK, KM
is equal to the square on MN,

and that DK is incommensurable in length with KM;
therefore the square on DM is greater than the square on MG
by the square on a straight line incommensurable with DM.

And DM, MG are commensurable in square only, and the
less, MG, is commensurable in length with DE.
Therefore DG is a fifth binomial.
\end{proof}

\begin{notes}

To prove that [cf. x. 40]

is ay/A binomial straight line.

1
For brevity denote it by - (a + vf, and put

(T . 2Z = 2«P.

Remembering that [x. 40] n'i?, («* + ii*) is medial, and 2uv is rational,
we proceed thus.

(o) o- (x ±y) is media] ;

therefore (x +y) is rational and ur (1).

Next, a . 2z is rational ;
therefore as is rational and * o- (2).

Thus (a - +,v), 2z are rational and <*- (3),

so that (jr +J 1 ) + 22 is a binomial straight line,
(J3) As before, * + y > 23,

*? = **,
and x uj 1 .

Therefore [x. 18]  '(*+.)')* - J'' „ (* +j<) (4).

Hence [(2), (3), (4)] (x +y) + iz is a fifth binomial straight line.

It is of course  - \ , + „i .

\end{notes}

\end{proposition}

\begin{proposition}
\label{propX_65}

\begin{statement}
The square on the side of the sum of two medial areas
applied to a rational straight line produces as breadth the
sixth binomial
\end{statement}

\begin{proof}

Let AB be the side of the sum of two medial areas,
divided at C,

let DE be a rational straight line,

and let there be applied to DE the parallelogram DF equal
to the Square on AB, producing DG as its breadth ;

I say that DG is a sixth binomial straight line.

For let the same construction be made as before.

Then, since AB is the side of
the sum of two medial areas, divided
at C,

therefore AC, CB are straight lines
incommensurable in square which
make the sum of the squares on
them medial, the rectangle contained
by them medial, and moreover the
sum of the squares on them incom-
mensurable with the rectangle contained by them, [x. 41]

so that, in accordance with what was before proved, each of
the rectangles DL, MF is medial.

And they are applied to the rational straight line DE ;

therefore each of the straight lines DM, MG is rational and
incommensurable in length with DE. [x. *a]

And, since the sum of the squares on AC, CB is incom-
mensurable with twice the rectangle AC, CB,

therefore DL is incommensurable with MF.

Therefore DM is also incommensurable with MG ;

[Vi, I, x. 11]

therefore DM, MG are rational straight lines commensurable
in square only ;

therefore DG is binomial. [x. 36]

I say next that it is also a sixth binomial straight line.
Similarly again we can prove that the rectangle DK, KM
is equal to the square on MN,

and that DK is incommensurable in length with KM;

and, for the same reason, the square on DM is greater than
the square on MG by the square on a straight line incom-
mensurable in length with DM.

And neither of the straight lines DM, MG is commen-
surable in length with the rational straight line DE set out.
Therefore DG is a sixth binomial straight line.
\end{proof}

\begin{notes}

To prove that [cf. x. 41]

<r J2 V Vl+<4 a -J* V <Jl + P>

is a sixth binomial straight line.

Denote it by - (u + v)', and put

fT.3Z= 2UV.

Now, by x. 41, «' v> tf, («* + ) is medial, 2«jj is medial, and

(«*+**) i/ 2«tf,

(a) In this case a(x+y) is medial ;

therefore (x +y) is rational and v a (t).

In like manner, 22 is rational and - c (2).

And, since <r (x +y)  <r . 2g,

(x+y) o 23 (3).

Therefore (x +y) + 22 is a binomial straight line.
(jS) As before, x +y > 2z,

xy = z*,

therefore [x. 18] 7(*+>)*- (22)* *, (x + y) (4).

Hence [(1), (2), (3), (4)] (x +y) + 2» is a rt'jfM binomial straight line.

It is obviously - \ JK + -yS — j

\end{notes}

\end{proposition}

\begin{proposition}
\label{propX_66}

\begin{statement}
A straight line commensurable in length with a binomial
straight line is itself also binomial and the same in order.
\end{statement}

\begin{proof}

Let AB be binomial, and let CD be commensurable in
length with AB ;

A 1 B

I say that CD is binomial and the same in order with AB.

For, since AB is binomial,
let it be divided into its terms at B,
and let AB be the greater term ;

therefore AE, EB are rational straight lines commensurable
in square only. [x. 36]

Let it be contrived that,
as AB is to CD, so is AE to CF\ [w. ia]

therefore also the remainder EB is to the remainder FD as
AB is to CD, [v. l9 ]

But AB is commensurable in length with CD;
therefore AE is also commensurable with CF, and EB with
FD. [x. 11]

And Zj, EB are rational ;
therefore CF, FD are also rational.

And, as AE is to CF, so is /TZ? to FD. [v. 11]

Therefore, alternately, as AE is to ,£7?, so is CF to Z 7 ``/?.

[v. 16]

But 4-ZT, ZTZ? are commensurable in square only ;

therefore CF t FD are also commensurable in square only.

[x. .I]
And they are rational ;

therefore CD is binomial. [x. 36]

I say next that it is the same in order with AB.

For the square on AE is greater than the square on EB
either by the square on a straight line commensurable with
AE or by the square on a straight line incommensurable
with it.

If then the square on AE is greater than the square on
EB by the square on a straight line commensurable with AE,
the square on CF will also be greater than the square on FD
by the square on a straight line commensurable with CF.

[x. '4]

And, if AE is commensurable with the rational straight
line set out, CF will also be commensurable with it, [x. 12]

and for this reason each of the straight lines AB, CD is a

first binomial, that is, the same in order. [x. Deff - ``  l ]

But, if EB is commensurable with the rational straight line

set out, FD is also commensurable with it, [x. i*]

and for this reason again CD will be the same in order with
AB,

for each of them will be a second binomial. [x, Deff. n. *]

But, if neither of the straight lines AE, EB is commen-
surable with the rational straight line set out, neither of the
straight lines CF, FD will be commensurable with it, [x. 13]

and each of the straight lines AB, CD is a third binomial,

[x. Deff. 11. 3]
But, if the square on AE is greater than the square on
EB by the square on a straight line incommensurable with
AE,

the square on CF is also greater than the square on FD by
the square on a straight line incommensurable with CF. [x. 14]

And, if AE is commensurable with the rational straight
line set out, CF is also commensurable with it,
and each of the straight lines AB, CD is a fourth binomial.

[x. Deff. 11. 4]

But, if EB is so commensurable, so is FD also,

and each of the straight lines AB, CD will be a fifth binomial.

[x. Deff. 11. 5]
But, if neither of the straight lines AE, EB is so com-
mensurable, neither of the straight lines CF, FD is commen-
surable with the rational straight line set out,

and each of the straight lines AB, CD will be a sixth binomial.

[x. Deff. it, 6]
Hence a straight line commensurable in length with a
binomial straight line is binomial and the same in order.
\end{proof}

\begin{notes}

The proofs of this and the following propositions up to x. 70 inclusive ate
easy and require no elucidation. They are equivalent to saying that, if in each

of the preceding irrational straight lines — p is substituted for p, the resulting

irrational is of the same kind as that from which it is altered.

\end{notes}

\end{proposition}

\begin{proposition}
\label{propX_67}

\begin{statement}
A straight line commensurable in length with a hi me dial
straight line is itself also bimedial and the same in order.
\end{statement}

\begin{proof}

Let AB be bimedial, and let CD be commensurable in
length with AB;

I say that CD is bimedial and the same * ? e

in order with AB. c £p

For, since AB is bimedial,
let it be divided into its mediate at E ;

i4» BOOK X [x. 67

therefore AE, EB are medial straight lines commensurable
in square only. [x. 37, 38]

And let it be contrived that,

as AB is to CD, so is AE to CF;

therefore also the remainder EB is to the remainder FD as
AB is to CD. [v. 19]

But AB is commensurable in length with CD ;

therefore AE, EB are also commensurable with CF t FD

respectively. [x. 11]

But AE, EB are medial ;
therefore CF, FD are also medial. [x. 23]

And since, as A E is to EB, so is CF to FD, [v. 1 r]

and AE, EB are commensurable in square only,
CF, FD are also commensurable in square only. [x. n]

But they were also proved medial ;
therefore CD is bimedial.

I say next that it is also the same in order with AB.
For since, as AE is to EB, so is CF to FD,

therefore also, as the square on AE is to the rectangle AE,
EB, so is the square on CF to the rectangle CF, FD ;

therefore, alternately,

as the square on AE is to the square on CF, so is the rect-
angle AE, EB to the rectangle CF, FD. [v. 16]
But the square on AE is commensurable with the square
on CF;

therefore the rectangle AE, EB is also commensurable with
the rectangle CF, FD,

If therefore the rectangle AE, EB is rational,

the rectangle CF, FD is also rational,

[and for this reason CD is a first bimedial] ; [x. 37]

but if medial, medial, [x. *3> P°r]

and each of the straight lines AB, CD is a second bimedial.

[X.J8]

And for this reason CD will be the same in order with AB.
\end{proof}

\end{proposition}

\begin{proposition}
\label{propX_68}

\begin{statement}
A straight line commensurable with a major straight
line is itself also major.
\end{statement}

\begin{proof}

Let AB be major, and let CD be commensurable with AB;
I say that CD is major.

Let AB be divided at E ;
therefore AE, EB are straight lines incommensur-
able in square which make the sum of the squares
on them rational, but the rectangle contained by
them medial. [x. 39]

Let the same construction be made as before.

Then since, as AB is to CD, so is AE to CF, and EB
to FD,
therefore also, as AE is to CF, so is EB to FD. [v. 11]

But AB is commensurable with CD;
therefore AE, EB are also commensurable with CF, FD
respectively, [x. n]

And since, as AE is to CF, so is EB to FD,
alternately also,

as AE is to EB, so is CF to FD \ [v, 16]

therefore also, componendo,

as AB is to BE, so is CD to DF; [v. 18]

therefore also, as the square on AB is to the square on BE,
so is the square on CD to the square on DF. [vi. 20]

Similarly we can prove that, as the square on AB is to
the square on AE, so also is the square on CD to the square
on CF.

Therefore also, as the square on AB is to the squares on
AE, EB, so is the square on CD to the squares on CF. FD ;
therefore also, alternately,

as the square on AB is to the square on CD, so are the
squares on AE, EB to the squares on CF, FD. [v. 16]

But the square on AB is commensurable with the square
on CD ;

therefore the squares on AE, EB are also commensurable
with the squares on CF, FD.

And the squares on AE, EB together are rational ;
therefore the squares on CF, FD together are rational,

* Similarly also twice the rectangle AE, EB is commen-
surable with twice the rectangle CF, FD.

And twice the rectangle AE, EB is medial ;
therefore twice the rectangle CF, FD is also medial.

[x. *3, Por.]

Therefore CF, FD are straight lines incommensurable in
square which make, at the same time, the sum of the squares
on them rational, but the rectangle contained by them medial;
therefore the whole CD is the irrational straight line called
major. [x. 39]

Therefore a straight line commensurable with the major
straight line is major.
\end{proof}

\end{proposition}

\begin{proposition}
\label{propX_69}

\begin{statement}
A straight line commensurable with the side of a rational
plus a medial area is itself also the side of a rational plus a
medial area.
\end{statement}

\begin{proof}

Let AB be the side of a rational plus a medial area,
and let CD be commensurable with AB;
it is to be proved that CD is also the side of a
rational plus a medial area.

Let AB be divided into its straight lines at E
therefore AE, EB are straight lines incommensur-
able in square which make the sum of the squares
on them medial, but the rectangle contained by them
rational. [x. 40]

Let the same construction be made as before.

We can then prove similarly that
CF, FD are incommensurable in square,
and the sum of the squares on AE, EB is commensurable
with the sum of the squares on CF, FD,
and the rectangle AE, EB with the rectangle CF, FD ;
so that the sum of the squares on CF, FD is also medial, and
the rectangle CF, FD rational.

Therefore CD is the side of a rational plus a medial area.
\end{proof}

\end{proposition}

\begin{proposition}
\label{propX_70}

\begin{statement}
4 straight line commensurable with the side of the sum
of two medial areas is the side of the sum of two medial areas.
\end{statement}

\begin{proof}

Let AB be the side of the sum of two medial areas, and
CD commensurable with AB

it is to be proved that CD is also the side of the
sum of two medial areas.

For, since AB is the side of the sum of two
medial areas,

let it be divided into its straight lines at E ;
therefore AE, EB are straight lines incommensur-
able in square which make the sum of the squares
on them medial, the rectangle contained by them
medial, and furthermore the sum of the squares on AE, EB
incommensurable with the rectangle AE, EB. [x. 41]

Let the same construction be made as before.

We can then prove similarly that
CE, ED are also incommensurable in square,
the sum of the squares on AE, EB is commensurable with
the sum of the squares on CE, FD,
and the rectangle AE, EB with the rectangle CF, FD
so that the sum of the squares on CF, FD is also medial,
the rectangle CF, FD is medial,

and moreover the sum of the squares on CF, FD is incom-
mensurable with the rectangle CF, FD.

Therefore CD is the side of the sum of two medial areas.
\end{proof}

\end{proposition}

\begin{proposition}
\label{propX_71}

\begin{statement}
If a rational and a medial area be added together, four
irrational straight lines arise, namely a binomial or a first
bimedial or a major or a side of a rational plus a medial
area,
\end{statement}

\begin{proof}

Let AB be rational, and CD medial ;
I say that the ``side'' of the area AD is a binomial or a first
bimedial or a major or a side of a rational plus a medial
area.

For IB is either greater or less than CD,

First, let it be greater ;
let a rational straight line EF be set out,
let there be applied to EF the rectangle EG equal to AB,
producing EH as breadth,

and let HI, equal to DC, be applied to EF, producing HK
as breadth.

Then, since AB is rational and is equal to EG,
therefore EG is also rational.

And it has been applied to EF, producing EH as breadth;
therefore EH is rational and commensurable in length with
EF. [x. 20]

Again, since CD is medial and is equal to HI,
therefore HI is also medial.

And it is applied to the rational straight line EF, pro-
ducing HK as breadth ;

therefore HK is rational and incommensurable in length
with EF [x. 22]

And, since CD is medial,
while AB is rational,

therefore AB is incommensurable with CD,
so that EG is also incommensurable with HI.

But, as EG is to HI, so is EH to HK; [vi. 1]

therefore EH is also incommensurable in length with HK.

[x. 11]

And both are rational ;

therefore EH, HK are rational straight lines commensurable

in square only ;

therefore EK is a binomial straight line, divided at H. [x. 36]

And, since AB is greater than CD,
while AB is equal to EG and CD to HI,
therefore EG is also greater than HI ;
therefore EH is also greater than HK.

The square, then, on EH is greater than the square on
HK either by the square on a straight line commensurable
in length with EH or by the square on a straight line in-
commensurable with it.

First, let the square on it be greater by the square on a
straight line commensurable with itself.

Now the greater straight line HE is commensurable in
length with the rational straight line EF set out ;
therefore EK is a first binomial. [x. Deff. 11. t]

But EF is rational ;
and, if an area be contained by a rational straight line and the
first binomial, the side of the square equal to the area is
binomial. [x. 34]

Therefore the ``side'' of EI is binomial ;
so that the ``side'' of AD is also binomial.

Next, let the square on EH be greater than the square
on HK by the square on a straight line incommensurable
with EH.

Now the greater straight line EH is commensurable in
length with the rational straight line EF set out ;
therefore EK is a fourth binomial. [x. Deff. 11. 4]

But EF is rational ;
and, if an area be contained by a rational straight line and the
fourth binomial, the ``side'' of the area is the irrational straight
line called major. [x. 57]

Therefore the ``side'' of the area EI is major ;
so that the ``side'' of the area AD is also major.

Next, let AB be less than CD ;
therefore EG is also less than HI
so that EH is also less than HK,

Now the square on HKis greater than the square on EH
either by the square on a straight line commensurable with
HK or by the square on a straight line incommensurable
with it.

First, let the square on it be greater by the square on a
straight line commensurable in length with itself.

Now the lesser straight line EH is commensurable in
length with the rational straight line EF set out ;
therefore EK is a second binomial. [x. Deff. 11. 2]

But EF is rational ,
and, if an area be contained by a rational straight line and
the second binomial, the side of the square equal to it is a
first bimedial ; x, 55]

therefore the ``side'' of the area EI is a first bimedial,
so that the ``side'' of AD Is also a first bimedial.

Next, let the square on HK be greater than the square
on HE by the square on a straight line incommensurable
with HK.

Now the lesser straight line EH is commensurable with
the rational straight line EF set out ;
therefore EK is a fifth binomial. [x. DeflT. u. 5]

But EF is rational ;
and, if an area be contained by a rational straight line and the
fifth binomial, the side of the square equal to the area is a
side of a rational plus a medial area. [x, 58)

Therefore the ``side'' of the area EI is a side of a rational
plus a medial area,

so that the ``side'' of the area AD is also a side of a rational
plus a medial area.

Therefore etc.
\end{proof}

\begin{notes}

A rationed area being of the form ip\ and a medial area of the form
J\ . p'', the problem is to classify

according to the different possible relations between k, A.
Put <rw = Ap*,

av = JK. p
Then, since the former rectangle is rational, the latter medial,
u is rational and « a,
v is rational and 'v <r.
Also the rectangles are incommensurable ;
so that u v v.

Hence a, v are rational and «-j
whence (u + v) is a binomial straight line.

The possibilities now are as follows :

I. u > v.
Then either
(1) -Jif-tf `` «,

or (z) J if -V  u,
while in both cases w « a-.

In case (1) (a + u) is a jf«/ binomial straight line,
and in case (2) (« + f>) is a fourth binomial straight line.

Thys -Jit(u + v) is either (1) a binomial straight line [x. 54] or (z) a major
irrational straight line [x. 57].

II. p > «.
Then either

(i) -Jv* - u 1 « p,
or (a) Jtf — ti' « i/,
while in both cases »  ir, but a `` o-.

Hence, in case (1), (v + it) is a JAVnif binomial straight line,
and, in case (2), (v + u) is  fifth binomial straight line.

Thus V*t(t' + u) is either (1) a first bimedial straight line [x. 55], or (2) a
side of a rational plus a medial area [x. 58].

\end{notes}

\end{proposition}

\begin{proposition}
\label{propX_72}

\begin{statement}
If two medial areas incommensurable with one another be
added together, the remaining two irrational straight lines
arise, namely either a second bimedial or a side of the sum 0/
two medial areas.
\end{statement}

\begin{proof}

For let two medial areas AB, CD incom mensurable with
one another be added together ;

I say that the ``side'' of the area AD is either a second
bimedial or a side of the sum of two medial areas.

For AB is either greater or less than CD,
First, if it so chance, let AB be greater than CD.
Let the rational straight line EEbe set out,
and to EF let there be applied the rectangle EG equal to

AB and producing EH as breadth, and the rectangle HI
equal to CD and producing HK as breadth.

Now, since each of the areas AB, CD is medial,

therefore each of the areas EG, HI is also medial.

And they are applied to the rational straight line FE,
producing EH, HK as breadth ;

therefore each of the straight lines EH, HK is rational and
incommensurable in length with EF. [x. n]

And, since AB is incommensurable with CD,
and AB is equal to EG, and CD to HI,
therefore EG is also incommensurable with HI.

But, as EG is to HI, so is EH to HK ; [vi. i]

therefore EH fa incommensurable in length with HK [x. it]

Therefore EH, HK are rational straight lines commen-
surable in square only ;
therefore EK is binomial. [x. 3G]

But the square on EH is greater than the square on HK
either by the square on a straight line commensurable with
EH or by the square on a straight line incommensurable
with it.

First, let the square on it be greater by the square on a
straight line commensurable in length with itself.

Now neither of the straight lines EH, II K is commen-
surable in length with the rational straight line EF set out ;
therefore EK is a third binomial. [x. Deff, n, 3]

But EF is rational ;
and, if an area be contained by a rational straight line and the
third binomial, the ``side'' of the area is a second bi medial ;

[x-5*]

therefore the ``side'' of EI, that is, of AD, is a second bimedial.

Next, let the square on EH be greater than the square
on HK by the square on a straight line incommensurable in
length with EH.

Now each of the straight lines EH, HK is incommen-
surable in length with EF;

therefore EK is a sixth binomial. [x. Deff. 11. 6]

But, if an area be contained by a rational straight line and

the sixth binomial, the ``side'' of the area is the side of the
sum of two medial areas; [x. 59]

so that the ``side'' of the area AD is also the side of the
sum of two medial areas.

Therefore etc.
\end{proof}

\begin{notes}

We have to classify, according to the different possible relations between
k, A, the straight line

where *jk . p 1 and ,/A   p 5 are incommensurable.

Suppose that au = Jk . p*,

trv = Jk . p : .

It is immaterial whether Jk.p 1 or ,/A. p- is the greater. Suppose, e.g.,
that the former is.

Now, J 'A . p\ Jk . p' being both media/ areas, and a rational,

u, ware both rational and  tr (1).

Again, by hypothesis, uu  of,

or u u v (2).

Hence [(1), (»)] (u + v) is a binomial straight line.

Next, Ju' - 1? is either commensurable or incommensurable in length
with «.

(a) Suppose v «* - ? rt «.

In this case (« + t>) is a /A«>rf binomial straight line,
and therefore [x, 56]

«/»(» + ») is a second bitntdial straight line.

03) If -Jtfi? ii a,

(« + ?') is a sf'.vrf binomial straight line,

and therefore [x. 59]

J a (k  + v) is a «ife 0/'' /A; jkot fir'' ftfff medial areas.

\end{notes}

The binomial straight line and the irrational straight lines
after it are neither the same with the medial nor with one
another.

For the square on a medial, if applied to a rational straight
line, produces as breadth a straight line rational and incom-
mensurable in length with that to which it is applied. [x. 22]

But the square on the binomial, if applied to a rational
straight line, produces as breadth the first binomial. [x. 60]

The square on the first bimedial, if applied to a rational
straight line, produces as breadth the second binomial, [x. 61]

The square on the second bimedial, if applied to a rational
straight line, produces as breadth the third binomial, [x. 61]

The square on the major, if applied to a rational straight
line, produces as breadth the fourth binomial, [x. 63]

The square on the side of a rational plus a medial area, if
applied to a rational straight line, produces as breadth the fifth
binomial. [x. 64]

The square on the side of the sum of two medial areas, if
applied to a rational straight line, produces as breadth the
sixth binomial. [x. 65]

And the said breadths differ both from the first and from
one another : from the first because it is rational, and from
one another because they are not the same in order ;
so that the irrational straight lines themselves also differ from
one another.

\begin{notes}

The explanation after x. 72 is for the purpose of showing that all the
irrational straight lines treated hitherto are different from one another, viz. the
medial, the six irrational straight lines beginning with the binomial, and the
six consisting of the first, second, third, fourth, fifth and sixth binomials.

\end{notes}

\end{proposition}

\begin{proposition}
\label{propX_73}

\begin{statement}
If from a rational straight line there be subtracted a
rational straight line commensurable with the whole in square
only, the remainder is irrational; and let it be called an
apotome.
\end{statement}

\begin{proof}

For from the rational straight line AB let the rational
straight line BC, commensurable with
the whole in square only, be sub- a c b

traded ;

I say that the remainder AC is the irrational straight line
called apotome.

For, since AB is incommensurable in length with BC,

and, as AB is to BC, so is the square on AB to the rectangle
AB, BC,

therefore the square on AB is incommensurable with the
rectangle AB, BC jx. 11]

But the squares on AB, BC are commensurable with the
square on AB, [x. 15]

and twice the rectangle AB, BC is commensurable with the
rectangle AB, BC. [x. 6]

x. 73. 74] PROPOSITIONS 72—74 159

And, inasmuch as the squares on AB, BC are equal to
twice the rectangle AB, BC together with the square on CA,

[«  7]
therefore the squares on AB, BC are also incommensurable
with the remainder, the square on AC. [x. 13, 16]

But the squares on AB, BC are rational ;
therefore AC is irrational. [x. Def. 4]

And let it be called an \textbf{apotome}.
\end{proof}

\begin{notes}

Euclid now passes to the irrational straight lines which are the difference
and not, as before, the sum of two straight lines. Apotome (``portion cut off'')
accordingly takes the place of binomial and the other terms follow mutatis
mutandis. The first hcxad of propositions (73 to 78) exhibit the six irrational
straight lines which are really the result of extracting the st/uare root of the six
irrationals in the later propositions 85 to 90 (or, strictly speaking, of finding
the sides of squares equal to the rectangles formed by each of those six
irrational straight lines respectively with a rational straight line). Thus, just
as in the corresponding propositions about the irrational straight lines formed
by addition, the further removed irrationals, so to speak, come first.

We shall denote the apotome etc. by (x -y), which is formed by subtracting
a certain lesser straight line y from a greater x. In x. 79 and later propositions
y is called by Euclid the annex (>J B-rxwa/joouo-o), being the straight line which,
when added to the apotome or other irrational formed by subtraction, makes
up the greater x.

The methods of proof are exactly the same as in the preceding propositions
about the irrational straight lines formed by addition.

In this proposition x, y are rational straight lines commensurable in square
only, and we have to prove that (x -y), the apotome, is irrational.

x «<  y , so that x v y :
therefore, since x -.y-x 1 : xy,

x* j xy.

But x 1 « (xr +yr), and xy *> txy ;
therefore x* + y' ~ sxy,

whence (x -y) ! ~ (x' +y*).

But (x? +y) is rational ,
therefore (x ~y)', and consequently (x -y), is irrational.

The apotome (x —y) is of the form p~ s /i . p, just as the binomial straight
line is of the form p+ JA.p,

\end{notes}

\end{proposition}

\begin{proposition}
\label{propX_74}

\begin{statement}
If from a medial straight line there be subtracted a medial
straight line which is commensurable with the whole in square
only, and which contains with the whole a rational rectangle,
the remainder is irrational. And let it be called a first
apotome of a medial straight line.
\end{statement}

\begin{proof}

For from the medial straight line AB let there be sub-
tracted the medial straight line BC

which is commensurable with AB in A c ?

square only and with AB makes the
rectangle AB, BC rational ;

I say that the remainder AC is irrational; and let it be
called a first apotome of a medial straight line.

For, since AB, BC are medial,

the squares on AB, BC are also medial.

But twice the rectangle AB, BC is rational ;

therefore the squares on AB, BC are incommensurable with
twice the rectangle AB, BC ;

therefore twice the rectangle AB, BC is also incommensurable
with the remainder, the square on AC, [cf. h. 7]

since, if the whole is incommensurable with one of the magni-
tudes, the original magnitudes will also be incommensurable.

[x. 16]
But twice the rectangle AB, BC is rational ;

therefore the square on AC is irrational ;

therefore AC is irrational. [x. Def. 4]

And let it be called a first apotome of a medial straight
line.
\end{proof}

\begin{notes}

The first apotome of a medial straight line is the difference between straight

lines of the form Hrp, Irp, which are medial straight lines commensurable in
square only and forming a rational rectangle.

By hypothesis, x 1 , y* are medial areas.

And, since xy is rational, (x 1 +y) v xy

~2xy,
whence (*—yY ** **V'

But 2xy is rational ;
therefore (x — y)', and consequently (x —y), is irrational.

This irrational, which is of the form (A*p ~ k'p), is the first apotome of a
medial straight line ; the term corresponding of course to first bimedial, which
applies where the sign is positive.

\end{notes}

\end{proposition}

\begin{proposition}
\label{propX_75}

\begin{statement}
If from a medial straight line there be subtracted a medial
straight line which is commensurable with the whole in square
only, and which contains with the whole a medial rectangle,
the remainder is irrational ; and let it be called a second
apotomc of a medial straight line.
\end{statement}

\begin{proof}

For from the medial straight line AB let there be sub-
tracted the medial straight line CB which is commensurable
with the whole AB in square only and such that the rectangle
AB, BC, which it contains with the whole AB, is medial; [x. 28]

I say that the remainder AC is irrational ; and let It be called
a second apotome of a medial straight line.

a c

FQ

For let a rational straight line DI be set out,
fet DE equal to the squares on AB, BC be applied to DI,
producing DG as breadth,

and let DH equal to twice the rectangle AB, BC be applied
to DI, producing DF as breadth ;

therefore the remainder FE is equal to the square on AC.

[`` 7]

Now, since the squares on AB, BC are medial and
commensurable,
therefore DE is also medial, [x. 15 and 13, Por.]

And it is applied to the rational straight line DI, producing
DG as breadth ;

therefore DG is rational and incommensurable in length
with DI. [x. **]

Again, since the rectangle AB, BC is medial,

therefore twice the rectangle AB, BC is also medial.

[x. »3, Por.]

And it is equal to DH ;
therefore DH is also medial.

And it has been applied to the rational straight line DI,
producing DF as breadth ;

therefore DF is rational and incommensurable in length
with DI, [x. i*]

And, since AB, BC are commensurable in square only,
therefore AB is incommensurable in length with BC;
therefore the square on AB is also incommensurable with the
rectangle AB, BC. [x. n]

But the squares on AB, BC are commensurable with the
square on AB, [x. 15]

and twice the rectangle AB, BC is commensurable with the
rectangle AB, BC ; [x. 6]

therefore twice the rectangle AB, BC is incommensurable with
the squares on AB, BC. |x- 13]

But DE is equal to the squares on AB, BC,
and DH to twice the rectangle AB, BC;
therefore DE is incommensurable with DH.

But, as DE is to DH, so is GD to DF; [vi. i]

therefore GD is incommensurable with DF. [x. 11]

And both are rational ;
therefore GD, DFare rational straight lines commensurable
in square only ;
therefore FG is an apotome. [x. 73]

But DI is rational,
and the rectangle contained by a rational and an irrational
Straight line is irrational, [deduction from x, 20]

and its ``side'' is irrational.

And AC is the ``side'' of FE ;
therefore AC is irrational.

And let it be called a second apotome of a medial
straight line.
\end{proof}

\begin{notes}

We have here the difference between p, J.pj, two medial straight
lines commensurable in square only and containing a medial rectangle.

Apply each of the areas (x'+j?), zxy to a rational straight line tr, i.e.
suppose that

x , +y = <ru,

2xy = av.

Then <rw, <rt> are medial areas,
so that u, v are both rational and « <r ,(1).

Again, x v y ;
therefore «* « xy,
and consequently x* +>'' ~ a*y,
or rav <ro,
and awo (i).

Thus ((1), (*)] », vare rational and «-;
therefore [x. 73] (« - v) is an apotome,
and, (« — 1>) being thus irrational,

(u — v) cr is an irrational area.

Hence (x -y) 1 , and consequently ( -y), is irrational.

The irrational straight line A*p ~ ** --  - is called a second apotome of a
medial straight line.

\end{notes}

\end{proposition}

\begin{proposition}
\label{propX_76}

\begin{statement}
If from, a straight line there be subtracted a straight line
which is incommensurable in square with the whole and which
with the whole makes the squares on them added together
rational, but the rectangle contained by them medial, the
remainder is irrational; and let it be called minor.
\end{statement}

\begin{proof}

For from the straight line AB let there be subtracted the
straight line BC which is incom-
mensurable in square with the whole a 4~ b
and fulfils the given conditions. [x. 33]

I say that the remainder A C is the irrational straight line
called minor.

For, since the sum of the squares on AB, BC is rational,
while twice the rectangle AB, BC is medial,
therefore the squares on AB, BC are incommensurable with
twice the rectangle AB, BC;

and, convertendo, the squares on AB, i?Care incommensurable
with the remainder, the square on AC [11. 7, x. 16]

But the squares on AB, BC are rational ;
therefore the square on A C is irrational ;
therefore AC is irrational.

And let it be called minor.

x, y are here of the form found in x. 33, viz.

P / T p / J~

J*V t I + JTTF TV 1 JTTe'

By hypothesis (a 3 + y*) is a rational, xy a medial, area.
Therefore (jp*+y)  2Ay,

whence (a -ji)'  (a 3 +1-

Therefore (a- -y)', and consequently (a — /), is irrational
The minor (irrational) straight line is thus of the form

Observe the use of tonvtrttndo (acnirTptyam) for the inference that, since
(a* +y) « 2aji, (a* +>> ) u (a -v) s . The use of the word corresponds exactly
to its use in proportions.
\end{proof}

\end{proposition}

\begin{proposition}
\label{propX_77}

\begin{statement}
If from a straight line there be subtracted a straight line
which is incommensurable in square with the whole, and which
with the whole makes the sum of the squares on them medial,
but twice the rectangle contained by them rational, the remainder
is irrational: and let it be called that which produces with
a rational area a medial whole.
\end{statement}

\begin{proof}

For from the straight line AB let there be subtracted the
straight line BC which is incommensurable in square
with AB and fulfils the given conditions ; [x. 34]

I say that the remainder AC is the irrational straight
line aforesaid.

For, since the sum of the squares on AB, BC is
medial,

while twice the rectangle AB, BC is rational,

therefore the squares on AB, BC are incommensurable
with twice the rectangle AB, BC ;

therefore the remainder also, the square on AC, is incom-
mensurable with twice the rectangle AB, BC. [11. 7, x. 16]

And twice the rectangle AB, BC is rational j
therefore the square on AC is irrational ;
therefore AC is irrational.

And let it be called that which produces with a
rational area a medial whole.
\end{proof}

\begin{notes}

Here x, y are of the form [cf. x. 34]

JjjTP + i,

 J2(t+P) -Ja(i+JP)

By hypothesis, (x 1 +y*) is a medial, xy a rational, area ;
thus (+y) m 2xc,

and therefore ( *->)* m My,

whence (x -yf, and consequently (x -y), is irrational.
The irrational straight line

is called thai which products with a rational area a medial whole or more
literally that which with a rational area makes the whole medial (jJ ft«ri pijicX
p. wav to ov troiovwn). Here `` produces `` means `` produces when a square
is described on it'' A clearer way of expressing the meaning would be to call
this straight line the `` side `` of a medial minus a rational area corresponding
to the ``side'' of a rational plus a medial area [x. 40].

\end{notes}

\end{proposition}

\begin{proposition}
\label{propX_78}

\begin{statement}
If from a straight line there be subtracted a straight line
 which is incommensurable in square with the whole and which
Tvith the whole makes the sum of the squares on them medial,
twice the rectangle contained by them medial, and further the
squares on them incommensurable with twice the rectangle
contained by them, the remainder is irrational ; and let it be
called that which produces with a medial area a
medial whole.
\end{statement}

\begin{proof}

For from the straight line AB let there be subtracted the
straight line BC incommensurable in

square with AB and fulfilling the p f q

given conditions ; [x. 35]

I say that the remainder AC is the
irrational straight line called that
which produces with a medial
area a medial whole.

For let a rational straight line DI * c a

be set out,

to DI let there be applied DE equal to the squares on AB,
BC, producing DG as breadth,

and let DH equal to twice the rectangle AB, BC be
subtracted.

Therefore the remainder FE is equal to the square
on AC, [u. 7]

so that AC is the `` side `` of FE.

Now, since the sum of the squares on AB, BC is medial
and is equal to DE,

therefore DE is medial.

And it is applied to the rational straight line DI, producing
DG as breadth ;

therefore DG is rational and incommensurable in length
with DI. [x. *»]

Again, since twice the rectangle AB, BC is medial and is
equal to DH,

therefore DH is medial.

And it is applied to the rational straight line DI, producing
DF as breadth ;

therefore DF is also rational and incommensurable in length
with DI. [x. 12]

And, since the squares on AB, BC are incommensurable
with twice the rectangle AB, BC,

therefore DE is also incommensurable with DH.

But, as DE is to DH, so also is DG to DF; [vi, 1]

therefore DG is incommensurable with DF. [x. n]

And both are rational ;

therefore GD, DF are rational straight lines commensurable
in square only.

Therefore FG is an apotome. [x. 73]

And FH is rational ;

but the rectangle contained by a rational straight line and an
apotome is irrational, [deduction from x. 10]

and its ``side'' is irrational.

And AC is the ``side'' of FE ;

therefore AC is irrational.

And let it be called that which produces with a
medial area a medial whole.
\end{proof}

\begin{notes}

In this case x, y have respectively the forms [cf. x. 35]
pX* / ~k~~ p* / k

Suppose that x* i-y* = <ru,

txy = ov.

By hypothesis, the areas <ru, av are medial ;
therefore w, v are both rational and ». <r (i).

Further au v <rv,
so that u ~j v , (*).

Hence [(1), (2)] u, v are rational and **- ,
so that (« -») is the irrational straight line called apotome [x. 73].

Thus <r (« - u) is an irrational area,
so that (x-yf, and consequently (.r-), is irrational.

The irrational straight line

pA*

is called that which products [i.e. when a square is described on it] with a
medial area a media/ whole, more literally that which with a medial area makes
the whole medial (15 /i<ri p-iaov fUvov to Skov jtoiouo-o). A clearer phrase (to
us) would be the ``side'' of the difference between two medial areas, correspond-
ing to the `` side'' of (the sum of) two mtdial areas [x, 4t],

\end{notes}

\end{proposition}

\begin{proposition}
\label{propX_79}

\begin{statement}
To an apotome only one rational straight line can be
annexed which is commensurable with the whole in square only.
\end{statement}

\begin{proof}

Let AB be an apotome, and BC an annex to it ;
therefore AC, CB are rational

straight lines commensurable in B c

square only. [x. 73] — '

I say that no other rational

straight line can be annexed to AB which is commensurable
with the whole in square only.

For, it'' possible, let BD be so annexed ;
therefore AD, DB are also rational straight lines commen-
surable in square only. [*  73]

Now, since the excess of the squares on AD, DB over
twice the rectangle AD, DB is also the excess of the squares
on AC, CB over twice the rectangle AC, CB,
for both exceed by the same, the square on AB, [11. 7]

therefore, alternately, the excess of the squares on AD, DB
over the squares on AC, CB is the excess of twice the rect-
angle AD, DB over twice the rectangle AC, CB.

But the squares on AD, DB exceed the squares on AC,
CB by a rational area,
for both are rational ;

therefore twice the rectangle AD, DB also exceeds twice the
rectangle A C, CB by a rational area :
which is impossible,

for both are medial [x. 21], and a medial area does not exceed
a medial by a rational area. [x. 26]

Therefore no other rational straight line can be annexed
to AB which is commensurable with the whole in square only.

Therefore only one rational straight line can be annexed
to an apotome which is commensurable with the whole in
square only.
\end{proof}

\begin{notes}

This proposition proves the equivalent of the well-known theorem of surds
that,

if a — Jb = x — ,Jy, then a - x, t=y;
and, if Ja- Je = Jx~ Jy, then a = x, b =y.

The method of proof corresponds to that of X. 42 for positive signs.

Suppose, if possible, that an apotome can be expressed as (x -y) and also
as (*'— y'), where x, y are rational straight lines commensurable in square only,
and *', / are so also.

Of x, x\ let x be the greater.

Now, since x—y = x —y'

3? +y - (x'' +/*) = ixy - zx'y.

But (2? +y), (x* +y*) are both rational, so that their difference is a
rational area.

On the other hand, nxy, xx'y' are both medial areas, being of the form

therefore the difference between two medial areas is rational :
which is impossible [x. 26].
Therefore etc

\end{notes}

\end{proposition}

\begin{proposition}
\label{propX_80}

\begin{statement}
To a first apotome of a medial straight line only one
medial straight line can be annexed which is commensurable
vnth the whole in square only and which contains with the
whole a rational rectangle.
\end{statement}

\begin{proof}

For let AB be a first apotome of a medial straight line,
and let BC be an annex to AB ;

therefore AC, CB are medial ? £— -

straight lines commensurable in

square only and such that the rectangle AC, CB which they
contain is rational ; [x. 74]

I say that no other medial straight line can be annexed to
AB which is commensurable with the whole in square only
and which contains with the whole a rational area.

For, if possible, let DB also be so annexed ;
therefore AD, DB are medial straight lines commensurable
in square only and such that the rectangle AD, DB which
they contain is rational. [x. 74]

Now, since the excess of the squares on AD, DB over
twice the rectangle AD, DB is also the excess of the squares
on AC, CB over twice the rectangle AC, CB,

for they exceed by the same, the square on AB, [n. 7]

therefore, alternately, the excess of the squares on AD, DB
over the squares on AC, CB is also the excess of twice the
rectangle AD, DB over twice the rectangle AC, CB.

But twice the rectangle AD, DB exceeds twice the rect-
angle AC, CB by a rational area,
for both are rational.

Therefore the squares on AD, DB also exceed the squares
on A C, CB by a rational area :

which is impossible,

for both are medial [x. 15 and 23, Por.], and a medial area does

not exceed a medial by a rational area. [x. 26]

Therefore etc.
\end{proof}

\begin{notes}

Suppose, if possible, that the same first apotome of a medial straight line
can be expressed in terms of the required character in two ways, so that

x-y = x'-y
and suppose that x > x'.

In this case x* +y', (x 11 +/*) are both media/ areas, and txy, ix'y 1 are both
rational areas ;
and x I +y t - (x'' 1 +/') = ixy - ix'y'.

Hence X. z6 is contradicted again ;
therefore etc.

\end{notes}

\end{proposition}

\begin{proposition}
\label{propX_81}

\begin{statement}
C D
\end{statement}

\begin{proof}

To a second apotome of a medial straight line only one
medial straight line can be annexed which is commensurable
with the whole in square only and which contains with the
whole a medial rectangle.

Let AB be a second apotome of a medial straight line
and BC an annex to AB ;
therefore AC, CB are medial straight
lines commensurable in square only and
such that the rectangle AC, CB which
they contain is medial. fi. 7s]

I say that no other medial straight line
can be annexed to AB which is commen-
surable with the whole In square only and
which contains with the whole a medial
rectangle.

For, if possible, let BD also be so
annexed ;

therefore AD, DB are also medial straight
lines commensurable in square only and
such that the rectangle AD, DB which
they contain is medial. [x, 75]

Let a rational straight line EF be set out,

let EG equal to the squares on AC, CB be applied to EF,
producing EM as breadth,

and let HG equal to twice the rectangle AC, CB be sub-
tracted, producing HM as breadth ;

therefore the remainder EL is equal to the square on AB,

[» 7]
so that AB is the ``side'' of EL.

Again, let EF equal to the squares on AD, DB be applied
to EF, producing EN as breadth.

But EL is also equal to the square on AB ;

therefore the remainder HI is equal to twice the rectangle
AD, DB. [11. 7]

Now, since AC, CB are medial straight lines,
therefore the squares on AC, CB are also medial.

And they are equal to EG ;

therefore EG is also medial. [x. 15 and 23, Por.]

And it is applied to the rational straight line EF, producing
EM as breadth ;

therefore EM is rational and incommensurable in length
with EF, [x. 23]

Again, since the rectangle AC, CB is medial,
twice the rectangle AC, CB is also medial. [x. *3, Por,]

And it is equal to HG ;

therefore HG is also medial.

And it is applied to the rational straight line EF, producing
HM as breadth ;

therefore HM is also rational and incommensurable in length
with EF. [x. j*]

And, since AC, CB are commensurable in square only,

therefore AC is incommensurable in length with CB.

But, as A C is to CB, so is the square on AC to the rect-
angle AC, CB;

therefore the square on AC is incommensurable with the
rectangle AC, CB. [x. 11]

But the squares on AC, CB are commensurable with the
square on AC,

while twice the rectangle AC, CB is commensurable with the
rectangle AC, CB ; [x. 6]

therefore the squares on AC, CB are incommensurable with
twice the rectangle AC, CB. [x. 13]

And EG is equal to the squares on AC, CB,
while GH is equal to twice the rectangle AC, CB ;
therefore EG is incommensurable with HG.

But, as EG Is to HG, so is EM to HM; [vi. 1]

therefore EM is incommensurable in length with MH. [x. it]

And both are rational ;
therefore EM, MH are rational straight Hnes commensurable
in square only ;
therefore EH is an apotome, and HM an annex to it [x. 73J

Similarly we can prove that HN is also an annex to it ;
therefore to an apotome different straight lines are annexed
which are commensurable with the wholes in square only :
which is impossible. [x. 79]

Therefore etc.
\end{proof}

\begin{notes}

As the irrationality of the setond apotome of a medial straight line was
deduced [x. 75] from the irrationality of an apotome, so the present theorem
is reduced to x. 79.

Suppose, if possible, that (x -y), (x' -y') are the same second apotome of
a medial straight line ;
and let (say) x be greater than x'.

Apply (Jf'+y), zxy and also (x* +y'*), 2x'y to a rational straight line a,
i.e. put

-''l and r™',].
txy = <rv ) 2xy = av )

Dealing with (x -y) first, we have :
(-v- +y'') is a medial area, and ixy is also a medial area.

Therefore u, v are both rational and  <r (1),

Also, since x «-y, x «y,
so that x 1  xy,
whence, as usual, x* +y' w txy,
that is, «u „ av,
and therefore u v v (s).

Thus [(1) and (2)] u, v are rational and «-,
so that (it — v) is ah apotome.

Similarly (u - if) is proved to be the same apotome.

Hence this apotome is formed in two ways :
which contradicts X. 79.

Therefore the original hypothesis is false, and a sttond apotome of a
media/ straight line is uniquely formed.

\end{notes}

\end{proposition}

\begin{proposition}
\label{propX_82}

\begin{statement}
To a minor straight line only one straight line can be
annexed which is incommensurable in square with the whole
and which makes, with the whole, the sum 0/ the squares on
them rational but twice the rectangle contained by them medial.
\end{statement}

\begin{proof}

Let AB be the minor straight line, and let BC be an
annex to AB ;

therefore AC, CB are straight *   ??

lines incommensurable in square
which make the sum of the squares on them rational, but
twice the rectangle contained by them medial. [x. 76]

I say that no other straight line can be annexed to AB
fulfilling the same conditions.

For, if possible, let BD be so annexed ;
therefore AD, DB are also straight lines incommensurable
in square which fulfil the aforesaid conditions. [x. 76]

Now, since the excess of the squares on AD, DB over
the squares on AC, CB is also the excess of twice the rect-
angle AD, DB over twice the rectangle AC, CB,
while the squares on AD, DB exceed the squares on AC,
CB by a rational area,
for both are rational,

therefore twice the rectangle AD, DB also exceeds twice
the rectangle AC, CB by a rational area :
which is impossible, for both are medial. [x. 36]

Therefore to a minor straight line only one straight
line can be annexed which is incommensurable in square with
the whole and which makes the squares on them added
together rational, but twice the rectangle contained by them
medial.
\end{proof}

\begin{notes}

Suppose, if possible, that, with the usual notation,
x-y = x~y ;
and let * (say) be greater than x.

In this case (jc* +y>), (*-'* +y*) are both rational areas,
and 2at, 2x y are both medial areas.

But, as before, (x*+y*) — (*'' +>'*) = xxy - zx'y',
so that the difference between two medial areas is rational :
which is impossible [x. 26].

Therefore etc.

\end{notes}

\end{proposition}

\begin{proposition}
\label{propX_83}

\begin{statement}
To a straight line which produces with a rational area a
medial whole only one straight line can be annexed which is
incommensurable in square with the whole straight line and
which with the whole straight line makes the sum 0/ the squares
on them medial, but twice the rectangle contained by them
rational.
\end{statement}

\begin{proof}

Let AB be the straight line which produces with a rational
area a medial whole,

and let BC be an annex to AB ; A P S—°

therefore AC, CB are straight lines

incommensurable in square which fulfil the given conditions.

I>- 77]

I say that no other straight line can be annexed to AB
which fulfils the same conditions.

For, if possible, let BD be so annexed ;
therefore AD, DB are also straight lines incommensurable in
square which fulfil the given conditions. [x. 77]

Since then, as in the preceding cases,
the excess of the squares on AD, DB over the squares on
AC, CB is also the excess of twice the rectangle AD, DB
over twice the rectangle AC, CB,

while twice the rectangle AD, DB exceeds twice the rectangle
AC, CB by a rational area,
for both are rational,

therefore the squares on AD, DB also exceed the squares
on AC, CB by a rational area :
which is impossible, for both are medial. [x. 26]

Therefore no other straight line can be annexed to AB
which is incommensurable in square with the whole and which
with the whole fulfils the aforesaid conditions;
therefore only one straight line can be so annexed.
\end{proof}

\begin{notes}

Suppose, with the same notation, that

x —y = * — y . (x > x' )

Here, (x'+y 1 ), (x+y'') being both medial areas, and 2xy, ix'y' both
rational areas,

while (x* +y*) — (i* 1 +y*)   zxy - ix'y,

x. 26 is contradicted again.

Therefore etc.

\end{notes}

\end{proposition}

\begin{proposition}
\label{propX_84}

\begin{statement}
To a straight line which produces with a medial area a
medial whole only one straight line can be annexed which is
incommensurable in square with the whole straight line and
which with the whole straight line makes the sum of the squares
on them medial and twice the rectangle contained by them both
medial and also incommensurable with the sum of the squares
on them.
\end{statement}

\begin{proof}

Let AB be the straight line which produces with a medial
area a medial, whole,
and BC an annex to it ;

therefore AC, t''Z?are straight lines incommensurable in square
which fulfil the aforesaid conditions. [x. 7 8 ]

c D

E H

F L

I say that no other straight line can be annexed to AB
which fulfils the aforesaid conditions.

For, if possible, let BD be so annexed,

so that AD, DB are also straight lines incommensurable in
square which make the squares on AD, DB added together
medial, twice the rectangle AD, DB medial, and also the
squares on AD, DB incommensurable with twice the rectangle
AD, DB, [x. 7 «]

Let a rational straight line EF be set out,

let EG equal to the squares on AC, CB be applied to EF,
producing EM as breadth,

and let HG equal to twice the rectangle AC, CB be applied
to EF, producing HM as breadth ;

therefore the remainder, the square on AB [11. 7], is equal

to EL ;

therefore AB is the ``side'' of EL.

Again, let EI equal to the squares on AD, DB be applied
to EF, producing EN as breadth.

But the square on AB is also equal to EL ;

therefore the remainder, twice the rectangle AD, DB [n. 7],
is equal to HI.

Now, since the sum of the squares on AC, CB is medial
and is equal to EG,
therefore EG is also medial.

And it is applied to the rational straight line EF, pro-
ducing EM as breadth ;

therefore EM is rational and incommensurable in length
with EF. [x. it]

Again, since twice the rectangle AC, CB is medial and is
equal to HG,

therefore HG is also medial.

And it is applied to the rational straight line EF, pro-
ducing HM as breadth ;

therefore HM is rational and incommensurable in length
with EF. [x. as]

And, since the squares on AC, CB are incommensurable
with twice the rectangle AC, CB,

EG is also incommensurable with HG ;

therefore EM is also incommensurable in length with MH.

[vi. i, x. 1 1]
And both are rational ;

therefore EM, MH are rational straight lines commensurable
in square only ;

therefore EH is an apotome, and HM an annex to it. [x. 73]

Similarly we can prove that EH is again an apotome and
HN an annex to it.

Therefore to an apotome different rational straight lines
are annexed which are commensurable with the wholes in
square only :

which was proved impossible. [x, 79J

Therefore no other straight line can be so annexed to AB.

Therefore to AB only one straight line can be annexed
which is incommensurable in square with the whole and which
with the whole makes the squares on them added together
medial, twice the rectangle contained by them medial, and
also the squares on them incommensurable with twice the
rectangle contained by them.
\end{proof}

\begin{notes}

With the usual notation, suppose that

x-y = x''-?. ( *>*')

Let x'+yem ] , x'' , +y* = <ru' 1

) and ``J , , ) .

Consider (x-y) first;
it follows, since (x* +y), 2xy are both medial areas, that
u, v are both rational and  j <r (1).

But x* +y >j axy,
that is, au  <ni,
and therefore M « v (a).

Therefore [(1) and (2)] v, v are rational and «- ;
hence (u-v) is an apotome.

Similarly (*' - r/) is proved to be the j»»w apotome.

Thus the same apotome is formed as such in two ways :
which is impossible [x. 79],

Therefore, etc.

\end{notes}

\end{proposition}

\chapter*{Definitions III}

\begin{enumerate}

\item Given: a rational straight line and an apotome, if the square on
  the whole be greater than the square on the annex by the square on a
  straight line commensurable in length with the whole, and the whole
  be commensurable in length with the rational straight line set out,
  let the apotome be called a first apotome.

\item But if the annex be commensurable in length with
the rational straight line set out, and the square on the whole
be greater than that on the annex by the square on a straight
line commensurable with the whole, let the apotome be called
a second apotome.

\item But if neither be commensurable in length with the
rational straight line set out, and the square on the whole be
greater than the square on the annex by the square on a
straight line commensurable with the whole, let the apotome
be called a third apotome,

\item Again, if the square on the whole be greater than
the square on the annex by the square on a straight line
incommensurable with the whole, then, if the whole be com-
mensurable in length with the rational straight line set out,
let the apotome be called a fourth apotome ;

\item if the annex be so commensurable, a fifth ;

\item and, if neither, a sixth.

\end{enumerate}

\begin{proposition}
\label{propX_85}

\begin{statement}
To find the first apotome.
\end{statement}

\begin{proof}

Let a rational straight line A be set out,
and let BG be commensurable in length with A ;
therefore BG is also rational.

A-
H-

E F D

Let two square numbers DE, EF be set out, and let their
difference FD not be square ;

therefore neither has ED to DF the ratio which a square
number has to a square number.

Let it be contrived that,

as ED is to DF, so is the square on BG to the square on GC;

[x. 6, Por.]
therefore the square on BG is commensurable with the square
on GC. [x. 6]

But the square on BG is rational ;
therefore the square on GC is also rational ;
therefore GC is also rational.

And, since ED has not to DF the ratio which a square
number has to a square number,

therefore neither has the square on BG to the square on GC
the ratio which a square number has to a square number ;
therefore BG is incommensurable in length with GC. [x. 9]

And both are rational ;
therefore BG, GC are rational straight lines commensurable
in square only ;
therefore BC is an apotome, [x. 73]

I say next that it is also a first apotome.

For let the square on H be that by which the square on
BG is greater than the square on GC.

Now since, as ED is to FD, so is the square on BG to
the square on GC,

therefore also, convertendo, [v. 19, Por.]

as DE is to EF, so is the square on GB to the square on H.

But DE has to EF the ratio which a square number has
to a square number,

for each is square ;

therefore the square on GB also has to the square on H the
ratio which a square number has to a square number ;

therefore BG is commensurable in length with H. [x. 9]

And the square on BG is greater than the square on GC
by the square on H

therefore the square on BG is greater than the square on GC
by the square on a straight Line commensurable in length
with BG.

And the whole BG is commensurable in length with the
rational straight line A set out.

Therefore BC is a first apotome. [x. Deff. in. 1]

Therefore the first apotome BC has been found.

(Being) that which it was required to find.
\end{proof}

\begin{notes}

Take kp commensurable in length with p, the given rational straight line.

Let m\ «' be square numbers such that (« , -« 1 ) is not square.

Take * such that m* : (n* -*) = ? : x 1 (1),

so that x = kp — —

= kpji-H, say.
Then shall ip-x, or kp -kp -/i-X 1 , be a. first apotome.
For (a) it follows rrom (i) that x is rational but incommensurable with kp,
whence kp, x are rational and <*-»
so that (kp—x) is an apotome.
(ft) lt/ = *Y - x*, then, by ( 1), toMwrtenJo,

m'-.n'Pp'.y,
whence y, that is, <Jk*p* - .**, is commensurable in length with kp.

And kp `` p ;
therefore kp-xh  first apotome.

As explained in the note to x. 48, the first apotome
kp — kp>Ji -A'
is one of the roots of the equation

x t -iAp.x + k 1  t ( r = o.

\end{notes}

\end{proposition}

\begin{proposition}
\label{propX_86}

\begin{statement}
To find the second apotome.
\end{statement}

\begin{proof}

Let a rational straight line A be set out, and GC com-
mensurable in length with A ;
therefore GC is rational.

Let two square numbers DE, ? — £ —   — ?

EF be set out, and let their h

difference DF not be square.

Now let it be contrived that, g f b

as FD is to DE, so is the square

on CG to the square on GB. [x. 6, Por.]

Therefore the square on CG is commensurable with the
square on GB. [x. 6]

But the square on CG is rational ;
therefore the square on GB is also rational ;
therefore BG is rational.

And, since the square on GC has not to the square on GB
the ratio which a square number has to a square number,
CG is incommensurable in length with GB, [x, 9]

And both are rational j
therefore CG, GB are rational straight lines commensurable
in square only ;
therefore EC is an apotome. [x. 73]

I say next that it is also a second apotome.

For let the square on H be that by which the square on
BG is greater than the square on GC.

Since then, as the square on BG is to the square on GC,
so is the number ED to the number DF,
therefore, convertendo,

as the square on BG is to the square on H, so is DE to EF.

[v. 19, Por.]

And each of the numbers DE, EF is square ;
therefore the square on BG has to the square on If the ratio
which a square number has to a square number ;
therefore BG is commensurable in length with H. [x. g]

And the square on BG is greater than the square on GC
by the square on H ;
therefore the square on BG is greater than the square on GC

by the square on a straight line commensurable in length
with BG.

And CG, the annex, is commensurable with the rational
straight line A set out.

Therefore BC is a second apotome. [x. Deff. 111. 2]

Therefore the second apotome BC has been found.
\end{proof}

\begin{notes}

Take, as before, kp commensurable in length with p.
Let m\ n 1 be again square numbers, but («' - « s ) not square.
Take * such that («*-«) ;*** = jy :.* (1),

whence x = kp ,

V *>' — »'
kp

Thus x is zrcater than kp.
kp

Then x-ip, or . — kp, is a second apotome,

Vi-X*

For (a), as before, x is rational and «- V-
fj9) If «'' - p'   y, we have, from ( 1 ),

m* :n*~x' :.
Thus,?, or Jxt-Pp*, is commensurable in length with x.
And kp is f> p.

Therefore x — kp is a second apotome.
As explained in the note on x. 49, the second apotome

is the lesser root of the equation

\end{notes}

\end{proposition}

\begin{proposition}
\label{propX_87}

\begin{statement}
To find the third apotome.
\end{statement}

\begin{proof}

Let a rational straight line A be set out,
let three numbers E, BC, CD be

set out which have not to one

another the ratio which a square ? - tj .

number has to a square number,

but let CB have to BD the ratio

which a square number has to a e

square number. .

Let it be contrived that, as E B D

is to BC, so is the square on A to the square on FG,

and, as BC is to CD, so is the square on FG to the square
on GH. [* 6, Por.]

Since then, as E is to BC, so is the square on A to the
square on FG,

therefore the square on A is commensurable with the square
on FG. [* 6]

But the square on A is rational ;
therefore the square on FG is also rational ;
therefore FG is rational.

And, since E has not to BC the ratio which a square
number has to a square number,

therefore neither has the square on A to the square on FG
the ratio which a square number has to a square number ;
therefore A is incommensurable in length with FG, [x. 9]

Again, since, as BC is to CD, so is the square on FG to
the square on GH,

therefore the square on FG is commensurable with the square
on GH. [x. 6]

But the square on FG is rational ;
therefore the square on GH is also rational ;
therefore GH is rational.

And, since BC has not to CD the ratio which a square
number has to a square number,

therefore neither has the square on FG to the square on GH
the ratio which a square number has to a square number ;
therefore FG is incommensurable in length with GH. [x. 9]

And both are rational ;
therefore FG, GH are rational straight lines commensurable
in square only ;

therefore FH is an apotome. [x. 73]

I say next that it is also a third apotome.
For since, as E is to BC, so is the square on A to the
square on FG,

and, as BC is to CD, so is the square on FG to the square
on HG,

therefore, ex aequali, as E is to CD, so is the square on A
to the square on HG. [v,

But E has not to CD the ratio which a square number
has to a square number ;

therefore neither has the square on A to the square on GH
the ratio which a square number has to a square number ;

therefore A is incommensurable in length with GH. [x. 9]

Therefore neither of the straight lines FG, GH is
commensurable in length with the rational straight line A
set out.

Now let the square on AT be that by which the square on
FG is greater than the square on GH.

Since then, as BC is to CD, so is the square on FG to
the square on GH,

therefore, convertendo, as BC is to BD, so is the square on
FG to the square on K. [v. 19, Por.]

But BC has to BD the ratio which a square number has
to a square number ;

therefore the square on FG also has to the square on K the
ratio which a square number has to a square number.

Therefore FG is commensurable in length with K, [x. 9]

and the square on FG is greater than the square on GH by
the square on a straight line commensurable with FG.

And neither of the straight lines FG, GH is commen-
surable in length with the rational straight line A set out ;
therefore FH is a third apotome. [* Deff. m. 3]

Therefore the third apotome FH has been found.
\end{proof}

\begin{notes}

Let p be a rational straight line.

Take numbers p, qm'', a (m* - *») which hare not to one another the ratio
of square to square.

Now let x, y be such that

P:qm' = ?:x* (i)

and ?«' :q (w , -» , ) =  , -.y (a).

Then shall (* —y\ be a tkird apotome.
For (a), from (i),

* is rational but v p (3),

And, from (z),y is rational but « x.
Therefore x, y are rational and -,
so that (x -y) is an apotome.

(0) By (1), (j), ex aequali,

pq(m*-) = p*:f,
whence y w p.

Thus, by this and (3), x, y are both vp (4)-

Lastly, let ** = ** —y*, so that, from (1), eomxrteitdo,
gm* : qn' = x* : **;

therefore *, or *Jx*-y*, n x (5).

Thus [(4) and (5)] (x-y) is a third apotome.
To find its form, we have, from (1) and (2),

J''p- — -jp>

so that *'' = /l'*'' , '`` |l " ,

This may be written in the form

mjk.p- mjk . pJi -X*.
As explained in the note on x. 50, this is the lesser root of the equation
x*— 2MjJk.px + k t m i ibp t = 0.

\end{notes}

\end{proposition}

\begin{proposition}
\label{propX_88}

\begin{statement}
To find the fourth apotome.
\end{statement}

\begin{proof}

Let a rational straight line A be set out, and BG com-
mensurable in length with it ;
therefore BG is also rational.

A ?  ?

H-

Let two numbers DF, FE be set out such that the whole
DE has not to either of the numbers DF, EF the ratio
which a square number has to a square number.

Let it be contrived that, as DE is to EF, so is the square
on BG to the square on GC ; [x. 6, Pot.]

therefore the square on BG is commensurable with the square
on GC. [x, 6]

But the square on BG is rational ;
therefore the square on GC is also rational ;
therefore GC is rational.

Now, since DE has not to EF the ratio which 3 square
number has to a square number,

therefore neither has the square on BG to the square on GC
the ratio which a square number has to a square number ;
therefore BG is incommensurable in length with GC. [x. 9]

And both are rational ;
therefore BG, GC are rational straight lines commensurable
in square only ;
therefore BC is an apotome. [x. 73]

Now let the square on H be that by which the square on
BG is greater than the square on GC.

Since then, as DE is to EF, so is the square on BG to
the square on GC,

therefore also, cmvertmdo, as ED is to DF, so is the square
on GB to the square on H. [v. 19, Por.]

But ED has not to DF the ratio which a square number
has to a square number ;

therefore neither has the square on GB to the square on H
the ratio which a square number has to a square number ;

therefore BG is incommensurable in length with H. [x. 9]

And the square on BG is greater than the square on GC
by the square on H;

therefore the square on BG is greater than the square on GC
by the square on a straight line incommensurable with BG.

And the whole BG is commensurable in length with the
rational straight line A set out

Therefore BC is a fourth apotome. [x. Deff, in. 4]

Therefore the fourth apotome has been found.
\end{proof}

\begin{notes}

Beginning with p, k P , as in x. 85, 86, we take numbers m, n such that
(m + n) has not to either of the numbers m, n the ratio of a square number to
a square number.

Take x such that (m + *) : n = ¥p' : x* ...(1),

whence * = ip* /

Then shall (tp-x), or (ip r 4— T ) , be  fourth aPctome.

\ Vi + A/

For, by (1), x is rational and *» kpt.
Also *Jf?~x* is incommensurable with kp, since
(» + ») :« = £y :(*y-*»),

and the ratio (« + *): m is not that of a square number to a square number.
And kp p.
As explained in the note on x, 51, ihtfottrtA apotome

is the lesser root of the quadratic equation

X* - 2kp , X + - *V = O.

r 1 + A r

\end{notes}

\end{proposition}

\begin{proposition}
\label{propX_89}

\begin{statement}
To find the fifth apotome.
\end{statement}

\begin{proof}

Let a rational straight line A be set out,
and let CG be commensurable in length
with A ;

therefore CG is rational.

Let two numbers DF, FE be set out
such that DE again has not to either of the
numbers DF, FE the ratio which a square
number has to a square number ;
and let it be contrived that, as FE is to ED,
so is the square on CG to the square on GB.

Therefore the square on GB is also
rational ; [x. 6]

therefore BG is also rational.

Now since, as DE is to EF, so is the square on BG to
the square on GC,

while DE has not to EF the ratio which a square number
has to a square number,

therefore neither has the square on BG to the square on GC
the ratio which a square number has to a square number ;
therefore BG is incommensurable in length with GC. fx. 9]

And both are rational ;
therefore BG, GC are rational straight lines commensurable
in square only ;
therefore BC is an apotome. [x. 73]

I say next that it is also a fifth apotome.

For let the square on H be that by which the square on
BG is greater than the square on GC.

Since then, as the square on BG is to the square on GC,
so is DE to EF,

therefore, eonverlendo, as ED is to DF, so is the square on
BG to the square on H. [v. 19, Por.]

But ED has not to DF the ratio which a square number
has to a square number ;

therefore neither has the square on BG to the square on H
the ratio which a square number has to a square number ;

therefore BG is incommensurable in length with H. [x. 9]

And the square on BG is greater than the square on GC
by the square on H

therefore the square on GB is greater than the square on GC
by the square on a straight line incommensurable in length
with GB.

And the annex CG is commensurable in length with the
rational straight line A set out ;

therefore BC is a fifth apotome. [x. Deff. m. 5]

Therefore the fifth apotome BC has been found.
\end{proof}

\begin{notes}

Let p, ip and the numbers m, n of the last proposition be taken.

Take x such that »:(*» + *) = *V : x* (1).

In this case x> kp, and x = kp,/

= kp*fi +A, say.

Then shall (x - kp), or (kpji + A — kp), be a, fifth apotome.

For, by (1), x is rational and «- kp.

And since, by (1), (m + ») : m m x* : (x* - k*p*),

1/3- k*f? is incommensurable with x.

Also kp « p.

As explained in the note on x. 51, Ihefi/tk apotome

kp*ft+k- kp

is the lesser root of the quadratic

** - ikpjl + X . X + XyP/>' = o.

\end{notes}

\end{proposition}

\begin{proposition}
\label{propX_90}

\begin{statement}
To find the sixth apotome.
\end{statement}

\begin{proof}

Let a rational straight line A be set out, and three
numbers E, BC, CD not having

to one another the ratio which A

a square number has to a square f * - a

number ;

and further let CB also not have

to BD the ratio which a square

number has to a square number. g g c

Let it be contrived that, as
E is to BC, so is the square on A to the square on FG,

and, as BC is to CD, so is the square on FG to the square
on GH. [x. 6, For.]

Now since, as E is to BC, so is the square on A to the
square on FG,

therefore the square on A is commensurable with the square
on FG. [x. 6]

But the square on A is rational ;

therefore the square on FG is also rational ;

therefore FG is also rational.

And, since E has not to BC the ratio which a square
number has to a square number,

therefore neither has the square on A to the square on FG
the ratio which a square number has to a square number ;
therefore A is incommensurable in length with FG. [x. 9]

Again, since, as BC is to CD, so is the square on FG to
the square on GH,

therefore the square on FG is commensurable with the square
on GH. [x. 6]

But the square on FG is rational ;

therefore the square on GH is also rational ;

therefore GH is also rational.

And, since BC has not to CD the ratio which a square
number has to a square number,

therefore neither has the square on FG to the square on GH
the ratio which a square number has to a square number ;
therefore FG is incommensurable in length with GH. [x. 9]

And both are rational ;
therefore FG, GH are rational straight lines commensurable
in square only ;
therefore FH is an apotome. [x. 73]

I say next that it is also a sixth apotome.

For since, as E is to BC, so is the square on A to the
square on FG,

and, as BC is to CD, so is the square on FG to the square
on GH,

therefore, ex aeguaii, as E is to CD, so is the square on A to
the square on GH. [v, *i]

But E has not to CD the ratio which a square number
has to a square number ;

therefore neither has the square on A to the square on GH
the ratio which a square number has to a square number ;

therefore A is incommensurable in length with GH; [x. 9]

therefore neither of the straight lines FG, GH is commen-
surable in length with the rational straight line A.

Now let the square on A!'' be that by which the square on
FG is greater than the square on GH.

Since then, as BC is to CD, so is the square on FG to
the square on GH,

therefore, conver tench, as CB is to BD, so is the square on
FG to the square on K. [v. 19, Por.]

But CB has not to BD the ratio which a square number
has to a square number ;

therefore neither has the square on FG to the square on K
the ratio which a square number has to a square number ;

therefore FG is incommensurable in length with K. [x. 9]

And the square on FG is greater than the square on GH
by the square on K ;

therefore the square on FG is greater than the square on GH
by the square on a straight line incommensurable in length
with FG.

And neither of the straight lines FG, GH is commen-
surable with the rational straight line A set out.

Therefore FH is a sixth apotome. [x. Deff. 111. 6]

Therefore the sixth apotome FH has been found.
\end{proof}

\begin{notes}

Let p be the given rational straight line.

Take numbers p, (m + it), n which have not to one another the ratio of a
square number to a square number, m, n being also chosen such that the
ratio ( m + n) ; m is not that of square to square.

Take x,y such that / : (» + «) = p*: x* (i),

(»! + n) : n = x t :f (?).

Then shall (x -y) be a sixth apotome.

For, by ( i ), x is rational and w p (3).

By (3), since x is rational,

y is rational and « x (4).

Thus [(3), (4)] (x -y) is an apotome.
Again, ex atquali, p : n = p* : v*,

whence y - p.

Thus x, y are both u p.
Lastly, (onvertendo from (a),

(m + «ri : m = «? : («* -''),
whence Jx* -y* ~ at.

Therefore (x-y) is a rucrA apotome.
From (1) and (2) we have

fm + n

so that the sixth apotomt may be written

/m + n In

p v ~r ~ p v ? '

or, more simply, Jk . p — ,/X . p.

As explained in the note on x. 53, the j/M apotome is the lesser root of

the equation

X s - i*   P* + (* ``* ty P* = *

\end{notes}

\end{proposition}

\begin{proposition}
\label{propX_91}

\begin{statement}
If'' «» area £« contained by a rational straight line and a
first apotome, the ``side'' of the area is an apotome.
\end{statement}

\begin{proof}

For let the area AB be contained by the rational straight
line AC and the first apotome AD ;

I say that the ``side'' of the area AB is an apotome.

For, since AD is a first apotome, let DG be its annex ;

therefore AG, GD are rational straight lines commensurable
in square only. [x. 73]

And the whole AG is commensurable with the rational
straight line AC set out,

and the square on AG is greater than the square on GD
by the square on a straight line commensurable in length
with AG; [x. Deff. m. 1]

if therefore there be applied to AG a parallelogram equal to
the fourth part of the square on DG and deficient by a square
figure, it divides it into commensurable parts. [x. 17]

Let DG be bisected at E,

let there be applied to AG a parallelogram equal to the square
on EG and deficient by a square figure,
and let it be the rectangle AF, FG ;
therefore AF is commensurable with FG.

And through the points E, F, G let EH, FI GK be drawn
parallel to AC,

Now, since AF is commensurable in length with FG,

therefore AG is also commensurable in length with each of
the straight lines AF, FG. [x. 15]

But AG is commensurable with AC;

therefore each of the straight lines AF, FG is commensurable
in length with AC. [x. ia]

And AC is rational ;
therefore each of the straight lines AF, FG is also rational,
so that each of the rectangles A I, FK is also rational, [x. 19]

Now, since DE is commensurable in length with EG,

therefore DG is also commensurable in length with each of
the straight lines DE, EG. [x. 15]

But DG is rational and incommensurable in length
with AC;

therefore each of the straight lines DE, EG is also rational
and incommensurable in length with AC] [x. *3]

therefore each of the rectangles DH, EK is medial. [x. u]

Now let the square LM be made equal to A I, and let
there be subtracted the square NO having a common angle
with it, the angle LPM, and equal to FK;

therefore the squares LM, NO are about the same diameter.

[vi. 26]

Let PR be their diameter, and let the figure be drawn.
Since then the rectangle contained by AF, FG is equal to

the square on EG,

therefore, as AF is to EG, so is EG to FG. [vi. 17]

But, as AF is to EG, so is A I to EK,

and, as EG is to FG, so is EK to KF; [vi. 1]

therefore EKis a mean proportional between A J, KF. [v. 1 1]

But MN is also a mean proportional between LM, NO,
as was before proved, [Lemma after x, 53]

and AI is equal to the square LM, and KF to NO ;

therefore MN is also equal to EK.

But EK is equal to DH, and MN to LO ;
therefore DK is equal to the gnomon UVW a.n NO.

But AK is also equal to the squares LM, NO ;
therefore the remainder AB is equal to ST.

But ST is the square on LN ;
therefore the square on LN is equal to AB ;
therefore LN is the ``side'' of AB,

I say next that LN is an apotome.

For, since each of the rectangles A I, FK is rational,
and they are equal to LM, NO,

therefore each of the squares LM, NO, that is, the squares on
LP, PN respectively, is also rational ;
therefore each of the straight lines LP, PN is also rational.

Again, since DH is medial and is equal to LO,
therefore LO is also medial.

Since then LO is medial,
while NO is rational,
therefore LO is incommensurable with NO.

But, as LO is to NO, so is LP to PN; [vi. i]

therefore LP is incommensurable in length with PN. [x. 1 1]

And both are rational ;
therefore LP, PN are rational straight lines commensurable
in square only ;
therefore LN is an apotome. [x. 73]

And it is the ``side'' of the area AB ;
therefore the ``side'' of the area AB is an apotome.

Therefore etc.
\end{proof}

\begin{notes}

This proposition corresponds to x. 54, and the problem solved in it is to
find and to classify the side of a square equal to the rectangle contained by a
first apotome and p, or (algebraically) to find

Jp(kp-kpJ7= ).
First find u, v from the equations

' . U  ¥ V = kp

»» = 1W(i-V),
If u, v represent the values so found, put

y = pvi w

and (x —y) shall be the square root required.

To prove this Euclid argues thus.

By (1), u : kp ViV = kp <J t~k* ; v,

whence ptt ! J ip* Ji -X s = J kp 1 Ji -A* : pv,

or x* : i V VFA'' = i V VrX'' : f.

But [Lemma after x. 53]

X? ; xy = xy : y 1 ,
so that *ji = ivVi-A» (3).

x-)) <*

Therefore (x -y) 1 - x? + j> ! - 2xy

-V-VVr-X*.

Thus — ) is equal to J p(kp - kp Ji —X').

It has next to be proved that (-,y) is an apotome.

From (1) it follows, by x. 17, that

u `` v;
thus v,  are both commensurable with (u + v) and therefore with p (4),

Hence u, v are both rational,
so that pu, pv are rational areas ;

therefore, by ( 2), xy* are rational and commensurable (5),

whence also x, y are rational straight lines (6).

Next, kp Vi - A* is rational and  p ;
therefore ikp'' 1 Ji -' is a medial area.

That is, by (3), xy is a medial area.

But [(5)] y is a rational area ;
therefore jtv  y 1 ,

Or x s y.

But [(6)] *, jy are both rational.

Therefore x, y are rational and f~ ;
so that (x — y) is an apotome.

To find the form of (x-y) algebraically, we have, by solving (1),
u = hkp (1 + A),
y = lp(i-A),

whence, from <*), * = p /- (1 + K),

and *-J''»py -(» + A >-py J (»-* >

As explained in the note on x. 54, (*-) is the lesser positive root of the
biquadratic equation

X t -*p*.X* + VPp'=t>.

\end{notes}

\end{proposition}

\begin{proposition}
\label{propX_92}

\begin{statement}
If an area be contained by a rational straight line and a
second apotome, the ``side'' of the area is a first apotome of a
medial straight line.
\end{statement}

\begin{proof}

For let the area AB be contained by the rational straight
line AC and the second apotome AD ;

I say that the ``side'' of the area AB is a first apotome of a
medial straight line.

For let DG be the annex to AD ;

therefore AG, GD are rational straight lines commensurable
in square only, [x. 73]

and the annex DG is commensurable with the rational straight
line AC set out,

while the square on the whole AG is greater than the square
on the annex GD by the square on a straight line commen-
surable in length with AG. [x. Deffi 111. a]

Since then the square on AG is greater than the square
on GD by the square on a straight line commensurable
with AG,

therefore, if there be applied to AG a parallelogram equal to
the fourth part of the square on GD and deficient by a square
figure, it divides it into commensurable parts. [x. 17]

Let then DG be bisected at F,
let there be applied to AG a parallelogram equal to the square
on EG and deficient by a square figure,

and let it be the rectangle AF, FG ;

therefore AF 'is commensurable in length with FG.

Therefore AG is also commensurable in length with each
of the straight lines AF, FG. [x. 15]

But AG is rational and incommensurable in length
with AC;

therefore each of the straight lines AF, FG is also rational
and incommensurable in length with AC; [x. 13]

therefore each of the rectangles A J, FK is medial. [x. 1 1]

Again, since DE is commensurable with EG,
therefore DG is also commensurable with each of the straight
lines DE, EG. [s. 15]

But DG is commensurable In length with AC.
Therefore each of the rectangles DH, EK is rational.

[x. 19]

Let then the square LM he constructed equal to A I,
and let there be subtracted NO equal to FK and being about
the same angle with LM, namely the angle LPM;

therefore the squares LM, NO are about the same diameter.

[vi. 26]

Let PR be their diameter, and let the figure be drawn.

Since then AI, FKK medial and are equal to the squares
on LP, PN,
the squares on LP, PN are also medial ;

therefore LP, PN are also medial straight lines commen-
surable in square only.

And, since the rectangle AF, EG is equal to the square
on EG,

therefore, as AF is to EG, so is EG to EG, [vi. 17]

while, as AF is to EG, so is AI to EK,

and, as EG is to EG, so is EK to FK ; [vi. 1]

therefore EK is a mean proportional between AI,FK. [v. u]

But MN is also a mean proportional between the squares
LM, NO,

and AI is equal to LM, and FK to NO ;

therefore MN is also equal to EK.

But DH is equal to EK, and L equal to MN ;

therefore the whole DK is equal to the gnomon UVW
and NO.

Since then the whole AK is equal to LM, NO,

and, in these, DK is equal to the gnomon UVW a,n NO,

therefore the remainder AB is equal to TS.

But TS is the square on LN;
therefore the square on LN is equal to the area AB ;
therefore LN is the ``side'' of the area AB.

I say that LN is a first apotome of a medial straight line.
For, since EK is rational and is equal to LO,

therefore LO, that is, the rectangle LP, PN, is rational.

But NO was proved medial ;
therefore L,0 is incommensurable with NO.

But, as LO is to NO, so is LP to PN ; [vi. 1]

therefore LP, PN are incommensurable in length. [x. 11]

Therefore LP, PN are medial straight lines commen-
surable in square only which contain a rational rectangle ;

therefore LN is a first apotome of a medial straight line.

[*  74]
And it is the ``side'' of the area AB.
Therefore the ``side'' of the area AB is a first apotome
of a medial straight line.
\end{proof}

\begin{notes}

There is an evident flaw in the text in the place (Heiberg, p. 28*,
11. 1 7 — ta : translation p. 1 96 above) where it is said that ``since then A I, FK
are medial and are equal to the squares on LP, PN, the squares on LP, PN
are also medial ; therefore LP, PN are also medial straight lines commensurable
in square only'' It is not till the last lines of the proposition (Heiberg, p. 284,
11. 17, 18) that it is proved that LP, /Ware incommensurabk in length. What
should have been proved in the former passage is that the squares on LP, PN
are commensurable, so that LP, PN are commensurable in square (not
commensurable in square only). I have supplied the step in the note below :
``Also x'y', since a « v.'' Theon seems to have observed the omission and
to have put ``and commensurable with one another'' after `` medial'' in the
passage quoted, though even this does not show why the squares on LP, PN
are commensurable. One ms. (V) also has ``only'' ()Uvoy) erased after
``commensurable in square.''

This proposition amounts to finding and classifying

The method is that of the last proposition. Euclid solves, first, the
equations

kp

uv = |iy J

Then, using the values of u, v so found, he puts

 -pu
f = pv

 ** ) <*>,

and (x -y) is the square root required.

is proved in the same way as is the corresponding fact in x. 91.

From (1) u:kp = kp : v,

so that pu : J p* - f? : pv.

But X? : xy = xy :y',
whence, by (j), xy = kp i (3)-

Therefore- (* — ?)* - *~ '

= p(u + v)-kp*

 '-*)

Next, we have to prove that (x — y) is a first apotomt of a medial straight
line.

From (1) it follows, by x, 17, that

K'' v , (4).

therefore u, v are both <''> (« + v).

But [(1)] (w + v) is rational and v p ;
therefore tf, » are both rational and v p (5).

Therefore pu, pv, or *'', j* 1 , are both medial areas, and .v, j* are medial
straight lines (6).

Also 3? ~y, sinre » « w [(4)] (7).

Now jy, or J ip', is a rational area ;
therefore «/,

and *  Jd

Hence [(6), (7), (3)] ,  are medial straight lines commensurable in square
only and containing a rational rectangle ;

therefore (x-y) is  first apotomt of a medial straight line.

Algebraical solution of the equations gives

. 1 -X .
» = 4 - - , *?

and x

->-v/[()'-,v/)'

As explained in the note on x. 55, this is the lesser positive root of the
equation

\end{notes}

\end{proposition}

\begin{proposition}
\label{propX_93}

\begin{statement}
If'' «k area be contained by a rational straight line and a
third apotome, the ``side'' of the area is a second apotome of a
medial straight line.
\end{statement}

\begin{proof}

For let the area AB be contained by the rational straight
line AC and the third apotome AD ;

I say that the ``side'' of the area AB is a second apotome of
a medial straight line.

For let DG be the annex to AD ;
therefore AG, GD are rational straight lines commensurable
in square only,

and neither of the straight lines AG, GD is commensurable
in length with the rational straight line AC set out,
while the square on the whole AG is greater than the square
on the annex DG by the square on a straight line commen-

surable with AG.

[x. Deff. 111. 3]

c

) , >

K

R T M

Since then the square on A G is greater than the square
on GD by the square on a straight line commensurable
with AG,

therefore, if there be applied to AG a parallelogram equal to
the fourth part of the square on DG and deficient by a square
figure, it will divide it into commensurable parts. [x. 17]

Let then DG be bisected at E,
let there be applied to AG a parallelogram equal to th«
square on EG and deficient by a square figure,
and let it be the rectangle AF, EG.

Let EH, FI, GK be drawn through the points E, F, G
parallel to AC.

Therefore AF, FG are commensurable ;

therefore AI is also commensurable with FK. [vi. 1, x. n]

And, since AF, FG are commensurable in length,

therefore AG is also commensurable in length with each of
the straight lines AF, FG. [x. 15]

But AG is rational and incommensurable in length
with AC;
so that AF, FG are so also. [x. 13]

Therefore each of the rectangles AI, FK is medial, [x. ai]

Again, since DE is commensurable in length with EG,
therefore DG is also commensurable in length with each of
the straight lines DE, EG. [x. 15)

But GD is rational and incommensurable in length
with AC;

therefore each of the straight lines DE, EG is also rational
and incommensurable in length with AC; [x. 13]

therefore each of the rectangles DH, EK is medial. [x. «]

And, since AG, GD are commensurable in square only ;

therefore AG is incommensurable in length with GD.

But AG is commensurable in length with AF, and DG
with EG ;

therefore AF is incommensurable in length with EG. [x. 13]

But, as AF is to EG, so is A I to EK; [vi. 1]

therefore AI is incommensurable with EK. [x. 1 1]

Now let the square LMbe, constructed equal to AI,

and let there be subtracted NO equal to FK and being about
the same angle with LM;

therefore LM, NO are about the same diameter, [v<. tt]

Let PR be their diameter, and let the figure be drawn.
Now, since the rectangle AF, FG is equal to the square
on EG,

therefore, as AF is to EG, so is EG to FG. [vi. 17]

But, as AF is to EG, so is A I to EK,
and, as EG is to EG, so is /JTif to FK ; [vi. 1]

therefore also, as A I is to EK, so is iTA'' to EK; [v. 11]

therefore EK is a mean proportional between AI, FK,

But J/7V'' is also a mean proportional between the squares
LM, NO,

and AI is equal to LM, and FK to jV(? ;
therefore EK is also equal to MN.

But jfl/iV is equal to LO, and EK equal to ZW ;
therefore the whole DK is also equal co the gnomon WW
and NO.

But  K is also equal to LM, NO ;
therefore the remainder AB is equal to ST, that is, to the
square on LN
therefore LN is the ``side'' of the area AB.

I say that LN is a second apotome of a medial straight
line.

For.since AI,FKvrere proved medial, and are equal to the
squares on LP, FN,

therefore each of the squares on LP, PN is also medial ;
therefore each of the straight lines LP, PN is medial.

And, since AI is commensurable with FK, [vi. 1, x. n]
therefore the square on LP is also commensurable with the
square on PN.

Again, since A I was proved incommensurable with EK,
therefore LM is also incommensurable with MN,
that is, the square on LP with the rectangle LP, PN ;
so that LP is also incommensurable in length with PN ;

[vi. i, x. 11]
therefore LP, PN are medial straight lines commensurable in
square only.

I say next that they also contain a medial rectangle.

For, since EK was proved medial, and is equal to the
rectangle LP, PN,

therefore the rectangle LP, PN is also medial,
so that LP, PN are medial straight lines commensurable in
square only which contain a medial rectangle.

Therefore LN is a second apotome of a medial straight
line; [x. 7S ]

and it is the ``side'' of the area AB.

Therefore the ``side'' of the area AB is a second apotome
of a medial straight line.
\end{proof}

\begin{notes}

Here we are to find and classify the irrational straight line

Jp[Ji.p~JA.pjT-k'').
Following the same method, we put

v + v=Jk.p 1

iw=)V(i -* 1 ) ) X ) '

Next, u, v being found, let

x* = pu

f = pv) (2h

then (x -y) is the square root required and is a second apotome of a medial
straight line.

That (x -y) is the square root required and that aft jr are medial areas, so
that x, y are medial straight lines, is proved exactly as in the last proposition.

The rectangle xy, being equal to \ Jk . p' «/ 1 - A 1 , is also medial.

Now, from (i), by x, 17, « `` V,

whence w+ v » u.

But ( u + *>), of J-P,'' i J   P vi- A* ;

therefore ujk.pji — A : ,

and consequently pu v   Ji.p* Ji-k',

or x*  xy,

whence x vy.

And, since a « f , pu  pv,

or .r 3 «*j£

Thus .v, r are medial straight lines commensurable in square only.
And .rji is a medial area.

Therefore (x —y) is a second apotome of a medial straight line.
Its actual form is found by solving equations (1), (2);
thus u = HJt.p + KJi.p),

v = (Jk.p-Kjk.p),

and * p yi* (l+ x) p y~ (l A).

As explained in the note on x. 56, this is the lesser positive root of the
equation

x i -2 l Jk.p>x> + \ i kp' = o.

\end{notes}

\end{proposition}

\begin{proposition}
\label{propX_94}

\begin{statement}
If an area be contained by a rational straight line and a
fourth apotome, the ``siaW of the area is minor.
\end{statement}

\begin{proof}

For let the area AB be contained by the rational straight
line AC and the fourth apotome AD ;
I say that the ``side'' of the area AB is minor.

For let DG be the annex to AD ;
therefore AG, GD are rational straight lines commensurable
in square only,

AG is commensurable in length with the rational straight line
AC set out,

and the square on the whole AG is greater than the square
on the annex DG by the square on a straight line incommen-
surable in length with AG, [x. Deff. m. 4]

Since then the square on AG is greater than the square
on GD by the square on a straight line incommensurable
in length with AG,

therefore, if there be applied to AG a parallelogram equal to
the fourth part of the square on DG and deficient by a square
figure, it will divide it into incommensurable parts. [x. 18]

Let then DG be bisected at E,
let there be applied te AGs. parallelogram equal to the square
on EG and deficient by a square figure,
and let it be the rectangle AF, FG ;
therefore AF is incommensurable in length with FG.

Let EH, FI, GK be drawn through E, F, G parallel to
AC, BD.

Since then AG is rational and commensurable in length
with AC,
therefore the whole AK is rational. [x. 19]

Again, since DG is incommensurable in length with AC,
and both are rational,
therefore DK is medial. [x. 21]

Again, since AF is incommensurable in length with FG,
therefore A I is also incommensurable with FK. [vi. 1, x. n]

Now let the square LM be constructed equal to A I,
and let there be subtracted NO equal to FK and about the
same angle, the angle LPM.

Therefore the squares LM, NO are about the same
diameter. [vi. »6]

Let PR be their diameter, and let the figure be drawn.

Since then the rectangle AF, FG is equal to the square
on EG,

therefore, proportionally, as AF is to EG, so is EG to FG.

[vi. 17]

But, as AF is to EG, so is AI to EK,
and, as EG is to FG, so is EK to FK ; [vi. 1]

therefore EK is a mean proportional between A I, FK. [v. n]

But MN is also a mean proportional between the squares
LM, NO,

and AI is equal to LM, and FK to NO ;
therefore EK is also equal to MN.

But DH is equal to EK, and LO is equal to  /A r ;

therefore the whole DK is equal to the gnomon UVW
and JVa

Since, then, the whole A K is equal to the squares
LM, NO,

and, in these, DK is equal to the gnomon UVW and the
square NO,

therefore the remainder AB is equal to ST, that is, to the
square on LN ;
therefore LN is the ``side'' of the area AB.

I say that LN is the irrational straight line called minor.

For, since AK is rational and is equal to the squares on
LP, PN,
therefore the sum of the squares on LP, PN is rational.

Again, since DK is medial,
and DK is equal to twice the rectangle LP, PN,
therefore twice the rectangle LP, PN is medial.

And, since A J was proved incommensurable with FK,

therefore the square on LP is also incommensurable with the
square on PN.

Therefore LP, PN are, straight lines incommensurable in
square which make the sum of the squares on them rational,
but twice the rectangle contained by them medial.

Therefore LN's the irrational straight line called minor;

[x. 76]
and it is the ``side'' of the area AB.

Therefore the ``side'' of the area AB is minor.
\end{proof}

\begin{notes}

We have here to find flhd classify the straight line

equations
1 = kp

* 1 + X )

As usual, we find u, v from the equations

u + v = kp

uv-
and then, giving u, v their values, we put

*'.*) »

Then (x— y) is the required square root.

This is proved in the same way as before, and, as before, it is proved that

Now, from (r), by x. 18, u *, v;

therefore pu w pv,

Or ``?,

so that x, y are incommensurable in square.

And .** +y*, or p (a + v), is a rational area (kp 1 ).

it
But ixy = 1 , which is a medial area.

Hence [x. 76] (x~y) is the irrational straight line called minor.

ao« BOOR X [x. 94, 9

Algebraical solution gives

whence .-pf. + s/-V' `` /rh)'

As explained in the note on x. 57, this is the lesser positive root of the
equation

X> - 2 V . X* + --r k*f? - O.

-

1 + A

\end{notes}

\end{proposition}

\begin{proposition}
\label{propX_95}

\begin{statement}
If an area be contained by a rational straight line and a
fifth apotome, the ``side'' of the area is a straight line which
produces with a rational area a medial whole.
\end{statement}

\begin{proof}

For let the area AB be contained by the rational straight
line AC and the fifth apotome AD

I say that the `` side `` of the area AB is a straight line which
produces with a rational area a medial whole.

For let DG be the annex to AD ;
therefore AG, GD are rational straight lines commensurable
in square only,

the annex GD is commensurable in length with the rational

straight line AC set out,

and the square on the whole AG is greater than the square

on the annex DG by the square on a straight line incommen-
surable with AG. [x. Deff. 111. 5]

Therefore, if there be applied to AG a parallelogram
equal to the fourth part of the square on DG and deficient
by a square figure, it will divide it into incommensurable
parts. [x. 18]

Let then DG be bisected at the point E,

let there be applied to AG a parallelogram equal to the
square on EG and deficient by a square figure, and let it be
the rectangle AF, FG ;
therefore AF is incommensurable in length with FG.

Now, since AG is incommensurable in length with CA,
and both are rational,

therefore AK is medial. [x. 21]

Again, since DG is rational and commensurable in length
with AC,

DKs rational. [x. 19]

Now let the square LM be constructed equal to A/, and
let the square NO equal to FK and about the same angle, the
angle LPM, be subtracted ;

therefore the squares LM, NO are about the same diameter.

[vi. 26]

Let PR be their diameter, and let the figure be drawn.

Similarly then we can prove that LN is the ``side'' of the
area AB.

I say that LN is the straight line which produces with a
rational area a medial whole.

For, since AK was proved medial and is equal to the
squares on LP, PN,

therefore the sum of the squares on LP, PN is medial.

Again, since DK is rational and is equal to twice the
rectangle LP, PN,
the latter is itself also rational.

And, since AI is incommensurable with FK,
therefore the square on LP is also incommensurable with the
square on PN ;
therefore LP, PN are straight lines incommensurable in
square which make the sum of the squares on them medial
hut twice the rectangle contained by them rational.

Therefore the remainder LN is the irrational straight line
called that which produces with a rational area a medial
whole ; [x. 77]

and it is the `` side `` of the area AB.

Therefore the ``side'' of the area AB is a straight line
which produces with a rational area a medial whole.
\end{proof}

\begin{notes}

Here the problem is to find and classify

 Jp(ipJTTk-kp).
As usual, we put

u + v = p<Ji+k) . .

uv*lt>? I ***

and, v, v being found, we take

* = **\ <«).

Then (x -y) so found is our required square root.
This fact is proved as before, and, as before, we see that

Now from (1), by x. -8, v ~ v,

whence P«  pv,

or u/,

and .v, y are incommensurable in square.

Next (x* + y*) = p (a + v) = kf? J i + K, which is a medial area.

And ixy = ip\ which is a rational area.

Hence (x-y) is the ``side'' of a medial, minus a rational, area. [x. 77]

Algebraical solution gives

2

v~ —
and therefore

f (Vn-X-VA),

which is, as explained in the note to x. 58, the lesser positive root of the
equation

\end{notes}

\end{proposition}

\begin{proposition}
\label{propX_96}

\begin{statement}
If an area be contained by a rational straight line and a
sixth apotome, the ``side'' of the area is a straight line which
produces with a medial area a medial whole.
\end{statement}

\begin{proof}

For let the area AB be contained by the rational straight
line AC and the sixth apotome AD ;

I say that the ``side'' of the area AB is a straight line which
produces with a medial area a medial whole.

A d e f a

H I K

For let DG be the annex to AD ;

therefore AG, GD are rational straight lines commensurable
in square only,

neither of them is commensurable in length with the rational
straight line AC set out,

and the square on the whole A G is greater than the square
on the annex DG by the square on a straight line incommen-
surable in length with AG. [x. Deff. in. 6]

Since then the square on AG is greater than the square
on GD by the square on a straight line incommensurable in
length with AG,

therefore, if there be applied to AG a parallelogram equal to
the fourth part of the square on DG and deficient by a square
figure, it will divide it into incommensurable parts. [x. 18;

Let then DG be bisected at E,

let there be applied to AG a parallelogram equal to the square

no BOOK X [x. 96

on EG and deficient by a square figure, and let it be the

rectangle AF, FG ;

therefore AF is incommensurable in length with FG.

But, as AFis to FG, so is AI to FK; [vi.i]

therefore A I is incommensurable with FK. [x. u]

And, sincere AC are rational straight lines commensur-
able in square only,
AK is medial. [x. 21]

Again, since AC, DG are rational straight lines and
incommensurable in length,
DK is also medial. [x. 21]

Now, since AG, GD are commensurable in square only,
therefore AG is incommensurable in length with GD.

But, as A G is to GD, so is AK to KD ; [vi. 1]

therefore A K is incommensurable with KD. [x. n]

Now let the square LM be constructed equal to 4/,
and let NO equal to AA'', and about the same angle, be
subtracted ;

therefore the squares LM, NO are about the same diameter.

[vi. 26]

Let PR be their diameter, and let the figure be drawn.

Then in manner similar to the above we can prove that
LN is the `` side `` of the area AB.

I say that LN is a straight line which produces with a
medial area a medial whole.

For, since AK was proved medial and is equal to the
squares on LP, PN,
therefore the sum of the squares on LP, PN is medial.

Again, since DK was proved medial and is equal to twice
the rectangle LP, PN,
twice the rectangle LP, PN is also medial.

And, since AK was proved incommensurable with DK,
the squares on LP, PN are also incommensurable with twice
the rectangle LP, PN.

And, since AT is incommensurable with FK,
therefore the square on LP is also incommensurable with the
square 011 PN;

therefore LP, PN are straight lines incommensurable in
square which make the sum of the squares on them medial,
twice the rectangle contained by them medial, and further the
squares on them incommensurable with twice the rectangle
contained by them.

Therefore LN is the irrational straight line called that
which produces with a medial area a medial whole; [x. 78]

and it is the `` side `` of the area AB.

Therefore the ``side'' of the area is a straight line which
produces with a medial area a medial whole.
\end{proof}

\begin{notes}

We have to find and classify

>Jp(.p-jK.p).
Put, as usual,

w=£V I [ A

and, u, v being thus found, let

J = '``) <*).

Then, as before, (x -y) is the square root required.
For, from (1), by x. iS, w ~ v,

whence pu j pv,

or wy,

and x, y are incommensurable in square.

Next, Jt 1 +y* = p (» +  i>) = JA . p'', which is a media) area.
Also 2xy = JX. . p', which is again a medial area.
Lastly, Jk . ft Jk . p are by hypothesis 1, so that

Jk.pv jk.p,

whence Jk . p' v J, p*,

or (x'+y r ) -j 2xy.

Thus (x-y) is the ``side'' of a medial, minus a medial, area [x. 78].

Algebraical solution gives

whence * -J- = pVi (V* + -J*K) - p  /j (V* `` V*A).

This, as explained in the note on x. 59, is the lesser positive root of the
equation

x l -iJk.p'x*+(i-K)p t = o.

\end{notes}

\end{proposition}

\begin{proposition}
\label{propX_97}

\begin{statement}
The square on an apotome applied to a rational straight
line produces as breadth a first apotome.
\end{statement}

\begin{proof}

Let AB be an apotome, and CD rational,
and to CD let there be applied CE equal to the square on
AB and producing CF as breadth ;
I say that CF is a first apotome.

A B Q

C F N K M

For let BG be the annex to AB ;
therefore AG, GB are rational straight lines commensurable
in square only. [x. 73]

To CD let there be applied CH equal to the square on
AG, and KL equal to the square on BG.

Therefore the whole CL is equal to the squares on AG, GB,
and, in these, CE is equal to the square on AB;
therefore the remainder FL is equal to twice the rectangle
AG, GB. [». 7]

Let FM be bisected at the point N,
and let NO be drawn through TV parallel to CD ;
therefore each of the rectangles FO, LN is equal to the
rectangle AG, GB.

Now, since the squares on AG, GB are rational,
and DM is equal to the sq uares on A G, GB, .
therefore DM is rational.

And it has been applied to the rational straight line CD,
producing CM as breadth ;

therefore CM is rational and commensurable in length with
CD. [x. ao]

Again, since twice the rectangle AG, GB is medial, and
FL is equal to twice the rectangle AG, GB,
therefore FL is medial.

And it is applied to the rational straight line CD, producing
FM as breadth ;

therefore FM is rational and incommensurable in length with
CD. [x. 22]

And, since the squares on AG, GB are rational,

while twice the rectangle AG, GB is medial,

therefore the squares on AG, GB are incommensurable with
twice the rectangle AG, GB.

And CL is equal to the squares on AG, GB,
and FL to twice the rectangle A G, GB ;
therefore DM is incommensurable with FL.

But, as DM is to FL, so is CM to FM; [vi. 1]

therefore CM is incommensurable in length with FM. [x. n]

And both are rational ;

therefore CM, MF are rational straight lines commensurable
in square only ;

therefore CF is an apotome. [x. 73]

I say next that it is also a first apotome.
For, since the rectangle AG, GB is a mean proportional
between the squares on AG, GB,

and CH is equal to the square on AG,

KL equal to the square on BG,

and NL equal to the rectangle AG, GB,

therefore NL is also a mean proportional between CH, KL ;

therefore, as CH is to NL, so is NL to KL.

But, as CH is to NL, so is CK to NM,

and, as NL is to KL, so is NM to KM; [vi. 1]

therefore the rectangle CK, KM is equal to the square on
NM [vi. 1 7], that is, to the fourth part of the square on FM.

And, since the square on AG is commensurable with the
square on GB,
CH is also commensurable with KL.

But, as CH is to KL, so is CK to KM ; [vi. 1]

therefore CK is commensurable with KM. [*  ``]

Since then CM, MF are two unequal straight lines,

and to CM there has been applied the rectangle CK, KM

equal to the fourth part of the square on FM and deficient by

a square figure,

while CK is commensurable with KM,

therefore the square on CM is greater than the square on MF

by the square on a straight line commensurable in length

with CM. Ex. 17]

And CM is commensurable in length with the rational
straight line CD set out ;
therefore CF is a first apotome. [x. Deff. in. 1]

Therefore etc.
\end{proof}

\begin{notes}

Here begins the hexad of propositions solving the problems which are the
converse of those in the hexad just concluded- Props. 97 to ioj correspond
of course to Props. 60 to 65 relating to the binomials etc.

We have in x. 97 to prove that, (p — Jk . p) being an apotome,

(p~Jk.pf
*
is a first apotome, and we have to find it geometrically.
Euclid's procedure may be represented thus.
Take x, y, z such that

«- J

<y = V ) (')

<r . n = 2 *jk . p* '

Thus (x + y)-2sJ- p -* /i - ( )t ,

and we have to prove that (x+y) - 22 is a first apotome.

(a) Now jj 5 + ifj\ or <r (x +y), is rational ;

therefore (x+y) is rational and *a (2)-

And 2 Jk . p*, or <r . 2 s, is medial :
therefore 22 is rational and  <r (3).

But, cr (x 4-y) being rational, and a .  z metlial,
<r(x + y) w<r. it,
whence (* + „v) <* ``

Therefore, since (x +y), 2Z are both rational [(2), (3)],
(x +J>), 22 are rational and < *- (4).

Hence (x +y)- 22 is an apotome.

(ft) Since Jk . p* is a mean proportional between p s , ip*,
o-a is a mean proportional between ax, ay [by (1)].

That is, . ax : an = az : ay,
or x : z = z  y,
and *y = *, or i(a*)' (5).

x. 97, 98]

And, since p* « kp 1 , <rx n ory,
or j; `` j

Hence [(5), (6)], by X. 17,

 J(x + >)' - (is)' `` (x +Jf).

And [(4)] (* +ji), iz are rational and '«-,
while [(2)] (*+.)') « a ;
therefore (.r +J>) — 2Z is ajf.r.ir' apotgme.

The actual value of (jt + y) - as is of course

aiS

...(6).

\end{notes}

\end{proposition}

\begin{proposition}
\label{propX_98}

\begin{statement}
The square on a first apotome of a medial straight line
applied to a rational straight line produces as breadth a second
apotome.
\end{statement}

\begin{proof}

Let AB be a first apotome of a medial straight line and
CD a rational straight line,

and to CD let there be applied CM equal to the square on
AB, producing CF as breadth ;
I say that CF is a second apotome.

For let BG be the annex to AB ;.
therefore AG, GB are medial straight lines commensurable in
square only which contain a rational rectangle. [x. 74]

A 6 G

F

< M

D E

: c

3

4 L

To CD let there be applied C.H equal to the square on
AG, producing CK as breadth, and KL equal to the square
on GB, producing KM as breadth ;

therefore the whole CL is equal to the squares on A G, GB ;
therefore CL is also medial. [``.15 and 23, Por.]

And it is applied to the rational straight line CD, pro-
ducing CM as breadth ;

therefore CM is rational and incommensurable in length with
CD. [x. 22]

Now, since CL is equal to the squares on AG, GB,
and, in these, the square on AB is equal to CE,
therefore the remainder, twice the rectangle A G, GB, is equal
to FL. [n. 7]

But twice the rectangle AG, GB is rational ;
therefore FL is rational.

And it is applied to the rational straight line FE, producing
FM as breadth ;

therefore FM is also rational and commensurable in length
with CD. [x. 20]

Now, since the sum of the squares on AG, GB, that is,
CL, is medial, while twice the rectangle AG, GB, that is, FL,
is rational,
therefore CL is incommensurable with FL.

But, as CL is to FL, so is CM to FM; [vi. 1]

therefore CM is incommensurable in length with FM. [x. n]

And both are rational ;
therefore CM, MF are rational straight lines commensurable
in square only ;

therefore CF is an apotome. [x. 73]

I say next that it is also a second apotome.

For let FM be bisected at N,
and let NO be drawn through A parallel to CD ;
therefore each of the rectangles FO, NL is equal to the
rectangle A G, GB.

Now, since the rectangle AG, GB is a mean proportional
between the squares on AG, GB,
and the square on AG is equal to CH,
the rectangle AG, GB to NL,
and the square on BG to KL,

therefore NL is also a mean proportional between CH, KL;
therefore, as CH is to NL, so is NL to KL.

But, as CH is to NL, so is CK to NM,
and, as NL is to KL, so is NM to MK; [vi. 1]

therefore, as CK is to NM, so is NM to KM '; [v. 11]

therefore the rectangle CK, KM is equal to the square on
NM [vi. 17], that is, to the fourth part of the square on FM,

Since then CM, MF are two unequal straight lines, and
the rectangle CK, KM equal to the fourth part of the square
on MF and deficient by a square figure has been applied to
the greater, CM, and divides it into commensurable parts,
therefore the square on CM is greater than the square on MF
by the square on a straight line commensurable in length with
CM. [x. .7]

And the annex FM is commensurable in length with the
rational straight line CD set out ;
therefore CF is a second apotome. [x, Deff. 111. a]

Therefore etc.
\end{proof}

\begin{notes}

In this case we have to find and classify

Take x, y, z such that

ax = p'

 ?=>**/ I i*y

<r , tz = Jjfp* I

(a) Now £y, 4V are medial areas ;

therefore <r (x + y) is medial,

whence +y) is rational and  <r , (2).

But zip 8 , and therefore o- . zsr, is rational,
whence iz is rational and * <r (3).

And, <r(x + y) being medial, and er . iz rational,
a ( x + y) vj <r . 2Z,
or (jc +y) v zz.

Hence (jc + y), za are rational straight lines com mens ura''ble in square only,
and therefore (x + y)-nz is an apvtofne,

() We prove, as before, that

«M4Ktf (4).

Also k?f? f* k'p', or ax « try,
so that :r''jr (5).

[This step is omitted in P, and Heiberg accordingly brackets it. The
result is, however, assumed.]
Therefore [(4), (5)], by x, 17,

 A* +y)' - (2Z) 1 « (* +y).
And 22 « <r.
Therefore (x +y) - iz is a second apotome.

o>
Obviously (x +y) - 2  - - JA (i + k) - 2k).

\end{notes}

\end{proposition}

\begin{proposition}
\label{propX_99}

\begin{statement}
The square on a second apotome of a medial straight line
applied to a rational straight line produces as breadth a third
apotome.
\end{statement}

\begin{proof}

Let AB be a second apotome of a medial straight line,
and CD rational,

and to CD let there be applied CE equal to the square on
AB, producing CF as breadth ;
I say that CF is a third apotome.

M

For let BG be the annex to AB ;
therefore AG, GB are medial straight lines commensurable
in square only which contain a medial rectangle. [x. 75]

Let CH equal to the square on AG be applied to CD,
producing CK as breadth,

and let KL equal to the square on BG be applied to KH t
producing KM as breadth ;

therefore the whole CL is equal to the squares on A G, GB ;
therefore CL is also medial- [x. 15 and 23, Por.]

And it is applied to the rational straight line CD, producing
CM as breadth ;

therefore CM is rational and incommensurable in length with
CD. [x. 22]

Now, since the whole CL is equal to the squares on AG,
GB, and, in these, CE is equal to the square on AB,
therefore the remainder LF is equal to twice the rectangle
AG,GB. [h.7]

Let then FM be bisected at the point N,
and let NO be drawn parallel to CD ;

therefore each of the rectangles FO, NL is equal to the rect-
angle AG, GB.

But the rectangle AG, GB is medial ;
therefore FL is also medial.

And it is applied to the rational straight line EF, producing
FM as breadth ;

therefore FM is also rational and incommensurable in length
with CD. [x. **]

And, since AG, GB are commensurable in square only,
therefore AG is incommensurable in length with GB;
therefore the square on A G is also incommensurable with the
rectangle AG, GB. [vi. i, x. n]

But the squares on AG, GB are commensurable with the
square on AG,

and twice the rectangle A G, GB with the rectangle A G, GB ;
therefore the squares on AG, GB are incommensurable with
twice the rectangle AG, GB. [x. 13]

But CL is equal to the squares on AG, GB,
and FL is equal to twice the rectangle AG, GB ;
therefore CL is also incommensurable with FL.

But, as CL is to FL, so is CM to FM; [vi. 1]

therefore CM is incommensurable in length with FM. [x. 11]

And both are rational ;

therefore CM, MFare rational straight lines commensurable
in square only ;

therefore CF is an apotome. [x, 73]

I say next that It is also a third apotome.
For, since the square on AG is commensurable with the
square on GB,

therefore CH is also commensurable with KL t
so that CK is also commensurable with KM. [vi. 1, x. it]

And, since the rectangle AG, GB Is a mean proportional
between the squares on AG, GB,

and CH is equal to the square on A G,

KL equal to the square on GB,

and NL equal to the rectangle AG, GB,

therefore NL is also a mean proportional between CH, KL ;

therefore, as CH is to NL, so is NL to KL.

But, as CH is to NL, so is CK to NM,
and, as NL is to KL, so is JVJSf to KM ; [vj. i]

therefore, as CK is to MN, so is /.A/ to KM; [v. n]

therefore the rectangle CK, KM is equal to [the square on
MN, that is, to] the fourth part of the square on FM.

Since then CM, MF are two unequal straight lines, and
a parallelogram equal to the fourth part of the square on FM
and deficient by a square figure has been applied to CM, and
divides it into commensurable parts,

therefore the square on CM is greater than the square on
MF by the square on a straight line commensurable with
CM. [x. 17]

And neither of the straight lines CM, MF is commensur-
able in length with the rational straight line CD set out ;
therefore CF is a third apotome. [x. Deff. 111. 3]

Therefore etc.
\end{proof}

\begin{notes}

We have to find and classify

Take x, y, 2 such that

Ji.p-

X

J*

*~W

(a) Then tr(x + y) is a medial area,

whence (x +y) is rational and  <r (1).

Also a . 2Z is medial,
whence az is rational and w a (z).

Again ~r>

whence Jk .(? <-> J\ . p*.

And Jk.pJk.ff + f?),

while /K . p' « 2 JK . (? ;

therefore f Jk . p* + —p, p* J  ik.p

or o- (jr +y) s/ <r . 2z,

and (* + /) ** « (3)

x. 99, ioo]

Thus f(i), (2), (3)] (x +y), 23 are rational and ``-,
so that (x +y) — 2Z is an apotome.

(£) trx f try, so that x  y.

And, as before, xy = (ts)

Therefore [x. 17] (x+yf-iiz)' « (* + y).
And neither (x +y) nor iz is « <r.
Therefore (* + y) - 22 is a Ai/rtf apotome.
It is of course equal to

\end{notes}

\end{proposition}

\begin{proposition}
\label{propX_100}

\begin{statement}
The square on a minor straight line applied to a rational
straight line produces as breadth a fourth apotome.
\end{statement}

\begin{proof}

Let AB be a minor and CD a rational straight line, and
to the rational straight line CD let CE be applied equal to the
square on AB and producing CF as breadth ;
I say that CF'xs, a fourth apotome.

For let BG be the annex to AB ;
therefore AG, GB are straight lines incommensurable in
square which make the sum of the squares on AG, GB
rational, but twice the rectangle AG, GB medial. [x. 76]

To CD let there be applied CH equal to the square on
A G and producing CK as breadth,

and KL equal to the square on BG, producing KM as breadth ;
therefore the whole CL is equal to the squares on AG, GB.

And the sum of the squares on AG, GB is rational ;
therefore CL is also rational.

And it is applied to the rational straight line CD, producing
CM as breadth ;

therefore CM is also rational and commensurable in length
with CD, [x. to]

AikI, since the whole CL is equal to the squares on AG,
GB, and, in these, CM is equal to the square on AB,

therefore the remainder FL is equal to twice the rectangle
AG, GB. [u. 7]

Let then FM be bisected at the point N,

and let NO be drawn through N parallel to either of the
straight lines CD, ML ;

therefore each of the rectangles FO, NL is equal to the rect-
angle AG, GB.

And, since twice the rectangle AG, GB is medial and is
equal to FL,

therefore FL is also medial.

And it is applied to the rational straight line FE, producing
FM as breadth ;

therefore FM is rational and incommensurable in length with
CD. [x. »]

And, since the sum of the squares on AG. GB is rational,
while twice the rectangle AG, GB is medial,
the squares on AG, GB are incommensurable with twice the
rectangle AG, GB.

But CL is equal to the squares on AG, GB,
and FL equal to twice the rectangle AG, GB ;
therefore CL is incommensurable with FL.

But, as CL is to FL, so is CM to MF; [vi. i]

therefore CM is incommensurable in length with MF. [x. n]

And both are rational ;
therefore CM, MF are rational straight lines commensurable
in square only ;
therefore CF is an apotome. [x. 73]

I say that it is also a fourth apotome.

For, since AG, GB are incommensurable in square,

therefore the square on AG is also incommensurable with the
square on GB,

And CH is equal to the square on A G,
and KL equal to the square on GB
therefore CH is incommensurable with KL.

But, as CH is to KL, so is CK to KM; [vi. 1)

therefore CA'' is incommensurable in length with KM. [x. n]

And, since the rectangle AG, GB is a mean proportional
between the squares on AG, GB,
and the square on AG is equal to CH,
the square on GB to KL,
and the rectangle AG, GB to NL,
therefore NL is a mean proportional between CH, KL j
therefore, as £7/ is to NL, so is jVX to KL.

But, as C7/ is to NL, so is CK to AW,

and, as NL is to ATZ, so is NM to AW; [vi i]

therefore, as CK is to MN, so is iJ/A'' to KM; [v, u]

therefore the rectangle CK, KM is equal to the square on
MN [vi. i 7], that is, to the fourth part of the square on FM.

Since then CM, MF are two unequal straight lines, and
the rectangle CK, KM equal to the fourth part of the square
on MF and deficient by a square figure has been applied to
CM and divides it into incommensurable parts,

therefore the square on CM is greater than the square on
MF by the square on a straight line incommensurable with
CM. [x. 18]

And the whole CM is commensurable in length with the
rational straight line CD set out ;

therefore CF is a fourth apotome. [x. Deff. m. 4]

Therefore etc.
\end{proof}

\begin{notes}

We have to find and classify
We will call this, for brevity,

Take x, y, z such that

ty =  ),
a . 2* = 2OTJ J

where it has to be remembered that tt\ 1? are incommensurable, («* + »/'') if
rational, and tuv medial.

It follows that tr(x+y) is rational and o- . 22 medial,

so that (x -t-y) is rational and « o-

while 2£ is rational and u<r

and a(x + y) ut.jj,

so that (x+y)zz

Thus [(i), (z), (3)] ( +>), 2J are rational and >*-,
so that (,* +>) — 2« is an apotome.

Next, since »*  i?,

<r jc -j <jy,
or * „ ji.

And it is proved, as usual, that

xy = s t = i(tz) t .

Therefore [x. 18] *J(x +;)-''( izf ~ (x+y).

But (x+y) « <r,
therefore jr+) p — 21 is a. fourth apotomt.

P*

 (0.

(3).

Its value is of course

\end{notes}

\end{proposition}

\begin{proposition}
\label{propX_101}

\begin{statement}
7''« square on the straight line which produces with a
rational area a medial whole, if applied to a rational straight
line, produces as breadth a fifth apotome.
\end{statement}

\begin{proof}

Let AB be the straight line which produces with a
rational area a medial whole, and CD a rational straight line,
rtnd to CD let C£ be applied equal to the square on AB and
producing CF as breadth ;
I say that CF is a fifth apotome.

a no

M

d e

For let BG be the annex to AB ;
therefore AG, GB are straight lines incommensurable in
square which make the sum of the squares on them medial
but twice the rectangle contained by them rational. [x. 77]

To CD let there be applied CM equal to the square on
A G, and KL equal to the square on GB ;
therefore the whole CL is equal to the squares on AG, GB.

But the sum of the squares on AG, GB together is
medial ;

therefore CL is medial.

And it is applied to the rational straight line CD, producing
CM as breadth ;

therefore CM is rational and incommensurable with CD. [x. 22]
And, since the whole CL is equal to the squares on AG, GB,
and, in these, CE is equal to the square on AB,
therefore the remainder FL is equal to twice the rectangle
AG, GB. [11. 7]

Let then FM be bisected at N,

and through N let NO be drawn parallel to either of the
straight lines CD, ML ;

therefore each of the rectangles FO, NL is equal to the rect-
angle AG, GB'.

And, since twice the rectangle AG, GB is rational and
equal to FL,
therefore FL is rational.

And it is applied to the rational straight line EF, producing
FM as breadth ;

therefore FM is rational and commensurable in length with
CD. [x. ao]

Now, since CL is medial, and FL rational,
therefore CL is incommensurable with FL.

But, as CL is to FL, so is CM to MF; [vi. 1]

therefore CM is incommensurable in length with MF. [x. n]

And both are rational ;
therefore CM, MF are rational straight lines commensurable
in square only ;
therefore CF is an apotome. [x. 73]

I say next that it is also a fifth apotome.

For we can prove similarly that the rectangle C K, KM
is equal to the square on NM, that is, to the fourth part of the
square on FM.

And, since the square on AG is incommensurable with the
square on GB,
while the square on AG is equal to CH,

and the square on GB to KL,

therefore CH is incommensurable with KL .

But, as CH is to KL, so is CK to KM ; [vi. i]

therefore CK 'is incommensurable in length with KM. [x. ``]

Since then CM, MFare two unequal straight lines,
and a parallelogram equal to the fourth part of the square
on FM and deficient by a square figure has been applied to
CM, and divides it into incommensurable parts,
therefore the square on CM is greater than the square on
MF by the square on a straight line incommensurable with
CM. [x. i8j

And the annex FM is commensurable with the rational
straight line CD set out ;
therefore CF is a fifth apotome. |x. Deff, hi. 5)
\end{proof}

\begin{notes}

We have to Pod and classify

«r Wa<t+)

*/ah +jP1 1

Call this - (« - r)*> and take a-, v, z such that

<rx = u' 1

<TV = *» .

cr. M- i«p )

In this case « s . * are incommensurable, (a* + 1?) is a medial area and iuv
a rational area.

Since it(x + j>) is medial and <r . 2z rational,
(x +y) is rational and w c
2* is rational and A tr,
while (*+>) - **

It follows that ( +>>), 2  are rational and «-,
so that (*+.?) - ** is an apotome.

Again, as before, xy = z a = \ (mf,

and, since «' v **, 03  try,

or x v y.

Hence [x. 18]- +>')'-(a«)' « (*+>).

And 32 `` <r.

Therefore ( * +.y) - 22 is a fifth apotemt.

It is of course equal to

\end{notes}

\end{proposition}

\begin{proposition}
\label{propX_102}

\begin{statement}
The square on the straight line which produces with a
medial area a medial whole, if applied to a rational straight
line, produces as breadth a sixth apotome.
\end{statement}

\begin{proof}

Let AB be the straight line which produces with a medial
area a medial whole, and CD a rational straight line,

and to CD let CE be applied equal to the square on AB and
producing CF as breadth ;

I say that CF is a sixth apotome.

M

For let BG be the annex to AB ;

therefore AG, GB are straight lines incommensurable in
square which make the sum of the squares on them medial,
twice the rectangle AG, GB medial, and the squares on AG,
GB incommensurable with twice the rectangle AG, GB. [x. j8]

Now to CD let there be applied CH equal to the square
on AG and producing CK as breadth,

and KL equal to the square on BG ;

therefore the whole CL is equal to the squares on AG, GB;

therefore CL is also medial.

And it is applied to the rational straight line CD, produc-
ing CM as breadth ;

therefore CM is rational and incommensurable in length
with CD. [x. »a)

Since now CL is equal to the squares on AG, GB,
and, in these, CF is equal to the square on AB,
therefore the remainder FL is equal to twice the rectangle
AG, GB. [11. 7]

And twice the rectangle AG, GB is medial ;
therefore FL is also medial.

And it is applied to the rational straight line FE, pro-
ducing FM as breadth ;

therefore FM is rational and incommensurable in length
with CD, [x. 21]

And, since the squares on AG, GB are incommensurable
with twice the rectangle AG, GB,
and CL is equal to the squares on AG, GB,
and FL equal to twice the rectangle AG, GB,
therefore CL is incommensurable with FL.

But, as CL is to FL, so is CM to MF; [n. 1]

therefore CM is incommensurable in length with MF, [x. n]

And both are rational.

Therefore CM, MF are rational straight lines commen-
surable in square only ;
therefore CF is an apotome. [x. 73]

I say next that it is also a sixth apotome.

For, since FL is equal to twice the rectangle AG, GB,
let FM be bisected at N,

and let NO be drawn through N parallel to CD ;
therefore each of the rectangles FO, NL is equal to the rect-
angle AG, GB.

And, since AG, GB are incommensurable in square,

therefore the square on AG is incommensurable with the
square on GB.

But CH is equal to the square on AG,
and KL is equal to the square on GB ;
therefore CH is incommensurable with KL.

But, as CH is to KL, so is CK to KM ; [vi, 1]

therefore CK is incommensurable with KM. [x. n]

And, since the rectangle AG, GB is a mean proportional
between the squares on AG, GB,

and CH is equal to the square on AG,

KL equal to the square on GB,

and NL equal to the rectangle AG, GB,

therefore NL is also a mean proportional between CH, KL ;

therefore, as CH is to NL, so is NL to KL.

And for the same reason as before the square on CM is
greater than the square on MF by the square on a straight
line incommensurable with CM. [x. 18]

And neither of them is commensurable with the rational
straight line CD set out ;
therefore CF is a sixth apotome. [x. Deff. hi. 6]
\end{proof}

\begin{notes}

We have to find and classify

i/ei/ I + jt f A *

rWiV VJ+s Tv TxWj

Call this - (u - v)', and put

<r* = **,

ff . 2S = XUV.

Here «*, w° are incommensurable,
(u* + s*), 2uv are both medial areas,
and (« a + »*)  iuv.

Since o- (a+), o- . 22 are medial and incommensurable,
(x + y) is rational and  o-,
zz is rational and w <r,
and (*+>) « as.

Hence (.* + ?), az are rational and ``-,
so that (x +y) — 22 is an apotome.

Again, since a', p*, or cms, ory, are incommensurable,

And, as before, xy   *'' = i (m)'.

Therefore [x. t8] V(i +.?)* - (22)' « (*+.>').
And neither (x+y) nor 22 is « z;
therefore (a +/) — 2* is a (£«C4 apotome.

It is of course - [ JK — ,    ) .

\end{notes}

\end{proposition}

\begin{proposition}
\label{propX_103}

\begin{statement}
A straight line commensurable in length with an apotome
is an apotome and the same in order.
\end{statement}

\begin{proof}

Let AB be an apotome,

and let CD be commensurable in a j E

length with AS ; Q p F

I say that CD is also an apotome and
the same in order with AB.

For, since AB is an apotome, let BE be the annex to it ;

therefore AE, EB are rational straight lines commensurable
in square only. [x. 73]

Let it be contrived that the ratio of BE to DF is the same
as the ratio of AB to CD ; [vi. 12)

therefore also, as one is to one, so are all to all ; [v. t'a]

therefore also, as the whole AE is to the whole CF, so is AB
to CD.

But AB is commensurable in length with CD.
Therefore AE is also commensurable with CF, and BE
with DF. [x. 11]

And AE, EB are rational straight lines commensurable in
square only ;

therefore CF, FD are aJso rational straight lines commensur-
able in square only. [x. 13]

Now since, as AE is to CF, so is BE to DF,
alternately therefore, as AE is to EB, so is CF to FD. [v. 16]

And the square on AE is greater than the square on EB
either by the square on a straight line commensurable with
AE or by the square on a straight line incommensurable
with it.

If then the square on AE is greater than the square on
EB by the square on a straight line commensurable with AE,
the square on CF will also be greater than the square on FD
by the square on a straight line commensurable with CF.

[x. 14]

And, if AE is commensurable in length with the rational
straight line set out,

CF is so also, [x. 1 3]

if BE, then DF also, [id.]

and, if neither of the straight lines AE, EB, then neither of
the straight lines CF, FD. [x. 13]

But, if the square on AE is greater than the square on EB
by the square on a straight line incommensurable with AE,

the square on CF will also be greater than the square on FD
by the square on a straight line incommensurable with CF.

[x. 14]

And, if AE is commensurable in length with the rational

straight line set out,
CF is so also,

if BE, then DFaho, [x. 13]

and, if neither of the straight lines AE, EB, then neither of
the straight lines CF, FD, [x. 13]

Therefore CD is an apotome and the same in order
with AB.
\end{proof}

\begin{notes}

This and the following propositions to 107 inclusive (like the correspond-
ing theorems x. 66 to 70) are easy and require no elucidation. They are
equivalent to saying that, if in any of the preceding irrational straight lines

- p is substituted for p, the resulting irrational is of the same kind and order

as that from which it is altered.

\end{notes}

\end{proposition}

\begin{proposition}
\label{propX_104}

\begin{statement}
A straight line commensurable with an apotome of a
medial straight line is an apotome of a medial straight line
and the same in order.
\end{statement}

\begin{proof}

Let AB be an apotome of a medial straight line,
and let CD be commensurable in

length with AB ; * .?

I say that CD is also an apotome of a (J BF

medial straight line and the same in
order with AB.

For, since AB is an apotome of a medial straight line, let
EB be the annex to it.

Therefore AE, EB are medial straight lines commensur-
able in square only. [x. 74, 75]

Let it be contrived that, as AB is to CD, so is BE to DF ;

[vt. 12]

therefore AE is also commensurable with CF, and BE
with DF. [v. i2, x. 11]

But AE, EB are medial straight lines commensurable in
square only ;

therefore CF, FD are also medial straight lines [x. 23] com-
mensurable in square only ; [x. 13]
therefore CD is an apotome of a medial straight line. [x. 74, 75]

I say next that it is also the same in order with AB.
Since, as AE is to EB, so is CF to FD,

therefore also, as the square on AE is to the rectangle AE,
EB, so is the square on CF to the rectangle CF, FD.

But the square on AE is commensurable with the square
on CF;

therefore the rectangle AE, EB is also commensurable with
the rectangle CF, FD. [v. 16, x. 11]

Therefore, if the rectangle AE, EB is rational, the rect-
angle CF, FD will also be rational, [x. Dtf. 4'

and if the rectangle AE, EB is medial, the rectangle CF, FD

is also medial. [x. 23, Por.]

Therefore CD is an apotome of a medial straight line and

the same in order with AB. [x. 74, 75]
\end{proof}

\end{proposition}

\begin{proposition}
\label{propX_105}

\begin{statement}
A straight line commensurable with a minor straight line
is minor.
\end{statement}

\begin{proof}

Let AB be a minor straight line, and CD commensurable
with AB ;
I say that CD is also minor.

i A B E

Let the same construction be made

as before ; ? 2 F

then, since AE, EB are incommensur-
able in square, [x. 76]
therefore CF, FD are also incommensurable in square, [x. 13]

Now since, as AE is to EB, so is CF to FD, [v. 12, v. 16]

therefore also, as the square on AE is to the square on EB,
so is the square on CF to the square on FD. [vi. 22]

Therefore, cumponendo, as the squares on AE, EB are to
the square on EB, so are the squares on CF, FD to the
square on FD. [v. 18]

But the square on BE is commensurable with the square
on DF;

therefore the sum of the squares on AE, EB is also commen-
surable with the sum of the squares on CF, FD. [v. 16, x. n]

But the sum of the squares on AE, EB is rational ; [x. 76]

therefore the sum of the squares on CF t FD is also rational.

[x. Def. 4]

Again, since, as the square onAE is to the rectangle AE,
EB, so is the squaru on CF to the rectangle CF, FD,

while the square on AE is commensurable with the square
on CF,

therefore the rectangle AE, EB is also commensurable with
the rectangle CF, FD.

But the rectangle AE, EB is medial ; [x. 76]

therefore the rectangle CF, FD is also medial ; [x. 13, Por.]

therefore CF, FD are straight lines incommensurable in square
which make the sum of the squares on them rational, but the
rectangle contained by them medial.

Therefore CD is minor. [x. j6]
\end{proof}

\end{proposition}

\begin{proposition}
\label{propX_106}

\begin{statement}
A straight line commensurable with that ``which produces
with a rational area a medial whole is a straight line which
produces with a rational area a medial whole.
\end{statement}

\begin{proof}

Let AB be a straight line which produces with a rational
area a medial whole,

and CD commensurable with AB ; . e E
I say that CD is also a straight line

which produces with a rational area a — ™

medial whole.

For let BE be the annex to AB ;

therefore AE, EB are straight lines incommensurable in
square which make the sum of the squares on AE, EB
medial, but the rectangle contained by them rational. [x. 77]

Let the same construction be made.

Then we can prove, in manner similar to the foregoing,
that CF, FD are in the same ratio as AE, EB,

the sum of the squares on AE, EB is commensurable with
the sum of the squares on CF, FD,

and the rectangle AE, EB with the rectangle CF, FD ;

so that CF, FD are also straight lines incommensurable in
square which make the sum of the squares on CF, FD medial,
but the rectangle contained by them rational.

Therefore CD is a straight line which produces with a
rational area a medial whole. [x. 77]
\end{proof}

\end{proposition}

\begin{proposition}
\label{propX_107}

\begin{statement}
A straight line commensurable with that which produces
with a medial area a medial whole is itself also a straight line
which produces with a medial area a medial whole.
\end{statement}

\begin{proof}

Let AB be a straight line which produces with a medial
area a medial whole,

and let CD be commensurable with AB;

I say that CD is also a straight line `` '

which produces with a medial area a '

medial whole.

For let BE be the annex to AB,

and let the same construction be made ;

therefore AE, EB are straight lines incommensurable in
square which make the sum of the squares on them medial,
the rectangle contained by them medial, and further the sum
of the squares on them incommensurable with the rectangle
contained by them. [x. 78]

Now, as was proved, AE, EB are commensurable with
CF, ED,

the sum of the squares on AE, EB with the sum of the
squares on CF, FD,

and the rectangle AE, EB with the rectangle CF, FD ;

therefore CF, FD are also straight lines incommensurable in
square which make the sum of the squares on them medial,
the rectangle contained by them medial, and further the sum
of the squares on them incommensurable with the rectangle
contained by them.

Therefore CD is a straight line which produces with a
medial area a medial whole. [x. 78]
\end{proof}

\end{proposition}

\begin{proposition}
\label{propX_108}

\begin{statement}
A e
\end{statement}

\begin{proof}

If from a rational area a medial area be subtracted, the
``side'' of the remaining area becomes one of two irrational
straight lines, either an apotome or a minor straight line.

For from the rational area BC let the medial area BD be
subtracted ;

I say that the `` side `` of the
remainder EC becomes one
of two irrational straight lines,
either an apotome or a minor
straight line.

For let a rational straight
line FG be set out,
to FG let there be applied the
rectangular parallelogram GH
equal to BC,

and let GK equal to DB be subtracted ;
therefore the remainder EC is equal to LH.

Since then BC is rational, and BD medial,
while BC is equal to GH, and BD to GK,
therefore GH is rational, and GK medial.

And they are applied to the rational straight line FG ;
therefore FH is rational and commensurable in length with
FG, [x. 20]

while FK is rational and incommensurable in length with FG;

[X. 2 3 ]

therefore FH is incommensurable in length with FK, [x. 13]

Therefore FH, FK are rational straight lines commen-
surable in square only ;
therefore KH is an apotome [x. 73], and KF the annex to it.

Now the square on HF is greater than the square on FK
by the square on a straight line either commensurable with
HF or not commensurable.

First, let the square on it be greater by the square on a
straight line commensurable with it.

Now the whole HF is commensurable in length with the
rational straight line FG set out ;
therefore KH is a first apotome. [x. Deff. in. 1]

But the ``side'' of the rectangle contained by a rational
straight line and a first apotome is an apotome. [x. 91]

Therefore the `` side `` oiLH, that is, of EC, is an apotome.

But, if the square on HF is greater than the square on
FK by the square on a straight line incommensurable
with HF,

while the whole FH is commensurable in length with the
rational straight line FG set out,

KH is a fourth apotome. [x. Deff. in. 4]

But the ``side'' of the rectangle contained by a rational

straight line and a fourth apotome is minor. [x. 94]
\end{proof}

\begin{notes}

A rational area being of the form kp*, and a medial area of the form
Jk.p't the problem is to classify

Jp 1 - ,J\ . p*
according to the different possible relations between k, K
Suppose that <ru = Ap 1 ,

<rv = jk . p'.
Since tru is rational and of medial,
u is rational and « a-,
while v is rational and v tr.

Therefore u « v ;

thus u, v are rational and au
whence (u — v) is an apotome.

The possibilities are now as follows.
(1) «y«*-* « u,
(a) Vw* — 1? -j u.
In both cases u « o-,
so that (« — v) is either (1) a. first apotome,
or (2) a fourth apotome.
In case (r) Joift — v) is an apotome [x. 91],
but in case (2) V<r (u - v) is a m/ztur irrational straight line [x. 94].

\end{notes}

\end{proposition}

\begin{proposition}
\label{propX_109}

\begin{statement}
If from a medial area a rational area be subtracted, there
arise two other irrational straight lines, either a first apotome
of a medial straight line or a straight line which produces with
a rational area a medial whole.
\end{statement}

\begin{proof}

For from the medial area BC let the rational area BD be
subtracted.

I say that the ``side'' of the remainder EC becomes one
of two irrational straight lines, either a first apotome of a
medial straight line or a straight line which produces with a
rational area a medial whole.

A 6

F K H

Q L

For let a rational straight line FG be set out,
and let the areas be similarly applied,

It follows then that FH is rational and incommensurable
in length with FG,

while KF is rational and commensurable in length with FG ;
therefore FH, FK are rational straight lines commensurable
in square only ; [x. 13]

therefore KH is an apotome, and FK the annex to it. [x. 73]

Now the square on HF is greater than the square on FK
either by the square on a straight line commensurable with
HF or by the square on a straight line incommensurable
with it.

H then the square on HF is greater than the square on
FK by the square on a straight line commensurable with HF,
while the annex FK is commensurable in length with the
rational straight line FG set out,
KH is a second apotome. [x. Deff. in. a]

But FG is rational ;
so that the `` side `` of LH, that is, of EC, is a first apotome of
a medial straight line. [x. 9a]

But, if the square on HF is greater than the square on
FK by the square on a straight line incommensurable with HF,
while the annex FK is commensurable in length with the
rational straight line FG set out,

KH is a fifth apotome ; [x. Deff. in. 5]

so that the ``side'' of EC is a straight line which produces
with a rational area a medial whole. [x. 95]
\end{proof}

\begin{notes}

In this case we have to classify
Suppose that av- Jk.ft,

[x. ro9, no

Thus, au being medial and av rational,
u is rational and <j <r,
while v is rational and « tr.

Thus, as before, a,  are rational and «-,
so that (w — v) is an apotome.
Now either

(r) Vw* —  * a,
or (2) i/u'-tP  a,

while in both cases » is commensurable with <r.
Therefore (a — v) is either (1) a second apotome,
or (z) a fifth apotome,
and hence in case (1) vcr(« - v) is the. first apotome of a medial straight line,

[* 9*]
and in case (a) /<r' (a — v) is the ``side'' of a medial, minus a rational, area.

\end{notes}

\end{proposition}

\begin{proposition}
\label{propX_110}

\begin{statement}
If from a medial area there be subtracted a medial area
incommensurable with the whole, the two remaining irrational
straight lines arise, either a second apotome of a medial straight
line or a straight line which produces with a medial area a
medial whole.
\end{statement}

\begin{proof}

For, as in the foregoing figures, let there be subtracted
from the medial area BC the medial area BD incommensur-
able with the whole ;

I say that the `` side `` of EC is one of two irrational straight
lines, either a second apotome of a medial straight line or a
straight line which produces with a medial area a medial whole.

For, since each of the rectangles BC, BD is medial,
and BC is incommensurable with BD,

it follows that each of the straight lines FH, FK will be
rational and incommensurable in length with FG. [x. m]

And, since BC is incommensurable with BD,
that is, GH with GK,
HFh also incommensurable with FK; [vi. 1, x. n]

therefore FH, FK are rational straight lines commensurable
in square only ;

therefore KH is an apotome. |x- 73]

If then the square on FH is greater than the square on
FK by the square on a straight line commensurable with FH,

while neither of the straight lines FH, FK is commensurable
in length with the rational straight line FG set out,

KH is a third apotome. [x. Deff. m. 3]

But KL is rational,

and the rectangle contained by a rational straight line and a
third apotome is irrational,

and the ``side'' of it is irrational, and is called a second
apotome of a medial straight line ; [x. 93]

so that the ``side `` of LH, that is, of EC, is a second apotome
of a medial straight line.

But, if the square on FH is greater than the square on
FK by the square on a straight line incommensurable with FH,

while neither of the straight lines HF, FK is commensurable
in length with FG,

KH is a sixth apotome. [x. Deff. iti. 6]

But the `` side `` of the rectangle contained by a rational
straight line and a sixth apotome is a straight line which
produces with a medial area a medial whole. [x. 96]

Therefore the `` side `` of LH, that is, of EC, is a straight
line which produces with a medial area a medial whole.
\end{proof}

\begin{notes}

We have to classify <Jji . p* — /A . p

where Jk . p'' is incommensurable with J\ , p*.

Put <TV=Jk. p>,

<r» = JK . p>.

Then u is rational and v <r,
v is rational and v <r,
and u yd v.

Therefore w, v are rational and «-.
so that (u - v) is an apotome.

Now either

(l) 'Jtt t -V l « u,
or (2) */w a - w 5 u u,
while in both cases both u and z> are v o-.
In case (1) (u~v) is a third apotome
and in case (2) (« - ti) is a j/;e/ apotome,

sc that i/ir(B-0 is cither (1) a second apotome 0/ a medial straight Hne [x. 93],
or (2) a ``side'' of the difference between two medial areas [x. 96]!

\end{notes}

\end{proposition}

\begin{proposition}
\label{propX_111}

\begin{statement}
The apotome is not the same with the binomial straight line.
\end{statement}

\begin{proof}

Let AB be an apotome ;
I say that AB is not the same with the
binomial straight line

For, if possible, let it be so ;
let a rational straight line DC be set out,
and tc CD let there be applied the
rectangle CE equal to the square on
AB and producing DE as breadth.

Then, since AB is an apotome,
DE is a first apotome. [x. 97]

Let EF be the annex to it ;
therefore DF, FE are rational straight
lines commensurable in square only;

the square on DF is greater than the square on FE by the
square on a straight line commensurable with DF,
and DF is commensurable in length with the rational straight
line DC set out. [x. Deff. in. 1]

Again, since AB is binomial,
therefore DE is a first binomial straight line. [x. 60]

Let it be divided into its terms at G,
and let DG be the greater term ;

therefore DG, GE are rational straight lines commensurable
in square only,

the square on DG is greater than the square on GE by the
square on a straight line commensurable with DG, and the
greater term DG is commensurable in length with the rational
straight line DC set out. - [s. Deff. h. i]

Therefore DF is also commensurable in length with DG ;

[x. ..]

therefore the remainder GF is also commensurable in length
with DF. [x. 15]

But DF is incommensurable in length with EF
therefore FG is also incommensurable in length with EF. [x. 13]

Therefore GF, FE are rational straight lines commensur-
able in square only ;
therefore EG is an apotome. [x. 73]

But it is also rational :
which is impossible.

Therefore the apotome is not the same with the binomial
straight line.
\end{proof}

\begin{notes}

This proposition proves the equivalent of the fact that
Jx + Jy cannot be equal to *Jx - Jy', and
x + Jy cannot be equal to x' - jy.

We should prove these results by squaring the respective expressions; and
Euclid's procedure corresponds to this exactly.

He has to prove that

p + i.p cannot be equal to p'—J.p'.

For, if possible, let this be so.

Take the straight lines W*-?? , <?!: J±llZ .
these must be equal, and therefore

£(l+k+3j/b) = £(t + A- a Vfc) (i)-

  1

Now — (1 + k\ — (1 + X) are rational and rt ;

therefore [ p - (1 + A) - t (, + k)\ ~ £ (1 + A)

w t tl :,

And, since both sides are rational, it follows that

-(i+X)--(i+i)|--.2 is an apotome.

But, by (i), this expression is equal to — . z Jk, which is rational.

Hence an apotome, which is irrational, is also rational:
which is impossible.

This proposition is the connecting link which enables Euclid to prove that
all the compound irrationals with positive signs above discussed are different
from all the corresponding compound irrationals with negative signs, while the
two sets are all different from one another and from the medial straight line.
The recapitulation following makes this clear.

\end{notes}

The apotome and the irrational straight lines following it
are neither the same with the medial straight line nor with one
another.

For the square on a medial straight line, if applied to a
rational straight line, produces as breadth a straight line
rational and incommensurable in length with that to which it
is applied, [x. 33]

while the square on an apotome, if applied to a rational
straight line, produces as breadth a first apotome, [x. 97]

the square on a first apotome of a medial straight line, if
applied to a rational straight line, produces as breadth a
second apotome, [x. 98]

the square on a second apotome of a medial straight line, if
applied to a rational straight line, produces as breadth a third
apotome, [x. 99]

the square on a minor straight line, if applied to a rational
straight line, produces as breadth a fourth apotome, [x. 100]
the square on the straight line which produces with a rational
area a medial whole, if applied to a rational straight line,
produces as breadth a fifth apotome, [x. 101]

and the square on the straight line which produces with a
medial area a medial whole, if applied to a rational straight
line, produces as breadth a sixth apotome. [x. 101]

Since then the said breadths differ from the first and from
one another, from the first because it is rational, and from one
another since they are not the same in order,
it is clear that the irrational straight lines themselves also
differ from one another.

And, since the apotome has been proved not to be the
same as the binomial straight line, - [x. 1 j 1]

but, if applied to a rational straight line, the straight lines

following the apotome produce, as breadtKs, each according
to its own order, apotomes, and those following the binomial
straight line themselves also, according to their order, produce
the binomials as breadths,

therefore those following the apotome are different, and those
following the binomial straight line are different, so that there
are, in order, thirteen irrational straight lines in all,

Medial,

Binomial,

First bimedial,

Second bimedial,

Major,

``Side'' of a rational plus a medial area,

`` Side `` of the sum of two medial areas,

Apotome,

First apotome of a medial straight line,

Second apotome of a medial straight line,

Minor,

Producing with a rational area a medial whole,

Producing with a medial area a medial whole.

\end{proposition}

\begin{proposition}
\label{propX_112}

\begin{statement}
The square on a rational straight lint applied to the
binomial straight line produces as breadth an apotome the
terms of which are commensurable with the terms of the bi-
nomial and moreover in the same ratio ; and further the
apotome so arising will have the same order as the binomial
straight line.
\end{statement}

\begin{proof}

Let A be a rational straight line,
let BC be a binomial, and let DC be its greater term ;
let the rectangle BC, EF be equal to the square on A ;

I say that EFis an apotome the terms of which are commen-
surable with CD, DB, and in the same ratio, and further EF
will have the same order as BC.

For again let the rectangle BD, G be equal to the square
on A.

Since then the rectangle BC, EF is equal to the rectangle
BD, G,
therefore, as CB is to BD, so is G to EF. [«. r6]

But CB is greater than BD ;
therefore G is also greater than EF. [v. 16, v. 14]

Let EH be equal to G ;

therefore, as CB is to BD, so is HE to EF;

therefore, separando, as CD is to BD, so is HF to FE. [v. 17]

Let it be contrived that, as HF is to FE, so is FK
to KE;

therefore also the whole HK is to the whole KF as FK
is to KE ;

for, as one of the antecedents is to one of the consequents, so
are all the antecedents to all the consequents. [v. 12]

But, as FK is to KE, so is CD to DB ; [v. 1 1]

therefore also, as HK is to KF, so is CD to DB. [id.]

But the square on CD is commensurable with the square
on DB; [x, 36]

therefore the square on HK is also commensurable with the
square on KF. [vi. aa, x. 11]

And, as the square on HK is to the square on KF, so is
HK to KE, since the three straight lines HK, KF, KE are
proportional. (v. Def. 9]

Therefore HK is commensurable in length with KE,

so that HE is also commensurable in length with EK. [x. 15]

Now, since the square on A is equal to the rectangle
EH,BD,

while the square on A is rational,

therefore the rectangle EH, BD is also rational.

And it is applied to the rational straight line BD ;

therefore EH is rational and commensurable in length
with BD ; [x. 20]

so that EK, being commensurable with it, is also rational and
commensurable in length with BD.

Since, then, as CD is to DB, so is FK to KE,

while CD, DB are straight lines commensurable in square
only,

therefore FK, KE are also commensurable in square only,

[x. 11]

But KE is rational ;

therefore FK is also rational.

Therefore FK, KE are rational straight lines commen-
surable in square only;

therefore EF is an apotome. [x. 73]

Now the square on CD is greater than the square on DB
either by the square on a straight line commensurable with
CD or by the square on a straight line incommensurable
with it.

If then the square on CD is greater than the square on
DB by the square on a straight line commensurable with CD,
the square on FK is also greater than the square on KE by
the square on a straight line commensurable with FK. [x. 14]

And, if CD is commensurable in length with the rational
straight line set out,

so also is FK; [x. u, 13]

if BD is so commensurable,

so also is KE ; [x. ia]

but, if neither of the straight lines CD, DB is so commensur-
able,
neither of the straight lines FK, KE is so.

But, if the square on CD is greater than the square on
DB by the square on a straight line incommensurable
with CD,

the square on FK is also greater than the square on KE by
the square on a straight line incommensurable with FK. [x. 14]

And, if CD is commensurable with the rational straight
line set out,
so also is FK;
if BD is so commensurable,
so also is KE ;

but, if neither of the straight lines CD, DJB is so commensur-
able,

neither of the straight lines FK t KE is so ;
so that FE is an apotome, the terms of which FK, KE are
commensurable with the terms CD, DB of the binomial
straight line and in the same ratio, and it has the same order
asC
\end{proof}

\begin{notes}

Heiberg considers that this proposition and the succeeding ones are inter-
polated, though the interpolation must have taken place before Theon's time.
His argument is that x. ti> — -115 are nowhere used, but that x. m rounds
off the complete discussion of the 1 3 irrationals (as indicated in the recapitu-
lation), thereby giving what was necessary for use in connexion with the
investigation of the five regular solids. For besides x. 73 (used in xm. 6, n)
x. 94 and 97 are used in xm. n, 6 respectively; and Euclid could not have
stopped at x. 97 without leaving the discussion of irrationals imperfect, for
X. 98 — ios are closely connected with x. 97,and x. 103 — 1 1 1 add, as it were,
the coping-stone to the whole doctrine. On the other hand, X. n a— 115 are
not connected with the rest of the treatise on the 13 irrationals and are not
used in the stereometric books. They are rather the germ of a new study and
a more abstruse investigation of irrationals in thansehts. Prop. 115 in
particular extends the number of the different kinds of irrationals. As
however x. n> — 115 are old and serviceable theorems, Heiberg thinks that,
though Euclid did not give them, they may have been taken from Apollonius.

I will only point out what seems to me open to doubt in the above, namely
that x. us — 1 14 (excluding 115) are not connected with the rest of the
exposition of the 13 irrationals. It seems tc me that they are so connected.
x. 1 1 1 has shown us that a binomial straight line cannot also be an apotome.
But X. ii2 — 1 14 show us how either of them ``can be used to rationalise the other,
thus giving what is surely an important relation between them.

x. nz is the equivalent of rationalising the denominators of the fractions
<* e*

JA+JB' a+JB'
by multiplying numerator and denominator by J A - JB and a — ,JB
respectively.

Euclid proves that — ,, — = p - k . p (h < 1 ), and his method enables

us to see that \ = <r*/(p* - Ap 1 ).

The proof is a remarkable instance of the dexterity of the Greeks in using
geometry as the equivalent of our algebra. Like so many proofs in Archimedes
and Apollonius, it leaves us completely in the dark as to how it was evolved.
That the Greeks must have had some analytical method which suggested -the
steps of such proofs seems certain ; but what it was must remain apparently
an insoluble mystery.

I will reproduce by means of algebraical symbols the exact course of
Euclid's proof.

He has to prove that fi — is an apotome related in a certain way to

p + J* . p

the binomial straight litiu p + Jk . p. If « be the straight line required,
(u + w) - w is shown to be an apotome of the kind described, where w is
determined in the following manner.

We have (p + Ji.p) u = tr*= Ji .p. x, say, 1
whence x > u. \ (i).

Let x = a + v. I

Then (p + Ji . p) : Jk . p = (» + *>): a,

and hence p : Jk.p = v : a (2).

  Let w be taken such that

v \ u = (a + w) : to (3).

Thus v : a = (a + y + to) \ (k + to) (4),

and therefore p : Ji . p = (a + v + to) : (a + to).

From the last proportion,

(a + v + to)* « (a + to)*,
and, from the two preceding, (« + to) is a mean proportional between
(u + v +  to), to, so that

(a + v + to)* : (a + to)* = (a + » +  to) : w.

Therefore (a +  y + to) « to,

whence (a + 1*) « «/.

Now Jk . p (a + v) = a 1 , which is rational ;
therefore (a + v) is rational and `` Jk . p;
hence to is also rational and rt Jk-p  (S)-

Next, by (a), (3), since p, Jk . p are «- ,
(a + w) «- to,
and to is rational ,

therefore (a + to) is rational,

and (« + to), to are rational and is* .

Hence (a + to) - to is an apotome.

Now either (I) Jp*-Ap' r> p,

or (II) 4f?-* - p.

In case (I) V(a + iv)*-v? « (« + a>), [(z), (3) and X. 14]

and in case (II) V(a + to)* - to 1  (a + to). [«£]

Then, since [(5)] w n Jk.p,
by X. 1 1 and (2), (3), (a + to) - p (6).

[This step is omitted in Euclid, but the result is assumed.]

If therefore p « cr, (a +to) `` or;
if JA.p « cr, hi «<r; [(5)]

and, if neither p nor Ji . p is n <r, neither (a + to) nor to will be `` <r.

Thus the order of the apotome (a + to) - w is the same as tht of the
binomial straight line p+ Jk.p; while [(2), (3)] the terms are proportional
and [(5), (6)] commensurable respectively.

We find (u + w),w algebraically thus.
u

By (i),

and. by (a), (3),
whence

Thus

Therefore

p * Jk . p '
« + w  p

U.Jk.p

p-Jk.p

U + «' = If

If'.

-

c

P 5

-V

I

<r>

  P

re .

7*~

P*-

``V

(« + «:)- W = a'' . — 3

p-Jk.p

[X. 112, 113

p'-V

\end{notes}

\end{proposition}

\begin{proposition}
\label{propX_113}

\begin{statement}
7 square on a rational straight line, if applied to an
apotome, produces us- breadth the binomial straight line the
terms of which are commensurable with the terms of the
apotome and in the same ratio ; and further the binomial
so arising has the same order as the apotome.
\end{statement}

\begin{proof}

Let A be a rational straight line and BD an apotome,
and let the rectangle BD, KH be equal to
the square on A, so that the square on the
rational straight line A when applied to the
apotome BD produces KH as breadth ;
I say that KH is a binomial straight line the
terms of which are commensurable with the
terms of BD and in the same ratio ; and
further KH has the same order as BD.

For let DC be the annex to BD ;

therefore BC, CD are rational straight lines commensurable
in square only. [x. 73]

Let the rectangle BC, G be also equal to the square on A .
But the square on A is rational ;

therefore the rectangle BC, G is also rational.

And it has been applied to the rational straight line BC ;

therefore G is rational and commensurable in length with BC

[x. so]

Since now the rectangle BC, G is equal to the rectangle
BD, KH,

therefore, proportionally, as CB is to BD, so is KH to G.

[VI. 16]

But BC is greater than BD ;
therefore KH is also greater than G. [v. 16, v. 14]

Let KE be made equal to G ;
therefore KE is commensurable in length with BC.

And since, as CB is to BD, so is HK to KE,
therefore, convertendo, as BC is to CD, so is .A'/f to HE.

[v. 19, Por.]

Let it be contrived that, as KH is to HE, so is HE
to EE;

therefore also the remainder KE is to EH as KH is to HE,
that is, as .ffC is to CD. [v. 19]

But BC, CD are commensurable in square only ;

therefore KE, EH are also commensurable in square only.

[x. 11]
And since, as KH is to HE, so is KE to EH,
while, as KH is to HE, so is HE to /vfi'',
therefore also, as KE is to ., so is HE to /-fi'', [v. it]

so that also, as the first is to the third, so is the square on the
first to the square on the second ; [v. Def. 9]

therefore also, as KE is to EE, so is the square on KE to the
square on EH.

But the square on KE is commensurable with the square
on EH,

for KE, EH are commensurable in square ;

therefore KE is also commensurable in length with EE, [x. n]

so that KE is also commensurable in length with KE. [x. 15]

But KE is rational and commensurable in length with BC;

therefore KE is also rational and commensurable in length
withC. [X. !2]

And, since, as BC Is to CD, so is KE to EH,
alternately, as BC is to KE, so is DC to EH. [v. 16]

But BC is commensurable with KE;
therefore EH is also commensurable in length with CD. fx. n]

But BC, CD are rational straight lines commensurable in
square only ;

therefore KF, FH are also rational straight lines [x. Def. 3]

commensurable in square only ;

therefore KH is binomial. [x. 36]

If now the square on BC is greater than the square on CD
by the square on a straight line commensurable with BC,
the square on KF will also be greater than the square on FH
by the square on a straight line commensurable with KF, [x 14]

And, if BC is commensurable in length with the rational
straight line set out,
so also is KF ;

if CD is commensurable in length with the rational straight
line set out,
so also is FH,

but, if neither of the straight lines BC, CD,
then neither of the straight lines KF, FH.

But, if the square on BC is greater than the square on CD
by the square on a straight line incommensurable with BC,
the square on KF is also greater than the square on FH by
the square on a straight line incommensurable with KF. [x. 14]

And, if BC is commensurable with the rational straight
line set out,
so also is KF ;
if CD is so commensurable,
so also is FH ;

but, if neither of the straight lines BC, CD,
then neither of the straight lines KF, FH.

Therefore KH is a binomial straight line, die terms of
which KF, FH are commensurable with the terms BC, CD of
the apotome and in the same ratio,
and further KH has the same order as BD.
\end{proof}

\begin{notes}

This proposition, which is companion to the preceding, gives us the equiva-
lent of the rationalisation of the denominator of

Euclid (or the writer) proves that
i
Ji —= i Xf>*XJA.p, (t<t)

and his method enables us to see that X = <r*/(p* - kp 1 ).

Let 7T~ n = * '

p-Jk.p

and it is proved that u is the binomial straight line (» - «') + w, where w is

determined as shown below.

u (p -. Jk . p) = <r* = px, say,

whence p : (p — JA ,p) = u  x (1),

so that x  -. u.

Let then x = a - v.

Since (u-v)p = a 1 , a rational area,

(u-v) is rational and `` p (*).

And [(1)] p:(p-JA.p) = u: (w-v),

so that, cenverfendoy p : Jk . p = u : v.

Suppose that « : v = w : (v - w),

so that [v. 19] (« - w) : w = u : v = tu : (v - w).

Thus, w being a mean proportional between (u - w), (f - w),
(» — wf : w 1 = (a — w) : (v - w).

But (u — ntf : it?= li 1 v

-«**V (3).

so that (u — 10)* *» w*.

Therefore (« - (c) « (w - a>)

`` ((» - w) - (v - w)
*(u- v).

Therefore [(2)] (u-w) is rational and « p (4)-

And, since p: Jk. p = (u-w):v>,

w is rational and.p (5).

Hence [(4), (5)] (u-w), w are rational and «-,
so that (k - w) + w is a binomial straight line.

Now either (I) <Jp* - kp* « p,

or (II) Vp* - V `` P-

Incase (I) /(« - a/)* - w'* « (« — a>),

and in case (II) V(« - w)* - «/'' « (»-«>). [(3) and x. 14]

And, if p `` tr, (u - w) « o- ; [(4)]

if ,/A . p `` er, a/ `` <r ; [5]

while, if neither p nor ,/i. p is « 0-, neither (u-w) nor w is `` a.

Hence (u-w) + if is a binomial straight line of the same order as the

apotome p — J A . />, its terms are proportional to those of the -''wtome [(3)],

and commensurable with them respectively [(4), (5)].

To find (a - w), w algebraically we have

<r*

tt -p-Jk.p'

u -if p

w ~ Jk.p'

From the latter w - —'-,.'-

P + Jk . P

a 1 . Jk ,p

Thus u - w - w   -n = .

Jk p-

Therefore (« - »/) + w = v* . p -. \ P

p*-v

\end{notes}

\end{proposition}

\begin{proposition}
\label{propX_114}

\begin{statement}
If an area be contained by an apotome and the binomial
straight line the terms of which are commensurable with the
terms of the apotome and in the same ratio, the `` side `` of the
area is rational.
\end{statement}

\begin{proof}

For let an area, the rectangle AB, CD, be contained by
the apotome AB and the binomial
straight line CD,

A B F

and let CE be the greater term of

the latter; C E p

let the terms CE, ED of the o

binomial straight Hne be commen- H
surable with the terms AF, FB of

the apotome and in the same ratio; * 1 *?

and let the ``side'' of the rectangle
AB, CBbeG;

I say that G is rational.

For let a rational straight line H be set out,
and to CD let there be applied a rectangle equal to the square
on If''and producing KL as breadth.

Therefore KL is an apotome.

Let its terms be KM, ML commensurable with the terms
CE, ED of the binomial straight line and in the same ratio.

[x. ,„]

But CE, ED are also commensurable with AF, FB and in
the same ratio ;

therefore, as AF is to FB, so is KM to ML.

Therefore, alternately, as AF is to KM, so is BF to L M ;
therefore also the remainder AB is to the remainder KL as
AF is to KM. [v. 19]

But AFis commensurable with KM; [x. ra]

therefore AB is also commensurable with KL. [x. n]

And, as AB is to KL, so is the rectangle CD, AB to the
rectangle CD, KL ; [vi. 1]

therefore the rectangle CD, AB is also commensurable with
the rectangle CD, KL. [x. u]

But the rectangle CD, KL is equal to the square on H ;
therefore the rectangle CD, AB is commensurable with the
square on H.

But the square on G is equal to the rectangle CD, AB ;
therefore the square on G is commensurable with the square
on H.

But the square on H is rational ;

therefore the square on G is also rational ;

therefore G is rational.

And it is the ``side'' of the rectangle CD, AB.
Therefore etc.

Porism. And it is made manifest to us by this also that
it is possible for a rational area to be contained by irrational
straight lines.
\end{proof}

\begin{notes}

This theorem is equivalent to the proof of the fact that

J(JA - JB) OQA + A JB) = Jk(A-B),

W»d JJa ~ JB) (ka + A JB) = Jk(B).

The result of the theorem x. 1 1 2 is used for the purpose thus.

We have to prove that

 J(j>-Jk.p)(p + Jk. P )
is rational.

By X. 11a we have, if a is a rational straight line,

Xp + A Ji . p

»--*>- *V*-P (0-

Now p : A'p = Jk . p : X' Jk . p = (p - Jk . p) : (A'p - A' Jk . p),
so that (p - Jk . p) n (A'p - X' Jk . p).

Multiplying each by (Xp + A Jk . p), we have

(p-Jk.p)()* + Jk.p)~(p + XJk.p)('p-k , Jk.p)
- <r*, by <r).

That is, (p- Jk .p)(p + Jk.p) is a rational area,
and therefore V(p - Jk. p) (Ap + kjk.p) is rational.

\end{notes}

\end{proposition}

\begin{proposition}
\label{propX_115}

\begin{statement}
From a medial straight line there arise irrational straight
lines infinite in number, and none of them is the same as any
of the preceding.
\end{statement}

\begin{proof}

Let /4bea medial straight line ;
I say that from A there arise

irrational straight lines infinite in A ,

number, and none of them is the
same as any of the preceding.

Let a rational straight line B c

be set out,

and let the square on C be equal

to the rectangle B, A ;

therefore C is irrational ; [x. Def. 4]

for that which is contained by an irrational and a rational

straight line is irrational. [deduction from x. 20]

And it is not the same with any of the preceding ;
for the square on none of the preceding, if applied to a rational
straight line produces as breadth a medial straight line.

Again, let the square on D be equal to the rectangle B, C;
therefore the square on D is irrational. [deduction from x. ao]

Therefore D is irrational ; [x, Def. 4]

and it is not the same with any of the preceding, for the
square on none of the preceding, if applied to a rational
straight line, produces C as breadth.

Similarly, if this arrangement proceeds ad infinitum, it
is manifest that from the medial straight line there arise
irrational straight lines infinite in number, and none is the
same with any of the preceding.
\end{proof}

\begin{notes}

Heiberg is clearly right in holding that this proposition, at all events, is
alien to the general scope of Book x, and is therefore probably an interpola-
tion, made however before Theon's time. It is of the same character as a
scholium at the end of the Book, which is (along with the interpolated proposi-
tion proving, in two ways, the incommensurability of the diagonal of a square
with its side) relegated by August as well as Heiberg to an Appendix.

The proposition amounts to this.

The straight line k*p being medial, if o- be a rational straight line, v pv
is a new irrational straight line. So is the mean proportional between this
and another rational straight line <r\ and so on indefinitely.

\end{notes}

\end{proposition}

\chapter*{Ancient Extensions of the Theory of Book X}

From the hints given by the author of the commentary found in Arabic
by Woepcke (cf. pp. 3 — 4 above) it would seem probable that Apollonius'
extensions of the theory of irrationals took two directions : (1 ) generalising
the medial straight line of Euclid, and (2) forming compound irrationals by the
addition and subtraction of more than two terms of the sort composing the
binomials, apotomes, etc The commentator writes (Woepcke's article, pp. 694
sqq.):

``It is also necessary that we should know that, not only when we join
together two straight lines rational and commensurable in square do we obtain
the binomial straight line, but three or four lines produce in an analogous
manner the same thing. In the first case, we obtain the trinomial straight
line, since the whole line is irrational ; and in the second case we obtain the
quadrinomial, and so im si infinitum. The proof of the (irrationality of the)
line composed of three lines rational and commensurable in square is exactly
the same as the proof relating to the combination of two lines.

`` But we must start afresh and remark that not only can we take one sole
medial line between two lines commensurable in square, but we can take three
or four of them and so on ad infinitum, since we can take, between any two
given straight lines, as many lines as we wish in continued proportion.

`` Likewise, in the lines formed by addition not only can we construct the
binomial straight line, but we can also construct the trinomial, as well as the
first and second tri medial ; and, further, the line composed of three straight
lines incommensurable in square and such that the one of them gives with
each of the two others a sum of squares (which is) rational, while the rectangle
contained by the two lines is medial, so that there results a major (irrational)
composed of three lines.

``And, in an analogous manner, we obtain the straight line which is the
' side of a rational plus a medial area, composed of three straight lines, and,
likewise, that which is the side of (the sum of) two medials. ``

The generalisation of the medial is apparently after the following manner.
Let x, y be two straight lines rational and commensurable in square only and
suppose that m means are interposed, so that

* : x, - Jt) : x, = x, : x = . . . = x m - t ; x m -x m : y.

We easily derive herefrom — = I — I ,
x r xj

~ w

x

y

*5« BOOK X

and hence *,' = x T . x r J ',

so that (jf r .* r - , ) l `` + 1 = (>.jc'' , )'' 1

and therefore x,™* 1 = *'' -F+1 .y,

or AV = (*''- r + '>T+ r ,

which is the generalised medial.

We now pass to the trinomial etc., with the commentator's further remarks
about them.

(i) Tht trinomial. ``Suppose three rational straight lines commensurable in
square only. The line composed of two of these lines, that is, the binomial
straight line, is irrational, and, in consequence, the area contained by this line
and the remaining line is irrational, and, likewise, the double of the area
contained by these two lines will be irrational. Thus the square on the
whole line composed of three lines is irrational and consequently the line is
irrational, and it is called a trinomial straight line.''

It is easy to see that this ``proof'' is not conclusive as stated. Nor does
Woepcke seem to show how the proposition can be proved on Euclidean
lines. But I think it would be somewhat as follows.

Suppose x, y, z to be rational and '*- .

Then «*, y*, «* are rational, and zyz, zzx, zxy are all medial.

First, (zyz + zzx + 2xy) cannot be rational.

For suppose this sum equal to a rational area, say a 3 .

Since zyz +  zzx + zxy = ff 1 ,

zzx + zxy = <r* - zy z,

or the sum of two medial areas incommensurable with one another is equal to
the difference between a rational area and a medial area.

But the `` side'' of the sum of the two medial areas must [x. 72] be one of
two irrationals with a positive sign ; and the `` side `` of the difference between a
rational area and a medial area must [x. 108] be one of two irrationals with a
negative sign.

And the first `` side `` cannot be the same as the second [x. 1 1 1 and ex-
planation following].

Therefore zzx + zxy  cr'' — zyz,

and zyz + zzx + zxy is consequently irrational.

Therefore (x 1 +>'' + «*) u (zyz + zex + zxy),

whence (x+y + *)* ~ (*  +y* +  3'),

so that (x + y + zf, and therefore also (x +y + z), is irrational.

The commentator goes on :

`` And, if we have four lines commensurable in square, as we have said, the
procedure will be exactly the same ; and we shall treat the succeeding lines in
an analogous manner.''

Without speculating further as to how the extension was made to the
quadrinsmial etc., we may suppose with Woepcke that Apollonius probably
investigated the multinomial

p+ Jk. p + ,J.p+ Jp.p + ...

ANCIENT EXTENSIONS »S7

(2) The first trimedial straight line.

The commentator here says : ``Suppose we have three medial lines com-
mensurable in square [only], one of which contains with each of the two others
a rational rectangle ; then the straight line composed of the two lines is
irrational and is called the first bi medial ; the remaining line is medial, and
the area contained by these two lines is irrational. Consequently the square
on the whole line is irrational.''

To begin with, the conditions here given are incompatible. If x, y, z be
medial straight lines such that xy, xz are both rational,

y : z = xy : xz = m : n,

and y, i are commensurable in length and not in square only.

Hence it seems that we must, with Woepcke, understand `` three medial

straight lines such that one is commensurable with each of the other two in

square only and makes with it a rational rectangle.''
If x, y, i be the three medial straight lines,

so that (?? +y + s') is medial.

Also we have txy, 2x1 both rational and 2ys medial.

Now (x' + y 1 + sfj + tyt + ixy + txz cannot be rational, for, if it were, the
sum of two medial areas, (x i +y t + «*), ays, would be rational: which is im-
possible, [Cf, x. 7 a.]

Hence (x +y + z) is irrational.

(3) The second trimedial straight line.

Suppose x, y, z to be medial straight lines commensurable in square only
and containing with each other medial rectangles.

Then (x 1 +v* + «) « x*, and is medial.

Also tyz, 2zx, txy are all medial areas.

To prove the irrationality in this case I presume that the metnod would
be like that of x. 38 about the second bimedial.
Suppose o- to be a rational straight line and let

(x' + f + z') = <rt
tyz = ITU
22X = <rv
3xy = aw ]

Here, since, e.g., xz : xy - v : w,

or ssy = v:w,

and similarly x :z -w :v,

u, v, w are tommensurable in square only.

Also, since (x 1 + >* + 1*) `` x*

w xy
t is incommensurable with w.

aj8 BOOK X

Similarly / is incommensurable with u, v.

But t, u, v, to are all rational and ``- er.

Therefore (t + u + v+w) is a quadrinomial and therefore irrational.
Therefore <r (/ + « + v+ w), or ( +,y + *)*, is irrational,
whence (x +y + *) is irrational.

(4) The major made up of three straight lines.

The commentator describes this as ``the line composed of three straight
lines incommensurable in square and. such that one of them gives with each
of the other two a sum of squares (which is) rational, while the rectangle
contained by the two lines is medial.''

If *, y, x are the three straight lines, this would indicate

(jc'+y) rational,

(jc 1 +  **) rational,

tyx medial.

Woepcke points out (pp. 696— 8, note) the difficulties connected with this
supposition or the supposition of

(«*+) rational,
(s? + e 1 ) rational,
ixy (or ixt) medial,
and concludes that what is meant is the supposition

(«* +jP) rational
xy medial
xz medial ,
(though the text is against this).

The assumption of (j.- 1 +y') and (a + **) being concurrently rational is
certainly further removed from Euclid, for x. 33 only enables us to find one
pair of lines having the property, as x, y.

But we will not pursue these speculations further-
As regards further irrationals formed by subtraction the commentator
writes as follows.

`` Again, it is not necessary that, in the irrational straight lines formed by
means of subtraction, we should confine ourselves to making one subtraction
only, so as to obtain the apotome, or the first apotome of the medial, or the
second apotome of the medial, or the minor, or the straight line which
produces with a rational area a medial whole, or that which produces with a
medial area a medial whole ; but we shall be able here to make two or three
or four subtractions.

``When we do that, we show in manner analogous to the foregoing that
the lines which remain are irrational and that each of them is one of the lines
formed by subtraction. That is to say that, if from a rational line we cut off
another rational line commensurable with the whole line in square, we obtain,
for remainder, an apotome ; and, if we subtract from this line (which is)
cut off and rational — that which Euclid calls the antux (irpowofj/uifowra) —
another rational line which is commensurable with it in square, we obtain, as
the remainder, an apotome ; likewise, if we cut off from the rational line cut
off from this line (i.e. the annex of the apotome last arrived at) another line
which is commensurable with it in square, the remainder is an apotome. The
same thing occurs in the subtraction of the other lines.''

As Woepcke remarks, the idea is the formation of the successive apotomes
,/a - Ji, jb — Jc, Je— Jd t etc. We should naturally have expected to see
the writer form and discuss the following expressions

\part{Book XI}

\chapter*{Definitions}

\begin{enumerate}

\item\label{def:XI_1} A solid is that which has length, breadth, and
  depth.

\item\label{def:XI_2} An extremity of a solid is a surface.

\item\label{def:XI_3} A straight line is at right angles to a plane,
  when it makes right angles with all the straight lines which meet it
  and are in the plane.

\item\label{def:XI_4} A plane is at right angles to a plane when the
  straight lines drawn, in one of the planes, at right angles to the
  common section of the planes are at right angles to the remaining
  plane.

\item\label{def:XI_5} The inclination of a straight line to a plane
  is, assuming a perpendicular drawn from the extremity of the
  straight line which is elevated above the plane to the plane, and a
  straight line joined from the point thus arising to the extremity of
  the straight line which is in the plane, the angle contained by the
  straight line so drawn and the straight line standing up.

\item\label{def:XI_6} The inclination of a plane to a plane is the
  acute angle contained by the straight lines drawn at right angles to
  the common section at the same point, one in each of the planes.

\item\label{def:XI_7} A plane is said to be similarly inclined to a
  plane as another is to another when the said angles of the inclina-
  tions are equal to one another,

\item\label{def:XI_8} Parallel planes are those which do not meet.

\item\label{def:XI_9} Similar solid figures are those contained by
  similar planes equal in multitude.

\item\label{def:XI_10} Equal and similar solid figures are those con-
  tained by similar planes equal in multitude and in magnitude,

\item\label{def:XI_11} A solid angle is the inclination constituted by
  more than two lines which meet one another and are not in the same
  surface, towards all the lines.

Otherwise : A solid angle is that which is contained by more than two
plane angles which are not in the same plane and are constructed to
one point.

\item\label{def:XI_12} A pyramid is a solid figure, contained by
  planes, which is constructed from one plane to one point.

\item\label{def:XI_13} A prism is a solid figure contained by planes
  two of which, namely those which are opposite, are equal, similar
  and parallel, while the rest are parallelograms.

\item\label{def:XI_14} When, the diameter of a semicircle remaining
  fixed, the semicircle is carried round and restored again to the
  same position from which it began to be moved, the figure so
  comprehended is a sphere.

\item\label{def:XI_15} The axis of the sphere is the straight line
  which remains fixed and about which the semicircle is turned.

\item\label{def:XI_16} The centre of the sphere is the same as that of
  the semicircle,

\item\label{def:XI_17} A diameter of the sphere is any straight line
  drawn through the centre and terminated in both directions by the
  surface of the sphere,

\item\label{def:XI_18} When, one side of those about the right angle
  in a right-angled triangle remaining fixed, the triangle is carried
  round and restored again to the same position from which it began to
  be moved, the figure so comprehended is a cone.

And, if the straight line which remains fixed be equal to
the remaining side about the right angle which is carried
round, the cone will be right-angled ; if less, obtuse-angled ;
and if greater, acute-angled.

\item\label{def:XI_19} The axis of the cone is the straight line which
  remains fixed and about which the triangle is turned.

\item\label{def:XI_20} And the base is the circle described by the
  straight line which is carried round.

\item\label{def:XI_21} When, one side of those about the right angle
  in a rectangular parallelogram remaining fixed, the parallelogram is
  carried round and restored again to the same position from which it
  began to be moved, the figure so comprehended is a cylinder.

\item\label{def:XI_22} The axis of the cylinder is the straight line
  which remains fixed and about which the parallelogram is turned.

\item\label{def:XI_23} And the bases are the circles described by the
  two sides opposite to one another which are carried round.

\item\label{def:XI_24} Similar cones and cylinders are those in which
  the axes and the diameters of the bases are proportional.

\item\label{def:XI_25} A cube is a solid figure contained by six equal
  squares.

\item\label{def:XI_26} An octahedron is a solid figure contained by
  eight equal and equilateral triangles.

\item\label{def:XI_27} An icosahedron is a solid figure contained by
  twenty equal and equilateral triangles.

\item\label{def:XI_28} A dodecahedron is a* solid figure contained by
  twelve equal, equilateral, and equiangular pentagons.

\end{enumerate}

\section*{Definition 1}

 rtp*6v tart to ftyKos *<u irAaTO? ko\ fjafios *X nv -

This definition was evidently traditional, as may be inferred from a number
of passages in Plato and Aristotle. Thus Plato speaks (Sophist, 235 d) of
making an imitation of a model (irapdSiiy(ia) `` in length and breadth and
depth and (Lams, 87 e) of ``the art of measuring length, surface and depth''
as one of three fiadraro. Depth, the third dimension, is used alone as a
description of `` body `` by Aristotle, the term being regarded as connoting the
other two dimensions ; thus (Afttaph. 1020 a 13, 11) ``length is a fine, breadth a
surface, and depth body'' ; `` that which is continuous in one direction is length,
in two directions breadth, and in three depth.'' Similarly Plato (Rep. 528 b, d),
when reconsidering his classification of astronomy as next to (plane) geometry:
``although the science dealing with the additional dimension of depth is next in
order, yet, owing to the fact that it is studied absurdly, I passed it over and
put next to geometry astronomy, the motion of (bodies having) depth.'' In
Aristotle (Topia vi. 5, 142 b 24) we find ``the definition of body, that which
has three dimensions (itafrraatis:) `` ; elsewhere he speaks of it as `` that which
has all the dimensions `` (Dt catlo 1. 1, 268 b 6), ``that which has dimension
every way'' (™ vivrrj Buurrao-ii' cyov, Afttaph. 1066 b 32) etc. In the Physics

(iv. 1, 208 b i3sqq.) he speaks of the `` dimensions `` as six, dividing each of
the three into two opposites, ``up and down, before and behind, right and left,''
though of course, as he explains, these terms are relative.

Heron, as might be expected, combines the two forms of the definition.
41 A solid body is that which has length, breadth, and depth : or that which
possesses the three dimensions.'' (Def. 1 1 .)

Similarly Theon of Smyrna (p. 1 1 1, 19, ed. Hiller) : `` that which is extended
(Suurrarav) and divisible in three directions is solid, having length, breadth
and depth.''

\section*{Definition 2}

In like manner Aristotle says (Metaph. 1066 b 13) that the notion (Ady«)
of body is ``that which is bounded by surfaces'' («rtiri£o« in this case) and
(Metaph. 1060 b 15) ``surfaces (brupviuu) are divisions of bodies.''

So Heron (Def. 11.): `` Every solid is bounded (Tripa.rovro,i) by surfaces, and
is produced when a surface is moved from a forward position in a backward
direction.''

\section*{Definition 3}

EiWtla irpot ttmwm* op6 tarty, Srav irpos irdtras ras durofttras aurs tiQttas
KaX atkraf iv t£ iiti-riStf ApSas ttoijj ywt'a?.

This definition and the next are given almost word for word by Heron
(Def. 115).

That a straight line can be so related to a plane as described in Def. 3 is
established in XI. 4. The fact has been made the basis of a definition of a
plane which is attributed by Crelle to Fourier, and is as follows. `` A plane is
formed by the totality of all the straight lines which, passing through one and
the same point of a straight line in space, stand perpendicular to it,'' Stated
in this form, the definition is open to the objection that the conception of a
right angle, involving the measurement of angles, presupposes a plane, inasmuch
as the measurement of angles depends ultimately upon the superposition of two
planes and their coincidence throughout when two lines in one coincide with
two lines in the other respectively. Cf. my note on 1. Def. 7, Vol. 1. pp. 1 73 — 5.

\section*{Definition 4}

'EtrinSor irpos tnirtSov ip6v itrnv, oto.v at rg itotvj) To/t]j iw twariuv -rpos
6p8a ayofuvai ntfcuu iv ivl tcuv imrtoW T Xoifftp i-rtiiretv wpot 6pas uttv.

Both this definition and Def. 6 use the common section of two planes,
though it is not till xi. 3 that this common section is proved to be a straight
line. The definition however, just like Def, 3, is legitimate, because the object
is to explain the meaning of terms, not to prove anything

The definition of perpendicular planes is made by Legendre a particular
case of Def. 6, the limiting case, namely, where the angle representing the
`` inclination of a plane to a plane `` is a right angle.

\section*{Definition 5}

EAoVn rpot iriwtSov «Aor« ioriv, orav djro toC /Lcrrupav Wparos nj?
tiOtlas iirl TO tvartBov t<atTO<; d)jj, * al * 7 ™ T °' ywoptvav OTjpttiov iwl to iv rid
luilfffg Tripat rffc tiBttas tuStia iwi£tvxg, 17 irfpif0)Urjj ywvia viro r!js ixSturtft

(a! Tljl (KffTWOTJt.

In other words, the inclination of a straight line to a plane is the angle
between the straight line and its projection on the plane. This angle is of
course less than the angle between the straight line and any other straight line
in the plane through the intersection of the straight line and plane ; and the
fact is sometimes made the subject of a proposition in modern text-books. It
is easily proved by means of the propositions xi. 4, 1. 19 and 18.

\section*{Definition 6}

'EffiWSou irpos tiTLTrtBov jcAuto itrrlv t) Trtpifortyy) o£tia. ywta vwb rwv irpos
opOas Tfj Koivfj Top-?/ dyopfywv irpos ty aurtu trqprtiw Iv tKaripw twv iiniriZtov.

When two planes meet in a straight line, they form what is called in
modern text-books a dihedral angle, which is defined as the opening or angular
opening between the two planes. This dihedral angle is an `` angle `` altogether
different in kind from a plane angle, as again it is different from a solid angle
as defined by Euclid (i.e. a trihedral, tetrahedral, etc. angle). Adopting for
the moment Apollonius' conception of an angle as the ``bringing together of a
surface or solid towards one point under a broken line or surface `` (Proclus,
p. 123, 16), we may regard a dihedral angle as the bringing together of the
broken surface formed by two intersecting planes not to a point but to a straight
line, namely the intersection of the planes. Legendre, in a proposition on the
subject, applied provisionally the term corner to describe the dihedral angle
between two planes ; and this would be a better word, I think, than opening
to use in the definition.

The distinct species of `` angle `` which we call dihedral is, however,
Measured by a certain plane angle, namely that which Euclid describes in the
present definition and calls the inclination 0/ a plane to a plane, and which in
some modern text-books is called the plane angle of the dihedral angle.

It is necessary to show that this plane angle is a proper measure of the
dihedral angle, and accordingly Legendre has a proposition to this effect In
order to prove it, it is necessary to show that, given two planes meeting in a
straight line,

( 1) the plane angle in question is the same at all points of the straight line
forming the common section ;

(2) if the dihedral angle between two planes increases or diminishes in a
certain ratio, the plane angle in question will increase or diminish in the same
ratio.

(1) If MAN, MAP be two planes intersecting in MA, and if AN, AP
be drawn in the planes respectively and at right angles to
MA, the angle NAP is the inclination of the plane to the
plane or the plane angle of the dihedral angle.

Let lifC, MB be also drawn in the respective planes
at right angles to MA.

Then since, in the plane MAN, MC and AN are
drawn at right angles to the same straight line MA,
MC, AN are parallel.

For the same reason, MB, AP are parallel.

Therefore [xt 10] the angle BMC is equal to the
angle PAN.

And M may be any point on MA. Therefore the
plane angle described in the definition is the same at all
points of AM.

(a) In the plane A P draw the arc NDP of any circle with centre A,
and draw the radius AD.

Now the planes NAP, CMB, being both at right angles to the straight
line MA, are parallel ; [xi, 14]

therefore the intersections AD, ME of these planes with the plane MAD are
parallel, [xi. 16]

and consequently the angles BME, PAD are equal. [xi. 10]

If now the plane angle NAD were equal to the plane angle DAP, the
dihedral angle NAMD would be equal to the dihedral angle DAMP;
for, if the angle PAD were applied to the angle DAN, AM remaining the
same, the corresponding dihedral angles would coincide.

Successive applications of this result show that, if the angles NAD, DAP
each contain a certain angle a certain number of times, the dihedral angles
NAMD, DAMP will contain the corresponding dihedral angle the same
number of times respectively.

Hence, where the angles NAD, DAP are commensurable, the dihedral
angles corresponding to them are in the same ratio.

Legendre then extends the proof to the case where the plane angles are
incommensurable by reference to an exactly similar extension in his proposition
corresponding to Euclid vi. 1, for which see the note on that proposition.

Modern text-books make the extension by an appeal to limits.

\section*{Definition 7}

*EtVcow Jfpos iriirtiov o/iouut k«A«tii iyvrai not inpov tt/w htpw, S™>
ai ftprjfiivat tiuv ffXurcuiy yfoviat la at dAAAats UKrtp.

\section*{Definition 8}

1 lapitkkrjkfi (irivcAa itm to. ocrvjUTrTorra.

Heron has the same definition of parallel planes (Def. r 1 5). The Greek
word which is translated `` which do not meet `` is daufHrruircL, the term which
has been adopted for the asymptotes of a curve.

\section*{Definition 9}

``O/tota trripia a-ij/iara hart, Ttt uiro 6/Wajy Iwvxvnt ircpic;(d/ii*a to- imp to

\section*{Definition 10}

*I<ra Se *at opota trrtpta. trfund fori ra xnrb afimw nrtircoaiy irfpieofjitva
taav Tip wAiffot tea\ rep fLcyiHtt.

These definitions, the second of which practically only substitutes the
words `` equal and simitar `` for the word `` similar `` in the first, have been the
mark of much criticism.

.Si sn son holds that the equality of solid figures is a thing which ought to be
primed, by the method of superposition, or otherwise, and hence that Def. 10
is not a definition but a theorem which ought not to have been placed among
the definitions. Secondly, he gives an example to show that the definition or
theorem is not universally true. He takes a pyramid and then erects on the
base, on opposite sides of it, two equal pyramids smaller than the first. The
addition and subtraction of these pyramids respectively from the first give two
solid figures which satisfy the definition but are clearly not equal (the smaller
having a re-entrant angle); whence it also appears that two unequal solid
angles may be contained by the same number of equal plane angles.

Maintaining then that Def. io is an interpolation by ``an unskilful hand,''
Simson transfers to a place before Def. 9 the definition of a solid angle, and
then defines similar solid figures as follows :

Similar solid figures are such as have all their solid angles equal, each to tack,
and which are contained by (he same number of similar planes.

Legendre has an invaluable discussion of the whole subject of these
definitions (Note xn., pp. 3*3 — 336, of the 14th edition of his Aliments de
Giomitrie). He remarks in the first place that, as Simson said, Def. 10 is not
properly a definition, but a theorem which it is necessary to prove ; for it is
not evident that two solids are equal for the sole reason that they have an
equal number of equal faces, and, if true, the fact should be proved by super-
position or otherwise. The fault of Def. 10 is also common to Def, 9. For,
if Def. 1 o is not proved, one might suppose that there exist two unequal and
dissimilar solids with equal faces ; but, in that case, according to Definition 9,
a solid having faces similar to those of the two first would be similar to both
of them, i.e. to two solids of different form : a conclusion implying a con-
tradiction or at least not according with the natural meaning of the word
``similar.''

What then is to be said in defence of the two definitions as given by
Euclid f It is to be observed that the figures which Euclid actually proves
equal or similar by reference to Deff. 9, io are such that their solid angles do
not consist of more than three plane angles ; and he proves sufficiently clearly
that, if three plane angles forming one solid angle be respectively equal to
three plane angles forming another solid angle, the two solid angles are equal.
If now two polyhedra have their faces equal respectively, the corresponding
solid angles will be made up of the same number of plane angles, and the
plane angles forming each solid angle in one polyhedron will be respectively
equal to the plane angles forming the corresponding solid angle in tbe other.
Therefore, if the plane angles in each solid angle are not more than three in
number, the corresponding solid angles will be equal. But if the correspond-
ing faces are equal, and the corresponding solid angles equal, the solids must
be equal ; for they can be superposed, or at least they will be symmetrical
with one another. Hence the statement of Defl''. 9, 10 is true and admissible
at all events in the case of figures with trihedral angles, which is the only case
taken by Euclid.

Again, the example given by Simson to prove the incorrectness of Def. 10
introduces a solid with a re-entrant angle. But it is more than probable that
Euclid deliberately intended to exclude such solids and to take cognizance of
convex polyhedra only ; hence Simson's example is not conclusive against the
definition.

Legendre observes that Simson's own definition, though true, has the
disadvantage that it contains a number of superfluous conditions. To get
over the difficulties, Legendre himself divides the definition of similar solids
into two, the first of which defines similar triangular pyramids only, and the
second (which defines similar polyhedra in general) is based on the first.

Tifo triangular pyramids are simitar when they have pairs of fates respectively
similar, similarly placed and equally inclined to one another.

Then, having formed a triangle with the vertices of three angles taken on
the same face or base of a polyhedron, we may imagine the vertices of the
different solid angles of the polyhedron situated outside of the plane of this
base to be the vertices of as many triangular pyramids which have the triangle
for common base, and each of these pyramids will determine the position of
one solid angle of the polyhedron. This being so,

Two polyhtdra art similar when they have similar bases, and the vertices of
their corresponding solid angles outside the bases art defermintd by triangular
pyramids similar each to each.

As a matter of fact, Cauchy proved that two convex solid figures are equal
if they are contained by equal plane figures similarly arranged. Legendre
gives a proof which, he says, is nearly the same as Cauchy's, depending on two
lemmas which lead to the theorem that, Givtn a convex polyhedron in which alt
the solid angles art made up of mort than thret plane angles, it is impossible to
vary the inclinations of tht plants of this solid so as to product a second polyht-
dron formed by the same plants arranged in the same manner as in the givtn
polyhedron. The convex polyhedron in which all the solid angles are made up
of more than three plane angles is obtained by cutting off from any given
polyhedron all the triangular pyramids forming trihedral angles (if one and the
same edge is common to two trihedral angles, only one of these angles is
suppressed in the first operation). This is legitimate because trihedral angles
are invariable from their nature.

Hence it would appear that Heron's definition of equal solid figures, which
adds `` similarly situated `` to Euclid's `` similar `` is correct, if it be understood to
apply to convex polyhedra only : Equal solid figures are those which are
contained by equal and similarly situated planes, equal in number and magnitude ;
where, however, the words `` equal and `` before `` similarly situated `` might be
dispensed with.

Heron (Def. 118) defines similar solid figures as those which art containtd
by planes similar and similarly situated. If understood of convex polyhedra,
there would not appear to be any objection to this, in view of the truth of
Cauchy's proposition about equal solid figures.

\section*{Definition 11}

Xttpto. ymvia. itrr'iv i\ vwb ttXiiovwv y St'o ypafi/Atiiv OrWTOfLfvtav aXki'ikivf kol fty
iv TJj ttvrjj t7ri<f>aytvL owrmv irpov Trafrcus Tats ypafLftais itktGis. *AAws   trrtptb
yavia i<rrv if wro jtVkWc ij  io ymviwv iiiurtav irtpuxoiiiw) prj ovtruiv cV T<ji
airr brtirif. wpos Jet m)/i(iig trvvuJTa.pivii>v.

Heiberg conjectures that the first of these two definitions, which is not in
Euclid's manner, was perhaps taken by him from some earlier Elements.

The phraseology of the second definition is exactly that of Plato when he
is speaking of solid angles in the Timatus (p. 55). Thus he speaks (1) of four
equilateral triangles so put together ((wttnafitva) that each set of three plane
angles makes one solid angle, (z) of eight equilateral triangles put together so
that each set of four plane angles makes one solid angle, and (3) of six squares
making eight solid angles, each composed of three plane right angles.

As we know, Apollonius defined an angle as the `` bringing together of a
surface or solid to one point under a broken line or surface.'' Heron (Def. 12)
even omits the word `` broken `` and says that A solid angle is in general (koifw*)
the bringing together of a surface which has its concavity in one and the samt
direction to ont point. It is clear from an allusion in Proelus (p. 123, r — 6) to
the half of a cone cut off by a triangle through the axis, and from a scholium to
this definition, that there was controversy as to the correct ness of describing as a
solid angle the `` angle `` enclosed by fewer than three surfaces (including curved
surfaces). Thus the scholiast says that Euclid's definition of a solid angle as
made up of three or more plane angles is deficient because it does not e.g. cover
the case of the angle of a `` fourth part of a sphere,'' which is contained by more
than two surfaces, though not all plane. But he declines to admit that the
half-cone forms a solid angle at the vertex, for in that case the vertex of the
cone would itself be an angle, and a solid angle would then be formed both
by two surfaces and by one surface; ``which is not tme.'' Heron on the
other hand (Def. 22) distinctly speaks of solid angles which arc not contained
by plane rectilineal angles, `` e.g. the angles of cones.'' The conception of the
latter `` angles `` as the limit of solid angles with an infinite number of infinitely
small constituent plane angles does not appear in the Greek geometers so far
as I know.

In modern text-books a polyhedral angle is usually spoken of as formed
(or bounded) by three or more plants meeting at a points or it is the angular
opening between such planes at the point where they meet.

\section*{Definition 12}

IIvpa/i« i(m ffJCJ/Ki tmptov fWnrtSow irtpuxa/uroY 6mb ivi/s ivariSou jrpos Ivl
<rj]jitttf <rw«fr(Js.

This definition is by no means too clear, nor is the slightly amplified
definition added to it by Heron (Def. 99). A pyramid is the figure brought
together to one point, by putting together triangles, from a triangular, quadri-
lateral or polygonal, that is, any rectilineal, base.

As we might expect, there is great variety in the definitions given in
modern text-books. Legendre says a pyramid is the solid formed when several
triangular planes start from one point and are terminated at the different sides
of one polygonal plane.

Mr H. M. Taylor and Smith and Bryant call it a polyhedron all but one of
whose faces meet in a point.

Mehler reverses Legendre's form and gives the content of Euclid's in
clearer language. ``An n-sided pyramid is bounded by an n-sided polygon as base
and n triangUs which connect its sides with one and the same point outside it.''

Rausenberger points out that a pyramid is the figure ci't off from a solid
angle formed of any number of plane angles by a plane which intersects the
solid angle.

\section*{Definition 13}

ITpc'cr/ia tcrrl trxjjfLa trrtptav liFiiritH irtpttofLtvoii, wv Si'o ri Awtvavriov Itra
T€ Kill ofioia i<m KO.I jrapaXXljka, Ttt Bi Xoilra TrajjaXXkoypaft/ia.

Mr H. M. Taylor, followed by Smith and Bryant, defines a prism as a
polyhedron all but two of the faces of which are parallel to one straight line.

Mehler calls an n-sided prism a body contained between two parallel planes
and enclosed by n other planes with parallel lines of intersection.

Heron's definition of a prism is much wider (lief. 105). Prisms are (hose
figures which are connected tnmavrorra.) from a rectilineal base to a rectilineal
area by rectilineal collocation (nar tWvypaniior (rirStatr). By this Heron must
apparently mean any convex solid formed by connecting the sides and angles
of two polygons in different planes, and
each having any number of sides, by
straight lines forming triangular faces
(where of course two adjacent triangles
may be in one plane and so form one
quadrilateral face) in the manner shown
in the annexed figure, where ABCD,
EFG represent the base and its
opposite.

Heron goes on to explain that, if
the face opposite to the base reduces to
a straight line, and a solid is formed by
connecting the base to its extremities by
straight lines, as in the other case, the
resulting figure is neither a pyramid nor
a prism.

Further, he defines parallelogrammic (in the body of the definition parallel-
sided) prisms as being those prisms which have six faces and have their
opposite planes parallel.

\section*{Definition 14}

 alpd t<JTW, OTtLV TjfJUKVKklav fltvavvr)*; Ttj BuLfAtTpOV Tttpiti'tBtv To

fUKVukutv tls to avro rdktv diroKanurra, o0*v rfp£aro tfttpea-Btii, to TrtpiXyftftBh/

``MP*.

The scholiast observes that this definition is not properly a definition 01 a
sphere but a description of the mode of generating it. But it will be seen, in
the last propositions of Book xiii., why, Euclid put the definition in this form.
It is because it is this particular view of a sphere which he uses to prove that
the vertices of the regular solids which he wishes to `` comprehend `` in certain
spheres do lie on the surfaces of those spheres. He proves in fact that the
said vertices lie on semicircles described on certain diameters of the spheres. For
the real definition the scholiast refers to Theodosius' Sphaerica. But of course
the proper definition was given much earlier. In Aristotle the characteristic
of a sphere is that its extremity is equally distant from its centre (to ttrov Avitu/
tou fiio-ou to hrxaror, De caelo ii. 14, 297 a 24). Heron (Def. 76) uses the
same form as that in which Euclid defines the circle : A sphere is a solid
figure bounded by one surface, such that ail the straight lines falling on it from
one point of those which lie within the figure are equal to one another. So the
usual definition in the text-books : A sphere is a closed surface such that all
points of it are equidistant from a fixed point within it.

\section*{Definition 15}

*A£u>v f T199 trfoitpa* itrriv ij jitvovva tvOtia, ir€p\ yv to -rjpXKVttkiov <TTpi<p£T(it.

That any diameter of a sphere may be called an axis is made clear by
Heron (Def. 78). The diameter of the sphere is called an axis, and is any
straight line drawn through the centre and bounded in both directions by the
sphere, immovable, about which the sphere is moved and turned. Cf. Euclid's
Def. 17.

\section*{Definition 16}

KcrTpov Si ttji erKupat ta-rl to airo, S xtu Ttru iJjumvKAu.

Heron, Def. 77. 7  middle (point) of the sphere is called its eentrt ; and
this same point is also the centre of the hemisphere.

\section*{Definition 17}

A«£urrpo? Si riff <TaLpas £<rrtv *vta tic Sii tou KtvTpov ify flirty naX ntpa-
roupirri itf,' inartpa to ft-ipi) wro ts fariavitat nj* craifjas.

\section*{Definition 18}

Kuvac forty, »rav opBoywvLov rptymvov fAtvovtrrfi >uas irXtvpaf Tt3v ir*p* tv
op6yy ytm-iiiv Trifittvfx'ii' to Tplywov <tv to avro iraXip <£irojcaTaaTajji, otv Tjparo
<f>ipttTai, to ir«piXtfl«i< ireful, xir piv ll fUvowra *v6iw. Iot) 7) Tij) Aoiirfl [171]
ircol tv opv TrtpuHpQfAfVfl t Gpo-ytvvtos iarat o kui-os, iav £f MdrrwK, afifikv-
yuytof, fit Si «t£«F, dfuyoSyiot.

This definition, or rather description of the genesis, of a (right) cone is
interesting on account of the second sentence distinguishing between right-
angled, obtuse-angled and acute-angled cones. This distinction is quite
unnecessary for Euclid's purpose and is not used by him in Book xii. ; it is no
doubt a relic of the method, still in use in Euclid's time, by which the earlier
Greek geometers produced conic sections, namely, by cutting right cones only
by sections always perpendicular to an edge. With this system the parabola
was a section of a right-angled cone, the hyperbola a section of an obtuse-angled
cone, and the ellipse a section of an acute-angled cone. The conic sections were
so called by Archimedes, and generally until Apollonius, who was the first to
give the complete theory of their generation by means of sections not perpen-
dicular to an edge, and from cones which are in general oblique circular cones.
Thus Apollonius begins his Conies with the more scientific definition of a cone.
If, he says, a straight line infinite in length, and passing always through a fixed
point, be made to move round the circumference of a circle which is not in the
same plane with the point, so as to pass successively through every point of
that circumference, the moving straight line will trace out the surface of a double
cone, or two similar cones lying in opposite directions and meeting in the fixed
point, which is the apex of each cone. The circle about which the straight line
moves is called the base of the cone lying between the said circle and the fixed
point, and the axis is defined as the straight line drawn from the fixed point,
or the apex, to the centre of the circle forming the base. Apollonius goes on
to say that the cone is a scalene or oblique cone except in the particular case
where the axis is perpendicular to the base. In this latter case it is a right
cone.

Archimedes called the right cone an isosceles cone. This fact, coupled
with the appearance in his treatise On Conoids and Spheroids (7, 8, 9) of
sections of acute-angled cones (ellipses) as sections of conical surfaces which are
proved to be oblique circular cones by finding their circular sections, makes it
sufficiently clear that Archimedes, if he had defined a cone, would have
defined it in the same way as Apollonius does.

\section*{Definition 19}

*A£ titV Si TOV KWVOV ItTTtV v) fltYQVfriX tVita, Tttpl TJV TO TplytaVQV OTpttjitTUU.

\section*{Definition 20}

BuiTLS S) 4 X UK A Of i VTTU T7J TT f f H<p f flOI K1J5 IV(LU<; ypU <f>6fl t V Of

\section*{Definition 21}

KuAivSpoc iffTik, cTrav opoyaiWov TrapaAA??oypd|iu)u /ifvotlarfs /uaf TrXrvpa?
T4iV wtpt Tr)k dpfjk ycjruip Trepupfxi' to TrapaAArjXoypapioy tts to avro iraAt*'
ijr<wo.T<wrroSB, ofl<c ripiaro aWp«ru, to irfptXtjic o''x/*a.

\section*{Definition 22}

Afuf SI tov ttvktypav (cTTif tj jivavtra. ivStia, iripl 17 k to npaAAiiXdypa/ipoi'

OTptyfTU.

\section*{Definition 23}

Bcureie St oi miicAoi ot vto tuk ctirfyaFrcov jrtpuiyofAtvimr £uo irXropfov

\section*{Definition 24}

'O/ioum jtuj p-oi rat KtiXtrSpot (iatf, wr oT ti uf out ™i al SuVtTpo< r«Sv fidatwr
iydkoydv riv.

\section*{Definition 25}

Ku/9ot furl o-xv/ia o*rtp«S* vjto If wrpaywewe fr™* ir«fHtxo/t(vof .

\section*{Definition 26}

'OdTatSpiJi' i<m ax*)t''> oTtprev vtto okt4 Tptytuwiiv i<rw ko.i ttroirXtiipojp

\section*{Definition 27}

EuiOO-OfSpOl' JO-Tt CXjw <TT(p€OV VTO (IkOI TplywFMC »0 , UH' KOi Io-twX(VpUH'

TT<ptfo/jlivar f

\section*{Definition 28}

AtaStKatSpoy fori o'x'Jf'' 1 or«p«o* fori Suima irtvrayojwitf uruiv Ktti Jo-oirX«iipMK

\part{Book XI. Propositions}

\begin{proposition}
\label{prop:XI_1}

\begin{statement}
A part of a straight line cannot be in Ike plane of reference
and a part in a plane more elevated.
\end{statement}

\begin{proof}

For, if possible, let a part AB of the straight line ABC
be in the plane of reference, and a part
BC in a plane more elevated.

There will then be in the plane of
reference some straight line continuous
with AB in a straight line.

Let it be BD ;
therefore AB is a common segment of the
two straight lines ABC, ABD :
which is impossible, inasmuch as, if we
describe a circle with centre B and distance
AB, the diameters will cut off unequal circumferences of the
circle.

Therefore a part of a straight line cannot be in the plane
of reference, and a part in a plane more elevated.
\end{proof}

\begin{annotations}

1. the plane of reference, to irroKtlfitvot (rfacSoy, the plane Inid down or assumed.

2. more elevated, /LereapoTtptp.

\end{annotations}

\begin{notes}

There is no doubt that the proofs of the first three propositions are
unsatisfactory owing to the fact that Euclid is not able to make any use of his
definition of a plane for the purpose of these proofs, and they really depend
upon truths which can only be assumed as axiomatic. The definition of a plane
as that surface which lie: evenly with the straight lines on itself, whatever its
exact meaning may be, is nowhere appealed to as a criterion to show whether
a particular surface is or is not a plane. If the meaning of it is what I conjec-
ture in the note on Book ]., Def. 7 (Vol. 1. p. 171), if, namely, it only tries to
express without an appeal to sight what Plato meant by the `` middle covering
the extremities `` (i.e. apparently, in the case of a plane, the fact that a plane
looked at edgewise takes the form of a straight line), then it is perhaps
possible to connect the definition with a method of generating a plane which
has commended itself to many writers as giving a better definition. Thus, if
we conceive a straight line in space and a point outside it placed so that, in
Plato's words, the line `` covers `` the point as we look at them, the line will
also ``cover'' every straight line which passes through the given point and
some one point on the given straight line. Hence, if a straight line passing
always through a fixed point moves in such a way as to pass successively
through every point of a given straight line which does not contain the given
point, the moving straight line describes a surface which satisfies the Euclidean
definition of a plane as I have interpreted it. But if we adopt the definition
of a plane as the surface described by a straight line which, passing through a
given faint, turns about it in such a -way as always to intersect a given straight
line not passing through the given point, this definition, though it would help us
to prove Eucl.\ *Xi. 2, does not give us the fundamental properties of a plane ;
some postulate is necessary in addition. The same is true even if we take a
definition which gives more than is required to determine a plane, the defini-
tion known as Simson's, though it is at least as early as the time of Theon of
Smyrna, who says (p. 112,5) tnat a P nt ** a surface such that, if a straight line
meet it in two points, the straight line lies wholly in it (oAij avrif ioppifrrat).
This is also called the axiom of the plane. (For some attempts to prove this on
the basis of other definitions of a plane see my note on the definition of a plane
surface, t. Def. 7.) If this definition or axiom be assumed. Prop. 1 becomes
evident, for, as I,egendre says, ``In accordance with the definition of the plane,
when a straight line has two points common with a plane, it lies wholly in the
plane.''

Euclid practically assumes the axiom when he says in this proposition
``there will be in the plane of reference some straight line continuous with
AB'' Clavius tries, unsuccessfully, to deduce this from Euclid's own
definition of a plane ; and he seems to admit his
failure, because he proceeds to try another tack.
Draw, he says, in the plane DE, the straight line
CG at right angles to AC, and, again in the plane
DE, CF at right angles to CG (t. 1 r]. Then AC,
CF make right angles with CG in the same plane ;
therefore (1. 14) ACFis a straight line. But this
does not really help, because Euclid assumes tacitly,
in Book i. as well as Book xi., that a straight line joining two points in a
plane lies wholly in that plane.

A curious point in Euclid's proof is the reason given why two straight lines
cannot have a common segment. The argument is precisely that of the
`` proof'' of the same thing given by Proclus on 1. 1 (see note on Book 1.
Post J, Vol. 1. p. r97) and is of course inconclusive. The fact that two
straight lines cannot have a common segment must be taken to be involved
in the definition of, and the postulates relating to, the straight line; and the
`` proof'' given here can hardly, I should say, be Euclid's, though the interpo-
lation, if it be such, must have been made very early.

The proof assumes too that a circle can be dpseribed so as to cut BA, BC
and SB, or, in other words, it assumes that AD, BC are in one plane; -that
is, Prop. 1 as we have it realty assumes the result of Prop. 2. There is there-
fore ground for Simson's alteration of the proof (after the point where 3D has
been taken in the given plane in a straight line with AB) to the following :

`` Let any plane pass through the straight line AD and be turned about it
until it pass through the point C

And, because the points B, C are in this plane, the straight line BC is
in it. [Simson's def.]

Therefore there are two straight lines ABC, ABD in the same plane that
have a common segment AB :
which is impossible.''

Simson, of course, justifies the last inference by reference to his Corollary
to i. i r, which, however, as we have seen, is not a valid proof of the assump-
tion, which is really implied in i. Post. 2.

An alternative reading, perhaps due to Theon, says, after the words
``which is impossible `` in the Greek text, ``for a straight line does not meet a
straight line in more points than one ; otherwise the straight lines will
coincide.'' Simson (who however does not seem to have had the second
clause beginning ``otherwise'' in the text which he used) attacks this alterna-
tive reading in a rather confused note chiefly directed against a criticism by
Thomas Simpson, without (as it seems to me) sufficient reason. It contains
surely a legitimate argument. The supposed straight lines ABC, ABD meet
in more than two points, namely in alt the points between A and B. But two
straight lines cannot have two points common without coinciding altogether ;
therefore ABC must coincide with ABD.

\end{notes}

\end{proposition}

\begin{proposition}
\label{prop:XI_2}

\begin{statement}
If two straight lines cut one another, they are in one plane,
and every triangle is in on* plane.
\end{statement}

\begin{proof}

For let the two straight lines AB, CD cut one another at
the point £ ;

I say that AB, CD are in one plane,
and every triangle is in one plane.

For let points F, G be taken at
random on EC, EB,
let CB, FG be joined,
and let FH, GK be drawn across ;
I say first that the triangle ECB is
in one plane.

For, if part of the- triangle ECB,
either FHC or GBK, is in the plane of reference, and the rest
in another,

a part also of one of the straight lines EC, EB will be in the
plane of reference, and a part in another.

But, if the part FCBG of the triangle ECB be in the
plane of reference, and the rest in another,
a part also of both the straight lines EC, EB will be in the
plane of reference and a part in another :
which was proved absurd. [su. 1]

Therefore the triangle ECB is in one plane.
But, in whatever plane the triangle ECB is, in that plane
also is each of the straight lines EC, EB,

and, in whatever plane each of the straight lines EC, EB is,
in that plane are AB, CD also. [xt. 1]

Therefore the straight lines AB, CD are in one plane,

and every triangle is in one plane.
\end{proof}

\begin{notes}

It must be admitted that the `` proof'' of this proposition is not of any
value. For one thing, Euclid only takes certain triangles and a certain
quadrilateral respectively forming part of the original triangle, and argues
about these. But, for anything we are supposed to know, there may be some
part of 'hi; triangle bounded (let us say) by some curve which is not in the
same plane with the triangle.

We may agree with Simson that it would be preferable to enunciate the
proposition as follows.

Tiw straight lints which intersect are in one plane, and three straight lines
which intersect two and two are in one plane.

Adopting Smith and Bryant's figure in preference to Simson's, we suppose
three straight lines PQ, RS, XYto intersect
two and two in A , B, C. R p

Then Simson's proof (adopted by Legen- *S

dre also) proceeds thus. Vf

Let any plane pass through the straight S
line PQ, and let this plane be turned about /
PQ (produced indefinitely) as axis until it /
passes through the point C. x / \ y

Then, since the points A, C are in this ~7 ``

plane, the straight line AC (and therefore /

the straight line RS produced indefinitely) * Jj

lies wholly in the plane. [Simson's def.j

For the same reason, since the points />', C are in the plane, the straight
line XYMes wholly in the plane.

Hence all three straight lines PQ, PS, XY (and of course any pair of
them) lie in one plane.

But it has still to be proved that there is only one plane passing through
the three straight lines.

This may be done, as in Mr Taylor's Euclid, thus.

Suppose, if possible, that there are two different planes through A, B, C.

The straight lines BC, CA, AB then lie wholly in each of the two planes.

Now any straight line in one of the two planes must intersect at least two
of the straight lines (produced if necessary) ;

let it intersect two of them in K, L.

Then, since K, L are also in the second plane, the line KL lies wholly in
that plane.

Hence every straight line in either of the planes lies wholly in the other
also ; and therefore the planes are coincident throughout their whole surface.

It follows from the above that

A plan/ is determined (i.e. uniquely determined) by any of the following data :
(1) by thru straight lines meeting one another two and two,
(t) by three points not in a straight line,

(3) by tivo straight lines meeting one another,

(4) by a straight line and a point without it.

\end{notes}

\end{proposition}

\begin{proposition}
\label{prop:XI_3}

\begin{statement}
If two planes cut one another, their common section is a
straight line.
\end{statement}

\begin{proof}

For let the two planes AB, BC cut one another,
and let the line DB be their common
section ;
I say that the line DB is a straight line.

For, if not, from D to B let the straight
line DEB be joined in the plane AB, and
in the plane BC the straight line DFB.

Then the two straight lines DEB, DFB
will have the same extremities, and will
clearly enclose an area :
which is absurd.

Therefore DEB, DFB are not straight lines.

Similarly we can prove that neither will there be any
other straight line joined from DtoB except DB the common
section of the planes AB, BC.

Therefore etc.
\end{proof}

\begin{notes}

I think Simson is right in objecting Co the words after `` which is absurd/'
to the effect that DEB, DFB are not straight lines, and that neither can there
be any other straight line joined from D to B except DB, as being unneces-
sary. It is right to conclude at once from the absurdity that 3D cannot but
be a straight line.

Legendre makes his proof depend on Prop. 2. `` For, if, among the points
common to the two planes, three should be found which are not in a straight
line, the two planes in question, each passing through three points, would only
amount to one and the same plane.'' [This of course assumes that three
points determine one and only one plane, which, strictly speaking, involves
more than Prop. a itself, as shown in the last note.]

A favourite proposition in modern text-books is the following. The proof
seems to be due to von Staudt (Killing, Grundlagen der Geometrie, Vol. 11.
P- «)

J] two plants meet in a point, they meet in a straight line.

Let ABC, ADE be two given planes meeting
at A.

Take any points B, C lying on the plane ABC,
and not on the plane ADE but on the same side
of it

Join AB, AC, and produce BA to F.

Join CF.

Then, since B, Fare on opposite sides of the
plane ADE,
C, Fare also on opposite sides of it.

Therefore CF must meet the plane ADE in
some point, say G.

Then, since A, G are both in each of the planes ABC, ADE, the straight
line AG is in both planes. [Simson's def.J

This is also the place to insert the proposition that, If three planes intersect
two and two, their lines of intersection either meet in a point or are parallel two
and two.

Let there be three planes intersecting in the straight lines AB, CD, EF.

cce?

Now AB, EF are in a plane ; therefore they either meet in a point or are
parallel.
(i) Let them meet in O.

Then O, being a point in AB, lies in the plane AD, and, being also a
point in EF, lies also in the plane ED.

Therefore O, being common to the planes AD, DE, must lie on CD, the
line of their intersection ;
i.e. CD, if produced, passes through O,
(3) Let AB, EF not meet, but let them be parallel.

Then CD cannot meet AB ; for, if it did, it must necessarily meet EF,
by the first case.

Therefore CD, AB, being in one plane, are parallel.

Similarly CD, EFte parallel.

\end{notes}

\end{proposition}

\begin{proposition}
\label{prop:XI_4}

\begin{statement}
If a straight line be set up at right angles to two straight
lines which cut one another, at their common point of section,
it will also lie at right angles to the plane through them.
\end{statement}

\begin{proof}

For let a straight line EF be set up at right angles to the
two straight lines AB, CD, which
cut one another at the point E,
from E ;

I say that EF is also at right
angles to the plane through AB,
CD.

For let AE, EB, CE, ED be
cut off equal to one another,

and let any straight line GEH be drawn across through E,
at random ;
let AD, CB be joined,

and further let FA, FG, FD, FC, FH, FB be joined from
the point F taken at random <on FF>.

Now, since the two straight lines AE, ED are equal to
the two straight lines CE, EB, and contain equal angles, [i. 15]
therefore the base AD is equal to the base CB,
and the triangle AED will be equal to the triangle CEB; [1.4]
so that the angle DAE is also equal to the angle EBC.

But the angle AEG is also equal to the angle BEH ;[i. 15]
therefore AGE, BEH are two triangles which have two
angles equal to two angles respectively, and one side equal
to one side, namely that adjacent to the equal angles, that
is to say, AE to EB ;

therefore they will also have the remaining sides equal to the
remaining sides. [' 26]

Therefore GE is equal to EH, and AG to BH.

And, since AE is equal to EB,
while FE is common and at right angles,
therefore the base FA is equal to the base FB. [1. 4]

For the same reason
FC is also equal to FD.

And, since AD is equal to CB,
and FA is also equal to FB,

the two sides FA, AD are equal to the two sides FB, BC
respectively ;

and the base FD was proved equal to the base FC ;
therefore the angle FAD is also equal to the angle FBC. [1. 8]

And since, again, AG was proved equal to BH,
and further FA also equal to FB,
the two sides FA, AG are equal to the two sides FB BH.

And the angle FAG was proved equal to the angle FBH;
therefore the base FG is equal to the base FH. [i. 4]

Now since, again, GE was proved equal to EH,
and EF is common,

the two sides GE, EFare equal to the two sides HE, EF;
and the base FG is equal to the base FH;
therefore the angle GEF is equal to the angle HEF. [1. 8]

Therefore each of the angles GEF, HEF is right.

Therefore FE is at right angles to GH drawn at random
through E.

Similarly we can prove that FE will also make right
angles with all the straight lines which meet it and are in the
plane of reference.

But a straight line is at right angles to a plane when it
makes right angles with all the straight lines which meet it
and are in that same plane ; [xi. Def. 3]

therefore FE is at right angles to the plane of reference.

But the plane of reference is the plane through the straight
lines AB, CD.

Therefore FE is at right angles to the plane through
AB, CD.

Therefore etc.
\end{proof}

\begin{notes}

The steps to be successively proved in order to establish this proposition
by Euclid's method are

(1) triangles A ED, EEC equal in all respects, [by 1. 4]

(t) triangles AEG, BE/fequl in all respects, [by 1. 26]

so that AG is equal to BH, and GE to EH,

(3) triangles AEF, BEF equal in all respects, [1. 4]
so that AFis equal to BE,

(4) likewise triangles CEF, DEF,
so that CFis equal to DF,

(5) triangles FAD, FBC equal in all respects, [1. 8]
so that the angles FAG, FBHm equal,

(6) triangles FAG, FBH equal in all respects, [by (1), (3), (5) and 1. 4]
so that FG is equal to FH,

(7) triangles FEG, FEH equal in all respects, [by (2), (6) and 1. 8]

so that the angles FEG, FEH are equal,
and therefore FE is at right angles to Gil.

In consequence of the length of the above proof others have been
suggested, and the proof which now finds most general acceptance is that of
Cauchy, which is as follows.

Let AB be perpendicular to two straight lines BC, BD in the plane MN
at their point of intersection B.

In the plane vlWdraw BE, any straight line
through B.

Join CD, and let CD meet BE in ML

Produce ABto F o that BF is equal to AB,

Join AC, AE, AD, CF, EF, DF.

Since BC is perpendicular to AF at its
middle point B,
AC is equal to CF.

Similarly AD is equal to DF.

Since in the triangles A CD, FCD the two
sides AC, CD are respectively equal to the two
sides FC, CD, and the third sides AD, ED are
also equal,

the angles ACD, FCD are equal. [i. 8]

The triangles ACE, FCE thus have two sides and the included angle
equal, whence

EA is equal to EF. [1. 4]

The triangles ABE, FBE have now all their sides equal respectively ;

therefore the angles ABE, FBE are equal, [1. 8]

and AB is perpendicular to BE.

And BE is in any straight line through B in the plane MN.

Legendre's proof is not so easy, but it is interesting. We are first required
to draw through any point E within the angle
CBD a straight line CD bisected at E.

To do this we draw EK parallel to DB
meeting B C in K, and then mark off KC equal
to BK.

CE is then joined and produced to D; and
CD is the straight line required.

Now, joining AC, AE, AD in the figure
above, we have, since CD is bisected at E,
(i) in the triangle ACD,

AC + AD 1 =2AE i +2ED',
and also (2) in the triangle BCD,

BC + BDP = 2BF? + tED 1 .

Subtracting, and remembering that the triangles ABC, ABD are right-
angled, so that

AC'-BCAB*,
and AD t -BD> = AB l ,

we have sAB? = 2AE> - 2BE 1 ,

or AE'AB' + BE 1 ,

whence [1. 48] the angle ABE is a right angle, and AB is perpendicular
to BE.

It follows of course from this proposition that the perpendicular AB is the
shortest distant from A to the plant MN.

And it can readily be proved that,

If f rem a point without a plane oblique straight lines be drawn to the plane,
( i ) those meeting the plane at equal distances from the foot of the perpendicular
art equal, and

(i) of ttvo straight lines meeting the plane at unequal distances from the foot of
the perpendicular, the more remote is the greater.

Lastly, it is easily seen that

From a point outside a plane only one perpendicular can be drawn to that
plane.

For, if possible, let there be two perpendiculars. Then a plane can be
drawn through them, and this will cut the original plane in a straight line.

This straight line and the two perpendiculars will form a plane triangle
which has two right angles : which is impossible.

\end{notes}

\end{proposition}

\begin{proposition}
\label{prop:XI_5}

\begin{statement}
If a straight line be set up at right angles to three straight
lines which meet one another, at their common point of section,
the three straight lines are in one plane.
\end{statement}

\begin{proof}

For let a straight line AB be set up at right angles to the
three straight lines BC, BD, BE, at
their point of meeting at B ;
I say that BC, BD, BE are in one plane.

For suppose they are not, but, if
possible, let BD, BE be in the plane of
reference and BC in one more elevated ;
let the plane through AB, BC be
produced ;

it will thus make, as common section in the plane of reference,
a straight line. [xi. 3]

Let it make BF.

Therefore the three straight lines AB, BC, BE are in one
plane, namely that drawn through AB, BC.

Now, since AB is at right angles to each of the straight
lines BD, BE,

therefore AB is also at right angles to the plane through
BD, BE. [xi. 4]

But the plane through BD, BE is the plane of reference ;
therefore AB is at right angles to the plane of reference.

Thus AB will also make right angles with all the straight
lines which meet it and are in the plane of reference.

[xi. De£ 3]

But BF which is in the plane of reference meets it ;

therefore the angle ABF is right.

But, by hypothesis, the angle ABC is also right ;
therefore the angle ABF is equal to the angle ABC.

And they are in one plane :
which is impossible.

Therefore the straight line BC is not in a more elevated
plane ;

therefore the three straight lines BC BD, BE are in one
plane.

Therefore, if a straight line be set up at right angles to
three straight lines, at their point of meeting, the three straight
lines are in one plane.
\end{proof}

\begin{notes}

It follows that, If a right angle be turned about one of the straight lines
containing it the other will describe a plane.

At any point in a straight line it is possible to draw only one plane which
is at right angles to the straight line.

One such plane can be found by taking any two planes through the given
straight line, drawing perpendiculars to the straight
line in the respective planes, e.g. BO, CO in the
planes AOB, AOC, each perpendicular to AO,
and then drawing a plane (BOC) through the
perpendiculars.

If there were another plane through O per-
pendicular to AO, it must meet the plane through
AO and some perpendicular to it as OC in a
straight line OC different from OC.

Then, by xi. 4, AOC is a right angle, and in
the same plane with the right angle AOC    which is impossible.

Next, one plane and only one can be drawn through a point outside a straight
line at right angUs to that line.

Let P be the given point, AB the given straight
line.

In the plane through P and AB, draw PO per-
pendicular to AB, and through draw another straight
line OQ at right angles to AB.

Then the plane through OP, OQ is perpendicular
to AB.

If there were another plane through P perpendicular
to AB, either

(1) it would intersect AB at O but not pass through OQ, or
(2) it would intersect AB At a point different from Q.

In either case, an absurdity would result

\end{notes}

\end{proposition}

\begin{proposition}
\label{prop:XI_6}

\begin{statement}
If two straight lines be at right angles to the same plane,
the straight lines  mill be parallel.
\end{statement}

\begin{proof}

For let the two straight lines AB, CD be at right angles
to the plane of reference ;
I say that AB is parallel to CD.

For let them meet the plane of
reference at the points B, D,
let the straight line BD be joined,
let DE be drawn, in the plane of
reference, at right angles to BD,
let DE be made equal to AB,
and let BE, AE, AD be joined.

Now, since AB is at right angles to the plane of reference,
it will also make right angles with all the straight lines which
meet it and are in the plane of reference. [xi. Def. 3]

But each of the straight lines BD, BE is in the plane of
reference and meets AB;
therefore each of the angles ABD t ABE is right.

For the same reason
each of the angles CDB, CDS is also right.

And, since AB is equal to DE,
and BD is common,

the two sides AB, BD are equal to the two sides ED, DB ;
and they include right angles ;
therefore the base AD is equal to the base BE, [1. 4]

And, since AB is equal to DE,
while AD is also equal to BE,

the two sides AB, BE are equal to the two sides ED, DA ;
and AE is their common base ;
therefore the angle ABE is equal to the angle EDA. [1. 8]

But the angle ABE is right ;
therefore the angle EDA is also right ;
therefore ED is at right angles to DA.

But it is also at right angles to each of the straight lines
BD, DC;

therefore ED is set up at right angles to the three straight
lines BD, DA, DC at their point of meeting;
therefore the three straight lines BD, DA, DC are in one
plane. [xi. 5]

But, in whatever plane DB, DA are, in that plane is AB
also,

for every triangle is in one plane ; [xi. *]

therefore the straight lines AB, BD t DC are in one plane

And each of the angles ABD, BDC is right ;
therefore AB is parallel to CD. [1. *

Therefore etc.
\end{proof}

\begin{notes}

If anyone wishes to convince himself of the real necessity for some
general agreement as to the order in which propositions in elementary
geometry should be taken, let him contemplate the hopeless result of too
much independence on the part of editors in the matter of this proposition
and its converse, xi. 8.

Legendre adopts a different, and elegant, method of proof ; but he applies
it to xi. 8, which he gives first, and then deduces xt. 6 from it by rtductio ad
absurdum. Dr Mehler uses Legendre's method of proof but applies it to
xi. 6, and then gives xi. 8 as a deduction from it Iardner follows Legendre.
Holgate, the editor of a recent American book, gives Euclid's proof of XL 6
and deduces xi. 8 by \emph{reductio ad absurdum}. Mis countrymen, Schultze and
Sevenoak, give xi, 8 first, but put it after, and deduce it from, Eucl.\ xi. 10;
they then give xi. 6, practically as a deduction from XL 8 by reduetio ad
absurdum, after a proposition corresponding to Eucl.\ xi. ti and Iff, and a
corollary to the effect that through a given point one and only one perpen-
dicular can be drawn to a given plane.

We will now give the proof of xi. 6 by Legendre's method (adopted by
Siiith and Bryant as well as by Mehler).

Let AB, CD be both perpendicular to the
same plane MN.

Join BD.

Now, since BD meets AB, CD, both of
which are perpendicular to the plane MN in
which BD is,
the angles ABD, CDS are right angles,

AB, CD will therefore be parallel provided
that they are in the same plane.

Through D draw EDF, in the plane MN,
at right angles to BD, and make ED equal to DF.

Join BE, BF, AE, AD, AF.

Then the triangles BDE, BDF are equal in all respects (by 1. 4), so that
BE is equal to BF,

It follows, since the angles ABE, ABFaxe right, that the triangles ABE,
ABFaie equal in all respects, and

AE is equal to AF.

[Mehler now argues elegantly thus. If C£, CF be also joined, it is clear
that

CE is equal to CF.

Hence each of the four points A, B, C, D is equidistant from the two
points E, F.

Therefore the points A, B, C, D are in one plane, so that AB, CD are
parallel.

If, however, we do not use the locus of points equidistant from two fixed
points, we proceed as follows.]

The triangles A ED, AFD have their sides equal respectively j
hence [1. 8] the angles ADE, ADF aje equal,
so that ED is at right angles to AD.

Thus ED is at right angles to BD, AD, CD;
therefore CD is in the plane through AD, BD. [xi. 5]

But AB is in that same plane; [xi. z]

therefore AB, CD are in the same plane.

And the angles ABD, CDB are right ;
therefore AB, CD are parallel.

\end{notes}

\end{proposition}

\begin{proposition}
\label{prop:XI_7}

\begin{statement}
If two straight lines be parallel and points be taken at
random on each of them, the straight line joining the points is
in the same plane with the parallel straight lines.
\end{statement}

\begin{proof}

Let AB, CD be two parallel straight lines,
and let points E, F be taken at random

on them respectively ;

I say that the straight line joining the
points E, F is in the same plane with
the parallel straight lines.

For suppose it is not, but, if possible, c
let it be in a more elevated plane as
EGF,

and let a plane be drawn through EGF;
it will then make, as section in the plane of reference, a
straight line. [xi. 3]

Let it make it, as EF;
therefore the two straight lines EGF, EF will enclose an
area ;
which is impossible.

Therefore the straight line joined from E to F is not in a
plane more elevated ;

therefore the straight line joined from E to F is in the plane
through the parallel straight lines AB, CD.

Therefore etc.
\end{proof}

\begin{notes}

It is true that this proposition, in the form in which Euclid enunciates it,
is hardly necessary if the plane is defined as a surface such that, if any two
points be taken in it, the straight line joining them lies wholly in the surface.
But Euclid did not give this definition ; and, moreover, Prop. 2 would be
usefully supplemented by a proposition which should prove that two parallel
straight lines determine a plane (i.e. one plane and one only) which also
contains alt the straight lines which join a point on one of the parallels to a point
on the other. That there cannot be two planes through a pair of parallels
would be proved in the same way as we prove that two or three intersecting
straight lines cannot be in two different planes, inasmuch as each transversal
lying in one of the two supposed planes through the parallels would lie wholly
in the other also, so that the two supposed planes must coincide throughout
(cf. note on Prop. 2 above).

But, whatever be' the value of the proposition as it is, Simson seems to
have spoilt it completely. He leaves out the construction of a plane through
EGF, which, as Euclid says, must cut the plane containing the parallels in
a straight line ; and, instead, he says, `` In the plane ABCD in which the
parallels are draw the straight line EHF from £ to F.'' Now, although we
can easily draw a straight line from E to F, to claim that we can draw it in
the plane in which the parallels are is surely to assume the very result which is
to be proved. All that we could properly say is that the straight line joining
E to F is in some plane which contains the parallels ; we do not know that
there is no more than one such plane, or that the parallels determine a plane
uniquely \ without some such argument as that which Euclid gives.

Nor can I subscribe to the remarks in Simson 's note on the proposition.
He says (i) ``This proposition has been put into this book by some unskilful
editor, as is evident from this, that straight lines which are drawn from one
point to another in a plane are, in the preceding books, supposed to be in that
plane ; and if they were not, some demonstrations in which one straight line
is supposed to meet another would not be conclusive. For instance, in
Prop. 30, Book 1, the straight line GK would not meet EF, if GK were not in
the plane in which are the parallels AB, CD, and in which, by hypothesis, the
straight line EF is.'' But the subject-matter of Book t. and Book xi. is quite
different ; in Book 1. everything is in one plane, and when Euclid, in defining
parallels, says they are straight lines in the same plant etc., he only does so
because he must, in order to exclude non-intersecting straight lines which are
not parallel. Thus in 1. 30 there is nothing wrong in assuming that there may
be three parallels in one plane, and that the straight line GHK cuts all three.

But in Book xi. it becomes a question whether there can be more than one
plane through parallel straight lines.

Simson goes on to say (2) ``Besides, this 7th Proposition is demonstrated
by the preceding 3rd ; in which the very same thing which is proposed to be
demonstrated in the 7th is twice assumed, vi., that the straight line drawn
from one point to another in a plane is in that plane.'' But there is nothing
in Prop. 3 about a plane in which two parallel straight lines are ; therefore
there is no assumption of the result of Prop. 7. What is assumed is that,
given two points in a plane, they can be joined by a straight line in the plane :
a legitimate assumption.

Lastly, says Simson, ``And the same thing is assumed in the preceding
6th Prop, in which the straight line which joins the points B, D that are in
the plane to which AB and CD are at right angles is supposed to be in that
plane.'' Here again there is no question of a plane in which two parallels are ;
so that the criticism here, as with reference to Prop, 3, appears to rest on a
misapprehension.

\end{notes}

\end{proposition}

\begin{proposition}
\label{prop:XI_8}

\begin{statement}
If two straight lines be parallel, and one of them be at
right angles to any plane, the remaining one will also be at
right angles to ike same plane.
\end{statement}

\begin{proof}

Let AB, CD be two parallel straight lines,
and let one of them, AB, be at right
angles to the plane of reference ;
I say that the remaining one, CD, will
also be at right angles to the same
plane.

For let AB, CD meet the plane of
reference at the points B, D,
and let BD be joined ;
therefore AB, CD, BD are in one plane. [xi 7]

Let DE be drawn, in the plane of reference, at right angles
to BD,

let DE be made equal to AB,
and let BE, AE, AD be joined.

Now, since AB is at right angles to the plane of reference,
therefore AB is also at right angles to all the straight lines
which meet it and are in the plane of reference ; [xi. Def. 3]
therefore each of the angles ABD, ABE is right.

And, since the straight line BD has fallen on the parallels
AB, CD,

therefore the angles ABD, CDB are equal to two right
angles. [   29]

But the angle ABD is right ;
therefore the angle CDB is also right ;
therefore CD is at right angles to BD.

And, since AB is equal to DE,
and BD is common,

the two sides AB, BD are equal to the two sides ED, DB ;
and the angle ABD is equal to the angle EDB,
for each is right ;
therefore the base AD is equal to the base BE,

And, since AB is equal to DE,
and BE to AD,

the two sides AB, BE are equal to the two sides ED, DA
respectively,

and AE is their common base ;
therefore the angle ABE is equal to the angle EDA.

But the angle ABE is right ;
therefore the angle EDA is also right ;
therefore ED is at right angles to AD.

But it is also at right angles to DB ;
therefore ED is also at right angles to the plane through
BD, DA, [xi. 4]

Therefore ED will also make right angles with all the
straight lines which meet it and are in the plane through
BD, DA.

But DC is in the plane through BD, DA, inasmuch as
AB, BD are in the plane through BD, DA, [xi. 1]

and DC is also in the plane in which AB, BD are.

Therefore ED is at right angles to DC,
so that CD is also at right angles to DE.

But CD is also at right angles to BD.

Therefore CD is set up at right angles to the two straight
lines DE, DB which cut one another, from the point of section
atD;

so that CD is also at right angles to the plane through
DE, DB. [xi. 4 ]

But the plane through DE, DB is the plane of reference ;
therefore CD is at right angles to the plane of reference.

Therefore etc.
\end{proof}

\begin{notes}

Simson objects to the words which explain why DC is in the plane through
BD, DA, viz. ``inasmuch as AB, BD are in the plane through BD, DA, and
DC is also in the plane in which AB, BD are,'' as being too roundabout.
He concludes that they are corrupt or interpolated, and that we ought only to
have the words `` because all three are in the plane in which are the parallels
AB, CD `` (by Prop. 7 preceding). But I think Euclid's words can be
defended. Prop. 7 says nothing of a plane determined by two transversals as
BD, DA are. Hence it is natural to say that DC is in the same plane in
which AB, BD are [Prop. j\ and AB, BD are in the same plane as BD,
DA [Prop, 3], so that DC is in the plane through BD, DA.

Legendre s alternative proof is split by him into two propositions.

(1) Let AB be a perpendicular to (heptane MN and EF a line situated in that
piane ; if from B, the foot of the perpendicular, BD be drawn perpendicular to
EF, and AD be joined, I say that AD will be perpendicular to EF.

(2) If AB is perpendicular to the plane MN, every straight line CD parallel to
AB will be perpendicular to the same plane.

To prove both propositions together we suppose CD given, join BD,
and draw EF perpendicular to BD in the
plane MN.

(t) As before, we make DE equal to DEand
join BE, BE, AE, AF.

Then, since the angles BDE, BDF are
right, and DE, DF equal,

BE is equal to BF. [1. 4]

And, since AB is perpendicular to the
plane,

the angles ABE, ABE are both right.
Therefore, in the triangles ABE, ABF,

AE is equal to AF. [1. 4]

Lastly, in the triangles ADE, ADF, since AE is equal to AF, and DE
to DF, while AD is common,

the angle ADE is equal to the angle ADF, [1. 3]

so that AD is perpendicular to EF,

(2) ED being thus perpendicular to DA, and also (by construction)
perpendicular to DB,

ED is perpendicular to the plane ADB. [xi. 4]

But CD, being parallel to AB, is in the plane ABD ;

therefore ED is perpendicular to CD. [xt. Def. 3]

Also, since AB, CD are parallel,
and ABD is a right angle,
CDB is also a right angle.

Thus CD is perpendicular to both DE and DB, and therefore to the
plane MN through DE, DB.

\end{notes}

\end{proposition}

\begin{proposition}
\label{prop:XI_9}

\begin{statement}
Straight lines which are parallel to the same straight line
and are not in the same plane with it are also parallel to one
another.
\end{statement}

\begin{proof}

For let each of the straight lines AB, CD be parallel to
EF, not being in the same plane

with it ; b h *

I say that AB is parallel to CD.

For let a point G be taken at f Q e

random on EF,

and from it let there be drawn

GH, in the plane through EF,

AB, at right angles to EF, and GK in the plane through

FE, CD again at right angles to EF.

Now, since EF is at right angles to each of the straight
lines GH, GK,

therefore EF is also at right angles to the plane through

GH, GK. [xi. 4 ]

And EF is parallel to AB ;

therefore AB is also at right angles to the plane through

HG, GK. [xi. 8]

For the same reason
CD is also at right angles to the plane through HG, GK ;
therefore each of the straight lines AB, CD is at right angles
to the plane through HG,. GK.

But, if two straight lines be at right angles to the same
plane, the straight lines are parallel ; [xi. 6]

therefore AB is parallel to CD.
\end{proof}

\end{proposition}

\begin{proposition}
\label{prop:XI_10}

\begin{statement}
If two straight lines meeting one another be parallel to
two straight lines meeting one another not in the same plane,
they will contain equal angles.
\end{statement}

\begin{proof}

For let the two straight lines AB, BC meeting one
another be parallel to the two straight lines DE, EF meeting
one another, not in the same plane ;
I say that the angle ABC is equal to the angle DBF,

For let BA, BC, ED, EF be cut off equal to one another,
and let AD, CF, BE, AC, DF be joined.

Now, since BA is equal and parallel to ED,
therefore AD is also equal and parallel to BE. [1. 33]

For the same reason
CF is also equal and parallel to BE.

Therefore each of the straight lines AD, CF is equal and
parallel to BE.

But straight lines which are parallel to the same straight
line and are not in the same plane with it are parallel to one
another ; [xi. 9]

therefore AD is parallel and equal to CF.

And AC, DF join them ;
therefore AC is also equal and. parallel to DF. [1. 33]

Now, since the two sides AB, BC are equal to the two
sides DE, EF,

and the base AC is equal to the base DF,
therefore the angle ABC is equal to the angle DEF. [1. 8]

Therefore etc.
\end{proof}

\begin{notes}

The result of this proposition does not appear to be quoted in Euclid until
xii. 3; but Euclsd no doubt inserted it here advisedly, because it has the
effect of incidentally proving that the ``inclination of two planes to one
another,'' as defined in XL Def. 6, is one and the same angle at whatever
point of the common section the plane angle measuring it is drawn.

\end{notes}

\end{proposition}

\begin{proposition}
\label{prop:XI_11}

\begin{statement}
Front a given elevated point to draw a straight line perpen-
dicular to a given plane.
\end{statement}

\begin{proof}

Let A be the given elevated point, and the plane of
reference the given plane ;
thus it is required to draw from the
points a straight line perpendicular to
the plane of reference.

Let any straight line BC be drawn,
at random, in the plane of reference,
and let AD be drawn from the point A
perpendicular to BC. [1. 12]

If then AD is also perpendicular to
the plane of reference, that which was
enjoined will have been done.

But, if not, let DE be drawn from the point D at right
angles to BC and in the plane of reference, [1. 1 1]

let AFbe drawn from A perpendicular to DE, [1. 11]

and let GH be drawn through the point F parallel to BC.

[' 3i]

Now, since BC is at right angles to each of the straight
lines DA, DE,

therefore BC is also at right angles to the plane through
ED, DA. [xi. 4 ]

And GH is parallel to it ;
but, if two straight lines be parallel, and one of them be at
right angles to any plane, the remaining one will also be at
right angles to the same plane ; [ate 8]

therefore GH is also at right angles to the plane through
ED, DA.

Therefore GH is also at right angles to all the straight
lines which meet it and are in the plane through ED, DA.

[xi. Def. 3]

But AF meets it and is in the plane through ED, DA ;
therefore GH is at right angles to FA,
so that FA is also at right angles to GH.

But AF is also at right angles to DE ;
therefore AF is at right angles to each of the straight lines
GH, DE.

But, if a straight line be set up at right angles to two
straight lines which cut one another, at the point of section,
it will also be at right angles to the plane through them ; [xi. 4]
therefore FA is at right angles to the plane through ED, GH.

But the plane through ED, GH is the plane of reference ;
therefore AF is at right angles to the plane of reference.

Therefore from the given elevated point A the straight
line AF has been drawn perpendicular to the plane of
reference.

Q.E.F.
\end{proof}

\begin{notes}

The text-books differ in the form which they give to this proposition rather
than in substance. They commonly assume the construction of a plane
through the point A at right angles to any straight line BC in the given plane
(the construction being effected in the manner shown at the end of the note
on xi. 5 above). The advantage of this method is that it enables a
perpendicular to be drawn from a point in the plane also, by the same
construction. (Where the letters for the two figures differ, those referring to
the second figure are put in brackets.)

ft

c

We can Include the construction of the plane through A perpendicular to
BC, and make the whole into one proposition, thus.

BC being any straight line in the given plane MN, draw AD perpendicu-
lar to JC.

In any plane passing through BC but not through /( draw DE at right
angles to BC.

Through DA, DE draw a plane ; this will intersect the given plane MN
in a straight line, as FD (AD).

In the plane AG draw AH perpendicular to FG (AD).

Then AH is the perpendicular required.

In the plane MN, through H in the first figure and A in the second, draw
KL parallel to BC.

Now, since BC is perpendicular to both DA and DE, BC is perpendicular
to the plane AG. [xr. 4]

Therefore KL, being parallel to BC, is also perpendicular to the plane
AG [xi..8], and therefore to AH which meets it and is in that plane.

Therefore AH is perpendicular to both FD (AD) and KL at their point
of intersection.

Therefore AH is perpendicular to the plane MM

Thus we have solved the problem in xt, 12 as well as that in XI. 11; and
this direct method of drawing a perpendicular to a plane from a point in it is
obviously preferable to Euclid's method by which the construction of a
perpendicular to a plane from a point withmtt it is assumed, and a line is
merely drawn from a point in the plane parallel to the perpendicular obtained

\end{notes}

\end{proposition}

\begin{proposition}
\label{prop:XI_12}

\begin{statement}
To set up a straight line at right angles to a given plane
from a given point in it.
\end{statement}

\begin{proof}

Let the plane of reference be the given plane,

and A the point in it;

thus it is required to set up from the point
A a straight line at right angles to the
plane of reference.

Let any elevated point B be conceived,

from B let BC be drawn perpendicular to
the plane of reference, [xi. it]

and through the point A let AD be drawn
parallel to BC. [t. 31]

Then, since AD, CB are two parallel straight lines,
while one of them, BC, is at right angles to the plane of
reference,

therefore the remaining one, AD, is also at right angles to
the plane of reference. [xi. 8]

Therefore AD has been set up at right angles to the given
plane from the point A in ,t.
\end{proof}

\end{proposition}

\begin{proposition}
\label{prop:XI_13}

\begin{statement}
From the same point two straight lines cannot be set up at
right angles to the same plane on the same side.
\end{statement}

\begin{proof}

For, if possible, from the same point A let the two straight
lines AB, AC be set up at right
angles to the plane of reference and on
the same side,

and let a plane be drawn through BA,
AC;

it will then make, as section through >?
in the plane of reference, a straight line.

[xi. 3]
Let it make DAE ;
therefore the straight lines AB, AC, DAE are in one plane.

And, since CA is at right angles to the plane of reference,
it will also make right angles with all the straight lines which
meet it and are in the plane of reference. [xi. Def. 3]

But DAE meets it and is in the plane of reference ;

therefore the angle CAE is right.

For the same reason
the angle BAE is also right ;
therefore the angle CA E is equal to the angle BAE.

And they are in one plane :
which is impossible.
Therefore etc.
\end{proof}

\begin{notes}

Simson added words to this as follows :

`` Also, from a point above a plane there can be but one perpendicular to
that plane ; for, if there could be two, they would be parallel to one another
[xi. 6], which is absurd.''

Euclid does not give this result, but we have already had it in the note
above to xi. 4 (ad fin.).

\end{notes}

\end{proposition}

\begin{proposition}
\label{prop:XI_14}

\begin{statement}
Planes to which the same straight line is at right angles
will be parallel.
\end{statement}

\begin{proof}

For let any straight line AB be at right angles to each of
the planes CD, EF;

I say that the planes are
parallel. /

For, if not, they will meet °~
when produced. r

Let them meet ; (

they will then make, as

common section, a straight line. [xi. 3]

Let them make GH ;
let a point K be taken at random on GH,
and let AK, BK be joined.

Now, since AB is at right angles to the plane EF,

therefore AB is also at right angles to BK which is a straight
line in the plane EF produced ; [xi. Def. 3]

therefore the angle ABK is right.

For the same reason
the angle BAK is also right.

Thus, in the triangle ABK, the two angles ABK, BAK
are equal to two right angles :

which is impossible. [1. 17]

Therefore the planes CD, EF will not meet when
produced ;

therefore the planes CD, EFare parallel. [xi. Def. 8]

Therefore planes to which the same straight line is at right
angles are parallel.
\end{proof}

\end{proposition}

\begin{proposition}
\label{prop:XI_15}

\begin{statement}
If (wo straight lines meeting one another be parallel to two
straight lines meeting one another, not being in the same plane,
the planes through them are parallel.
\end{statement}

\begin{proof}

For let the two straight lines AB, BC meeting one another
be parallel to the two straight lines
DE, EF meeting one another, not
being in the same plane ;
I say that the planes produced
through AB, BC and DE, EFw
not meet one another.

For let BG be drawn from the
point B perpendicular to the plane
through DE, £F[x.i. 11], and let it
meet the plane at the point G ;
through G let GH be drawn
parallel to ED, and GK parallel to EF. [1. 31]

Now, since BG is at right angles to the plane through
DE,EF,

therefore it will also make right angles with all the straight
lines which meet it and are in the plane through DE, EF.

[xi. Def. 3]

But each of the straight lines GH, GK meets it and is in
the plane through DE, EF;
therefore each of the angles BGH, BGK is right.

And, since BA is parallel to GH, [xi. 9]

therefore the angles GBA, BGH are equal to two right angles.

[1. 2 9 ]
But the angle BGH is right ;
therefore the angle GBA is also right ;
therefore GB is at right angles to BA.

For the same reason
GB is also at right angles to BC.

Since then the straight line GB is set up at right angles
to the two straight lines BA, BC which cut one another,
therefore GB is also at right angles to the plane through
BA, BC. [xi. 4]

But planes to which the same straight line is at right
angles are parallel ; [xi. 14]

therefore the plane through AB, BC is parallel to the plane
through DE t EF.

Therefore, if two straight lines meeting one another be
parallel to two straight lines meeting one another, not in the
same plane, the planes through them are parallel.
\end{proof}

\begin{notes}

This result is arrived at in the American text-books already quoted by
starting from the relation between a plane and a straight line parallel to it.
The series of propositions is worth giving. A straight line and a plane being
parallel if they do not meet however far they may be produced, we have the
following propositions.

t. Any plane containing one t and only one, of two parallel straight lines is
parallel to the other.

For suppose AB, CD to be parallel and CD to lie in the plane MN.

Then A B, CD determine a plane intersecting MN in the straight line CD

Thus, if AB meets MN, it must meet
it at some point in CD. fy

But this is impossible, since AB is
parallel to CD.

Therefore A B will not meet the plane
MN, and is therefore parallel to it.

[This proposition and the proof are in
Legendre,]

The following theorems follow as corollaries.

2. Through a given straight line a plane can he drawn parallel to any other
given straight line; and, if the lines are not parallel, only one such plane can he
drawn.

We have simply to draw through any point on the first line a straight line
parallel to the second line and then pass a plane through these two intersecting
lines. This plane is then, by the above proposition, parallel to the second
given straight line.

3. Through a given point a plane can he drawn parallel to any two straight
lines in space ; and, if the tatter are not parallel, only one such plane can be
drawn.

Here we draw through the point straight lines parallel respectively to the
given straight lines and then draw a plane through the lines so drawn.
Next we have the partial converse of the first proposition above.

4. If a straight line is parallel to a plane, it is also parallel to the inter-
section of any plane through it with the given plane.

Let AB be parallel to the plane MN, and let
any plane through AB intersect MNn CD.

Now AB and CD cannot meet, because, if
they did, AB would meet the plane MN,

And AB, CD are in one plane.

Therefore AB, CD are parallel.

From this follows as a corollary :

5. If each of two intersecting straight lines is parallel to a given plane,
the plane containing them is parallel to the given
plane.

Let AB, AC be parallel to the plane
MN.

Tben, if the plane ABC were to meet the
plane MN, the intersection would be parallel
both to AB and to AC; which is impossible.

Lastly, we have Euclid's proposition.

6. If two straight lines forming an angle are respectively parallel to tivo
other straight lines forming an angle, the plane of

the first angle is parallel to the plane of the second.

Let ABC, DEF be the angles formed by
straight lines parallel to one another respectively.

Then, since AB is parallel to DE,
the plane of DEF is parallel to AB [(r) above].

Similarly the plane of DEF is parallel to
BC.

Hence the plane of DEF is parallel to the
plane of ABC [(5)].

Legendre arrives at the result by yet another method. He first proves
Eucl.\ xi. 16 to the effect that, if two parallel planes are cut by a third, the lines
of intersection are parallel, and then deduces from this that, (/ two parallel
straight lines are terminated by tivo parallel planes, the straight lines are equal
in length,

(The latter inference is obvious because the plane through the parallels
cuts the parallel planes in parallel lines, which
therefore, with the given parallel lines, form a
parallelogram.)

Legendre is now in a position to prove
Euclid's proposition Xi. 15.

If ABC, DEF be the angles, make AB
equal to DE, and BC equal to EF, and join
CA, ED, BE, CF, AD.

Then, as in Eucl.\ xi. 10, the triangles
ABC, DEFzie equal in all respects;
and AD, BE, CFa.m all equal.

It is now proved that the planes are
parallel by reduclio ad absurdum from the
last preceding result. For, if the plane ABC

is not parallel to the plane DEF, let the plane drawn through B parallel to the
plane DEF meet CF, AD in H, G respectively.

Then, by the last result BE, HF, GD will all be equal.

But BE, CF, AD are all equal 1
which is impossible.

Therefore etc.

\end{notes}

\end{proposition}

\begin{proposition}
\label{prop:XI_16}

\begin{statement}
If two parallel planes be cut by any plane, their common
sections are parallel.
\end{statement}

\begin{proof}

For let the two parallel planes AB, CD be cut by the
plane EFGH,

and let EF, GH be their common sections ;
I say that EF is parallel to GH.

-)B

For, if not, EF, GH will, when produced, meet either in
the direction of F, H or of E, G.

Let them be produced, as in the direction of F, H, and
let them, first, meet at K.

Now, since EFK is in the plane AB,
therefore all the points on EFK are also in the plane AB.

[XI. ,]

But K is one of the points on the straight line EFK;
therefore K is in the plane AB.

For the same reason
K is also in the plane CD ;
therefore the planes AB, CD will meet when produced.

But they do not meet, because they are, by hypothesis,
parallel ;

therefore the straight lines EF, GH will not meet when
produced in the direction of F t H.

Similarly we can prove that neither will the straight lines
EF, GH meet when produced in the direction of E, G.

But straight lines which do not meet in either direction
are parallel. [i. Def. 23]

Therefore EF is parallel to GH,

Therefore etc.
\end{proof}

\begin{notes}

Simson points oat that, in here quoting 1. Def. 23, Euclid should have
said ``But straight lines in one plant which do not meet in either direction are
parallel''

From this proposition is deduced the converse of xi. 14.

If a straight line is perpendicular to one of two parallel planes, it is
perpendicular to the other also.

For suppose that MN, PQ are two parallel planus, and that AB is perpen-
dicular to MN.

Through AB draw any plane, and let it intersect
the planes MN, PQ in AC, BD respectively. / -T~7N

Therefore AC, BD are parallel. [xi. 16]

But A C is perpendicular to AB ; Hf''

therefore AB is also perpendicular to BD.

That is, AB is perpendicular to any line in PQ
passing through B ;
therefore AB is perpendicular to PQ.

It follows as a corollary that

Through a given point one plane, and only one, aw be drawn parallel to a
given plane.

In the above figure let A be the given point and PQ the given plane.

Draw AB perpendicular to PQ.

Through A draw a plane MN at right angles to AB (see note on xi. 5
above).

Then MNh parallel to PQ. [xi, 14]

If there could pass through A a second plane parallel to PQ, AB would
also be perpendicular to it.

That is, AB would be perpendicular to two different planes through A :
which is impossible (see the same note).

Also it is readily proved that,

If two planes are parallel to a third plane, they are parallel to one another.

\end{notes}

\end{proposition}

\begin{proposition}
\label{prop:XI_17}

\begin{statement}
If two straight lines be cut by parallel planes, (key will be
cut in the same ratios.
\end{statement}

\begin{proof}

For let the two straight
lines AB, CD be cut by the
parallel planes GH, KL, MN
at the points A, E, B and C,
F,D;

I say that, as the straight line
AE is to EB, so is CF to FD.
For let AC, BD, AD be
joined,

let AD meet the plane KL
at the point O,
and let EO, OF be joined.

Now, since the two parallel planes KL, MN are cut by
the plane EBDO,
their common sections EO, BD are parallel. [xi. 16]

For the same reason, since the two parallel planes GH,
KL are cut by the plane AOFC,
their common sections AC, OF are parallel. [id.]

*

And, since the straight line EO has been drawn parallel to
BD, one of the sides of the triangle ABD,

therefore, proportionally, as AE is to EB, so is AO to OD.

[vi. a]

Again, since the straight line OF has been drawn parallel
to AC, one of the sides of the triangle ADC,
proportionally, as AO is to OD, so is CF to FD. [id.]

But it was also proved that, as AO is to OD, so is AE
to EB;
therefore also, as AE is to EB, so is CF to FD. [v. 1 1]

Therefore etc.
\end{proof}

\end{proposition}

\begin{proposition}
\label{prop:XI_18}

\begin{statement}
If a straight line be at right angles to any plane, all the
planes through it will also be at right angles to the same plane.
\end{statement}

\begin{proof}

For let any straight line AB be at right angles to the
plane of reference;

I say that all the planes through n n m
AB are also at right angles to the
plane of reference.

For let the plane DE be drawn
through AB,

let CE be the common section of
the plane DE and the plane of
reference,

let a point F'he. taken at random on CM,
and from F let FG be drawn in the plane DE at right
angles to CE. [1. n]

Now, since AB is at right angles to the plane of reference,
AB is also at right angles to all the straight lines which meet
it and are in the plane of reference ; [xi. Def. 3]

so that it is also at right angles to CE ;
therefore the angle ABF is right.

But the angle GFB is also right ;
therefore AB is parallel to FG. [1. *8]

But AB is at right angles to the plane of reference ;
therefore FG is also at right angles to the plane of reference.

[XL 8]

Now a plane is at right angles to a plane, when the
straight lines drawn, in one of the planes, at right angles to
the common section of the planes are at right angles to the
remaining plane. [xi. Def. 4]

And FG, drawn in one of the planes DE at right angles
to CE, the common section of the planes, was proved to be
at right angles to the plane of reference ;

therefore the plane DE is at right angles to the plane of
reference.

Similarly also it can be proved that all the planes through
AB are at right angles to the plane of reference.
Therefore etc.
\end{proof}

\begin{notes}

Starting as Euclid does from the definition of perpendicular planes as
planes such that all straight lines drawn in one of the planes at right angles to
the common section are at right angles to the other plane, it is necessary for
him to show that, if F be any point in CE, and FG be drawn in the plane
DE at right angles to CE, FG will be perpendicular to the plane to which
AB Is perpendicular.

It is perhaps more scientific to make the definition, as Legendre makes it,
a particular case of the definition of the inclination of plants. Perpendicular
plants would thus be planes such that the angle which (when it is acute)
Euclid calls the inclination of a plane to a plane is a right angle. When to this
is added the fact incidentally proved in xi, 10 that the `` inclination of a plane to
a plane `` is the same at whatever point in their common section it is drawn, it
is sufficient to prove the perpendicularity of two planes if one straight line
drawn, in one of them, perpendicular to their common section is perpendicular
to the other.

If this point of view is taken, Props. 18, 19 are much simplified (cf.
Legendre, H. M. Taylor, Smith and Bryant, Rausenberger, Schultze and
Seven oak, Holgate), The alternative proof is as follows.

Let AB be perpendicular to the plane MN, and CE any plane through
AB, meeting the plane MN'm the straight line CD.

In the plane MNsm BFt right angles to CD.

Then ABF'm the angle which Euclid calls (in the case where it is acute)
the `` inclination of the plane to the plane.''

But, since AB is perpendicular to the plane MN. it is perpendicular to
BF in it.

Therefore the angle ABF is a right angle ;
whence the plane CE is perpendicular to the plane MN.

\end{notes}

\end{proposition}

\begin{proposition}
\label{prop:XI_19}

\begin{statement}
If two planes which cut one another be at right angles to
any plane, their common section will also be at right angles to
the same plane.
\end{statement}

\begin{proof}

For let the two planes AB, BC be at right angles to the
plane of reference,

and let BD be their common section ;
I say that BD is at right angles to the
plane of reference.

For suppose it is not, and from the
point D let DE be drawn in the plane
AB at right angles to the straight line
AD, and DF in the plane BC at right
angles to CD.

Now, since the plane AB is at right
angles to the plane of reference,

and DE has been drawn in the plane AB at right angles to
AD, their common section,

therefore DE is at right angles to the plane of reference.

[xi. Def. 4]

Similarly we can prove that
DF is also at right angles to the plane of reference.

Therefore from the same point D two straight lines have
been set up at right angles to the plane of reference on the
same side :

which is impossible. [xi. 13]

Therefore no straight line except the common section DB
of the planes AB, BC can be set up from the point D at right
angles to the plane of reference.

Therefore etc.
\end{proof}

\begin{notes}

Legendre, followed by other writers already quoted, uses a preliminary
proposition equivalent to Euclid's definition of planes at right angles to one
another.

If two planes are perpendicular to one another, a straight line drawn in one
of them perpendicular to their common section -will be perpendicular to the other.

Let the perpendicular planes CE, MN (figure of last note) intersect in
CD, and let AB be drawn in CE perpendicular to CD.

In the plane MN draw BFsX right angles to CD.

'then, since the planes are perpendicular, the angle A BF (their inclination)
is a right angle.

Therefore AB is perpendicular to both CD and BF, and therefore to the
plane MN.

We are now in a position to prove XI. 19, viz. If two planes lie perpendicular
to a third, their intersection is also perpen-
dicular to that third plane. A

Let each of the two planes AC, AD
intersecting in A B be perpendicular to the
plane MN.

Let A C, AD intersect MN in BC, BD
respectively.

In the plane MN draw BE at right
angles to BC and BF at right angles to
BD.

Now, since the planes AC, MN are at
right angles, and BE is drawn in the latter perpendicular to BC, BE is
perpendicular to the plane AC.

Hence AB is perpendicular to BE. [xi. 4]

Similarly AB is perpendicular to BF.

Therefore AB is perpendicular to the plane through BF, BF, i.e. 10 the
plane MN.

An useful problem is that of drawing a common perpendicular to two
straight lines not in one plane, and in connexion with this the following
proposition may be given.

Given a plane and a straight lint not perpendicular to it, one plant, and only
one, can be drawn through the straight line perpen-
dicular to the plane.

Let AB be the given straight line, MN the
given plane.

From any point C in AB draw CD perpen-
dicular to the plane MN.

Through AB and CD draw a plane AE.

Then the plane AE is perpendicular to the
plane MN [xi. 18]

If any other plane could be drawn through
AB perpendicular to MN, the intersection AB of
the two planes perpendicular to MN would itself
be perpendicular to MN:
which contradicts the hypothesis.

To draw a common perpendicular to two straight lines not in the same plane.

Let AB, CD he the given straight lines.

Through CD draw the plane MN parallel to AB (Prop, a in note
to Xi. 15).

Through AB draw the plane AE perpendicular to the plane MN(ee the
last preceding proposition).

AH KB

[XI. 19]

Let the planes AE, MN intersect in EF, and let EFmX CD in G.

From G, in the plane AE, draw GHai, right angles to EF, meeting AB in H.

GH is then the required perpendicular.

For AB it parallel to EF (Prop. 4 in note to xi. 15) ; therefore GH,
being perpendicular to EF, is also perpendicular to AB,

But, the plane AE being perpendicular to 'he plane MN, and GH being
perpendicular to EF, their intersection,

GH is perpendicular to the plane MN, and therefore to CD.

Therefore GH is perpendicular to both AB and CD.

Only one common perpendicular can be drawn to two straight lines not in
one plane.

For, if possible, let KL also be perpendicular to both AB and CD.

Let the plane through KL, AB meet the plane MN in LQ,

Then AB is parallel to LQ (Prop. 4 in note to xi. 15), so that KL, being
perpendicular to AB, is also perpendicular to LQ.

Therefore KL is perpendicular to both CL and LQ, and consequently to
the plane MN.

But, if KP be drawn in the plane AF perpendicular to EF, KP is also
perpendicular to the plane MN.

Thus there are two perpendiculars from the point K to the plane MN:
which is impossible.

Rausenberger's construction for the same problem is more elegant
he says, through each straight line a plane parallel
to the other. Then draw through each straight line
a plane perpendicular to the plane through the
other. The two planes last drawn will intersect
in a straight line, and this straight line is the
common perpendicular required.

Draw,

CDF,

The form of the construction best suited for
examination purposes, because the most self-
contained, is doubtless that given by Smith and
Bryant.

Let AB, CD be the two given straight lines.

Through any point E in CD draw EF parallel to AB.

From any point G in AB draw GH perpendicular to the plane
meeting the plane in H.

Through H in the plane CDF draw
HK parallel to FE or AB, to cut CD
in K.

Then, since AB, HK are parallel,
AGHK is a plane.

Complete the parallelogram GHKL.

Now, since LK, GHare parallel, and
GH is perpendicular to the plane CDF,

LK is perpendicular to the plane
CDF.

Therefore LK is perpendicular to CD and KH, and therefore loAB which
is parallel to KH

\end{notes}

\end{proposition}

\begin{proposition}
\label{prop:XI_20}

\begin{statement}
If a solid angle be contained by three plane angles, any two,
taken together in any manner, are greater than the remaining one.
\end{statement}

\begin{proof}

For let the solid angle at A be contained by the three
plane angles BAC, CAD, DAB ;
I say that any two of the angles
BAC, CAD, DAB, taken to-
gether in any manner, are greater
than the remaining one.

If now the angles BAC, CAD,
DAB are equal to one another,
it is manifest that any two are greater than the remaining one.

But, if not, let BAC be greater,
and on the straight line AB, and at the point A on it, let the

angle BAE be constructed, in the plane through BA, AC,

equal to the angle DAB ;

let AE be made equal to AD,

and let BEC, drawn across through the point E, cut the

straight lines AB, AC at the points B, C

let DB, DC be joined.

Now, since DA is equal to AE,
and AB is common,
two sides are equal to two sides ;
and the angle DAB is equal to the angle BAE ;
therefore the base DB is equal to the base BE. [i. 4]

And, since the two sides BD, DC are greater than BC,

[1. »]
and of these DB was proved equal to BE,

therefore the remainder DC is greater than the remainder EC,

Now, since DA is equal to AE,
and AC is common,

and the base DC is greater than the base EC,
therefore the angle DAC is greater than the angle EAC.

Li- »«]

But the angle DAB was also proved equal to the angle
BAE;

therefore the angles DAB, DAC are greater than the angle
BAC.

Similarly we can prove that the remaining angles also,
taken together two and two, are greater than the remaining
one.

Therefore etc
\end{proof}

\begin{notes}

After excluding the obvious case in which all three angles are equal,
Euclid goes on to say ``If not, let the angle BAC be greater,'' without adding
greater than what. Heiberg is clearly right in saying that he means greater
than BAD, i.e. greater than one of the adjacent angles. This is proved by
the words at the end `` Similarly we can prove,'' etc. Euclid thus excludes
as obvious the case where one of the three angles is not greater than either of
the other two, but proves the remaining cases. This is scientific, but he might
further have excluded as obvious the case in which one angle is greater than
one of the others but equal to or less than the remaining one.

Si m 50 n remarks that the angle BAC may happen to be equal to one of
the other two and writes accordingly `` If they [all three angles] are not [equal],
let BAC be that angle which is not less than either of the other two, and is
greater than one of them DAB''' He then proves, in the same way as Euclid
does, that the angles DAB, DAC are greater than the angle BAC, adding
finally : ``But BAC is not less than either of the angles DAB, DAC; there-
fore BAC, with either of them, is greater than the other.''

It would he better, as indicated by Legendre and Rausenberger, to begin
by saying that, ``If one of the three angles is either equal to or less than either
of the other two, it is evident that the sum of those two 15 greater than the
6rst. It is therefore only necessary to prove, for the case in which one angle is
greater than each of the others, that the sum of the two latter is greater than
the former.

Accordingly let BA C be greater than each of the other angles.'' We then
proceed as in Euclid.

\end{notes}

\end{proposition}

\begin{proposition}
\label{prop:XI_21}

\begin{statement}
Any solid angle is contained by plane angles less than four
right angles.
\end{statement}

\begin{proof}

Let the angle at A be a solid angle contained by the plane
angles BAC, CAD, DAB ;
I say that the angles BAC, CAD,
DAB are less than four right angles.

For let points B, C, D be taken
at random on the straight lines AB,
AC, AD respectively.
and let BC, CD, DB be joined. B

Now, since the solid angle at B is contained by the three
plane angles CBA, ABD, CBD,

any two are greater than the remaining one ; [xi. 10]

therefore the angles CBA, ABD are greater than the angle
CBD.

For the same reason
the angles BCA, ACD are also greater than the angle BCD,
and the angles CD A, ADB are greater than the angle CDS ;
therefore the six angles CBA, ABD, BCA, ACD, CD A,
ADB are greater than the three angles CBD, BCD, CDB.

But the three angles CBD, BDC, BCD are equal to two
right angles ; [1. 3*]

therefore the six angles CBA, ABD, BCA, ACD, CD A,
ADB are greater than two right angles.

And, since the three angles of each of the triangles ABC,
A CD, ADB are equal to two right angles,
therefore the nine angles of the three triangles, the angles
CBA, ACB, BAC, ACD, CD A, CAD, ADB, DBA, BAD
are equal to six right angles ;

and of them the six angles ABC, BCA t ACD, CD A, ADB,
DBA are greater than two right angles ;
therefore the remaining three angles BAC, CAD, DAB
containing the solid angle are less than four right angles.

Therefore etc.
\end{proof}

\begin{notes}

It will be observed that, although Euclid enunciates this proposition for
any solid angle, he only proves it for the particular case of a trihedral angle.
This is in accordance with his manner of proving one case and leaving the
others to the reader. The omission of the convex polyhedral angle here
corresponds to the omission, after i. 32, of the proposition about the interior
angles of a convex polygon given by Prod us and in most books. The proof
of the present proposition for any convex polyhedral angle can of course be
arranged so as not to assume the proposition that the interior angles of a
convex polygon together with four right angles are equal to twice as many
right angles as the figure has sides.

Let there be any convex polyhedral angle with V as vertex, and let it be
cut by any plane meeting its faces in, say, the
polygon ABCDE.

Take O any point within the polygon, and
in its plane, and join OA, OB, OC, OD, OE.

Then all the angles of the triangles with
vertex O are equal to twice as many right angles
as the polygon has sides ; [1. 3a]

therefore the interior angles of the polygon to-
gether with all the angles round are equal to
twice as many right angles as the polygon has
sides.

Also the sum of the angles of the triangles
VAB, VBC, etc, with vertex P'are equal to twice as many right angles as the
polygon has sides ;

and all the said angles are equal to the sum of (1) the plane angles at V
forming the polyhedral angle and (2) the base angles of the triangles with
vertex V.

This latter sum is therefore equal to the sum of (3) all the angles
round O and (4) all the interior angles of the polygon.

Now, by Euclid's proposition, of the three angles forming the solid angle at
A, the angles VAE, VAB are together greater than the angle EAB.

Similarly, at B, the angles VBA, VBC are together greater than the angle
ABC.

And so on.

Therefore, by addition, the base angles of the triangles with vertex V

!(x) above] are together greater than the sum of the angles of the polygon
(4) above].

Hence, by way of compensation, the sum of the plane angles at V[(i)
above] is less than the sum of the angles round [(3) above].

But the latter sum is equal to four right angles; therefore the plane angles
forming the polyhedral angle are together less than four right angles.

The proposition is only true of convex polyhedral angles, i.e. those in
which the plane of any face cannot, if produced, ever cut the solid angle.

There are certain propositions relating to equal (and symmetrical) trihe-
dral angles which are necessary to the consideration of the polyhedra dealt
with by Euclid, all of which (as before remarked) have trihedral angles only.

1. Two trihedral angles are equal if two face angles and the included
dihedral angle of the one are respectively equal to two face angles anil the included
dihedral angle of the other, the equal parts being arranged in the same order.

2. Two trihedral angles are equal if two dihedral angles and the included
face angle of the one are respectively equal to two dihedral angles and the included
fate angle of the other, all equal parts being arranged in the same order.

These propositions are proved immediately by superposition.

3. Two trihedral angles are equal if the three face angles of the one are
respectively equal to the three face angles of the other, and all are arranged in the
same order.

Let V— ABC and V—A'B'C be two trihedral angles such that the angle
A VB is equal to the angle A' VB', the angle BVC to the angle B'V'C, and
the angle CVA to the angle C V'A'.

We first prove that corresponding pairs of face angles include equal dihedral
angles.

E.g, the dihedral angle formed by the plane angles CVA, AVB is equal
to that formed by the plane angles C V A , A' VB'.

Take points A, B, C on VA, VB, VC and points A', If, C on V'A',
VB 1 , VC, such that VA, VB, VC. V'A', VB\ VC are all equal.

Join BC, CA, AB, SC, C'A\ A'B'.

Take any point Don AV, and measure A'D' along A' V equal to AD.

From D draw DE in the plane AVB, and DF in the plane CVA,
perpendicular to A V. Then DE, DF will meet AB, AC respectively, the
angles VAB, VAC, the base angles of two isosceles triangles, being i-ss than
right angles.

Join EF.

Draw the triangle DEF in the same way.

Now, by means of the hypothesis and construction, it appears that the
triangles VAB, V'A'B'' are equal in all respects.

So are the triangles VAC, VA'C', and the triangles VBC, VBC.

Thus BC, CA, AB are respectively equal to B'C, C'A\ A'B'', and the
triangles ABC, A'ffC are equal in all respects.

Now, in the triangles ADE, A'DE,
the angles ADE, DAE are equal to the angles A'DE, DA'S respectively,
and AD is equal to A'D.

Therefore the triangles ADE, A'DE are equal in all respects.

Similarly the triangles ADE, A'DE are equal in all respects.

Thus, in the triangles AEE, A'E'F',
EA, AEsre respectively equal to EA', A'E' t
and the angle EAF is equal to the angle E'A'E (from above) ;
therefore the triangles AEE, A' E E are equal in all respects.

Lastly, in the triangles DEE, DE'F, the three sides are respectively
equal to the three sides *
therefore the triangles are equal in all respects.

Therefore the angles EDE, EDE'' are equal.

But these angles are the measures of the dihedral angles formed by the
planes CVA, A VB and by the planes CVA', A' VB'' respectively.
Therefore these dihedral angles are equal.

Similarly for the other two dihedral angles.

Hence the trihedral angles coincide if one is applied to the other ;
that is, they are equal.

To understand what is implied by `` taken in the same order `` we may
suppose ourselves to be placed at the vertices, and to take the faces in clock-
wise direction, or the reverse, for both angles.

If the face angles and dihedral angles are taken in reverse directions, i.e.
in clockwise direction in one and in counterclockwise direction in the other,
then, if the other conditions in the above three propositions are fulfilled, the
trihedral angles are not equal but symtnetrical.

If the faces of a trihedral angle be produced beyond trie vertex, they form
another trihedral angle. It is easily seen that these vertical trihedral angles
are symmetrical.

\end{notes}

\end{proposition}

\begin{proposition}
\label{prop:XI_22}

\begin{statement}
Jf there be three plane angles of ``which two, taken together
in any manner, are greater than the remaining one, and they
are contained by equal straight lines, it is possible to construct
a triangle out of the straight lines joining the extremities of
the equal straight lines.
\end{statement}

\begin{proof}

Let there be three piane angles ABC, DEF, GHK, of
which two, taken together in any manner, are greater than
the remaining one, namely

the angles ABC, DEF greater than the angle GHK,
the angles DEF, GHK greater than the angle ABC,

and, further, the angles GHK, ABC greater than the angle

DEF;

let the straight lines AB, BC, DE, EF, GH, HK be equal,

and let AC, DF, GK be joined ;

I say that it is possible to construct a triangle out of straight
lines equal to AC, DF, GK, that is, that any two of the
straight lines AC, DF, GK are greater than the remaining
one.

Now, if trre angles ABC, DEF, GHK are equal to one
another, it is manifest that, AC, DF, GK being equal also,
it is possible to construct a triangle out of straight lines equal
to AC, DF,GK.

But, if not, let them be unequal,
and on the straight line HK, and at the point H on it, let
the angle KHL be constructed equal
to the angle ABC;
let HL be made equal to one of the
straight lines AB, BC, DE, EF, GH,
HK,
and let KL, GL be joined.

Now, since the two sides AB, BC
are equal to the two sides KH, HL,
and the angle at B is equal to the angle KHL,
therefore the base AC is equal to the base KL.

And, since the angles ABC, GHK are greater than the
angle DEF,

while the angle ABC is equal to the angle KHL,
therefore the angle GHL is greater than the angle DEF.

And, since the two sides GH, HL are equal to the two
sides DE, EF,

and the angle GHL is greater than the angle DEF,
therefore the base GL is greater than the base DF. [t. 84]

But GK, KL are greater than GL.

Therefore GK, KL are much greater than DF.

But KL is equal to AC;
therefore AC. GK are greater than the remaining straight
line DF.

Similarly we can prove that
AC, DFare greater than GK,
and further DF, GK are greater than A C.

Therefore it is possible to construct a triangle out of
straight lines equal to AC, DF, GK.
\end{proof}

\begin{notes}

The Greek text gives an alternative proof, which is relegated by Heiberg
to the Appendix. Simson selected the alternative proof in preference to that
given above ; he objected however to words near the beginning, `` If not, let
the angles at the points B, E, H be unequal and that at B greater than either
of the angles at B, H,'' and altered the words so as to take account of the
possibility that the angle at B might be equal to one of the other two.

As will be seen, Euclid takes no account of the relative magnitude of the
angles except as regards the case when all three are equal. Having proved
that one base is less than the sum of the two others, he says that `` similarly
we can prove'' the same thing for the other two bases.

If a distinction is to be made according to the relative magnitude of the
three angies, we may say, as in the corresponding place in xi. 21, that, if one
of the three angles is either equal to or less than either of the other two, the
bases subtending those two angles must obviously be together greater than the
base subtending the first. Thus it is only necessary to prove, for the case in
which one angle is greater than either of the others, that the sum of the bases
subtending those others is greater than that subtending the first. This is
practically the course taken in the interpolated alternative proof.

\end{notes}

\end{proposition}

\begin{proposition}
\label{prop:XI_23}

\begin{statement}
To construct a solid angle out of three plane angles two oj
which, taken together in any manner, are greater than the
remaining one : thus the three angles must be less than four
right angles.
\end{statement}

\begin{proof}

Let the angles ABC, DEF, GHK be the three given
plane angles, and let two of these, taken together in any
manner, be greater than the remaining one, while, further,
the three are less than four right angles ;

thus it is required to construct a solid angle out of angles
equal to the angles ABC, DBF, GHK.

Let AB, BC, DE, EF, GH, HK be cut off equal to one
another,

and let AC, DF, GK be joined ;

it is therefore possible to construct a triangle out of straight
lines equal to AC, DF, GK. [xi. is]

Let LMN be so constructed that
AC is equal to LM, DF to MN,an<
further GK to NL,
let the circle LMNbe described about
the triangle LMN,
let its centre be taken, and let it be 0;
let LO, MO, NO be joined ;
I say that AB is greater than LO.

For, if not, AB is either equal to LO, or less.

First, let it be equal.

Then, since AB is equal to LO,

while AB is equal to BC, and OL to OM,

the two sides AB, BC are equal to the two sides LO. OM
respectively ;

and, by hypothesis, the base AC is equal to the base LM ;
therefore the angle ABC is equal to the angle L OM. [1. 8]

For the same reason
the angle DEF is also equal to the angle MON,
and further the angle GHK to the angle NOL ;

3»6 BOOK XI [xi. 23

therefore the three angles ABC, DEF, GHK are equal to
the three angles LOM, MON, NOL,

But the three angles LOM, MON, NOL are equal to
four right angles ;

therefore the angles ABC, DEF, GHK are equal to four
right angles.

But they are also, by hypothesis, less than four right angles :
which is absurd.

Therefore AB is not equal to LO.

I say next that neither is AB less than LO.

For, if possible, let it be so,
and let OP be made equal to AB, and OQ equal to BC,
and let PQ be joined.

Then, since AB is equal to BC,
OP is also equal to OQ,
so that the remainder LP is equal to QM.

Therefore LM is parallel to PQ, (vi. a]

and LMO is equiangular with PQO ; [1. 39]

therefore, as OL is to LM, so is OP to PQ ; [vi. 4]

and alternately, as LO is to OP, so is LM to PQ. [v. 16]

But LO is greater than OP ;
therefore LM is also greater than PQ.

But LM was made equal to AC;
therefore AC is also greater than PQ.

Since, then, the two sides AB, BC are equal to the two

sides PO, OQ,

and the base AC is greater than the base PQ,

therefore the angle ABC is greater than the angle POQ.

['  2 S]
Similarly we can prove that

the angle DEF is also greater than the angle MON,

and the angle GHK greater than the angle NOL.

Therefore the three angles A BC, DEF, GHK ak greater
than the three angles LOM, MON, NOL.

But, by hypothesis, the angles ABC, DEF, GHK are
less than four right angles;

therefore the angles LOM, MON, NOL are much less than
four right angles.

But they are also equal to four right angles :

which is absurd.

Therefore AB is not less than LO.

And it was proved that neither is it equal ;

therefore AB is greater than LO.

Let then OR be set up from the point at right angles
to the plane of the circle LMN, [xi. u]

and let the square on OR be equal to that area by which
the square on AB is greater than the square on LO ; [Lemma]
let RL, RM, RN be joined.

Then, since RO is at right angles to the plane of the circle

LMN,

therefore RO is also at right angles to each of the straight
lines LO, MO, NO.

And, since LO is equal to OM,

while OR is common and at right angles,

therefore the base RL is equal to the base RM. [1. 4]

For the same reason

RN is also equal to each of the straight lines RL, RM ;

therefore the three straight lines RL, RM, RN are equal to
one another.

Next, since by hypothesis the square on OR is equal to
that area by which the square on AB is greater than the
square on LO,
therefore the square on A B is equal to the squares on LO, OR,

But the square on LR is equal to the squares on LO, OR,
for the angle LOR is right ; [1. 47]

therefore the square on AB is equal to the square on RL ;
therefore AB is equal to RL.

But each of the straight lines BC, DE, EF, GH, HK'is
equal to AB,

while each of the straight lines RM, RN is equal to RL ;
therefore each of the straight lines AB, BC, DE, EF, GH,
HK is equal to each of the straight lines RL, RM, RN.

And, since the two sides LR, RM are equal to the two
sides AB, BC,

and the base LM is by hypothesis equal to the base A C,
therefore the angle LRM is equal to the angle ABC. [t- 8]

For the same reason
the angle MRN is also equal to the angle DEF,
and the angle LRN to the angle GHK.

Therefore, out of the three plane angles LRM, MRN,
LRN, which are equal to the three given angles ABC, DEF,
GHK, the solid angle at R has been constructed, which is
contained by the angles LRM, MRN, LRN.

Q.E.F.
\end{proof}

\begin{lemma*}

But how it is possible to take the square on OR equal to
that area by which the square on AB is
greater than the square on LO, we can show
as follows.

Let the straight lines AB, LO be
set out,
and let AB be the greater ;
let the semicircle ABC be described on AB,
and into the semicircle ABC let AC be fitted equal to the
straight line LO, not being greater than the diameter AB;yv. i]
let CB be joined

Since then the angle ACB is an angle in the semicircle
ACB,
therefore the angle ACB is right. [hi. 31]

Therefore the square on AB is equal to the squares on
AC,CB. [1.47]

Hence the square on AB is greater than the square on
AC by the square on CB.

But AC is equal to LO.

Therefore the square on AB is greater than the square on
LO by the square on CB.

If then we cut off OR equal to BC, the square on AB will
be greater than the square on LO by the square on OR.

Q.E.F.

\end{lemma*}

\begin{notes}

The whole difficulty in this proposition is the proof of a fact which makes
the construction possible, viz. the fact that, if LMN\ a triangle with sides
respectively equal to the bases of the isosceles triangles which have the
given angles as vertical angles and the equal sides all of the same length, then
one of these equal sides, as AB, is greater than the radius LO of the circle
circuit) scribing the triangle LMN

Assuming that AB is greater than LO, we have only to draw from O a
perpendicular OR to the plane of the triangle LMN, to make OR of such a
length that the sum of the squares on LO, OR is equal to the square on AB,
and to join RL, RM, RN. (The manner of finding OR such that the square
on it is equal to the difference between the squares on AB and LO is shown
in the Lemma at the end of the text of the proposition. We have already
had the same construction in the Lemma after x. 1 3.)

Then clearly RL, RM, RN are equal to AB and to one another [1. 4
and 1. 47].

Therefore the triangles LRM, MRN, NRL have their three sides
respectively equal to those of the triangles ABC, DEF, GHK respectively.

Hence their vertical angles are equal to the three given angles respectively;
and the required solid angle is constructed.

We return now to the proposition to be proved as a preliminary to the
construction, viz. that, in the figures, AB is greater than LO.

It will be observed that Euclid, as his manner is, proves it for one case
only, that, namely, in which 0, the centre of the circle circumscribing the
triangle LMN, falls within the triangle, leaving the other cases for the reader
to prove. As usual, however, the two other cases ate found in the Greek text,
after the formal conclusion of the proposition, as above, ending with the words
Swtp (Set iroi>/<rai. This position for the proofs itself suggests that they are not
Euclid's but are interpolated ; and this is rendered certain by the fact that
words distinguishing three cases at the point where the centre of the
circumscribing circle is found, `` It [the centre] will then be either within the
triangle LMN or on one of its sides or without. First let it be within,'' are
found in the mss. B and V only and are manifestly interpolated. Nevertheless
the additional two cases must have been inserted very early, as they are found
in all the best mss.

In order to give a clear view ol the proof of all three cases as given in the
text, we will reproduce all three (Euclid's as well as the others) with abbrevia-
tions to make them catch the eye better.

In all three cases the proof is by reduetio ad cbsurdum, and it is proved
first that AB cannot be equal to LO, and secondly that AB cannot be less
than LO.

Case I.

(1) Suppose, if possible, that AB - LO,

Then AB, BC are respectively equal to LO, OM;
and AC - LM(by construction).

Therefore 4 ABC - u LOM.

Similarly L DEF= l MON,

i-GHK=t.NOL,
Adding, we have

lABC + l DEF+ L GHK= U LOM+ L MON + L NOL
m four right angles :
which contradicts the hypothesis.
Therefore AB*LO.

[xi. 23

(a) Suppose that AS < LO.

Make OP, OQ (measured along OL, OM) each equal to AB.

Thus, OL, OM being equal also, it follows that
PQ is || to LM.

Hence LM ': PQ = LO i OP;

and, since LO > OP,

LM,i,e.AC,> PQ.

Thus, in As POQ, ABC, two sides aie equal to two sides, and base
AC> base PQ;

therefore l ABC > l POQ, i.e. L LOM.

Similarly l. DEF> l. MON,

L GHK> L NOL,
and it follows by addition that

l ABC+l. DEF+ l GHK> (four right angles)
which again contradicts the hypothesis.

Case II.

(i) Suppose, if possible, that AB~ LO.

Then (AB + BC), or (DE + EF) = M0+0L

= MN
= DF:
which contradicts the hypothesis.
(2) The supposition that AB < LO is even more
impossible ; for in this case it would result that
DE + EF<. DF.

Case III.

(1) Suppose, if possihle, that AB = LO.

Then, in the triangles ABC, LOM, two sides AB, BC are respectively
equal to two sides LO, OM, and the bases
A C, LM are equal ;
therefore l ABC= t- LOM.

Similarly lGHK=lNOL.
Therefore, by addition,

L MON= L ABC+ l GHK

>i.DEF(by hypothesis).
But, in the triangles DEF, MON, which
are equal in all respects,

i. M0N= l DEF
But it was proved that L MON> L DEF:
which is impossible.

(2) Suppose, if possible, that AB < LO.
Along OL, OM measure OP, OQ each equal to AB,

Then LM, PQ are parallel, and

LM:PQ=LO: OP,
whence, since LO> OP,

LM, or AC, > .P(?.
Thus, in the triangles ABC, POQ,

lA£C>lPOQ, i.e. lLOM.
Similarly, by taking OR along ON equal to .4.5, we prove that

i-GHK>t-LON.
Now, at 0, make lPOS equal to lABC, and lPOT equal to

Make OS, CTeach equal to 0/'', and join ST, SP, TP.
Then, in the equal triangles ABC, POS,

AC=PS,

so that LM= PS.

Similarly LN - PT.

Therefore in the triangles MLN, SPT, since lMLN> lSPT [this is
assumed, but should have been explained],

MAT> ST,
or DF>ST.

Lastly, in As DEF, SOT, which have two sides equal to two sides, since
DF> ST,

l L>EF> l SOT

> L ABC + l GHK (by construction) :
which contradicts the hypothesis.

Simson gives rather different proofs for all tliree cases ; but the essence of
them can be put, I think, a little more shortly than in his text, as well as more
clearly

Case I. (O within ALMN.)
(i) Let AB be, if possible, equal to LO.

Then the as ABC, DEF, GHK must be identically equal to the As
LOM, AfON, NOL respectively.

H

K 6 F

Therefore the vertical angles at in the

latter triangles are equal respectively to the angles

at B, E, H.

The tatter are therefore together equal to four

right angles ;

which is impossible.

(2) If AB be less than LO, construct on the

bases LM, MN, NL triangles with vertices

P, Q, R and identically equal to the As ABC,

DEF, GHK respectively.

3« BOOK XI [xi. 13

Then P, Q, R will fall within the respective angles at O, since PL = PM
and < LO, and similarly in the other cases.

Thus [1. ii] the angles at P, Q, R are respectively greater than the angles
at O in which they lie.

Therefore the sum of the angles at P, Q, R, i.e. the sum of the angles at
B, E, H, is greater than four right angles :
which again contradicts the hypothesis.

Case II. (O lying on MN.)

In this case, whether (1) AS - LO, or (2) A B < LO, a triangle cannot
be formed with MN as base and each of the other sides equal to AB. In other
words, the triangle DEF either reduces to a straight line or is impossible.

H

Case III. (O lying outside the LMN.)

(1) Suppose, if possible, that AB = LO.

Then the triangles LOM, MON, NOL are identically equal to the
triangles ABC, DEF, GHK.

Since L LOM+ 1. LON= l MON,

lABC + lGHK=lDEF:

which contradicts the hypothesis.

(a) Suppose that AB < OL.

Draw, as before, on LM, MN, NL as bases triangles with vertices P, Q, R
and identically equal to the As ABC, DEF, GHK.

Next, at N on the straight line NR, make L RNS equal to the angle
PLM, cut off NS equal to Zand join RS, LS.

Then A NRSis identically equal to A LPAfot A ABC,

Now ( l LNR + L RNS) < ( i, NLO + l. OLM),

that is, i. LNS < l NLM.

Thus, in As LNS, NLM, two sides are equal to two sides, and the included
angle in the former is less than the included angle in the other.

Therefore LS<MN.

Hence, in the triangles MQN, LRS, two sides are equal to two sides, and
MN>LS..

Therefore

That is,
which is impossible.

l MQN  > l. LRS

>(lLRN+lSRN)
> \ l LRN± l. LPM).

l. DEF> ( l. GHK+ L ABC) :

\end{notes}

\end{proposition}

\begin{proposition}
\label{prop:XI_24}

\begin{statement}
If a solid be contained by parallel planes, the opposite planes
in it are equal and parallelogrammic.
\end{statement}

\begin{proof}

For let the solid CDHG be contained by the parallel planes
AC, GF, AH. DF, BF t A£;
I say that the opposite planes
in it are equal and parallelo-
grammic.

For, since the two parallel
planes BG, CE are cut by the
plane AC,

their common sections are
parallel. [xl 16]

Therefore AB is parallel to DC.

Again, since the two parallel planes BF, AE are cut by
the plane AC,
their common sections are parallel. [xi. 16]

Therefore BC is parallel to AD.

But AB was also proved parallel to DC ;
therefore A C is a parallelogram.

Similarly we can prove that each of the planes DF, FG,
GB, BF, AE is a parallelogram.

Let AH, DF be joined.

Then, since AB is parallel to DC, and BH to CF,
the two straight lines AB, BH which meet one another are
parallel to the two straight lines DC, CF which meet one
another, not in the same plane ;

therefore they will contain equal angles ; [xi. 10]

therefore the angle ABH is equal to the angle DCF.

And, since the two sides AB, BH are equal to the two
sides DC, CF, [1. 34]

and the angle ABH is equal to the angle DCF,
therefore the base AH 1 equal to the base DF,
and the triangle ABH is equal to the triangle DCF, [1. 4]

And the parallelogram BG is double of the triangle ABH,
and the parallelogram CE double of the triangle DCF; [1. 34]
therefore the parallelogram BG is equal to the parallelo-
gram CE.

Similarly we can prove that
A C is also equal to GF,
and AE to BF.

Therefore etc.
\end{proof}

\begin{notes}

As Heiberg says, this proposition is carelessly enunciated. Euclid means
a solid contained by six planes and not more, the planes are parallel two and
two, and the opposite faces are equal in the sense of identically equal, or, as
Simson puts it, equal and similar. The similarity is necessary in order to
enable the equality of the parallelepipeds in the next proposition to be inferred
from the roth definition of Book xi. Hence a better enunciation would be:

If a solid be contained by six planes parallel two and two, the opposite faces
respectively are equal and similar parallelograms.

The proof is simple and requires no elucidation.

\end{notes}

\end{proposition}

\begin{proposition}
\label{prop:XI_25}

\begin{statement}
If a parallelepipedal solid be cut by a plane ``which is
parallel to the opposite planes, then, as the base is to the base, so
wilt the solid be to the solid.
\end{statement}

\begin{proof}

For let the parallelepipedal solid A BCD be cut by the
plane FG which is parallel to the opposite planes RA, DH

I say that, as the base A EFV is to the base ENCF, so is the
solid ABFUX.Q the solid EGCD.

L K A e

For let AH be produced in each direction,

let any number of straight lines whatever, AK, KL, be made
equal to AE,

and any number whatever, HM, MN, equal to EH;

and let the parallelograms LP, KV, HW, MS and the solids
LQ, KR, DM, MTbe completed.

Then, since the straight lines LK, KA, AE are equal to
one another,

the parallelograms LP, KV, A Fare also equal to one another,
KO, KB, AG are equal to one another,

and further LX, KQ, AR are equal to one another, for they
are opposite. [xi. 24]

For the same reason

the parallelograms EC, HW, MS are also equal to one another,

HG, HI, IN are equal to one another,

and further DH, MY, NT are equal to one another.

Therefore in the solids LQ, KR, AU three planes are
equal to three planes.

But the three planes are equal to the three opposite ;
therefore the three solids LQ, KR, AU are equal to one
another.

For the same reason
the three solids ED, DM, MT are also equal to one another.

Therefore, whatever multiple the base LF is of the base
AF t the same multiple also is the solid L U of the solid A U.

For the same reason,
whatever multiple the base NF is of the base FH, the same
multiple also is the solid NU of the solid HU.

And, if the base LF is equal to the base NF, the solid L U
is also equal to the solid NU;

if the base LF exceeds the base NF, the solid LU also
exceeds the solid NU;
and, if one falls short, the other falls short.

Therefore, there being four magnitudes, the two bases
AF, FH, and the two solids A U, UN,
equimultiples have been taken of the base AF and the solid
A U, namely the base LF and the solid L U,
and equimultiples of the base NF and the solid NU, namely
the base NF and the solid NU,

and it has been proved that, if the base LF exceeds the base
FN, the solid LU also exceeds the solid NU,
if the bases are equal, the solids are equal,
and if the base falls short, the solid falls short.

Therefore, as the base AF is to the base FN, so is the
solid AUto the solid UN. [v, Def. 5]
\end{proof}

\begin{notes}

It is to be observed that, as the word p arollelogrammie was used in Book I.
without any definition of its meaning, so n-opoXX'ifA.tTtiriSos, paralkkpipedal, is
here used without explanation. While it means simply ``with parallel planes,''
i.e. `` faces,'' the term is appropriated to the particular solid which has six
plane faces parallel two and two. The proper translation of trMpeo*
TapaWifXtriirtSoir is paralUlepipedal solid, not solid paralltkpiped, as it is
usually translated. Still less is the solid a parallelepiped, as the word is not
uncommonly written.

The opposite faces in each set of parallelepipedal solids in this proposition
are not only equal but equal and similar. Euclid infers that the solids in each
set are equal from Def. 10; but, as we have seen in the note on Deft*. 9, io,
though it is true, where no solid angle in the figures is contained by more
than three plane angles, that two solid figures are equal and simitar which are
contained by the same number of equal and similar faces, similarly arranged,
the fact should have been proved. To do this, we hare only to prove the
proposition, given above in the note on XI. z I, that two trihedral angles are
equal if the three face angks of the one are respectively equal to the three face
angles in the other, and all are arranged in the same order, and then to prove
equality by applying one figure to the other as is done by Simson in his
proposition C.

Application will also, of course, establish what is assumed by Euclid of
the solids formed by the multiples of the original solids, namely that, if
LF  NF, the solid ZC/  the solid JVCl.

\end{notes}

\end{proposition}

\begin{proposition}
\label{prop:XI_26}

\begin{statement}
On a given straight line, and at a given point on it, to
construct a solid angle equal to a given solid angle.
\end{statement}

\begin{proof}

Let AB be the given straight line, A the given point on
it, and the angle at D, contained by the angles EDC, EDF,
FDC, the given solid angle ;

thus it is required to construct on the straight line AB, and at
the point A on it, a solid angle equal to the solid angle at D.

For let a point F be taken at random on DF,
let FG be drawn from F perpendicular to the plane through
ED, DC, and let it meet the plane at G, [xi. u]

let DG be joined,

let there be constructed on the straight line AB and at the
point A on it the angle BAL equal to the angle EDC, and
the angle BAK equal to the angle EDG, [1. 23]

let AK be made equal to DG,

let KH be set up from the point K at right angles to the
plane through BA, AL, [xi. uj

let KH be made equal to GF,

and let HA be joined ;

I say that the solid angle at A, contained by the angles BAL,
BAH, HAL is equal to the solid angle at D contained by
the angles EDC, EDF, FDC.

For let AB, DE be cut off equal to one another,
and let HB, KB, FE, GE be joined.

Then, since FG is at right angles to the plane of reference,
it will also make right angles with all the straight lines which
meet it and are in the plane of reference ; [xt Def- 3]

therefore each of the angles FGD, FGE is right.

For the same reason
each of the angles HKA, HKB is also right.

And, since the two sides KA, AB are equal to the two
sides GD, DE respectively,

and they contain equal angles,

therefore the base KB is equal to the base GE. [1. 4]

But KH is also equal to GF,
and they contain right angles ;
therefore HB is also equal to FE. [1. 4]

Again, since the two sides AK, KH are equal to the two
sides DG, GF,

and they contain right angles,

therefore the base AH is equal 10 the base FD. [1. 4]

But AB is also equal to DE ;
therefore the two sides HA, AB are equal to the two sides
DF, DE.

And the base HB is equal to the base FE ;
therefore the angle BAH is equal tc the angle EDF. [1. 8]

For the same reason
the angle HAL is also equal to the angle FDC.

And the angle BAL is also equal to the angle EDC.

Therefore on the straight line AB, and at the point A on
it, a solid angle has been constructed equal to the given solid
angle at D.

Q.E.F.
\end{proof}

\begin{notes}

This proposition again assumes trie equality of two trihedral angles which
have the three plane angles of the one respectively equal to the three plane
angles of the other taken in the same order.

\end{notes}

\end{proposition}

\begin{proposition}
\label{prop:XI_27}

\begin{statement}
On a given straight line to describe a parallelepipedal solid
simitar and similarly situated to a given parallelepipedal solid.
\end{statement}

\begin{proof}

Let AB be the given straight line and CD the given
parallelepipedal solid ;

thus it is required to describe on the given straight line AB
a parallelepipedal solid similar and similarly situated to the
given parallelepipedal solid CD,

M

For on the straight line AB and at the point A on it let
the solid angle, contained by the angles BA H, HAK, KAB,
be constructed equal to the solid angle at C, so that the angle
BAH is equal to the angle ECF, the angle BAK equal to
the angle ECG, and the angle KAH to the angle GCF ;
and let it be contrived that,
as EC is to CG, so is BA to AK,
and, as GC is to CF, so is KA to AH, [vi. 1*]

Therefore also, ex aequali,
as EC is to CF, so is BA to AH. [v. «]

Let the parallelogram HB and the solid ALhe completed.

Now since, as EC is to CG, so is BA to AK,
and the sides about the equal angles ECG, BAK are thus
proportional,

33P BOOK Xt [xi. *7, a8

therefore the parallelogram GE is similar to the parallelo-
gram KB.

For the same reason
the parallelogram KH is also similar to the parallelogram GF,
and further FE to HB ;

therefore three parallelograms of the solid CD are similar to
three parallelograms of the solid AL.

But the former three are both equal and similar to the
three opposite parallelograms,

and the latter three are both equal and similar to the three
opposite parallelograms ;

therefore the whole solid CD is similar to the whole solid AL.

[xi. Def. 9]
Therefore on the given straight line AB there has been
described AL similar and similarly situated to the given
parallelepipedal solid CD.

Q.E.F.
\end{proof}

\end{proposition}

\begin{proposition}
\label{prop:XI_28}

\begin{statement}
If a parallelepipedal solid be cut by a plane through the
diagonals of the opposite planes, the solid will be bisected by the
plane.
\end{statement}

\begin{proof}

For let the parallelepipedal solid AB be cut by the plane
CDEF through the diagonals CF, DE of
opposite planes ;

I say that the solid AB will be bisected by
the plane CDEF.

For, since the triangle CGF is equal
to the triangle CFB, [1. 34]

and ADE to DEH,
while the parallelogram CA is also equal
to the parallelogram EB, for they are opposite,
and GE to CH,

therefore the prism contained by the two triangles CGF,
ADE and the three parallelograms GE, AC, CE is also equal
to the prism contained by the two triangles CFB, DEH and
the three parallelograms CH, BE, CE ;
for they are contained by planes equal both in multitude and

in magnitude. [xi. Def. 10]

Hence the whole solid AB is bisected by the plane CDEF.
\end{proof}

\begin{notes}

Simson properly observes that it ought to be proved that the diagonals of
two opposite faces are in one plane, before we speak of drawing a plane
through them. Clavius supplied the proof, which is of course simple enough.

Since EF, CD are both parallel to AG or BH, they are parallel to one
another.

Consequently a plane can be drawn through CD, EF and the diagonals
DE, CF are in that plane [xi. 7]. Moreover CD, EF are equal as well as
parallel ; so that CF, DE are also equal and parallel.

Simson does not, however, seem to have noticed a more serious difficulty.
The two prisms are shown by Euclid to be contained by equal faces — the faces
are in fact equal and similar — and Euclid then infers at once that the prisms
are equal. But they are not equal in the only sense in which we have, at
present, a right to speak of solids being equal, namely in the sense that they
can be applied, the one to the other. They cannot be so applied because the
faces, though equal respectively, are not similarly arranged ; consequently the
prisms are symmetrical, and it ought to be proved that they are, though not
equal and similar, equal in content, or equivalent, as Legendre has it.

Legendre addressed himself to proving that the two prisms are equivalent,
and his method has been adopted, though his
name is not mentioned, by Schultze and Seven-
oak and by Holgate. Certain preliminary pro-
positions are necessary.

1. The sections of a prism made by parallel
planes cutting all the lateral edges art equal
polygons.

Suppose a prism MNcut by parallel planes
which make sections ABCDE, A'BCDE.

1iowAB,BC, CD, . . . are respectively parallel
to A'B 1 , BC, CD',.... [xt. 16]

Therefore the angles ABC, BCD, ... are
equal to the angles A'BC, BCD, ... respec-
tively, [xi. 10]

Also AB, BC, CD, ... are respectively equal
to A'B, BC, CD 1 ,.... [1.34]

Thus the polygons ABCDE, A'BCDE' are equilateral and equiangular
to one another.

1. Two prisms are equal when they have a solid angle in each contained by
three faces equal each to each and similarly arranged.

Let the faces ABCDE, AG, AL be equal and similarly placed to the
faces A'BCDE, A'G'', A'L'.

Since the three plane angles at A, A' are equal respectively and are
similarly placed, the trihedral angle at A is equal to the trihedral angle at A'.

[(3) in note to xi. 21]

Place the trihedral angle at A on that at A'.
Then the face ABCDE coincides with the face A'ffC'DE, the
with the face A'G\ and the face AL with the face A'L'.
The point C falls on C'' and D on D.

[xi. 28
lace v4£7

Since the lateral edges of a prism are parallel, CH will fall an CJT, and
DKonDK'.

And the points - G, L coincide respectively with F, G', L', so that
the planes GfC, G'K' coincide.

Hence H, K coincide with H', K' respectively.

Thus the prisms coincide throughout and are equal.

In the same way we can prove that two truncated prisms with three faces
forming a solid angle related to one another as in the above proposition are
identically equal.

In particular,

Cor. Two right prisms having equal bases and equal heights are equal.

3. An oblique prism is equivalent to a right prism whose base is a right
section of the oblique prism and whose
height is equal to a lateral edge of the
oblique prism.

Suppose GL to be a right section of
the oblique prism AD, and let GL be
a right prism on GL as base and with
height equal to a lateral edge of AD.

Now the lateral edges of GL are
equal to the lateral edges of AD',

Therefore AG=A'G\ BH=BH',
CK= C'/C, etc.

Thus the faces AH, BK, CL are
equal respectively to the faces A' IT,
BK 1 , CL'.

Therefore [by the proposition
above]

(truncated prism AL) = (truncated

prism A'L').
Subtracting each from the whole solid AL', we see that
the prisms AD, GL' are equivalent.

Now suppose the parallelepiped of Euclid's proposition to be cut by the
plane through AG, DF

Let KLMNhe. a right section of the parallelepiped
cutting the edges AD, BC, GF, HE.

Then KLMN is a parallelogram; and, if the
diagonal KM be drawn,

C±KLM=MNK.

Now the prism of which the As ABG, DCF are
the bases is equal to the right prism on A KLM as
base and of height AD.

Similarly the prism of which the As AGH, DFE
are the bases is equal to the right prism on MNK
as base and with height AD. [(3) above]

And the right prisms on A s KLM, MNK as bases and of equal height
AD are equal. [(*), Cor. above]

Consequently the two prisms into which the parallelepiped is divided are
equivalent.

\end{notes}

\end{proposition}

\begin{proposition}
\label{prop:XI_29}

\begin{statement}
Parallelepipedal solids wkich are on the same base and of
the same height, and in wkich the extremities of the sides which
stand up are on the same straight lines, are equal to one
another.
\end{statement}

\begin{proof}

Let CM, CN be parallelepipedal solids on the same base
AB and of the same height,
and let the extremities of their
sides which stand up, namely
AG, AF, LM, LN, CD, CE,
BH,BJC,be on the same straight
lines FN,DK;

I say that the solid CM is equal
to the solid CN.

For, since each of the figures
CH, CK is a parallelogram, CB
is equal to each of the straight lines DH, EK ,

hence DH is also equal to EK.

Let EH be subtracted from each ;
therefore the remainder DE is equal to the remainder HK.

Hence the triangle DCE is also equal to the triangle
HBK, [i 8, 4 ]

and the parallelogram DG to the parallelogram HN. [1. 3°]

For the same reason
the triangle AFG is also equal to the triangle MLN.

But the parallelogram CF is equal to the parallelogram BM,
and CG to BN, for they are opposite ;

therefore the prism contained by the two triangles AFG, DCE
and the three parallelograms AD, DG, CG is equal to the
prism contained by the two triangles MLN, HBK and the
three parallelograms BM, HN, BN,

Let there be added to each the solid of which the
parallelogram AB is the base and GEHM its opposite ;
therefore the whole parallelepipedal solid CM is equal to the
whole parallelepipedal solid CN.

Therefore etc.
\end{proof}

\begin{notes}

As usual, Euclid takes one case only and leaves the reader to prove for
himself the two other possible cases shown in the subjoined figures. Euclid's
proof holds with a very slight change in each case. With the first figure, the

only difference is that the prism of which the As GAL, ECB are the bases
takes the place of `` the solid of which the parallelogram AB is the base and
GEHM its opposite''; while with the second figure we have to subtract the
prisms which are proved equal successively from the solid of which the
parallelogram A 3 is the base and FDKN its opposite.

Simson, as usual, suspects mutilation by ``some unskilful editor,'' but gives
a curious reason why the case in which the two parallelograms opposite to
AB have a side common ought not to have been omitted, namely that this
case ``is immediately deduced from the preceding 18th Prop which seems for
this purpose to have been premised to the 19th.'' But, apart from the fact that
Euclid's Prop. 28 does not prove the theorem which it enunciates (as we have
seen), that theorem is not in the least necessary for the proof of this case of
Prop. 29, as Euclid's proof applies to it perfectly well.

\end{notes}

\end{proposition}

\begin{proposition}
\label{prop:XI_30}

\begin{statement}
Parallelepipedal solids which are on the same base and of
ike same keigkl, and in which ike extremities of the sides wkuk
stand up are not on the same straight lines, are equal to one
another.
\end{statement}

\begin{proof}

Let CM, CN be parallelepipedal solids on the same base
AB and of the same height,
and lettheextremitiesoftheir
sides which stand up, namely
AF,AG,LM,LN,CD,CE,
BH t BK, not be on the same
straight lines ;

I say that the solid CM is
equal to the solid CN,

For let NK, DH be pro-
duced and meet one another
at R,

and further let FM, GE be
produced to P, Q ;
let AO, LP, CQ, BR be joined.

Then the solid CM, of which the parallelogram ACBL is
the base, and FDHM its opposite, is equal to the solid CP,
of which the parallelogram ACBL is the base, and OQRP its
opposite ;

for they are on the same base A CBL and of the same height,
and the extremities of their sides which stand up, namely AF t
AO, LM, LP, CD, CQ, BH, BR, are on the same straight
lines FP, DR. [xi. 39]

But the solid CP, of which the parallelogram ACBL is
the base, and OQRP its opposite, is equal to the solid CN,
of which the parallelogram ACBL is the base and GEKN its
opposite ;

for they are again on the same base A CBL and of the same
height, and the extremities of their sides which stand up,
namely AG, AO, CE, CQ, LN, LP, BK. BR, are on the
same straight lines GQ, NR.

Hence the solid CM is also equal to the solid CN.
Therefore etc.
\end{proof}

\begin{notes}

This proposition completes the proof of the theorem that
Tkoo parallelepipeds on the same base and of the same height are equivalent.
Legendte deduced the useful theorem that

Every parallelepiped can be changed into an equivalent rectangular parallele-
piped having the same iteight and an equivalent base.

For suppose we have a parallelepiped on the base ABCD with EFGH for
the opposite face.

Draw A I, BK, CL, DM perpendicular to the plane through EFGH and
all equal to the height of the parallelepiped AG. Then, on joining IK, KL,
LM, MI, we have a parallelepiped equivalent to the original one and having
its lateral faces AK, BL, CM, Dl rectangles.

If ABCD is not a rectangle, draw AO, DN in the plane AC perpendicu-
lar to BC, and IP, MQ in the plane IL perpendicular to KL.

Joining OP, NQ, we have a rtdangular parallelepiped on AOND as base
which is equivalent to the parallelepiped with ABCD as ba 1  and IKLM as
opposite face, since we may regard these parallelepipeds as being on the same
base ADM J 'and of the same height (AO).

That is, a rectangular parallelepiped has been constructed which is
equivalent to the given parallelepiped and has (1) the same height, (2) an
equivalent base.

The American text-books which I have quoted adopt a somewhat different
construction shown in the subjoined figure.

The edges AB, DC, EF, HG of the original parallelepiped are produced
and cut at right angles by two parallel planes at a distance apart A B equal
to AB.

Thus a parallelepiped is formed in which all the faces are rectangles except
A' IT, BG

Next produce I/A', CB, G'F, H'E 'and cut them perpendicularly by two
parallel planes at a distance apart B'C equai to BC.

The points of section determine a rectangular parallelepiped.

The equivalence of the three parallelepipeds is proved, not by EucL xi.
*9> 30, but by the proposition about a right section of a prism given above in
the note to xi. 18 (3 in that note).

\end{notes}

\end{proposition}

\begin{proposition}
\label{prop:XI_31}

\begin{statement}
Parallelepipedal solids which are on equal bases and of the
same height are equal to one another.
\end{statement}

\begin{proof}

Let the parallelepipedal solids AE, CF, of the same height,
be on equal bases AB, CD.

I say that the solid AE is equal to the solid CF.

First, let the sides which stand up, HK, BE, AG, LJIf,
PQ, DF, CO, RS, be at right angles to the bases AB, CD ;
let the straight line RT be produced in a straight line
with CR

on the straight line RT, and at the point R on it, let the
angle TR U be constructed equal to the angle ALB, [1. 23]
let RT be made equal to AL, and RU equal to LB,
and let the base fand the solid XC/he completed.

Now, since the two sides TR, RU are equal to the two
sides AL, LB,

and they contain equal angles,

therefore the parallelogram R W is equal and similar to the

parallelogram HL.

Since again AL is equal to RT, and LM to RS,
and they contain right angles,

therefore the parallelogram RX is equal and similar to the
parallelogram AM.

For the same reason
LE is also equal and similar to SU

therefore three parallelograms of the solid AE are equal and
similar to three parallelograms of the solid XU.

But the former three are equal and similar to the three
opposite, and the latter three to the three opposite ; [xi. 24]
therefore the whole parallelepipedal solid AE is equal to the
whole parallelepipedal solid XU. [xt Def. 10]

Let DR, WU be drawn through and meet one another
at Y,

let a Tb be drawn through T parallel to D Y,
let PD be produced to a,
and let the solids YX, RI be completed.

Then the solid XY, of which the parallelogram RX is the
base and Yc its opposite, is equal to the solid XU of which
the parallelogram RX is the base and UV its opposite,
for they are on the same base RX and of the same height, and
the extremities of their sides which stand up, namely RY, RU,
Tb, TW, Se t Sd, Xc, XV, are on the same straight lines
YW,eV. [xi. 29]

But the solid XU is equal to AE :
therefore the solid XY is also equal to the solid AE.

And, since the parallelogram RUWT is equal to the
parallelogram YT

for they are on the same base RT and in the same parallels
RT, YW, [.. 35]

while R UWT is equal to CD, since it is also equal to AB,
therefore the parallelogram YT is also equal to CD.

But DTis another parallelogram ;
therefore, as the base CD is to D T, so is YT to D T. [v. 7]

And, since the parallelepipedal solid CI has been cut by
the plane RF which is parallel to opposite planes,
as the base CD is to the base DT, so is the solid CF to the
solid RI. [xi. a S ]

For the same reason,

since the parall el epi pedal solid YI has been cut by the plane
RX which is parallel to opposite planes,

as the base YT is to the base TD, so is the solid YX to the
solid RI. [xt 25]

But, as the base CD is to DT, so is Kto DT;

therefore also, as the solid CF is to the solid RI, so is the
solid YX to RI. [v. n]

Therefore each of the solids CF, YX has to RI the same
ratio ;
therefore the solid CF is equal to the solid YX. [v. 9]

But YX was proved equal to AE ;
therefore AE is also equal to CF.

Next, let the sides standing up, AG, HK, BE, LM, CN,
PQ, DF, RS, not be at right angles to the bases AB, CD ;

I say again that the solid AE is equal to the solid CF.

Q F

For from the points K, E, G, M, Q, F, N, S let KO, ET,
GU,MV, QW, FX, NY, SI be drawn perpendicular to the
plane of reference, and let them meet the plane at the points
0, T, U, V W, X, Y, I,
and let OT, OU, UV, TV, WX, WY, YI, IX be joined.

Then the solid KV'vs, equal to the solid QI,

for they are on the equal bases KM, QS and of the same
height, and their sides which stand up are at right angles to
their bases. [First part of this Prop.]

But the solid KV is equal to the solid AE,
and QI to CF;

for they are on the same base and of the same height, while
the extremities of their sides which stand up are not on the
same straight lines. [3''. 3°]

Therefore the solid AE is also equal to the solid CF.
Therefore etc.
\end{proof}

\begin{notes}

It is interesting to observe that, in the figure of this proposition, the bases
are represented as lying `` in the plane of the paper,'' as it were, and the third
dimension as `` standing up `` from that plane. The figure is that of the
manuscript P slightly corrected as regards the solid AE.

Nothing could well be more ingenious than the proof of this proposition,
which recalls the brilliant proposition t. 44 and the proofs of vi. 14 and 33.

A3 the proof occupies considerable space in the text, it will no doubt be
well to give a summary.

I. First, suppose that the edges terminating at the angular points of the
bases are perpendicular to the bases.

AB, CD being the bases, Euclid constructs a solid identically equal to
AE (he might simply have moved AE itself), placing it so that RS is the edge
corresponding to HK (RS- HK because the heights are equal), and the face
RX corresponding to HE is in the plane of CS.

The faces CD t RW are in one plane because both are perpendicular to
RS. Thus DR, WV meet, if produced, in Y say.

Complete the parallelograms YT, £>Tand the solids YX, FT.

Then (solid YX) = (solid UX),

because they are on the same base ST and of the same height. [xi. 29]

Also, CI, YI being parallelepipeds cut by planes RF, RX parallel to pairs
of opposite faces respectively,

(solid CF) \ (solid RI)=UCD:U DT, [xi. as]

and (solid YX) : (solid RI)=C3 YT-.CJDT.

But [1.35] UYT=aUT

=UAB

=CJC£>, by hypothesis.
Therefore (solid CF) = (solid YX)

m (solid UX)
= (solid AE).

II, If the edges terminating at the base are net perpendicular to it, turn
each solid into an equivalent one on the same base with edges perpendicular
to it (by drawing four perpendiculars from the angular points of the base to
the plane of the opposite face), (xi. 29, 30 prove the equivalence.)

Then the equivalent solids are equal, by Part 1. ; so that the original solids
are also equal.

Simson observes that Euclid has made no mention of the case in which
the bases of the two solids are equiangular, and he prefixes this case to Part L
in the text. This is surety unnecessary, as Part 1. covers it well enough : the
only difference in the figure is that UW would coincide with Yb and dV
with et.

Simson further remarks that in the demonstration of Part 11. it is not
proved that the new solids constructed in the manner described art parallele-
pipeds. The proof is, however, so simple that it scarcely needed insertion
into the text He is correct in his remark that the words ``while the
extremities of their sides which stand up are not on the same straight lines ``
just before the end of the proposition would be better absent, since they may
be `` on the same straight lines.''

\end{notes}

\end{proposition}

\begin{proposition}
\label{prop:XI_32}

\begin{statement}
Parallelepipedal solids which are of the same height are to
one another as their hoses.
\end{statement}

\begin{proof}

Let AB, CD be parallelepipedal solids of the same height ;
I say that the parallelepipedal solids AB, CD are to one
another as their bases, that is, that, as the base AE is to the
base CF, so is the solid AB to the solid CD,

For let FH equal to AE be applied to FG, [1. 45]

and on FH as base, and with the same height as that of CD,
let the parallelepipedal solid GK be completed.

Then the solid AB is equal to the solid GK ;
for they are on equal bases AE, FH and of the same height.

[xi. 31]

And, since the parallelepipedal solid CK is cut by the plane
DG which is parallel to opposite planes,
therefore, as the base CF is to the base FH, so is the solid
CD to the solid DH. [xi. 25]

But the base FH is equal to the base AE,
and the solid GK to the solid AB ;

therefore also, as the base AE is to the base CF, so is the
solid AB to the solid CD.

Therefore etc.
\end{proof}

\begin{notes}

As Clavius observed, Euclid should have said, in applying the parallelo-
gram FH to FG, that it should be applied `` in the angle FGH equal to the
angle LCG.'' Simson is however, I think, hypercritical when he states as
regards the completion of the solid GK that it ought to be said, `` complete
the solid of which the base is FH, and one of its insisting straight lines is FD.''
Surely, when we have two faces DG, FH meeting in an edge, to say ``complete
the solid `` is quite sufficient, though the words `` on FH as base `` might
perhaps as well be left out. The same `` completion `` of a paralletepipedal
solid occurs in xi. 31 and 33.

\end{notes}

\end{proposition}

\begin{proposition}
\label{prop:XI_33}

\begin{statement}
Similar parallekpipedal solids are to one another in the
triplicate ratio of their corresponding sides.
\end{statement}

\begin{proof}

Let AB, CD be similar parallelepipedal solids,
and let AE be the side corresponding to CF;
I say that the solid AB has to the solid CD the ratio triplicate
of that which AE has to CF.

For let EK, EL, EM be produced in a straight line with
AE, GE, HE,

let EK be made equal to CF, EL equal to FN, and further

EM equal to FR,

and let the parallelogram KL and the solid KP be completed.

Now, since the two sides KE, EL are equal to the two
sides CF, FN,

while the angle KEL is also equal to the angle CFN,
inasmuch as the angle AEG is also equal to the angle CFN
because of the similarity of the solids AB, CD,

therefore the parallelogram KL is equal < and similar > to the
parallelogram CN.

For the same reason
the parallelogram KM is also equal and similar to CR,
and further EP to DF;

therefore three parallelograms of the solid KP are equal and
similar to three parallelograms of the solid CD.

But the former three parallelograms are equal and similar
to their opposites, and the latter three to their opposites ; [xi. 24]
therefore the whole solid KP is equal and similar to the whole
solid CD, [xi. Def. ro]

Let the parallelogram GK be completed,
and on the parallelograms GK, KL as bases, and with the
same height as that of AB, let the solids EO, LQ be
completed.

Then since; owing to the similarity of the solids AB, CD,
as AB is to CF, so is EG to FN, and EH to FR,
while CF is equal to EK, FN to EL, and FR to EM,
therefore, as AE is to EK, so is GE to EL, and HE to EM.

But, as AE is to EK, so is A G to the parallelogram GK,
as GE is to EL, so is GK to KL,

and, as HE is to EM, so is QE to ATAf ; [?i. 1]

therefore also, as the parallelogram AG is to GK, so is GK
to AX, and QE to A'AT.

But, as A G is to GK, so is the solid AB to the solid Zj 0,
as GK is to A''Z, so is the solid OE to the solid QL,
and, as QE is to AT./B/, so is the solid QL to the solid KP ;

[xi. 31]
therefore also, as the solid AB is to EO, so is EO to QL, and
0£ to AV.

But, if four magnitudes be continuously proportional, the
first has to the fourth the ratio triplicate of that which it has
to the second ; [v. Def. 10]

therefore the solid AB has to KP the ratio triplicate of that
which AB has to EO.

But, as AB is to EO, so is the parallelogram AG to GK,
and the straight line AE to EK [vi. 1] ;

344 BOOK XT [xi. 33

hence the solid AB has also to KP the ratio triplicate of that
which AE has to EK.

But the solid KP is equal to the solid CD,
and the straight line EK to CF

therefore the solid AB has also to the solid CD the ratio
triplicate of that which the corresponding side of it, AE, has
to the corresponding side CF.

Therefore etc.
\end{proof}

\begin{porism*}

From this it is manifest that, if four straight
lines be < continuously > proportional, as the first is to the
fourth, so will a parallelepipedal solid on the first be to the
similar and similarly described parallelepipedal solid on the
second, inasmuch as the first has to the fourth the ratio
triplicate of that which it has to the second.

\end{porism*}

\begin{notes}

The proof may be summarised as follows.

The three edges AE, GE, HE of the parallelepiped AB which meet at
E, the vertex corresponding to R in the other parallelepiped, are produced,
and lengths EK, EL, EM are marked off equal respectively to the edges CF,
FN, ER of CD.

The parallelograms and solids are then completed as shown in the figure.
Euclid first shows that the solid CD and the new solid PK are equal and
similar according to the criterion in xi. Def. 10, viz. that they are contained
by the same number of equal and similar planes. (They are arranged in the
same order, and it would be easy to prove equality by proving the equality of
a pair of solid angles and then applying one solid to the other.)
We have now, by hypothesisj

AE: CE= EG : EN '= EH ': ER ;
that is, AE : EK= EG: EL = EH : EM.

But AE;EK = E3AG:CDGK, [vi. 1]

EG: EL=E3GK:EJKL,
EH : EM=n HK : U KM.
Again, by xi. 25 or 32,

CJAG-.CJGK (solid AB) : (solid EO),
CJGK-.U KL = (solid EO) : (solid QL),
CJHK-.CD KM= (solid QL) : (solid KP).
Therefore
(solid AB) : (solid EO) = (solid EO) : (solid QL) = (solid QL) : (solid KP),
or the solid AB is to the solid KP (that is, CD) in the ratio triplicate of that
which the solid AB has to the solid EO, i.e. the ratio triplicate of that which
AE has to EK (or CF).

Heiberg doubts whether the Porism appended to this proposition is
genuine.

Simson adds, as Prop. D, a useful theorem which we should have expected
to find here, on the analogy of vi. 23 following vi. 19, 20, viz. that Solid
parallelepipeds contained by parallelograms equiangular to one another, each to
each, that is, of which the solid angles are equal, each to each, have to one another
the ratio compounded of the ratios of their sides.

The proof follows the method of the proposition xi. 33, and we can use
the same figure. In order to obtain one ratio between lines to represent the
ratio compounded of the ratios of the sides, after the manner of vi. 13, we
take any straight lint a, and then determine three other straight lines b, c, d,
such that

AE: CF=a:b,

EG:FN=b;c,

EH:FRc:d,

whence a : d represents the ratio compounded of the ratios of the sides.

We obtain, in the same manner as above,

(solid AB) : (solid EO) =C3 AG :U GK= AE : EK= AE : CF

= a:b,
(solid EO) : (solid QL)=a GK-.CJ KL = GE : EL = GE : FN

= b:c,
(solid QL): (solid KF)=CJ HK:CJKM=EH;EM=EH:FR

= cid,
whence, by composition [v. 2a],

(solid AB) : (solid KP) = a:d,
or (solid AB) : (solid CD) = a:d.

\end{notes}

\end{proposition}

\begin{proposition}
\label{prop:XI_34}

\begin{statement}
In equal parallelepipedal solids the bases are reciprocally
proportional to the heights; and those parallelepipedal solids in
which the bases are reciprocally proportional to the heights are
equal.
\end{statement}

\begin{proof}

Let AB, CD be equal parallelepipedal solids ;
I say that in the paralieiepipedal solids AB, CD the bases are
reciprocally proportional to the heights,

that is, as the base EH is to the base NQ, so is the height
of the solid CD to the height of the solid AB.

First, let the sides which stand up, namely AG, EF, LB,
HK, CM, NO, PD, QR, be at right angles to their bases
I say that, as the base EH is to the base NQ, so is CM
to AG.

If now the base EH is equal to the base NQ,
while the solid AB is also equal to the solid CD,
CM will also be equal to AG.

For parallelepipedal solids of the same height are to
one another as the bases ; [xi. 3*]

and, as the base EH is to NQ, so will CM be to AG,
and it is manifest that in the parallelepipedal solids AB, CD
the bases are reciprocally proportional to the heights.

Next, let the base EH not be equal to the base NQ,
but let EH be greater.

Now the solid AB is equal to the solid CD ;
therefore CM is also greater than AG.

Let then CT be made equal to A G,
and let the parallelepipedal solid VC be completed on NQ as
base and with CT as height

Now, since the solid AB is equal to the solid CD,
and C V is outside them,

while equals have to the same the same ratio, [v. 7]

therefore, as the solid AB is to the solid CV, so is the solid
CD to the solid CV.

But, as the solid AB is to the solid CV, so is the base
EH to the base NQ,

for the solids AB, CV are of equal height ; [xi. 32]

and, as the solid CD is to the solid CV,o is the base MQ to
the base TQ [xi. 25] and CM to CT [vi. 1] ;
therefore also, as the base EH is to the base NQ, so is MC
to CT

But CT is equal to A G ;
therefore also, as the base EH is to the base NQ, so is MC
to AG,

Therefore in the para) 1 el epi pedal solids AB, CD the bases
are reciprocally proportional to the heights.

Again, in the parallelepipedal solids AB CD let the bases
be reciprocally proportional to the heights, chat is, as the base
EH is to the base NQ, so let the height of the solid CD be
to the height of the solid AB
I say that the solid AB is equal to the solid CD.

Let the sides which stand up be again at riglit angles to
the bases.

Now, if the base EH is equal to the base NQ,
and, as the base EH is to the base NQ, so is the height of
the solid CD to the height of the solid AB,
therefore the height of the solid CD is also equal to the
height of the solid AB,

But parallelepipedal solids on equal bases and of the same
height are equal to one another ; [xi. 31]

therefore the solid AB is equal to the solid CD.

Next, let the base EH not be equal to the base NQ,
but let EH be greater ;

therefore the height of the solid CD is also greater than the

height of the solid AB,

that is, CM is greater than AG.

Let CT'be again made equal to AG,
and let the solid CVbe similarly completed.

Since, as the base EH is to the base NQ, so is MC
to AG,

while AG is equal to CT,

therefore, as the base EH is to the base NQ, so is CM
to CT.

But, as the base EH is to the base NQ, so is the solid
AB to the solid CV,

for the solids AB, C V are of equal height ; [x jz]

and, as CM is to CT, so is the base MQ to the base Q T [n. 1]
and the solid CD to the solid CV. [ax as]

Therefore also, as the solid AB is to the solid CV, so is
the solid CD to the solid CV;

therefore each of the solids AB, CD has to CJ' the same
ratio.

Therefore the solid AB is equal to the solid CD. [v. 9]

Now let the sides which stand up, FE, BL, GA, HK,
ON, DP, MC, RQ, not be at right angles to their bases ;
let perpendiculars be drawn from the points F, G, B, K, 0,
M, D, R to the planes through EH, NQ, and let them meet
the planes at S, T, U, V, W, X, Y, a,
and let the solids FV, Oa be completed ;
I say that, in this case too, if the solids AB, CD are equal,
the bases are reciprocally proportional to the heights, that is,
as the base EH is to the base NQ, so is the height of the
solid CD to the height of the solid AB.

Since the solid AB is equal to the solid CD,
while AB is equal to BT,

for they ate on the same base FK and of the same height;

[xi. 29, 30]
and the solid CD is equal to DX,

for they are again on the same base RO and of the same
height ; []

therefore the solid BT is also equal to the solid DX.

Therefore, as the base FK is to the base OR, so is the
height of the solid DX to the height of the solid BT.

[Part 1.]

But the base FK is equal to the base EH,
and the base OR to the base NQ ;

therefore, as the base EH is to the base NQ, so is the height
of the solid DX to the height of the solid BT.

But the solids DX, BT and the solids DC, BA have the
same heights respectively ;

therefore, as the base EH is to the base NQ, so is the height
of the solid DC to the height of the solid AB,

Therefore in the parallelepipedal solids AB, CD the bases
are reciprocally proportional to the heights.

Again, in the parallelepipedal solids AB, CD let the bases
be reciprocally proportional to the heights,
that is, as the base EH is to the base NQ, so let the height
of the solid CD be to the height of the solid AB ;
I say that the solid AB is equal to the solid CD.

For, with the same construction,
since, as the base EH is to the base NQ, so is the height of
the solid CD to the height of the solid AB,
while the base EH is equal to the base FK,
and NQ to OR,

therefore, as the base FK is to the base OR, so is the height
of the solid CD to the height of the solid AB.

But the solids AB, CD and BT, DX have the same
heights respectively ;

therefore, as the base FK is to the base OR, so is the height
of the solid DX to the height of the solid BT.

Therefore in the parallelepipedal solids BT, DX the bases
are reciprocally proportional to the heights ;
therefore the solid BT is equal to the solid DX. [Part 1.]

But BTis equal to BA,

for they are on the same base FK and of the same height ;

[xi. 29, 30]
and the solid DX is equal to the solid DC. [id.]

Therefore the solid AB is also equal to the solid CD.
\end{proof}

\begin{notes}

In this proposition Euclid makes two assumptions which require notice,
(j) that, if two parallelepipeds are equal, and have equal bases, their heights
are equal, and (2) that, if the bases of two equal parallelepipeds are unequal,
that which has the lesser base has the greater height In justification of the
former statement Euclid says, according to what Heiberg holds to be the
genuine reading, `` for parallelepipedal solids of the same height are to one
another as their bases'' [xi. 3*]. This apparently struck some very early
editor as not being sufficient, and he added the explanation appearing in
Simson's text, `` For if, the bases EH, NQ being equal, the heights AG, CM
were not equal, neither would the solid AB be equal to CD. But it is by
hypothesis equal. Therefore the height CM is not unequal to the height AG;
therefore it is equal.'' Then, it being perceived that there ought not to be two
explanations, the genuine one was erased from the inferior mss. While the
interpolated explanation does not take us very far, the truth of the statement
may be deduced with perhaps greater case from xi. 31 than from xi. 32
quoted by Euclid. For, assuming one height greater than the other, while the
bases are equal, we have only to cut from the higher solid so much as will
make its height equal to that of the other. Then this part of the higher solid
is equal to the whole of the other solid which is by hypothesis equal to the
higher solid itself. That is, the whole is equal to its part ; which is impossible.

The genuine text contains no explanation of the second assumption that,
if the base EH be greater than the base NQ, while the solids are equal, the
height CM is greater than the height AG; for the added words ``for, if not,
neither again will the solids AB, CD be equal ; but they are equal by
hypothesis `` are no doubt interpolated. In this case the truth of the assump-
tion is easily deduced from xi. 32 by redudie ad absurdum. If the height CM
were equal to the height AG, the solid AB would be to the solid CD as the
base EH is to the base NQ, i.e. as a greater to a less, so that the solids would
not be equal, as they are by hypothesis. Again, if the height CM were less
than the height AG, we could increase the height of CD till it was equal to
that of AB, and it would then appear that AB is greater than the heightened
solid and \emph{a fortiori} greater than CD : which contradicts the hypothesis.

Clavius rather ingeniously puts the first assumption the other way, saying
that, if the heights are equal in the equal parallelepipeds, the bases must be
equal This follows directly from xi. 32, which proves that the parallelepipeds
are to one another as their bases ; though Clavius deduces it indirectly from
xi. 31. The advantage of Clavius' alternative is that it makes the second
assumption unnecessary. He merely says, if the heights be not equal, let CM
be the greater, and then proceeds with Euclid's construction.

It is also to be observed that, when Euclid comes to the corresponding
proposition for cones and cylinders [xn. 15 J he begins by supposing the
heights equal, inferring by HI. 11 (corresponding to xi. 31) that, the solids
being equal, the bases are also equal, and then proceeds to the case where the
heights are unequal without making any preliminary inference about the
bases. The analogy then of xn, 1 5, and the fact that he quotes xi. 3* here
(which directly proves that, if the solids are equal, and also their heights, their
bases are also equal), make Clavius' form the more convenient to adopt.

The two assumptions being proved as above, the proposition can be put
shortly as follows.

I. Suppose the edges terminating at the comers of the base to be per-
pendicular to it.

Then (a), if the base EH be equal to the base NQ, the parallelepipeds
being also equal, the heights must be equal (converse of xi. 31), so that the
bases are reciprocally proportional to the heights, the ratio of the bases and
the ratio of the heights being both ratios of equality.

(£) If the base EH be greater than the base NQ, and consequently (by
deduction from xi. 32) the height CM greater than the height AG, cut off
CT from CM equal to AG, and draw the plane TV through T parallel to the
base NQ, making the parallelepiped CV, with CT(= AG) for its height.

Then, since the solids AB, CD are equal,

(solid AB) : (solid CV) = (solid CD) : (solid CV). [v. 7]

Bui (solid AB) : (solid CV)=C3 HE -.a NQ, [xi. 32]

and (solid CD) : (solid C V) = O MQ : O TQ [xi. 35]

= CM:CT. [vi. 1 J

Therefore E3 HE-.U NQ=CM;CT

= CM:AG.
Conversely («), if the bases EH, NQ be equal and reciprocally proportional
to the heights, the heights must be equal.

Consequently (solid AB) = (solid CD), [xi. 31]

(») If the bases EH, NQ be unequal, if, e.g. O EH>£JNQ,
then, since O -ff-ff : CD NQ = CjV : A G,

CM>AG.
Make the same construction as before.

Then CJEH:ONQ = (solid AB) : (solid CV\ [xi. 32]

and CM:AG=CM: CT

= nDMQ;C]TQ [vi. 1]

= (solid CD)   (solid CF). [xi. 25]

Therefore

(solid AB) : (solid CV) = (solid CZ>) : (solid CF),
whence (solid -4 J) = solid CD. [v. 9]

II. Suppose that the edges terminating at the corners of the bases are not
perpendicular to it

Drop perpendiculars on the bases from the corners of the faces opposite
to the bases.

We thus have two parallelepipeds equal to AB, CD respectively, since
they are on the same bases FK, RO and of the same height respectively.

[xi. so, 30]
If then (1) the solid AB is equal to the solid CD,
(solid BT) = (solid i?*),
and, by the first part of this proposition,

O KF : a OB = MX : GT,

or £7 HE :0 NQ = MX : GT.

(a) If SDHE-.aNQMX-.GT,

then o KF: CJOR = MX : GT,

so that, by the first half of the proposition, the solids BT, DX are equal, and
consequently

(solid AB) m (solid CD).

The text of the second part of the proposition four times contains, after
the words `` of the same height,'' the words `` in which the sides which stand
up are not on the same straight lines.'' As Simson observed, they are inept,
as the extremities of the edges may or may not be `` on the same straight
lines''; cf. the similar words incorrectly inserted at the end of XI. 31.

Words purporting to quote the result of the first part of the proposition
are also twice inserted; but they are rejected as unnecessary and as containing
an absurd expression — ``(solids) in which the heights are at right angles to their
bases,'' as if the heights could be otherwise than perpendicular to the bases.

\end{notes}

\end{proposition}

\begin{proposition}
\label{prop:XI_35}

\begin{statement}
If there be two equal plane angles, and on their vertices
there be set up elevated straight lines containing equal angles
with the original straight lines respectively, if on the elevated
straight lines points be taken at random and perpendiculars be
drawn from them to the planes in which the original angles
are, and if from the points so arising in the planes straight
lines be joined to the vertices of the original angles, they will
contain, with the elevated straight lines, equal angles.
\end{statement}

\begin{proof}

Let the angles BA C, EDF be two equal rectilineal angles,
and from the points A, D let the elevated straight lines AG,
DM be set up containing, with the original straight lines,
equal angles respectively, namely, the angle MDE to the
angle GAB and the angle MDF to the angle GAC,
let points G, M be taken at random on AG, DM,
let GL, MN be drawn from the points G, M perpendicular to
the planes through BA, AC and ED, DF, and let them meet
the planes at L, N,
and let LA, ND be joined ;
I say that the angle GAL is equal to the angle MDN.

Let AH be made equal to DM,
and let HKx, drawn through the point H parallel to GL.

But GL is perpendicular to the plane through BA, AC ;
therefore HK is also perpendicular to the plane through,
BA, AC. [x., 8]

From the points K, N let KC, NF, KB, NE be drawn
perpendicular to the straight lines AC, DF, A3, DE,
and let HC, CB, MF, FE be joined.

Since the square on HA is equal to the squares on HK,
KA,

and the squares on KC, CA are equal to the square on KA,

['  47]
therefore the square on HA is also equal to the squares on
HK, KC, CA.

But the square on HC is equal to the squares on
HK, KC; [1.47]

therefore the square on HA is equal to the squares on
HC, CA.

Therefore the angle HCA is right. [1. 48 J

For the same reason
the angle DFM is also right.

Therefore the angle A CH is equal to the angle DFM.

But the angle HAC is also equal to the angle MDF.

Therefore MDF t HAC are two triangles which have two
angles equal to two angles respectively, and one side equal to
one side, namely, that subtending one of the equal angles,
that is, HA equal to MD

therefore they will also have the remaining sides equal to the
remaining sides respectively. [1. 26]

Therefore AC is equal to DF.

Similarly we can prove that AB is also equal to DE,

Since then AC is equal to DF, and AB to DE,
the two sides CA, AB are equal to the two sides FD, DE.

But the angle CAB is also equal to the angle FDE ;
therefore the base BC is equal to the base EF t the triangle to
the triangle, and the remaining angles to the remaining
angles ; [1. 4]

therefore the angle ACB is equal to the angle DFE.

But the right angle ACK is also equal to the right angle
DFN;

therefore the remaining angle BCK is also equal to the
remaining angle EFN.

For the same reason
the angle CBK is also equal to the angle FEN.

Therefore BCK, EFN are two triangles which have two
angles equal to two angles respectively, and one side equal to
one side, namely, that adjacent to the equal angles, that is,
BC equal to EF;

therefore they will also have the remaining sides equal to the
remaining sides. [i. *6]

Therefore CK is equal to FN.

But AC is also equal to DF ;
therefore the two sides AC, CK are equal to the two sides
DF, FN;
and they contain right angles.

Therefore the base AKs equal to the base DN. [i. 4]

And. since AH is equal to DM,
the square on AH is also equal to the square on DM,

But the squares on AK, KH are equal to the square
on AH,

for the angle AKH is right ; [1. 47]

and the squares on DN, NM are equal to the square
on DM,

for the angle DNM is right ; [1. 47]

therefore the squares on AK, KH are equal to the squares
on DN, NM ;

and of these the square on A K is equal to the square on DN;
therefore the remaining square on KH is equal to the square
on NM;
therefore HK is equal to MN.

And, since the two sides HA, AK are equal to the two
sides MD, DN respectively,

and the base HK was proved equal to the base MN,
therefore the angle HAK is equal to the angle MDN. [1. 8]

Therefore etc.

\begin{porism*}
From this it is manifest that, if there be two
equal plane angles, and if there be set up on them elevated
straight lines which are equal and contain equal angles with
the original straight lines respectively, the perpendiculars
drawn from their extremities to the planes in which are
the original angles are equal to one another.
\end{porism*}
\end{proof}

\begin{notes}

This proposition is required for the next, where it is necessary to know
that, if in two equiangular parallelepipeds equal angles, one in each, be
contained by three plane angles respectively, one of which is an angle of the
parallelogram forming the base in one parallelepiped, while its equal is likewise
in the base of the other, and the edges in which the two remaining angles
forming the solid angles meet are equal, the parallelepipeds are of the same
height.

Bearing in mind the definition of the inclination of a straight line to a
plane, we might enunciate the proposition more shortly thus.

If there be two trihedral angles identically equal to one another, corresponding
edges in each are equally inclined to the planes through the other two edges
respectively.

The proof, which is necessarily somewhat long, may be summarised thus.

It is required to prove that the angles GAL, MDN in the figure are equal,
G, M being any points on AG, DM, and GL, MN perpendicular to the
planes BAC, EDF respectively.

If AH s made equal to DM, and HK is drawn in the plane GAL parallel
toCZ,

HK is also perpendicular to the plane BAC. [xi. 8]

Draw KB, KC perpendicular to AB, AC respectively and NE, NF
perpendicular to DE, irrespectively, and complete the figures,

Now(i) HA* = HFC + KA*

= HK* + KC* + CA>   [i. 47]

= HC + CA* )

Therefore l HCA = a right angle.

Similarly l. MFD m a right angle.

(a) As If AC, MDFha.ve therefore two angles equal and one side.

Therefore AHACsAMDF, and AC=DF. [i. *6]

(3) Similarly AHAB s AMDE, and AB - DE.

(4) Hence as ABC, DEFre equal in all respects, so that BC = EF,
and lABC=lDEF,

uACBDFE.

(5) Therefore the complements of these angles are equal,
i.e. lKBC=lNEF,

and l.KCB = l.NFE.

(6) The A s KBC, NEF have two angles equal and one side, and are
therefore equal in all respects, so that

KB = NE,

KC=NF.

(7) The right-angled triangles KAC, NDFiat equal in all respects, since
A C= DF[(2) above], KC=NF.

Consequently AK=DN.

(8) In As HAK, MDN,

HK* + KA* = HA*

= ML?, by hypothesis,
=-MN* + ND'.

Subtracting the equals KA\ NIP,
we have HK* = MN

or HK= MN.

(9) As HAK, MDNzxt, now equal in all respects, by 1. 8 and 1. 4, and
therefore

lHAK=lMDN.

The Porism is merely a statement of the result arrived at in (8).

Lege rid re uses, practically, the construction and argument of this propo-
sition to prove the theorem given under (3) of the note on xi. 21 above that
In two equal trihedral angles, corresponding pairs of fate angles include equal
dihedral angles. This fact is readily deduced from the above proposition.

Since [(1)] HC, KC are both perpendicular to AC, and MF, NF both
perpendicular to DF, the angles HCK, MFN are the measures of the
dihedral angles between the planes HAC, BAC, and MDF, EDF respec-
tively, [xi. Def. 6]

By (6), KC=NF,

and, by (8), HK= MN,

while the angles HKC, MNF, both being right, are equal.

Consequently the A s HCK, MFNk equal in all respects, [1. 4]

so that lHCK=lMFN

Simson substituted a different proof of (i) in the above summary, as
follows.

Since HK is perpendicular to the plane BAC, the plane If BK, passing
through HK, is also perpendicular to the plane BAC. [xi. 18]

And AB, being drawn in the plane BAC perpendicular to BK, the
common section of the planes HBK, BAC, is perpendicular to the plane
HBK [xi. Def. 4], and is therefore perpendicular to every straight line
meeting it in that plane [xi. Def. 3].

Hence the angle ABH'xs a right angle.

I think Euclid's proof much preferable to this with its references to
definitions which are more of the nature of theorems.

\end{notes}

\end{proposition}

\begin{proposition}
\label{prop:XI_36}

\begin{statement}
ff three straight lines be proportional, the parallelepipedal
solid formed out of the three is equal to the parallelepipedal
solid on the mean which is equilateral, but equiangular with
the aforesaid solid.
\end{statement}

\begin{proof}

Let A, B, C be three straight lines in proportion, so that,
as A is to B, so is B to C

I say that the solid formed out of A, B, C is equal to the
solid on B which is equilateral, but equiangular with the
aforesaid solid.

Let there be set out the solid angle at E contained by the
angles DEG, GEF, FED,

let each of the straight lines DE, GE, EF be made equal to

B, and let the parallelepipedal solid EK be completed,

let LMhtt made equal to A,

and on the straight line LM, and at the point L on it, let there

be constructed a solid angle equal to the solid angle at E,

namely that contained by NLO, OEM, MLN

let L be made equal to B, and LN equal to C.

8-

c-

Now, since, as A is to B, so is B to C,
while A is equal to LM, B to each of the straight lines LO,
ED, and C to LN,

therefore, as LM is to EF, so is DE to LN,

Thus the sides about the equal angles NLM, DEE are
reciprocally proportional ;

therefore the parallelogram MN is equal to the parallelogram
DF [vi. 14]

And, since the angles DEE, NLM are two plane recti-
lineal angles, and on them the elevated straight lines L O, EG
are set up which are equal to one another and contain equal
angles with the original straight lines respectively,
therefore the perpendiculars drawn from the points G, O to
the planes through NL, LM and DE, EF are equal to one
another ; [xi, 35. Por.]

hence the solids LH, EK are of the same height.

But parallelepipedal solids on equal bases and of the same
height are equal to one another ; [xi. 31]

therefore the solid HL is equal to the solid EK.

And LH is the solid formed out of A, B, C, and EK the
solid on B ;

therefore the parallelepipedal solid formed out of A, B, -C is
equal to the solid on B which is equilateral, but equiangular
with the aforesaid solid,
\end{proof}

\begin{notes}

The edges of the parallelepiped HL being respectively equal to A, B, C,
and those of the equiangular parallelepiped KE being ail equal to B, we
regard MN (net containing the edge OL equal to B) as the base of the first
parallelepiped, and consequently FD, equiangular to MN, as the base of KE.

Then the solids have the same height [xi. 35, Por.]

Hence (solid HL) : (solid KE) = O MN : O ED. [xi. 3 2]

But, since A, B, C are in continued proportion,
A : B = B : C,
or LM:EFDE:LN.

Thus the sides of the equiangular Di MN, ED are reciprocally pro-
portional, whence

UMN=CJFD, [vi. 14]

and therefore (solid HL) = (solid KE).

\end{notes}

\end{proposition}

\begin{proposition}
\label{prop:XI_37}

\begin{statement}
If four straight lines be proportional, the parallelepipedal
solids on them which are similar and similarly described will
also be proportional; and, if the parallelepipedal solids on them
which are similar and similarly described be proportional, the
straight lines will themselves also be proportional.
\end{statement}

\begin{proof}

Let AB, CD, EF, GH be four straight lines in proportion,
so that, as AB is to CD, so is EF to GH ;
and let there be described on AB, CD, EF, GH the similar
and similarly situated parallelepipedal solids KA, LC, ME,
NG;

I say that, as KA is to LC, so is ME to NG.

For, since the parallelepipedal solid KA is similar to LC,
therefore KA has to LC the ratio triplicate of that which AB
has to CD. [xi. 33]

For the same reason

ME also has to NG the ratio triplicate of that which EF has
to GH, [«]

And, as AB is to CD, so is EF to GH.
Therefore also, as AK is to LC, so is ME to NG.

Next, as the solid AK is to the solid LC, so let the solid
ME be to the solid NG ;
I say that, as the straight line AB is to CD, so is EF to <7/£

For since, again, KA has to Z.C the ratio triplicate of that
which AB has to CD, [xi. 33]

and jfc27T also has to jV(? the ratio triplicate of that which EF
has to GH, [id.]

and, as KA is to ZC, so is ME to AG,
therefore also, as AB is to CD, so is -ST 7 `` to GH.

Therefore etc.
\end{proof}

\begin{notes}

In this proposition it is assumed that, if two ratios be equal, the ratio
triplicate of one is equal to the ratio triplicate of the other and, conversely,
that, if ratios which are the triplicate of two other ratios are equal, those other
ratios are themselves equal.

To avoid the necessity for these assumptions Simson adopts the alternative
proof found in the MS- which Heiberg calls b, and also adopted by Clavius,
who, howevet, gives Euclid's proof as well, attributing it to Theon. The
alternative proof proceeds after the manner of vi. »*, thus.

Make AB, CD, 0, P continuous proportionals, and also EF, GH, Q, R.

i

D E

Q

I. Then, since

AB:CD = EF: GH,
we have, tx aequali,

AB:P=EF:R. [v. «]

But (solid AK)  . (solid CL) = AB:P,

[xi. 33 and For.]
and (solid EM) : (solid GN) = EF:R.

Therefore

(solid AK) \ (solid CL) - (solid EM) : (solid GN)

II. If the solids are proportional, take .ST* such that
AB.CD = EF: ST,
and on ST describe the parallelepiped SV similar and similarly situated to
either of the parallelepipeds EM, GN.

Then, by the first part,

(solid AK) -. (solid CL) = (solid EM) : (solid SV),
whence it follows that

(solid GN) = (solid SV).

Hut these solids are similar and similarly situated ;
therefore their faces are similar and equal; [xi. Def. 10]

therefore the corresponding sides GH, ST are equal.

[For this inference cf. note on vi. 22. The equality of GH, ST may
readily be proved by application of the two parallelepipeds to one another,
since, being similar, they are equiangular.]

Hence AB : CD = EF. GH.

The text of the mss. has here a proposition which is as badly placed as it
is unnecessary. If a plane be at right angles to a plant, and from any one of the
points in out of the planes a perpendicular be drawn to the other plane, the
perpendicular so drawn will fall on the. common section of the planes. It is of
the nature of a lemma to xii. 17, where
alone the fact is made use of. Heiberg
observes that it is omitted in b and that the
copyist of P knew other texts which did not
contain it, From these facts it is fairly con-

cluded that the proposition was interpolated. B

The truth of it is of course immediately

obvious by rtductio ad absurdum. Let the plane CAD be perpendicular to
the plane AB, and let a perpendicular be drawn to the latter from any point
E in the former.

If it does not fall on AD, the common section, let it meet the plane AB
inF.

Draw FG in AB perpendicular to AD, and join EG.

Then FG is perpendicular to ihe plane CAD[x.. Def. 4], and therefore
to GE[x. Def. 3]. Therefore t, EGF'is right

Also, since .Eis perpendicular to AB,
the angle EFG is right

That is, the triangle EGF has two right angles :
which is impossible.

\end{notes}

\end{proposition}

\begin{proposition}
\label{prop:XI_38}

\begin{statement}
If Ihe sides of the opposite planes of a cube be bisected, and
planes be carried through the points of section, the common
section of the planes and the diameter of the cube bisect one
another.
\end{statement}

\begin{proof}

For let the sides of the opposite planes CF, AH of the
cube AF be bisected at the points K, L, -M, N t O, Q, P, R,
and through the points of section let the planes KN t OR be
carried ;

let US be the common section of the planes, and DG the
diameter of the cube AF.

I say that UT is equal to TS, and DT to TG.

For let DU, UE, BS, SG be joined.

Then, since DO is parallel to PE,

the alternate angles DOU, UPE are equal to one another.

[1. 29]
And, since DO is equal to PE, and OU to UP,
and they contain equal angles,
therefore the base DU is equal to the base UE,
the triangle DOU is equal to the triangle PUE,
and the remaining angles are equal to the remaining angles ;

[«  4]
therefore the angle OUD is equal to the angle PUE.

For this reason DUE is a straight line. 0- 14]

For the same reason, BSG is also a straight line,
and BS is equal to SG.

Now, since CA is equal and parallel to DB,
while CA is also equal and parallel to EG,
therefore DB is also equal and parallel to EG. [xi. 9]

36a BOOK XI [xi. 38

And the straight lines DE, BG join their extremities ;

therefore DE is parallel to BG. [1. 33]

Therefore the angle ED T is equal to the angle BG T,
for they are alternate ; [1. 29]

and the angle DTU is equal to the angle GTS. [1. 15]

Therefore D TU, G TS are two I iangles which have two
angles equal to two angles, and one side equal to one side,
namely that subtending one of the equal angles, that is, D U
equal to GS,

for they are the halves of DE, BG ;

therefore they will also have the remaining sides equal to the

remaining sides. [1. 26]

Therefore DT is equal to TG, and C/Tto TS.

Therefore etc.
\end{proof}

\begin{notes}

Euclid enunciates this proposition of a cube only, though it is true of any
parallelepiped, no doubt because its truth for a cube is all that was wanted for
the only proposition where it is needed, viz. xm. 17.

Simson remarks that it should be proved that the straight lines bisecting
the corresponding opposite sides of opposite planes art in one plane. This is,
however, clear because e.g. since DK, CL are equal and parallel, KL is equal
and parallel to CD. And, since KL, AB are both parallel to DC, KL is
parallel to AB. And lastly, since KL, MNiXK both parallel to AB, KL is
parallel to MNtsA therefore in one plane with it.

The essential thing to be proved is that the plane passing through the
opposite edges DB, EG passes through the straight line US, since, only if
this be the case, can US, DG intersect one another.

To prove this we have only to prove that, if DU, UE and BS, SG be
joined, DUE and BSG are both straight lines.

Now, since DO is parallel to PE,

lDOU=lEPU.

Thus, in the As DUO, EUP, two sides DO, OU are equal to two sides
EP, PU, and the included angles are equal.

Therefore DUO ~ A EUP,

DU UE,
and i.nUO = LEUP,

so that DUE is a straight line, bisected at U. Similarly BSG is a straight
line, bisected at S.

Thus the plane through DB, EG (DB, EG being equal and parallel)
contains the straight lines DUE, BSG (which are therefore equal and parallel
also) and also [xi. 7] the straight lines US, DG (which accordingly intersect).

In As DTU, GTS, the angles UDT, SGT sue equal (being alternate),
and the angles UTD, STG are also equal (being vertically opposite), while
D U (half of DE) is equal to GS (half of BG).

Therefore [1. 16] the triangles D TU, GTS are equal in all respects, so that
DT= TG,
UT= TS.

\end{notes}

\end{proposition}

\begin{proposition}
\label{prop:XI_39}

\begin{statement}
If there be two prisms of equal height, and one have a
parallelogram as base and the vtker a triangle, and if the
parallelogram be double of the triangle, the prisms will be
equal.
\end{statement}

\begin{proof}

Let ABCDEF, GHKLMN be two prisms of equal
height,

let one have the parallelogram AF as base, and the other the
triangle GHK,

and let the parallelogram AF be double of the triangle GHK;
1 say that the prism ABCDEF is equal to the prism
GHKLMN,

For let the solids A0 t GP be completed.

Since the parallelogram AF is double of the triangle GHK,
while the parallelogram HK is also double of the triangle
GHK, [,. 34]

therefore the parallelogram AF is equal to the parallelogram
HK

But parallelepipedal solids which are on equal bases and

of the same height are equal to one another ;
therefore the solid AO is equal to the solid GP.

And the prism ABCDEF is half of the solid A 0,
and the prism GHKLMN is half of the solid GP ;
therefore the prism ABCDEF is equal to the
GHKLMN.

Therefore etc.
\end{proof}

\begin{notes}

This proposition is made use of in xn. 3, 4. The phraseology is interest-
ins because we find one of the paralttlogrammic faces of one of the triangular
prisms called its base, and the perpendicular on this plane from that vertex of
either triangular face which is not in this plane the height.

The proof is simple because we have only to complete parallelepipeds
which are double the prisms respectively and then use xi. 31. It has to be
borne in mind, however, that, if the parallelepipeds are not rectangular, the
proof in xi. z8 is not sufficient to establish the fact that the parallelepipeds
are double of the prisms, but has to be supplemented as shown in the note on
that proposition. X11. 4 does, however, require the theorem in its general
form.

\end{notes}

\end{proposition}

\part{Book XII}

\chapter*{Historical Note}

The predominant feature of Book xii. is the use of the method of
exhaustion, which is applied in Propositions 2, 3 — 5, 10, 11, 12, and (in a
slightly different form) in Propositions 16—18. We conclude therefore that
for the content of this Book Euclid was greatly indebted to Eudoxus, to whom
the discovery of the method of exhaustion is attributed. The evidence for
this attribution comes mainly from Archimedes. (1) In the preface to On
the Sphere and Cylinder 1., after stating the main results obtained by himself
regarding the surface of a sphere or a segment thereof, and the volume and
surface of a right cylinder with height equal to its diameter as compared with
those of a sphere with the same diameter, Archimedes adds : `` Having now
discovered that the properties mentioned are true of these figures, I cannot
feel any hesitation in setting them side by side both with my former investiga-
tions and with those of the theorems of Eudoxus on solids which are held to be
most irrefragably established, namely that any pyramid is one third part of the
prism which has the same base with the pyramid and equal height [i.e. Eucl.
xii. 7], and that any cone is one third part of the cylinder which has the same
base with the tone and equal height [i.e. Eucl.\ xii. 10]. For, though these
properties also were naturally inherent in the figures all along, yet they were
in fact unknown to all the many able geometers who lived before Eudoxus
and had not been observed by any one'' (*) In the preface to the treatise
known as the Quadrature of the Parabola Archimedes states the ``lemma''
assumed by him and known as the ``Axiom of Archimedes'' (see note on x. 1
above) and proceeds : `` Earlier geometers (oi irpoTtpov yeeu/« rpai) have also
used this lemma; for it is by the use of this same lemma that they have
shown that circles are to one another in the duplicate ratio of their diameters
[Eucl.\ XII. 2], and that spheres are to one another in the triplicate ratio of their
diameters [Eucl.\ xii. 18], and further that eitery pyramid is one third part of the
prism which has the same base with the pyramid and equal height [Eucl.\ xii. 7];
also, that every cone is one third part of the cylinder which has the same base
with the eone and equal height [Eucl, xii. 10] they proved by assuming a certain
lemma similar to that aforesaid.'' Thus in the first passage two theorems of
Eucl.\ xii. are definitely attributed to Eudoxus ; and, when Archimedes says,
in the second passage, that `` earlier geometers `` proved these two theorems
by means of the lemma known as the ``Axiom of Archimedes'' and of a
lemma similar to it respectively, we can hardly suppose him to be alluding to
any other proof than that given by Eudoxus. As a matter of fact, the lemma
used by Euclid to prove both propositions (xti. 3 — 5 and 7, and xn. 10) is the
theorem of Eucl.\ x. 1. As regards the connexion between the two ``lemmas''
see note on x. 1.

We are not, however, to suppose that none of the results obtained by
the method of exhaustion had been discovered before the time of Eudoxus
(fl. about 368 — s B.C.). Two at least are of earlier date, those of Eucl.\ xn. a
and xn. 7.

(a) Simplicius (Comment, in Aristot, Phys. p. 6i, ed. Diels) quotes
Eudemus as saying, in his History of Geometry, that Hippocrates of Chios
(fl. say 430 B.c.) first laid it down (Wno) that similar segments of circles are
in the ratio of the squares on their bases and that he proved this (Ihtiwviv) by
proving (U toO Scifat) that the squares on the diameters have the same ratio
as the (whole) circles. We know nothing of the method by which Hippo-
crates proved this proposition ; but, having regard to the evidence from
Archimedes quoted above, it is not permissible to suppose that the method
was the fully developed method of exhaustion as we know it.

(b) As regards the two theorems about the volume of a pyramid and of
a cone respectively, which Eudoxus was the first to prove, we now have
authentic evidence in the short treatise by Archimedes discovered by Heiberg
in a MS. at Constantinople in 1906 and published in Hermes the following
year (see now Archimedis opera omnia, ed. Heiberg, 2, ed,, Vol. 11 ., 1913,
pp. 425 — 507; '1''. L. Heath, The Method of Archimedes, Cambridge, 191 2).
The said treatise, complete in all essentials, bears the title 'Apxtwovs irtpt rmr
fHflfin*** 8vapr)it£.Twv irpos ``EpaToatiirrpr «o>y. This ``Method'' (or ``Plan of
attack ``), addressed to Eratosthenes, is none other than the ftfuSSio* on which,
according to Suidas, Theodosius wrote a commentary, and which is several
times cited by Heron in his Metrica ; its discovery adds a new and important
chapter to the history of the integral calculus. In the preface to this work
Archimedes alludes to the theorems which he first discovered by means of
mechanical considerations, but proved afterwards by geometry, because the
investigation by means of mechanics did not constitute a rigid proof; he
observes, however, that the mechanical method is of great use for the discovery
of theorems, and it is much easier to provide the rigid proof when the fact
to be proved has once been discovered than it would be if nothing were
known to begin with. He goes on : `` Hence too, in the case of those
theorems the proof of which was first discovered by Eudoxus, namely those
relating to the cone and the pyramid, that the cone is one third part of the
cylinder, and the pyramid one third part of the prism, having the same base
and equal height, no small part of the credit will naturally be assigned to
Democritus, who was the first to make the statement (of the fact) regarding
the said figure [i.e. property], though without proving it.'' Hence the discovery
of the two theorems must now be attributed to Democritus (fl. towards the
end of 5th cent. B.C.). The words ``without proving it `` (x«P« £iroB«'£«i>5) do
not mean that Democritus gave no sort of proof, but only that he did not give
a proof on the rigorous lines required later ; for the same words are used by
Archimedes of his own investigations by means of mechanics, which, however,
do constitute a reasoned argument. The character of Archimedes' mechanical
arguments combined with a passage of Plutarch about a particular question in
infinitesimals said to have been raised by Democritus may perhaps give a clue
to the line of Democritus' argument as regards the pyramid. The essential
feature of Archimedes' mechanical arguments in this tract is that he regards
an area as the sum of an infinite number of straight lines parallel to one
another and terminated by the boundary or boundaries of the closed figure
the area of which is to be found, and a volume as the sum of an infinite
number of plane sections parallel to one another : which is of course the same
thing as taking (as we do in the integral calculus) the sum of an infinite
number of strips of breadth dx (say), when dx becomes indefinitely small, or
the sum of an infinite number of parallel laminae of depth dz (say), wlwn dz
becomes indefinitely small. To give only one instance, we may take the
case of the area of a segment of a parabola cut off by a chord.

Let CBA be the parabolic segment, CE the tangent at C meeting the

diameter EBD through the middle point of the chord CA in E, so that

EB = BD.

Draw AF parallel to ED meeting CE produced in F. Produce CB to
H so that CK=.KH, where K is the point in which CH meets AF; and
suppose CH 'to be a lever.

Let any diameter MNPO be drawn meeting the curve in P and CF, CK,
CA in M, N, respectively.

Archimedes then observes that

CA.AO = MO: OF
(`` for this is proved in a lemma ``),
whence 3. K : KJV= MO : OF,

so that, if a straight line TG equal to PO be placed with its middle point at
H, the straight line MO with centre of gravity at N, and the straight line TG
 vith centre of gravity at H, will balance about K.

Taking all other parts of diameters like PO intercepted between the curve
and CA, and placing equal straight lines with their centres of gravity at H,
these straight lines collected at H will balance (about K) all the lines like
MO parallel to FA intercepted within the triangle CFA in the positions in
which they severally lie in the figure.

Hence Archimedes infers that an area equal to that of the parabolic
segment hung at H will balance (about K) the triangle CFA hung at its
centre of gravity, the point X (a point on CK such that CK=XK), and
therefore that

(area of triangle CFA) : (area of segment) = HK: KX

= 3:1.

from which it follows that

area of parabolic segment = ABC,

The same sort of argument is used for solids, plane itetions taking the
place of straight lines.

Archimedes is careful to state once more that this method of argument
does not constitute  proof. Thus, at the end of the above proposition about
the parabolic segment, he adds : `` This property is of course not proved by
what has just been said; but it has furnished a sort of indication (1/iacriV riva)
that the conclusion is true.''

Let us now turn to the passage of Plutarch (De Comm. Not. adv. Stoieos
xxx tx j) about Democritus above referred to. Plutarch speaks of Democritus
as having raised the question in natural philosophy (wo«5s) : `` if a cone
were cut by a plane parallel to the base [by which is clearly meant a plane
indefinitely near to the base], what must we think of the surfaces of the
sections, that they are equal or unequal ? For, if they are unequal, they will
make the cone irregular, as having many indentations, like steps, and uneven-
nesses ; but, if they are equal, the sections will be equal, and the cone will
appear to have the property of the cylinder and to be made up of equal, not
un equal circles, wh ic h is very absu rd , `` The phrase'' made up of eq uat . . . circl es ``
(i£ law o-iry *«/«» <«,., mIkXcov) shows that Democritus already had the idea of
a solid being the sum of an infinite number of parallel planes, or indefinitely
thin laminae, indefinitely near together : a most important anticipation of the
same thought which led to such fruitful results in Archimedes. If then one
may hazard a conjecture as to Democritus' argument with regard to a pyramid,
it seems probable that he would notice that, if two pyramids of the same
height and equal triangular bases are respectively cut by planes parallel to the
base and dividing the heights in the same ratio, the corresponding sections of
the two pyramids are equal, whence he would infer that the pyramids are
equal as being the sum of the same infinite number of equal plane sections
or indefinitely thin laminae. (This would be a particular anticipation of
Cavalieri's proposition that the areal or solid contents of two figures are equal
if two sections of them taken at the same height, whatever the height may be,
always give equal straight lines or equal surfaces respectively.) And
Democritus would of course see that the three pyramids into which a prism
on the same base and of equal height with the original pyramid is divided (as
in End. xn. 7) satisfy this test of equality, so that the pyramid would be one
third part of the prism. The extension to a pyramid with a polygonal base
would be easy. And Democritus may have stated the proposition for the
cone (of course without an absolute proof) as a natural inference from the
result of increasing indefinitely the number of sides in a regular polygon
forming the base of a pyramid.

\part*{Book XII. Propositions}

\begin{proposition}
\label{prop:XII_1}

\begin{statement}
Similar polygons inscribed in circles are to one another as
the squares on the diameters.
\end{statement}

\begin{proof}

Let ABC, FGH be circles,
let ABCDE, FGHKL be similar polygons inscribed in them,
and let BM, GN be diameters of the circles ;
I say that, as the square on BM is to the square on GN, so
is the polygon ABCDE to the polygon FGHKL.

For let BE, AM, GL, FN be joined.

Now, since the polygon ABCDE is similar to the polygon
FGHKL,

the angle BAE is equal to the angle GFL,
and, as BA is to AE, so is GF to FL. [vi. Def. i]

Thus BAE, GFL are two triangles which have one angle
equal to one angle, namely the angle BAE to the angle
GFL, and the sides about the equal angles proportional ;
therefore the triangle ABE is equiangular with the triangle
FGL. [vi. 6)

Therefore the angle AEB is equal to the angle FLG.

But the angle AEB is equal to the angle AMB,
for they stand on the same circumference ; [ni. 27]

and the angle FLG to the angle FNG ;
therefore the angle AMB is also equal to the angle FNG.

But the right angle BAM is also equal to the right angle
GFN; [hi. 31]

therefore the remaining angle is equal to the remaining angle.

[«  3*]

Therefore the triangle ABM is equiangular with the
triangle FGN.

Therefore, proportionally, as BM is to GN, so is BA
to GF. [vi. 4]

But the ratio of the square on BM to the square on GN
is duplicate of the ratio of BM to GN,

and the ratio of the polygon ABCDE to the polygon FGHKL
is duplicate of the ratio of BA to GF; [vi. *o]

therefore also, as the square on BM is to the square on GN,
so is the polygon ABCDE to the polygon FGHKL.

Therefore etc.
\end{proof}

\begin{notes}

As, from this point onward, the text of each proposition usually occupies
considerable space, I shall generally give in the notes a summary of [he
argument, ``to enable it to be followed more easily.

Here we have to prove that a pair of corresponding sides are in the ratio
of the corresponding diameters.

Since i.s BAE, GFL are equal, and the sides about those angles
proportional,

As ABE, FGL are equiangular,

so that ii AEB = l FLG

Hence their equals in the same segments, l s AMB, FNG, are equal.
And the right angles BAM, GFN ate equal-
Therefore As ABM, FGN Me equiangular, so that

BM : GN= BA : GF.

The duplicates of these ratios are therefore equal,
whence (polygon ABCDE) : (polygon FGHKL)

= duplicate ratio of BA to GF
- duplicate ratio of BM to GN
= BM' : GN'.

XII. t]

\end{notes}

\end{proposition}

\begin{proposition}
\label{prop:XII_2}

\begin{statement}
Circles are to one another as the squares on the diameters.
\end{statement}

\begin{proof}

Let ABCD, EFGH be circles, and BD, FH theii
diameters ;

I say that, as the circle ABCD is to the circle EFGH, so is
the square on BD to the square on FH.

A

8

T

1

For, if the square on BD is not to the square on FH as
the circle ABCD is to the circle EFGH,
then, as the square on BD is to the square on FH, so will
the circle ABCD be either to some less area than the circle
EFGH, or to a greater.

First, let it be in that ratio to a less area 5.

Let the square EFGH be inscribed in the circle EFGH
then the inscribed square is greater than the half of the circle
EFGH, inasmuch as, if through the points E, F, G, H we
dn*w tangents to the circle, the square EFGH is half the
square circumscribed about the circle, and the circle is less
than the circumscribed square ;

hence the inscribed square EFGH is greater than the half of
the circle EFGH.

Let the circumferences EF, EG, GH, HE be bisected at
the points K, L, M, N,

and let EK, KF, FL, LG, GM, MH, HN, NE be joined ;
therefore each of the triangles EKF, FLG, GMH, HNE is
also greater than the half of the segment of the circle about
it, inasmuch as, if through the points K, L, M, N we draw
tangents to the circle and complete the parallelograms on the
straight lines EF, FG, GH, HE, each of the triangles EKF,

FLG, GMH, HNE wilt be half of the parallelogram
about it,

while the segment about it is less than the parallelogram ;
hence each of the triangles EKF, FLG, GMH, HNE
is greater than the half of the segment of the circle
about it.

Thus, by bisecting the remaining circumferences and
joining straight lines, and by doing this continually, we shall
leave some segments of the circle which will be less than the
excess by which the circle EFGH exceeds the area S.

For it was proved in the first theorem of the tenth book
that, if two unequal magnitudes be set out, and if from the
greater there be subtracted a magnitude greater than the half,
and from that which is left a greater than the half, and if this
be done continually, there will be left some magnitude which
will be less than the lesser magnitude set out.

Let segments be left such as described, and let the
segments of the circle EFGH on EK, KF, FL, LG, GM,
MH, HN, NE be less than the excess by which the circle
EFGH exceeds the area S.

Therefore the remainder, the polygon EKFLGMHN, is
greater than the area S.

Let there be inscribed, also, in the circle A BCD the poly-
gon AOBPCQDR similar to the polygon EKFLGMHN;
therefore, as the square on BD is to the square on FH, so is
the polygon AOBPCQDR to the polygon EKFLGMHN.

[xn. i]

But, as the square on BD is to the square on FH, so also
is the circle ABCD to the area S ;

therefore also, as the circle ABCD is to the area 5, so is the
polygon AOBPCQDR to the polygon EKFLGMHN;

[V.U]

therefore, alternately, as the circle ABCD is to the polygon
inscribed in it, so is the arei S to the polygon EKFLGMHN.

[v. 16]
But the circle ABCD is greater than the polygon inscribed
in it;

therefore the area 5 is also greater than the polygon
EKFLGMHN.

But it is also less :
which is impossible.

Therefore, as the square on BD is to the square on FH,
so is not the circle ABCD to any area less than the circle
EFGH,

Similarly we can prove that neither is the circle EFGH
to any area less than the circle ABCD as the square on FH
is to the square on BD.

I say next that neither is the circle ABCD to any area
greater than the circle EFGH as the square on BD is to the
square on FH,

For, if possible, let it be in that ratio to a greater area 5.

Therefore, inversely, as the square on FH is to the square
on DB, so is the area 5 to the circle ABCD.

But, as the area S is to the circle ABCD, so is the circle
EFGH to some area less than the circle ABCD ;
therefore also, as the square on FH is to the square on BD,
so is the circle EFGH to some area less than the circle
ABCD: [v. ii]

which was proved impossible.

Therefore, as the square on BD is to the square on FH,
so is not the circle ABCD to any area greater than the circle
EFGH.

And it was proved that neither is it in that ratio to any
area less than the circle EFGH ;

therefore, as the square on BD is to the square on FH, so ts
the circle ABCD to the circle EFGH.

Therefore etc.
\end{proof}

\begin{lemma*}

I say that, the area 5'' being greater than the circle
EFGH, as the area 5 is to the circle ABCD, so is the circle
EFGH to some area less than tho circle ABCD,

For let it be contrived that, as the area S is to the circle
ABCD, so is the circle EFGH to the area T.

I say that the area T is less than the circle ABCD.

For since, as the area S is to the circle ABCD, so is the
circle EFGH to the area T,

therefore, alternately, as the area S is to the circle EFGH, so
is the circle A BCD to the area T. [v. 16]

But the area .S is greater than the circle EFGH ;
therefore the circle ABCD is also greater than the area T.

Hence, as the area S is to the circle ABCD, so is the
circle EFGH to some area less than the circle ABCD.

Q.E.D.

\end{lemma*}

\begin{notes}

Though this theorem is said to have been proved by Hippocrates, we may
with tolerable certainty attribute the proof of it given by Euclid to Eudoxus,
to whom xn. 7 For. and Xll. 10 (which Euclid proves in exactly the same
manner) are specifically attributed by Archimedes. As regards the lemma
used herein (Eucl.\ x. i) and the somewhat different lemma by means of which
Archimedes says that the theorems of xii. 2, xll. 7 Par. and Xll. 18 were
proved, see my note on x, 1 above.

The first essential in this proposition is to prove that we can exhaust a
circle, in the sense of x. 1, by successively inscribing in it regular polygons,
each of which has twice as many sides as the preceding one. We take first
an inscribed square, then bisect the arcs subtended by the sides and so form
an equilateral polygon of eight sides, then do the same with the latter, forming
a polygon of 16 sides, and so on. And we have to prove that what is left
over when any one of these polygons is taken away from the circle is more
than half exhausted when the next polygon is made and subtracted from the
circle.

Euclid proves that the inscribed square is greater than half the circle and
that the regular octagon when subtracted takes away more than half of what
was left by the square. He then infers that the same
thing will happen whenever the number of sides is
doubled.

This can be seen generally by taking any arc of a
circle cut off by a chord AB. Bisect the arc in C.
Draw a tangent to the circle at C, and let AD, BE
be drawn perpendicular to the tangent. Join AC, CB.

Then D E is parallel to AB, since
l ECB = l CAB, in alternate segment, [in. 32]
= l CBA. [ill. 29, 1, 5]

Thus A BED is a CD;
and it is greater than the segment ACB,

Therefore its half, the a A CB, is greater than half the segment.

Thus, by x, 1, Euclid's construction of successive regular polygons in
a circle, if continued far enough, will at length leave segments which are
together less than any given area.

Now let X, X' be the areas of the circles, d, d' their diameters, respectively.

Then, if X : X' * d* : d' 1 ,

d*:d'* = X :S,
where S is some area either greater or less than X '.

I. Suppose 5 < X'.

Continue the construction of polygons in X' until we arrive at one which
leaves over segments together less than the excess of X' over S, i.e. a polygon
such that

X > (polygon in X') > S.

Inscribe in the circle X a polygon similar to that in X .

Then (polygon in X) : (polygon in X') = d*:<f* [xn. 1]

= X : S, by hypothesis ;
and, alternately,

(polygon in X ) : A'' = (polygon in X') : S.

But (polygon in X) < X;

therefore (polygon in X') < S.

But, by construction. (polygon in X) > S :
which is impossible.

Hence S cannot be less than X' as supposed.

II. Suppose S> X

Since d* : d* = X : S,

we have, inversely, d'    d* = S: X.

Suppose that S:X=X':T,

whence, since S>X', X > T. [v. 14]

Consequently d'* : d* = X' : T,

where T < X.

This can be proved impossible in exactly the same way as shown in Part I.

Hence S cannot be greater than X' as supposed.

Since then .S' is neither greater nor less than X

sx;

and therefore d> ; d'   = X : X'.

With reference to the assumption that there is some space S such that
d':d''X: S t
i.e. that there is a fourth proportional to the areas if, d' 1 , X, Simson observes
that it is sufficient, in this and the like cases, that a thing made use of in the
reasoning can possibly exist, though it cannot be exhibited by a geometrical
construction. As regards the assumption see note on v. 18 above.

There is grave reason for suspecting the genuineness of the Lemma at the
end of the proposition ; though, if it be rejected, it will be necessary to delete
the words ``as was before proved `` in corresponding places in xn. 5, 18.

It will be observed that Euclid proves the impossibility in the second case
by reducing it to the first If it is desired to prove the second case indepen-
dently, we must circumscribe successive polygons to the circles instead ol
inscribing them, in the way shown by Archimedes in his first proposition on
the Measurement of a circle. Of course we require, as a preliminary, the
proposition corresponding to xn. 1, that
Similar polygons circumscribed about
circles are to one another as the squares
on the diameters.

Let AB, A'ff be corresponding sides
of the two similar polygons. Then L s
OAB, OA'S are equal, since AO, A'O
bisect equal angles.

Similarly lABOlABO.

Therefore As A OB, A' OB are similar, so that their areas are in the
duplicate ratio of AB to A'B.

The radii OC, OC drawn to the points of contact are perpendicular to
AB, A'B, and it follows that

AB : A'B = CO : CO.

Thus the polygons are to one another in the duplicate ratio of the radii,
and therefore of the diameters.

Now suppose a square ABCD described about
a circle.

Make an octagon described about the circle by
drawing tangents at the points £ etc., where OA etc.
meet the circle.

Then shall the tangent at £ cut off more than
half of the area between AX', AH and the arc
HEK.

For the angle AEG is right, and is therefore
? uEAG.

Therefore AG>EG

> G K.

Therefore £.AGE> EGK.

Similarly AFE > EEH.

Hence A A EG > £ (re-entrant quadrilateral A HEK),

and \emph{a fortiori}, A AEG > | (area between AH, AK sxA the arc).

Thus the octagon takes from the square more than half the space between
the square and the circle.

Similarly, if a figure of 16 equal sides be circumscribed by cutting off
symmetrically the corners of the octagon, it will take away more than half of
the space between the octagon and circle.

Suppose now, with the original notation, that
P.d'tX-.S,
where S is greater than X'.

Continue the construction of circumscribed polygons about X' until the
total area between the polygon and the circle is less than the difference
between S and A'', i.e. till

S> (polygon about X') > X'.

Circumscribe a similar polygon about X.

Then (polygon about X ) : (polygon about X') = iP-.d' 1

-X:S, by hypothesis,
and, alternately,

(polygon about X ) \ X = (polygon about X') : S.

But (polygon about X) > X.

Therefore (polygon about X') > S.

But S > (polygon about X ') : [above]

which is impossible.

Hence S cannot be greater than X'.

Legendre proves this proposition by a method equally rigorous but not, I
think, possessing any advantages over Euclid's. It depends on a lemma
corresponding to Eucl.\ xu. 16, but with another part added to it.

Initio concentric circles being given, we can always inscribe in the greater a
regular polygon such that its sides do net meet the circumference of the lesser, and
we can also circumscribe about the lesser a regular
polygon such thai its sides do net meet the circum- u B

ference of the greater.

Let CA, CB be the radii of the circles.

I. At A on the inner circle draw the tangent
DE meeting the outer circle in D, E.

Inscribe in the outer circle any of the regular
polygons which we can inscribe, e.g. a square.

Bisect the arc subtended by a side, bisect
the half, bisect that again, and so on, until we
arrive at an arc less than the arc DBE.

Let this arc be MN, and suppose it so placed
that B is its middle point.

Then the chord MN is clearly more distant from the centre C than DE
is ; and the regular polygon, of which MN is a side, does not anywhere meet
the circumference of the inner circle.

II. Join CM, CN, meeting DE in P, Q.

Then PQ will be the side of a polygon circumscribed about the inner
circle and similar to the polygon inscribed in the outer ;
and the circumscribed polygon of which PQ is a side will not anywhere meet
the outer circle.

Legendre now proves xil. a after the following manner.
For brevity, let us denote the area of the circle with radius CA by
(circ CA),

Then it is required to prove that, if OB be the radius of a second circle,
(circ. CA) : (circ OB) = CA* : OB 1 .

Suppose, if possible, that this relation is not true. Then CA 1 will be to
OB 1 as (circ, CA) is to an area greater or less than (circ, OB).

I. Suppose, first, that

CA' t OB 1 = (circ CA)   (circ OB),
where OD is less than OB.

Inscribe in the circle with radius OS a regular polygon such that its sides
do not anywhere meet the circumference of the circle with centre OD ;

[Lemma]
and inscribe a similar polygon in the other circle.

The areas of the polygons will then be in the duplicate ratio of CA to OB,
or [xil. 1]

(polygon in circ. CA) : (polygon in circ. OS)
= CA?.OB*

= (circ. CA) : (circ. OD), by hypothesis.
But this is impossible, because the polygon in (circ. CA) is less than (cite.
CA), but the polygon in (circ. OB) is greater than (circ. OD).

Therefore CA 1 cannot be to OS 1 as (circ. CA) ts to a less circle than
(circ. OS).

1 1. Suppose, if possible, that

CA* : OS 1 = (circ. CA) : (some circle > circ. OS).
Then inversely

OS* : CA* - (circ. OS) : (some circle < circ. CA),
and this is proved impossible exactly as in Part I.

Therefore CA* : OS* = (circ. CA) : (circ. OS).

\end{notes}

\end{proposition}

\begin{proposition}
\label{prop:XII_3}

\begin{statement}
Any pyramid which has a triangular base is divided into
two pyramids equal and similar to one another, similar to the
whole and having triangular bases, and into two equal prisms ;
and the two prisms are greater than the half of the whole
pyramid.
\end{statement}

\begin{proof}

Let there be a pyramid of which the triangle ABC is the
base and the point D the vertex ;
I say that the pyramid ABCD is
divided into two pyramids equal to
one another, having triangular bases
and similar to the whole pyramid,
and into two equal prisms ; and the
two prisms are greater than the half
of the whole pyramid.

For let AB, BC, CA, AD, DB,
DC be bisected at the points E, F,
G, H, K, L, and let HE, EG, GH, HK, KL, LH, KF, FG
be joined.

Since AE is equal to EB, and AH to DH,
therefore EH is parallel to DB. [n 3]

For the same reason
HK is also parallel to AB,

Therefore HEBK is a parallelogram ;
therefore HK is equal to EB. [t- 34]

But EB is equal to EA ;
therefore AE is also equal to HK.

But AH is also equal to HD ;
therefore the two sides EA, AH are equal to the two sides
KH, HD respectively ,

and the angle EAH is equal to the angle KHD ;
therefore the base EH is equal to the base KD. [t 4]

Therefore the triangle A EH is equal and similar to the
triangle HKD.

For the same reason
the triangle AHG is also equal and similar to the triangle
HLD.

Now, since two straight lines EH, HG meeting one
another are parallel to two straight lines KD, DL meeting
one another, and are not in the same plane, they w:il contain
equal angles. [xi. 10]

Therefore the angle EHG is equal to the angle KDL.

And, since the two straight lines EH, HG are equal to the
two KD, DL respectively,
and the angle EHG is equal to the angle KDL,
therefore the base EG is equal to the base KL ; [1. 4]

therefore the triangle EHG is equal and similar to the
triangle KDL.

For the same reason

the triangle AEG is also equal and similar to the triangle
HKL.

Therefore the pyramid of which the triangie AEG is the
base and the point H the vertex is equal and similar to the
pyramid of which the triangle HKL is the base and the point
D the vertex. [xi. Def. 10]

And, since HK has been drawn parallel, to AB, one of the
sides of the triangle ADB,

380 BOOK XII [xn. 3

the triangle ADB is equiangular to the triangle DHK, [i. 29]

and they have their sides proportional ;

therefore the triangle ADB is similar to the triangle DHK.

[vi. Del. 1]

For the same reason
the triangle DBC is also similar to the triangle DKL, and
the triangle ADC to the triangle DLH.

Now, since the two straight lines BA, AC meeting one
another are parallel to the two straight lines KH, HL meeting
one another, not in the same plane, they will contain equal
angles. [xi. 10]

Therefore the angle BAC is equal to the angle KHL.

And, as BA is to AC, so is KH to HL ;
therefore the triangle ABC is similar to the triangle HKL.

Therefore also the pyramid of which the triangle ABC is
the base and the point D the vertex is similar to the pyramid
of which the triangle HKL is the base and the point D the
vertex.

But the pyramid of which the triangle HKL is the base
and the point D the vertex was proved similar to the pyramid
of which the triangle AEG is the base and the point H the
vertex.

Therefore each of the pyramids AEGH, HKLD is
similar to the whole pyramid A BCD.

Next, since BE is equal to EC,
the parallelogram EBFG is double of the triangle GEC.

And since, if there be two prisms of equal height, and one
have a parallelogram as base, and the other a triangle, and if
the parallelogram be double of the triangle, the prisms are
equal, [xi. 39]

therefore the prism contained by the two triangles BKF,
EHG, and the three parallelograms EBFG, EBKH, HKEG
is equal to the prism contained by the two triangles GEC,
HKL and the three parallelograms KECL, LCGH, HKEG.
And it is manifest that each of the prisms, namely that in
which the parallelogram EBFG is the base and the straight
line HK is its opposite, and that in which the triangle GEC is
the base and the triangle HKL its opposite, is greater than
each of the pyramids of which the triangles AEG, HKL are
the bases and the points H, D the vertices,
inasmuch as, if we join the straight lines £F, EK, the prism
in which the parallelogram EBFG is the base and the straight
line HKits opposite is greater than the pyramid of which the
triangle EBF is the base and the point K the vertex.

But the pyramid of which the triangle EBF is the base
and the point K the vertex is equal to the pyramid of which
the triangle AEG is the base and the point H the vertex ;
for they are contained by equal and similar planes.

Hence also the prism in which the parallelogram EBFG
is the base and the straight line HK its opposite is greater
than the pyramid of which the triangle AEG is the base and
the point H the vertex.

Bat the prism in which the parallelogram EBFG is the
base and the straight line HK its opposite is equal to the
prism in which the triangle GFC is the base and the triangle
HKL its opposite,

and the pyramid of which the triangle AEG is the base and
the point H the vertex is equal to the pyramid of which the
triangle HKL is the base and the point D the vertex.

Therefore the said two prisms are greater than the said
two pyramids of which the triangles AEG, HKL are the
bases and the points H, D the vertices.

Therefore the whole pyramid, of which the triangle ABC
is the base and the point D the vertex, has been divided into
two pyramids equal to one another and into two equal prisms,
and the two prisms are greater than the half of the whole
pyramid.
\end{proof}

\begin{notes}

We will denote a pyramid with vertex D and base ABC by D (ABC) or
D-ABC and the triangular prism with triangles GCF, HLK for bases by
(GCF, HLK).

The following are the steps of the proof.

I. To prove pyramid H(AEG) equal and similar to pyramid D(HKL).
Since sides of .DAB are bisected at N, E, K,
HE || DB, and HK\ AB.
Hence HK=EB = EA,

HE = KB = DK.
Therefore (1) As HAE, DHKz.it equal and similar.
Similarly (2) As HAG, DHL are equal and similar.
Again, Lff, HK are respectively || to GA, AE in a different plane ;
therefore l. GAE = L LHK.

And LH, If/Cue respectively equal to GA, AE.
Therefore (3) As GAE, LHK are equal and similar.
Similarly (4) As HGE, £>LK are equal and similar.
Therefore [xi. Def. 10] the pyramids If(AEG) and D(HKL) are equal
and similar.

II. To prove the pyramid D (HKL) similar to the pyramid D (ABC).
(1) The As DHK, DAB are equiangular and therefore similar.
Similarly (*) As DLH, DCA are similar, as also (3) the As DLK, DCB.
Again, BA, AC are respectively parallel to KH, HL in a different plane ;

therefore lBACl. KHL.

And BA : AC= KH : HL.

Therefore (4) As BAC, KHL are similar.

Consequently the pyramid D(ABC) is similar to the pyramid D (HKL),
and therefore also to the pyramid H(AEG).

III. To prove prism (GCF, HLK) equal to prism (HGE, KFB).

The prisms may be regarded as having the same hdht (the distance
between the planes HKL, ABC) and having for bases (i) the ACCand
(2) the O EBFG, which is the double of the A CGF.

Therefore, by XI. 39, the prisms are equal.

IV. To prove the prisms greater than the small pyramids.

Prism (HGE, KFB) is clearly greater than pyramid K(EFB) and there-
fore greater than pyramid H(AEG).

Therefore each of the prisms is greater than each of the small pyramids ;
and the sum of the two prisms is greater than the sum of the two small
pyramids, which, with the two prisms, make up the whole pyramid.

\end{notes}

\end{proposition}

\begin{proposition}
\label{prop:XII_4}

\begin{statement}
If there be two pyramids of the same height which have
triangular bases, and each of them be divided into two pyramids
equal to one another and similar to the whole, and into two
equal prisms, then, as the base of the one pyramid is to the
base of the other pyramid, so will all the prisms in the one
pyramid be to all the prisms, being equal in multitude, in the
other pyramid.
\end{statement}

\begin{proof}

Let there be two pyramids of the same height which
have the triangular bases ABC, DEF, and vertices the
points G, H,

and let each of them be divided into two pyramids equal to
one another and similar to the whole and into two equal
prisms ; [xil 3]

I say that, as the base ABC is to the base DEF, so are
all the prisms in the pyramid ABCG to all the prisms, being
equal in multitude, in the pyramid DEFH,

For, since BO is equal to OC, and AL to LC,
therefore LO is parallel to AB,
and the triangle ABC is similar to the triangle LOC.

3»3

For the same reason
the triangle DEFis also similar to the triangle RVF.

And, since BC is double of CO, and EF of FV,
therefore, as BC is to CO, so is EF to i 7 .

And on BC, CO are described the similar and similarly
situated rectilineal figures ABC, LOC,

and on EF, FV the similar and similarly situated figures
DEF, RVF;

therefore, as the triangle ABC is to the triangle LOC, so is
the triangle DEFto the triangle RVF; [vi. 22]

therefore, alternately, as the triangle ABC is to the triangle
DEF, so is the triangle LOC to the triangle RVF. [v. 16]

But, as the triangle LOC is to the triangle RVF, so Is
the prism in which the triangle LOC is the base and PMN its
opposite to the prism in which the triangle R VF is the base
and STU its opposite ; [Lemma following]

therefore also, as the triangle ABC is to the triangle DEF,
so is the prism in which the triangle LOC is the base and
PMN its opposite to the prism in which the triangle RVF
is the base and S TU its opposite.

But, as the said prisms are to one another, so is the prism
in which the parallelogram KBOL is the base and the straight
line PM its opposite to the prism in which the parallelogram
QEVR is the base and the straight line ST its opposite.

[xi. 39 ; cf. xii. 3]

Therefore also the two prisms, that in which the parallelo-
gram KBOL is the base and PM its opposite, and that in
which the triangle LOC is the base and PMN its opposite,
are to the prisms in which QEVR is the base and the straight
line ST its opposite and in which the triangle RVF is the
base and STU its opposite in the same ratio [v. u]

Therefore also, as the base ABC is to the base DEF, so
are the said two prisms to the said two prisms.

And similarly, if the pyramids PMNG, STUH be divided
into two prisms and two pyramids,

as the base PMN is to the base STU, so will the two prisms
in the pyramid PMNG be to the two prisms in the pyramid
STUH.

But, as the base PMN is to the base STU, so is the base
ABC to the base DEF ;

for the triangles PMN, STU are equal to the triangles LOC,
R VF respectively.

Therefore also, as the base ABC is to the base DEF, so
are the four prisms to the four prisms.

And similarly also, if we divide the remaining pyramids
into two pyramids and into two prisms, then, as the base
ABC is to the base DEF, so will all the prisms in the
pyramid ABCG be to all the prisms, being equal in multitude,
in the pyramid DEFH.
\end{proof}

\begin{lemma*}

But that, as the triangle LOC is to the triangle RVF,
so is the prism in which the triangle LOC is the base and
PMN its opposite to the prism in which the triangle RVF is
the base and STU its opposite, we must prove as follows.

For in the same figure let perpendiculars be conceived
drawn from G, H to the planes ABC, DEF; these are of
course equal because, by hypothesis, the pyramids are of equal
height.

Now, since the two straight lines <7Cand the perpendicular
from G are cut by the parallel planes ABC, PMN,
they will be cut in the same ratios. [xt. 17)

And GC is bisected by the plane PMN at N

therefore the perpendicular from G to the plane ABC will
also be bisected by the plane PMN.

For the name reason

the perpendicular from H to the plane DEF will also be
bisected by the plane ST if.

And the perpendiculars from G t H to the planes ABC,
DEF are equal ;

therefore the perpendiculars from the triangles PMN, STU
to the planes ABC, DEF are also equal.

Therefore the prisms in which the triangles LOC, RVF
are bases, and PMN, STU their opposites, are of equal
height.

Hence also the parallelepi pedal solids described from the
said prisms are of equal height and are to one another as their
bases ; [xi. 3*]

therefore their halves, namely the said prisms, are to one another
as the base LOC is to the base RVF,

Q.E.D.

\end{lemma*}

\begin{notes}

We can incorporate the lemma at the end of the proposition and sum-
marise the proof thus.

Since LO is parallel to AB,

As ABC, LOC ate simitar.
In like manner As DEF, RVF are similar.

And, since BC : CO = EF : FV,

A ABC : A LOC = A DEF: A RVF, [vi. 22]

and, alternately,

A ABC : DEF= ALOC.ARVF.
Now the prisms (LOC, PUN) and (RVF, STU) are equal in height :
for the perpendiculars from G, H on the bases ABC, DEF are divided by
the planes PMN, STU (parallel to the bases) in the same proportion as GC,
fi/Faxe divided by those planes [xi. 17], i.e. they are bisected j
hence the heights of the prisms, being half the equal heights of the pyramids,
are equal.

And the prisms are the halves respectively of parallelepipeds of the same
height on parallelogrammic bases double of the As LOC, RVF respectively :

[xi. 28 and note]
hence they are in the same ratio as those parallelepipeds, and therefore as
their bases [xi. J»J
Therefore

(prism LOC, PMN) : (prism RVF, STU) = A LOC: A RVF

= AABC:ADEF.

And since the other prisms in the pyramids are equal to these prisms
respectively,

(sum of prisms in GABC) : (sum of prisms in HDEF) =AABC:£)EF.
Similarly, if the pyramids GPMN, HSTU be divided in like manner, and
also the pyramids PAKL, SDQJt, we shall have e.g.

(sum of prisms in GPMN) : (sum of prisms in HSTU) = PMN:h.STU

= AABC:A£>EF,
and similarly for the second pair of pyramids.

The process may be continued indefinitely, and we shall always have
(sum of prisms in GABC) : (sum of prisms in HDEF) =A ABC :ADEF.

\end{notes}

\end{proposition}

\begin{proposition}
\label{prop:XII_5}

\begin{statement}
Pyramids which are of the same height and have triangular
bases are to one another as the bases.
\end{statement}

\begin{proof}

Let there be pyramids of the same height, of which the
triangles ABC, DEF are the bases and the points G, H the
vertices ;

I say that, as the base ABC ts to the base DEF, so is the
pyramid ABCG to the pyramid DEFH,

For, if the pyramid ABCG is not to the pyramid DEFH
as the base ABC is to the base DEF,
then, as the base ABC is to the base DEF, so will the
pyramid ABCG be either to some solid less than the pyramid
DEFH or to a greater.

Let it, first, be in that ratio to a less solid W, and let the
pyramid DEFH be divided into two pyramids equal to one
another and similar to the whole and into two equal prisms ;

then the two prisms are greater than the half of the whole
pyramid, [xii. 3]

Again, let the pyramids arising from the division be
similarly divided,

and let this be done continually until there are left over from
the pyramid DEFH some pyramids which are less than the
excess by which the pyramid DEFH exceeds the solid W.

[x, 1]

Let such be left, and let them be, for the sake of argument,
DQRS, STUH

therefore the remainders, the prisms in the pyramid DEFH,
are greater than the solid IV.

Let the pyramid ABCG also be divided similarly, and a
similar number of times, with the pyramid DEFH;
therefore, as the base ABC is to the base DEF, so are the
prisms in the pyramid ABCG to the prisms in the pyramid
DEFH. [xii. 4]

But, as the base ABC is to the base DEF, so also is the
pyramid ABCG to the solid W

therefore also, as the pyramid ABCG is to the solid W, so
are the prisms in the pyramid ABCG to the prisms in the
pyramid DEFH; [v. 11]

therefore, alternately, as the pyramid ABCG is to the prisms
in it, so is the solid W to the prisms in the pyramid DEFH.

[v. .6]

But the pyramid ABCG is greater than the prisms in it ;
therefore the solid W is also greater than the prisms in the
pyramid DEFH.

But it is also less :
which is impossible.

Therefore the prism ABCG is not to any solid less than
the pyramid DEFH as the base ABC is to the base DEF.

Similarly it can be proved that neither is the pyramid
DEFH to any solid less than the pyramid ABCG as the base
DEF is to the base ABC.

I say next that neither is the pyramid ABCG to any
solid greater than the pyramid DEFH as the base ABC is
to the base DEF.

For, if possible, let it be in that ratio to a greater solid W;
therefore, inversely, as the base DEF is to the base ABC,
so is the solid W to the pyramid ABCG.

But, as the solid W is to the solid ABCG, so is the
pyramid DEFH to some solid less than the pyramid ABCG,
as was before proved ; [xn. a, Lemma]

therefore also, as the base DEF is to the base ABC, so is
the pyramid DEFH to some solid less than the pyramid
ABCG: [v. m]

which was proved absurd.

Therefore the pyramid ABCG is not to any solid greater
than the pyramid DEFH as the base ABC is to the base
DEF.

But it was proved that neither is it in that ratio to a less
solid.

Therefore, as the base ABC is to the base DEF, so is
the pyramid ABCG to the pyramid DEFH
\end{proof}

\begin{notes}

In the two preceding propositions it has been shown how we can divide a
py ram id with a triangular base into (1) two equal prisms which are together
greater than half the pyramid and (2) two equal pyramids similar to the
original one, and that, if this process be continued with the two pyramids,
then with the four resulting pyramids, and so on, and if, further, another
pyramid of the same height as the original one be similarly divided, the sub-
division being made the same number of times, the sum of all the prisms in
one pyramid is to the sum of all the prisms in the other as the base of the
first is to the base of the second.

We can now prove in the manner of xn. 2 that the volumes of the
pyramids themselves are as the bases.

Let us call the pyramids P, P and their respective bases B, B.

If P.P + B.,

suppose that B-.B-P; W.

I. Let W be < P.

Divide P into two prisms and two pyramids, subdivide the latter similarly,
and so on, until the sum of the pyramids remaining is less than the difference
between P and W[x, 1], so that

P > (prisms in P) a- W.

Then divide P similarly, the same number of times.

Now (prisms in P) : (prisms in P) = B : B [xn. 4]

= P : W, by hypothesis,
and, alternately,

(prisms in P):P= (prisms in P) : W.

But (prisms in P) < P;

therefore (prisms in P) < W.

But, by construction, (prisms in P) > W.

Hence W cannot be less than P.

II. Suppose, if possible, that W> f.

Then, inversely, B : B = W.P,

= P'': V t
where V is some solid less than P. [Cf. xii. z. Lemma, and note.]

But this can be proved impossible exactly as in Part I.

Therefore Wis neither less nor greater than P'',

sothat B.BP.P.

Legendre, followed by the American editors already mentioned, and by
others, approaches the subject by a different route, proving the following
propositions.

r. If a pyramid be cut by a plant parallel to the base, (a) the lateral edges
and the height will be cut in the same proportion, (b) the section by the plant
will be a polygon similar to the base.

(a) Since a lateral face VAB of the pyramid V(ABCDE) is cut by two
parallel planes in AB, ab,

ABab;
Similarly BC j be, and so on.
Therefore VA : Va = VB: Vb= VC: Vem„„

And, if VO the height be cut in O, 0,

BO I! bo j and each of the above ratios is equal to VO : Vo.

(b) Since BA ba, znd BC\ be,

(,ABC= Lobe. [xi. 10]

Similarly for all the other angles of the polygons, which are therefore
equiangular.

Also, by similar triangles,

VA : Va = AB : ab,
and so on.

Therefore, by the ratios above,

AB:ab = BC; bc= ....
Therefore the polygons are similar.

2. If two pyramids of the same height be cut by plants which art at tht
same perpendicular distance from the vertices, the sections are as the rcsptctivt
baits.

For, if we place the pyramids so that the vertices coincide and the bases
are in one plane, the planes of the sections will coincide.

If, e.g. t the base of the second pyramid be XYZ and the section xyt, we
shall have, by the argument of the last proposition,

VX : Vx = VY : Vy = VZ: V* = VO : Ve= VA : Va = ... ,
and X YZ, xyz will be similar.

Now (polygon ABCDE) : (polygon abate) = AS 1 : aP

= FA' : Va 1 ,
and A XYZ \ bxy* = XY* : xy*

= VX* : Vx?
= VA* : Va*.
Therefore

(polygon ABCDE) : (polygon abcde) =AXYZ: xyt.
As a particular case, if the bases of the two pyramids are equivalent, the
sections art aha equivalent.

3. 7itx> triangular pyramids which have equivalent bases and equal heights
are equivalent.

Let VABC, vabc be pyramids with equivalent bases ABC, abe, which for
convenience we will suppose placed in one plane, and let TA be the common

height.

Then, if the pyramids are not equivalent, one must be greater than the other.

Let VABC be the greater ; and let AX be the height of a prism on ABC
as base which is equal in volume to the difference of the pyramids.

Divide the height AT into equal parts such that each is less than AX, and
let each part be equal to z.

Through the points of division draw planes parallel to the bases cutting
both pyramids in the sections DEF, GHI,... and def, ghi, ....

The sections DEF, def mi\ then be equivalent ; so will the sections GHI,
ghi, and so on. [(2) above]

On the triangles ABC, DEF, GHI, ... as bases draw exterior prisms
having for edges the parts AD, DG, GIC, ... of the edge A V;
and on the triangles def, ghi, ... as bases draw interior prisms having for edges
the parts aif, dg, ... of av.

All the partial prisms will then have the same height 1.

Now the sum of the exterior prisms of the pyramid VABC is greater than
that pyramid ;

and the sum of the interior prisms in the pyramid vabe is less than that
pyramid.

Consequently the difference between the sum of the first set of prisms and
the sum of the second set of prisms is greater than the difference between the
two pyramids.

Again, if we start from the bases ABC, air, the second exterior prism
DEJFG is equivalent to the first interior prism defa, since their bases are
equivalent and they have the same height z. [xt. 28 and note ; xi. 32]

Similarly the third exterior prism is equivalent to the second interior
prism, and so on, until we arrive at the last of each.

Therefore the prism ABCD   the first exterior prism, is the difference
between the sums of the exterior and interior prisms respectively.

Therefore the difference between the two pyramids is less than the prism
ABCD, which should therefore be greater than the prism with base ABC
and height AX.

But the prism ABCD is, by hypothesis, less than the latter prism :
which is impossible.

Consequently the pyramid VABC cannot be greater than the pyramid
k.

Similarly it may be proved that vabe cannot be greater than VABC.

Therefore the pyramids are equivalent.

Legendre next establishes a proposition corresponding to Eucl.\ XII. 7, vu.

4. Any triangular pyramid is one third of the triangular prism on the same
base and of the same height,

and from this he deduces that

Cor. The volume of a triangular pyramid is equal to a third of the product
of its base by its height.

He has previously proved that the volume of a triangular prism is equal to
the product of its base and height, since (1) the prism is half of a parallele-
piped of the same height and with a parallelogram for base which is double of
the base of the prism, and (2) this parallelepiped can be transformed into an
equivalent rectangular parallelepiped with the same height and an equivalent
base.

The theorem (4) is then extended to any pyramid in the proposition

5. Any pyramid has for its measure the third part of the product of its base
and its height, from which follow

Cor. I. Any pyramid is the third part of the prism on the same base and
of the same height.

Cor. II. Two pyramids of the same height are to one another as their
bases, and huo pyramids on the same base are to one another as their heights.

The first part of the second corollary corresponds to the present
proposition as extended by the next, xu. 6.

\end{notes}

\end{proposition}

\begin{proposition}
\label{prop:XII_6}

\begin{statement}
Pyramids which are of the same height and have polygonal
bases are to one another as the bases.
\end{statement}

\begin{proof}

Let there be pyramids of the same height of which the
polygons ABCDE, FGHKL are the bases and the points
M, N the vertices ;

I say that, as the base ABCDE is to the base FGHKL,
so is the pyramid ABCDEM to the pyramid FGHKLN.

For let AC, AD, FH, FK be joined.

Since then ABCM, ACDM are two pyramids which have
triangular bases and equal height,

they are to one another as the bases ; [xn. 5]

therefore, as the base ABC is to the base A CD, so is the
pyramid ABCM to the pyramid ACDM.

And, componendo, as the base A BCD is to the base A CD,
so is the pyramid ABCDM to the pyramid A CDM. [v. 18]

But also, as the base A CD is to the base ADE, so is the
pyramid ACDM to the pyramid ADEM. [xn. 5]

Therefore, ex aequali, as the base A BCD is to the base
ADE, so is the pyramid ABCDM to the pyramid ADEM.

[v. m]

And again componendo, as the base ABCDE is to the
base ADE, so is the pyramid ABCDEM to the pyramid
ADEM. [v. 18)

Similarly also it can be proved that, as the base FGHKL
is to the base FGH, so is the pyramid FGHKLN to the
pyramid FGHN.

And, since A DEM, FGHN are two pyramids which have
triangular bases and equal height,

therefore, as the base ADE is to the base FGH, so is the
pyramid A DEM to the pyramid FGHN. [xii. 5]

But, as the base ADE is to the base ABCDE, so was
the pyramid ADEM to the pyramid ABCDEM.

Therefore also, ex aequali, as the base ABCDE is to the
base FGH, so is the pyramid ABCDEM to the pyramid
FGHN. [v. i»]

But further, as the base FGH is to the base FGHKL, so
also was the pyramid FGHN to the pyramid FGHKL N.

Therefore also, ex atquali, as the base ABCDE is to the
base FGHKL, so is the pyramid ABCDEM to the pyramid
FGHKLN. [v. »]
\end{proof}

\begin{notes}

It will be seen that, in order to obtain the proportion

(base ABCDE) : A ADE - (pyramid M ABCDE) : (pyramid MADE),

Euclid employs v. 18 (componendo\ twice over, with an ex atquali step [v. aj]
intervening.

We might arrive at it more concisely by using v. 24 extended to any
number of antecedents,

Thus

AABC-.AADE = (pyramid MABC) : (pyramid MADE),

AACD :AADE = (pyramid MA CD) : (pyramid MADE),

and lastly

A ADE : AADE = (pyramid MADE) ; (pyramid MADE).

Therefore, adding the antecedents [v, 24], we have

(polygon ABCDE) :AADE = (pyramid M ABCDE) : (pyramid MADE).

Again, since the pyramids MADE, NFGHare of the same height,

A ADE ; AFGH (pyramid MADE) ; (pyramid JVFGH).

Lastly, using the same argument for the pyramid NFGHKL as for
MABCDE, and inverting, we have

A FGH: (polygon FGHKL) m (pyramid NFGH) : (pyramid NFGHKL).

Thus from the three proportions, ex aequali,

(polygon ABCDE) : (polygon FGHKL)

m (pyramid MABCDE) : (pyramid JVFGHJpL).

\end{notes}

\end{proposition}

\begin{proposition}
\label{prop:XII_7}

\begin{statement}
Any prism which has a triangular base is divided into three
pyramids equal to one another which have triangular bases.
\end{statement}

\begin{proof}

Let there be a prism in which the triangle ABC is the
base and DEF its opposite ;
I say that the prism ABCDEF is
divided into three pyramids equal to
one another, which have triangular
bases.

For let BD, EC, CD be joined.

Since ABED is a parallelogram,
and BD is its diameter,
therefore the triangle ABD is equal
to the triangle EBD \ [1. 34]

therefore also the pyramid of which the triangle ABD is the
base and the point C the vertex is equal to the pyramid of
which the triangle DEB is the base and the point C the
vertex. [so. 5]

But the pyramid of which the triangle DEB is the base
and the point C the vertex is the same with the pyramid of
which the triangle EBC is the base and the point D the
vertex ;
for they are contained by the same planes.

Therefore the pyramid of which the triangle ABD is the
base and the point C the vertex is also equal to the pyramid
of which the triangle EBC is the base and the point D the
vertex.

Again, since FCBE is a parallelogram,
and CE is its diameter,
the triangle CEF is equal to the triangle CBE. [1. 34]

Therefore also the pyramid of which the triangle BCE is
the base and the point D the vertex is equal to the pyramid
of which the triangle ECF is the base and the point D the
vertex. [*«, 5]

But the pyramid of which the triangle BCE is the base
and the point D the vertex was proved equal to the pyramid
of which the triangle ABD is the base and the point C the
vertex ;

therefore also the pyramid of which the triangle CEF is the
base and the point D the vertex is equal to the pyramid of
which the triangle ABD is the base and the point C the
vertex ;

therefore the prism ABCDEF has been divided into three
pyramids equal to one another which have triangular bases.

And, since the pyramid of which the triangle ABD is the
base and the point C the vertex is the same with the pyramid
of which the triangle CAB is the base and the point D the
vertex,

for they are contained by the same planes,
while the pyramid of which the triangle ABD is the base and
the point C the vertex was proved to be a third of the prism
in which the triangle ABC is the base and DBF its opposite,
therefore also the pyramid of which the triangle ABC is the
base and the point D the vertex is a third of the prism which
has the same base, the triangle ABC, and DBF as its
opposite.

Porism. From this it is manifest that any pyramid is a
third part of the prism which has the same base with it and
equal height
\end{proof}

\begin{notes}

If we denote by C-ABD a pyramid with vertex C and base ABD, Euclid's
argument is easily followed thus.

The CD ABED being bisected by BD,

(pyramid C-ABD) = (pyramid C-DEB) |xil. 5]

3 (pyramid DEBC).
And, the E3 EBCF being bisected by EC,

(pyramid DEBC) = (pyramid D-ECF).
Thus (pyramid C-ABD) = (pyramid D-EBC) = (pyramid D-ECF), and
these three pyramids make up the whole prism, so that each is one- third of the
prism.

And, since (pyramid C-ABD) a (pyramid D-ABC),

(pyramid D-ABC) = \ (prism ABC, DEE).

\end{notes}

\end{proposition}

\begin{proposition}
\label{prop:XII_8}

\begin{statement}
Similar pyramids which have triangular bases are in the
triplicate ratio of their corresponding sides.
\end{statement}

\begin{proof}

Let there be similar and similarly situated pyramids of
which the triangles ABC, DEF. are the bases and the points
G, H the vertices ;

I say that the pyramid ABCG has to the pyramid DEFH
the ratio triplicate of that which BC has to EF.

For let the parallelepipedal solids BGML, EHQP be
completed.

Now, since the pyramid ABCG is similar to the pyramid
DEFH,

therefore the angle ABC is equal to the angle DEF,

the angle GBC to the angle HEF,

and the angle AUG to the angle DEB ;

and, as AB is to DE, so is BC to EF, and BG to EH.

And since, as AB is to DE, so is BC to EF,

and the sides are proportional about equal angles,

therefore the parallelogram BM is similar to the parallelo-
gram EQ.

For the same reason

BN is also similar to ER, and BK to EO ;

therefore the three parallelograms MB, BK, BN are similar
to the three EQ, EO, ER.

But the three parallelograms MB, BK, BN are equal and
similar to their three opposkes,

and the three EQ, EO, ER are equal and similar to their

three opposkes. [xi. 14]

Therefore the solids BGML, EHQP are contained by
similar planes equal in multitude.

Therefore the solid BGML is similar to the solid EHQP.

But similar parallelepipedal solids are in the triplicate ratio
of their corresponding sides. [xi. 33]

Therefore the solid BGML has to the solid EHQP the
ratio triplicate of that which the corresponding side BC has to
the corresponding side EF.

But, as the solid BGML is to the solid EHQP, so is the
pyramid ABCG to the pyramid DEFH,
inasmuch as the pyramid is a sixth part of the solid, because
the prism which is half of the parallelepipedal solid [xi. 28J is
also triple of the pyramid. [xii. 7]

Therefore the pyramid ABCG also has to the pyramid
DEFH the ratio triplicate of that which BC has to EF.
\end{proof}

\begin{porism*}

Porism. From this it is manifest that similar pyramids
which have polygonal bases are also to one another in the
triplicate ratio of their corresponding sides.

For, if they are divided into the pyramids contained in
them which have triangular bases, by virtue of the fact that
the similar polygons forming their bases are also divided into
similar triangles equal in multitude and corresponding to the
wholes fvi. 20],

then, as the one pyramid which has a triangular base in the
one complete pyramid is to the one pyramid which has a
triangular base in the other complete pyramid, so also will all
the pyramids which have triangular bases contained in the
one pyramid be to all the pyramids which have triangular
bases contained in the other pyramid [v, u], that is, the
pyramid itself which has a polygonal base to the pyramid
which has a polygonal base.

But the pyramid which has a triangular base is to the
pyramid which has a triangular base in the triplicate ratio of
the corresponding sides ;

therefore also the pyramid which has a polygonal base has to
the pyramid which has a similar base the ratio triplicate of
that which the side has to the side.

\end{porism*}

\begin{notes}

It is at once proved that, the pyramids being similar, the parallelepipeds
constructed as shown in the figure are also similar.

Consequently, as these latter are In the triplicate ratio of their corre-
sponding sides [xi. 33], so are the pyramids which are their sixth parts
respectively (being one third of the respective prisms on the same bases, i.e.
of the halves of the respective parallelepipeds, xi. 28).

As the Porism is not used where Euclid might have been expected to use
it (see note on xn. 12, p, 416), there is some reason to doubt its genuineness.
P only has it in the margin, though in the first hand.

\end{notes}

\end{proposition}

\begin{proposition}
\label{prop:XII_9}

\begin{statement}
In equal pyramids which have triangular bases the bases
are reciprocally proportional to the heights ; and those pyramids
in which the bases are reciprocally proportional to the heights
are equal.
\end{statement}

\begin{proof}

For let there be equal pyramids which have the triangular
bases ABC, DEF and vertices the points G, H ;
I say that in the pyramids ABCG, DEFH the bases are
reciprocally proportional to the heights, that is, as the base
ABC is to the base DBF, so is the height of the pyramid
DEFH to the height of the pyramid ABCG.

For let the parallelepipedal solids SGML, EHQP be
completed.

Now, since the pyramid ABCG is equal to the pyramid
DEFH,

and the solid BGML is six times the pyramid ABCG.

and the solid EHQP six times the pyramid DEFH,

therefore the solid BGML is equal to the solid EHQP,

But in equal parallelepipedal solids the bases are recipro-
cally proportional to the heights ; [xi. 34]

therefore, as the base BM is to the base EQ, so is the height
of the solid EHQP to the height of the solid BGML.

But, as the base BM is to EQ, so is the triangle ABC to
the triangle DEF. [1. 34]

Therefore also, as the triangle ABC is to the triangle
DEF, so is the height of the solid EHQP to the height of
the solid BGML. [v. u]

But the height of the solid EHQP is the same with the
height of the pyramid DEFH,

and the height of the solid BGML is the same with the
height of the pyramid ABCG t

therefore, as the base ABC is to the base DEF, so is the
height of the pyramid DEFH to the height of the pyramid
ABCG.

Therefore in the pyramids ABCG, DEFH the bases are
reciprocally proportional to the heights.

Next, in the pyramids ABCG, DEFH let the bases be
reciprocally proportional to the heights ;
that is, as the base ABC is to the base DEF, so let the height
of the pyramid DEFH be to the height of the pyramid
ABCG ;

F say that the pyramid ABCG is equal to the pyramid
DEFH

For, with the same construction,
since, as the base ABC is to the base DEF, so is the height
of the pyramid DEFH to the height of the pyramid ABCG,

while, as the base ABC is to the base DEF, so is the
parallelogram BM to the parallelogram EQ,

therefore also, as the parallelogram BM is to the parallelogram
EQ, so is the height of the pyramid DEFH to the height of
the pyramid ABCG. [v. ti]

But the height of the pyramid DEFH is the same with
the height of the parallelepiped EHQP,

and the height of the pyramid ABCG is the same with the
height of the parallelepiped BGML ;

therefore, as the base BM is to the base EQ, so is the height
of the parallelepiped EHQP to the height of the parallelepi-
ped BGML.

But those parallelepipedal solids in which the bases are
reciprocally proportional to the heights are equal ; [xi. 34]

therefore the parallelepipedal solid BGML is equal to the
parallelepipedal solid EHQP.

And the pyramid ABCG is a sixth part of BGML, and
the pyramid DEFH a sixth part of the parallelepiped
EHQP;
therefore the pyramid ABCG is equal to the pyramid DEFH,
Therefore etc
\end{proof}

\begin{notes}

The volumes of the pyramids are respectively one sixth part of the volumes
of the parallelepipeds described, as in the figure, on double the bases and with
the same heights as the pyramids.

I. Thus the parallelepipeds are equal if the pyramids are equal.

And, the parallelepipeds being equal, their bases are reciprocally propor-
tional to their heights ; [xi. 34]
hence the bases of the equal pyramids (which are the halves of the bases of
the parallelepipeds) are proportional to their heights.

II. If the bases of the pyramids are reciprocally proportional to their
heights, so are the bases of the parallelepipeds to their heights (since the bases
of the parallelepipeds are double of the bases of the pyramids respectively).

Consequently the parallelepipeds are equal. fxi. 34]

Therefore their sixth parts, the pyramids, are also equal.

\end{notes}

\end{proposition}

\begin{proposition}
\label{prop:XII_10}

\begin{statement}
Any cone is a third part of the cylinder which has the same
base with it and equal height.
\end{statement}

\begin{proof}

For let a cone have the same base, namely the circle
ABCD, with a cylinder and equal
height ;

I say that the cone is a third part
of the cylinder, that is, that the
cylinder is triple of the cone.

For if the cylinder is not triple
of the cone, the cylinder will be
either greater than triple or less
than triple of the cone.

First let it be greater than
triple,

and let the square ABCD be
inscribed in the circle ABCD ; [iv. 6]

then the square ABCD is greater than the half of the circle
ABCD.

From the square ABCD let there be set up a prism of
squal height with the cylinder.

Then the prism so set up is greater than the half of the
cylinder,

inasmuch as, if we also circumscribe a square about the circle
ABCD[iv. 7], the square inscribed in the circle A BCD is half
of that circumscribed about it,

and the solids set up from them are parallelepipedal prisms of
equal height,

while parallelepipedal solids which are of the same height are
to one another as their bases ; [xi. 3*]

therefore also the prism set up on the square ABCD is half
of the prism set up from the square circumscribed about the
circle ABCD ; [cf. xi. 28, or xii. 6 and 7, Por.]

and the cylinder is less than the prism set up from the square
circumscribed about the circle ABCD;

therefore the prism set up from the square ABCD and of
equal height with the cylinder is greater than the half of the
cylinder.

Let the circumferences AB, BC, CD, DA be bisected at
the points E, E, G, H,

and let A £, EB, BE, EC, CG, GD, DH, HA be joined ;

then each of the triangles AEB, BEC, CGD, DMA is greater
than the half of that segment of the circle ABCD which is
about It, as we proved before. [xii. a]

On each of the triangles AEB, BEC, CGD, DMA let
prisms be set up of equal height with the cylinder ;

then each of the prisms so set up is greater than the half part
of that segment of the cylinder which is about it,

inasmuch as, if we draw through the points E, E, G, H
parallels to AB, BC, CD, DA, complete the parallelograms
on AB, BC, CD, DA, and set up from them parallelepipedal
solids of equal height with the cylinder, the prisms on the
triangles AEB, BEC, CGD, DMA are halves of the several
solids set up ;

and the segments of the cylinder are less than the parallelepi-
pedal solids set up ;

hence also the prisms on the triangles AEB, BEC, CGD,
DHA are greater than the half of the segments of the
cylinder about them.

Thus, bisecting the circumferences that are left, joining

straight lines, setting up on each of the triangles prisms of
equal height with the cylinder,

and doing this continually,

we shall leave some segments of the cylinder which will be
less than the excess by which the cylinder exceeds the triple
of the cone. [x, i]

Let such segments be left, and let them be AE, £B, BF,
FC, CG, GD, DH, HA ;

therefore the remainder, the prism of which the polygon
AEBFCGDH is the base and the height is the same as that
of the cylinder, is greater than triple of the cone.

But the prism of which the polygon AEBFCGDH is the
base and the height the same as that of the cylinder is triple
of the pyramid of which the polygon AEBFCGDH is the
base and the vertex is the same as that of the cone ; [xn. 7, Por.]

therefore also the pyramid of which the polygon AEBFCGDH
is the base and the vertex is the same as that of the cone is
greater than the cone which has the circle A BCD as base.

But it is also less, for it is enclosed by it:

which is impossible.

Therefore the cylinder is not greater than triple of the cone.

I say next that neither is the cylinder less than triple of
the cone,

For, if possible, let the cylinder be less than triple of the
cone ,

therefore, inversely, the cone is greater than a third part of

the cylinder.

Let the square ABCD be inscribed in the circle ABCD ;

therefore the square ABCD is greater than the half of the
circle ABCD.

Now let there be set up from the square ABCD a pyramid
having the same vertex with the cone ;

therefore the pyramid so set up is greater than the half part
of the cone,

seeing that, as we proved before, if we circumscribe a square
about the circle, the square A BCD will be half of the square
circumscribed about the circle,

and if we set up from the squares parallelepipedal solids of
equal height with the cone, which are also called prisms, the
solid set up from the square A BCD will be half of that set up
from the square circumscribed about the circle ;
for they are to one another as their bases. [xi. 32]

Hence also the thirds of them are in that ratio ;
therefore also the pyramid of which the square A BCD is the
base is half of the pyramid set up from the square circum-
scribed about the circle.

And the pyramid set up from the square about the circle
is greater than the cone,

for it encloses it.

Therefore the pyramid of which the square A BCD is the
base and the vertex is the same with that of the cone is
greater than the half of the cone.

Let the circumferences AB, BC, CD, DA be bisected at
the points E, F, G, H,

and let AE, EB, BF, FC, CG, GD, DH, HA be joined ;
therefore also each of the triangles AEB, BFC, CGD, DHA
is greater than the half part of that segment of the circle
ABCD which is about it

Now, on each of the triangles AEB, BFC, CGD, DHA
let pyramids be set up which have the same vertex as the
cone ;

therefore also each of the pyramids so set up is, in the same
manner, greater than the half part of that segment of the cone
which is about it.

Thus, by bisecting the circumferences that are left, joining
straight lines, setting up on each of the triangles a pyramid
which has the same vertex as the cone,

and doing this continually,

we shell leave some segments of the cone which will be less
than the excess by which the cone exceeds the third part of
the cylindei. [x. 1]

Let such be left, and let them be the segments on AE,
EB, BF, FC, CG, GD, DH, HA ;
therefore the remainder, the pyramid of which the polygon
AEBFCGDH is the base and the vertex the same with that
of the cone, is greater than a third part of the cylinder.

But the pyramid of which the polygon AEBFCGDH is
the base and the vertex the same with that of the cone is a
third part of the prism of which the polygon AEBFCGDH
is the base and the height is the same with that of the
cylinder ;

therefore the prism of which the polygon AEBFCGDH is
the base and the height is the same with that of the cylinder
is greater than the cylinder of which the circle A BCD is the
base.

But it is also less, for it is enclosed by it :
which is impossible.

Therefore the cylinder is not less than triple of the cone.

But it was proved that neither is it greater than triple ;
therefore the cylinder is triple of the cone ;
hence the cone is a third part of the cylinder.

Therefore etc
\end{proof}

\begin{notes}

We observe the use in this proposition of the term `` parallel pi pedal
prism,'' which recalls Heron's `` parallelogrammic `` or `` parallel-sided prism.''

The course of the proof is exactly the same as in xn. a, except that an
arithmetical fraction takes the place of a ratio which, being incommensurable,
could only be expressed as a ratio. Consequently we do not need proportions
in this proposition, as we did in xn. z, and shall again in xn. 1 1 , etc.

Euclid exhausts the cylinder and cone respectively by setting up prisms
and pyramids of the same height on the successive regular polygons inscribed
in the circle which is the common base, viz. the square, the regular polygon
of 8 sides, that of 16 sides, etc.

If A B be the side of one polygon, we obtain two sides of the next by
bisecting the arc ACB and joining A C, CB. Draw the
tangent DE at C and complete the parallelogram
A BED.

Now suppose a prism erected on the polygon of
which AB is a side, and of the same height as that of
the cylinder.

To obtain the prism of the same height on the next
polygon we add all the triangular prisms of the same
height on the bases A CB and the rest.

Now the prism on ACB is half the prism of the
same height on the CD ABED as base.

[cf. xi. a8)

And the prism on O ABED includes, and is greater than, the portion of
the cylinder standing on the segment ACB of the circle.

The same thing is true in regard to the other sides of the polygon of
which A B is one side.

Thus the process begins with a prism on the square inscribed in the circle,
which is more than half the cylinder, the next prism (with eight lateral faces)
takes away more than half the remainder, and so on ;

hence [x. 1], if we proceed far enough, we shall ultimately arrive at a prism
leaving over portions of the cylinder together less than any assigned volume.

The construction of pyramids on the successive polygons exhausts the cone
in exactly the same way.

Now, if the cone is not equal to one-third of the cylinder, it must be either
greater or less.

I. Suppose, if possible, that, V, O being their volumes respectively,

Construct successive inscribed polygons in the bases and prisms on them
until we arrive at a prism P leaving over portions of the cylinder together less
than (0- 3), i.e. such that

But P is triple of the pyramid on the same base and of the same height ;
and this pyramid is included by, and is therefore less than, V;
therefore P<.V.

But, by construction, P> 3 V

which is impossible.

Therefore O > 3 V.

II. Suppose, if possible, that O <. 3 V.
Therefore V> 0.

Construct successive pyramids in the cone in the manner described until
we arrive at a pyramid n leaving over portions of the cone together less than
(V-0\ i.e. such that

Now 11 is one-third of the prism oh the same base and of the same height;
and this prism is included by, and is therefore less than, the cylinder ;
therefore n < 0.

But, by construction, II > Oi

which is impossible.

Therefore is neither greater nor less than 3 V, so that

= V.

It will be observed that here, as in xii. 2, Euclid always ex/musts the solid
by (as it were) building up to it from inside. Hence the solid to be exhausted
must, with him, be supposed greater than the solid to which it is to be proved
equal ; and this is the reason why, in the second part, the initial supposition
is turned round.

In this case too Euclid might have approximated to the cone and cylinder
by circumscribing successive pyramids and prisms in the way shown, after
Archimedes, in the note on xn. z.

\end{notes}

\end{proposition}

\begin{proposition}
\label{prop:XII_11}

\begin{statement}
Cones and cylinders which are of the same height are to
one another as their bases.
\end{statement}

\begin{proof}

Let there be cones and cylinders of the same height,

let the circles A BCD, EFGH be their bases, XL, MN their
axes and AC, EG the diameters of their bases ;

I say that, as the circle A BCD is to the circle EFGH, so is
the cone AL to the cone EN.

For, if not, then, as the circle A BCD is to the circle
EFGH, so will the cone AL be either to some solid less
than the cone EN'' or to a greater.

First, let it be in that ratio to a less solid 0, and let the
solid X be equal to that by which the solid O is less than the
cone EN;

therefore the cone EN is equal to the solids O, X.

Let the square EFGH he inscribed in the circle EFGH;

therefore the square is greater than the half of the circle-
Let there be set up from the square EFGH a pyramid of

equal height with the cone ;

therefore the pyramid so set up is greater than the halfofthe
cone,

inasmuch as, if we circumscribe a square about the circle, and
set up from it a pyramid of equal height with the cone, the

inscribed pyramid is half of the circumscribed pyramid,

for they are to one another as their bases, [xn. 6]

while the cone is iess than the circumscribed pyramid.

Let the circumferences EF, FG, GH, HE be bisected at
the points P, Q, R, S,
and let HP, PE, EQ, QF, FR, FG, GS, SH be joined.

Therefore each of the triangles HPE, EQF, FRG, GSH
is greater than the half of that segment of the circle which is
about it.

On each of the triangles HPE, EQF, FRG, GSH let
there be set up a pyramid of equal height with the cone ;
therefore, also, each of the pyramids so set up is greater than
the half of that segment of the cone which is about it.

Thus, bisecting the circumferences which are left, joining
straight lines, setting up on each of the triangles pyramids of
equal height with the cone,
and doing this continually,

we shall leave some segments of the cone which will be less
than the solid X. [x. 1]

Let such be left, and let them be the segments on HP,
PE. EQ, QF, FR, RG, GS, SH;

therefore the remainder, the pyramid of which the polygon
HPFQFRGS is the base and the height the same with that
of the cone, is greater than the solid O.

Let there also be inscribed in the circle ABCD the
polygon D TA UBVCW similar and similarly situated to the
polygon HPEQFRGS,

and on it let a pyramid be set up of equal height with the cone
AL.

Since then, as the square on AC is to the square on EG, so
is the polygon DTA UBVCW to the polygon HPEQFRGS,

[xii. 1]

while, as the square on AC is to the square on EG, so is the
circle ABCD to the circle EFGH, [xii. a]

therefore also, as the circle ABCD is to the circle EFGH, so
is the polygon D TA UB VCW to the polygon HPEQFRGS.
But, as the circle ABCD is to the circle EFGH, so is the
cone AL to the solid 0,

and, as the polygon DTAUBVCW is to the polygon
HPEQFRGS, so is the pyramid of which the polygon
DTAUBVCW 'is the base and the point L the vertex to the
pyramid of which the polygon HPEQFRGS is the base and
the point A T the vertex. [w, 6]

Therefore also, as the cone AL is to the solid O, so is the
pyramid of which the polygon DTA UBVCW is the base and
the point L the vertex to the pyramid of which the polygon
HPEQFRGS is the base and the point N the vertex ; |v. 1 1]
therefore, alternately, as the cone AL is to the pyramid in it,
so is the solid O to the pyramid in the cone EN. [v. i6|

But the cone AL is greater than the pyramid in it ;
therefore the solid O is also greater than the pyramid in the
cone EN.

But it is also less:
which is absurd.

Therefore the cone AL is not to any solid less than the
cone EN as the circle A BCD is to the circle EFGH.

Similarly we can prove that neither is the cone EN to
any solid less than the cone AL as the circle EFGH is to the
circle ABCD.

I say next that neither is the cone AL to any solid greater
than the cone EN as the circle ABCD is to the circle
EFGH.

For, if possible, let it be in that ratio to a greater solid O

therefore, inversely, as the circle EFGH is to the circle
ABCD, so is the solid O to the cone AL.

But, as the solid O is to the cone AL, so is the cone EN
to some solid less than the cone AL ;

therefore also, as the circle EFGH is to the circle ABCD, so
is the cone EN to some solid less than the cone AL :

which was proved impossible.

Therefore the cone AL is not to any solid greater than
the cone EN as the circle ABCD is to the circle EFGH.

But it was proved that neither is it in this ratio to a less
solrd ;

therefore, as the circle ABCD is to the circle EFGH, so is
the cone AL to the cone EN.

But, as the cone is to the cone, so is the cylinder to the
cylinder,
for each is triple of each ; [xii. 10]

Therefore also, as the circle ABCD is to the circle
EFGH, so are the cylinders on them which are of equal
height

Therefore etc.
\end{proof}

\begin{notes}

We need not again repeat the preliminary construction of successive
pyramids and prisms exhausting the cones and cylinders.

Let Z, Z'' he the volumes of the two cones, ft ft their respective bases.

If P-.P*Z:Z',

then must ; = ZQ,

where O is either less or greater than Z'.

I. Suppose, if possible, that is less than Z .

Inscribe in Z' a pyramid (II') leaving over portions of it together less than
(Z' - O), i.e. such that

Z' =* W > 0.

Inscribe in Z a. pyramid II on a polygon inscribed in the circular base of
Z similar to the polygon which is the base of IT.
Now, if d, d' be the diameters of the bases,

J9 : /T =  : rf'' [xii. 1]

= (polygon in ft):(polygon in ft) [xn. 1]

= 11:11'. [xii. 6]

Therefore Z: = 11:11',

and, alternately, Z: U = : II'.

But ``/, > n, since it includes it ;

therefore 0> II'.

But, by construction, <W :

which is impossible.

Therefore 0<fcZ

II. Suppose, if possible, that

yJ:ft = Z:0,
where O is greater than Z'.

Therefore : ft = O : Z

where is some solid less than Z

That is, ft : = Z' : 0',

where O < Z

This is proved impossible exactly in the same way as the assumption in
Part I. was proved impossible.

Therefore Z has not either to a less solid than Z' or to a greater solid than
Z' the ratio of fl to ft j

therefore 0:ft = Z:Z'.

The same is true of the cylinders which are equal to Z, Z' respectively.

4™

\end{notes}

\end{proposition}

\begin{proposition}
\label{prop:XII_12}

\begin{statement}
Similar cones and cylinders are to one another in the
triplicate ratio of the diameters in their bases.
\end{statement}

\begin{proof}

Let there be similar cones and cylinders,
let the circles ABCD, EFGH be their bases, BD, FH the

diameters of the bases, and KL t MN the axes of the cones
and cylinders ;

I say that the cone of which the circle ABCD is the base and
the point L the vertex has to the cone of which the circle
EFGH is the base and the point N the vertex the ratio
triplicate of that which BD has to FH.

H

For, if the cone ABCDL has not to the cone EFGHN
the ratio triplicate of that which BD has to FH,

the cone ABCDL will have that triplicate ratio either to
some solid less than the cone EFGHN or to a greater.

First, let it have that triplicate ratio to a less solid O.

Let the square EFGH be inscribed in the circle EFGH;

[iv. 6]
therefore the square EFGH is greater than the half of the
circle EFGH.

Now let there be set up on the square EFGH a pyramid
having the same vertex with the cone ;

therefore the pyramid so set up is greater than the half part
of the cone.

Let the circumferences EF, FG, GH, HE be bisected at
the points P, Q, R, S,
and let EP, PF, FQ, QG, GR, RH, HS, SE be joined.

Therefore each of the triangles EPF, F JG, GRH, HSE
is also greater than the half part of that segment of the circle
EFGH which is about it.

Now on each of the triangles EPF, FQG, GRH, HSE
let a pyramid be set up having the same vertex with the cone;
therefore each of the pyramids so set up is also greater than
the half part of that segment of the cone which is about it.

Thus, bisecting the circumferences so left, joining straight
lines, setting up on each of the triangle- pyramids having the
same vertex with the cone,
and doing this continually,

we shall leave some segments of the cone which will be less
than the excess by which the cone EFGHN exceeds the
solid O. [x. 1]

Let such be left, and let them be the segmet: i on EP,
PF, FQ, QG, GR, RH, HS, SE ;

therefore the remainder, the pyramid of whirh the polygon
EPFQGRHS is the base and the point N the vertex, is
greater than the solid O.

Let there be also inscribed in the circle ABCD the
polygon ATBUCVDW similar and similarly situated to the
polygon EPFQGRHS,

and let there be set up on the polygon ATBUCVDW
pyramid having the same vertex with the cone ;

of the triangles containing the pyramid of which the polygon
ATBUCVDW is the base an 1 'ie point L the vertex let
LB The one,

and of the triangles containing the pyramid of which the
polygon EPFQGRHS is the base and the point N the vertex
let NFP be one ;
and let KT, MP be joined.

Now, since the cone ABCDL is similar to the cone
EFGHN,

therefore, as BD is to FH, so is the axis KL to th axis MN.

[xi. Def. 24]

4ia BOOK XII [xii. la

But, as BD is to FN, so is BK to FM;
therefore also, as BK is to FM, so is KL to MN.

And, alternately, as BK is to AX, so is FM to  /!.

[v. ,6]

And the sides are proportional about equal angles, namely
the angles BKL, FMN ;
therefore the triangle BKL is similar to the triangle FMN..

[vl 6]

Again, since, as BK is to K T, so is FM to MP,
and they are about equal angles, namely the angles BKT,
FMP,

inasmuch as, whatever part the angle BKT is of the four
right angles at the centre K, the same part also is the angle
FMP of the four right angles at the centre M ;
since then the sides are proportional about equal angles,
therefore the triangle BKT is similar to the triangle FMP.

[vi. 6]

Again, since it was proved that, as BK is to KL, so is FM

to MN,

while BK is equal to KT, and FM to PM,
therefore, as TK is to KL, so is PM to MN ;
and the sides are proportional about equal angles, namely
the angles TKL, PMN, for they are right;
therefore the triangle LKT is similar to the triangle NMP.

[vi. 6]

And since, owing to the similarity of the triangles LKB,
NMF,

as LB is to BK, so is NF to FM,

and, owing to the similarity of the triangles BKT, FMP,
as KB is to B T, so is MF to FP,
therefore, ex aequali, as LB is to BT, so is NF to FP, [v. as]

Again since, owing to the similarity of the triangles L TK,
NPM,

as L T is to TK, so is NP to PM,

and, owing to the similarity of the triangles TKB, PMF,
as KT is to TB, so is MP to  ;
therefore, ex aequali, as LT is to 7!, so is AT/ 3 to /. [v. m]

But it was also proved that, as TB is to BL, so is PF
to FN.

Therefore, ex aequali, as TL is to LB, so is PN to NF.

[v. ix]

Therefore in the triangles LTB, NPF the sides are

proportional ;

therefore the triangles LTB, NPF arc equiangular; [vi. 5]

hence they are also similar, [vi. Def. 1]

Therefore the pyramid of which the triangle BKT is the
base and the point L the vertex is also similar to the pyramid
of which the triangle FMP is the base and the point A'' the
vertex,

for they are contained by similar planes equal in multitude.

[xi. Def. 9]

But similar pyramids which have triangular bases are to
one another in the triplicate ratio of their corresponding sides.

[xii. 8]

Therefore the pyramid BKTL has to the pyramid FMPN
the ratio triplicate of that which BK has to FM.

Similarly, by joining straight lines from A, IV, D, V, C, U
to K, and from F, S, H, R, G, Q to M, and setting up on
each of the triangles pyramids which have the same vertex
with the cones,

we can prove that each of the similarly arranged pyramids
will also have to each similarly arranged pyramid the ratio
triplicate of that which the corresponding side BK has to the
corresponding side FM, that is, which BD has to FH.

And, as one of the antecedents is to one of the conse-
quents, so are all the antecedents to all the consequents ;

[v. 12]
therefore also, as the pyramid BKTL is to the pyramid
FMPN, so is the whole pyramid of which the polygon
ATBUCVDW'xs the base and the point L the vertex to the
whole pyramid of which the polygon EPFQGRHS is the
base and the point N the vertex ;

hence also the pyramid of which A TB UCVD IV is the base
and the point L the vertex has to the pyramid of which the
polygon EPFQGRHS is the base and the point N the
vertex the ratio triplicate of that which BD has to FH.

But, by hypothesis, the cone of which the circle ABCD
is the base and the point L the vertex has also to the solid
the ratio triplicate of that which BD has to FH;

therefore, as the cone of which the circle ABCD is the base
and the point L the vertex is to the solid O, so is the pyramid
of which the polygon ATBUCVDW is the base and L the
vertex to the pyramid of which the polygon EPFQGRHS is
the base and the point N the vertex ;

therefore, alternately, as the cone of which the circle ABCD
is the base and L the vertex is to the pyramid contained in
it of which the polygon A TB UC VD W is the base and L
the vertex, so is the solid O to the pyramid of which the
polygon EPFQGRHS is the base and N the vertex, [v. 16]

But the said cone is greater than the pyramid in it ;

for it encloses it.

Therefore the solid O is also greater than the pyramid of
which the polygon EPFQGRHS is the base and N the
vertex.

But it is also less :

which is impossible.

Therefore the cone of which the circle ABCD is the base
and /, the vertex has not to any solid less than the cone of
which the circle EFGH is the base and the point N the
vertex the ratio triplicate of that which BD has to FH:

Similarly we can prove that neither has the cone EFGHN
to any solid less than the cone ABCDL the ratio triplicate
of that which FH has to BD.

I say next that neither has the cone ABCDL to any
solid greater than the cone EFGHN the ratio triplicate of
that which BD has to FH,

For, if possible, let it have that ratio to a greater solid 0.

Therefore, inversely, the solid has to the cone ABCDL
the ratio triplicate of that which FH has to BD.

But, as the solid O is to the cone ABCDL, so is the
cone EFGHN to some solid less than the cone ABCDL.

Therefore the cone EFGHN also has to some solid less
than the cone ABCDL the ratio triplicate of that which FH
has to BD :
which was proved impossible.

Therefore the cone ABCDL has not to any solid greater
than the cone EFGHN the ratio triplicate of that which BD
has to FH.

But it was proved that neither has it this ratio to a less
solid than the cone EFGHN.

Therefore the cone ABCDL has to the cone EFGHN
the ratio triplicate of that which BD has to FH.

But, as the cone is to the cone, so is the cylinder to the
cylinder,

for the cylinder which is on the same base as the cone and
of equal height with it is triple of the cone; [xn. 10J

therefore the cylinder also has to the cylinder the ratio
triplicate of that which BD has to FH.
Therefore etc.
\end{proof}

\begin{notes}

The method of proof is precisely that of the previous proposition. The
only addition is caused by the necessity of proving that, if similar equilateral
polygons be inscribed in the bases of two similar cones, and pyramids be
erected on them with the same vertices as those of the cones, the pyramids
(are similar and) are to one another in the triplicate ratio of corresponding
edges.

Let KL, MJVba the axes of the cones, L, iVthe vertices, and let BT, FP
be sides of similar polygons inscribed in the bases. Join BK, TK, BL, TL,
PM, FM, PN, FN.

I  / '[  J

Now BKL, FMN are right-angled triangles, and, since the cones are
similar,

BK : KL = FM : MN. [xi. Def. 14]

Therefore (1) As BKL, FMN are similar. [vi. 6]

Similarly (2) As TKL, PMNxk, similar.

Next, in As BKT, FMP, the angles BKT, FMP are equal, since each is
the same fraction of four right angles ; and the sides about the equal angles are
proportional ;
therefore (3) As BKT, FMP are similar.

Again, since from the similar A s BKL, FMN, and the similar A s BKT,
FMP respectively,

LB : BK = NF ': FM,

BK:BT=MF:FP,

ex aequali, LB : BT= NF: FP.

Similarly LT : TB = NP ; PF.

Inverting the latter ratio and compounding it with the preceding one, we
have, ex aequali,

LB:LT= NF: NP.

Thus in As LTB, NPFthe sides are proportional in pairs;
therefore (4) As LTB, NPFate similar.

Thus the partial pyramids L-BKT, N-FMP are similar.

In exactly the same way it is proved that all the other partial pyramids are
similar.

Now
(pyramid L-BKT) ! (pyramid N-FMP) = ratio triplicate of ( BK: FM).

The other partial pyramids are to one another in the same triplicate ratio.

The sum of the antecedents is therefore to the sum of the consequents in
the same triplicate ratio,
i.e. (pyramid L-A TBU...): (pyramid N-EPFQ.. . )

= ratio triplicate of ratio (BK: FM)
= ratio triplicate of ratio (BD : FH).

[The fact that Euclid makes this transition from the partial pyramids to
the whole pyramids in the body of this proposition seems to me to suggest
grave doubts as to the genuineness of the Porism to xu. 8, which contains a
similar but rather more general extension from the case of triangular pyramids
to pyramids with polygonal bases. Were that Porism genuine, Euclid would
have been more likely to refer to it than to repeat here the same arguments
which it contains,]

Now we are in a position to apply the method of exhaustion.

If A'', X' be the volumes of the cones, d, d' the diameters of their bases, and if
(ratio triplicate of d : d') * X : X',
then must (ratio triplicate of d : d') = X : O,

where O is either less or greater than X'.

I. Suppose that O is lets than X'.

Construct in the way described a pyramid (IT) in X' leaving over portions
of X' together less than (Jf - O), so that X > W > O,
and construct in X a pyramid (II), with the same vertex as X has, on a
polygon inscribed in its base similar to the base of IT.

Then, by what has just been proved,

II : n' = (ratio triplicate of d : d')
= X:0, by hypothesis,
and, alternately, U:X=W:0.

But X includes, and is therefore greater than, II ;
therefore O > n'.

But, by construction, < U' :

which is impossible. .

Therefore cannot be less than X'.

II. Suppose, if possible, that

(ratio triplicate of d-.d') = X: O,
where is greater than X' ;

then (ratio triplicate of d : d' ) = Z : X ',

or, inversely, (ratio triplicate of d' : d) = X' : Z,

where Z is some solid less than X.

This is proved impossible by the exact method of Part I.

Hence O cannot be either greater or less than X',
and X : X' = (ratio triplicate of ratio d : d').

\end{notes}

\end{proposition}

\begin{proposition}
\label{prop:XII_13}

\begin{statement}
If a cylinder be cut by a plane which is parallel to its
apposite planes, then, as the cylinder is to the cylinder, so will
the axis be to the axis.
\end{statement}

\begin{proof}

For let the cylinder AD be cut by the plane GH which
is parallel to the opposite planes AB, CD,
and let the plane GH meet the axis at the point K
I say that, as the cylinder BG is to the cylinder GD, so is
the axis EK to the axis KF.

For let the axis EF be produced in both directions to the
points L, M,

and let there be set out any number whatever of axes EN, NL
equal to the axis EK,

and any number whatever FO, OM equal to FK ;
and let the cylinder PW on the axis LM be conceived of
which the circles PQ, VW me the bases.

Let planes be carried through the points N, O parallel to
AB, CD and to the bases of the cylinder PW,
and let them produce the circles RS, TU about the centres
N, 0.

Then, since the axes LN, NE, EK are equal to one
another,
therefore the cylinders QR, RB, BG are to one another as
their bases. [xu. 1 1]

But the bases are equal ;
therefore the cylinders QR, RB, BG are also equal to one

another.

Since then the axes LN, NE, EK are equal to one
another,

and the cylinders QR, RB, BG are also equal to one another,
and the multitude of the former is equal to the multitude of
the latter,

therefore, whatever multiple the axis KL is of the axis EK,
the same multiple also will the cylinder QG be of the
cylinder GB.

For the same reason, whatever multiple the axis MK is
of the axis KF, the same multiple also is the cylinder WG
of the cylinder GD.

And, if the axis KL is equal to the axis KM, the cylinder
QG will also be equal to the cylinder GW,

if the axis is greater than the axis, the cylinder will also be

greater than the cylinder,
and if less, less.

Thus, there being four magnitudes, the axes EK, KF
and the cylinders BG, GD,

there have been taken equimultiples of the axis EK and of
the cylinder BG, namely the axis Z.A'and the cylinder QG,
and equimultiples of the axis KF and of the cylinder GD,
namely the axis KM and the cylinder GW ;
and it has been proved that,

if the axis KL is in excess of the axis KM, the cylinder QG
is also in excess of the cylinder GW,

if equal, equal,
and if less, less.

Therefore, as the axis EK is to the axis KF, so is the
cylinder BG to the cylinder GD. [v. Def. 5]
\end{proof}

\begin{notes}

It is not necessary to reproduce the proof, as it follows exactly the method
of vi. 1 and xi. as.

The fact that cylinders described about axes of equal length and having
equal bases are equal is inferred from xii. 11 to the effect that cylinders of
equal height are to one another as their bases.

That, of two cylinders with unequal axes but equal bases, the greater is
that which has the longer axis is of course obvious either by application or by
cutting off from the cylinder with the longer axis a cylinder with an axis of the
same length as that of the other given cylinder.

\end{notes}

\end{proposition}

\begin{proposition}
\label{prop:XII_14}

\begin{statement}
Cones and cylinders which are on equal bases are to one
anotker as their heights.
\end{statement}

\begin{proof}

For let EB, FD be cylinders on equal bases, the circles
AB, CD ;

I say that, as the cylinder EB is
to the cylinder FD, so is the axis
GH to the axis KL.

For let the axis KL be pro-
duced to the point N,
let LN be made equal to the axis aITTOb riiTiM

GH, -

and let the cylinder CM be conceived about LN as axis.

Since then the cylinders EB, CM are of the same height,
they are to one another as their bases. [xii. 1 1]

But the bases are equal to one another ;
therefore the cylinders EB, CM 'are also equal.

And, since the cylinder FM has been cut by the plane
CD which is parallel to its opposite planes,
therefore, as the cylinder CM is to the cylinder FD, so is the
axis LN to the axis KL, [xii. 13]

But the cylinder CM is equal to the cylinder EB,
and the axis LN to the axis GH;

therefore, as the cylinder EB is to the cylinder FD, so is the
axis GH to the axis KL.

But, as the cylinder EB is to the cylinder FD, so is the
cone ABG to the cone CDK, [xii. 10]

Therefore also, as the axis GH is to the axis KL, so is
the cone ABG to the cone CDK and the cylinder EB to the
cylinder FD.
\end{proof}

\begin{notes}

No separate proposition corresponding to this is necessary in the case of
parallelepipeds, for xi. 25 really contains the property corresponding to that in
this proposition as well as the property corresponding to that in XI 1. 13.

\end{notes}

\end{proposition}

\begin{proposition}
\label{prop:XII_15}

\begin{statement}
In equal cones and cylinders the bases are reciprocally
proportional to the heights ; and those cones and cylinders in
which the bases are reciprocally proportional to the lieights are
equal.
\end{statement}

\begin{proof}

Let there be equal cones and cylinders of which the circles
ABCD, EFGH are the bases ;
let AC, EG be the diameters of the bases,
and KL, MN the axes, which are also the heights of the
cones or cylinders ;
let the cylinders AO, EP be completed.

I say that in the cylinders AO, EP the bases are re-
ciprocally proportional to the heights,

that is, as the base ABCD is to the base EFGH, so is the
height MN to the height KL.

For the height LK is either equal to the height MN or
not equal.

First, let it be equal.

Now the cylinder AO is also equal to the cylinder EP.

But cones and cylinders which are of the same height are
to one another as their bases ; [xii. 1 1]

therefore the base ABCD is also equal to the base EFGH.

Hence also, reciprocally, as the base ABCD is to the base
EFGH, so is the height MN to the height KL.

Next, let the height LK not be equal to MN,
but let MN be greater ;

from the height MN let QN be cut off equal to KL,
through the point Q let the cylinder EP be cut by the plane
TUS parallel to the planes of the circles EFGH, RP,

and let the cylinder ES be conceived erected from the circle
EFGH as base and with height NQ.

Now, since the cylinder AO is equal to the cylinder EP,
therefore, as the cylinder AO is to the cylinder ES, so is the
cylinder EP to the cylinder ES. [v. 7]

But, as the cylinder AO is to the cylinder ES, so is the
base ABCD to the base EFGH,
for the cylinders AO, ES are of the same height ; [xu. 11]

and, as the cylinder EP is to the cylinder ES, so is the height
MN to the height QN,

for the cylinder EP has been cut by a plane which is parallel
to its opposite planes. [xu. 13]

Therefore also, as the base ABCD is to the base EFGH,
so is the height MN to the height QN. [v. n]

But the height QN is equal to the height KL
therefore, as the base ABCD is to the base EFGH, so is the
height MN to the height KL.

Therefore in the cylinders AO, EP the bases are re-
ciprocally proportional to the heights.

Next, in the cylinders AO, EPet the bases be reciprocally
proportional to the heights,

that is, as the base ABCD is to the base EFGH, so let the
height MN be to the height KL

I say that the cylinder AOis equal to the cylinder EP.

For, with the same construction,

since, as the base ABCD is to the base EFGH, so is the
height MN to the height KL,

while the height KL is equal to the height QN,

therefore, as the base ABCD is to the base EFGH, so is the

height MN to the height QN

But, as the base ABCD is to the base EFGH, so is the
cylinder AO to the cylinder ES,

for they are of the same height ; [xu. n]

and, as the height MN is to QN, so is the cylinder EP to the
cylinder ES; [xu. 13]

therefore, as the cylinder A O is to the cylinder ES, so is the
cylinder EP to the cylinder ES. [v. n]

Therefore the cylinder AO is equal to the cylinder EP.
And the same is true for the cones also.
\end{proof}

\begin{notes}

I. If the heights of the two cylinders are equal, and their volumes are
equal, the bases are equal, since the latter are proportional to the volumes,

fxn. u]
If the heights are not equal, cut off from the higher cylinder a cylinder of
the same height as the lower.

Then, if LK, QNYx the equal heights,
we have, by xir. n,

(base ASCD) : (base EFGH) = (cylinder AO) :

= (cylinder EP) I

= MN
= MN

ON
KL.

i (cylinder ES)
l (cylinder ES),
by hypothesis,

[in. 13]

II. In the converse part of the proposition, Euclid omits the case where
the cylinders have equal heights. In this case of course the reciprocal ratios
are both ratios of equality ; the bases are therefore equal, and consequently the
cylinders.

If the heights are not equal, we have, with the same construction as before,

(base A BCD)  (base EFGH) = MN : KL.
But [xn. nj

(base ABCD) : (base EFGH) - (cylinder AO) : (cylinder ES),
and MN:KL = MNQN

m (cylinder EP) : (cylinder ES). [xn. 13]
Therefore

(cylinder AO)   (cylinder ES) = (cylinder EP) : (cylinder ES),
and consequently (cylinder AO) = (cylinder EP).

Similarly for the cones, which are equal to one-third of the cylinders
respectively.

Legendre deduces these propositions about cones and cylinders
others which he establishes by a method similar
to that adopted by him for the theorem of xn. a
(see note on that proposition).

The first (for the cylinder) is as follows.

The volume of a cylinder is equal to the
product of its base by its height.

Suppose CA to be the radius of the base of
the given cylinder, h its height.

For brevity let us denote by (surf. CA) the
area of the circle of which CA is the radius.

If (surf. CA)x h is not the measure of the
given cylinder, it will be the measure of a
cylinder greater or less than it.

L First let it be the measure of a less
cylinder, that, for example, of which the circle with radius CD is the base, and
h is the height

from two

Circumscribe about the circle with radius CD a regular polygon GHI ...
such that its sides do not anywhere meet the circle with radius CA. [See note
on xii. a, p. 393 above, for Legendre's lemma relating to this construction.]

Imagine a prism erected on the polygon as base and with height h.

Then (volume of prism) = (polygon GHI...) x h.

[Legendre has previously proved this proposition, first for a parallelepiped
(by transforming it into a rectangular one), then for a triangular prism (half of
a parallelepiped of the same height), and lastly for a prism with a polygonal
base.]

But (polygon GHI. ..)< (surf. CA).

Therefore (volume of prism) < (surf. CA) x A

< (cylinder on circle of rad. CD),
by hypothesis.

But the prism is greater than the latter cylinder, since it includes it ;
which is impossible.

II. In order not to multiply figures let us, in this second cose, suppose
that CD is the radius of the base of the given cylinder, and that (surf. CD) * h
is the measure of a cylinder greater than it, e.g. a cylinder on the circle with
radius CA as base and of height h.

Then, with the same construction,

(volume of prism) = (polygon GHI...) x h.

And (polygon GHI...t> (surf. CD

Therefore (volume of prism) > (surf. CD) x k

> (cylinder on surf. CA), by hypothesis.

But the volume of the prism is also less than that cylinder, being included
by it:
which is impossible.

Therefore (volume of cylinder) = (its base) x (its height).

It follows as a corollary that

Cylinders of the same height are to one another as their bases fxn. 13], and
cylinders on the same base are to one another as their heights [xti. 14].

Also

Similar cylinders are as the cubes of their heights, or as the cubes of the
diameters of their oases [Eucl.\ XII. 13].

For the bases are as the squares on their diameters ; and, since the
cylinders are similar, the diameters of the bases are as their heights.

Therefore the bases are as the squares on the heights, and the bases
multiplied by the heights, or the cylinders themselves, are as the cubes of the
heights.

I need not reproduce Legendre's proofs of the corresponding propositions
for the cone.

\end{notes}

\end{proposition}

\begin{proposition}
\label{prop:XII_16}

\begin{statement}
Given two circles about the same centre, to inscribe in the
greater circle an equilateral polygon with an even number of
sides which does not touch the lesser circle.
\end{statement}

\begin{proof}

Let ABCD, EFGH be the two given circles about the
same centre K;

thus it is required to inscribe in the
greater circle ABCD an equilateral
polygon with an even number of
sides which does not touch the circle
EFGH.

For let the straight line BKD
be drawn through the centre K,

and from the point G let GA be
drawn at right angles to the straight
line BD and carried through to C ;

therefore AC touches the circle EFGH. [in. 16, Por.]

Then, bisecting the circumference BAD, bisecting the
half of it, and doing this continually, we shall leave a circum-
ference less than AD. [x, 1]

Let such be left, and let it be LD ;

from L let LM be drawn perpendicular to BD and carried
through to N,

and let LD, DN be joined ;

therefore LD is equal to DN. [in. 3, 1. 4]

Now, since LN is parallel to AC,

and AC touches the circle EFGH,

therefore LN does not touch the circle EFGH;

therefore LD, DN are far from touching the circle EFGH.

If then we fit into the circle ABCD straight lines equal
to the straight line LD and placed continuously, there will
be inscribed in the circle ABCD an equilateral polygon with
an even number of sides which does not touch the lesser
circle EFGH.

Q.E.F.
\end{proof}

\begin{notes}

It must be carefully observed that the polygon inscribed in the outer circle
in this proposition is such that not only do its own sides not touch the inner
circle, but also the chords, as LN, joining angular points next hut out to each
other do not touch the inner circle either. In other words, the polygon is the
second in order, not the first, which satisfies the condition of the enunciation.
This is important, because such a polygon is wanted in the next proposition ;
hence in that proposition the exact construction here given must be followed.

\end{notes}

\end{proposition}

\begin{proposition}
\label{prop:XII_17}

\begin{statement}
Given two spheres about the same centre, to inscribe in the
greater sphere a polyhedral solid which does not touch the
lesser sphere at its surface.
\end{statement}

\begin{proof}

Let two spheres be conceived about the same centre A
thus it is required to inscribe in the greater sphere a poly-
hedral solid which does not touch the lesser sphere at its
surface.

Let the spheres be cut by any plane through the centre ;
then the sections will be circles,
inasmuch as the sphere was produced by the diameter
remaining fixed and the semicircle being carried round it ;

[xi. Def. 14]
hence, in whatever position we conceive the semicircle to be,
the plane carried through it will produce a circle on the
circumference of the sphere.

And it is manifest that this circle is the greatest possible,
inasmuch as the diameter of the sphere, which is of course
the diameter both of the semicircle and of the circle, is greater
than all the straight lines drawn across in the circle or the
sphere.

Let then BCDE be the circle in the greater sphere,
and FGH the circle in the lesser sphere ;
let two diameters in them, BD, CE. je drawn at right angles
to one another ;

then, given the two circles BCDE, FGH about the same
centre, let there be inscribed in the greater circle BCDE an
equilateral polygon with an even number of sides which does
not touch the lesser circle FGH,

let BK, KL, LM, ME be its sides in the quadrant BE.
let KA be joined and carried through to N,
let AO be set up from the point A at right angles to the
plane of the circle BCDE, and let it meet the surface of the
sphere at O,

and through AO and each of the straight lines BD, KN let
planes be carried ;

they will then make greatest circles on the surface of the
sphere, for the reason stated.

Let them make such,
and in them let BOD, KON be the semicircles on BD, KN.

Now, since OA is at right angles to the plane of the circle
BCDE,

therefore all the planes through OA are also at right angles
to the plane of the circle BCDE ; [xi. 18]

hence the semicircles BOD, KON are also at right angles to
the plane of the circle BCDE.

And, since the semicircles BED, BOD, KON are equal,
for they are on the equal diameters BD, KN,
therefore the quadrants BE, BO, KO are also equal to one
another.

Therefore there are as many straight lines in the quadrants
BO, KO equal to the straight lines BK, KL, LM, ME as
there are sides of the polygon in the quadrant BE.

Let them be inscribed, and let them be BP, PQ, QR, RO
and KS, ST, TU, UO,
let SP, TQ, UR be joined,
and from P, S let perpendiculars be drawn to the plane of the
circle BCDE ; [xi. 1 1]

these will fall on BD, KN, the common sections of the planes,
inasmuch as the planes of BOD, KONare also at right angles
to the plane of the circle BCDE. [c£ xi. Def. 4]

Let them so fall, and let them be PV, SW,
and let WVbe joined.

Now since, in the equal semicircles BOD, KON, equal
straight lines BP, KS have been cut off,
and the perpendiculars PV, SWhaxe been drawn,
therefore PV is equal to SW, and BV to KW. [in. 27, 1. 26]

But the whole BA is also equal to the whole KA ;
therefore the remainder VA is also equal to the remainder WA ;
therefore, as BV  to VA, so is Ato WA ;
therefore WVs parallel to KB. [vi. 2]

And, since each of the straight lines PV, SW is at right
angles to the plane of the circle BCDE,
therefore PV is parallel to SW. [xi. 6]

But it was also proved equal to it ;
therefore WV, SP are also equal and parallel. [1. 33]

And, since WV is parallel to SP,
while WV is parallel to KB,
therefore SP is also parallel to KB. [xi. 9]

And BP, Adjoin their extremities;
therefore the quadrilateral KBPS is in one plane,
inasmuch as, if two straight lines be parallel, and points be
taken at random on each of them, the straight line joining the
points is in the same plane with the parallels. [xi. 7]

For the same reason
each of the quadrilaterals SPQ T, TQR U is also in one plane.
But the triangle URO is also in one plane. [xi. 2]

If then we conceive straight lines joined from the points
P, S, Q, T, R, U to A, there will be constructed a certain
polyhedral solid figure between the circumferences BO, KO,
consisting of pyramids of which the quadrilaterals KBPS,
SPQT, TQR U and the triangle URO are the bases and the
point A the vertex.

And, if we make the same construction in the case of each
of the sides KL, LM, ME as in the case of BK, and further
in the case of the remaining three quadrants,
there will be constructed a certain polyhedral figure in-
scribed in the sphere and contained by pyramids, of which
the satd quadrilaterals and the triangle URO, and the others
corresponding to them, are the bases and the point A the
vertex.

1 say that the said polyhedron will not touch the lesser
sphere at the surface on which the circle FGH is.

Let AX be drawn from the point A perpendicular to the
plane of the quadrilateral KB PS, and let it meet the plane at
the point X ; [xi. 11]

let XB, XK be joined.

Then, since AX is at right angles to the plane of the
quadrilateral KBPS,

therefore it is also at right angles to all the straight lines
which meet it and are in the plane of the quadrilateral.

[xi. Def. 3]

Therefore AX is at right angles to each of the straight
lines BX, XK.

And, since AB is equal to AK,

the square on AB is also equal to the square on AK.

And the squares on AX, XB are equal to the square
on AB,

for the angle at X is right ; [1. 47]

and the squares on AX, XK are equal to the square on AK.

[U]

Therefore the squares on A X, XB are equal to the squares
on AX, XK.

Let the square on AX be subtracted from each ;

therefore the remainder, the square on BX, is equal to the
remainder, the square on XK;

therefore BX is equal to XK.

Similarly we can prove that the straight lines joined
from X to P, S are equal to each of the straight lines BX,
XK.

Therefore the circle described with centre X and distance
one of the straight lines XB, XK will pass through P, S also,
and KB PS will be a quadrilateral in a circle.

Now, since KB is greater than WV,
while WVs equal to SP,
therefore KB is greater than SP.

But KB is equal to each of the straight lines KS, BP
therefore each of the straight lines KS, BP is greater than SP,

And, since KB PS is a quadrilateral in a circle,
and KB, BP, KS are equal, and PS less,
and BX is the radius of the circle,

therefore the square on KB is greater than double of the
square on BX.

Let KZ be drawn from K perpendicular to B V.

Then, since BD is less than double of DZ,
and, as BD is to DZ, so is the rectangle DB, BZ to the
rectangle DZ, ZB,

if a square be described upon BZ and the parallelogram on
ZD be completed,

then the rectangle DB, BZ is also less than double of the
rectangle DZ, ZB,

And, if KD be joined,
the rectangle DB, BZ is equal to the square on BK,

and the rectangle DZ, ZB equal to the square on KZ ;

[in. 31, vi. 8 and Por.J
therefore the square on KB is less than double of the square
on KZ.

But the square on KB is greater than double of the square
on BX ;
therefore the square on KZ is greater than the square on BX.

And, since BA is equal to KA,
the square on BA is equal to the square on AK.

And the squares on BX, XA are equal to the square on BA,
and the squares on KZ, ZA equal to the square on KA ;

[' 47]
therefore the squares on BX, XA are equal to the squares on
KZ. ZA,

and of these the square on KZ is greater than the square
on BX

therefore the remainder, the square on ZA, is less than the

square on XA.

Therefore AX is greater than AZ ;
therefore AX is much greater than AG,

And AX is the perpendicular on one base of the poly-
hedron,

and AG on the surface of the lesser sphere ;

hence the polyhedron will not touch the lesser sphere on its
surface.

Therefore, given two spheres about the same centre, a
polyhedral solid has been inscribed in the greater sphere
which does not touch the lesser sphere at its surface.

Q.E.F.
\end{proof}

\begin{porism*}

But if in another sphere also a polyhedral solid
be inscribed similar to the solid in the sphere BCDE,

the polyhedral solid in the sphere BCDE has to the poly-
hedral solid in the other sphere the ratio triplicate of that
which the diameter of the sphere BCDE has to the diameter
of the other sphere.

For, the solids being divided into their pyramids similar
in multitude and arrangement, the pyramids will be similar.

But similar pyramids are to one another in the triplicate
ratio of their corresponding sides ; [xn. 8, Pot.]

therefore the pyramid of which the quadrilateral KBPS is
the base, and the point A the vertex, has to the similarly
arranged pyramid in the other sphere the ratio triplicate of
that which the corresponding side has to the corresponding
side, that is, of that which the radius AB of the sphere about
A as centre has to the radius of the other sphere.

Similarly also each pyramid of those in the sphere about
A as centre has to each similarly arranged pyramid of those
in the other sphere the ratio triplicate of that which AB has
to the radius of the other sphere.

And, as one of the antecedents is to one of the conse-
quents, so are all the antecedents to all the consequents ;
hence the whole polyhedral solid in the sphere about A as
centre has to the whole polyhedral solid in the other sphere
the ratio triplicate of that which AB has to the radius of the
other sphere, that is, of that which the diameter BD has to
the diameter of the other sphere.

Q.E.D.

\end{porism*}

\begin{notes}

This proposition is of great length and therefore requires summarising in
order to make it easier to grasp. Moreover there are some assumptions in it
which require to be proved, and some omissions to be supplied. The figure
also is one of some complexity, and, in addition, the text and the figure treat
two points Z and V, which are really one and the same, as different.

The first thing needed is to know that all sections of a sphere by planes
through the centre are circles and equal to one another (great circles or
`` greatest circles `` as Euclid calls them, more appropriately). Euclid uses his
definition of a sphere as the figure described by a semicircle revolving about
its diameter. This of course establishes that alt planes through the particular
diameter make equal circular sections ; but it is also assumed that the same
sphere is generated by any other semicircle of the same size and with its
centre at the same point.

The construction and argument of the proposition may be shortly given
as follows.

A plane through the centre of two concentric spheres cuts them in great
circles of which BE, GEzie quadrants.

A regular polygon with an even number of sides is inscribed (exactly as in
Prop. 16) to the outer circle such that its sides do not touch the inner circle.
BK, KL, LM, AfE are the sides in the quadrant BE.

AO is drawn at right angles to the plane ABE, and through AO are
drawn planes passing through B, K, L, M, E, etc., cutting the sphere in great
circles.

OB, OK are quadrants of two of these great circles.

As these quadrants are equal to the quadrant BE, they wilt be divisible
into arcs equal in number and magnitude to the arcs UK, KL, LM, ME.

Dividing the other quadrants of these circles, and also all the quadrants of
the other circles through CM, in this way we shall have in all the circles a
polygon equal to that in the circle of which BE is a quadrant

BP, PQ, QR, RO and KS, ST, TU, UO are the sides of these polygons
in the quadrants BO, KO.

Joining PS, QT, RU, and making the same construction all round the
circles through AO, we have a certain polyhedron inscribed in the outer
sphere.

Draw PV perpendicular to AB and therefore (since the planes OAB,
BAE are at right angles) perpendicular to the plane BAE; [xr. Def. 4 J

draw SW perpendicular to AK and therefore (for a like reason) perpendicular
to the plane BAE.

Draw KZ perpendicular to BA. (Since BK =BP, and DB.BV=BP t ,
DB.BZ=BJP, it follows that BV=BZ, and Z, V coincide.)

Now, since l. s PA V, SA W, being angles subtended at the centre by
equal arcs of equal circles, are equal,
and since L s PVA, SWA are right,

while AS=AP t

A s PA F, SAW are equal in all respects, [1. 26]

and A V= A W.

Consequently AB: AV = AK-.AW;

and VW   BKzie parallel.

But PV, SW are parallel (being both perpendicular to one plane) and
equal (by the equal As PA V, SA W),
therefore VW   PS are equal and parallel.

Therefore BK (being parallel to VW) is parallel to PS.

Consequently (1) BPSK is a quadrilateral in one plane.

Similarly the other quadrilaterals PQTS, QRl/T a.se in one plant ; and
the triangle OR U is in one plane.

In order now to prove that the plane BPSK does not anywhere touch the
inner sphere we have to prove that the shortest distance from A to the plane
is greater than AZ, which by the construction in XU. 16 is greater than AG,

Draw AX perpendicular to the plane BPSK.

Then AX* + XB* = A X* + XK' = AX'+XS , = AX t + XP* = AB 1 ,
whence XB = XK= XS = XP,

or (2) the quadrilateral BPSK is inscri bable in a circle with X as centre and
radius XB.

Now BK> VW

>PS;
therefore in the quadrilateral BPSK three sides BK, BP, KS are equal, but
PS is less.

Consequently the angles about X are three equal angles and one smaller
angle ;
therefore any one of the equal angles is greater than a right angle, i.e. u BXK
is obtuse.

Therefore (3) BK*>iBX*. [11. u]

Next, consider the semicircle BKD with KZ drawn perpendicular to BD.

We have BD < 2 DZ,

so that D B . BZ < 2DZ ,-ZB,

or BK - < iXZ 1 ;

therefore, \emph{a fortiori}, [by (3) above]

(4) BX* < KZK

Now AJO = AB t ;

therefore AZ* + ZK* = AX   + XB>.

And BX*<KZ*i

therefore AXAZ 1 ,

or (s) AXAZ.

But, by the construction in xii. 16, AZ>AG; therefore, \emph{a fortiori},
AX>AG.

And, since the perpendicular AX is the shortest distance from A to the
plane BPSK,
(6) the plane BPSK does not anywhere meet the inner sphere.

Euclid omits to prove that, \emph{a fortiori}, the other quadrilaterals PQTS,
QRUT, and the triangle ROU, do not anywhere meet the inner sphere.

For this purpose it is only necessary to show that the radii of the circles
circumscribing BPSK, PQTS, QR UT and ROU we in descending order of
magnitude.

We have therefore to prove that, if A BCD, A' BCD' are two quadrilaterals
inscribable in circles, and

AD = BC=A'jy = B'C,
while AB is not greater than AD, A'B = CD, and AB> CD> CV,
then the radius OA of the circle circumscribing the first quadrilateral is greater
than the radius OA' of the circle circumscribing the second,

Clavius, and Simson after him, prove this by \emph{reductio ad absurdum}.

(1) HOAA',
it follows that M 40D, BOC, A' OH, B'ffC are all equal.
Also t-AOBi.A'OB',

LCODucan',

whence the four angles about O are together greater than the four angles
about 0', i.e. greater than four right angles ;
which is impossible.

(z) If a A' > OA,
cut off from 0A\ US, OC\ O'D lengths equal to OA, and draw the inner
quadrilateral as shown in the figure (XYZW).

Then AB>A , ff>XY,

CD>C'D>ZW,
AD = A'JDr* WX,
BC=SC> YZ

Consequently the same absurdity as in (1) follows \emph{a fortiori}.

Therefore, since OA is neither equal to nor less than OA',
OA > OA'.

The fact is also sufficiently clear if we draw MO, NO bisecting DA, DC
perpendicularly and therefore meeting in O, the centre of the circumscribed
circle, and then suppose the side DA with the perpendicular MO to turn
inwards about D as centre. Then the intersection of MO and NO, as P, will
gradually move towards N.

Simson gives his proof as `` Lemma n.'' immediately before xit. 1 7.
He adds to the Porism some words explaining how we may construct a
similar polyhedron in another sphere and how we may prove that the
polyhedra are similar.

The Porism is of course of the essence of the matter because it is the
porism which as much as the construction is wanted in the next proposition.
It would therefore not have been amiss to include the Porism in the enuncia-
tion of Xii. 1 7 so as to call attention to it

\end{notes}

\end{proposition}

\begin{proposition}
\label{prop:XII_18}

\begin{statement}
Spheres are to one another in the triplicate ratio of their
respective diameters.
\end{statement}

\begin{proof}

Let the spheres ABC, DEFbe conceived,
and let BC, EF be their diameters ;

I say that the sphere ABC has to the sphere DEF the ratio
triplicate of that which BC has to EF,

Foi, if the sphere ABC has not to the sphere DEF the
ratio triplicate of that which BC has to EF,
then the sphere ABC will have either to some less sphere
than the sphere DEF, or to a greater, the ratio triplicate of
that which BC has to EF.

First, let it have that ratio to a less sphere GHK,
let DEF be conceived about the same centre with GHK,
let there be inscribed in the greater sphere DEF a poly-
hedral solid which does not touch the lesser sphere GHK at
its surface, [xu. 17]
and let there also be inscribed in the spnere ABC a poly-
hedral solid similar to the polyhedral solid in the sphere DEF ;

therefore the polyhedral solid in ABC has to the polyhedral
solid in DEF the ratio triplicate of that which Z?Chas to EF.

[xii, 17, Por.]

But the sphere ABC also has to the sphere GHK the
ratio triplicate of that which BC has to EF;

therefore, as the sphere ABC is to the sphere GHK, so is
the polyhedral solid in the sphere ABC to the polyhedral
solid in the sphere DEF;

and, alternately, as the sphere ABC is to the polyhedron in
it, so is the sphere GHK to the polyhedral solid in the
sphere DEF. [v. 16]

But the sphere ABC is greater than the polyhedron in it ;

therefore the sphere GHK is also greater than the polyhedron
in the sphere DEF.

But it is also less,

for it is enclosed by it.

Therefore the sphere ABC has not to a less sphere than
the sphere DEF the ratio triplicate of that which the diameter
BC has to EF.

Similarly we can prove that neither has the sphere DEF
to a less sphere than the sphere ABC the ratio triplicate of
that which EF has to BC.

I say next that neither has the sphere ABC to any greater
sphere than the sphere DEF the ratio triplicate of that which
BC has to EF.

For, if possible, let it have that ratio to a greater, LMN;

therefore, inversely, the sphere LMN has to the sphere ABC
the ratio triplicate of that which the diameter EF has to the
diameter BC.

But, inasmuch as LMN is greater than DEF,

therefore, as the sphere LMN is to the sphere ABC, so is the
sphere DEF to some less sphere than the sphere ABC, as
was before proved. [xn. a, Lemma]

Therefore the sphere DEF also has to some less sphere
than the sphere ABC the ratio triplicate of that which EF
has to BC:

which was proved impossible.

Therefore the sphere ABC has not to any sphere greater
than the sphere DEF the ratio triplicate of that which BC
has to EF.

But it was proved that neither has it that ratio to a less
sphere.

Therefore the sphere ABC has to the sphere DEF the
ratio triplicate of that which BC has to EF.
\end{proof}

\begin{notes}

It is the method of this proposition which Legendre adopted for his proof
of xn. 2 (see note on that proposition).

The argument can be put very shortly. We will suppose S, S' to be the
volumes of the spheres, and d, d' to be their diameters ; and we will for brevity
express the triplicate ratio of d to d' by d* : d''.

ir rf*:* s.s 1 ,

then J*:J*mS:T,

where T is the volume of some sphere either greater or less than S'.

I. Suppose, if possible, that 7''< S.
Let T be supposed concentric with S'.

As in xn. 17, inscribe a polyhedron in 5' such that its faces do not any-
where touch T;

and inscribe in S a polyhedron similar to that in S'.

xii. i8] PROPOSITION 18 437

Then S:T=d 1 -. rf' s

  (polyhedron in S) : (polyhedron in S') j
or, alternately,

S : (polyhedron in S) = T: (polyhedron in S').
And S > (polyhedron in S) ;

therefore T> (polyhedron in 5').

But, by construction, T< (polyhedron in S') ;
which is impossible.

Therefore T-. S'.

II. Suppose, if possible, that T > S'.

Now d*;d' l = S: T

= XS
where X is the volume of some sphere less than S, [xn. 2, Lemma]

or, inversely, d'   : d* = S'' : X,

where X < S.

This is proved impossible exactly as in Part 1.

Therefore T   S.

Hence T, not being greater or less than S', is equal to it, and
d* : d* = S : S''.

\end{notes}

\end{proposition}

\part{Book XIII}

\chapter*{Historical Note}

I have already given, in the note to iv. 10, the evidence upon which the
construction of the live regular solids is attributed to the Pythagoreans. Some
of them, the cube, the tetrahedron (which is nothing but a pyramid), and the
octahedron (which is only a double pyramid with a square base), cannot but
have been known to the Egyptians. And it appears that dodecahedra have
been found, of bronze or other material, which may belong to periods earlier
than Pythagoras' time by some centuries (for references see Cantor's Geschichte
der Mathematik I,, pp, 175 — 6).

It is true that the author of the scholium No. 1 to Eucl.\ xm. says that the
Book is about ``the five so-called Platonic figures, which however do not
belong to Plato, three of the aforesaid five figures being due to the Pythagoreans,
namely the cube, the pyramid and the dodecahedron, while the octahedron
and the icosahed:on are due to Theaetetus.'' This statement (taken probably
from Geminus) may perhaps rest on the fact that Theaetetus was the first to
write at any length about the two last-mentioned solids. We are told indeed
by Suidas (s. v. Sfamrro*) that Theaetetus `` first wrote on the five solids as
they are called.'' This no doubt means that Theaetetus was the first to write
a complete and systematic treatise on all the regular solids ; it .does not
exclude the possibility that Hippasus or others had already written on the
dodecahedron. The fact that Theaetetus wrote upon the regular solids agrees
very well with the evidence which we possess of his contributions to the
theory of irrationals, the connexion between which and the investigation of
the regular solids is seen in Euclid's Book xm.

Theaetetus flourished about 380 B.C., and his work on the regular solids
was soon followed by another, that of Aristaeus, an elder contemporary of
Euclid, who also wrote an important book on Solid Loci, i.e. on conies treated
as loci. This Aristaeus (known as ``the elder'') wrote in the period about
310 B.C. We hear of his Comparison of the five regular solids from Hypsicles
(2nd cent B.C.), the writer of the short book commonly included in the editions
of the Elements as Book Xiv. Hypsicles gives in this Book some six proposi-
tions supplementing Eucl.\ xm. ; and he introduces the second of the
propositions (Heiberg's Euclid, Vol. v, p. 6) as follows :

`` The same circle circumscribes both the pentagon of the dodecahedron and the
triangle of the icosahedron when both are inscribed in the same sphere. This is
proved by Aristaeus in the book entitled Comparison f the five figures''
Hypsicles proceeds (pp. 7 sqq.) to give a proof of this theorem. Allman
pointed out (Greek Geometry from Thaies to Euclid, 1889, pp. 201 — 2) that this
proof depends on eight theorems, six of which appear in Eudid's Book xm.
(in Propositions 8, ro, 12, 15, 16 with Por., 17) ; two other propositions not
mentioned by Allman are also used, namely xm. 4 and 9. This seems, as
Allman says, to confirm the inference of Bretschneider (p. 171) that, as
Aristaeus' work was the newest and latest in which, before Euclid's time, this
subject was treated, we have in Eucl.\ xm. at least a partial recapitulation of
the contents of the treatise of Aristaeus.

After Euclid, Apollonius wrote on the comparison of the dodecahedron
and the icosahedron inscribed in one and the same sphere. This we also
learn from Hypsicles, who says in the next words following those about
Aristaeus above quoted : ``But it is proved by Apollonius in the second
edition of his Comparison of the dodecahedron with the icosahedron that, as the
surface of the dodecahedron is to the surface of the icosahedron [inscribed
in the same sphere], so is the dodecahedron itself [i.e. its volume] to the
icosahedron, because the perpendicular is the same from the centre of the
sphere to the pentagon of the dodecahedron and to the triangle of the
icosahedron.''

\part{Book XIII. Propositions}

\begin{proposition}
\label{prop:XIII_1}

\begin{statement}
If a straight line be cut in extreme and mean ratio, the
square on the greater segment added to the half of the whole
is five times the square on the half
\end{statement}

\begin{proof}

For let the straight line AB be cut in extreme and mean
ratio at the point C,
and let AC be the greater segment ;
let the straight line AD be pro-
duced in a straight line with CA,
and let AD be made half of AB;
F say that the square on CD is
five times the square on AD.

For let the squares AE, DF
be described on AB, DC,
and let the figure in DF be drawn ;
let FC be carried through to G.

Now, since AB has been cut in
extreme and mean ratio at C,
therefore the rectangle AB, BC is
equal to the square on AC.

[vi. Def. 3, vi. 17]

And CM is the rectangle AB, BC, and FH the square
on AC;

therefore CE is equal to FH.

And, since BA is double of AD,
while BA is equal to KA, and AD to AH,
therefore KA is also double of AH.

But, as KA is to AH, so is CK to CH;
therefore CK is double of CH.

But LH, HC are also double of CH.

Therefore KC is equal to LH, HC.

But CE was also proved equal to HF ;
therefore the whole square AE is equal to the gnomon MNO.

And, since BA is double of AD,
the square on BA is quadruple of the square on AD,
that is, AE is quadruple of DH.

But AE is equal to the gnomon MNO ;
therefore the gnomon MNO is also quadruple of AP;
therefore the whole DF is five times AP.

And DF is the square on DC, and AP the square on DA ;
therefore the square on CD is five times the square on DA.

Therefore etc,
\end{proof}

\begin{notes}

The first five propositions are in the nature of lemmas, which are required
for later propositions but are not in themselves of much importance.

It will be observed that, while the method of the propositions is that
of Book ii., being strictly geometrical and not algebraical, none of
the results of that Book are made use of (except indeed in the Lemma
to xin. a, which is probably not genuine). It would therefore appear
as though these propositions were taken from an earlier treatise
without being revised or rewritten in the light of Book 11. It will be
remembered that, according to Proclus (p. 67, 6), Eudoxus `` greatly
added to the number of the theorems which originated with Plato
regarding the section `` (i.e. presumably the ``go/den section ``) ;
and it is therefore probable that the five theorems are due to
Eudoxus.

That, if AB is divided at C in extreme and mean ratio, the rectangle
AB, BC is equal to the square on AC is inferred from vi. 17.

AD is made equal to half AB, and we have to prove that
(sq. on CD) = 5 (sq. on AD).

The figure shows at once that

cdchcdhl, s <-

so that CJCH+njHL = i(CjCH)
= £JAG.
Also sq. HF= (sq. on AC)

= rect AB, BC *

= CE.
By addition,

(gnomon MNO) - sq. on AB R

= 4 (sq. on AD) ;
whence, adding the sq. on AD to each, we have
(sq. on CD) = 5 (sq. on AD).
The result here, and in the next propositions,
is really seen more readily by means of the figure
of it. 11.

In this figure SR = AC + AB, by construction;

and we have therefore to prove that

(sq. on SR) = s (sq. on AH).

This is obvious, for

(sq. on SR) = (sq. on RB)

t AB AD

- sum of sqs. on AB, AR
= 5 (sq. on AR).

The mss. contain a. curious addition to xin, i — 5 in the shape of analyses
and syntheses for each proposition prefaced by the heading :

`` What is analysis and what is synthesis.

`` Analysis is the assumption of that which is sought as if it were admitted
-; and the arrival > by means of its consequences at something admitted to
be true.

`` Synthesis is an assumption of that which is admitted < and the arrival >
by means of its consequences at something admitted to be true.''

There must apparently be some corruption in the text ; it does not, in the
case of synthesis, give what is wanted. B and V have, instead of `` something
admitted to be true,'' the words `` the end or attainment of what is sought.''

The whole of this addition is evidently interpolated. To begin with, the
analyses and syntheses of the five propositions are placed all together in four
mss. ; in P, q they come after an alternative proof of xui. 5 (which alternative
proof P gives after xiit. 6, while q gives it instead of xm, 6), in B (which has
not the alternative proof of xni. 5) after xm. 6, and in b (in which xm. 6 is
wanting, and the alternative proof of xm. 5 is in the margin, in the first hand)
after xjii. 5, while V has the analyses of 1 — 3 in the text after xm. 6 and
those of 4 — 5 in the same place in the margin, by the second hand-. Further,
the addition is altogether alien from the plan and manner of the Elements.
The interpolation took place before The oil's time, and the probability is that
it was originally in the margin, whence it crept into the text of P after xm. 5.
Heiberg (after Bretschn eider) suggested in his edition (VoL v. p. Ixxxiv.) that
it might be a relic of analytical investigations by Theaetetus or Eudoxus, and
he cited the remark of Pappus (v. p. 410) at the beginning of his
``comparisons of the five [regular solid] figures which have an equal surface,''
to the effect that he will not use `` the so-called analytical investigation by
means of which some of the ancients effected their demonstrations.'' More
recently (Ptxralipomena zu Rukiid in Hermes xxxvm., 1903) Heiberg con-
jectures that the author is Heron, on the ground that the sort of analysis and
synthesis recalls Heron's remarks on analysis and synthesis in his commentary
on the beginning of Book it. (quoted by an-Nairīzī, ed. Curtze, p. 89) and his
quasi-algebraical alternative proofs of propositions in that Book.

To show the character of the interpolated matter I need only give the
and synthesis of one proposition. In the case of xni. 1 it is in
substance as follows. The figure is a mere
straight line. DA C 1

Let A B be divided in extreme and mean 1 1 '

ratio at C, AC being the greater segment ;
and let AD=AB.

I say that (sq. on CD) = 5 (sq. on AD).

(Analysis.)

`` For, since (sq. on CD) = 5 (sq. on AD)''

and (sq. on CD) = (sq. on CA) + (sq. on AD) + 2 (rect. CA, AD),
therefore (sq. on CA) + a (rect. CA, AD) = 4 (sq. on AD).

But rect. BA . A C = z (rect. CA . AD),

and (sq. on CA) = (rect. AB, £C),

Therefore

(rect BA, AC) + (rect. AB, BC) = 4 (sq. on AD),
or (sq. on AB) = 4 (sq. on AD) :

and this is true, since AD = AB.

(Synthesis.)

Since (sq. on AB) = 4(sq. on AD),

and (sq. on AB) = (rect. BA, ,4(7) + (rect. AB, BC),

therefore 4(sq. on AD) = 2 (rect. DA, AC) + sq. on AC.

Adding to each the square on AD, we have

(sq. on CD) = 5 (sq. on AD).

\end{notes}

\end{proposition}

\begin{proposition}
\label{prop:XIII_2}

\begin{statement}
If the square on a straight line be five times the square on
a segment of it, then, when the double of the said segment is cut
in extreme and mean ratio, the greater segment is the remaining
part of the original straight line.
\end{statement}

\begin{proof}

For let the square on the straight line AB be five times
the square on the segment AC
of it,

and let CD be double of AC ;
I say that, when CD is cut in extreme
and mean ratio, the greater segment
is CB.

Let the squares AF, CG be de-
scribed on AB, CD respectively,
let the figure in AFbe drawn,
and let BE be drawn through.

Now, since the square on BA is
five times the square on AC,
AF is five times AH.

Therefore the gnomon MNO is
quadruple of AH.

And, since DC is double of CA,
therefore the square on DC is quadruple of the square on CA,
that is, CG is quadruple of AH.

But the gnomon MNO was also proved quadruple of AH;
therefore the gnomon MNO is equal to CG,

And, since DC is double of CA,
while DC is equal to CK, and AC to CH,
therefore KB is also double of BH. [vi. 1].

But LH, HB are also double of HB ;
therefore KB is equal to LH, HB.

But the whole gnomon MNO was also proved equal to
the whole CG ;
therefore the remainder HF is equal to BG.

And BG is the rectangle CD, DB,
for CD is equal to DG ;
and HF is the square on CB
therefore the rectangle CD, DB is equal to the square on CB.

Therefore, as DC is to CB, so is CB to BD.

But DC is greater than CB ;
therefore CB is also greater than BD.

Therefore, when the straight line CD is cut in extreme and
mean ratio, CB is the greater segment.

Therefore etc.
\end{proof}

\begin{lemma*}

That the double of A C is greater than BC is to be proved
thus.

If not, let BC be, if possible, double of CA.

Therefore the square on BC is quadruple of the square
on CA ;

therefore the squares on BC, CA are five times the square
on CA.

But, by hypothesis, the square on BA is also five times
the square on CA ;

therefore the square on BA is equal to the squares on BC, CA :
which is impossible. [n. 4]

Therefore CB is not double of AC.

Similarly we can prove that neither is a straight line less
than CB double of CA ;
for the absurdity is much greater.

Therefore the double of A C is greater than CB.

Q.E.D.

\end{lemma*}

\begin{notes}

This proposition is the converse of Prop. 1. We have to prove that, if
AB be so divided at C that

(sq. on AS) = 5 (sq. on AC),
and if CD = a AC,
then (rect. CD, DB) = (sq. on CB).

Subtract from each side the sq. on AC;
then (gnomon MNO) = 4(sq. on AC)

= (sq. on CD).
Now, as in the last proposition,

CJCE = 2(E) BH)

= a BH '+ O HL.

Subtracting these equals from the equals, the square on CD and the
gnomon MNO respectively, we have

a BG= (square HF),

i.e. (rect. CD, DB) = (sq. on CB).

Here again the proposition can readily be proved by means of a figure
similar to that of It. u.

Draw CA through C at right angles to CB and of length equal to CA in
the original figure ; make CD double of CA ;

produce AC to R so that CR= CB.

Complete the squares on CB and CD, and
join AD.

Now we are given the fact that

(sq. on AS) = 5 (sq. on CA).
But
5 (sq. on AC) = (sq. on AC)+ (sq. on CD)
- (sq. on AD).
Therefore

(sq, on AR) = (sq. on AD),
or AR = AD.

Now

(rect KR, RC) + (sq. onAC) = (sq. on AR)

- (sq, on AD)

= (sq. on AC) + (sq. on CD).
Therefore (rect. KR . RC) = (sq. on CD).

That is, (rectangle RE) = (square CG).

Subtract the common part CE,
and (rect. BG) = (sq. RB),

or rect, CD, DB = (sq. on CB).

Heiberg, with reason, doubts the genuineness of the Lemma following this
proposition.

\end{notes}

\end{proposition}

\begin{proposition}
\label{prop:XIII_3}

\begin{statement}
If a straight line be cut in extreme and mean ratio, the
square on the lesser segment added to the half of the greater
segment is five times the square on the half of the greater
segment.
\end{statement}

\begin{proof}

For let any straight line AB be cut in extreme and mean
ratio at the point C,
let AC be the greater segment,
and let AC be bisected at D ;
I say that the square on BD is
five times the square on DC.

For let the square AE be
described on AB,
and let the figure be drawn
double.

Since AC is double of DC,
therefore the square on AC is
quadruple of the square on DC,
that is, RS is quadruple of FG.

And, since the rectangle AB, BC is equal to the square
on AC,

and CE is the rectangle AB, BC,
therefore CE is equal to RS.

But RS is quadruple of FG ;
therefore CE is also quadruple of FG.

Again, since AD is equal to DC,
HK is also equal to KF.

Hence the square GF is also equal to the square HL,

Therefore GK is equal to KL, that is, MN to NE ;
hence MF is also equal to FE.

But MF is equal to CG ;
therefore CG is also equal to FE.

Let CN be added to each ;
therefore the gnomon OPQ is equal to CE.

But CE was proved quadruple of GF;
therefore the gnomon OPQ is also quadruple of the square FG.

Therefore the gnomon OPQ and the square FG are
five times FG.

But the gnomon OPQ and the square FG are the
square DN,

And DN is the square on DB, and 67*'' the square on DC.

Therefore the square on DB is five times the square
on DC.
\end{proof}

\begin{notes}

In this case we have

(sq. on SD) = (sq. FG) + (rect. CG) + (rect. CN)
= (sq. FG) + (rect. J) + (rect. CN)
a (sq. G) + (rect. C£)
= (sq. FG) 4 (rect. , BC)
= (sq. iiff) + (sq. on AC), by hypothesis,
= S (sq. oni>C).
The theorem is still more obvious if the figure
of ii. 1 1 be used. Let CF be divided in extreme
and mean ratio at E, by the method of n. n.
Then, since

(rect. AB, BC) + (sq. on CD)
-- sq. on JiD
= sqs. on CD, CF,
(rect AB, BC) = (sq. on CF)
= (sq. on CA),
and AB is divided at C in extreme and mean ratio.
And (sq. on BD) = (sq. on DF)
- s (sq. on CD).

c

E F

/

\end{notes}

\end{proposition}

\begin{proposition}
\label{prop:XIII_4}

\begin{statement}
If'' a straight line be cut in extreme and mean ratio, the
square on the whole and the square on the lesser segment together
are triple of the square on the greater segment.
\end{statement}

\begin{proof}

Let AB be a straight line,
let it be cut in extreme and mean ratio at C,
and let AC be the greater segment ;
I say that the squares on AB, BC are
triple of the square on CA.

For let the square ADEB be de-
scribed on AB,
and let the figure be drawn.

Since then AB has been cut in extreme
and mean ratio at C,
and A C is the greater segment,

therefore the rectangle AB, BC is equal to the square on A C,

[vi. Def. 3, vi. 17]
And AK is the rectangle AB, BC, and HG the square
on AC-,
therefore AK is equal to HG.

And, since AF'is equal to FE,
let CK be added to each ;

therefore the whole AK is equal to the whole CE ;
therefore AK, CE are double of AK.

But AK, CE are the gnomon LMN and the square CK;
therefore the gnomon LMN and the square CK are double
of AK.

But, further, AK was also proved equal to HG ;
therefore the gnomon LMN and the squares CK, HG are
triple of the square HG.

And the gnomon LMN and the squares CK, HG are
the whole square AE and CK, which are the squares on
AB, BC,
while HG is the square on AC.

Therefore the squares on AB, BC are triple of the square
on AC.
\end{proof}

\begin{notes}
Here, as in the preceding propositions, the results are proved tie novo by
the method of Book n., without reference to that Book, Otherwise the proof
might have been shorter.
For, by It. 7,

(sq. on AB) + (sq. on BC) = 1 (rect, AB, BC) + (sq. on AC)
= 3 (sq. on A C),

\end{notes}

\end{proposition}

\begin{proposition}
\label{prop:XIII_5}

\begin{statement}
Lf a straight line be cut in extreme and mean ratio, and
there be added to it a straight line equal to the greater segment,
the whole straight line has been cut in extreme and mean ratio,
and the original straight line is the greater segment.
\end{statement}

\begin{proof}

For let the straight line AB be cut in extreme and mean
ratio at the point C,
let AC be the greater segment,
and let AD be equal to AC.

I say that the straight line
DB has been cut in extreme and
mean ratio zxA, and the original
straight line AB is the greater
segment.

For let the square AE be described on AB,
and let the figure be drawn.

Since AB has been cut in extreme and mean ratio at C,

therefore the rectangle AB, BC is equal to the square on AC.

[vi. Def. 3, vi. 17]

And CE is the rectangle AB, BC. and CH the square
on AC;
therefore CE is equal to HC.

But HE is equal to CE,
and DH is equal to HC ;
therefore DH is also equal to HE,

Therefore the whole DK is equal to the whole AE.

And DK is the rectangle BD, DA,
for AD is equal to DL ;
and AE is the square on AB ;

therefore the rectangle BD, DA is equal to the square
on AB.

Therefore, as DB is to BA, so is BA to AD. fvi. 17]

And DB is greater than BA ;
therefore BA is also greater than AD. [v. 14]

Therefore DB has been cut in extreme and mean ratio at
A, and AB is the greater segment,
\end{proof}

\begin{notes}

We have (sq. DH) - (sq. HC)

= (rect. CE), by hypothesis,
= (rect. HE).
Add to each side the rectangle A K, and

(rect. DK) = (sq. AE),
or (rect. AO, Z>) = (sq. on AS).

The result is of course obvious from 11. 11.

There is an alternative proof given in P after xm. 6, which depends on
Book v.

By hypothesis, BA : AC= AC : CS,

or, inversely, AC : AB = CB -, AC.

Compo»enHo, (AB + AC):AB = AB : AC,

or DB:BA = BA . AD.

\end{notes}

\end{proposition}

\begin{proposition}
\label{prop:XIII_6}

\begin{statement}
If a rational straight line be cut in extreme and mean ratio,
each of the segments is the irrational straight line called
apotome.
\end{statement}

\begin{proof}

Let AB be a rational straight line,
let it be cut in extreme and mean

ratio at C, o   ?

and let AC be the greater segment ;

I say that each of the straight lines AC, CB is the irrational

straight line called apotome.

For let BA be produced, and let AD be made half of BA.

Since then the straight line AB has been cut in extreme
and mean ratio,

and to the greater segment AC is added AD which is half
of AB,
therefore the square on CD is five times the square on DA.

[xm. l]

Therefore the square on CD has to the square on DA the
ratio which a number has to a number ;

therefore the square on CD is commensurable with the square
on DA. [x. 6]

But the square on DA is rational,

for DA is rational, being half of AB which is rational ;

therefore the square on CD is also rational ; [x. Def. 4]

therefore CD is also rational.

And, since the square on CD has not to the square on
DA the ratio which a square number has to a square number,
therefore CD is incommensurable in length with DA ; [x. 9)
therefore CD, DA are rational straight lines commensurable
in square only ;
therefore A C is an apotome. [x. 73]

Again, since AB has been cut in extreme and mean ratio,

and AC is the greater segment,

therefore the rectangle AB, BC is equal to the square on AC

[vi. Def. 3, vi. 17]

Therefore the square on the apotome AC, if applied to
the rational straight line AB, produces BC as breadth.

But the square on an apotome, if applied to a rational
straight line, produces as breadth a first apotome ; [x. 97)

therefore CB is a first apotome.

xiii. 6, 7] PROPOSITIONS 6, 7 451

And CA was also proved to be an apotome.
Therefore etc.
\end{proof}

\begin{notes}

It seems certain that this proposition is an interpolation. P has it, but the
copyist (or rather the copyist of its archetype) says that `` this theorem is not
found in most copies of the new recension, but is found in those of the old.''
In the first place, there is a scholium to xitt. 17 in P itself which proves the
same thing as XIII. 6, and which would therefore have been useless if xiii. 6
had preceded. Hence, when the scholium was written, this proposition had
not yet been interpolated. Secondly, P has it before the alternative proof of
xiii. 5 ; this proof is considered, on general grounds, to be interpolated, and
it would appear that it must have been a later interpolation (xiti. 6) which
divorced it from the proposition to which it belonged. Thirdly, there is cause
for suspicion in the proposition itself, for, while the enunciation states that
each segment of the straight line is an apotome, the proposition adds that the
lesser segment is a first apotome. The scholium in P referred to has not this
blot. What is actually wanted in xiii. 1 7 is the fact that the greater segment
is an apotome. It is probable that Euclid assumed this fact as evident enough
from xn 1. 1 without further proof, and that he neither wrote XIII. 6 nor the
quotation of its enunciation in xiii. 1 7.

\end{notes}

\end{proposition}

\begin{proposition}
\label{prop:XIII_7}

\begin{statement}
If three angles of an equilateral pentagon, taken either in
order or not in order, be equal, the pentagon will be equiangular.
\end{statement}

\begin{proof}

For in the equilateral pentagon ABCDE let, first, three
angles taken in order, those sxA,B, C,
be equal to one another ;
I say that the pentagon ABCDE is
equiangular.

For let AC, BE, FD be joined.

Now, since the two sides CB, BA
are equal to the two sides BA, AE
respectively,

and the angle CBA is equal to the
angle BAE,
therefore the base AC is equal to the base BE,
the triangle ABC is equal to the triangle ABE,
and the remaining angles will be equal to the remaining angles,
namely those which the equal sides subtend, [1. 4]

that is, the angle BCA to the angle BEA, and the angle
ABE to the angle CAB;
hence the side AF is also equal to the side BE. [1. 6]

But the whole ACv/as also proved equal to the whole BE;
therefore the remainder FC is also equal to the remainder FE.

But CD is also equal to DE.

Therefore the two sides FC, CD are equal to the two
sides FE, ED ;

and the base FD is common to them ;
therefore the angle FCD is equal to the angle FED. [i. 8]

But the angle BCA was also proved equal to the angle
AEB;

therefore the whole angle BCD is also equal to the whole
angle A ED.

But, by hypothesis, the angle BCD is equal to the angles
at A, B ;
therefore the angle A ED is also equal to the angles at A, B.

Similarly we can prove that the angle CDE is also equal
to the angles at A, B, C ;
therefore the pentagon ABCDE is equiangular.

Next, let the given equal angles not be angles taken in
order, but let the angles at the points A, C, D be equal ;
I say that in this case too the pentagon ABCDE is equiangular.

For let BD be joined.

Then, since the two sides BA, AE are equal to the two
sides BC, CD,

and they contain equal angles,
therefore the base BE is equal to the base BD,
the triangle ABE is equal to the triangle BCD,
and the remaining angles will be equal to the remaining angles,
namely those which the equal sides subtend ; [i. 4]

therefore the angle AEB is equal to the angle CDB.

But the angle BED is also equal to the angle BDE,
since the side BE is also equal to the side BD. [1. 5]

'Therefore the whole angle AED is equal to the whole
angle CDE.

But the angle CDE is, by hypothesis, equal to the angles
at A, C;
therefore the angle AED is also equal to the angles at A, C.

For the same reason
the angle ABC is also equal to the angles at A, C, D.
Therefore the pentagon ABCDE is equiangular.
\end{proof}

\begin{notes}

This proposition is required in mil 17.
The steps of the proof may be shown thus.

I. Suppose that the angles at A, B, C are all equal.

Then the isosceles triangles BAE, ABC are equal in all respects ;
thus BE = AC, lBCA = lBEA, lCABlEBA.

By the last equality, FA = FB,

so that, since BE = AC, FC= FE.

The As FED, FCD are now equal in all respects, [1. 8, 4)

and £ FCD - 4 FED.

But l ACB = l. ABB, from above,

whence, by addition, l BCD - L AED.

Similarly it may be proved that U CDE is also equal to any one of the
angles at A, B, C.

II. Suppose the angles at A, C, D to be equal.

Then the isosceles triangles ABE, CBD are equal in all respects, and
hence BE  BD (so that l BDE = t BED),
and l CDB = u AEB.

By addition of the equal angles,

l CDE = l DEA.

Similarly it may be proved that l ABC is also equal to each of the angles
at A, C,D.

\end{notes}

\end{proposition}

\begin{proposition}
\label{prop:XIII_8}

\begin{statement}
If in an equilateral and equiangular pentagon straight
lines subtend two angles taken in order, they cut one another
in extreme and mean ratio, and their greater segments are equal
to the side of the pentagon.
\end{statement}

\begin{proof}

For in the equilateral and equiangular pentagon ABCDE
let the straight lines AC, BE, cutting
one another at the point H, subtend
two angles taken in order, the angles
at A, B ;

I say that each of them has been
cut in extreme and mean ratio at
the point H, and their greater seg-
ments are equal to the side of the
pentagon.

For let the circle ABCDE be
circumscribed about the pentagon ABCDE. [iv. 14]

Then, since the two straight lines EA, AB are equal to
the two AS, BC,
and they contain equal angles,
therefore the base BE is equal to the base AC,
the triangle ABE is equal to the triangle ABC,
and the remaining angles will be equal to the remaining angles
respectively, namely those which the equal sides subtend, i. a

Therefore the angle BA C is equal to the angle ABE ;
therefore the angle A HE is double of the angle BAH. [i. 3 a]

But the angle EAC is also double of the angle BAC,
inasmuch as the circumference EDC is also double of the
circumference CB ; [in. 38, vi. ]

therefore the angle HAE is equal to the angle A HE ;
hence the straight line HE is also equal to EA, that is, to AB.

And, since the straight line BA is equal to AE,
the angle ABE is also equal to the angle AEB. [1. 5]

But the angle ABE was proved equal to the angle BAH;
therefore the angle BE A is also equal to the angle BAH,

And the angle ABE is common to the two triangles ABE
and ABH;

therefore the remaining angle BAE is equal to the remaining
angle A HB; [1.33]

therefore the triangle ABE is equiangular with the triangle
ABH;

therefore, proportionally, as EB is to BA, so is AB to BH.

[vi. 4]
But BA is equal to EH ;

therefore, as BE is to EH, so is EH to HB.

And BE is greater than EH;
therefore EH is also greater than HB. [v. 14)

Therefore BE has been cut in extreme and mean ratio at
H and the greater segment HE is equal to the side of the
pentagon.

Similarly we can prove that AC has also been cut in
extreme and mean ratio at H, and its greater segment CH
is equal to the side of the pentagon.
\end{proof}

\begin{notes}

In order to prove this theorem we have to show (1) that the As AEB,
HAB are similar, and (2) that £H= EA (= AB).
To prove (2) we have

As AEB, B AC equa\ in all respects,
whence EB=AC,

and lBAClABE.

Therefore i. A HE = ilBAC

= lEAC,
so that EH= EA

= AB.
To prove (1) we have, in the As AEB, HAB,
lBAH=l.EBA
= l AEB,
and l ABE is common ,

therefore the third  s AHB, EAB are equal,
and A s AEB, HAB are simitar.

Now, since these triangles are similar,

EB;BA = BA: BH,
or (rect. EB, BH) = (sq. on BA)

= (sq. on EH),
so that EB is divided in extreme and mean ratio at H

Similarly its equal, CA, is divided in extreme and mean ratio at H,

\end{notes}

\end{proposition}

\begin{proposition}
\label{prop:XIII_9}

\begin{statement}
If ike side of the hexagon and that of the decagon inscribed
in the same circle be added together \ the whole straight line
has been cut in extreme and mean ratio, and its greater segment
is the side of the hexagon.
\end{statement}

\begin{proof}

Let ABC be a circle ;

of the figures inscribed in the circle ABC let BC be the side
of a decagon, CD that of a hexagon,

and let them be in a straight line ;

I say that the whole straight line
BD has been cut in extreme and
mean ratio, and CD is its greater
segment.

For let the centre of the circle,
the point E, be taken,

let EB, EC, ED be joined,

and let BE be carried through to A. o

Since BC is the side of an equilateral decagon,

456 BOOK XIII [xiil 9

therefore the circumference ACB is five times the circum-
ference BC;
therefore the circumference A C is quadruple of CB.

But, as the circumference AC is to CB, so is the angle
ABC to the angle CEB ; [vi. 33]

therefore the angle A EC is quadruple of the angle CEB.

And, since the angle EEC is equal to the angle ECB, [1. 5]
therefore the angle A EC is double of the angle ECB. [1. 32]

And, since the straight line EC is equal to CD,
for each of them is equal to the side of the hexagon inscribed
in the circle ABC, [iv. 15, Por.]

the angle CED is also equal to the angle CDE ; [1. 5]

therefore the angle ECB is double of the angle EDC. [1. 31]

But the angle A EC was proved double of the angle ECB;
therefore the angle A EC is quadruple of the angle EDC.

But the angle AEC was also proved quadruple of the
angle BEC;
therefore the angle EDC is equal to the angle BEC.

But the angle EBD is common to the two triangles BEC
and BED ;

therefore the remaining angle BED is also equal to the
remaining angle ECB ; [1. 3*]

therefore the triangle EBD is equiangular with the triangle
EEC.

Therefore, proportionally, as DB is to BE, so is EB to BC.

[vi. 4]
But EB is equal to CD.
Therefore, as BD is to DC, so is DC to CB.
And BD is greater than DC ;

therefore DC is also greater than CB,

Therefore the straight line BD has been cut in extreme
and mean ratio, and DC is its greater segment.
\end{proof}

\begin{notes}

BC is the side of a regular decagon inscribed in the circle ; CD is the
side of the inscribed regular hexagon, and is therefore equal to the radius BE
or EC.

Therefore, in order to prove our theorem, we have only to show that tht
triangles EBC, DBE arc similar.

Since BC is the side of a regular decagon,

(arc BCA) = 5 (arc BC),
so that (arc CEA) m 4 (arc BC),

whence l CEA = 4 l BEC.

But i.CEA = n.ECB.

Theretore i.ECB = 2i.BEC fi).

But, since CD = CE,

L CDE = l CED,
so that lECB=2lCDE.

It follows from (1) that lBEC=lCDE.
Now, in the As EBC, DBE,

lBEC=l.BDE,
and  EBC is common,

so that i. £C = i. /?£,,

and As EBC, DBE are similar.

Hence DB : BE = EB ; BC,

or (rect. X>5, JC) = (sq. on ££)

= (sq. on CD),
and Z)2? is divided at C in extreme and mean ratio.

To find the side of the decagon algebraically in terms of the radius we
have, if x be the side required,

v + *)* = *I

m

whence x = -(J — 1).

\end{notes}

\end{proposition}

\begin{proposition}
\label{prop:XIII_10}

\begin{statement}
If an equilateral pentagon be inscribed in a circle, the
square on the side of the pentagon is equal to ike squares on
the side of the hexagon and on that of the decagon inscribed in
ike same circle.
\end{statement}

\begin{proof}

Let ASCDE be a circle,
and let the equilateral pentagon ABCDE be inscribed in the
circle ABCDE.

I say that the square on the side of the pentagon ABCDE
is equal to the squares on the side of the hexagon and on
that of the decagon inscribed in the circle ABCDE.

For let the centre of the circle, the point F, be taken,
let AF be joined and carried through to the point G,
let FB be joined,

let FH be drawn from F perpendicular to AB and be carried
through to K,

let AK, KB be joined,

let FL be again drawn from /*'' perpendicular to AK, and be
carried through to M,
and let KN be joined.

Since the circumference
ABCG is equal to the circum-
ference AEDG,
and in them ABC is equal to
AED,

therefore the remainder, the
circumference CG, is equal to
the remainder GD.

But CD belongs to a pen-
tagon ;

therefore CG belongs to a
decagon.

And, since FA is equal to FB,
and FH is perpendicular,
therefore the angle AFK is also equal to the angle KFB,

[i. S->-*«]

Hence the circumference AK is also equal to KB \ [in. 26]

therefore the circumference AB is double of the circumference

BK,

therefore the straight line AK is a side of a decagon.

For the same reason
AK is also double of KM,

Now, since the circumference AB is double of the circum-
ference BK,
while the circumference CD is equal to the circumference AB,

therefore the circumference CD is also double of the circum-
ference BK.

But the circumference CD is also double of CG ;

therefore the circumference CG is equal to the circumference
BK.

But BK is double of KM, since KA is so also ;
therefore CG is also double of KM,

But, further, the circumference CB is also double of the
circumference BK,

for the circumference CB is equal to BA.

Therefore the whole circumference GB is also double
of BM;

hence the angle GFB is also double of the angle BFM. [vi. 33]
But the angle GFB is also double of the angle FAB,

for the angle FAB is equal to the angle ABF.

Therefore the angle BFN is also equal to the angle FAB.
But the angle ABF is common to the two triangles ABF

and BFN

therefore the remaining angle AFB is equal to the remaining
angle BNF; [' 3*]

therefore the triangle ABF is equiangular with the triangle
BFN.

Therefore, proportionally, as the straight line AB is to BF,
so is FB to BN ; [vi. 4]

therefore the rectangle AB, BN is equal to the square on BF.

[vi. 17]

Again, since AL is equal to LK,
while LN is common and at right angles,
therefore the base KN is equal to the base A N ; [1. 4]

therefore the angle LKN is also equal to the angle LAN.

But the angle LAN is equal to the angle KBN ;
therefore the angle LKN is also equal to the angle KBN.

And the angle at A is common to the two triangles A KB
znAAKN.

Therefore the remaining angle AKB is equal to the
remaining angle KNA ; [1. 32]

therefore the triangle KB A is equiangular with the triangle
KNA.

Therefore, proportionally, as the straight line BA is to
AK, so is KA to AN; [vi. 4J

therefore the rectangle BA, AN is equal to the square on AK.

[vi. 17]
But the rectangle AB, BN was also proved equal to the
square on BF;
therefore the rectangle AB, BN together with the rectangle
BA, AN, that is, the square on BA [u. a\ is equal to the
square on BF together with the square on AK.

And BA is a side of the pentagon, BF of the hexagon
[iv. is, Por.], and AK of the decagon.

Therefore etc.
\end{proof}

\begin{notes}

ABCDE being a regular pentagon inscribed in a circle, and AG the
diameter through A, tt follows that

(arc CG) = (arc GD),
and CG t GD are sides of an inscribed regular decagon.

FHK being drawn perpendicular to AB, it follows, by I. 26, that
L s AFK, BFK are equal, and BK, KA are sides of the regular decagon.
Similarly it may he proved that, FLM being perpendicular to AK,

AL - LK,
and (arc AM) = (arc MK).

The main facts to prove are that
(1) the triangles ABF, FBN are similar, and (3) the triangles A BK, AKN
are simitar.

(1) 2 (arc Ct?) = (arc CD)

= (arc AB)
= 2 (arc BK),
or (arc CG) = (arc BK) = (arc AK)

= 1 (arc KM).
And (arc CB) - 2 (arc BK).

Therefore, by addition,

(arc BCG) = 1 (arc BKM).
Therefore u BFG -it- BFN.

But lBFG=ilFAB,

so that L FAB = 4 BFN.

Hence, in the ABF, FBN,

l FAB = l BFN,
and l ABF is common ;

therefore l AFB = l BNF,

and As ABF, FBNvte. similar.

(2) Since AL- LK, and the angles at L are right,

AN= NK,
and LNKA = i.NAK

m L KBA.

Hence, in the As ABK, AKN,

±A£K=lAKN,

and l. KAN is common,

whence the third angles are equal ;

therefore the triangles ABK, AKN are similar.

Now from the similarity of A s A BF, FBN it follows that
AB :BF=BF:BN,

or (rect. AB, BN)   (sq. on BF).

And, from the similarity of ABK, AKN,

BA : AK AK : AN,
or (rect. BA, AN)   (sq. on AK).

Therefore, by addition,

(rect, AB, BN) + (rect. BA, AN)   (sq. on BF) + (sq. on AK),
that is, (sq. on AB) = (sq. on BF) + (sq. on AK).

If r be the radius of the circle, we have seen (xiii. 9, note) that
AA'-Us-i).

Therefore (side of pentagon) 1 = r* + - (6 - 2 5)

4

= 7(*«-*VsX

4

so that (side of pentagon) = - n/10 - 2 V5

\end{notes}

\end{proposition}

\begin{proposition}
\label{prop:XIII_11}

\begin{statement}
V« a *m ml£ las f diameter rational an equilateral
pentagon be inscribed, the side of the pentagon is the irrational
straight line called minor.
\end{statement}

\begin{proof}

For in the circle ABCDE which has its diameter rational
let the equilateral pentagon ABCDE be inscribed ;
1 say that the side of the pentagon is the irrational straight
line called minor.

For let the centre of the circle, the point F, be taken,
let AF, FB be joined and carried through to the points, G, H,
let AC be joined,
and let FK be made a fourth part of AF.

Now AF is rational ;
therefore FK is also rational.

But BF is also rational ;
therefore the whole BK is rational.

And, since the circumference ACG is equal to the circum-
ference ADG,

and in them ABC is equal to A ED,
therefore the remainder CG is equal to the remainder GD.

And, if we join AD, we conclude that the angles at L
are right,
and CD is double of CL.

For the same reason
the angles at M are also right,
and AC is double of CM.

Since then the angle ALC is equal to the angle AMF,

and the angle LAC is common to the two triangles ACL

and AMF,

therefore the remaining angle ACL is equal to the remaining

angle MFA ; [i. 32]

therefore the triangle ACL Is equiangular with the triangle
AMF;

therefore, proportionally, as LC is to CA, so is MF to /v?.

And the doubles of the antecedents may be taken ;

therefore, as the double of LC is to CA, so is the double of
MF to FA.

But, as the double of MF is to FA, so is MF to the half
oiFA

therefore also, as the double of LC is to CA, so is MF to the
half of TW.

And the halves of the consequents may be taken ;

therefore, as the double of LC is to the half of CA, so is MF
to the fourth of FA.

And DC is double of LC, CM is half of CA, and FK -a.
fourth part of FA ;
therefore, as DC is to CM, so is MF to FK.

Componendo also, as the sum of DC, CM is to CM, so is
MM to KF\ [v. 18]

therefore also, as the square on the sum of DC, CM is to the
square on CM, so is the square on MK to the square on KF.

And since, when the straight line subtending two sides of
the pentagon, as AC, is cut in extreme and mean ratio, the
greater segment is equal to the side of the pentagon, that is,
to DC, [xui. 8]

while the square on the greater segment added to the half
of the whole is five times the square on the half of the
whole, [xtu. 1]

and CM is half of the whole AC,

therefore the square on DC, CM taken as one straight line is
five times the square on CM.

But it was proved that, as the square on DC, CM taken
as one straight line is to the square on CM, so is the square
on MK to the square on KF ;
therefore the square on MK is five times the square on KF.

But the square on KF is rational,
for the diameter is rational ;
therefore the square on MK is also rational ;
therefore MK is rational

And, since BF is quadruple of FK,

therefore BK is five times KF ;

therefore the square on BK is twenty-five times the square
on KF.

But the square on MK is five times the square on KF;
therefore the square on BK is five times the square on KM;
therefore the square on BK has not to the square on KM
the ratio which a square number has to a square number ;

therefore BK is incommensurable in length with KM. [x. 9]

And each of them is rational.

Therefore BK, KM are rational straight lines commen-
surable in square only.

But, if from a rational straight line there be subtracted a
rational straight line which is commensurable with the whole
in square only, the remainder is irrational, namely an apotome;
therefore MB is an apotome and MK the annex to it. [x. 73]

I say next that MB is also a fourth apotome.
Let the square on N be equal to that by which the square
on BK is greater than the square on KM ;

therefore the square on BK is greater than the square on KM
by the square on N.

And, since KF is commensurable with FB,
eomponendo also, KB is commensurable with FB. [x. 15]

But BF is commensurable with BH;

therefore BK is also commensurable with BH. [x. 12]

And, since the square on BK is five times the square
on KM,

therefore the square on BK has to the square on KM the
ratio which 5 has to 1.

Theretore, convertendo, the square on BK has to the square
on N the ratio which 5 has to 4 [v. 19, Por.], and this is not the
ratio which a square number has to a square number ;
therefore BK is incommensurable with N\ [x. 9)

therefore the square on BK is greater than the square on KM
by the square on a straight line incommensurable with BK.

Since then the square on the whole BK is greater than
the square on the annex KM by the square on a straight line
incommensurable with BK,

and the whole BK is commensurable with the rational straight

line, BH, set out,

therefore MB is a fourth apotome. [x. Deff. in. 4]

But the rectangle contained by a rational straight line and
a fourth apotome is irrational,

and its square root is irrational, and is called minor. [x. 94]

But the square on AB is equal to the rectangle HB t BM,

because, when AH is joined, the triangle ABH is equiangular
with the triangle ABM, and, as HB is to BA, so is AB
to BM.

Therefore the side AB of the pentagon is the irrational
straight line called minor,
\end{proof}

\begin{notes}

Here we require certain definitions and propositions of Book x.

First we require the definition of an apotome [see x. 73], which is a straight
line of the form (p ~ Jk . p), where p is a `` rational `` straight line and A is any
integer or numerical fraction, the square root of which is not integral or
expressible in integers. The lesser of the straight lines p, N Ik, p is the <tn/itx.

Next we require the definition of th fourth apotome [x. Deff. Ill (after
x, 84)], which is a straight line of the form (x-y), where x, y (being both
rational and commi able in square only) are also such that Jx'-j'' is
incommensurable wiu. x, while x is commensurable with a given rational
straight line p. As shown on X. 88 (note), lY. fourth apotomt is of the form

V vi + A/

Lastly the miner (straight line) is the irrational straight line defined in
x. 76. It is of the form (jc — y), where x, y are incommensurable in stjuart,
and (x'' +/*) is rational,' while xy is medial.' As shown in the note on
x. 76, the miner irrational straight line is of the form

±. /, + *- --E- A - -— -
V 3 V J7+ V* V /!*'

The proposition may be put as follows. ABCDE being a regular
pentagon inscribed in a circle, AG, BH the diameters through A, B meeting
CD in L and AC in At respectively, FK is made equal to AF.

Now, the radius AF(r) being rational, so are FK, BK.

The arcs CG, GD are equal ;
hence l s at L are right, and CD -zCL

Similarly l s at M are right, and AC-sCM.

We have to prove

(1) that BM is an apotome,

(a) that BM is a fourth apotome,

(3) that BA is a minor irrational straight line.

Remembering that, if L'A is divided in extreme and mean ratio, the
greater segment is equal to the side of the pentagon [xm. 8], and that accord
ingly [xm. 1] (CD + CA)   = 5 (faCA)'', we work towards a proportion con-
taining the ratio (CD + CM) 1 : CAP, thus.

The As ACL, AFAfte equiangular and therefore similar.

Therefore LC: CA = MF:FA,

and accordingly iZC: CA = MF: FA ;

thus  LC:CA = MF.FA,

or DC:CM=MF:FK;

whence, componendo, and squaring,

(DC + CMf : CAP - MK* : KF*.

But (DC+CAf)* = S CAP;

therefore MK? - XF*>

[This means that MK* = - r 1 ,

or JttT = 3r.]

4
It follows that,  AT*' being rational, MK*, and therefore MK, is rational.

(1) To prove that BMte an apotome and jWA* its annex.
We have BF=aFK;

therefore BX=FK,

BK* = aFK s

- MK\ from above ;
therefore BK* has not to AfJC the ratio of a square number tn a square
number ;
therefore BK, MKare incommensurable in length.

They are therefore rational and commensurable in square only ;
accordingly BMfc an apotome.

[BK 1 = MK> - ?V, and BK= r.

Consequently BK - MK= (-* - — fY ]

(2) To prove that BM is a fourth apotome.
First, since KF, FB are commensurable,

BK, BF are commensurable, i-e. BK is commensurable with BH, a given
rational straight line.

Secondly, if W = BK* - KM

since BK*: KM* =  : 1,

it follows that BK 1 : JV 1 = 5 ; 4,

whence A'', /V are incommensurable.

Therefore BMis a. fourth apotome.

(3) To prove that BA is a minor irrational straight line.

If a fourth apotome form a rectangle with a rational straight line, the side
of the square equivalent to the rectangle is minor [x. 94].

Now BA*HB.BM,

HB is rational, and BMis a fourth apotome;
therefore BA is a minor irrational straight line.

If this is separated into the difference between two straight lines, we have
BA= r - sis + 2 VS - r ~ 5 ~ * VS-]

\end{notes}

\end{proposition}

\begin{proposition}
\label{prop:XIII_12}

\begin{statement}
If an equilateral triangle be inscribed in a circle, the square
on the side of the triangle is triple of the square on the radius
of the circle.
\end{statement}

\begin{proof}

Let ABC be a circle,
and let the equilateral triangle ABC be inscribed in it ;
I say that the square on one side of
the triangle ABC is triple of the square
on the radius of the circle.

For let the centre D of the circle
ABC be taken,

let AD be joined and carried through

to E,

and let BE be joined.

Then, since the triangle ABC is
equilateral,

therefore the circumference BEC is a third part of the circum-
ference of the circle ABC

Therefore the circumference BE is a sixth part of the
circumference of the circle ;

therefore the straight line BE belongs to a hexagon ;

therefore it is equal to the radius DE. [iv. 15, Por.]

And, since AE is double of DE,

the square on AE is quadruple of the square on ED, that is,
of the square on BE.

But the square on AE is equal to the squares on AB, BE;

[111. 31, 1. 47]
therefore the squares on AB, BE are quadruple of the square
on BE.

Therefore, separando, the square on AB is triple of the
square on BE.

But BE is equal to DE ;

therefore the square on AB is triple of the square on DE.

Therefore the square on the side of the triangle is triple
of the square on the radius.
\end{proof}

\end{proposition}

\begin{proposition}
\label{prop:XIII_13}

\begin{statement}
To construct a pyramid, to comprehend it in a given sphere,
and to prove that the square on the diameter of the sphere is
one and a half times the square on the side of the pyramid.
\end{statement}

\begin{proof}

Let the diameter AB of the given sphere be set out,
and let it be cut at the point C so that AC is double of CB ;
let the semicircle ADB be described on AB,
let CD be drawn from the point C at right angles to AB,

and let DA be joined ;

let the circle EFG which has its radius equal to DC be
set out,

let the equilateral triangle EFG be inscribed in the circle EFG,

[iv. 1]

let the centre of the circle, the point H, be taken, [ui. 1]

let EH, HF, HG be joined ;

from the point H let HK be set up at right angles to the plane
of the circle EFG, [it is]

let HK equal to the straight line AC be cut off from HK,
and let KE, KF, KG be joined.

Now, since KH is at right angles to the plane of the
circle EFG,

therefore it will also make right angles with all the straight
lines which meet it and are in the plane of the circle EFG.

[xi. Def. 3]

But each of the straight lines HE, HF, HG meets it :

therefore HK is at right angles to each of the straight lines
HE, HF, HG.

And, since AC is equal to HK, and CO to HE,
and they contain right angles,
therefore the base DA is equal to the base KE. [1. 4]

xiii. i j] PROPOSITION 13 469

For the same reason

each of the straight lines KF, KG is also equal to DA ;

therefore the three straight lines KE, KF, KG are equal to
one another.

And, since AC is double of CB,

therefore AB is triple of BC.

But, as AB is to BC, so is the square on AD to the square
on DC, as will be proved afterwards.

Therefore the square on AD is triple of the square on DC.
But the square on FE is also triple of the square on EH,

[xm. 13]
and DC is equal to EH ;

therefore DA is also equal to EF.

But DA was proved equal to each of the straight lines
KE, KF, KG ;

therefore each of the straight lines EF, FG, GE is also equal

to each of the straight lines KE, KF, KG ;

therefore the four triangles EFG, KEF, KFG, KEG are

equilateral.

Therefore a pyramid has been constructed out of four
equilateral triangles, the triangle EFG being its base and the
point K its vertex.

It is next required to comprehend it in the given sphere
and to prove that the square on the diameter of the sphere
is one and a half times the square on the side of the pyramid.

For let the straight line HL be produced in a straight
line with KH,
and let HL be made equal to CB.

Now, since, as AC is to CD, so is CD to CB, [vi. 8, Por]

while AC is equal to KH, CD to HE, and CB to HL,

therefore, as KH is to HE, so is EH to HL ;

therefore the rectangle KH, HL is equal to the square on
EH. [vi. 17]

And each of the angles KHE. EHL is right ;

therefore the semicircle described on KL will pass through
E also. [cf. vi. 8, 111. 31.]

If then, KL remaining fixed, the semicircle be carried round
and restored to the same position from which it began to be
moved, it will also pass through the points F, G,

since, if FL, LG be joined, the angles at F, G similarly become
right angles )

and the pyramid will be comprehended in the given sphere.

For KL, the diameter of the sphere, rs equal to the
diameter AB of the given sphere, inasmuch as KH was
made equal to AC, and HL to CB.

I say next that the square on the diameter of the sphere
is one and a half times the square on the side of the
pyramid

For, since AC is double of CB,

therefore AB is triple of BC ;

and, converiendo, BA is one and a half times AC.

But, as BA is to AC, so is the square on BA to the square
on AD.

Therefore the square on BA is also one and a half times
the square on AD.

And BA is the diameter of the given sphere, and AD is
equal to the side of the pyramid.

Therefore the square on the diameter of the sphere is
one and a half times the square on the side of the pyramid.
\end{proof}

\begin{lemma*}

It is to be proved that, as AB is to BC, so is the square
on AD to the square on DC.

For let the figure of the semi-
circle be set out,
let DB be joined,
let the square EC be described
on AC,

and let the parallelogram FB be
completed.

Since then, because the tri-
angle DAB is equiangular with
the triangle DAC,
as BA is to AD, so is DA to AC,

[vi. 8, vi. 4]

therefore the rectangle BA, AC is equal to the square on AD.

[vi. 17]

And since, as AB is to BC, so is EB to BF, [vi. i]

and EB is the rectangle BA, AC, for EA is equal to AC,

and BF is the rectangle AC, CB,

therefore, as AB is to BC, so is the rectangle BA, AC xo the
rectangle AC, CB.

And the rectangle BA, AC is equal to the square on AD,
and the rectangle 4C, CB to the square on DC,

for the perpendicular Z?C is a mean proportional between the
segments AC, CB of the base, because the angle ADB is
right. [vi. 8, Por.]

Therefore, as AB is to BC, so is the square on AD to
the square on DC.

Q.E.D.

\end{lemma*}

\begin{notes}

The Lemma is with reason suspected. Euclid commonly takes more
difficult theorems for granted in the stereometrical Books. It is also clumsy
in itself, while, from a gloss in the proposition rejected as an interpolation, it
is clear that the interpolator of the gloss had not the Lemma. With the
Lemma should disappear the words ``as will be proved afterwards `` (p. 460),

In the figure of the proposition, the semicircle really represents half of
a section of the sphere through its centre and one edge of the inscribed
tetrahedron (AD being the length of that edge).

The proof is in three parts, the object of which is to prove

(1) that KEFG is a tetrahedron with all its edges equal to AD,

(2) that it is inscribable in a sphere of diameter equal to AB,

(3) that AB* =  AD*.
To prove (1) we have to show

(a) that KE = KF= KG = AD,

(b) that AD = EF.

(a) Since HE = HF=HG=CD,

KH=AC,
and usACD, KHE, KHF, KHG are right,

As A CD, KHE, KHF, KHG are equal in all respects ;

therefore

KE = KF= KG = AD.

(6) Since

AB=BC,

and

AB:BC=AB.AC:AC.CB

- AD* : CD

it follows that

AD' = 3 CD

But [xiii, 12]

EF i =zEH i ,

and EH= CD, by construction

47 BOOK XIII [xin, 13

Therefore ADEF.

Thus EFGK'v, a regular tetrahedron.

(2) We now observe the usefulness of Euclid's description of a sphere

[in Xt. Def. re-
producing KJf(= AC) to L so that HL = CB,

we have KL equal to AB ;

thus KL is a diameter of the sphere which should circumscribe out tetra-
hedron,

and we have only to prove that E, F, G lie on semicircles described on KL

as diameter.

E.g. for the point £,

since AC-.CDCD.CB,

while AC=KH, CDHE, CB = HL,

we have KH:HE=HE. HL,

or KH.HL=HE',

whence, the angles KHE, EHL being right,

EKL is a triangle right-angled at E [cf. vi, 8].

Hence E lies on a semicircle on KL as diameter.

Similarly for F, G.

Thus a semicircle on KL as diameter revolving round KL passes

successively through E, F, G.

(3) AB = iBC,

therefore BA = AC.

And BA:AC = BA> zBA.AC

= BA*:AD

Therefore BA* = AD

If r be the radius of the circumscribed sphere,

(edge of tetrahedron) = -H— . r = f/6 . r.

It will be observed that, although in these cases Euclid's construction is
equivalent to inscribing the particular regular solid in a given sphere, he does
not actually construct the solid in the sphere but constructs a solid which a
sphere equal to the given sphere will circumscribe. Pappus, on the other
hand, in dealing with the same problems, actually constructs the respective
solids in the given spheres. His method is to find circular sections in the
given spheres containing a certain number of the angular points of the given
solids. His solutions are interesting, although they require a knowledge of
some properties of a sphere which are of course not found in the Elements
but belonged to treatises such as the Sphaerica of Theodosius.

Pappus' solution of the problem of Eucl.\ XIII. 13.

In order to inscribe a regular pyramid or tetrahedron in a given sphere,
Pappus (hi. pp. 142 — 144) finds two circular sections equal and parallel to one
another, each of which contains one of two opposite edges as its diameter. In
this and the other similar problems he proceeds in the orthodox manner by
analysis and synthesis. The following is a reproduction of his solution of
this case.

Analysis.

Suppose the problem solved, A, B, C, D being the angular points of the
required pyramid.

Through A draw EF parallel to CD; this will make equal angles with
AC, AD; and, since AB does so too, EF
is perpendicular to AB [Pappus has a lemma
for this, p. 140, ii — 34], and is therefore a
tangent to the sphere (for EF is parallel to
CD, the base of the triangle A CD, and
therefore touches the circle circumscribing
it, while it also touches the circular section
AB made by the plane passing through AB
and EF perpendicular to it).

Similarly GH drawn through D parallel
to AB touches the sphere.

And the plane through GH, CD makes
a circular section equal and parallel to AB.

Through the centre K of that circular
section, and in the plane of the section, draw LM perpendicular to CD and
therefore parallel to AB. Join BL, BM.

BM  then perpendicular to AB, LM, and LB is a diameter of the sphere.

Join MC.

Then LAP=iMC

and BC=AB = LM,

so that BC* = 2MC.

And BM, being perpendicular to the plane of the circle LAf, is perpen-
dicular to CM,

whence BC 1 = BM 1 + MC,

so that BM= MC.

But BC=ZM;

therefore LM* = zBM'' 1 .

And, since the angle LMB is right,

BL>=LM* + MB» =   LM

Synthesis.

Draw two parallel circular sections of the sphere with diameter a'',
such that

where d is the diameter of the sphere.

[This is easily done by dividing BL, any diameter of the sphere, at P, so
that LP= 2PB, and then drawing PM at right angles to LB meeting the
great circle LMB of the sphere in M. Then LM « : LB* - LP: LB = 2 : 3.]

Draw sections through M, B perpendicular to MB, and in these sections
respectively draw the parallel diameters LM, AB.

Lastly, in the section LM draw CD through the centre K perpendicular
to LM.

ABCD is then the required regular pyramid or tetrahedron.

\end{notes}

\end{proposition}

\begin{proposition}
\label{prop:XIII_14}

\begin{statement}
To construct an octahedron and comprehend it in a sphere,
as in the preceding case ; and to prove that the square on the
diameter of the sphere is double of the square on the side of the
octahedron.
\end{statement}

\begin{proof}

Lei the diameter AB of the given sphere be set out,

and let it be bisected at C ;

let the semicircle ADB be described on AB,

let CD be drawn from C at right angles to AB,

let DB be joined ;

let the square EFGH, having each of its sides equal to DB,
be set out,

let HE, EG be joined,

from the point K let the straight line KL be set up at rig In
angles to the plane of the square EFGH [xi. 12], and let it be
carried through to the other side of the plane, as KM ;

from the straight lines KL, KM let KL, KM be respectively
cut off equal to one of the straight lines EK, FK, GK, HK,

and let LE, LF, LG, LH, ME, MF, MG, MM be joined.

Then, since KE is equal to KH,
and the angle EKH is right,
therefore the square on HE is double of the square on EK.

['  47]
Again, since LK is equal to KE,

and the angle LKE is right,

therefore the square on EL is double of the square on EK.

[id.]

But the square on HE was also proved double of the
square on EK;

therefore the square on LE is equal to the square on EH;
therefore LE is equal to EH.

For the same reason
LH is also equal to HE ;
therefore the triangle LEH is equilateral.

Similarly we can prove that each of the remaining tri-
angles of which the sides of the square EFGH are the bases,
and the points L, M the vertices, is equilateral ;
therefore an octahedron has been constructed which is con-
tained by eight equilateral triangles.

It is next required to comprehend it in the given sphere,
and to prove that the square on the diameter of the sphere is
double of the square on the side of the octahedron.

For, since the three straight lines LK, KM, KE are equal
to one another,

therefore the semicircle described on LM will also pass
through E.

And for the same reason,

if, LM remaining fixed, the semicircle be carried round and
restored to the same position from which it began to be
moved,

it will also pass through the points F, G, H,

and the octahedron will have been comprehended in a sphere.

I say next that it is also comprehended in the given sphere.
For, since LK is equal to KM,

while KE is common,

and they contain right angles,

therefore the base LE is equal to the base EM. [1. 4]

And, since the angle LEM is right, for it is in a semicircle,

[in. 31J
therefore the square on LM is double of the square on LE.

['- 47]
Again, since AC is equal to CB,
A Bis double of BC.

But, as AB is to BC, so is the square on AB to the square
on BD ;
therefore the square on AB is double of the square on BD.

But the square on LM was also proved double of the
square on LE.

And the square on DB is equal to the square on LE, for
EH was made equal to DB,

Therefore the square on AB is also equal to the square
on LM;
therefore AB is equal to LM.

And AB is the diameter of the given sphere ;
therefore LM is equal to the diameter of the given sphere.

Therefore the octahedron has been comprehended in the
given sphere, and it has been demonstrated at the same time
that the square on the diameter of the sphere is double of the
square on the side of the octahedron.
\end{proof}

\begin{notes}

I think [he accompanying figure will perhaps be clearer than that in
Euclid's text.

EFGH being a square with side equal to BD, it follows that K£, KF,
KG, KH are all equal to CB.

So are KL, KM, by construction ;
hence LE, LF, LG, i/fand ME, MF, MG, MHslk all equal to EF or BD.

Thus (1) the figure is made up of eight equilateral triangles and is therefore
a regular octahedron.

(a) Since KEKL = KM,

the semicircle on LM in the plane LKE passes through E.

Similarly F, G, -tf lie on semicircles on LAfas diameter.

Thus all the vertices of the tetrahedron lie on the sphere of which LM is
a diameter.

(3) LE = EM=BD;

therefore LM 1 = iED = t£D'

or

= AB
LM=AB.

(4) AS* m iBD i

= 2£F'.

If r be the radius of the circumscribed sphere,

(edge of octahedron) = J 2 . r.

Pappus' method.

Pappus (in. pp. 143 — 150) finds the two equal and parallel sections of the
sphere which circumscribe two opposite faces of the octahedron thus.

Analysis.

Suppose the octahedron inscribed, A, B, C; D, E, F being the vertices.

Through ABC, DEF describe planes
making the circular sections ABC, DEF.

Since the straight lines DA, DB, DE, DF
are equal, the points A, E, F, B lie on a circle
of which D is the pole.

Again, since AB, BF, FE, EA are equal,
A BEE is a square inscribed in the said circle,
and AB, EFre parallel.

Similarly DE is parallel to BC, and DE
to AC.

Therefore the circles through D, E, .Fand
A, B, C are parallel ; and they are also equal
because the equilateral triangles inscribed in
them are equal.

Now, ABC, DEF being equal and parallel circular sections, and AB, EF
equal and parallel chords `` not on the same side of the centres,''
``is a diameter of the sphere.

[Pappus has a lemma for this, pp. 136 — 138],

And AE = EF, so that AF 1 = tFE

But, if a'' be the diameter of the circle DEF,

d' , = EF t . [cf. xhi. 12]

Therefore, if d be the diameter of the sphere,

<f*:<r=-3:*-

Now d is given, and therefore d' is given ; hence the circles DEF, ABC
are given.

Synthesis.

Draw two equal and parallel circular sections with diameter d', such that

where d is the diameter of the sphere.

Inscribe an equilateral triangle ABC in either circle (ABC).

In the other circle draw EF equal and parallel to AB but on the opposite
side of the centre, and complete the inscribed equilateral triangle DEF.

ABCDEF'w the octahedron required.

It will be observed that, whereas in the problem of xnt, 13 Euclid first
finds the circle circumscribing a face and Pappus first finds an edge, in this
problem Euclid finds the edge first and Pappus the circle circumscribing
a face.

\end{notes}

\end{proposition}

\begin{proposition}
\label{prop:XIII_15}

\begin{statement}
To construct a cube and comprehend it in a sphere, like the
pyramid ; and to prove that the square on the diameter of the
sphere is triple of the square on the side of the cube.
\end{statement}

\begin{proof}

Let the diameter AB of the given sphere be set out,
and let it be cut at C so that AC is double of CB ;
let the semicircle ADB be described on AB,
let CD be drawn from C at right angles to AB,
and let DB be joined ;

let the square EFGH having its side equal to DB be set out,
from E, F, G, H let EK, FL, GM, HN be drawn at right
angles to the plane of the square EFGH,
from EK, FL, GM, HN let EK, FL, GM, HN respectively
be cut off equal to one of the straight lines EF, EG,
GH, HE,

and let KL, LM, MN, NK be joined ;
therefore the cube FN has been constructed which is contained
by six equal squares.

E H .

It is then required to comprehend it in the given sphere,
and to prove that the square on the diameter of the sphere is
triple of the square on the side of the cube.

For let KG, EG be joined.

Then, since the angle KEG is right, because KE is also
at right angles to the plane EG and of course to the straight
line EG also, [xi. Def. 3]

therefore the semicircle described on KG will also pass through
the point E.

Again, since GF is at right angles to each of the straight
lines FL, FE,

GF is also at right angles to the plane FK ;

hence also, if we join FK, GF will be at right angles to FK ;

and for this reason again the semicircle described on GK will
also pass through F.

Similarly it will also pass through the remaining angular
points of the cube.

If then, KG remaining fixed, the semicircle be carried
round and restored to the same position from which It began
to be moved,
the cube will be comprehended in a sphere.

I say next that it is also comprehended in the given
sphere.

For, since GF is equal to FE,
and the angle at F is right,
therefore the square on EG is double of the square on EF.

But EF is equal to EK;
therefore the square on EG is double of the square on EK;
hence the squares on GE, EK, that is the square on GATj.47],
is triple of the square on EK

And, since AB is triple of BC,
while, as AB Is to BC, so is the square on A B to the square
on BD.
therefore the square on AB is triple of the square on BD.

But the square on GK was also proved triple of the square
onKE.

And KE was made equal to DB ;
therefore KG is also equal to AB.

And AB is the diameter of the given sphere ;
therefore KG is also equal to the diameter of the given
sphere.

Therefore the cube has been comprehended in the given
sphere ; and it has been demonstrated at the same time that
the square on the diameter of the sphere is triple of the square
on the side of the cube.
\end{proof}

\begin{notes}

AB is divided so that AC = iCB; CD is drawn at right angles to AB,
and BD is joined

KG is, by construction, a cube of side equal to BD.

To prove ( 1) that it is inscribable in a sphere.

Since KE is perpendicular to EH, EF,
KE is perpendicular to EG.

Thus, KEG being a right angle, £ lies on a semicircle with diameter KG.
The same thing is proved in the same way of the other vertices
F, N, L, M, N.

Thus the cube is inscribed in the sphere of which KG is a diameter,

KG* = KE' + EG*
= KE* + 2EF*
= 3EK*.
AB=BC,
AB.BC= AB* -AB.BC
= AB 1 :BJ? 1 ;
AB* = BD
BD=EK;
KG = AB.
AB' = 3BD'
= SKE'.
If r be the radius of the circumscribed sphere,

(edge of cube)   A~ .r«f</. r.

Also
while

therefore

But
therefore

(3)

<J3

Pappus' solution.

In this case too Pappus (ill. pp. 144 — 148) gives the full analysis and
synthesis.

Analysis.

Suppose the problem solved, and let the vertices of the cube be
A, B, C, A £, F, G, H.

Draw planes through A, B, C, D and
E, F t G, H respectively ; these will produce
parallel circular sections, which are also equal
since the inscribed squares are equal.

And CE will be a diameter of the sphere.

Join EG.

Now, since EG* = 2EB* - iGC,
and the angle CGE is right,

CE'' m GC + EG' =  EG

But CE* is given ;
therefore EG' is given, so that the circles
EFGH, ABCD, and the squares inscribed in them, are given.

Synthtsis.

Draw two parallel circular sections with equal diameters d\ such that

where d is the diameter of the given sphere.

Inscribe a square in one of the circles, as ABCD.

In the other circle draw FG equal and parallel to BC, and complete the
square on FG inscribed in the circle EFGH.

The eight vertices of the required cube are thus determined.

\end{notes}

\end{proposition}

\begin{proposition}
\label{prop:XIII_16}

\begin{statement}
To construct an icosakedron and comprehend it in a sphere,
tike the aforesaid figures ; and to prove that the side of the
icosakedron is the irrational straight line called minor.
\end{statement}

\begin{proof}

Let the diameter AB of the given sphere be set out,

and let it be cut at C so that AC is quadruple of CB,

let the semicircle ADB be described on AB,

let the straight line CD be drawn from C at right angles
to AB,

and let DB be joined ;

let the circle EFGHK be set out and let its radius be equal

toX,

let the equilateral and equiangular pentagon EFGHK be

inscribed in the circle EFGHK,

let the circumferences EF, FG, GH, HK, KE be bisected at

the points L, M, N, O, P,

and let LM, MN, NO, OP, PL, EP be joined.

4 8a BOOK XIII [xhi. 16

Therefore the pentagon LMNOP is also equilateral,
and the straight line EP belongs to a decagon.

Now from the points E, F, G, H, Ket the straight lines
EQ, FR, GS, HT, KU be set up at right angles to the plane
of the circle, and let them be equal to the radius of the circle
EFGHK,

let QR, RS, ST, TU, UQ, QL, LR, RM, MS, SN, N'T,
TO, OU, UP, PQ be joined.

Now, since each of the straight lines EQ, KU is at right
angles to the same plane,
therefore EQ is parallel to KU. [xi. 6]

But it is also equal to it ;
and the straight lines joining those extremities of equal and
parallel straight lines which are in the same direction are equal
and parallel. [' 33]

Therefore Q U is equal and parallel to EK,

But EK belongs to an equilateral pentagon ;
therefore £?f/a!so belongs to the equilateral pentagon inscribed
in the circle EFGHK.

For the same reason
each of the straight lines QR, RS, ST, TU also belongs to
the equilateral pentagon inscribed in the circle EFGHK
therefore the pentagon QRSTU is equilateral.

And, since QE belongs to a hexagon,
and EP to a decagon,
and the angle QEP is right,
therefore QP belongs to a pentagon ;

for the square on the side of the pentagon is equal to the
square on the side of the hexagon and the square on the side
of the decagon inscribed in the same circle. [xm. 10]

For the same reason
PU is also a side of a pentagon.

But QU also belongs to a pentagon ;
therefore the triangle QP U is equilateral.

For the same reason
each of the triangles QLR, RMS, SNT, TOU'is also equi-
lateral.

xni. i6] PROPOSITION t6 483

And, since each of the straight lines QL, QP was proved
to belong to a pentagon,
and LP also belongs to a pentagon,
therefore the triangle QLP is equilateral.

For the same reason
each of the triangles LRM, MSN, NTO, O UP is also equi-
lateral.

Let the centre of the circle EFGHK. the point V, be
taken ;

from V let VZ be set up at right angles to the plane of the
circle,

let it be produced in the other direction, as VX,
let there be cut off VW, the side of a hexagon, and each of
the straight lines VX, WZ, being sides of a decagon,
and let QZ, QW, UZ, EV, L V, LX, XM be joined.

Now, since each of the straight lines VW, QE is at right
angles to the plane of the circle,
therefore VWis parallel to QE, [xi. 6]

But they are also equal ;
therefore EV, Q W are also equal and parallel. [1. 33]

But E V belongs to a hexagon ;
therefore QW also belongs to a hexagon.

And, since Q W belongs to a hexagon,
and WZ to a decagon,
and the angle Q WZ is right,
therefore QZ belongs to a pentagon. [xm. 10]

For the same reason
UZ also belongs to a pentagon,

inasmuch as, if we join VK, WU, they will be equal and
opposite, and VK, being a radius, belongs to a hexagon ;

[iv. 15, Por]
therefore WU also belongs to a hexagon.

But WZ belongs to a decagon,
and the angle U WZ is right ;
therefore UZ belongs to a pentagon. [xm. 10]

But Q U also belongs to a pentagon ;
therefore the triangle QUZ is equilateral.

484 BOOK XIII [xm. id

For the same reason
each of the remaining triangles of which the straight lines
QR, RS, ST, 7''£/are the bases, and the point Z the vertex,
is also equilateral.

Again, since VL belongs to a hexagon,
and VX to a decagon,
and the angle L VX is right,
therefore LX belongs to a pentagon. [xm. w]

For the same reason,
if we join MV, which belongs to a hexagon,
MX is also inferred to belong to a pentagon.

But LM also belongs to a pentagon ;
therefore the triangle LMX is equilateral.

Similarly it can be proved that each of the remaining
triangles of which MN, NO, OP, PL are the bases, and the
point X the vertex, is also equilateral.

Therefore an icosahedron has been constructed which is
contained by twenty equilateral triangles.

It is next required to comprehend it in the given sphere,
and to prove that the side of the icosahedron is the irrational
straight line called minor.

For, since VW belongs to a hexagon,
and WZ to a decagon,

therefore VZ has been cut in extreme and mean ratio at W,
and VW is its greater segment ; [xm 9]

therefore, as ZV'vs, to VW, so is VWto WZ.

But V Wis equal to VE, and WZ to VX;
therefore, as ZVis to VE, so is EVto VX.

And the angles ZVE, EVX are right ;
therefore, if we join the straight line EZ, the angle XEZ
will be right because of the similarity of the triangles XEZ,
VEZ.

For the same reason,
since, as ZVis to VW, so is f-to WZ,
and ZVs equal to XW, and VWto WQ,
therefore, as XW is to WQ, so is QW to WZ.

xur. 16] PROPOSITION 16 485

And for this reason again,
if we join QX, the angle at Q will be right ; [vi. 8]

therefore the semicircle described on XZ will also pass
through Q. [in. 31]

And if, XZ remaining fixed, the semicircle be carried
round and restored to the same position from which it began
to be moved, it will also pass through Q and the remaining
angular points of the icosahedron,

and the icosahedron will have been comprehended in a
sphere.

I say next that it is also comprehended in the given sphere.
For let VW be bisected at A'.

Then, since the straight line VZ has been cut in extreme
and mean ratio at W,

and ZW is its lesser segment,

therefore the square on ZW added to the half of the greater
segment, that is WA', is five times the square on the half
of the greater segment ; [xm. 3]

therefore the square on ZA' is five times the square on
A'W.

And ZX is double of ZA\ and VW double of A'W;

therefore the square on ZX is five times the square on

wv.

And, since AC is quadruple of CB,

therefore AB is five times BC.

But, as A B is to BC, so is the square on AB to the square
on BD ; [vi. 8, v. Def. 9 ]

therefore the square on AB is five times the square on BD,

But the square on ZX was also proved to be five times
the square on VW.

And DB is equal to VW,

for each of them is equal to the radius of the circle EFGHK

therefore AB is also equal to XZ.

And AB is the diameter of the given sphere ;
therefore XZ is also equal to the diameter of the given sphere.

Therefore the icosahedron has been comprehended in the
given sphere

4 86 BOOK XIII (xili. 16

I say next that the side of the icosahedron is the irrational
straight line called minor.

For, since the diameter of the sphere is rational,
and the square on it is five times the square on the radius
of the circle EFGHK,

therefore the radius of the circle EFGHK is also rational ;
hence its diameter is also rational.

But, if an equilateral pentagon be inscribed in a circle
which has its diameter rational, the side of the pentagon is
the irrational straight line called minor. [xm. 11]

And the side of the pentagon EFGHK is the side of the
icosahedron.

Therefore the side of the icosahedron is the irrational
straight line called minor.

\begin{porism*}
From this it is manifest that the square on the
diameter of the sphere is five times the square on the radius
of the circle from which the icosahedron has been described,
and that the diameter of the sphere is composed of the side
of the hexagon and two of the sides of the decagon inscribed
in the same circle.
\end{porism*}
\end{proof}

\begin{notes}

Euclid's method is
(1) to find the pentagons in the two parallel circular sections of the sphere,
the sides of which form ten (five in each circle) of the edges of the icosahedron,
(2) to find the two points which are the poles of the two circular sections,

(3) to prove that the triangles formed by joining the angular points of the
pentagons which are nearest to one another two and two are equilateral,

(4) to prove that the triangles of which the poles are the vertices and the
sides of the pentagons the bases are also equilateral,

(5) that all Che angular points Other than the poles lie on a sphere the
diameter of which is the straight line joining the poles,

(6) that this sphere is of the same size as the given sphere,

(7) that, if the diameter of the sphere is rational, the edge of the icosahedron
is the minor irrational straight line.

I have drawn another figure which will perhaps show the pentagons, and
the position of the poles with regard to them, more clearly than does Euclid's
figure.

(1) If A3 is the diameter of the given sphere,, divide AB at C so that

AC=a,CB;
draw CD at right angles to AB meeting the semicircle on AB in D.
Join BD.

BD is the radius of the circular sections containing the pentagons.

[If r is the radius of the sphere,
since AB :BC=AB*:AB .BC

= AB':BD
while AB=sBC,

it follows that AB* = 3D

or (radius of section)'  r*.

f*

Thus [xiu. 10, note] (side of pentagon)' = — (10- tJs)-]

Inscribe the regular pentagon EFGHK in the circle EFGHK of radius
equal to BD.

Bisect the arcs EF, FG, . . ., so forming a decagon in the circle.

Joining successive points of bisection, we obtain another regular pentagon
LMNOP.

LMNOP is one of the pentagons containing five edges of the icasahedron.

The other circle and inscribed pentagon are obtained by drawing perpen-
diculars from E, F, G, H, K to the plane of the circle, as EQ, FB, GS,
HT, KU, and making each of these perpendiculars equal to the radius of the
circle, or, as Euclid says, the side of the regular hexagon in it.

QRSTU is the second pentagon (of course equal to the first) containing five,
edges of the icosahedron.

Joining each angular point of one of the two pentagons to the two nearest
angular points in the other pentagon, we complete ten triangles each of which
has for one side a side of one or other of the two pentagons.

V, W are the centres of the two circles, and VW is of course perpen-
dicular to the planes of both.

(2) Produce VW in both directions, making VX and IVZ both equal to
a side of the regular decagon in the circle (as EL).

Joining X, Z to the angular points of the corresponding pentagons, we

4 88 BOOK XIII [xih. 16

have five more triangles formed with the sides of each pentagon as bases, ten
more triangles in all.

Now we come to the proof.
(3) Taking two adjacent perpendiculars, EQ, KU, to the plane of the circle
EFGHK, we see that they are parallel as well as equal;
therefore QU, EK are equal and parallel.
Similarly for QR, EPetc.
Thus the pentagons have their sides equal.

To prove that the triangles QPL etc., are equilateral, we have, e.g.
QL* = ££* + £()*

= (side of decagon) 1 + (side of hexagon)*
= (side of pentagon) 5 , [xill. 10]

Le, QL - side of pentagon in circle)

= LP.
Similarly QP = LP,

and A QPL is equilateral.

So for the other triangles between the two pentagons.
(4) Since VW, EQ are equal and parallel,

VE, WQ are equal and parallel.
Thus WQ is equal to the side of a regular hexagon in the circles.
Now the angle ZWQ is right ;
therefore ZQ = ZW* + WQ 1

= (side of decagon)* + (side of hexagon)*
= (side of pentagon)*. [xiti. 10]

Thus ZQ, ZR, ZS, ZT, ZU are all equal to QR, RS etc. ; and the
triangles with Z as vertex and bases QR, RS etc. are equilateral.

Similarly for the triangles with A'' as vertex and LM, AfN fc. as bases.
Hence the figure is an icosahedron, being contained by twenty equal
equilateral triangles.

(5) To prove that all the vertices of the icosahedron lie on the sphere
which has XZ for diameter.

VW being equal to the side of a regular hexagon, and WZ to the side of
a regular decagon inscribed in the same circle,
VZ is divided at Wx\ extreme and mean ratio. [xm. 9]

Therefore ZV \ VW = VW : WZ,

or, since VW = VE, WZ=VX,

ZV: VE= VE: VX.
Thus £ lies on the semicircle on ZX as diameter. [vi. 8]

Similarly for all the other vertices of the icosahedron.
Hence the sphere with diameter XZ circumscribes it,

(6) To prove XZ.r.AS.

Since VZ is divided in extreme and mean ratio at W, and VW is
bisected at A',

AT=iA'W i . [xm, 3]

Taking the doubles of A'Z, A' W, we have

= A. [see under (1) above]

xiii. i6] PROPOSITION 16 4»9

That is, XZ=AB.

[If r is the radius of the sphere,

VW=£D=--r,

Vs

VX = (side of decagon in circle of radius BD)

SD Us-i) [xm. 9 ,note]

a

-£«*-*

Consequently

XZ= VW+2VX

= i''7s rU5 -

«>

*«*]

2

(6) The radius of the circle EFGHK is equal to -j- r, and is therefore

`` rational `` in Euclid's sense.

Hence the side of the inscribed pentagon is the irrational straight line
Called minor. [xlll. nj

[The side of this pentagon is the edge of the icosahedron, and its value is
(note on xm. 10)

BD / — —r
= gVio(5-V5)-]

Pappus* solution.

This solution (Pappus, in. pp. 150 — 6) differs considerably from that of
Euclid. Whereas Euclid uses two circular sections of the sphere (those-
circumscribing the pentagons of his construction), Pappus finds four parallel
circular sections each passing through three of the vertices of the icosahedron;
two of the circles are small circles circumscribing two opposite triangular
faces respectively, and the other two circles are between these two circles,
parallel to them and equal to one another.

Analysis.

Suppose the problem solved, the vertices of the icosahedron being A, B, C;
D, E, F; G,H,K,L,M,N.

Since the straight lines BA, BC, BF, BO, BE drawn from B to the
surface of the sphere are equal,

A, C, F, G, E are in one plane.

And AC, CF, FG, GF, FA are equal ;

therefore ACFGE is an equilateral and equiangular pentagon.

So are the figures KEBCD, DHFBA, AKLGB, AKNHC, and
CHMGB.

Join EF, KH.

Now AC will be parallel to EF (in the pentagon ACFGE) and to KH
(in the pentagon AKNHC), so that EF, KH are also parallel ;

and further Aiff is parallel to LM (in the pentagon LKDHM).

Similarly .8C, £Z», GH, ZiVare all parallel;
and likewise BA, FD, GK, J/Ware all parallel.

Since BC is equal and parallel to LN t and BA to JtfjV, the circles ABC,
LMNk equal and parallel.

Similarly the circles DEF, KG Hare equal and parallel ; for the triangles
inscribed in them are equal (since each of the Sides in both is the chord
subtending an angle of equal pentagons), and their sides are parallel re-
spectively.

Now in the equal and parallel circles DEF, KGH the chords EF, KH
are equal and parallel, and on opposite sides of the centres ;
therefore FK is a diameter of the sphere [Pappus' lemma, pp. 136 — 8], and the
angle FEK  right [Pappus' lemma, p. 138, 20 — 26].

[The diameter FKis not actually drawn in the figure.]

In the pentagon GEACF, if EF be divided in extreme and mean ratio,

the greater segment is equal to AC. [Eucl.\ xm, 8]

Therefore EF : AC= (side of hexagon) ; (side of decagon in same circle).

[xm. t,]
And EF 1 + AC 1 = EF> + £/[* = <?,

where d is the diameter of the sphere.

Thus FK, EF, AC are as the sides of the pentagon, hexagon and decagon
respectively inscribed in the same circle. [xm. 10]

But FK, the diameter of the sphere, is given ;

therefore EF, AC ate given respectively ;

thus the radii of the circles EFD, A CB are given (if r, r 1 are their radii,

r> = EF*,r> = bAC).

Hence the circles are given '
and so are the circles KHG, LMN which are equal and parallel to them
respectively.

Synthesis.

If d be the diameter of the sphere, set out two straight lines x, y, such
that d, x, y are in the ratio of the sides of the pentagon, hexagon and decagon
respectively inscribed in one and the same circle.

Draw (r) two equal and parallel circular sections in the sphere, with radii
equal to r, where f'' = J, as DEF, KGH,

and (a) two equal and parallel circular sections is ABC, LMN, with radius t*
such that r* = y*.

In the circles (1) draw EF. KHzs sides of inscribed equilateral triangles,
parallel to one another, and on opposite sides of the centres ;
and in the circles (z) draw AC, LM sides of inscribed equilateral triangles
parallel to one another and to EF, KH, and so that AC, EFaiz on opposite
sides of the centres, and likewise KH, LM.

Complete the figure.

The correctness of the construction is proved as in the analysis.

It follows also (says Pappus) that

(diam. of sphere)* = 3 (side of pentagon in DEF) 1 .

For, by construction, KF : FE =p : h,

where/, h are the sides of the pentagon and hexagon inscribed in the same
circle DEF.

And FE : h = the ratio of the side of an equilateral triangle to that of a
hexagon inscribed in the same circle ;
that is, FE 1 Am Jfj s r,

whence KF -.p = Ji   i,

or KF*=tf.

Another construction.

Mr H. M. Taylor has a neat construction for an icosahedron of edge a.

Let / be the length of the diagonal of a regular pentagon with side equal
to a.

Then (figure of xnt. 8), by Ptolemy's theorem,
P = ia + a*.

Construct a cube with edge equal to /.

Let O be the centre of the cube.

From O draw OL, OM, ON perpendicular to three adjacent faces, and in
these draw PF, QQ, RR parallel to AB, AD, AE respectively.

Make LP, LP', MQ, MQ, NR, NK all equal to Jo.

Le [ Pi P> 9i 9> r i r be the reflexes of P, P', Q, Q, K, R' respectively.

Then will P, P\ Q, Q, P, K, p, p', g, q\ r, r be the vertices of a regular
icosahedron.

The projections of PQ on AB, AD, AE are equal to j (/-«), J a, l
rc s dgc ti vcl Vj

Therefore PCf = i (/ - a)' + J a* + \ />

= «*.

Therefore PQ = a.

Similarly it may be proved that every other edge is equal to a

All the angular points tie on a sphere with radius OP, and

op* = i («* + />.

Each solid pentahedral angle is composed of five equal plane angles, each
of which is the angle of an equilateral triangle.
Therefore the icosahedron is regular.

And, from the equation i*-la + a\ we derive

Therefore, if r-be the radius of the sphere,

4»''.

whence
as above.]

= 4r ls /io-s i ys/V8o
=  Vio(5–7sJ,

\end{notes}

\end{proposition}

\begin{proposition}
\label{prop:XIII_17}

\begin{statement}
To construct a dodecahedron and comprehend, it in a sphere,
like the aforesaid figures, and to prove that the side of the
dodecahedron is the irrational straight line called apotome.
\end{statement}

\begin{proof}

Let ABCD, CBEE, two planes of the aforesaid cube at
right angles to one another, be set out,

let the sides AB, BC, CD, DA, EF, EB, EC be bisected at
G, H, K, L, M, N, respectively,
let GK, HL, MH, NO be joined,

let the straight lines NP, PO, HQ be cut in extreme and
mean ratio at the points R, S, T respectively,
and let RP, PS, TQ be their greater segments ;
from the points R, S, 7* let RU, SV, TlVbe set up at right
angles to the planes of the cube towards the outside of the
cube,

let them be made equal to RP, PS, TQ,

and let UB, BW, JVC, CV, VU be joined.

I say that the pentagon UBWCV is equilateral, and in
one plane, and is further equiangular.
For let RB, SB, VB be joined.

494 BOOK XIII [xiu. 17

Then, since the straight line NP has been cut in extreme
and mean ratio at R,
and RP is the greater segment,

therefore the squares on PN, NR are triple of the square
on RP. [xiu. 4J

But PN is equal to NB, and PR to RU;
therefore the squares on BN, NR are triple of the square
on RU.

But the square on BR is equal to the squares on BN, NR;

[I- «]

therefore the square on BR is triple of the square on RU;
hence the squares on BR, RU are quadruple of the square
on RU,

But the square on B U is equal to the squares on BR, RU;
therefore the square on BUis quadruple of the square on RU;
therefore BU Is double of RU.

But VU is also double of UR,
inasmuch as SR is also double of PR, that is, of RU;
therefore BU is equal to UV.

Similarly it can be proved that each of the straight lines
BfV, WC, CV is also equal to each of the straight lines
BU, UV.

Therefore the pentagon BUVCWs equilateral.

I say next that it is also in one plane.

For let PX be drawn from P parallel to each of the
straight lines RU, SV and towards the outside of the cube,
and let XH, HWbe, joined ;
I say that XHW is a straight line.

For, since HQ has been cut in extreme and mean ratio at
T, and Q T is its greater segment,
therefore, as HQ is to Q T, so is £?7to TH.

But HQ is equal to HP, and QT to each of the straight
lines TW,PX
therefore, as HP is to PX, so is WT to TH.

And HP is parallel to TW,
for each of them is at right angles to the plane BD ; fxi. 6]
and TH is parallel to PX,
for each of them is at right angles to the plane BF. [**]

xiii. 17] PROPOSITION 17 495

But if two triangles, as XPH, HTW, which have two
sides proportional to two sides be placed together at one
angle so that their corresponding sides are also parallel,
the remaining straight lines will be in a straight line ; [vi.
therefore XH is in a straight line with HW.

But every straight line is in one plane ; [xi. 1]

therefore the pentagon [/BWC Vis in one plane.

I say next that it is also equiangular.

For, since the straight line NP has been cut in extreme
and mean ratio at R, and PR is the greater segment,
while PR is equal to PS,

therefore AfS has also been cut in extreme and mean ratio
at P,

and NP is the greater segment ; [xiii. 5]

therefore the squares on NS, SP are triple of the square
on NP. [xiii. 4]

But NP is equal to NS, and PS to SV;
therefore the squares on NS, SV are triple of the square
on NB;

hence the squares on VS, SN, NB are quadruple of the square
on NB.

But the square on SB is equal to the squares on SN, NB ;
therefore the squares on BS, SV, that is, the square on BV
—for the angle VSB is right — is quadruple of the square
on NB;

therefore VB is double of BN.

But BC is also double of BN ;

therefore BV is equal to BC.

And, since the two sides BU, UV are equal to the two
sides BW, WC,

and the base BV is equal to the base BC,

therefore the angle BUV is equal to the angle BWC. [1. 8]

Similarly we can prove that the angle (JVC is also equal
to the angle BWC;

therefore the three angles BWC BUV, UVC are equal to
one another.

49*5 BOOK XIII [xih. 17

But if in an equilateral pentagon three angles are equal to
one another, the pentagon will be equiangular, [xui. 7]

therefore the pentagon BUVCWs equiangular.

And it was also proved equilateral ;
therefore the pentagon BUVCW is equilateral and equi-
angular, and it is on one side BC of the cube.

Therefore, if we make the same construction in the case
of each of the twelve sides of the cube,

a solid figure will have been constructed which is contained
by twelve equilateral and equiangular pentagons, and which is
called a dodecahedron.

It is then required to comprehend it in the given sphere,
and to prove that the side of the dodecahedron is the irrational
straight line called apotome.

For let XP be produced, and let the produced straight
line be XZ ;

therefore PZ meets the diameter of the cube, and they bisect
one another,

for this has been proved in the last theorem but one of the
eleventh book. [xi. 38]

Let them cut at Z ;
therefore Z is the centre of the sphere which comprehends
the cube,
and ZP is half of the side of the cube.

Let UZ be joined.

Now, since the straight line NS has been cut in extreme
and mean ratio at P,
and NP is its greater segment,

therefore the squares on NS, SP are triple of the square
on NP. [xui. 4]

But NS is equal to XZ,
inasmuch as NP is also equal to PZ, and XP to PS,

But further PS is also equal to XU,
since it is also equal to RP ;

therefore the squares on ZX, XU are triple of the square
on NP.

But the square on UZ is equal to the squares on ZX, XU;
therefore the square on UZ is triple of the square on NP.

xni. 17] PROPOSITION r7 497

But the square on the radius of the sphere which compre-
hends the cube is also triple of the square on the half of the
side of the cube,

for it has previously been shown how to construct a cube and
comprehend it in a sphere, and to prove that the square on
the diameter of the sphere is triple of the square on the side
of the cube. [xm, 15]

But, if whole is so related to whole, so is half to half also ;
and NP is half of the side of the cube ;

therefore UZ is equal to the radius of the sphere which com-
prehends the cube.

And Z is the centre of the sphere which comprehends the
CHbe ;
therefore the point U is on the surface of the sphere.

Similarly we can prove that each of the remaining angles
of the dodecahedron is also on the surface of the sphere ;
therefore the dodecahedron has been comprehended in the
given sphere.

I say next that the side of the dodecahedron is the irrational
straight line called apotome.

For since, when NP has been cut in extreme and mean
ratio, RP is the greater segment,

and, when PO has been cut in extreme and mean ratio, PS
is the greater segment,

therefore, when the whole NO is cut in extreme and mean
ratio, RS is the greater segment.

[Thus, since, as NP is to PR, so is PR to RN,
the same is true of the doubles also,

for parts have the same ratio as their equimultiples ; [v. 15)
therefore as NO is to RS, so is RS to the sum of NR, SO.

But NO is greater than RS ;
therefore RS is also greater than the sum of NR, SO ;
therefore NO has been cut in extreme and mean ratio,
and RS is its greater segment.]

But RS is equal to UV
therefore, when NO is cut in extreme and mean ratio, UV is
the greater segment.

498 BOOK XIII [xiii. 17

And, since the diameter of the sphere is rational,
and the square on it is triple of the square on the side of the
cube,
therefore NO, being a side of the cube, is rational.

[But if a rational line be cut in extreme and mean ratio,
each of the segments is an irrational apotome.]

Therefore UV, being a side of the dodecahedron, is an
irrational apotome. [xiii. 6]

\begin{porism*}
From this it is manifest that, when the side of
the cube is cut in extreme and mean ratio, the greater segment
is the side of the dodecahedron.
\end{porism*}
\end{proof}

\begin{notes}

In this proposition we find Euclid using two propositions which precede
but are used nowhere else, notably vi. 32, which some authors, in consequence
of their having overlooked its use here, have been hard put to it to explain.

Euclid's construction in this case is really identical with that given by
Mr H. M, Taylor, and also referred to by Henrici and Treutlein under `` crystal-
formation.''

Euclid starts from the cube inscribed in a sphere, as in xiii. 15, and then
finds the side of the regular pentagon in which the side of the cube is a
diagonal

Mr Taylor takes / to be the diagonal of a regular pentagon of side a,
so that, by Ptolemy's theorem,

P - a I + a*,
constructs a cube of wnich / is the edge, and gets the side of the pentagon
by drawing ZX from Z, the centre of the cube, perpendicular to the face BF
and equal to J (/+«), then drawing UV through X parallel to BC, and
making UX, XV both equal to a.

Euclid finds UVtinxt.

Draw NO, Mil bisecting pairs of opposite sides in the square BF and
meeting in P.

Draw GK, HL bisecting pairs of opposite sides in the square BD and
meeting in Q.

Divide FN, FO, QH respectively in extreme and mean ratio at R, S, T
(FR, FS, QT being the greater segments); draw RU, SV, TW outwards
perpendicular to the respective faces of the cube, and all equal in length
to FR.FS.TQ.

Join BU, UV, VC, CW, WB.

Then BUVCW is one of the pentagonal faces of the dodecahedron ;
and the Others can be constructed in the same way.

Euclid now proves

(1) that the pentagon BUVCW\ equilateral,

(1) that it is in one plane,

(3) that it is equiangular,

(4) that the vertex U is on the sphere which circumscribes the cube, and
hence

(5) that all the other vertices lie on the same sphere,
and (6) that the side of the dodecahedron is an apotome,

(1) To prove that the pentagon BUVCW'vz equilateral.
We have BU*= BR 1 + RU*

= ( BN* + MR*) -h RP*

- (FN* + NR*) + RP*

= SUP* * RP* [xui. 4]

= 4/''

= UV*.
Therefore BU = UV.

Similarly it may be proved that BW, JVC, CV are all equal to UV
or BU.

J Mr Taylor proceeds in this way. With his notation, the projections of
on BA, BC, BE are respectively Jo, J (/- a), l.

Therefore BU* = ia , + J(/-a) , + l*

= (l t -ai+a t )

= a*.
Similarly for BW, SVC etc.]

(2) To prove that the pentagon BUVCW\ in one plane.
Draw PX parallel to R Uoi S'V meeting UV in X.
Join XH, HW.

TKen we have to prove that XH, HW are in one straight line.

Now HP, WT, being both perpendicular to the face BD, are parallel.

For the same reason XP, HT are parallel.

Also, since QH is divided at T in extreme and mean ratio,
QH:QT=QT:Tff.

And QH= HP, QT= WT= PX.

Therefore HP .PXWT. TH.

Consequently the triangles HPX, WTH satisfy the conditions of w. 32 ;
hence XHW is a straight line.

[Mr Taylor proves this as follows :

The projections of WH, WX on BE are \ a and \ (a + /),
and the projections of WH, WX on 3 A are (/-«) and |/;
and a :(a + l)-(l-a):t,

since aJ-/*-a t .

Therefore WHX is a straight line.]

(3) To prove that the pentagon BUVCW  equiangular.
We have BV* = BS*  +  SF«

= (BN* + jVS*) + SF*
= PN* + (NS* + SP*)
= PN* + sPN
since NS is divided in extreme and mean ratio at P [xm. 5], so that

JVS* + J , /'' = 3 /W. [xm. 4]

Consequently By = 4PW

=bc;

or BV=BC.

The As fSf, WBCam therefore equal in all respects,
and lBUV= lBWC.

Similarly l CVU > i.,5 WC.

Therefore the pentagon is equiangular. (xm. 7]

(4) To prove that the sphere which circumscribes the cube also circum-
scribes the dodecahedron we have only to prove that, if X be the centre of
the sphere, ZU= ZB, for example.

Now, by xi. 38, XP produced meets the diagonal of the cube, and the
portion of XP produced which is within the cube and the diagonal bisect
one another.

And ZU* = ZX , + XU*

= IfS' + PS'
= 3 PH
as before.

Also (cf. xm. 15)

ZB' = ZP*+PB>

= ZP* + PflP + NB*

Hence ZU=ZB.

(5) Similarly for ZV, ZW etc.

xin. 17] PROPOSITION 17 5°'

(6) Since FN is divided in extreme and mean ratio at R,
NP:PR = PR: RN.

Doubling the terms, we have

NO:RS=RS: (NR + SO),
so that, if NO is divided in extreme and mean ratio, the greater segment
is equal to RS.

Now, since the diameter of the sphere is rational,
and (diam. of sphere)' = 3 (edge of cube)*,

the edge of the cube (i.e. NO) is rational.

Consequently RS is an apotome.

SThis is proved in the spurious xm. 6 above ; Euclid assumes it, and the
Is purporting to quote the theorem are probably interpolated, like xm. 6
itself.]

As a matter of fact, with Mr Taylor's notation,
P = la + a 1 ,

and a=— I.

2

Since, if r is the radius of the circumscribing sphere, >''= V3   ~>

Pappus' solution.

Here too Pappus (m. pp. 156 — 162) finds four circular sections of the
sphere all parallel to one another and all passing through five of the vertices
of the dodecahedron.

Analyst).

Suppose (he says) the problem solved, and let the vertices of the
dodecahedron be A, B, C, D, E; F, G, H, K, L; M, N, O, P, Q;
R, S, T, U, V.

Then, as before, ED is parallel to PL, and AE to PG ; therefore the
planes ABODE, FGHKL are parallel.

But, since PA is parallel to BH, and BH to 00, PA is parallel to OC ;
and they are equal ; therefore PO, AC are parallel, so that ST, ED are also
parallel.

Similarly RS, DC are parallel, and likewise the pairs (TV, EA),
(UV,AB), (VR,BC).

Therefore the planes ABODE, RSTUV are parallel ; and the circles
ABCDE, RSTUVexe equal, since the inscribed pentagons are equal.

Similarly the circles FGHKL, MNOPQ are equal, since the pentagons
inscribed in them are equal.

Now CL, OU are parallel because each is parallel to KN;
therefore L, C, O, U are in one plane.

And LC, CO, OU, UL are all equal, since they subtend angles of equal
pentagons.

Also L, C, O, U are on a plane section, i.e. a circle ;
therefore LCOU'  a square.

Therefore 01? = 2LC* = zLF*

(for LC, LF subtend angles of equal pentagons).

And the angle OLF is right; for J>0, LF are equal and parallel chords
in two equal and parallel circular sections of a sphere [Pappus' lemma, p. 138,
30 — 26].

Therefore OF* = OL* + FD - )FL f . [from above]

And OF is a diameter of the sphere ; for PO, Ft are on opposite sides
of the centres of the circles in which they are [Pappus' lemma, pp. 136 — 8],

Now suppose /, /, A to he the sides of an equilateral pentagon, triangle
and hexagon in the circle FGHKL, d the diameter of the sphere.

Then d:FL = Js>.i [from above]

= t : k; [Eucl.\ xni. u]

and it follows a/temanda (since FL -p) that

d-.tp-.h.

Now let d\ p\ H be the sides of a regular decagon, pentagon and hexagon
respectively inscribed in any one circle.

Since, if FL be divided in extreme and mean ratio, the greater segment is

equal to ED, [xm. S]

FL : ED = X :£. [vi. Def. 3, xm. o]

And FL : ED is the ratio of the sides of the regular pentagons inscribed
in the circles FGHKL, ABCDE, and is therefore equal to the ratio of the
sides of the equilateral triangles inscribed in the same circles.

Therefore / : (side of A in ABCDE) = A' ;d

But d:t=p:k

therefore, ex aequali, d : (side of A in ABCDE) -p' : dl.

Now d is given ;
therefore the sides of the equilateral triangles inscribed in the circles ABCDE,
FGHKL respectively are given, whence the radii of those circles are also
given.

Thus the two circles are given, and so accordingly are the equal and
parallel circular sections.

Synthesis.

Set out two straight lines x, y such that d, x, y are in the ratio of the sides
of a regular pentagon, hexagon and decagon respectively inscribed in one and
the same circle.

Find two circular sections of the sphere with radii r, /, where

Let these be the circles FGHKL, ABODE respectively) and draw the
equal and parallel circles on the other side of the centre, namely MNOPQ,
RSTUV,

In the first two circles inscribe regular pentagons with their sides respec-
tively parallel, ED being parallel to FL.

Draw equal and parallel chords (on the other sides of the centres) in the
other two circles, namely ST equal and parallel to ED, and PO equal and
parallel to FL ; and complete the regular pentagons on ST, PO inscribed in
the circles.

Thus all the vertices of the dodecahedron are determined.

The proof of the correctness of the construction is clear from the analysis.

Pappus adds that the construction shows that the circles containing five
vertices of the dodecahedron are the same respectively as those containing
three vertices of the icosahedron, and that the same circle circumscribes the
triangle of the icosahedron and the pentagonal face of the dodecahedron in
the same sphere.

\end{notes}

\end{proposition}

\begin{proposition}
\label{prop:XIII_18}

\begin{statement}
To set out the sides of the five figures and to compare them
with one another.
\end{statement}

\begin{proof}

Let AB, the diameter of the given sphere, be set out,

and let it be cut at C so that
AC is equal to CB, and at D
so that AD ts double of DB

let the semicircle AEB be de-
scribed on AB,

from C, D let CE, DEbe drawn
at right angles to AB,

and Jet AF, FB t EB be joined.

Then, since AD is double
ofZ?,

therefore AB is triple of BD.

Convertendo, therefore, BA is one and a half times AD.

But, as BA is to AD, so is the square on BA to the
square on AF, [v. Def. g, vi. 8]

for the triangle AFB is equiangular with the triangle AFD ;
therefore the square on BA is one and a half times the square
on AF.

But the square on the diameter of the sphere is also one
and a half times the square on the side of the pyramid.

[xiil i 3 ]

And AB is the diameter of the sphere ;
therefore AF is equal to the side of the pyramid.

Again, since AD is double of DB,
therefore AB is triple of BD.

But, as AB is to BD, so is the square on AB to the square
on BF; [vi. 8, v. Def. 9 ]

therefore the square on AB is triple of the square on BF.

But the square on the diameter of the sphere is also triple
of the square on the side of the cube. [xm. 15]

And AB is the diameter of the sphere ;
therefore BF is the side of the cube.

And, since AC is equal to CB,
therefore AB is double of BC.

But, asAB is to BC, so is the square on AB to the square
on BE;
therefore the square on AB is double of the square on BE.

But the square on the diameter of the sphere is also double
of the square on the side of the octahedron, |xm. 14]

And AB is the diameter of the given sphere ;
therefore BE is the side of the octahedron.

Next, let AG be drawn from the point A at right angles
to the straight line AB,
let AG be made equal to AB,
let GC be joined,
and from H let HK be drawn perpendicular to AB.

Then, since GA is double of AC,
for GA is equal to AB,
and, as GA is to AC, so is HK to KC,
therefore HK is also double of KC.

Therefore the square on HK is quadruple of the square
on KC

therefore the squares on HK, KC, that is, the square on HC,
is five times the square on KC.

But HC is equal to CB ;
therefore the square on BC is five times the square on CK.

And, since AB is double of CB,
8D0, in them, AD is double of DB,
therefore the remainder BD is double of the remainder DC.

Therefore BC is triple of CD
therefore the square on BC is nine times the square on CD.

But the square on BC is five times the square on CK;
therefore the square on CK is greater than the square on CD ;
therefore CK is greater than CD.

Let CL be made equal to CK,
from L let LM be drawn at right angles to AB,
and let MB be joined.

Now, since the square on BC is five times the square
on CK.

and AB is double of BC, and KL double of CK,
therefore the square on A B is five times the square on KL.

But the square on the diameter of the sphere is also five
times the square on the radius of the circle from which the
icosahedron has been described. [xm. 16, Por.]

And AB is the diameter of the sphere ;
therefore KL is the radius of the circle from which the icosa-
hedron has been described ;
therefore KL is a side of the hexagon in the said circle.

[iv. 15, Por.]

And, since the diameter of the sphere is made up of the
side of the hexagon and two of the sides of the decagon
inscribed in the same circle, [xm. 16, Por.]

and AB is the diameter of the sphere,
while KL is a side of the hexagon,
and AK is equal to LB,

therefore each of the straight lines AK, LB is a side of the
decagon inscribed in the circle from which the icosahedron
has been described.

And, since LB belongs to a decagon, and ML to a
hexagon, ,

for ML is equal to KL, since it is a!so equal to HK, being
the same distance from the centre, and each of the straight
lines HK, KL is double of KC,

therefore MB belongs to a pentagon, [xm. 10]

But the side of the pentagon is the side of the icosa-
hedron ; [xm. 16]
therefore MB belongs to the icosahedron.

Now, since FB is a side of the cube,
let it be cut in extreme and mean ratio at N,
and let NB be the greater segment ;
therefore NB is a side of the dodecahedron. [xm. 17, Por.]

And, since the square on the diameter of the sphere was
proved to be one and a half times the square on the side AF
of the pyramid, double of the square on the side BE of the
octahedron and triple of the side FB of the cube,
therefore, of parts of which the square on the diameter of the
sphere contains six, the square on the side of the pyramid
contains four, the square on the side of the octahedron three,
and the square on the side of the cube two.

Therefore the square on the side of the pyramid is four-
thirds of the square on the side of the octahedron, and double
of the square on the side of the cube ;

and the square on the side of the octahedron is one and a half
times the square on the side of the cube.

The said sides, therefore, of the three figures, I mean the
pyramid, the octahedron and the cube, are to one another in
rational ratios.

But the remaining two, I mean the side of the icosa-
hedron and the side of the dodecahedron, are not in rational
ratios either to one another or to the aforesaid sides ;
for they are irrational, the one being minor [xm. 16] and the.
other an apotome [xm. 17],

That the side MB of the icosahedron is greater than the
side NB of the dodecahedron we can prove thus.

For, since the triangle FDB is equiangular with the
triangle FAB, [vi. 8]

proportionally, as DB is to BF, so Is BF to BA. [vr. 4]

And, since the three straight lines are proportional,
as the first is to the third, so is the square on the first to the
square on the second ; [v. Def. 9, vi. 30, Por,]

therefore, as DB is to BA, so is the square on DB to the
square on BF;

therefore, inversely, as AB is to BD, so is the square on FB
to the square on BD.

But AB is triple of BD ;
therefore the square on FB is triple of the square on BD.

But the square on AD is also quadruple of the square
on DB,

for AD is double of DB

therefore the square on AD is greater than the square on FB;
therefore AD is greater than FB ;
therefore A L is by far greater than FB.

And, when AL is cut in extreme and mean ratio,
KL is the greater segment,

inasmuch as LK belongs to a hexagon, and KA to a decagon;

[xm. 9]
and, when FB is cut in extreme and mean ratio, NB is the
greater segment ;
therefore KL is greater than NB.

But KL is equal to LM
therefore LM is greater than NB.

Therefore MB, which is a side of the icosahedron, is by
far greater than NB which is a side of the dodecahedron,

Q.E.D.

I say next that no other figure, besides the said five figures,
can be constructed which is contained by equilateral and equi-
angular figures equal to one another.

For a solid angle cannot be constructed with two triangles,
or indeed planes.

With three triangles the angle of the pyramid is constructed,
with four the angle of the octahedron, and with five the angle
of the icosahedron ;

but a solid angle cannot be formed by six equilateral and equi-
angular triangles placed together at one point,

for, the angle of the equilateral triangle being two-thirds of a
right angle, the six will be equal to four right angles :
which is impossible, for any solid angle is contained by angles
less than four right angles. [xi. 31]

For the same reason, neither can a solid angle be con-
structed by more than six plane angles.

By three squares the angle of the cube is contained, but
by four it is impossible for a solid angle to be contained,
for they will again be four right angles.

By three equilateral and equiangular pentagons the angle
of the dodecahedron is contained ;

but by four such it is impossible for any solid angle to be
contained,

for, the angle of the equilateral pentagon being a right angle
and a fifth, the four angles will be greater than four right
angles :
which is impossible.

Neither again will a solid angle be contained by other
polygonal figures by reason of the same absurdity.
Therefore etc.
\end{proof}

\begin{lemma*}

But that the angle of the equilateral and equiangular
pentagon is a right angle and a fifth we must prove thus.

Let ABCDE be an equilateral and equiangular
pentagon,

let the circle ABCDE be cir-
cumscribed about it,
let its centre F be taken,
and let FA, FB, FC, FD, FE
be joined.

Therefore they bisect the
angles of the pentagon at A,
B, C, D, E.

And, since the angles at F
are equal to four right angles
and are equal,

therefore one of them, as the angle AFB, is one right angle
less a fifth ;

therefore the remaining angles FAB, ABF consist of one
right angle and a fifth.

But the angle FAB is equal to the angle FBC
therefore the whole angle ABC of the pentagon consists of
one right angle and a fifth.

Q.E.D.

\end{lemma*}

\begin{notes}

We have seen in the preceding notes that, if r be the radius of the sphere
circumscribing the five solid figures,

(edge of tetrahedron) = | ,/6 . r,
(edge of octahedron) - Ja.r,
(edge of cube) = | Jx, . r,

(edge of icosahedron)   - */ 10 (5 — 5),

(edge of dodecahedron) = - (>Jj — 3).
Euclid here exhibits the edges of all the five regular solids in one figure.
(1) Make AD equal to zDB.

Thus

BA = AD,

and

BA .AD = £A*;AF*i

therefore

BA* = AF*.

Thus

AF= Vf . ar - J6 . r = (edge of tetrahedron).

w

AB»:BF' = AB:BD

Therefore

= 3:1.
BJ-Ab*,

or

2 2,

BF= —7- . r m - ,/3 . r m (edge of. cube).
v3 3

(3)

AB* = iBEK

Therefore

BE - J 3 ,r = (edge of octahedron).

(4) Draw AG perpendicular and equal to AB. Join GC, meeting the
semicircle in H, and draw UK perpendicular to AB.

Then GA~*AC

therefore, by similar triangles, HK= 2KC.

Hence HK* = tKC*,

and therefore KC* = UK* + KC*

= HC 1
= CB i .

Again, since AB = iCB, and AD = iD£,
by subtraction, BD= 2DC,

or BC=sDC

5io BOOK XIII | xiii. 18

Therefore gUCBC

= KC   .
Hence KC> CD.

Make CL equal to KC, draw LM at right angles to AB, and join
AM, MB.

Since CBPsKC*,

AB 1 = KL

It follows that KL (= J,r) is the radius of, or the side of the regular
hexagon in, the circle containing the pentagonal sections of the icosahedron.

[xiii. 1 6]
And, since

zr = (side of hexagon) + i (side of decagon in same circle)

[xiii. 16, Por.]

AK'' LB = (side of decagon in the said circle).

But LM = HK = KL - (side of hexagon in circle).

Therefore LM * + LB* (- BM*) = (side of pentagon in circle)* [xiii. 10]

= (edge of icosahedron )'',

and BM= (edge of icosahedron).

[More shortly, HK= *KC>

whence HK* = *KC,

and KC I = HC = r*.

Also

Thug

AK-,-CK-r(*-

BM* = HK 1 + AK*

-!-''(;-.*)'

= **( r )

= -(10–35),

BM-- */io (S - Vs) = Wjf* oJ icosahedron).]

(5) Cut />7-'(the edge of the cube) in extreme and mean ratio at N,
Then, if BN be the greater segment,

BN = (edge of dodecahedron) . [xi 11. 1 7 ]

[Solving, we obtain

BN= `` . BF
a

- Vs - 1 *

= 7<VIS-V3)

= (flgv  dodecahedron). ]

xm. i8] PROPOSITION 18 511

(6) If /, 0, c are the edges of the tetrahedron, octahedron and cube
respectively,

If each of these equals is put equal to X,

>*=! x,

whence 4* : r* : 0* : ** = 6 : 4 : 3 : 2,

and the ratios between 2r, t, 0, c are all rational (in Euclid's sense).

The ratios between these and the edges of the icosahedron and the
dodecahedron are irrational.

(7) To prove that

(edge of icosahedron) > (edge of dodecahedron),
ie. that MB> NB.

By similar A s FDB, AFB,

DB;BF=BFBA t
or DB:BA = DB* \ BF*.

But iDB = BA;

therefore BF 1 = iDB*.

By hypothesis, AD' = 4DB*

therefore AD > BF,

and, \emph{a fortiori}, AL-> BF.

Now LK is the side of a hexagon, and AK the side of a decagon in the
same circle ;

therefore, when AL is divided in extreme and mean ratio, KL is the greater
segment.

And, when BF is divided in extreme and mean ratio, BN is the greater
segment.

Therefore, since AL > BF,

KL > BN,

at LM>BN,

And therefore, \emph{a fortiori}, MB > BN.

\end{notes}

\end{proposition}

\appendix

\chapter*{I. The Contents of the So-called Book XIV.
by Hyps1cles}

This supplement to Euclid's Rook xm. is worth reproducing for the sake
not only of the additional theorems proved in it but of the historical notices
contained in the preface and in one or two later passages. Where I translate
literally from the Greek text, I shall use inverted commas; except in such
passages I reproduce the contents in briefer form.

1 have already quoted from the Preface (Vol. I. pp. 5-*-6), but I will
repeat it here.

`` Basil ides of Tyre, Protarchus, when he came to Alexandria and met
my father, spent the greater part of his sojourn with him on account of the
bond between them due to their common interest in mathematics. And on
one occasion, when looking into the tract written by Apollonius about the
comparison of the dodecahedron and icosahedron inscribed in one and the
same sphere, that is to say, on the question what ratio they bear to one
another, they came to the conclusion that Apollonius' treatment of it in this
book was not correct ; accordingly, as I understood from my father, they
proceeded to amend and rewrite it. But I myself afterwards came across
another book published by Apollonius, containing a demonstration of the
matter in question, and I was greatly attracted by his investigation of the
problem. Now the book published by Apollonius is accessible to all ; for it
has a large circulation in a form which seems to have been the result of later
careful elaboration.

``For my part I determined to dedicate to you what 'I deem to be
necessary by way of commentary, partly because you will be able, by reason
of your proficiency in all mathematics and particularly in geometry, to pass an
expert judgment upon what I am about to write, and partly because, on
account of your intimacy with my father and your friendly feeling towards
myself, you will lend a kindly ear to my disquisition. But it is time to have
done with the preamble and to begin my treatise itself.

[Prop, i.] `` The perpendicular drawn /ram the centre of any circle to the
side of the pentagon inscribed in the same circle is half the sum of the side of the
hexagon and of the side of the decagon inscribed in the same circle.''

I. THE SO-CALLED ``BOOK XIV''

513

Let ABC be a circle, and BC the side of the inscribed regular pentagon.

Take D the centre of the circle, draw DE from D perpendicular to BC,
and produce DE both ways to meet the circle in F, A.

I say that DE is half the sum of the side of the hexagon and of the side
of the decagon inscribed in the same circle.

Let DC, CF be joined ; make GE equal to EF, and join GC.

Since the circumference of the circle is five
times the arc BFC,

and half the circumference of the circle is the arc
ACF,

while the arc EC is half the arc BFC,
therefore (arc ACF) - 5 (arc FC)
or (arc AC)- 4 (arc CF).

Hence l ADC = 4 c CDF,

and therefore l AFC = 2 l. CDF.

Thus 4 CGF= lAFC=x i. CDF;
therefore [1. 31] i.CDG = i.DCG t
so that DG-GC-CF,

And GE=EF;
therefore DE = EF + FC.

Add DE to each ;

therefore tDE = DE+ FC.

And DF is the side of the regular hexagon, and FC the side of the regular
decagon, inscribed in the same circle.
Therefore etc.

``Next it is manifest from the theorem [12] in Book xm. that the perpen-
dicular drawn from the centre of the circle to the tide of the equilateral triangle
[inscribed in it] is half of the radius of the circle.

[Prop, a.] `` The same circle circumscribes both the pentagon of the dodeca-
hedron and the triangle of the icosahedron inscribed in the same sphere.

``This is proved by Aristaeus in his work entitled Comparison of the five
figures. But Apollonius proves in the second edition of his comparison of the
dodecahedron with the icosahedron that, as the surface of the dodecahedron
is to the surface of the icosahedron, so also is the dodecahedron itself to the
icosahedron, because the perpendicular from the centre of the sphere to the
pentagon of the dodecahedron and to the triangle of the icosahedron is the
same.

`` But it is right that I too should prove that

[Prop. 2] The same circle circumscribes both the pentagon of the dodecahedron
and the triangle of the icosahedron inscribed in the same sphere-

U'' For this I need the following

Lemma.

`` Jf an equilateral and equiangular pentagon be inscribed in a circle, the sum
of the squares on the straight line subtending two sides and on the side of the
pentagon is Jive times the square on the radius.''

5«4

APPENDIX

Let ABC be a circle, AC the side of the pentagon, Z> the centre ;
draw 2V perpendicular to AC and produce it to
B,E;

join AS, AE.
I say that

BA i + AC*=sDE 1 .
For, since BE = tED,

BE* = 4EZP.
And £E l = BA* + AE';

therefore BA X -t- AE 1 + ED 1 = E£P.
But AC* = DE t + EA 2 ;

[Eucl.\ xiii. to]
therefore A4' +  C s = siXE*.

``This being proved, it is required to prove that the same circle circum-
scribes both the pentagon of the dodecahedron and the triangle of the
icosahedron inscribed in the same sphere.''

Let AB be the diameter of the sphere, and let a dodecahedron and an
icosahedron be inscribed.

A, B

O ``

M

N

Let CDEFG be one pentagon of the dodecahedron, and KLH one
triangle of the icosahedron.

I say that the radii of the circles circumscribing them are equal.

Join DG ; then DG is the side of a cube inscribed in the sphere.

[Eucl.\ xiii. 17]

Take a straight line .AW such that AB I = iMN*.

Now the square on the diameter of the sphere is five times the square on
the radius of the circle from which the icosahedron is described.

[xm. r6, Por.]

Therefore MN'k equal to the radius of the circle passing through the five
vertices of the icosahedron which form a pentagon.

Cut MN''n extreme and mean ratio at O, MO being the greater segment

Therefore MO is the side of the decagon in the circle with radius MN.

[xiii. 9 and 5, converse]

Now MN*=AB > =  DG*. [xm. 15]

But 3 Z>C 1 3 C<?= sMN* 1 SM

(since, if DG is cut in extreme and mean ratio, the greater segment is equal
to CG, arid, if two straight lines are cut in extreme and mean ratio, their
segments are in the same ratio : see lemma later, op. 518 — 9).

I. THE SO CALLED ``BOOK XIV''

sis

And sMO 1 + MN' = t,KD.

[This follows from xm. to, since KL is, by the construction of xm. 16, the
side of the regular pentagon in the circle with radius equal to MN, that is, the
circle in which MN is the side of the inscribed hexagon and MO the side of
the inscribed decagon.]

Therefore  KD = 3CC + DG.

But KD = 1 5 (radius of circle about KLHf, [xm. 1 2]

and iDG* + 3CC = 15 (radius of circle about CDEFGf.

[Lemma above]
Therefore the radii of the two circles are equal.

Q.E.D.

[Prop. 3.] `` If (hire be an equilateral and equiangular pentagon and a
circle circumscribed about it, and if a perpendicular be drawn from the centre to
one tide, then

30 times the rectangle contained by the side and the perpendicular is equal to
the surface of the dodecahedron.''

Let ABCDE be the pentagon, F the centre of the circle, FG the
perpendicular on a side CD.
I say that

30CD , FG - 1 2 (area of pentagon).
Let CF, FD be joined.
Then, since

CD .FG=i(CDF),
CD.FG = io(ACDF),
whence 30CZ? . FG =12 (area of pentagon).

Similarly we can prove that,

[Prop. 4] If ABC be an equilateral triangle in a
circle, D the centre, and DE perpendicular to BC,
30BC. DE = (surface of kosahedron).
For DE.BC=*(D3C);

therefore DE . BC= 6(DBC)
= i(AEC),
whence 30.D.E . BC= jo (A ABC).

It follows that [Prop. 5]

(surface of dodecahedron) ! (surface of icosaliedron)

= (side of pentagon) . (its perpendicular) : (side of triangle) , (its Perp.).

``This being clear, we have next to prove that,
[Prop. 6] As the surface of the dodecahedron is to the surface of the kosahedron,
so is the side of the cube to the side of the icosahedron.''

5i6

APPENDIX

Let ABC be the circle circumscribing the pentagon of the dodecahedron
and the triangle of the icosahedron, and let CD
be the side of the triangle, AC that of the
pentagon.

Let E be the centre, and EF, EG perpen-
diculars to CD, AC.

Produce EG to meet the circle in B and
join BC.

Set out H equal to the side of the cube in-
scribed in the same sphere.

I say that

(surface of dodecahedron) : (surface of icosahedron)

= H : CD.

For, since the sum of EB, BC is divided at B in extreme and mean ratio,
and BE is the greater segment, [xtlt. 9]

and EG = ) (EB + BC), [Prop, 1]

while EF= BE, [see p. 513 above]

therefore, if EG is divided in extreme and mean ratio, the greater segment is
equal to EF [that is to say, since EB is the greater segment of EB + BC
divided in extreme and mean ratio, EB is the greater segment of
(EB + BC) similarly divided].

But, if H is also divided in extreme and mean ratio, the greater segment

is equal to CA.

Therefore
or

And, since
and
therefore

 EGiEF,

1 CA . EG.

a

H: CA--
FE.H--

CD = FE.H:FE. CD,
F£.Sf=CA.EG,
H.CD=CA.EG:FE.CD

= (surface of dodecahedron)

[xiii. 17, Por.]

(surf, of icos.).
[Prop. 5]

Another proof of the same theorem.
Preliminary.

Let ABC he a circle and AB, AC sides of an inscribed regular pentagon.

Join BC ; take D the centre of the circle, join AD ana produce it to

meet the circle at E. Join BD.

Let DF be made equal to AD, and CH equal
to JCff.
I say that

rect AF. BH= (area of pentagon).
For, since AD ~ 2DF,

AF=  AD.
And, since GC= iHC x

GC = GH.
Therefore FA : AD=CG: Gff,

so that AF. GH=AD. CG

=AD.BG
= *(ABD).

I. THE SO-CALLED ``BOOK XIV

5«7

Therefore

5 AF. GH
And GH= iHC;
therefore AF. HC

AF.BH

10 ( A ABD) - 2 (area of pentagon).

or

(area of pentagon),
(area o'f pentagon).

Proof of theorem.

This being clear, let the circle be set out which circumscribes the pentagon
of the dodecahedron and the triangle of the icosahe-
dron inscribed in the same sphere.

Let ABC be the circle, and AB, A C two sides of
the pentagon; join BC.

Take £ the centre of the circle, join AE and
produce it to F.

Let AE=2EG, KC=CH.

Through G draw DM at right angles to AF
meeting the circle at D, M ';

DM is then the side of the inscribed equilateral
triangle.

Join AD, AM, which are equal to DM,

Now, since AG . BH= (area of pentagon),

and AG . GD=> (area of triangle),

therefore BH : GD = (area of pentagon) : (area of triangle),
and 1 2BH : 20GD - (surface of dod.) : (surface of icos.).

But i2BH=toBC, since BH=sHC, and BC=bHC;
and 2oGD= 10DM;
therefore (surface of dodecahedron) : (surface of icosahedron)

= (side of cube) ; (side of icosahedron).

`` Next we have to prove that,

[Prop. 7] If any straight line whatever be cut in extreme aud mean ratio, then,
as is ( 1 ) tie straight line the square on whieh is equal to the sum of the squares
on the whole line and on the greater segment to (2) the straight line the square en
which is equal to the sunt of tlie squares on the whole and on the lesser segment,
so is (3) the side of the cube to (4) the side of the icosahedron.''

Let AHB be the circle circumscribing both the pentagon of the dodeca-
hedron and the triangle of the icosahedron inscribed
in the same sphere, C the centre of the circle, and
CB any radius divided at D in extreme and mean
ratio, CD being the greater segment.

CD is then the side of the decagon inscribed in
the circle. [xm. 9 and 5, converse]

Let E be the side of the icosahedron, F that of
the dodecahedron, and G that of the cube, inscribed
in the sphere.

Then E, Fare the sides of the equilateral triangle
and pentagon inscribed in the circle, and, if G is
divided in extreme and mean ratio, the greater
segment is equal to F. [xiii. 17, For.]

E
F

G

Si8 APPENDIX

Thus E? = zBC\ [xtu. la]

and CB* + BD» = 3CD*. [xm, 4]

Therefore E?   CB 1 = (CB 1 + BIT) : CD*,

or E* : (CB* + Bn*)=CB a : CUP

= G*.F>.

Therefore, alternately and inversely,

G* : E* = f* 1 ( CB* + BIT).

But F* = BC + CD* ; for the square on the side of the pentagon is equal
to the sum of the squares on the sides of the hexagon and decagon inscribed
in the same circle. [xm. 10]

Therefore G*:E*=(BC 7 + CD*) : ( CB' + BD*

which is the result required.

It has now to be proved that
[Prop. 8] (Side of cube) : (side of icosahedron)

= (content of dodecahedron) : (content of icosahedron).

Since equal circles circumscribe the pentagon of the dodecahedron and
the triangle of the kosahedron inscribed in the same sphere,
and in a sphere equal circular sections are equally distant from the centre,
the perpendiculars from the centre of the sphere to the faces of the two solids
are equal;

in other words, the pyramids with the centre as vertex and the pentagons of
the dodecahedron and the triangles of the icosahedron respectively as bases
are of equal height.

Therefore the pyramids are to one another as their bases.
Thus (12 pentagons) : (20 triangles)

= (12 pyramids on pentagons) : (20 pyramids on triangles),
or (surface of dodecahedron) : (surface of icosahedron)

= (content of dad.) : (content of icos.).
Therefore

(content of dodecahedron) : (content of icosahedron)

  (side of cube) : (side of icosahedron). [Prop. 6]

Lemma.

Jf two straight lines be cut in extreme and mean ratio, the segments of both
are in one and the same ratio.

Let AB be cut in extreme and mean ratio at C, AC being the greater
segment ;

and let DE be cut in extreme and mean ratio at F, DF being the greater
segment.

I say that AB : AC= DE : DF. A C B

Since AB.BC-AC*,

and DE . EF= DF*, *

AB.BC: AC* = DE.EF: DF*,
and 4 AB ,BC:AC' = *DE . EF : DF*.

II. THE SO-CALLED ``BOOK XV'' 519

Comfanendo,

(AB .BC+AC*):AC* = (4DE . EF+ DF 1 ) : DF 1 ,
or (AB + BCf :AC = (DE + EFf : DF 1 ; [li. 8]

therefore (AB + BC) : AC = (DE + EF) : DF,

Compotundo,

(AB + BC + AC):AC = (DE + EF + DF) : DF,
or 2AB ': AC= xDE : DF;

that is, AB : AC= DE : DF.

Summary of results.

If AB be arty straight line divided at C in extreme and mean ratio, AC
being the greater segment, and if we have a cube, a dodecahedron and an
icosahedron inscribed in one and the same sphere, then :

(1) (side of cube) : (side of icosahedron) = J(AB 1 -i-AC 1 ) : ,J (AB* + BC);

(2) (surface of dod.) : (surface of icos.)

= (side of cube) : (side of icosahedron) ;

(3) (content of dod.) : (content of icos.)

= (surface of dod.) : (surface of icos.) ;
and (4) (content of dodecahedron) : (content of icos.)

= J(AB> + AC*) ; (AB' + BC).

\chapter*{II. Note on the So-called ``Book XV.''}

The second of the two Books added to the genuine thirteen is also
supplementary to the discussion of the regular solids, but is much inferior
to the first, `` Book XIV.'' Its contents are of less interest and the exposition
leaves much to be desired, being in some places obscure and in others
actually inaccurate. It consists of three portions unequal in length. The
first (Hetberg, Vol. v, pp. 40 — 48) shows how to inscribe certain of the
regular solids in certain others, (a) a tetrahedron (`` pyramid ``) in a cube,
(A) an octahedron in a tetrahedron (``pyramid''), () an octahedron in a cube,
(J) a cube in an octahedron and (<) a dodecahedron in an icosahedron.
The second portion (pp. 48 — 50) explains how to calculate the number of
edges and the number of solid angles in the five solids respectively. The
third (pp. 50 — 66) shows how to determine the angle of inclination between
faces meeting in an edge of any one of the solids. The method is to con-
struct an isosceles triangle with vertical angle equal to the said angle of
inclination ; from the middle point of any edge two perpendiculars are drawn
to it, one in each of the two faces intersecting in that edge ; these perpen-
diculars (forming an angle which is the inclination of the two faces to one
another) are used to determine the two equal sides of an isosceles triangle,
and the base of the triangle is easily found from the known properties of the
particular solid. The rules for drawing the respective isosceles triangles are
first given all together in general terms (pp. 50—5*); and the special interest
of the passage consists in the fact that the rules are attributed to `` Isidorus
our great teacher.'' This Isidorus is no doubt Isidorus of Miletus, the
architect of the Church of St Sophia at Constantinople (about 532 a.u.),
whose pupil Eutocius also' was; he is often referred to by Eutocius (Comtn,
on Arckimeiits) as o MtAijVtos ftnvamfSs 'l<riStopoi if/itTipos iMaxatK. Thus
the third portion of the Book at all events was written by a pupil of Isidorus
in the sixth century. Kluge (Dt Eudidis eltmetUsrum libris qui ftruntur XIV
tt XV, Leipzig, 1 891) has closely examined the language and style of the
three portions and conjectures that they may be the work of different authors;
the first portion may, he thinks, date from the end of the third century (the
time of Pappus), and the second portion too may be older than the third.
Hultsch however (art. `` Eukleides `` in Pauly-Wissowa's Real-Eneyclopddie dtr
dassischtn Altertumrwissenuhaft, 1907) does not think his arguments con-
vincing.

It may be worth while to set out the particulars of Isidorus' rules for
constructing isosceles triangles with vertical angles equal respectively to
the angles of inclination between faces meeting in an edge of the several
regular solids. A certain base is taken, and then with its extremities as
centres and a certain other straight line as radius two circles are drawn ;
their point of intersection determines the vertex of the particular isosceles
triangle. In the case of the cube the triangle is of course right-angled ; in
the other cases the bases and the equal sides are as shown below.

For the tetrahedron
For the octahedron
For the icosahedron

Fur the dodecahedron

Bast of isgsttlts irianglt
the side of a triangular face

the diagonal of the square
on one side of a triangular
face

the chord joining two non-
consccutive angular points
of the regular pentagon on
an edge (the ``pentagon of
the icosahedron >N )

the chord joining two non-
consecutive angular points
of a pentagonal face [BC
in the figure of Eocl. XJIi.
 7l

Eqitat sides of
isttsteles trinnglc

the perpendicular from the
vertex of a triangular face
to its base

ditto

ditto

the perpendicular from the
middle point of the chord
joining two non-consecu-
tive angular points of a
face to the parallel side of
that face [NX in the figure
of Eucl.\ xiti. 17]

\end{document}
